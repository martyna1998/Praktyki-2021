\documentclass[12pt, a4paper]{article}
\usepackage[utf8]{inputenc}
\usepackage{polski}

\usepackage{amsthm}  %pakiet do tworzenia twierdzeń itp.
\usepackage{amsmath} %pakiet do niektórych symboli matematycznych
\usepackage{amssymb} %pakiet do symboli mat., np. \nsubseteq
\usepackage{amsfonts}
\usepackage{graphicx} %obsługa plików graficznych z rozszerzeniem png, jpg
\theoremstyle{definition} %styl dla definicji
\newtheorem{zad}{} 
\title{Multizestaw zadań}
\author{Jacek Jabłoński}
%\date{\today}
\date{}
\newcounter{liczniksekcji}
\newcommand{\kategoria}[1]{\section{#1}} %olreślamy nazwę kateforii zadań
\newcommand{\zadStart}[1]{\begin{zad}#1\newline} %oznaczenie początku zadania
\newcommand{\zadStop}{\end{zad}}   %oznaczenie końca zadania
%Makra opcjonarne (nie muszą występować):
\newcommand{\rozwStart}[2]{\noindent \textbf{Rozwiązanie (autor #1 , recenzent #2): }\newline} %oznaczenie początku rozwiązania, opcjonarnie można wprowadzić informację o autorze rozwiązania zadania i recenzencie poprawności wykonania rozwiązania zadania
\newcommand{\rozwStop}{\newline}                                            %oznaczenie końca rozwiązania
\newcommand{\odpStart}{\noindent \textbf{Odpowiedź:}\newline}    %oznaczenie początku odpowiedzi końcowej (wypisanie wyniku)
\newcommand{\odpStop}{\newline}                                             %oznaczenie końca odpowiedzi końcowej (wypisanie wyniku)
\newcommand{\testStart}{\noindent \textbf{Test:}\newline} %ewentualne możliwe opcje odpowiedzi testowej: A. ? B. ? C. ? D. ? itd.
\newcommand{\testStop}{\newline} %koniec wprowadzania odpowiedzi testowych
\newcommand{\kluczStart}{\noindent \textbf{Test poprawna odpowiedź:}\newline} %klucz, poprawna odpowiedź pytania testowego (jedna literka): A lub B lub C lub D itd.
\newcommand{\kluczStop}{\newline} %koniec poprawnej odpowiedzi pytania testowego 
\newcommand{\wstawGrafike}[2]{\begin{figure}[h] \includegraphics[scale=#2] {#1} \end{figure}} %gdyby była potrzeba wstawienia obrazka, parametry: nazwa pliku, skala (jak nie wiesz co wpisać, to wpisz 1)

\begin{document}
\maketitle


\kategoria{Wikieł/z1.84n}
\zadStart{Zadanie z Wikieł z1.84n) moja wersja nr [nrWersji]}
%[p1]:[2,3,4]
%[p2]:[3,4]
%[p3]:[2,3]
%[p4]:[2,3]
%[p5]:[2,3]
%[p6]:[0,1,2]
%[p7]:[2,3]
%[p8]:[0,1,2]
%[r1]=int(math.pow([p4],[p1]))
%[delta]=abs([p2]*[p2]-4*([p3]))
%[pdelta]=int(math.pow([delta],(1/2)))
%[t1]=int(([p2]-[pdelta])/2)
%[t2]=int(([p2]+[pdelta])/2)
%[c1]=math.sqrt([delta])
%[c2]=math.isqrt([delta])
%[c3]=[p2]*[p2]-4*[p3]
%[e1]=int(math.pow([p5],[p6]))
%[e2]=int(math.pow([p7],[p8]))
%[f1]=[p1]+1
%[f2]=[p1]+2
%[f3]=[p1]+3
%[f4]=[p1]+4
%[c3]>0 and not([c1]!=[c2]) and not([e1]!=[t1]) and not([e2]!=[t2]) and not([p6]!=0) and not([p5]!=[p4]) and not([p7]!=[p4])
Rozwiązać równanie:
n) $[r1]^{2x}-[p2] \cdot [r1]^{x} + [p3] = 0$
\zadStop
\rozwStart{Jacek Jabłoński}{}
$$[r1]^{2x}-[p2] \cdot [r1]^{x} + [p3] = 0$$
$$t = [r1]^x$$
$$t^2 -[p2]t + [p3] = 0 $$
$$\Delta = [delta]$$
$$\sqrt{\Delta} = [pdelta]$$
$$t_1 = \frac{[p2]-[pdelta]}{2 \cdot 1} = [t1]$$
$$t_2 = \frac{[p2]+[pdelta]}{2 \cdot 1} =[t2]$$
$$t_1 = [r1]^x \ \ \ \ t_2 = [r1]^x$$
$$[t1] = [p4]^{[p1]x} \ \ \ \ [t2] = [p4]^{[p1]x}$$
$$[p5]^{[p6]} = [p4]^{[p1]x} \ \ \ \ [p7]^{[p8]} = [p4]^{[p1]x}$$
$$[p6] = [p1]x \ \ \ \ [p8] = [p1]x$$
$$0 = x \ \ \ \ \frac{[p8]}{[p1]} = x$$
\rozwStop
\odpStart
$$ x = 0 \ \ lub \ \ x = \frac{[p8]}{[p1]} $$
\odpStop
\testStart
A. $$ x = 0 \ \ lub \ \ x = \frac{[p8]}{[p1]} $$
B. $$ x = 0 \ \ lub \ \ x = \frac{[p8]}{[f1]} $$
C. $$ x = 0 \ \ lub \ \ x = \frac{[p8]}{[f2]} $$
D. $$ x = 0 \ \ lub \ \ x = \frac{[p8]}{[f3]} $$
E. $$ x = 0 \ \ lub \ \ x = \frac{[p8]}{[f4]} $$
F. $$ x = 0 \ \ lub \ \ x = [f1] $$
G. $$ x = 0 \ \ lub \ \ x = [f2] $$
H. $$ x = 0 \ \ lub \ \ x = [f3] $$
I. $$ x = 0 \ \ lub \ \ x = [f4] $$
\testStop
\kluczStart
A
\kluczStop



\end{document}