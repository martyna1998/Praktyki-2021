\documentclass[12pt, a4paper]{article}
\usepackage[utf8]{inputenc}
\usepackage{polski}
\usepackage{amsthm}  %pakiet do tworzenia twierdzeń itp.
\usepackage{amsmath} %pakiet do niektórych symboli matematycznych
\usepackage{amssymb} %pakiet do symboli mat., np. \nsubseteq
\usepackage{amsfonts}
\usepackage{graphicx} %obsługa plików graficznych z rozszerzeniem png, jpg
\theoremstyle{definition} %styl dla definicji
\newtheorem{zad}{} 
\title{Multizestaw zadań}
\author{Radosław Grzyb}
%\date{\today}
\date{}
\newcounter{liczniksekcji}
\newcommand{\kategoria}[1]{\section{#1}} %olreślamy nazwę kateforii zadań
\newcommand{\zadStart}[1]{\begin{zad}#1\newline} %oznaczenie początku zadania
\newcommand{\zadStop}{\end{zad}}   %oznaczenie końca zadania
%Makra opcjonarne (nie muszą występować):
\newcommand{\rozwStart}[2]{\noindent \textbf{Rozwiązanie (autor #1 , recenzent #2): }\newline} %oznaczenie początku rozwiązania, opcjonarnie można wprowadzić informację o autorze rozwiązania zadania i recenzencie poprawności wykonania rozwiązania zadania
\newcommand{\rozwStop}{\newline}                                            %oznaczenie końca rozwiązania
\newcommand{\odpStart}{\noindent \textbf{Odpowiedź:}\newline}    %oznaczenie początku odpowiedzi końcowej (wypisanie wyniku)
\newcommand{\odpStop}{\newline}                                             %oznaczenie końca odpowiedzi końcowej (wypisanie wyniku)
\newcommand{\testStart}{\noindent \textbf{Test:}\newline} %ewentualne możliwe opcje odpowiedzi testowej: A. ? B. ? C. ? D. ? itd.
\newcommand{\testStop}{\newline} %koniec wprowadzania odpowiedzi testowych
\newcommand{\kluczStart}{\noindent \textbf{Test poprawna odpowiedź:}\newline} %klucz, poprawna odpowiedź pytania testowego (jedna literka): A lub B lub C lub D itd.
\newcommand{\kluczStop}{\newline} %koniec poprawnej odpowiedzi pytania testowego 
\newcommand{\wstawGrafike}[2]{\begin{figure}[h] \includegraphics[scale=#2] {#1} \end{figure}} %gdyby była potrzeba wstawienia obrazka, parametry: nazwa pliku, skala (jak nie wiesz co wpisać, to wpisz 1)
\begin{document}
\maketitle
\kategoria{Wikieł/Z1.83d}
\zadStart{Zadanie z Wikieł Z 1.83d moja wersja nr [nrWersji]}
%[p1]:[1,2,3,4,5,6,7,8,9,10,11,12]
%[p2]:[1,2,3,4,5,6,7,8,9,10,11,12]
%[p3]:[2,3]
%[g]=[p3]**5
%[delta]=int(math.pow([p2],2)-4*[p1]*(-1))
%[x1]=(-[p2]+math.sqrt([delta]))/(2*[p1])
%[x2]=(-[p2]-math.sqrt([delta]))/(2*[p1])
%[w]=[x2]-2
%[delta]>0 and (math.sqrt([delta])).is_integer() is True and ([x1]*1000).is_integer() is True and ([x2]*1000).is_integer() is True
Wyznaczyć wartości x, dla których funkcja $y=2^{[p1]x^{2}-[p2]x-6}$przyjmuje wartość $\frac{1}{32}$.
\zadStop
\rozwStart{Radosław Grzyb}{}
Oznaczamy:
$$[p3]^{[p1]x^{2}-[p2]x-6}=\frac{1}{[g]}$$
$$[p3]^{[p1]x^{2}-[p2]x-6}=[p3]^{-5}$$
Logarytmując obie strony równania otrzymujemy:\\
$$[p1]x^{2}-[p2]x-6=-5$$
$$[p1]x^{2}-[p2]x-1=0$$
Liczymy deltę:\\
$$\Delta=[p2]^2-4\cdot[p1]\cdot(-1)=[delta]$$\\
Czas na miejsca zerowe i zarazem odpowiedź:
$$x_{1}=\frac{-[p2]+\sqrt{[delta]}}{2\cdot[p1]}=[x1]$$
$$x_{2}=\frac{-[p2]-\sqrt{[delta]}}{2\cdot[p1]}=[x2]$$
\rozwStop
\odpStart
$[x1],[x2]$
\odpStop
\testStart
A.$$1$$
B.$$[x1],[x2$$
C.$$2$$
D.$$[x1],[w]$$
\testStop
\kluczStart
B
\kluczStop
\end{document}