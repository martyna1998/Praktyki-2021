\documentclass[12pt, a4paper]{article}
\usepackage[utf8]{inputenc}
\usepackage{polski}

\usepackage{amsthm}  %pakiet do tworzenia twierdzeń itp.
\usepackage{amsmath} %pakiet do niektórych symboli matematycznych
\usepackage{amssymb} %pakiet do symboli mat., np. \nsubseteq
\usepackage{amsfonts}
\usepackage{graphicx} %obsługa plików graficznych z rozszerzeniem png, jpg
\theoremstyle{definition} %styl dla definicji
\newtheorem{zad}{} 
\title{Multizestaw zadań}
\author{Robert Fidytek}
%\date{\today}
\date{}
\newcounter{liczniksekcji}
\newcommand{\kategoria}[1]{\section{#1}} %olreślamy nazwę kateforii zadań
\newcommand{\zadStart}[1]{\begin{zad}#1\newline} %oznaczenie początku zadania
\newcommand{\zadStop}{\end{zad}}   %oznaczenie końca zadania
%Makra opcjonarne (nie muszą występować):
\newcommand{\rozwStart}[2]{\noindent \textbf{Rozwiązanie (autor #1 , recenzent #2): }\newline} %oznaczenie początku rozwiązania, opcjonarnie można wprowadzić informację o autorze rozwiązania zadania i recenzencie poprawności wykonania rozwiązania zadania
\newcommand{\rozwStop}{\newline}                                            %oznaczenie końca rozwiązania
\newcommand{\odpStart}{\noindent \textbf{Odpowiedź:}\newline}    %oznaczenie początku odpowiedzi końcowej (wypisanie wyniku)
\newcommand{\odpStop}{\newline}                                             %oznaczenie końca odpowiedzi końcowej (wypisanie wyniku)
\newcommand{\testStart}{\noindent \textbf{Test:}\newline} %ewentualne możliwe opcje odpowiedzi testowej: A. ? B. ? C. ? D. ? itd.
\newcommand{\testStop}{\newline} %koniec wprowadzania odpowiedzi testowych
\newcommand{\kluczStart}{\noindent \textbf{Test poprawna odpowiedź:}\newline} %klucz, poprawna odpowiedź pytania testowego (jedna literka): A lub B lub C lub D itd.
\newcommand{\kluczStop}{\newline} %koniec poprawnej odpowiedzi pytania testowego 
\newcommand{\wstawGrafike}[2]{\begin{figure}[h] \includegraphics[scale=#2] {#1} \end{figure}} %gdyby była potrzeba wstawienia obrazka, parametry: nazwa pliku, skala (jak nie wiesz co wpisać, to wpisz 1)

\begin{document}
\maketitle


\kategoria{Wikieł/Z3.6a}
\zadStart{Zadanie z Wikieł Z 3.6 a) moja wersja nr [nrWersji]}
%[p1]:[2,3,4,5,6,7,8,9,10,12]
%[a]=random.randint(2,25)
%[b]=random.randint(2,15)
%[d]=random.randint(1,15)
%[1pd]=[p1]+[d]
%[2pd]=[p1]*2+[d]
%[3pd]=[p1]*3+[d]
%[21pd]=[2pd]-[1pd]
%[32pd]=[3pd]-[2pd]
%[a]!=[b]
Sprawdź czy ciąg ($a_{n}$) jest ciągiem geometrycznym, gdy:\\
b)$a_{n}=[a]\cdot[b]^{[p1]n+[d]}$
\zadStop
\rozwStart{Wojciech Przybylski}{Jacek Jabłoński}
$$a_{1}=[a]\cdot[b]^{[p1]\cdot1+[d]}\hspace{3mm}a_{2}=[a]\cdot[b]^{[p1]\cdot2+[d]}\hspace{3mm}a_{3}=[a]\cdot[b]^{[p1]\cdot3+[d]} $$
$$\frac{a_{2}}{a_{1}}=\frac{[a]\cdot[b]^{[2pd]}}{[a]\cdot[b]^{[1pd]}}=[b]^{[21pd]}  \hspace{3mm} \frac{a_{3}}{a_{2}}=\frac{[a]\cdot[b]^{[3pd]}}{[a]\cdot[b]^{[2pd]}}=[b]^{[32pd]}  $$
$$[b]^{[21pd]}=[b]^{[32pd]} $$ 
\rozwStop
\odpStart
ciąg ($a_{n}$) jest ciągiem geometrycznym
\odpStop
\testStart
A. Ciąg ($a_{n}$) jest ciągiem geometrycznym\\
B. Ciąg ($a_{n}$) nie jest ciągiem geometrycznym\\
C. Ciąg ($a_{n}$) jest ciągiem arytmetycznym\\
D. Nie da się określić, czy ciąg ($a_{n}$) jest ciągiem geometrycznym\\
E. To nie jest ciąg
\testStop
\kluczStart
A
\kluczStop



\end{document}