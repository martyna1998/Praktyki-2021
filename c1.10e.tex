\documentclass[12pt, a4paper]{article}
\usepackage[utf8]{inputenc}
\usepackage{polski}
\usepackage{amsthm}  %pakiet do tworzenia twierdzeń itp.
\usepackage{amsmath} %pakiet do niektórych symboli matematycznych
\usepackage{amssymb} %pakiet do symboli mat., np. \nsubseteq
\usepackage{amsfonts}
\usepackage{graphicx} %obsługa plików graficznych z rozszerzeniem png, jpg
\theoremstyle{definition} %styl dla definicji
\newtheorem{zad}{} 
\title{Multizestaw zadań}
\author{Robert Fidytek}
%\date{\today}
\date{}
\newcounter{liczniksekcji}
\newcommand{\kategoria}[1]{\section{#1}} %olreślamy nazwę kateforii zadań
\newcommand{\zadStart}[1]{\begin{zad}#1\newline} %oznaczenie początku zadania
\newcommand{\zadStop}{\end{zad}}   %oznaczenie końca zadania
%Makra opcjonarne (nie muszą występować):
\newcommand{\rozwStart}[2]{\noindent \textbf{Rozwiązanie (autor #1 , recenzent #2): }\newline} %oznaczenie początku rozwiązania, opcjonarnie można wprowadzić informację o autorze rozwiązania zadania i recenzencie poprawności wykonania rozwiązania zadania
\newcommand{\rozwStop}{\newline}                                            %oznaczenie końca rozwiązania
\newcommand{\odpStart}{\noindent \textbf{Odpowiedź:}\newline}    %oznaczenie początku odpowiedzi końcowej (wypisanie wyniku)
\newcommand{\odpStop}{\newline}                                             %oznaczenie końca odpowiedzi końcowej (wypisanie wyniku)
\newcommand{\testStart}{\noindent \textbf{Test:}\newline} %ewentualne możliwe opcje odpowiedzi testowej: A. ? B. ? C. ? D. ? itd.
\newcommand{\testStop}{\newline} %koniec wprowadzania odpowiedzi testowych
\newcommand{\kluczStart}{\noindent \textbf{Test poprawna odpowiedź:}\newline} %klucz, poprawna odpowiedź pytania testowego (jedna literka): A lub B lub C lub D itd.
\newcommand{\kluczStop}{\newline} %koniec poprawnej odpowiedzi pytania testowego 
\newcommand{\wstawGrafike}[2]{\begin{figure}[h] \includegraphics[scale=#2] {#1} \end{figure}} %gdyby była potrzeba wstawienia obrazka, parametry: nazwa pliku, skala (jak nie wiesz co wpisać, to wpisz 1)

\begin{document}
\maketitle


\kategoria{Wikieł/C1.10e}
\zadStart{Zadanie z Wikieł C1.10 e) moja wersja nr [nrWersji]}
%[a]:[2,3,4,5,6,7,8,9,10]
Obliczyć całki funkcji trygonometrycznej.\\
$\int [a] ctg^{3}(x) dx$\\
\zadStop
\rozwStart{Martyna Czarnobaj}{}
	$\int [a] ctg^{3}(x) dx = |ctg(x) = \frac{1}{tg(x)}| = [a] \int \frac{1}{tg^{3}(x)} dx = [a] \int \frac{1}{cos^{2}(x)tg^{3}(x)(tg^{2}(x)+1)} dx = |u=tg(x), du=\frac{1}{cos^{2}(x)} dx| = [a] \int \frac{1}{u^{5}+u^{3}} du = [a] \int \frac{1}{u^{3}(u^{2}+1)} du = [a] \int \frac{u}{u^{2}+1} - \frac{1}{u} + \frac{1}{u^{3}} du = [a] (\int \frac{u}{u^{2}+1} du - \int \frac{1}{u} du + \int \frac{1}{u^{3}} du) = [a] (\frac{\ln (u^{2}+1)}{2} - \ln(u) + (-\frac{1}{2u^{2}})) = [a]\frac{\ln (u^{2}+1)}{2} - [a]\ln(u) - [a]\frac{1}{2u^{2}} = [a]\frac{\ln (tg^{2}(x)+1)}{2} - [a]\ln(tg(x)) - [a]\frac{1}{2tg^{2}(x)}$\\ 


Koniec rozwiązania.\\
\rozwStop
\odpStart
$[a]\frac{\ln (tg^{2}(x)+1)}{2} - [a]\ln(tg(x)) - [a]\frac{1}{2tg^{2}(x)}$\\
\odpStop
\testStart
A.$[a]\frac{\ln (tg^{2}(x)+1)}{2} - [a]\ln(tg(x)) - [a]\frac{1}{2tg^{2}(x)}$\\
B.$[a]\frac{\ln (ctg^{2}(x)+1)}{2} - [a]\ln(tg(x)) - [a]\frac{1}{2tg^{2}(x)}$\\
C.$[a]\frac{\ln (tg^{2}(x)+1)}{2} - [a]\ln(ctg(x)) - [a]\frac{1}{2tg^{2}(x)}$\\
.
\testStop
\kluczStart
A
\kluczStop
\end{document}