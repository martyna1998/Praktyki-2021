\documentclass[12pt, a4paper]{article}
\usepackage[utf8]{inputenc}
\usepackage{polski}
\usepackage{amsthm}  %pakiet do tworzenia twierdzeń itp.
\usepackage{amsmath} %pakiet do niektórych symboli matematycznych
\usepackage{amssymb} %pakiet do symboli mat., np. \nsubseteq
\usepackage{amsfonts}
\usepackage{graphicx} %obsługa plików graficznych z rozszerzeniem png, jpg
\theoremstyle{definition} %styl dla definicji
\newtheorem{zad}{} 
\title{Multizestaw zadań}
\author{Patryk Wirkus}
%\date{\today}
\date{}
\newcommand{\kategoria}[1]{\section{#1}}
\newcommand{\zadStart}[1]{\begin{zad}#1\newline}
\newcommand{\zadStop}{\end{zad}}
\newcommand{\rozwStart}[2]{\noindent \textbf{Rozwiązanie (autor #1 , recenzent #2): }\newline}
\newcommand{\rozwStop}{\newline}                                           
\newcommand{\odpStart}{\noindent \textbf{Odpowiedź:}\newline}
\newcommand{\odpStop}{\newline}
\newcommand{\testStart}{\noindent \textbf{Test:}\newline}
\newcommand{\testStop}{\newline}
\newcommand{\kluczStart}{\noindent \textbf{Test poprawna odpowiedź:}\newline}
\newcommand{\kluczStop}{\newline}
\newcommand{\wstawGrafike}[2]{\begin{figure}[h] \includegraphics[scale=#2] {#1} \end{figure}}

\begin{document}
\maketitle

\kategoria{Wikieł/1.62a}


\zadStart{Zadanie z Wikieł Z 1.62 a) moja wersja nr 1}

Rozwiązać nierówności $(x-1)(x-2)(x-3)\ge0$.
\zadStop
\rozwStart{Patryk Wirkus}{}
Miejsca zerowe naszego wielomianu to: $1, 2, 3$.\\
Wielomian jest stopnia nieparzystego, ponadto znak współczynnika przy\linebreak najwyższej potędze x jest dodatni.\\ W związku z tym wykres wielomianu zaczyna się od lewej strony poniżej osi OX. A więc $$x \in [1,2] \cup [3,\infty).$$
\rozwStop
\odpStart
$x \in [1,2] \cup [3,\infty)$
\odpStop
\testStart
A.$x \in [1,2] \cup [3,\infty)$\\
B.$x \in (1,2) \cup [3,\infty)$\\
C.$x \in (1,2] \cup [3,\infty)$\\
D.$x \in [1,2) \cup [3,\infty)$\\
E.$x \in [1,2] \cup (3,\infty)$\\
F.$x \in (1,2) \cup (3,\infty)$\\
G.$x \in [1,2) \cup (3,\infty)$\\
H.$x \in (1,2] \cup (3,\infty)$
\testStop
\kluczStart
A
\kluczStop



\zadStart{Zadanie z Wikieł Z 1.62 a) moja wersja nr 2}

Rozwiązać nierówności $(x-1)(x-2)(x-4)\ge0$.
\zadStop
\rozwStart{Patryk Wirkus}{}
Miejsca zerowe naszego wielomianu to: $1, 2, 4$.\\
Wielomian jest stopnia nieparzystego, ponadto znak współczynnika przy\linebreak najwyższej potędze x jest dodatni.\\ W związku z tym wykres wielomianu zaczyna się od lewej strony poniżej osi OX. A więc $$x \in [1,2] \cup [4,\infty).$$
\rozwStop
\odpStart
$x \in [1,2] \cup [4,\infty)$
\odpStop
\testStart
A.$x \in [1,2] \cup [4,\infty)$\\
B.$x \in (1,2) \cup [4,\infty)$\\
C.$x \in (1,2] \cup [4,\infty)$\\
D.$x \in [1,2) \cup [4,\infty)$\\
E.$x \in [1,2] \cup (4,\infty)$\\
F.$x \in (1,2) \cup (4,\infty)$\\
G.$x \in [1,2) \cup (4,\infty)$\\
H.$x \in (1,2] \cup (4,\infty)$
\testStop
\kluczStart
A
\kluczStop



\zadStart{Zadanie z Wikieł Z 1.62 a) moja wersja nr 3}

Rozwiązać nierówności $(x-1)(x-2)(x-5)\ge0$.
\zadStop
\rozwStart{Patryk Wirkus}{}
Miejsca zerowe naszego wielomianu to: $1, 2, 5$.\\
Wielomian jest stopnia nieparzystego, ponadto znak współczynnika przy\linebreak najwyższej potędze x jest dodatni.\\ W związku z tym wykres wielomianu zaczyna się od lewej strony poniżej osi OX. A więc $$x \in [1,2] \cup [5,\infty).$$
\rozwStop
\odpStart
$x \in [1,2] \cup [5,\infty)$
\odpStop
\testStart
A.$x \in [1,2] \cup [5,\infty)$\\
B.$x \in (1,2) \cup [5,\infty)$\\
C.$x \in (1,2] \cup [5,\infty)$\\
D.$x \in [1,2) \cup [5,\infty)$\\
E.$x \in [1,2] \cup (5,\infty)$\\
F.$x \in (1,2) \cup (5,\infty)$\\
G.$x \in [1,2) \cup (5,\infty)$\\
H.$x \in (1,2] \cup (5,\infty)$
\testStop
\kluczStart
A
\kluczStop



\zadStart{Zadanie z Wikieł Z 1.62 a) moja wersja nr 4}

Rozwiązać nierówności $(x-1)(x-2)(x-6)\ge0$.
\zadStop
\rozwStart{Patryk Wirkus}{}
Miejsca zerowe naszego wielomianu to: $1, 2, 6$.\\
Wielomian jest stopnia nieparzystego, ponadto znak współczynnika przy\linebreak najwyższej potędze x jest dodatni.\\ W związku z tym wykres wielomianu zaczyna się od lewej strony poniżej osi OX. A więc $$x \in [1,2] \cup [6,\infty).$$
\rozwStop
\odpStart
$x \in [1,2] \cup [6,\infty)$
\odpStop
\testStart
A.$x \in [1,2] \cup [6,\infty)$\\
B.$x \in (1,2) \cup [6,\infty)$\\
C.$x \in (1,2] \cup [6,\infty)$\\
D.$x \in [1,2) \cup [6,\infty)$\\
E.$x \in [1,2] \cup (6,\infty)$\\
F.$x \in (1,2) \cup (6,\infty)$\\
G.$x \in [1,2) \cup (6,\infty)$\\
H.$x \in (1,2] \cup (6,\infty)$
\testStop
\kluczStart
A
\kluczStop



\zadStart{Zadanie z Wikieł Z 1.62 a) moja wersja nr 5}

Rozwiązać nierówności $(x-1)(x-2)(x-7)\ge0$.
\zadStop
\rozwStart{Patryk Wirkus}{}
Miejsca zerowe naszego wielomianu to: $1, 2, 7$.\\
Wielomian jest stopnia nieparzystego, ponadto znak współczynnika przy\linebreak najwyższej potędze x jest dodatni.\\ W związku z tym wykres wielomianu zaczyna się od lewej strony poniżej osi OX. A więc $$x \in [1,2] \cup [7,\infty).$$
\rozwStop
\odpStart
$x \in [1,2] \cup [7,\infty)$
\odpStop
\testStart
A.$x \in [1,2] \cup [7,\infty)$\\
B.$x \in (1,2) \cup [7,\infty)$\\
C.$x \in (1,2] \cup [7,\infty)$\\
D.$x \in [1,2) \cup [7,\infty)$\\
E.$x \in [1,2] \cup (7,\infty)$\\
F.$x \in (1,2) \cup (7,\infty)$\\
G.$x \in [1,2) \cup (7,\infty)$\\
H.$x \in (1,2] \cup (7,\infty)$
\testStop
\kluczStart
A
\kluczStop



\zadStart{Zadanie z Wikieł Z 1.62 a) moja wersja nr 6}

Rozwiązać nierówności $(x-1)(x-2)(x-8)\ge0$.
\zadStop
\rozwStart{Patryk Wirkus}{}
Miejsca zerowe naszego wielomianu to: $1, 2, 8$.\\
Wielomian jest stopnia nieparzystego, ponadto znak współczynnika przy\linebreak najwyższej potędze x jest dodatni.\\ W związku z tym wykres wielomianu zaczyna się od lewej strony poniżej osi OX. A więc $$x \in [1,2] \cup [8,\infty).$$
\rozwStop
\odpStart
$x \in [1,2] \cup [8,\infty)$
\odpStop
\testStart
A.$x \in [1,2] \cup [8,\infty)$\\
B.$x \in (1,2) \cup [8,\infty)$\\
C.$x \in (1,2] \cup [8,\infty)$\\
D.$x \in [1,2) \cup [8,\infty)$\\
E.$x \in [1,2] \cup (8,\infty)$\\
F.$x \in (1,2) \cup (8,\infty)$\\
G.$x \in [1,2) \cup (8,\infty)$\\
H.$x \in (1,2] \cup (8,\infty)$
\testStop
\kluczStart
A
\kluczStop



\zadStart{Zadanie z Wikieł Z 1.62 a) moja wersja nr 7}

Rozwiązać nierówności $(x-1)(x-2)(x-9)\ge0$.
\zadStop
\rozwStart{Patryk Wirkus}{}
Miejsca zerowe naszego wielomianu to: $1, 2, 9$.\\
Wielomian jest stopnia nieparzystego, ponadto znak współczynnika przy\linebreak najwyższej potędze x jest dodatni.\\ W związku z tym wykres wielomianu zaczyna się od lewej strony poniżej osi OX. A więc $$x \in [1,2] \cup [9,\infty).$$
\rozwStop
\odpStart
$x \in [1,2] \cup [9,\infty)$
\odpStop
\testStart
A.$x \in [1,2] \cup [9,\infty)$\\
B.$x \in (1,2) \cup [9,\infty)$\\
C.$x \in (1,2] \cup [9,\infty)$\\
D.$x \in [1,2) \cup [9,\infty)$\\
E.$x \in [1,2] \cup (9,\infty)$\\
F.$x \in (1,2) \cup (9,\infty)$\\
G.$x \in [1,2) \cup (9,\infty)$\\
H.$x \in (1,2] \cup (9,\infty)$
\testStop
\kluczStart
A
\kluczStop



\zadStart{Zadanie z Wikieł Z 1.62 a) moja wersja nr 8}

Rozwiązać nierówności $(x-1)(x-2)(x-10)\ge0$.
\zadStop
\rozwStart{Patryk Wirkus}{}
Miejsca zerowe naszego wielomianu to: $1, 2, 10$.\\
Wielomian jest stopnia nieparzystego, ponadto znak współczynnika przy\linebreak najwyższej potędze x jest dodatni.\\ W związku z tym wykres wielomianu zaczyna się od lewej strony poniżej osi OX. A więc $$x \in [1,2] \cup [10,\infty).$$
\rozwStop
\odpStart
$x \in [1,2] \cup [10,\infty)$
\odpStop
\testStart
A.$x \in [1,2] \cup [10,\infty)$\\
B.$x \in (1,2) \cup [10,\infty)$\\
C.$x \in (1,2] \cup [10,\infty)$\\
D.$x \in [1,2) \cup [10,\infty)$\\
E.$x \in [1,2] \cup (10,\infty)$\\
F.$x \in (1,2) \cup (10,\infty)$\\
G.$x \in [1,2) \cup (10,\infty)$\\
H.$x \in (1,2] \cup (10,\infty)$
\testStop
\kluczStart
A
\kluczStop



\zadStart{Zadanie z Wikieł Z 1.62 a) moja wersja nr 9}

Rozwiązać nierówności $(x-1)(x-2)(x-11)\ge0$.
\zadStop
\rozwStart{Patryk Wirkus}{}
Miejsca zerowe naszego wielomianu to: $1, 2, 11$.\\
Wielomian jest stopnia nieparzystego, ponadto znak współczynnika przy\linebreak najwyższej potędze x jest dodatni.\\ W związku z tym wykres wielomianu zaczyna się od lewej strony poniżej osi OX. A więc $$x \in [1,2] \cup [11,\infty).$$
\rozwStop
\odpStart
$x \in [1,2] \cup [11,\infty)$
\odpStop
\testStart
A.$x \in [1,2] \cup [11,\infty)$\\
B.$x \in (1,2) \cup [11,\infty)$\\
C.$x \in (1,2] \cup [11,\infty)$\\
D.$x \in [1,2) \cup [11,\infty)$\\
E.$x \in [1,2] \cup (11,\infty)$\\
F.$x \in (1,2) \cup (11,\infty)$\\
G.$x \in [1,2) \cup (11,\infty)$\\
H.$x \in (1,2] \cup (11,\infty)$
\testStop
\kluczStart
A
\kluczStop



\zadStart{Zadanie z Wikieł Z 1.62 a) moja wersja nr 10}

Rozwiązać nierówności $(x-1)(x-2)(x-12)\ge0$.
\zadStop
\rozwStart{Patryk Wirkus}{}
Miejsca zerowe naszego wielomianu to: $1, 2, 12$.\\
Wielomian jest stopnia nieparzystego, ponadto znak współczynnika przy\linebreak najwyższej potędze x jest dodatni.\\ W związku z tym wykres wielomianu zaczyna się od lewej strony poniżej osi OX. A więc $$x \in [1,2] \cup [12,\infty).$$
\rozwStop
\odpStart
$x \in [1,2] \cup [12,\infty)$
\odpStop
\testStart
A.$x \in [1,2] \cup [12,\infty)$\\
B.$x \in (1,2) \cup [12,\infty)$\\
C.$x \in (1,2] \cup [12,\infty)$\\
D.$x \in [1,2) \cup [12,\infty)$\\
E.$x \in [1,2] \cup (12,\infty)$\\
F.$x \in (1,2) \cup (12,\infty)$\\
G.$x \in [1,2) \cup (12,\infty)$\\
H.$x \in (1,2] \cup (12,\infty)$
\testStop
\kluczStart
A
\kluczStop



\zadStart{Zadanie z Wikieł Z 1.62 a) moja wersja nr 11}

Rozwiązać nierówności $(x-1)(x-2)(x-13)\ge0$.
\zadStop
\rozwStart{Patryk Wirkus}{}
Miejsca zerowe naszego wielomianu to: $1, 2, 13$.\\
Wielomian jest stopnia nieparzystego, ponadto znak współczynnika przy\linebreak najwyższej potędze x jest dodatni.\\ W związku z tym wykres wielomianu zaczyna się od lewej strony poniżej osi OX. A więc $$x \in [1,2] \cup [13,\infty).$$
\rozwStop
\odpStart
$x \in [1,2] \cup [13,\infty)$
\odpStop
\testStart
A.$x \in [1,2] \cup [13,\infty)$\\
B.$x \in (1,2) \cup [13,\infty)$\\
C.$x \in (1,2] \cup [13,\infty)$\\
D.$x \in [1,2) \cup [13,\infty)$\\
E.$x \in [1,2] \cup (13,\infty)$\\
F.$x \in (1,2) \cup (13,\infty)$\\
G.$x \in [1,2) \cup (13,\infty)$\\
H.$x \in (1,2] \cup (13,\infty)$
\testStop
\kluczStart
A
\kluczStop



\zadStart{Zadanie z Wikieł Z 1.62 a) moja wersja nr 12}

Rozwiązać nierówności $(x-1)(x-2)(x-14)\ge0$.
\zadStop
\rozwStart{Patryk Wirkus}{}
Miejsca zerowe naszego wielomianu to: $1, 2, 14$.\\
Wielomian jest stopnia nieparzystego, ponadto znak współczynnika przy\linebreak najwyższej potędze x jest dodatni.\\ W związku z tym wykres wielomianu zaczyna się od lewej strony poniżej osi OX. A więc $$x \in [1,2] \cup [14,\infty).$$
\rozwStop
\odpStart
$x \in [1,2] \cup [14,\infty)$
\odpStop
\testStart
A.$x \in [1,2] \cup [14,\infty)$\\
B.$x \in (1,2) \cup [14,\infty)$\\
C.$x \in (1,2] \cup [14,\infty)$\\
D.$x \in [1,2) \cup [14,\infty)$\\
E.$x \in [1,2] \cup (14,\infty)$\\
F.$x \in (1,2) \cup (14,\infty)$\\
G.$x \in [1,2) \cup (14,\infty)$\\
H.$x \in (1,2] \cup (14,\infty)$
\testStop
\kluczStart
A
\kluczStop



\zadStart{Zadanie z Wikieł Z 1.62 a) moja wersja nr 13}

Rozwiązać nierówności $(x-1)(x-2)(x-15)\ge0$.
\zadStop
\rozwStart{Patryk Wirkus}{}
Miejsca zerowe naszego wielomianu to: $1, 2, 15$.\\
Wielomian jest stopnia nieparzystego, ponadto znak współczynnika przy\linebreak najwyższej potędze x jest dodatni.\\ W związku z tym wykres wielomianu zaczyna się od lewej strony poniżej osi OX. A więc $$x \in [1,2] \cup [15,\infty).$$
\rozwStop
\odpStart
$x \in [1,2] \cup [15,\infty)$
\odpStop
\testStart
A.$x \in [1,2] \cup [15,\infty)$\\
B.$x \in (1,2) \cup [15,\infty)$\\
C.$x \in (1,2] \cup [15,\infty)$\\
D.$x \in [1,2) \cup [15,\infty)$\\
E.$x \in [1,2] \cup (15,\infty)$\\
F.$x \in (1,2) \cup (15,\infty)$\\
G.$x \in [1,2) \cup (15,\infty)$\\
H.$x \in (1,2] \cup (15,\infty)$
\testStop
\kluczStart
A
\kluczStop



\zadStart{Zadanie z Wikieł Z 1.62 a) moja wersja nr 14}

Rozwiązać nierówności $(x-1)(x-2)(x-16)\ge0$.
\zadStop
\rozwStart{Patryk Wirkus}{}
Miejsca zerowe naszego wielomianu to: $1, 2, 16$.\\
Wielomian jest stopnia nieparzystego, ponadto znak współczynnika przy\linebreak najwyższej potędze x jest dodatni.\\ W związku z tym wykres wielomianu zaczyna się od lewej strony poniżej osi OX. A więc $$x \in [1,2] \cup [16,\infty).$$
\rozwStop
\odpStart
$x \in [1,2] \cup [16,\infty)$
\odpStop
\testStart
A.$x \in [1,2] \cup [16,\infty)$\\
B.$x \in (1,2) \cup [16,\infty)$\\
C.$x \in (1,2] \cup [16,\infty)$\\
D.$x \in [1,2) \cup [16,\infty)$\\
E.$x \in [1,2] \cup (16,\infty)$\\
F.$x \in (1,2) \cup (16,\infty)$\\
G.$x \in [1,2) \cup (16,\infty)$\\
H.$x \in (1,2] \cup (16,\infty)$
\testStop
\kluczStart
A
\kluczStop



\zadStart{Zadanie z Wikieł Z 1.62 a) moja wersja nr 15}

Rozwiązać nierówności $(x-1)(x-2)(x-17)\ge0$.
\zadStop
\rozwStart{Patryk Wirkus}{}
Miejsca zerowe naszego wielomianu to: $1, 2, 17$.\\
Wielomian jest stopnia nieparzystego, ponadto znak współczynnika przy\linebreak najwyższej potędze x jest dodatni.\\ W związku z tym wykres wielomianu zaczyna się od lewej strony poniżej osi OX. A więc $$x \in [1,2] \cup [17,\infty).$$
\rozwStop
\odpStart
$x \in [1,2] \cup [17,\infty)$
\odpStop
\testStart
A.$x \in [1,2] \cup [17,\infty)$\\
B.$x \in (1,2) \cup [17,\infty)$\\
C.$x \in (1,2] \cup [17,\infty)$\\
D.$x \in [1,2) \cup [17,\infty)$\\
E.$x \in [1,2] \cup (17,\infty)$\\
F.$x \in (1,2) \cup (17,\infty)$\\
G.$x \in [1,2) \cup (17,\infty)$\\
H.$x \in (1,2] \cup (17,\infty)$
\testStop
\kluczStart
A
\kluczStop



\zadStart{Zadanie z Wikieł Z 1.62 a) moja wersja nr 16}

Rozwiązać nierówności $(x-1)(x-2)(x-18)\ge0$.
\zadStop
\rozwStart{Patryk Wirkus}{}
Miejsca zerowe naszego wielomianu to: $1, 2, 18$.\\
Wielomian jest stopnia nieparzystego, ponadto znak współczynnika przy\linebreak najwyższej potędze x jest dodatni.\\ W związku z tym wykres wielomianu zaczyna się od lewej strony poniżej osi OX. A więc $$x \in [1,2] \cup [18,\infty).$$
\rozwStop
\odpStart
$x \in [1,2] \cup [18,\infty)$
\odpStop
\testStart
A.$x \in [1,2] \cup [18,\infty)$\\
B.$x \in (1,2) \cup [18,\infty)$\\
C.$x \in (1,2] \cup [18,\infty)$\\
D.$x \in [1,2) \cup [18,\infty)$\\
E.$x \in [1,2] \cup (18,\infty)$\\
F.$x \in (1,2) \cup (18,\infty)$\\
G.$x \in [1,2) \cup (18,\infty)$\\
H.$x \in (1,2] \cup (18,\infty)$
\testStop
\kluczStart
A
\kluczStop



\zadStart{Zadanie z Wikieł Z 1.62 a) moja wersja nr 17}

Rozwiązać nierówności $(x-1)(x-2)(x-19)\ge0$.
\zadStop
\rozwStart{Patryk Wirkus}{}
Miejsca zerowe naszego wielomianu to: $1, 2, 19$.\\
Wielomian jest stopnia nieparzystego, ponadto znak współczynnika przy\linebreak najwyższej potędze x jest dodatni.\\ W związku z tym wykres wielomianu zaczyna się od lewej strony poniżej osi OX. A więc $$x \in [1,2] \cup [19,\infty).$$
\rozwStop
\odpStart
$x \in [1,2] \cup [19,\infty)$
\odpStop
\testStart
A.$x \in [1,2] \cup [19,\infty)$\\
B.$x \in (1,2) \cup [19,\infty)$\\
C.$x \in (1,2] \cup [19,\infty)$\\
D.$x \in [1,2) \cup [19,\infty)$\\
E.$x \in [1,2] \cup (19,\infty)$\\
F.$x \in (1,2) \cup (19,\infty)$\\
G.$x \in [1,2) \cup (19,\infty)$\\
H.$x \in (1,2] \cup (19,\infty)$
\testStop
\kluczStart
A
\kluczStop



\zadStart{Zadanie z Wikieł Z 1.62 a) moja wersja nr 18}

Rozwiązać nierówności $(x-1)(x-2)(x-20)\ge0$.
\zadStop
\rozwStart{Patryk Wirkus}{}
Miejsca zerowe naszego wielomianu to: $1, 2, 20$.\\
Wielomian jest stopnia nieparzystego, ponadto znak współczynnika przy\linebreak najwyższej potędze x jest dodatni.\\ W związku z tym wykres wielomianu zaczyna się od lewej strony poniżej osi OX. A więc $$x \in [1,2] \cup [20,\infty).$$
\rozwStop
\odpStart
$x \in [1,2] \cup [20,\infty)$
\odpStop
\testStart
A.$x \in [1,2] \cup [20,\infty)$\\
B.$x \in (1,2) \cup [20,\infty)$\\
C.$x \in (1,2] \cup [20,\infty)$\\
D.$x \in [1,2) \cup [20,\infty)$\\
E.$x \in [1,2] \cup (20,\infty)$\\
F.$x \in (1,2) \cup (20,\infty)$\\
G.$x \in [1,2) \cup (20,\infty)$\\
H.$x \in (1,2] \cup (20,\infty)$
\testStop
\kluczStart
A
\kluczStop



\zadStart{Zadanie z Wikieł Z 1.62 a) moja wersja nr 19}

Rozwiązać nierówności $(x-1)(x-3)(x-4)\ge0$.
\zadStop
\rozwStart{Patryk Wirkus}{}
Miejsca zerowe naszego wielomianu to: $1, 3, 4$.\\
Wielomian jest stopnia nieparzystego, ponadto znak współczynnika przy\linebreak najwyższej potędze x jest dodatni.\\ W związku z tym wykres wielomianu zaczyna się od lewej strony poniżej osi OX. A więc $$x \in [1,3] \cup [4,\infty).$$
\rozwStop
\odpStart
$x \in [1,3] \cup [4,\infty)$
\odpStop
\testStart
A.$x \in [1,3] \cup [4,\infty)$\\
B.$x \in (1,3) \cup [4,\infty)$\\
C.$x \in (1,3] \cup [4,\infty)$\\
D.$x \in [1,3) \cup [4,\infty)$\\
E.$x \in [1,3] \cup (4,\infty)$\\
F.$x \in (1,3) \cup (4,\infty)$\\
G.$x \in [1,3) \cup (4,\infty)$\\
H.$x \in (1,3] \cup (4,\infty)$
\testStop
\kluczStart
A
\kluczStop



\zadStart{Zadanie z Wikieł Z 1.62 a) moja wersja nr 20}

Rozwiązać nierówności $(x-1)(x-3)(x-5)\ge0$.
\zadStop
\rozwStart{Patryk Wirkus}{}
Miejsca zerowe naszego wielomianu to: $1, 3, 5$.\\
Wielomian jest stopnia nieparzystego, ponadto znak współczynnika przy\linebreak najwyższej potędze x jest dodatni.\\ W związku z tym wykres wielomianu zaczyna się od lewej strony poniżej osi OX. A więc $$x \in [1,3] \cup [5,\infty).$$
\rozwStop
\odpStart
$x \in [1,3] \cup [5,\infty)$
\odpStop
\testStart
A.$x \in [1,3] \cup [5,\infty)$\\
B.$x \in (1,3) \cup [5,\infty)$\\
C.$x \in (1,3] \cup [5,\infty)$\\
D.$x \in [1,3) \cup [5,\infty)$\\
E.$x \in [1,3] \cup (5,\infty)$\\
F.$x \in (1,3) \cup (5,\infty)$\\
G.$x \in [1,3) \cup (5,\infty)$\\
H.$x \in (1,3] \cup (5,\infty)$
\testStop
\kluczStart
A
\kluczStop



\zadStart{Zadanie z Wikieł Z 1.62 a) moja wersja nr 21}

Rozwiązać nierówności $(x-1)(x-3)(x-6)\ge0$.
\zadStop
\rozwStart{Patryk Wirkus}{}
Miejsca zerowe naszego wielomianu to: $1, 3, 6$.\\
Wielomian jest stopnia nieparzystego, ponadto znak współczynnika przy\linebreak najwyższej potędze x jest dodatni.\\ W związku z tym wykres wielomianu zaczyna się od lewej strony poniżej osi OX. A więc $$x \in [1,3] \cup [6,\infty).$$
\rozwStop
\odpStart
$x \in [1,3] \cup [6,\infty)$
\odpStop
\testStart
A.$x \in [1,3] \cup [6,\infty)$\\
B.$x \in (1,3) \cup [6,\infty)$\\
C.$x \in (1,3] \cup [6,\infty)$\\
D.$x \in [1,3) \cup [6,\infty)$\\
E.$x \in [1,3] \cup (6,\infty)$\\
F.$x \in (1,3) \cup (6,\infty)$\\
G.$x \in [1,3) \cup (6,\infty)$\\
H.$x \in (1,3] \cup (6,\infty)$
\testStop
\kluczStart
A
\kluczStop



\zadStart{Zadanie z Wikieł Z 1.62 a) moja wersja nr 22}

Rozwiązać nierówności $(x-1)(x-3)(x-7)\ge0$.
\zadStop
\rozwStart{Patryk Wirkus}{}
Miejsca zerowe naszego wielomianu to: $1, 3, 7$.\\
Wielomian jest stopnia nieparzystego, ponadto znak współczynnika przy\linebreak najwyższej potędze x jest dodatni.\\ W związku z tym wykres wielomianu zaczyna się od lewej strony poniżej osi OX. A więc $$x \in [1,3] \cup [7,\infty).$$
\rozwStop
\odpStart
$x \in [1,3] \cup [7,\infty)$
\odpStop
\testStart
A.$x \in [1,3] \cup [7,\infty)$\\
B.$x \in (1,3) \cup [7,\infty)$\\
C.$x \in (1,3] \cup [7,\infty)$\\
D.$x \in [1,3) \cup [7,\infty)$\\
E.$x \in [1,3] \cup (7,\infty)$\\
F.$x \in (1,3) \cup (7,\infty)$\\
G.$x \in [1,3) \cup (7,\infty)$\\
H.$x \in (1,3] \cup (7,\infty)$
\testStop
\kluczStart
A
\kluczStop



\zadStart{Zadanie z Wikieł Z 1.62 a) moja wersja nr 23}

Rozwiązać nierówności $(x-1)(x-3)(x-8)\ge0$.
\zadStop
\rozwStart{Patryk Wirkus}{}
Miejsca zerowe naszego wielomianu to: $1, 3, 8$.\\
Wielomian jest stopnia nieparzystego, ponadto znak współczynnika przy\linebreak najwyższej potędze x jest dodatni.\\ W związku z tym wykres wielomianu zaczyna się od lewej strony poniżej osi OX. A więc $$x \in [1,3] \cup [8,\infty).$$
\rozwStop
\odpStart
$x \in [1,3] \cup [8,\infty)$
\odpStop
\testStart
A.$x \in [1,3] \cup [8,\infty)$\\
B.$x \in (1,3) \cup [8,\infty)$\\
C.$x \in (1,3] \cup [8,\infty)$\\
D.$x \in [1,3) \cup [8,\infty)$\\
E.$x \in [1,3] \cup (8,\infty)$\\
F.$x \in (1,3) \cup (8,\infty)$\\
G.$x \in [1,3) \cup (8,\infty)$\\
H.$x \in (1,3] \cup (8,\infty)$
\testStop
\kluczStart
A
\kluczStop



\zadStart{Zadanie z Wikieł Z 1.62 a) moja wersja nr 24}

Rozwiązać nierówności $(x-1)(x-3)(x-9)\ge0$.
\zadStop
\rozwStart{Patryk Wirkus}{}
Miejsca zerowe naszego wielomianu to: $1, 3, 9$.\\
Wielomian jest stopnia nieparzystego, ponadto znak współczynnika przy\linebreak najwyższej potędze x jest dodatni.\\ W związku z tym wykres wielomianu zaczyna się od lewej strony poniżej osi OX. A więc $$x \in [1,3] \cup [9,\infty).$$
\rozwStop
\odpStart
$x \in [1,3] \cup [9,\infty)$
\odpStop
\testStart
A.$x \in [1,3] \cup [9,\infty)$\\
B.$x \in (1,3) \cup [9,\infty)$\\
C.$x \in (1,3] \cup [9,\infty)$\\
D.$x \in [1,3) \cup [9,\infty)$\\
E.$x \in [1,3] \cup (9,\infty)$\\
F.$x \in (1,3) \cup (9,\infty)$\\
G.$x \in [1,3) \cup (9,\infty)$\\
H.$x \in (1,3] \cup (9,\infty)$
\testStop
\kluczStart
A
\kluczStop



\zadStart{Zadanie z Wikieł Z 1.62 a) moja wersja nr 25}

Rozwiązać nierówności $(x-1)(x-3)(x-10)\ge0$.
\zadStop
\rozwStart{Patryk Wirkus}{}
Miejsca zerowe naszego wielomianu to: $1, 3, 10$.\\
Wielomian jest stopnia nieparzystego, ponadto znak współczynnika przy\linebreak najwyższej potędze x jest dodatni.\\ W związku z tym wykres wielomianu zaczyna się od lewej strony poniżej osi OX. A więc $$x \in [1,3] \cup [10,\infty).$$
\rozwStop
\odpStart
$x \in [1,3] \cup [10,\infty)$
\odpStop
\testStart
A.$x \in [1,3] \cup [10,\infty)$\\
B.$x \in (1,3) \cup [10,\infty)$\\
C.$x \in (1,3] \cup [10,\infty)$\\
D.$x \in [1,3) \cup [10,\infty)$\\
E.$x \in [1,3] \cup (10,\infty)$\\
F.$x \in (1,3) \cup (10,\infty)$\\
G.$x \in [1,3) \cup (10,\infty)$\\
H.$x \in (1,3] \cup (10,\infty)$
\testStop
\kluczStart
A
\kluczStop



\zadStart{Zadanie z Wikieł Z 1.62 a) moja wersja nr 26}

Rozwiązać nierówności $(x-1)(x-3)(x-11)\ge0$.
\zadStop
\rozwStart{Patryk Wirkus}{}
Miejsca zerowe naszego wielomianu to: $1, 3, 11$.\\
Wielomian jest stopnia nieparzystego, ponadto znak współczynnika przy\linebreak najwyższej potędze x jest dodatni.\\ W związku z tym wykres wielomianu zaczyna się od lewej strony poniżej osi OX. A więc $$x \in [1,3] \cup [11,\infty).$$
\rozwStop
\odpStart
$x \in [1,3] \cup [11,\infty)$
\odpStop
\testStart
A.$x \in [1,3] \cup [11,\infty)$\\
B.$x \in (1,3) \cup [11,\infty)$\\
C.$x \in (1,3] \cup [11,\infty)$\\
D.$x \in [1,3) \cup [11,\infty)$\\
E.$x \in [1,3] \cup (11,\infty)$\\
F.$x \in (1,3) \cup (11,\infty)$\\
G.$x \in [1,3) \cup (11,\infty)$\\
H.$x \in (1,3] \cup (11,\infty)$
\testStop
\kluczStart
A
\kluczStop



\zadStart{Zadanie z Wikieł Z 1.62 a) moja wersja nr 27}

Rozwiązać nierówności $(x-1)(x-3)(x-12)\ge0$.
\zadStop
\rozwStart{Patryk Wirkus}{}
Miejsca zerowe naszego wielomianu to: $1, 3, 12$.\\
Wielomian jest stopnia nieparzystego, ponadto znak współczynnika przy\linebreak najwyższej potędze x jest dodatni.\\ W związku z tym wykres wielomianu zaczyna się od lewej strony poniżej osi OX. A więc $$x \in [1,3] \cup [12,\infty).$$
\rozwStop
\odpStart
$x \in [1,3] \cup [12,\infty)$
\odpStop
\testStart
A.$x \in [1,3] \cup [12,\infty)$\\
B.$x \in (1,3) \cup [12,\infty)$\\
C.$x \in (1,3] \cup [12,\infty)$\\
D.$x \in [1,3) \cup [12,\infty)$\\
E.$x \in [1,3] \cup (12,\infty)$\\
F.$x \in (1,3) \cup (12,\infty)$\\
G.$x \in [1,3) \cup (12,\infty)$\\
H.$x \in (1,3] \cup (12,\infty)$
\testStop
\kluczStart
A
\kluczStop



\zadStart{Zadanie z Wikieł Z 1.62 a) moja wersja nr 28}

Rozwiązać nierówności $(x-1)(x-3)(x-13)\ge0$.
\zadStop
\rozwStart{Patryk Wirkus}{}
Miejsca zerowe naszego wielomianu to: $1, 3, 13$.\\
Wielomian jest stopnia nieparzystego, ponadto znak współczynnika przy\linebreak najwyższej potędze x jest dodatni.\\ W związku z tym wykres wielomianu zaczyna się od lewej strony poniżej osi OX. A więc $$x \in [1,3] \cup [13,\infty).$$
\rozwStop
\odpStart
$x \in [1,3] \cup [13,\infty)$
\odpStop
\testStart
A.$x \in [1,3] \cup [13,\infty)$\\
B.$x \in (1,3) \cup [13,\infty)$\\
C.$x \in (1,3] \cup [13,\infty)$\\
D.$x \in [1,3) \cup [13,\infty)$\\
E.$x \in [1,3] \cup (13,\infty)$\\
F.$x \in (1,3) \cup (13,\infty)$\\
G.$x \in [1,3) \cup (13,\infty)$\\
H.$x \in (1,3] \cup (13,\infty)$
\testStop
\kluczStart
A
\kluczStop



\zadStart{Zadanie z Wikieł Z 1.62 a) moja wersja nr 29}

Rozwiązać nierówności $(x-1)(x-3)(x-14)\ge0$.
\zadStop
\rozwStart{Patryk Wirkus}{}
Miejsca zerowe naszego wielomianu to: $1, 3, 14$.\\
Wielomian jest stopnia nieparzystego, ponadto znak współczynnika przy\linebreak najwyższej potędze x jest dodatni.\\ W związku z tym wykres wielomianu zaczyna się od lewej strony poniżej osi OX. A więc $$x \in [1,3] \cup [14,\infty).$$
\rozwStop
\odpStart
$x \in [1,3] \cup [14,\infty)$
\odpStop
\testStart
A.$x \in [1,3] \cup [14,\infty)$\\
B.$x \in (1,3) \cup [14,\infty)$\\
C.$x \in (1,3] \cup [14,\infty)$\\
D.$x \in [1,3) \cup [14,\infty)$\\
E.$x \in [1,3] \cup (14,\infty)$\\
F.$x \in (1,3) \cup (14,\infty)$\\
G.$x \in [1,3) \cup (14,\infty)$\\
H.$x \in (1,3] \cup (14,\infty)$
\testStop
\kluczStart
A
\kluczStop



\zadStart{Zadanie z Wikieł Z 1.62 a) moja wersja nr 30}

Rozwiązać nierówności $(x-1)(x-3)(x-15)\ge0$.
\zadStop
\rozwStart{Patryk Wirkus}{}
Miejsca zerowe naszego wielomianu to: $1, 3, 15$.\\
Wielomian jest stopnia nieparzystego, ponadto znak współczynnika przy\linebreak najwyższej potędze x jest dodatni.\\ W związku z tym wykres wielomianu zaczyna się od lewej strony poniżej osi OX. A więc $$x \in [1,3] \cup [15,\infty).$$
\rozwStop
\odpStart
$x \in [1,3] \cup [15,\infty)$
\odpStop
\testStart
A.$x \in [1,3] \cup [15,\infty)$\\
B.$x \in (1,3) \cup [15,\infty)$\\
C.$x \in (1,3] \cup [15,\infty)$\\
D.$x \in [1,3) \cup [15,\infty)$\\
E.$x \in [1,3] \cup (15,\infty)$\\
F.$x \in (1,3) \cup (15,\infty)$\\
G.$x \in [1,3) \cup (15,\infty)$\\
H.$x \in (1,3] \cup (15,\infty)$
\testStop
\kluczStart
A
\kluczStop



\zadStart{Zadanie z Wikieł Z 1.62 a) moja wersja nr 31}

Rozwiązać nierówności $(x-1)(x-3)(x-16)\ge0$.
\zadStop
\rozwStart{Patryk Wirkus}{}
Miejsca zerowe naszego wielomianu to: $1, 3, 16$.\\
Wielomian jest stopnia nieparzystego, ponadto znak współczynnika przy\linebreak najwyższej potędze x jest dodatni.\\ W związku z tym wykres wielomianu zaczyna się od lewej strony poniżej osi OX. A więc $$x \in [1,3] \cup [16,\infty).$$
\rozwStop
\odpStart
$x \in [1,3] \cup [16,\infty)$
\odpStop
\testStart
A.$x \in [1,3] \cup [16,\infty)$\\
B.$x \in (1,3) \cup [16,\infty)$\\
C.$x \in (1,3] \cup [16,\infty)$\\
D.$x \in [1,3) \cup [16,\infty)$\\
E.$x \in [1,3] \cup (16,\infty)$\\
F.$x \in (1,3) \cup (16,\infty)$\\
G.$x \in [1,3) \cup (16,\infty)$\\
H.$x \in (1,3] \cup (16,\infty)$
\testStop
\kluczStart
A
\kluczStop



\zadStart{Zadanie z Wikieł Z 1.62 a) moja wersja nr 32}

Rozwiązać nierówności $(x-1)(x-3)(x-17)\ge0$.
\zadStop
\rozwStart{Patryk Wirkus}{}
Miejsca zerowe naszego wielomianu to: $1, 3, 17$.\\
Wielomian jest stopnia nieparzystego, ponadto znak współczynnika przy\linebreak najwyższej potędze x jest dodatni.\\ W związku z tym wykres wielomianu zaczyna się od lewej strony poniżej osi OX. A więc $$x \in [1,3] \cup [17,\infty).$$
\rozwStop
\odpStart
$x \in [1,3] \cup [17,\infty)$
\odpStop
\testStart
A.$x \in [1,3] \cup [17,\infty)$\\
B.$x \in (1,3) \cup [17,\infty)$\\
C.$x \in (1,3] \cup [17,\infty)$\\
D.$x \in [1,3) \cup [17,\infty)$\\
E.$x \in [1,3] \cup (17,\infty)$\\
F.$x \in (1,3) \cup (17,\infty)$\\
G.$x \in [1,3) \cup (17,\infty)$\\
H.$x \in (1,3] \cup (17,\infty)$
\testStop
\kluczStart
A
\kluczStop



\zadStart{Zadanie z Wikieł Z 1.62 a) moja wersja nr 33}

Rozwiązać nierówności $(x-1)(x-3)(x-18)\ge0$.
\zadStop
\rozwStart{Patryk Wirkus}{}
Miejsca zerowe naszego wielomianu to: $1, 3, 18$.\\
Wielomian jest stopnia nieparzystego, ponadto znak współczynnika przy\linebreak najwyższej potędze x jest dodatni.\\ W związku z tym wykres wielomianu zaczyna się od lewej strony poniżej osi OX. A więc $$x \in [1,3] \cup [18,\infty).$$
\rozwStop
\odpStart
$x \in [1,3] \cup [18,\infty)$
\odpStop
\testStart
A.$x \in [1,3] \cup [18,\infty)$\\
B.$x \in (1,3) \cup [18,\infty)$\\
C.$x \in (1,3] \cup [18,\infty)$\\
D.$x \in [1,3) \cup [18,\infty)$\\
E.$x \in [1,3] \cup (18,\infty)$\\
F.$x \in (1,3) \cup (18,\infty)$\\
G.$x \in [1,3) \cup (18,\infty)$\\
H.$x \in (1,3] \cup (18,\infty)$
\testStop
\kluczStart
A
\kluczStop



\zadStart{Zadanie z Wikieł Z 1.62 a) moja wersja nr 34}

Rozwiązać nierówności $(x-1)(x-3)(x-19)\ge0$.
\zadStop
\rozwStart{Patryk Wirkus}{}
Miejsca zerowe naszego wielomianu to: $1, 3, 19$.\\
Wielomian jest stopnia nieparzystego, ponadto znak współczynnika przy\linebreak najwyższej potędze x jest dodatni.\\ W związku z tym wykres wielomianu zaczyna się od lewej strony poniżej osi OX. A więc $$x \in [1,3] \cup [19,\infty).$$
\rozwStop
\odpStart
$x \in [1,3] \cup [19,\infty)$
\odpStop
\testStart
A.$x \in [1,3] \cup [19,\infty)$\\
B.$x \in (1,3) \cup [19,\infty)$\\
C.$x \in (1,3] \cup [19,\infty)$\\
D.$x \in [1,3) \cup [19,\infty)$\\
E.$x \in [1,3] \cup (19,\infty)$\\
F.$x \in (1,3) \cup (19,\infty)$\\
G.$x \in [1,3) \cup (19,\infty)$\\
H.$x \in (1,3] \cup (19,\infty)$
\testStop
\kluczStart
A
\kluczStop



\zadStart{Zadanie z Wikieł Z 1.62 a) moja wersja nr 35}

Rozwiązać nierówności $(x-1)(x-3)(x-20)\ge0$.
\zadStop
\rozwStart{Patryk Wirkus}{}
Miejsca zerowe naszego wielomianu to: $1, 3, 20$.\\
Wielomian jest stopnia nieparzystego, ponadto znak współczynnika przy\linebreak najwyższej potędze x jest dodatni.\\ W związku z tym wykres wielomianu zaczyna się od lewej strony poniżej osi OX. A więc $$x \in [1,3] \cup [20,\infty).$$
\rozwStop
\odpStart
$x \in [1,3] \cup [20,\infty)$
\odpStop
\testStart
A.$x \in [1,3] \cup [20,\infty)$\\
B.$x \in (1,3) \cup [20,\infty)$\\
C.$x \in (1,3] \cup [20,\infty)$\\
D.$x \in [1,3) \cup [20,\infty)$\\
E.$x \in [1,3] \cup (20,\infty)$\\
F.$x \in (1,3) \cup (20,\infty)$\\
G.$x \in [1,3) \cup (20,\infty)$\\
H.$x \in (1,3] \cup (20,\infty)$
\testStop
\kluczStart
A
\kluczStop



\zadStart{Zadanie z Wikieł Z 1.62 a) moja wersja nr 36}

Rozwiązać nierówności $(x-1)(x-4)(x-5)\ge0$.
\zadStop
\rozwStart{Patryk Wirkus}{}
Miejsca zerowe naszego wielomianu to: $1, 4, 5$.\\
Wielomian jest stopnia nieparzystego, ponadto znak współczynnika przy\linebreak najwyższej potędze x jest dodatni.\\ W związku z tym wykres wielomianu zaczyna się od lewej strony poniżej osi OX. A więc $$x \in [1,4] \cup [5,\infty).$$
\rozwStop
\odpStart
$x \in [1,4] \cup [5,\infty)$
\odpStop
\testStart
A.$x \in [1,4] \cup [5,\infty)$\\
B.$x \in (1,4) \cup [5,\infty)$\\
C.$x \in (1,4] \cup [5,\infty)$\\
D.$x \in [1,4) \cup [5,\infty)$\\
E.$x \in [1,4] \cup (5,\infty)$\\
F.$x \in (1,4) \cup (5,\infty)$\\
G.$x \in [1,4) \cup (5,\infty)$\\
H.$x \in (1,4] \cup (5,\infty)$
\testStop
\kluczStart
A
\kluczStop



\zadStart{Zadanie z Wikieł Z 1.62 a) moja wersja nr 37}

Rozwiązać nierówności $(x-1)(x-4)(x-6)\ge0$.
\zadStop
\rozwStart{Patryk Wirkus}{}
Miejsca zerowe naszego wielomianu to: $1, 4, 6$.\\
Wielomian jest stopnia nieparzystego, ponadto znak współczynnika przy\linebreak najwyższej potędze x jest dodatni.\\ W związku z tym wykres wielomianu zaczyna się od lewej strony poniżej osi OX. A więc $$x \in [1,4] \cup [6,\infty).$$
\rozwStop
\odpStart
$x \in [1,4] \cup [6,\infty)$
\odpStop
\testStart
A.$x \in [1,4] \cup [6,\infty)$\\
B.$x \in (1,4) \cup [6,\infty)$\\
C.$x \in (1,4] \cup [6,\infty)$\\
D.$x \in [1,4) \cup [6,\infty)$\\
E.$x \in [1,4] \cup (6,\infty)$\\
F.$x \in (1,4) \cup (6,\infty)$\\
G.$x \in [1,4) \cup (6,\infty)$\\
H.$x \in (1,4] \cup (6,\infty)$
\testStop
\kluczStart
A
\kluczStop



\zadStart{Zadanie z Wikieł Z 1.62 a) moja wersja nr 38}

Rozwiązać nierówności $(x-1)(x-4)(x-7)\ge0$.
\zadStop
\rozwStart{Patryk Wirkus}{}
Miejsca zerowe naszego wielomianu to: $1, 4, 7$.\\
Wielomian jest stopnia nieparzystego, ponadto znak współczynnika przy\linebreak najwyższej potędze x jest dodatni.\\ W związku z tym wykres wielomianu zaczyna się od lewej strony poniżej osi OX. A więc $$x \in [1,4] \cup [7,\infty).$$
\rozwStop
\odpStart
$x \in [1,4] \cup [7,\infty)$
\odpStop
\testStart
A.$x \in [1,4] \cup [7,\infty)$\\
B.$x \in (1,4) \cup [7,\infty)$\\
C.$x \in (1,4] \cup [7,\infty)$\\
D.$x \in [1,4) \cup [7,\infty)$\\
E.$x \in [1,4] \cup (7,\infty)$\\
F.$x \in (1,4) \cup (7,\infty)$\\
G.$x \in [1,4) \cup (7,\infty)$\\
H.$x \in (1,4] \cup (7,\infty)$
\testStop
\kluczStart
A
\kluczStop



\zadStart{Zadanie z Wikieł Z 1.62 a) moja wersja nr 39}

Rozwiązać nierówności $(x-1)(x-4)(x-8)\ge0$.
\zadStop
\rozwStart{Patryk Wirkus}{}
Miejsca zerowe naszego wielomianu to: $1, 4, 8$.\\
Wielomian jest stopnia nieparzystego, ponadto znak współczynnika przy\linebreak najwyższej potędze x jest dodatni.\\ W związku z tym wykres wielomianu zaczyna się od lewej strony poniżej osi OX. A więc $$x \in [1,4] \cup [8,\infty).$$
\rozwStop
\odpStart
$x \in [1,4] \cup [8,\infty)$
\odpStop
\testStart
A.$x \in [1,4] \cup [8,\infty)$\\
B.$x \in (1,4) \cup [8,\infty)$\\
C.$x \in (1,4] \cup [8,\infty)$\\
D.$x \in [1,4) \cup [8,\infty)$\\
E.$x \in [1,4] \cup (8,\infty)$\\
F.$x \in (1,4) \cup (8,\infty)$\\
G.$x \in [1,4) \cup (8,\infty)$\\
H.$x \in (1,4] \cup (8,\infty)$
\testStop
\kluczStart
A
\kluczStop



\zadStart{Zadanie z Wikieł Z 1.62 a) moja wersja nr 40}

Rozwiązać nierówności $(x-1)(x-4)(x-9)\ge0$.
\zadStop
\rozwStart{Patryk Wirkus}{}
Miejsca zerowe naszego wielomianu to: $1, 4, 9$.\\
Wielomian jest stopnia nieparzystego, ponadto znak współczynnika przy\linebreak najwyższej potędze x jest dodatni.\\ W związku z tym wykres wielomianu zaczyna się od lewej strony poniżej osi OX. A więc $$x \in [1,4] \cup [9,\infty).$$
\rozwStop
\odpStart
$x \in [1,4] \cup [9,\infty)$
\odpStop
\testStart
A.$x \in [1,4] \cup [9,\infty)$\\
B.$x \in (1,4) \cup [9,\infty)$\\
C.$x \in (1,4] \cup [9,\infty)$\\
D.$x \in [1,4) \cup [9,\infty)$\\
E.$x \in [1,4] \cup (9,\infty)$\\
F.$x \in (1,4) \cup (9,\infty)$\\
G.$x \in [1,4) \cup (9,\infty)$\\
H.$x \in (1,4] \cup (9,\infty)$
\testStop
\kluczStart
A
\kluczStop



\zadStart{Zadanie z Wikieł Z 1.62 a) moja wersja nr 41}

Rozwiązać nierówności $(x-1)(x-4)(x-10)\ge0$.
\zadStop
\rozwStart{Patryk Wirkus}{}
Miejsca zerowe naszego wielomianu to: $1, 4, 10$.\\
Wielomian jest stopnia nieparzystego, ponadto znak współczynnika przy\linebreak najwyższej potędze x jest dodatni.\\ W związku z tym wykres wielomianu zaczyna się od lewej strony poniżej osi OX. A więc $$x \in [1,4] \cup [10,\infty).$$
\rozwStop
\odpStart
$x \in [1,4] \cup [10,\infty)$
\odpStop
\testStart
A.$x \in [1,4] \cup [10,\infty)$\\
B.$x \in (1,4) \cup [10,\infty)$\\
C.$x \in (1,4] \cup [10,\infty)$\\
D.$x \in [1,4) \cup [10,\infty)$\\
E.$x \in [1,4] \cup (10,\infty)$\\
F.$x \in (1,4) \cup (10,\infty)$\\
G.$x \in [1,4) \cup (10,\infty)$\\
H.$x \in (1,4] \cup (10,\infty)$
\testStop
\kluczStart
A
\kluczStop



\zadStart{Zadanie z Wikieł Z 1.62 a) moja wersja nr 42}

Rozwiązać nierówności $(x-1)(x-4)(x-11)\ge0$.
\zadStop
\rozwStart{Patryk Wirkus}{}
Miejsca zerowe naszego wielomianu to: $1, 4, 11$.\\
Wielomian jest stopnia nieparzystego, ponadto znak współczynnika przy\linebreak najwyższej potędze x jest dodatni.\\ W związku z tym wykres wielomianu zaczyna się od lewej strony poniżej osi OX. A więc $$x \in [1,4] \cup [11,\infty).$$
\rozwStop
\odpStart
$x \in [1,4] \cup [11,\infty)$
\odpStop
\testStart
A.$x \in [1,4] \cup [11,\infty)$\\
B.$x \in (1,4) \cup [11,\infty)$\\
C.$x \in (1,4] \cup [11,\infty)$\\
D.$x \in [1,4) \cup [11,\infty)$\\
E.$x \in [1,4] \cup (11,\infty)$\\
F.$x \in (1,4) \cup (11,\infty)$\\
G.$x \in [1,4) \cup (11,\infty)$\\
H.$x \in (1,4] \cup (11,\infty)$
\testStop
\kluczStart
A
\kluczStop



\zadStart{Zadanie z Wikieł Z 1.62 a) moja wersja nr 43}

Rozwiązać nierówności $(x-1)(x-4)(x-12)\ge0$.
\zadStop
\rozwStart{Patryk Wirkus}{}
Miejsca zerowe naszego wielomianu to: $1, 4, 12$.\\
Wielomian jest stopnia nieparzystego, ponadto znak współczynnika przy\linebreak najwyższej potędze x jest dodatni.\\ W związku z tym wykres wielomianu zaczyna się od lewej strony poniżej osi OX. A więc $$x \in [1,4] \cup [12,\infty).$$
\rozwStop
\odpStart
$x \in [1,4] \cup [12,\infty)$
\odpStop
\testStart
A.$x \in [1,4] \cup [12,\infty)$\\
B.$x \in (1,4) \cup [12,\infty)$\\
C.$x \in (1,4] \cup [12,\infty)$\\
D.$x \in [1,4) \cup [12,\infty)$\\
E.$x \in [1,4] \cup (12,\infty)$\\
F.$x \in (1,4) \cup (12,\infty)$\\
G.$x \in [1,4) \cup (12,\infty)$\\
H.$x \in (1,4] \cup (12,\infty)$
\testStop
\kluczStart
A
\kluczStop



\zadStart{Zadanie z Wikieł Z 1.62 a) moja wersja nr 44}

Rozwiązać nierówności $(x-1)(x-4)(x-13)\ge0$.
\zadStop
\rozwStart{Patryk Wirkus}{}
Miejsca zerowe naszego wielomianu to: $1, 4, 13$.\\
Wielomian jest stopnia nieparzystego, ponadto znak współczynnika przy\linebreak najwyższej potędze x jest dodatni.\\ W związku z tym wykres wielomianu zaczyna się od lewej strony poniżej osi OX. A więc $$x \in [1,4] \cup [13,\infty).$$
\rozwStop
\odpStart
$x \in [1,4] \cup [13,\infty)$
\odpStop
\testStart
A.$x \in [1,4] \cup [13,\infty)$\\
B.$x \in (1,4) \cup [13,\infty)$\\
C.$x \in (1,4] \cup [13,\infty)$\\
D.$x \in [1,4) \cup [13,\infty)$\\
E.$x \in [1,4] \cup (13,\infty)$\\
F.$x \in (1,4) \cup (13,\infty)$\\
G.$x \in [1,4) \cup (13,\infty)$\\
H.$x \in (1,4] \cup (13,\infty)$
\testStop
\kluczStart
A
\kluczStop



\zadStart{Zadanie z Wikieł Z 1.62 a) moja wersja nr 45}

Rozwiązać nierówności $(x-1)(x-4)(x-14)\ge0$.
\zadStop
\rozwStart{Patryk Wirkus}{}
Miejsca zerowe naszego wielomianu to: $1, 4, 14$.\\
Wielomian jest stopnia nieparzystego, ponadto znak współczynnika przy\linebreak najwyższej potędze x jest dodatni.\\ W związku z tym wykres wielomianu zaczyna się od lewej strony poniżej osi OX. A więc $$x \in [1,4] \cup [14,\infty).$$
\rozwStop
\odpStart
$x \in [1,4] \cup [14,\infty)$
\odpStop
\testStart
A.$x \in [1,4] \cup [14,\infty)$\\
B.$x \in (1,4) \cup [14,\infty)$\\
C.$x \in (1,4] \cup [14,\infty)$\\
D.$x \in [1,4) \cup [14,\infty)$\\
E.$x \in [1,4] \cup (14,\infty)$\\
F.$x \in (1,4) \cup (14,\infty)$\\
G.$x \in [1,4) \cup (14,\infty)$\\
H.$x \in (1,4] \cup (14,\infty)$
\testStop
\kluczStart
A
\kluczStop



\zadStart{Zadanie z Wikieł Z 1.62 a) moja wersja nr 46}

Rozwiązać nierówności $(x-1)(x-4)(x-15)\ge0$.
\zadStop
\rozwStart{Patryk Wirkus}{}
Miejsca zerowe naszego wielomianu to: $1, 4, 15$.\\
Wielomian jest stopnia nieparzystego, ponadto znak współczynnika przy\linebreak najwyższej potędze x jest dodatni.\\ W związku z tym wykres wielomianu zaczyna się od lewej strony poniżej osi OX. A więc $$x \in [1,4] \cup [15,\infty).$$
\rozwStop
\odpStart
$x \in [1,4] \cup [15,\infty)$
\odpStop
\testStart
A.$x \in [1,4] \cup [15,\infty)$\\
B.$x \in (1,4) \cup [15,\infty)$\\
C.$x \in (1,4] \cup [15,\infty)$\\
D.$x \in [1,4) \cup [15,\infty)$\\
E.$x \in [1,4] \cup (15,\infty)$\\
F.$x \in (1,4) \cup (15,\infty)$\\
G.$x \in [1,4) \cup (15,\infty)$\\
H.$x \in (1,4] \cup (15,\infty)$
\testStop
\kluczStart
A
\kluczStop



\zadStart{Zadanie z Wikieł Z 1.62 a) moja wersja nr 47}

Rozwiązać nierówności $(x-1)(x-4)(x-16)\ge0$.
\zadStop
\rozwStart{Patryk Wirkus}{}
Miejsca zerowe naszego wielomianu to: $1, 4, 16$.\\
Wielomian jest stopnia nieparzystego, ponadto znak współczynnika przy\linebreak najwyższej potędze x jest dodatni.\\ W związku z tym wykres wielomianu zaczyna się od lewej strony poniżej osi OX. A więc $$x \in [1,4] \cup [16,\infty).$$
\rozwStop
\odpStart
$x \in [1,4] \cup [16,\infty)$
\odpStop
\testStart
A.$x \in [1,4] \cup [16,\infty)$\\
B.$x \in (1,4) \cup [16,\infty)$\\
C.$x \in (1,4] \cup [16,\infty)$\\
D.$x \in [1,4) \cup [16,\infty)$\\
E.$x \in [1,4] \cup (16,\infty)$\\
F.$x \in (1,4) \cup (16,\infty)$\\
G.$x \in [1,4) \cup (16,\infty)$\\
H.$x \in (1,4] \cup (16,\infty)$
\testStop
\kluczStart
A
\kluczStop



\zadStart{Zadanie z Wikieł Z 1.62 a) moja wersja nr 48}

Rozwiązać nierówności $(x-1)(x-4)(x-17)\ge0$.
\zadStop
\rozwStart{Patryk Wirkus}{}
Miejsca zerowe naszego wielomianu to: $1, 4, 17$.\\
Wielomian jest stopnia nieparzystego, ponadto znak współczynnika przy\linebreak najwyższej potędze x jest dodatni.\\ W związku z tym wykres wielomianu zaczyna się od lewej strony poniżej osi OX. A więc $$x \in [1,4] \cup [17,\infty).$$
\rozwStop
\odpStart
$x \in [1,4] \cup [17,\infty)$
\odpStop
\testStart
A.$x \in [1,4] \cup [17,\infty)$\\
B.$x \in (1,4) \cup [17,\infty)$\\
C.$x \in (1,4] \cup [17,\infty)$\\
D.$x \in [1,4) \cup [17,\infty)$\\
E.$x \in [1,4] \cup (17,\infty)$\\
F.$x \in (1,4) \cup (17,\infty)$\\
G.$x \in [1,4) \cup (17,\infty)$\\
H.$x \in (1,4] \cup (17,\infty)$
\testStop
\kluczStart
A
\kluczStop



\zadStart{Zadanie z Wikieł Z 1.62 a) moja wersja nr 49}

Rozwiązać nierówności $(x-1)(x-4)(x-18)\ge0$.
\zadStop
\rozwStart{Patryk Wirkus}{}
Miejsca zerowe naszego wielomianu to: $1, 4, 18$.\\
Wielomian jest stopnia nieparzystego, ponadto znak współczynnika przy\linebreak najwyższej potędze x jest dodatni.\\ W związku z tym wykres wielomianu zaczyna się od lewej strony poniżej osi OX. A więc $$x \in [1,4] \cup [18,\infty).$$
\rozwStop
\odpStart
$x \in [1,4] \cup [18,\infty)$
\odpStop
\testStart
A.$x \in [1,4] \cup [18,\infty)$\\
B.$x \in (1,4) \cup [18,\infty)$\\
C.$x \in (1,4] \cup [18,\infty)$\\
D.$x \in [1,4) \cup [18,\infty)$\\
E.$x \in [1,4] \cup (18,\infty)$\\
F.$x \in (1,4) \cup (18,\infty)$\\
G.$x \in [1,4) \cup (18,\infty)$\\
H.$x \in (1,4] \cup (18,\infty)$
\testStop
\kluczStart
A
\kluczStop



\zadStart{Zadanie z Wikieł Z 1.62 a) moja wersja nr 50}

Rozwiązać nierówności $(x-1)(x-4)(x-19)\ge0$.
\zadStop
\rozwStart{Patryk Wirkus}{}
Miejsca zerowe naszego wielomianu to: $1, 4, 19$.\\
Wielomian jest stopnia nieparzystego, ponadto znak współczynnika przy\linebreak najwyższej potędze x jest dodatni.\\ W związku z tym wykres wielomianu zaczyna się od lewej strony poniżej osi OX. A więc $$x \in [1,4] \cup [19,\infty).$$
\rozwStop
\odpStart
$x \in [1,4] \cup [19,\infty)$
\odpStop
\testStart
A.$x \in [1,4] \cup [19,\infty)$\\
B.$x \in (1,4) \cup [19,\infty)$\\
C.$x \in (1,4] \cup [19,\infty)$\\
D.$x \in [1,4) \cup [19,\infty)$\\
E.$x \in [1,4] \cup (19,\infty)$\\
F.$x \in (1,4) \cup (19,\infty)$\\
G.$x \in [1,4) \cup (19,\infty)$\\
H.$x \in (1,4] \cup (19,\infty)$
\testStop
\kluczStart
A
\kluczStop



\zadStart{Zadanie z Wikieł Z 1.62 a) moja wersja nr 51}

Rozwiązać nierówności $(x-1)(x-4)(x-20)\ge0$.
\zadStop
\rozwStart{Patryk Wirkus}{}
Miejsca zerowe naszego wielomianu to: $1, 4, 20$.\\
Wielomian jest stopnia nieparzystego, ponadto znak współczynnika przy\linebreak najwyższej potędze x jest dodatni.\\ W związku z tym wykres wielomianu zaczyna się od lewej strony poniżej osi OX. A więc $$x \in [1,4] \cup [20,\infty).$$
\rozwStop
\odpStart
$x \in [1,4] \cup [20,\infty)$
\odpStop
\testStart
A.$x \in [1,4] \cup [20,\infty)$\\
B.$x \in (1,4) \cup [20,\infty)$\\
C.$x \in (1,4] \cup [20,\infty)$\\
D.$x \in [1,4) \cup [20,\infty)$\\
E.$x \in [1,4] \cup (20,\infty)$\\
F.$x \in (1,4) \cup (20,\infty)$\\
G.$x \in [1,4) \cup (20,\infty)$\\
H.$x \in (1,4] \cup (20,\infty)$
\testStop
\kluczStart
A
\kluczStop



\zadStart{Zadanie z Wikieł Z 1.62 a) moja wersja nr 52}

Rozwiązać nierówności $(x-1)(x-5)(x-6)\ge0$.
\zadStop
\rozwStart{Patryk Wirkus}{}
Miejsca zerowe naszego wielomianu to: $1, 5, 6$.\\
Wielomian jest stopnia nieparzystego, ponadto znak współczynnika przy\linebreak najwyższej potędze x jest dodatni.\\ W związku z tym wykres wielomianu zaczyna się od lewej strony poniżej osi OX. A więc $$x \in [1,5] \cup [6,\infty).$$
\rozwStop
\odpStart
$x \in [1,5] \cup [6,\infty)$
\odpStop
\testStart
A.$x \in [1,5] \cup [6,\infty)$\\
B.$x \in (1,5) \cup [6,\infty)$\\
C.$x \in (1,5] \cup [6,\infty)$\\
D.$x \in [1,5) \cup [6,\infty)$\\
E.$x \in [1,5] \cup (6,\infty)$\\
F.$x \in (1,5) \cup (6,\infty)$\\
G.$x \in [1,5) \cup (6,\infty)$\\
H.$x \in (1,5] \cup (6,\infty)$
\testStop
\kluczStart
A
\kluczStop



\zadStart{Zadanie z Wikieł Z 1.62 a) moja wersja nr 53}

Rozwiązać nierówności $(x-1)(x-5)(x-7)\ge0$.
\zadStop
\rozwStart{Patryk Wirkus}{}
Miejsca zerowe naszego wielomianu to: $1, 5, 7$.\\
Wielomian jest stopnia nieparzystego, ponadto znak współczynnika przy\linebreak najwyższej potędze x jest dodatni.\\ W związku z tym wykres wielomianu zaczyna się od lewej strony poniżej osi OX. A więc $$x \in [1,5] \cup [7,\infty).$$
\rozwStop
\odpStart
$x \in [1,5] \cup [7,\infty)$
\odpStop
\testStart
A.$x \in [1,5] \cup [7,\infty)$\\
B.$x \in (1,5) \cup [7,\infty)$\\
C.$x \in (1,5] \cup [7,\infty)$\\
D.$x \in [1,5) \cup [7,\infty)$\\
E.$x \in [1,5] \cup (7,\infty)$\\
F.$x \in (1,5) \cup (7,\infty)$\\
G.$x \in [1,5) \cup (7,\infty)$\\
H.$x \in (1,5] \cup (7,\infty)$
\testStop
\kluczStart
A
\kluczStop



\zadStart{Zadanie z Wikieł Z 1.62 a) moja wersja nr 54}

Rozwiązać nierówności $(x-1)(x-5)(x-8)\ge0$.
\zadStop
\rozwStart{Patryk Wirkus}{}
Miejsca zerowe naszego wielomianu to: $1, 5, 8$.\\
Wielomian jest stopnia nieparzystego, ponadto znak współczynnika przy\linebreak najwyższej potędze x jest dodatni.\\ W związku z tym wykres wielomianu zaczyna się od lewej strony poniżej osi OX. A więc $$x \in [1,5] \cup [8,\infty).$$
\rozwStop
\odpStart
$x \in [1,5] \cup [8,\infty)$
\odpStop
\testStart
A.$x \in [1,5] \cup [8,\infty)$\\
B.$x \in (1,5) \cup [8,\infty)$\\
C.$x \in (1,5] \cup [8,\infty)$\\
D.$x \in [1,5) \cup [8,\infty)$\\
E.$x \in [1,5] \cup (8,\infty)$\\
F.$x \in (1,5) \cup (8,\infty)$\\
G.$x \in [1,5) \cup (8,\infty)$\\
H.$x \in (1,5] \cup (8,\infty)$
\testStop
\kluczStart
A
\kluczStop



\zadStart{Zadanie z Wikieł Z 1.62 a) moja wersja nr 55}

Rozwiązać nierówności $(x-1)(x-5)(x-9)\ge0$.
\zadStop
\rozwStart{Patryk Wirkus}{}
Miejsca zerowe naszego wielomianu to: $1, 5, 9$.\\
Wielomian jest stopnia nieparzystego, ponadto znak współczynnika przy\linebreak najwyższej potędze x jest dodatni.\\ W związku z tym wykres wielomianu zaczyna się od lewej strony poniżej osi OX. A więc $$x \in [1,5] \cup [9,\infty).$$
\rozwStop
\odpStart
$x \in [1,5] \cup [9,\infty)$
\odpStop
\testStart
A.$x \in [1,5] \cup [9,\infty)$\\
B.$x \in (1,5) \cup [9,\infty)$\\
C.$x \in (1,5] \cup [9,\infty)$\\
D.$x \in [1,5) \cup [9,\infty)$\\
E.$x \in [1,5] \cup (9,\infty)$\\
F.$x \in (1,5) \cup (9,\infty)$\\
G.$x \in [1,5) \cup (9,\infty)$\\
H.$x \in (1,5] \cup (9,\infty)$
\testStop
\kluczStart
A
\kluczStop



\zadStart{Zadanie z Wikieł Z 1.62 a) moja wersja nr 56}

Rozwiązać nierówności $(x-1)(x-5)(x-10)\ge0$.
\zadStop
\rozwStart{Patryk Wirkus}{}
Miejsca zerowe naszego wielomianu to: $1, 5, 10$.\\
Wielomian jest stopnia nieparzystego, ponadto znak współczynnika przy\linebreak najwyższej potędze x jest dodatni.\\ W związku z tym wykres wielomianu zaczyna się od lewej strony poniżej osi OX. A więc $$x \in [1,5] \cup [10,\infty).$$
\rozwStop
\odpStart
$x \in [1,5] \cup [10,\infty)$
\odpStop
\testStart
A.$x \in [1,5] \cup [10,\infty)$\\
B.$x \in (1,5) \cup [10,\infty)$\\
C.$x \in (1,5] \cup [10,\infty)$\\
D.$x \in [1,5) \cup [10,\infty)$\\
E.$x \in [1,5] \cup (10,\infty)$\\
F.$x \in (1,5) \cup (10,\infty)$\\
G.$x \in [1,5) \cup (10,\infty)$\\
H.$x \in (1,5] \cup (10,\infty)$
\testStop
\kluczStart
A
\kluczStop



\zadStart{Zadanie z Wikieł Z 1.62 a) moja wersja nr 57}

Rozwiązać nierówności $(x-1)(x-5)(x-11)\ge0$.
\zadStop
\rozwStart{Patryk Wirkus}{}
Miejsca zerowe naszego wielomianu to: $1, 5, 11$.\\
Wielomian jest stopnia nieparzystego, ponadto znak współczynnika przy\linebreak najwyższej potędze x jest dodatni.\\ W związku z tym wykres wielomianu zaczyna się od lewej strony poniżej osi OX. A więc $$x \in [1,5] \cup [11,\infty).$$
\rozwStop
\odpStart
$x \in [1,5] \cup [11,\infty)$
\odpStop
\testStart
A.$x \in [1,5] \cup [11,\infty)$\\
B.$x \in (1,5) \cup [11,\infty)$\\
C.$x \in (1,5] \cup [11,\infty)$\\
D.$x \in [1,5) \cup [11,\infty)$\\
E.$x \in [1,5] \cup (11,\infty)$\\
F.$x \in (1,5) \cup (11,\infty)$\\
G.$x \in [1,5) \cup (11,\infty)$\\
H.$x \in (1,5] \cup (11,\infty)$
\testStop
\kluczStart
A
\kluczStop



\zadStart{Zadanie z Wikieł Z 1.62 a) moja wersja nr 58}

Rozwiązać nierówności $(x-1)(x-5)(x-12)\ge0$.
\zadStop
\rozwStart{Patryk Wirkus}{}
Miejsca zerowe naszego wielomianu to: $1, 5, 12$.\\
Wielomian jest stopnia nieparzystego, ponadto znak współczynnika przy\linebreak najwyższej potędze x jest dodatni.\\ W związku z tym wykres wielomianu zaczyna się od lewej strony poniżej osi OX. A więc $$x \in [1,5] \cup [12,\infty).$$
\rozwStop
\odpStart
$x \in [1,5] \cup [12,\infty)$
\odpStop
\testStart
A.$x \in [1,5] \cup [12,\infty)$\\
B.$x \in (1,5) \cup [12,\infty)$\\
C.$x \in (1,5] \cup [12,\infty)$\\
D.$x \in [1,5) \cup [12,\infty)$\\
E.$x \in [1,5] \cup (12,\infty)$\\
F.$x \in (1,5) \cup (12,\infty)$\\
G.$x \in [1,5) \cup (12,\infty)$\\
H.$x \in (1,5] \cup (12,\infty)$
\testStop
\kluczStart
A
\kluczStop



\zadStart{Zadanie z Wikieł Z 1.62 a) moja wersja nr 59}

Rozwiązać nierówności $(x-1)(x-5)(x-13)\ge0$.
\zadStop
\rozwStart{Patryk Wirkus}{}
Miejsca zerowe naszego wielomianu to: $1, 5, 13$.\\
Wielomian jest stopnia nieparzystego, ponadto znak współczynnika przy\linebreak najwyższej potędze x jest dodatni.\\ W związku z tym wykres wielomianu zaczyna się od lewej strony poniżej osi OX. A więc $$x \in [1,5] \cup [13,\infty).$$
\rozwStop
\odpStart
$x \in [1,5] \cup [13,\infty)$
\odpStop
\testStart
A.$x \in [1,5] \cup [13,\infty)$\\
B.$x \in (1,5) \cup [13,\infty)$\\
C.$x \in (1,5] \cup [13,\infty)$\\
D.$x \in [1,5) \cup [13,\infty)$\\
E.$x \in [1,5] \cup (13,\infty)$\\
F.$x \in (1,5) \cup (13,\infty)$\\
G.$x \in [1,5) \cup (13,\infty)$\\
H.$x \in (1,5] \cup (13,\infty)$
\testStop
\kluczStart
A
\kluczStop



\zadStart{Zadanie z Wikieł Z 1.62 a) moja wersja nr 60}

Rozwiązać nierówności $(x-1)(x-5)(x-14)\ge0$.
\zadStop
\rozwStart{Patryk Wirkus}{}
Miejsca zerowe naszego wielomianu to: $1, 5, 14$.\\
Wielomian jest stopnia nieparzystego, ponadto znak współczynnika przy\linebreak najwyższej potędze x jest dodatni.\\ W związku z tym wykres wielomianu zaczyna się od lewej strony poniżej osi OX. A więc $$x \in [1,5] \cup [14,\infty).$$
\rozwStop
\odpStart
$x \in [1,5] \cup [14,\infty)$
\odpStop
\testStart
A.$x \in [1,5] \cup [14,\infty)$\\
B.$x \in (1,5) \cup [14,\infty)$\\
C.$x \in (1,5] \cup [14,\infty)$\\
D.$x \in [1,5) \cup [14,\infty)$\\
E.$x \in [1,5] \cup (14,\infty)$\\
F.$x \in (1,5) \cup (14,\infty)$\\
G.$x \in [1,5) \cup (14,\infty)$\\
H.$x \in (1,5] \cup (14,\infty)$
\testStop
\kluczStart
A
\kluczStop



\zadStart{Zadanie z Wikieł Z 1.62 a) moja wersja nr 61}

Rozwiązać nierówności $(x-1)(x-5)(x-15)\ge0$.
\zadStop
\rozwStart{Patryk Wirkus}{}
Miejsca zerowe naszego wielomianu to: $1, 5, 15$.\\
Wielomian jest stopnia nieparzystego, ponadto znak współczynnika przy\linebreak najwyższej potędze x jest dodatni.\\ W związku z tym wykres wielomianu zaczyna się od lewej strony poniżej osi OX. A więc $$x \in [1,5] \cup [15,\infty).$$
\rozwStop
\odpStart
$x \in [1,5] \cup [15,\infty)$
\odpStop
\testStart
A.$x \in [1,5] \cup [15,\infty)$\\
B.$x \in (1,5) \cup [15,\infty)$\\
C.$x \in (1,5] \cup [15,\infty)$\\
D.$x \in [1,5) \cup [15,\infty)$\\
E.$x \in [1,5] \cup (15,\infty)$\\
F.$x \in (1,5) \cup (15,\infty)$\\
G.$x \in [1,5) \cup (15,\infty)$\\
H.$x \in (1,5] \cup (15,\infty)$
\testStop
\kluczStart
A
\kluczStop



\zadStart{Zadanie z Wikieł Z 1.62 a) moja wersja nr 62}

Rozwiązać nierówności $(x-1)(x-5)(x-16)\ge0$.
\zadStop
\rozwStart{Patryk Wirkus}{}
Miejsca zerowe naszego wielomianu to: $1, 5, 16$.\\
Wielomian jest stopnia nieparzystego, ponadto znak współczynnika przy\linebreak najwyższej potędze x jest dodatni.\\ W związku z tym wykres wielomianu zaczyna się od lewej strony poniżej osi OX. A więc $$x \in [1,5] \cup [16,\infty).$$
\rozwStop
\odpStart
$x \in [1,5] \cup [16,\infty)$
\odpStop
\testStart
A.$x \in [1,5] \cup [16,\infty)$\\
B.$x \in (1,5) \cup [16,\infty)$\\
C.$x \in (1,5] \cup [16,\infty)$\\
D.$x \in [1,5) \cup [16,\infty)$\\
E.$x \in [1,5] \cup (16,\infty)$\\
F.$x \in (1,5) \cup (16,\infty)$\\
G.$x \in [1,5) \cup (16,\infty)$\\
H.$x \in (1,5] \cup (16,\infty)$
\testStop
\kluczStart
A
\kluczStop



\zadStart{Zadanie z Wikieł Z 1.62 a) moja wersja nr 63}

Rozwiązać nierówności $(x-1)(x-5)(x-17)\ge0$.
\zadStop
\rozwStart{Patryk Wirkus}{}
Miejsca zerowe naszego wielomianu to: $1, 5, 17$.\\
Wielomian jest stopnia nieparzystego, ponadto znak współczynnika przy\linebreak najwyższej potędze x jest dodatni.\\ W związku z tym wykres wielomianu zaczyna się od lewej strony poniżej osi OX. A więc $$x \in [1,5] \cup [17,\infty).$$
\rozwStop
\odpStart
$x \in [1,5] \cup [17,\infty)$
\odpStop
\testStart
A.$x \in [1,5] \cup [17,\infty)$\\
B.$x \in (1,5) \cup [17,\infty)$\\
C.$x \in (1,5] \cup [17,\infty)$\\
D.$x \in [1,5) \cup [17,\infty)$\\
E.$x \in [1,5] \cup (17,\infty)$\\
F.$x \in (1,5) \cup (17,\infty)$\\
G.$x \in [1,5) \cup (17,\infty)$\\
H.$x \in (1,5] \cup (17,\infty)$
\testStop
\kluczStart
A
\kluczStop



\zadStart{Zadanie z Wikieł Z 1.62 a) moja wersja nr 64}

Rozwiązać nierówności $(x-1)(x-5)(x-18)\ge0$.
\zadStop
\rozwStart{Patryk Wirkus}{}
Miejsca zerowe naszego wielomianu to: $1, 5, 18$.\\
Wielomian jest stopnia nieparzystego, ponadto znak współczynnika przy\linebreak najwyższej potędze x jest dodatni.\\ W związku z tym wykres wielomianu zaczyna się od lewej strony poniżej osi OX. A więc $$x \in [1,5] \cup [18,\infty).$$
\rozwStop
\odpStart
$x \in [1,5] \cup [18,\infty)$
\odpStop
\testStart
A.$x \in [1,5] \cup [18,\infty)$\\
B.$x \in (1,5) \cup [18,\infty)$\\
C.$x \in (1,5] \cup [18,\infty)$\\
D.$x \in [1,5) \cup [18,\infty)$\\
E.$x \in [1,5] \cup (18,\infty)$\\
F.$x \in (1,5) \cup (18,\infty)$\\
G.$x \in [1,5) \cup (18,\infty)$\\
H.$x \in (1,5] \cup (18,\infty)$
\testStop
\kluczStart
A
\kluczStop



\zadStart{Zadanie z Wikieł Z 1.62 a) moja wersja nr 65}

Rozwiązać nierówności $(x-1)(x-5)(x-19)\ge0$.
\zadStop
\rozwStart{Patryk Wirkus}{}
Miejsca zerowe naszego wielomianu to: $1, 5, 19$.\\
Wielomian jest stopnia nieparzystego, ponadto znak współczynnika przy\linebreak najwyższej potędze x jest dodatni.\\ W związku z tym wykres wielomianu zaczyna się od lewej strony poniżej osi OX. A więc $$x \in [1,5] \cup [19,\infty).$$
\rozwStop
\odpStart
$x \in [1,5] \cup [19,\infty)$
\odpStop
\testStart
A.$x \in [1,5] \cup [19,\infty)$\\
B.$x \in (1,5) \cup [19,\infty)$\\
C.$x \in (1,5] \cup [19,\infty)$\\
D.$x \in [1,5) \cup [19,\infty)$\\
E.$x \in [1,5] \cup (19,\infty)$\\
F.$x \in (1,5) \cup (19,\infty)$\\
G.$x \in [1,5) \cup (19,\infty)$\\
H.$x \in (1,5] \cup (19,\infty)$
\testStop
\kluczStart
A
\kluczStop



\zadStart{Zadanie z Wikieł Z 1.62 a) moja wersja nr 66}

Rozwiązać nierówności $(x-1)(x-5)(x-20)\ge0$.
\zadStop
\rozwStart{Patryk Wirkus}{}
Miejsca zerowe naszego wielomianu to: $1, 5, 20$.\\
Wielomian jest stopnia nieparzystego, ponadto znak współczynnika przy\linebreak najwyższej potędze x jest dodatni.\\ W związku z tym wykres wielomianu zaczyna się od lewej strony poniżej osi OX. A więc $$x \in [1,5] \cup [20,\infty).$$
\rozwStop
\odpStart
$x \in [1,5] \cup [20,\infty)$
\odpStop
\testStart
A.$x \in [1,5] \cup [20,\infty)$\\
B.$x \in (1,5) \cup [20,\infty)$\\
C.$x \in (1,5] \cup [20,\infty)$\\
D.$x \in [1,5) \cup [20,\infty)$\\
E.$x \in [1,5] \cup (20,\infty)$\\
F.$x \in (1,5) \cup (20,\infty)$\\
G.$x \in [1,5) \cup (20,\infty)$\\
H.$x \in (1,5] \cup (20,\infty)$
\testStop
\kluczStart
A
\kluczStop



\zadStart{Zadanie z Wikieł Z 1.62 a) moja wersja nr 67}

Rozwiązać nierówności $(x-1)(x-6)(x-7)\ge0$.
\zadStop
\rozwStart{Patryk Wirkus}{}
Miejsca zerowe naszego wielomianu to: $1, 6, 7$.\\
Wielomian jest stopnia nieparzystego, ponadto znak współczynnika przy\linebreak najwyższej potędze x jest dodatni.\\ W związku z tym wykres wielomianu zaczyna się od lewej strony poniżej osi OX. A więc $$x \in [1,6] \cup [7,\infty).$$
\rozwStop
\odpStart
$x \in [1,6] \cup [7,\infty)$
\odpStop
\testStart
A.$x \in [1,6] \cup [7,\infty)$\\
B.$x \in (1,6) \cup [7,\infty)$\\
C.$x \in (1,6] \cup [7,\infty)$\\
D.$x \in [1,6) \cup [7,\infty)$\\
E.$x \in [1,6] \cup (7,\infty)$\\
F.$x \in (1,6) \cup (7,\infty)$\\
G.$x \in [1,6) \cup (7,\infty)$\\
H.$x \in (1,6] \cup (7,\infty)$
\testStop
\kluczStart
A
\kluczStop



\zadStart{Zadanie z Wikieł Z 1.62 a) moja wersja nr 68}

Rozwiązać nierówności $(x-1)(x-6)(x-8)\ge0$.
\zadStop
\rozwStart{Patryk Wirkus}{}
Miejsca zerowe naszego wielomianu to: $1, 6, 8$.\\
Wielomian jest stopnia nieparzystego, ponadto znak współczynnika przy\linebreak najwyższej potędze x jest dodatni.\\ W związku z tym wykres wielomianu zaczyna się od lewej strony poniżej osi OX. A więc $$x \in [1,6] \cup [8,\infty).$$
\rozwStop
\odpStart
$x \in [1,6] \cup [8,\infty)$
\odpStop
\testStart
A.$x \in [1,6] \cup [8,\infty)$\\
B.$x \in (1,6) \cup [8,\infty)$\\
C.$x \in (1,6] \cup [8,\infty)$\\
D.$x \in [1,6) \cup [8,\infty)$\\
E.$x \in [1,6] \cup (8,\infty)$\\
F.$x \in (1,6) \cup (8,\infty)$\\
G.$x \in [1,6) \cup (8,\infty)$\\
H.$x \in (1,6] \cup (8,\infty)$
\testStop
\kluczStart
A
\kluczStop



\zadStart{Zadanie z Wikieł Z 1.62 a) moja wersja nr 69}

Rozwiązać nierówności $(x-1)(x-6)(x-9)\ge0$.
\zadStop
\rozwStart{Patryk Wirkus}{}
Miejsca zerowe naszego wielomianu to: $1, 6, 9$.\\
Wielomian jest stopnia nieparzystego, ponadto znak współczynnika przy\linebreak najwyższej potędze x jest dodatni.\\ W związku z tym wykres wielomianu zaczyna się od lewej strony poniżej osi OX. A więc $$x \in [1,6] \cup [9,\infty).$$
\rozwStop
\odpStart
$x \in [1,6] \cup [9,\infty)$
\odpStop
\testStart
A.$x \in [1,6] \cup [9,\infty)$\\
B.$x \in (1,6) \cup [9,\infty)$\\
C.$x \in (1,6] \cup [9,\infty)$\\
D.$x \in [1,6) \cup [9,\infty)$\\
E.$x \in [1,6] \cup (9,\infty)$\\
F.$x \in (1,6) \cup (9,\infty)$\\
G.$x \in [1,6) \cup (9,\infty)$\\
H.$x \in (1,6] \cup (9,\infty)$
\testStop
\kluczStart
A
\kluczStop



\zadStart{Zadanie z Wikieł Z 1.62 a) moja wersja nr 70}

Rozwiązać nierówności $(x-1)(x-6)(x-10)\ge0$.
\zadStop
\rozwStart{Patryk Wirkus}{}
Miejsca zerowe naszego wielomianu to: $1, 6, 10$.\\
Wielomian jest stopnia nieparzystego, ponadto znak współczynnika przy\linebreak najwyższej potędze x jest dodatni.\\ W związku z tym wykres wielomianu zaczyna się od lewej strony poniżej osi OX. A więc $$x \in [1,6] \cup [10,\infty).$$
\rozwStop
\odpStart
$x \in [1,6] \cup [10,\infty)$
\odpStop
\testStart
A.$x \in [1,6] \cup [10,\infty)$\\
B.$x \in (1,6) \cup [10,\infty)$\\
C.$x \in (1,6] \cup [10,\infty)$\\
D.$x \in [1,6) \cup [10,\infty)$\\
E.$x \in [1,6] \cup (10,\infty)$\\
F.$x \in (1,6) \cup (10,\infty)$\\
G.$x \in [1,6) \cup (10,\infty)$\\
H.$x \in (1,6] \cup (10,\infty)$
\testStop
\kluczStart
A
\kluczStop



\zadStart{Zadanie z Wikieł Z 1.62 a) moja wersja nr 71}

Rozwiązać nierówności $(x-1)(x-6)(x-11)\ge0$.
\zadStop
\rozwStart{Patryk Wirkus}{}
Miejsca zerowe naszego wielomianu to: $1, 6, 11$.\\
Wielomian jest stopnia nieparzystego, ponadto znak współczynnika przy\linebreak najwyższej potędze x jest dodatni.\\ W związku z tym wykres wielomianu zaczyna się od lewej strony poniżej osi OX. A więc $$x \in [1,6] \cup [11,\infty).$$
\rozwStop
\odpStart
$x \in [1,6] \cup [11,\infty)$
\odpStop
\testStart
A.$x \in [1,6] \cup [11,\infty)$\\
B.$x \in (1,6) \cup [11,\infty)$\\
C.$x \in (1,6] \cup [11,\infty)$\\
D.$x \in [1,6) \cup [11,\infty)$\\
E.$x \in [1,6] \cup (11,\infty)$\\
F.$x \in (1,6) \cup (11,\infty)$\\
G.$x \in [1,6) \cup (11,\infty)$\\
H.$x \in (1,6] \cup (11,\infty)$
\testStop
\kluczStart
A
\kluczStop



\zadStart{Zadanie z Wikieł Z 1.62 a) moja wersja nr 72}

Rozwiązać nierówności $(x-1)(x-6)(x-12)\ge0$.
\zadStop
\rozwStart{Patryk Wirkus}{}
Miejsca zerowe naszego wielomianu to: $1, 6, 12$.\\
Wielomian jest stopnia nieparzystego, ponadto znak współczynnika przy\linebreak najwyższej potędze x jest dodatni.\\ W związku z tym wykres wielomianu zaczyna się od lewej strony poniżej osi OX. A więc $$x \in [1,6] \cup [12,\infty).$$
\rozwStop
\odpStart
$x \in [1,6] \cup [12,\infty)$
\odpStop
\testStart
A.$x \in [1,6] \cup [12,\infty)$\\
B.$x \in (1,6) \cup [12,\infty)$\\
C.$x \in (1,6] \cup [12,\infty)$\\
D.$x \in [1,6) \cup [12,\infty)$\\
E.$x \in [1,6] \cup (12,\infty)$\\
F.$x \in (1,6) \cup (12,\infty)$\\
G.$x \in [1,6) \cup (12,\infty)$\\
H.$x \in (1,6] \cup (12,\infty)$
\testStop
\kluczStart
A
\kluczStop



\zadStart{Zadanie z Wikieł Z 1.62 a) moja wersja nr 73}

Rozwiązać nierówności $(x-1)(x-6)(x-13)\ge0$.
\zadStop
\rozwStart{Patryk Wirkus}{}
Miejsca zerowe naszego wielomianu to: $1, 6, 13$.\\
Wielomian jest stopnia nieparzystego, ponadto znak współczynnika przy\linebreak najwyższej potędze x jest dodatni.\\ W związku z tym wykres wielomianu zaczyna się od lewej strony poniżej osi OX. A więc $$x \in [1,6] \cup [13,\infty).$$
\rozwStop
\odpStart
$x \in [1,6] \cup [13,\infty)$
\odpStop
\testStart
A.$x \in [1,6] \cup [13,\infty)$\\
B.$x \in (1,6) \cup [13,\infty)$\\
C.$x \in (1,6] \cup [13,\infty)$\\
D.$x \in [1,6) \cup [13,\infty)$\\
E.$x \in [1,6] \cup (13,\infty)$\\
F.$x \in (1,6) \cup (13,\infty)$\\
G.$x \in [1,6) \cup (13,\infty)$\\
H.$x \in (1,6] \cup (13,\infty)$
\testStop
\kluczStart
A
\kluczStop



\zadStart{Zadanie z Wikieł Z 1.62 a) moja wersja nr 74}

Rozwiązać nierówności $(x-1)(x-6)(x-14)\ge0$.
\zadStop
\rozwStart{Patryk Wirkus}{}
Miejsca zerowe naszego wielomianu to: $1, 6, 14$.\\
Wielomian jest stopnia nieparzystego, ponadto znak współczynnika przy\linebreak najwyższej potędze x jest dodatni.\\ W związku z tym wykres wielomianu zaczyna się od lewej strony poniżej osi OX. A więc $$x \in [1,6] \cup [14,\infty).$$
\rozwStop
\odpStart
$x \in [1,6] \cup [14,\infty)$
\odpStop
\testStart
A.$x \in [1,6] \cup [14,\infty)$\\
B.$x \in (1,6) \cup [14,\infty)$\\
C.$x \in (1,6] \cup [14,\infty)$\\
D.$x \in [1,6) \cup [14,\infty)$\\
E.$x \in [1,6] \cup (14,\infty)$\\
F.$x \in (1,6) \cup (14,\infty)$\\
G.$x \in [1,6) \cup (14,\infty)$\\
H.$x \in (1,6] \cup (14,\infty)$
\testStop
\kluczStart
A
\kluczStop



\zadStart{Zadanie z Wikieł Z 1.62 a) moja wersja nr 75}

Rozwiązać nierówności $(x-1)(x-6)(x-15)\ge0$.
\zadStop
\rozwStart{Patryk Wirkus}{}
Miejsca zerowe naszego wielomianu to: $1, 6, 15$.\\
Wielomian jest stopnia nieparzystego, ponadto znak współczynnika przy\linebreak najwyższej potędze x jest dodatni.\\ W związku z tym wykres wielomianu zaczyna się od lewej strony poniżej osi OX. A więc $$x \in [1,6] \cup [15,\infty).$$
\rozwStop
\odpStart
$x \in [1,6] \cup [15,\infty)$
\odpStop
\testStart
A.$x \in [1,6] \cup [15,\infty)$\\
B.$x \in (1,6) \cup [15,\infty)$\\
C.$x \in (1,6] \cup [15,\infty)$\\
D.$x \in [1,6) \cup [15,\infty)$\\
E.$x \in [1,6] \cup (15,\infty)$\\
F.$x \in (1,6) \cup (15,\infty)$\\
G.$x \in [1,6) \cup (15,\infty)$\\
H.$x \in (1,6] \cup (15,\infty)$
\testStop
\kluczStart
A
\kluczStop



\zadStart{Zadanie z Wikieł Z 1.62 a) moja wersja nr 76}

Rozwiązać nierówności $(x-1)(x-6)(x-16)\ge0$.
\zadStop
\rozwStart{Patryk Wirkus}{}
Miejsca zerowe naszego wielomianu to: $1, 6, 16$.\\
Wielomian jest stopnia nieparzystego, ponadto znak współczynnika przy\linebreak najwyższej potędze x jest dodatni.\\ W związku z tym wykres wielomianu zaczyna się od lewej strony poniżej osi OX. A więc $$x \in [1,6] \cup [16,\infty).$$
\rozwStop
\odpStart
$x \in [1,6] \cup [16,\infty)$
\odpStop
\testStart
A.$x \in [1,6] \cup [16,\infty)$\\
B.$x \in (1,6) \cup [16,\infty)$\\
C.$x \in (1,6] \cup [16,\infty)$\\
D.$x \in [1,6) \cup [16,\infty)$\\
E.$x \in [1,6] \cup (16,\infty)$\\
F.$x \in (1,6) \cup (16,\infty)$\\
G.$x \in [1,6) \cup (16,\infty)$\\
H.$x \in (1,6] \cup (16,\infty)$
\testStop
\kluczStart
A
\kluczStop



\zadStart{Zadanie z Wikieł Z 1.62 a) moja wersja nr 77}

Rozwiązać nierówności $(x-1)(x-6)(x-17)\ge0$.
\zadStop
\rozwStart{Patryk Wirkus}{}
Miejsca zerowe naszego wielomianu to: $1, 6, 17$.\\
Wielomian jest stopnia nieparzystego, ponadto znak współczynnika przy\linebreak najwyższej potędze x jest dodatni.\\ W związku z tym wykres wielomianu zaczyna się od lewej strony poniżej osi OX. A więc $$x \in [1,6] \cup [17,\infty).$$
\rozwStop
\odpStart
$x \in [1,6] \cup [17,\infty)$
\odpStop
\testStart
A.$x \in [1,6] \cup [17,\infty)$\\
B.$x \in (1,6) \cup [17,\infty)$\\
C.$x \in (1,6] \cup [17,\infty)$\\
D.$x \in [1,6) \cup [17,\infty)$\\
E.$x \in [1,6] \cup (17,\infty)$\\
F.$x \in (1,6) \cup (17,\infty)$\\
G.$x \in [1,6) \cup (17,\infty)$\\
H.$x \in (1,6] \cup (17,\infty)$
\testStop
\kluczStart
A
\kluczStop



\zadStart{Zadanie z Wikieł Z 1.62 a) moja wersja nr 78}

Rozwiązać nierówności $(x-1)(x-6)(x-18)\ge0$.
\zadStop
\rozwStart{Patryk Wirkus}{}
Miejsca zerowe naszego wielomianu to: $1, 6, 18$.\\
Wielomian jest stopnia nieparzystego, ponadto znak współczynnika przy\linebreak najwyższej potędze x jest dodatni.\\ W związku z tym wykres wielomianu zaczyna się od lewej strony poniżej osi OX. A więc $$x \in [1,6] \cup [18,\infty).$$
\rozwStop
\odpStart
$x \in [1,6] \cup [18,\infty)$
\odpStop
\testStart
A.$x \in [1,6] \cup [18,\infty)$\\
B.$x \in (1,6) \cup [18,\infty)$\\
C.$x \in (1,6] \cup [18,\infty)$\\
D.$x \in [1,6) \cup [18,\infty)$\\
E.$x \in [1,6] \cup (18,\infty)$\\
F.$x \in (1,6) \cup (18,\infty)$\\
G.$x \in [1,6) \cup (18,\infty)$\\
H.$x \in (1,6] \cup (18,\infty)$
\testStop
\kluczStart
A
\kluczStop



\zadStart{Zadanie z Wikieł Z 1.62 a) moja wersja nr 79}

Rozwiązać nierówności $(x-1)(x-6)(x-19)\ge0$.
\zadStop
\rozwStart{Patryk Wirkus}{}
Miejsca zerowe naszego wielomianu to: $1, 6, 19$.\\
Wielomian jest stopnia nieparzystego, ponadto znak współczynnika przy\linebreak najwyższej potędze x jest dodatni.\\ W związku z tym wykres wielomianu zaczyna się od lewej strony poniżej osi OX. A więc $$x \in [1,6] \cup [19,\infty).$$
\rozwStop
\odpStart
$x \in [1,6] \cup [19,\infty)$
\odpStop
\testStart
A.$x \in [1,6] \cup [19,\infty)$\\
B.$x \in (1,6) \cup [19,\infty)$\\
C.$x \in (1,6] \cup [19,\infty)$\\
D.$x \in [1,6) \cup [19,\infty)$\\
E.$x \in [1,6] \cup (19,\infty)$\\
F.$x \in (1,6) \cup (19,\infty)$\\
G.$x \in [1,6) \cup (19,\infty)$\\
H.$x \in (1,6] \cup (19,\infty)$
\testStop
\kluczStart
A
\kluczStop



\zadStart{Zadanie z Wikieł Z 1.62 a) moja wersja nr 80}

Rozwiązać nierówności $(x-1)(x-6)(x-20)\ge0$.
\zadStop
\rozwStart{Patryk Wirkus}{}
Miejsca zerowe naszego wielomianu to: $1, 6, 20$.\\
Wielomian jest stopnia nieparzystego, ponadto znak współczynnika przy\linebreak najwyższej potędze x jest dodatni.\\ W związku z tym wykres wielomianu zaczyna się od lewej strony poniżej osi OX. A więc $$x \in [1,6] \cup [20,\infty).$$
\rozwStop
\odpStart
$x \in [1,6] \cup [20,\infty)$
\odpStop
\testStart
A.$x \in [1,6] \cup [20,\infty)$\\
B.$x \in (1,6) \cup [20,\infty)$\\
C.$x \in (1,6] \cup [20,\infty)$\\
D.$x \in [1,6) \cup [20,\infty)$\\
E.$x \in [1,6] \cup (20,\infty)$\\
F.$x \in (1,6) \cup (20,\infty)$\\
G.$x \in [1,6) \cup (20,\infty)$\\
H.$x \in (1,6] \cup (20,\infty)$
\testStop
\kluczStart
A
\kluczStop



\zadStart{Zadanie z Wikieł Z 1.62 a) moja wersja nr 81}

Rozwiązać nierówności $(x-1)(x-7)(x-8)\ge0$.
\zadStop
\rozwStart{Patryk Wirkus}{}
Miejsca zerowe naszego wielomianu to: $1, 7, 8$.\\
Wielomian jest stopnia nieparzystego, ponadto znak współczynnika przy\linebreak najwyższej potędze x jest dodatni.\\ W związku z tym wykres wielomianu zaczyna się od lewej strony poniżej osi OX. A więc $$x \in [1,7] \cup [8,\infty).$$
\rozwStop
\odpStart
$x \in [1,7] \cup [8,\infty)$
\odpStop
\testStart
A.$x \in [1,7] \cup [8,\infty)$\\
B.$x \in (1,7) \cup [8,\infty)$\\
C.$x \in (1,7] \cup [8,\infty)$\\
D.$x \in [1,7) \cup [8,\infty)$\\
E.$x \in [1,7] \cup (8,\infty)$\\
F.$x \in (1,7) \cup (8,\infty)$\\
G.$x \in [1,7) \cup (8,\infty)$\\
H.$x \in (1,7] \cup (8,\infty)$
\testStop
\kluczStart
A
\kluczStop



\zadStart{Zadanie z Wikieł Z 1.62 a) moja wersja nr 82}

Rozwiązać nierówności $(x-1)(x-7)(x-9)\ge0$.
\zadStop
\rozwStart{Patryk Wirkus}{}
Miejsca zerowe naszego wielomianu to: $1, 7, 9$.\\
Wielomian jest stopnia nieparzystego, ponadto znak współczynnika przy\linebreak najwyższej potędze x jest dodatni.\\ W związku z tym wykres wielomianu zaczyna się od lewej strony poniżej osi OX. A więc $$x \in [1,7] \cup [9,\infty).$$
\rozwStop
\odpStart
$x \in [1,7] \cup [9,\infty)$
\odpStop
\testStart
A.$x \in [1,7] \cup [9,\infty)$\\
B.$x \in (1,7) \cup [9,\infty)$\\
C.$x \in (1,7] \cup [9,\infty)$\\
D.$x \in [1,7) \cup [9,\infty)$\\
E.$x \in [1,7] \cup (9,\infty)$\\
F.$x \in (1,7) \cup (9,\infty)$\\
G.$x \in [1,7) \cup (9,\infty)$\\
H.$x \in (1,7] \cup (9,\infty)$
\testStop
\kluczStart
A
\kluczStop



\zadStart{Zadanie z Wikieł Z 1.62 a) moja wersja nr 83}

Rozwiązać nierówności $(x-1)(x-7)(x-10)\ge0$.
\zadStop
\rozwStart{Patryk Wirkus}{}
Miejsca zerowe naszego wielomianu to: $1, 7, 10$.\\
Wielomian jest stopnia nieparzystego, ponadto znak współczynnika przy\linebreak najwyższej potędze x jest dodatni.\\ W związku z tym wykres wielomianu zaczyna się od lewej strony poniżej osi OX. A więc $$x \in [1,7] \cup [10,\infty).$$
\rozwStop
\odpStart
$x \in [1,7] \cup [10,\infty)$
\odpStop
\testStart
A.$x \in [1,7] \cup [10,\infty)$\\
B.$x \in (1,7) \cup [10,\infty)$\\
C.$x \in (1,7] \cup [10,\infty)$\\
D.$x \in [1,7) \cup [10,\infty)$\\
E.$x \in [1,7] \cup (10,\infty)$\\
F.$x \in (1,7) \cup (10,\infty)$\\
G.$x \in [1,7) \cup (10,\infty)$\\
H.$x \in (1,7] \cup (10,\infty)$
\testStop
\kluczStart
A
\kluczStop



\zadStart{Zadanie z Wikieł Z 1.62 a) moja wersja nr 84}

Rozwiązać nierówności $(x-1)(x-7)(x-11)\ge0$.
\zadStop
\rozwStart{Patryk Wirkus}{}
Miejsca zerowe naszego wielomianu to: $1, 7, 11$.\\
Wielomian jest stopnia nieparzystego, ponadto znak współczynnika przy\linebreak najwyższej potędze x jest dodatni.\\ W związku z tym wykres wielomianu zaczyna się od lewej strony poniżej osi OX. A więc $$x \in [1,7] \cup [11,\infty).$$
\rozwStop
\odpStart
$x \in [1,7] \cup [11,\infty)$
\odpStop
\testStart
A.$x \in [1,7] \cup [11,\infty)$\\
B.$x \in (1,7) \cup [11,\infty)$\\
C.$x \in (1,7] \cup [11,\infty)$\\
D.$x \in [1,7) \cup [11,\infty)$\\
E.$x \in [1,7] \cup (11,\infty)$\\
F.$x \in (1,7) \cup (11,\infty)$\\
G.$x \in [1,7) \cup (11,\infty)$\\
H.$x \in (1,7] \cup (11,\infty)$
\testStop
\kluczStart
A
\kluczStop



\zadStart{Zadanie z Wikieł Z 1.62 a) moja wersja nr 85}

Rozwiązać nierówności $(x-1)(x-7)(x-12)\ge0$.
\zadStop
\rozwStart{Patryk Wirkus}{}
Miejsca zerowe naszego wielomianu to: $1, 7, 12$.\\
Wielomian jest stopnia nieparzystego, ponadto znak współczynnika przy\linebreak najwyższej potędze x jest dodatni.\\ W związku z tym wykres wielomianu zaczyna się od lewej strony poniżej osi OX. A więc $$x \in [1,7] \cup [12,\infty).$$
\rozwStop
\odpStart
$x \in [1,7] \cup [12,\infty)$
\odpStop
\testStart
A.$x \in [1,7] \cup [12,\infty)$\\
B.$x \in (1,7) \cup [12,\infty)$\\
C.$x \in (1,7] \cup [12,\infty)$\\
D.$x \in [1,7) \cup [12,\infty)$\\
E.$x \in [1,7] \cup (12,\infty)$\\
F.$x \in (1,7) \cup (12,\infty)$\\
G.$x \in [1,7) \cup (12,\infty)$\\
H.$x \in (1,7] \cup (12,\infty)$
\testStop
\kluczStart
A
\kluczStop



\zadStart{Zadanie z Wikieł Z 1.62 a) moja wersja nr 86}

Rozwiązać nierówności $(x-1)(x-7)(x-13)\ge0$.
\zadStop
\rozwStart{Patryk Wirkus}{}
Miejsca zerowe naszego wielomianu to: $1, 7, 13$.\\
Wielomian jest stopnia nieparzystego, ponadto znak współczynnika przy\linebreak najwyższej potędze x jest dodatni.\\ W związku z tym wykres wielomianu zaczyna się od lewej strony poniżej osi OX. A więc $$x \in [1,7] \cup [13,\infty).$$
\rozwStop
\odpStart
$x \in [1,7] \cup [13,\infty)$
\odpStop
\testStart
A.$x \in [1,7] \cup [13,\infty)$\\
B.$x \in (1,7) \cup [13,\infty)$\\
C.$x \in (1,7] \cup [13,\infty)$\\
D.$x \in [1,7) \cup [13,\infty)$\\
E.$x \in [1,7] \cup (13,\infty)$\\
F.$x \in (1,7) \cup (13,\infty)$\\
G.$x \in [1,7) \cup (13,\infty)$\\
H.$x \in (1,7] \cup (13,\infty)$
\testStop
\kluczStart
A
\kluczStop



\zadStart{Zadanie z Wikieł Z 1.62 a) moja wersja nr 87}

Rozwiązać nierówności $(x-1)(x-7)(x-14)\ge0$.
\zadStop
\rozwStart{Patryk Wirkus}{}
Miejsca zerowe naszego wielomianu to: $1, 7, 14$.\\
Wielomian jest stopnia nieparzystego, ponadto znak współczynnika przy\linebreak najwyższej potędze x jest dodatni.\\ W związku z tym wykres wielomianu zaczyna się od lewej strony poniżej osi OX. A więc $$x \in [1,7] \cup [14,\infty).$$
\rozwStop
\odpStart
$x \in [1,7] \cup [14,\infty)$
\odpStop
\testStart
A.$x \in [1,7] \cup [14,\infty)$\\
B.$x \in (1,7) \cup [14,\infty)$\\
C.$x \in (1,7] \cup [14,\infty)$\\
D.$x \in [1,7) \cup [14,\infty)$\\
E.$x \in [1,7] \cup (14,\infty)$\\
F.$x \in (1,7) \cup (14,\infty)$\\
G.$x \in [1,7) \cup (14,\infty)$\\
H.$x \in (1,7] \cup (14,\infty)$
\testStop
\kluczStart
A
\kluczStop



\zadStart{Zadanie z Wikieł Z 1.62 a) moja wersja nr 88}

Rozwiązać nierówności $(x-1)(x-7)(x-15)\ge0$.
\zadStop
\rozwStart{Patryk Wirkus}{}
Miejsca zerowe naszego wielomianu to: $1, 7, 15$.\\
Wielomian jest stopnia nieparzystego, ponadto znak współczynnika przy\linebreak najwyższej potędze x jest dodatni.\\ W związku z tym wykres wielomianu zaczyna się od lewej strony poniżej osi OX. A więc $$x \in [1,7] \cup [15,\infty).$$
\rozwStop
\odpStart
$x \in [1,7] \cup [15,\infty)$
\odpStop
\testStart
A.$x \in [1,7] \cup [15,\infty)$\\
B.$x \in (1,7) \cup [15,\infty)$\\
C.$x \in (1,7] \cup [15,\infty)$\\
D.$x \in [1,7) \cup [15,\infty)$\\
E.$x \in [1,7] \cup (15,\infty)$\\
F.$x \in (1,7) \cup (15,\infty)$\\
G.$x \in [1,7) \cup (15,\infty)$\\
H.$x \in (1,7] \cup (15,\infty)$
\testStop
\kluczStart
A
\kluczStop



\zadStart{Zadanie z Wikieł Z 1.62 a) moja wersja nr 89}

Rozwiązać nierówności $(x-1)(x-7)(x-16)\ge0$.
\zadStop
\rozwStart{Patryk Wirkus}{}
Miejsca zerowe naszego wielomianu to: $1, 7, 16$.\\
Wielomian jest stopnia nieparzystego, ponadto znak współczynnika przy\linebreak najwyższej potędze x jest dodatni.\\ W związku z tym wykres wielomianu zaczyna się od lewej strony poniżej osi OX. A więc $$x \in [1,7] \cup [16,\infty).$$
\rozwStop
\odpStart
$x \in [1,7] \cup [16,\infty)$
\odpStop
\testStart
A.$x \in [1,7] \cup [16,\infty)$\\
B.$x \in (1,7) \cup [16,\infty)$\\
C.$x \in (1,7] \cup [16,\infty)$\\
D.$x \in [1,7) \cup [16,\infty)$\\
E.$x \in [1,7] \cup (16,\infty)$\\
F.$x \in (1,7) \cup (16,\infty)$\\
G.$x \in [1,7) \cup (16,\infty)$\\
H.$x \in (1,7] \cup (16,\infty)$
\testStop
\kluczStart
A
\kluczStop



\zadStart{Zadanie z Wikieł Z 1.62 a) moja wersja nr 90}

Rozwiązać nierówności $(x-1)(x-7)(x-17)\ge0$.
\zadStop
\rozwStart{Patryk Wirkus}{}
Miejsca zerowe naszego wielomianu to: $1, 7, 17$.\\
Wielomian jest stopnia nieparzystego, ponadto znak współczynnika przy\linebreak najwyższej potędze x jest dodatni.\\ W związku z tym wykres wielomianu zaczyna się od lewej strony poniżej osi OX. A więc $$x \in [1,7] \cup [17,\infty).$$
\rozwStop
\odpStart
$x \in [1,7] \cup [17,\infty)$
\odpStop
\testStart
A.$x \in [1,7] \cup [17,\infty)$\\
B.$x \in (1,7) \cup [17,\infty)$\\
C.$x \in (1,7] \cup [17,\infty)$\\
D.$x \in [1,7) \cup [17,\infty)$\\
E.$x \in [1,7] \cup (17,\infty)$\\
F.$x \in (1,7) \cup (17,\infty)$\\
G.$x \in [1,7) \cup (17,\infty)$\\
H.$x \in (1,7] \cup (17,\infty)$
\testStop
\kluczStart
A
\kluczStop



\zadStart{Zadanie z Wikieł Z 1.62 a) moja wersja nr 91}

Rozwiązać nierówności $(x-1)(x-7)(x-18)\ge0$.
\zadStop
\rozwStart{Patryk Wirkus}{}
Miejsca zerowe naszego wielomianu to: $1, 7, 18$.\\
Wielomian jest stopnia nieparzystego, ponadto znak współczynnika przy\linebreak najwyższej potędze x jest dodatni.\\ W związku z tym wykres wielomianu zaczyna się od lewej strony poniżej osi OX. A więc $$x \in [1,7] \cup [18,\infty).$$
\rozwStop
\odpStart
$x \in [1,7] \cup [18,\infty)$
\odpStop
\testStart
A.$x \in [1,7] \cup [18,\infty)$\\
B.$x \in (1,7) \cup [18,\infty)$\\
C.$x \in (1,7] \cup [18,\infty)$\\
D.$x \in [1,7) \cup [18,\infty)$\\
E.$x \in [1,7] \cup (18,\infty)$\\
F.$x \in (1,7) \cup (18,\infty)$\\
G.$x \in [1,7) \cup (18,\infty)$\\
H.$x \in (1,7] \cup (18,\infty)$
\testStop
\kluczStart
A
\kluczStop



\zadStart{Zadanie z Wikieł Z 1.62 a) moja wersja nr 92}

Rozwiązać nierówności $(x-1)(x-7)(x-19)\ge0$.
\zadStop
\rozwStart{Patryk Wirkus}{}
Miejsca zerowe naszego wielomianu to: $1, 7, 19$.\\
Wielomian jest stopnia nieparzystego, ponadto znak współczynnika przy\linebreak najwyższej potędze x jest dodatni.\\ W związku z tym wykres wielomianu zaczyna się od lewej strony poniżej osi OX. A więc $$x \in [1,7] \cup [19,\infty).$$
\rozwStop
\odpStart
$x \in [1,7] \cup [19,\infty)$
\odpStop
\testStart
A.$x \in [1,7] \cup [19,\infty)$\\
B.$x \in (1,7) \cup [19,\infty)$\\
C.$x \in (1,7] \cup [19,\infty)$\\
D.$x \in [1,7) \cup [19,\infty)$\\
E.$x \in [1,7] \cup (19,\infty)$\\
F.$x \in (1,7) \cup (19,\infty)$\\
G.$x \in [1,7) \cup (19,\infty)$\\
H.$x \in (1,7] \cup (19,\infty)$
\testStop
\kluczStart
A
\kluczStop



\zadStart{Zadanie z Wikieł Z 1.62 a) moja wersja nr 93}

Rozwiązać nierówności $(x-1)(x-7)(x-20)\ge0$.
\zadStop
\rozwStart{Patryk Wirkus}{}
Miejsca zerowe naszego wielomianu to: $1, 7, 20$.\\
Wielomian jest stopnia nieparzystego, ponadto znak współczynnika przy\linebreak najwyższej potędze x jest dodatni.\\ W związku z tym wykres wielomianu zaczyna się od lewej strony poniżej osi OX. A więc $$x \in [1,7] \cup [20,\infty).$$
\rozwStop
\odpStart
$x \in [1,7] \cup [20,\infty)$
\odpStop
\testStart
A.$x \in [1,7] \cup [20,\infty)$\\
B.$x \in (1,7) \cup [20,\infty)$\\
C.$x \in (1,7] \cup [20,\infty)$\\
D.$x \in [1,7) \cup [20,\infty)$\\
E.$x \in [1,7] \cup (20,\infty)$\\
F.$x \in (1,7) \cup (20,\infty)$\\
G.$x \in [1,7) \cup (20,\infty)$\\
H.$x \in (1,7] \cup (20,\infty)$
\testStop
\kluczStart
A
\kluczStop



\zadStart{Zadanie z Wikieł Z 1.62 a) moja wersja nr 94}

Rozwiązać nierówności $(x-1)(x-8)(x-9)\ge0$.
\zadStop
\rozwStart{Patryk Wirkus}{}
Miejsca zerowe naszego wielomianu to: $1, 8, 9$.\\
Wielomian jest stopnia nieparzystego, ponadto znak współczynnika przy\linebreak najwyższej potędze x jest dodatni.\\ W związku z tym wykres wielomianu zaczyna się od lewej strony poniżej osi OX. A więc $$x \in [1,8] \cup [9,\infty).$$
\rozwStop
\odpStart
$x \in [1,8] \cup [9,\infty)$
\odpStop
\testStart
A.$x \in [1,8] \cup [9,\infty)$\\
B.$x \in (1,8) \cup [9,\infty)$\\
C.$x \in (1,8] \cup [9,\infty)$\\
D.$x \in [1,8) \cup [9,\infty)$\\
E.$x \in [1,8] \cup (9,\infty)$\\
F.$x \in (1,8) \cup (9,\infty)$\\
G.$x \in [1,8) \cup (9,\infty)$\\
H.$x \in (1,8] \cup (9,\infty)$
\testStop
\kluczStart
A
\kluczStop



\zadStart{Zadanie z Wikieł Z 1.62 a) moja wersja nr 95}

Rozwiązać nierówności $(x-1)(x-8)(x-10)\ge0$.
\zadStop
\rozwStart{Patryk Wirkus}{}
Miejsca zerowe naszego wielomianu to: $1, 8, 10$.\\
Wielomian jest stopnia nieparzystego, ponadto znak współczynnika przy\linebreak najwyższej potędze x jest dodatni.\\ W związku z tym wykres wielomianu zaczyna się od lewej strony poniżej osi OX. A więc $$x \in [1,8] \cup [10,\infty).$$
\rozwStop
\odpStart
$x \in [1,8] \cup [10,\infty)$
\odpStop
\testStart
A.$x \in [1,8] \cup [10,\infty)$\\
B.$x \in (1,8) \cup [10,\infty)$\\
C.$x \in (1,8] \cup [10,\infty)$\\
D.$x \in [1,8) \cup [10,\infty)$\\
E.$x \in [1,8] \cup (10,\infty)$\\
F.$x \in (1,8) \cup (10,\infty)$\\
G.$x \in [1,8) \cup (10,\infty)$\\
H.$x \in (1,8] \cup (10,\infty)$
\testStop
\kluczStart
A
\kluczStop



\zadStart{Zadanie z Wikieł Z 1.62 a) moja wersja nr 96}

Rozwiązać nierówności $(x-1)(x-8)(x-11)\ge0$.
\zadStop
\rozwStart{Patryk Wirkus}{}
Miejsca zerowe naszego wielomianu to: $1, 8, 11$.\\
Wielomian jest stopnia nieparzystego, ponadto znak współczynnika przy\linebreak najwyższej potędze x jest dodatni.\\ W związku z tym wykres wielomianu zaczyna się od lewej strony poniżej osi OX. A więc $$x \in [1,8] \cup [11,\infty).$$
\rozwStop
\odpStart
$x \in [1,8] \cup [11,\infty)$
\odpStop
\testStart
A.$x \in [1,8] \cup [11,\infty)$\\
B.$x \in (1,8) \cup [11,\infty)$\\
C.$x \in (1,8] \cup [11,\infty)$\\
D.$x \in [1,8) \cup [11,\infty)$\\
E.$x \in [1,8] \cup (11,\infty)$\\
F.$x \in (1,8) \cup (11,\infty)$\\
G.$x \in [1,8) \cup (11,\infty)$\\
H.$x \in (1,8] \cup (11,\infty)$
\testStop
\kluczStart
A
\kluczStop



\zadStart{Zadanie z Wikieł Z 1.62 a) moja wersja nr 97}

Rozwiązać nierówności $(x-1)(x-8)(x-12)\ge0$.
\zadStop
\rozwStart{Patryk Wirkus}{}
Miejsca zerowe naszego wielomianu to: $1, 8, 12$.\\
Wielomian jest stopnia nieparzystego, ponadto znak współczynnika przy\linebreak najwyższej potędze x jest dodatni.\\ W związku z tym wykres wielomianu zaczyna się od lewej strony poniżej osi OX. A więc $$x \in [1,8] \cup [12,\infty).$$
\rozwStop
\odpStart
$x \in [1,8] \cup [12,\infty)$
\odpStop
\testStart
A.$x \in [1,8] \cup [12,\infty)$\\
B.$x \in (1,8) \cup [12,\infty)$\\
C.$x \in (1,8] \cup [12,\infty)$\\
D.$x \in [1,8) \cup [12,\infty)$\\
E.$x \in [1,8] \cup (12,\infty)$\\
F.$x \in (1,8) \cup (12,\infty)$\\
G.$x \in [1,8) \cup (12,\infty)$\\
H.$x \in (1,8] \cup (12,\infty)$
\testStop
\kluczStart
A
\kluczStop



\zadStart{Zadanie z Wikieł Z 1.62 a) moja wersja nr 98}

Rozwiązać nierówności $(x-1)(x-8)(x-13)\ge0$.
\zadStop
\rozwStart{Patryk Wirkus}{}
Miejsca zerowe naszego wielomianu to: $1, 8, 13$.\\
Wielomian jest stopnia nieparzystego, ponadto znak współczynnika przy\linebreak najwyższej potędze x jest dodatni.\\ W związku z tym wykres wielomianu zaczyna się od lewej strony poniżej osi OX. A więc $$x \in [1,8] \cup [13,\infty).$$
\rozwStop
\odpStart
$x \in [1,8] \cup [13,\infty)$
\odpStop
\testStart
A.$x \in [1,8] \cup [13,\infty)$\\
B.$x \in (1,8) \cup [13,\infty)$\\
C.$x \in (1,8] \cup [13,\infty)$\\
D.$x \in [1,8) \cup [13,\infty)$\\
E.$x \in [1,8] \cup (13,\infty)$\\
F.$x \in (1,8) \cup (13,\infty)$\\
G.$x \in [1,8) \cup (13,\infty)$\\
H.$x \in (1,8] \cup (13,\infty)$
\testStop
\kluczStart
A
\kluczStop



\zadStart{Zadanie z Wikieł Z 1.62 a) moja wersja nr 99}

Rozwiązać nierówności $(x-1)(x-8)(x-14)\ge0$.
\zadStop
\rozwStart{Patryk Wirkus}{}
Miejsca zerowe naszego wielomianu to: $1, 8, 14$.\\
Wielomian jest stopnia nieparzystego, ponadto znak współczynnika przy\linebreak najwyższej potędze x jest dodatni.\\ W związku z tym wykres wielomianu zaczyna się od lewej strony poniżej osi OX. A więc $$x \in [1,8] \cup [14,\infty).$$
\rozwStop
\odpStart
$x \in [1,8] \cup [14,\infty)$
\odpStop
\testStart
A.$x \in [1,8] \cup [14,\infty)$\\
B.$x \in (1,8) \cup [14,\infty)$\\
C.$x \in (1,8] \cup [14,\infty)$\\
D.$x \in [1,8) \cup [14,\infty)$\\
E.$x \in [1,8] \cup (14,\infty)$\\
F.$x \in (1,8) \cup (14,\infty)$\\
G.$x \in [1,8) \cup (14,\infty)$\\
H.$x \in (1,8] \cup (14,\infty)$
\testStop
\kluczStart
A
\kluczStop



\zadStart{Zadanie z Wikieł Z 1.62 a) moja wersja nr 100}

Rozwiązać nierówności $(x-1)(x-8)(x-15)\ge0$.
\zadStop
\rozwStart{Patryk Wirkus}{}
Miejsca zerowe naszego wielomianu to: $1, 8, 15$.\\
Wielomian jest stopnia nieparzystego, ponadto znak współczynnika przy\linebreak najwyższej potędze x jest dodatni.\\ W związku z tym wykres wielomianu zaczyna się od lewej strony poniżej osi OX. A więc $$x \in [1,8] \cup [15,\infty).$$
\rozwStop
\odpStart
$x \in [1,8] \cup [15,\infty)$
\odpStop
\testStart
A.$x \in [1,8] \cup [15,\infty)$\\
B.$x \in (1,8) \cup [15,\infty)$\\
C.$x \in (1,8] \cup [15,\infty)$\\
D.$x \in [1,8) \cup [15,\infty)$\\
E.$x \in [1,8] \cup (15,\infty)$\\
F.$x \in (1,8) \cup (15,\infty)$\\
G.$x \in [1,8) \cup (15,\infty)$\\
H.$x \in (1,8] \cup (15,\infty)$
\testStop
\kluczStart
A
\kluczStop



\zadStart{Zadanie z Wikieł Z 1.62 a) moja wersja nr 101}

Rozwiązać nierówności $(x-1)(x-8)(x-16)\ge0$.
\zadStop
\rozwStart{Patryk Wirkus}{}
Miejsca zerowe naszego wielomianu to: $1, 8, 16$.\\
Wielomian jest stopnia nieparzystego, ponadto znak współczynnika przy\linebreak najwyższej potędze x jest dodatni.\\ W związku z tym wykres wielomianu zaczyna się od lewej strony poniżej osi OX. A więc $$x \in [1,8] \cup [16,\infty).$$
\rozwStop
\odpStart
$x \in [1,8] \cup [16,\infty)$
\odpStop
\testStart
A.$x \in [1,8] \cup [16,\infty)$\\
B.$x \in (1,8) \cup [16,\infty)$\\
C.$x \in (1,8] \cup [16,\infty)$\\
D.$x \in [1,8) \cup [16,\infty)$\\
E.$x \in [1,8] \cup (16,\infty)$\\
F.$x \in (1,8) \cup (16,\infty)$\\
G.$x \in [1,8) \cup (16,\infty)$\\
H.$x \in (1,8] \cup (16,\infty)$
\testStop
\kluczStart
A
\kluczStop



\zadStart{Zadanie z Wikieł Z 1.62 a) moja wersja nr 102}

Rozwiązać nierówności $(x-1)(x-8)(x-17)\ge0$.
\zadStop
\rozwStart{Patryk Wirkus}{}
Miejsca zerowe naszego wielomianu to: $1, 8, 17$.\\
Wielomian jest stopnia nieparzystego, ponadto znak współczynnika przy\linebreak najwyższej potędze x jest dodatni.\\ W związku z tym wykres wielomianu zaczyna się od lewej strony poniżej osi OX. A więc $$x \in [1,8] \cup [17,\infty).$$
\rozwStop
\odpStart
$x \in [1,8] \cup [17,\infty)$
\odpStop
\testStart
A.$x \in [1,8] \cup [17,\infty)$\\
B.$x \in (1,8) \cup [17,\infty)$\\
C.$x \in (1,8] \cup [17,\infty)$\\
D.$x \in [1,8) \cup [17,\infty)$\\
E.$x \in [1,8] \cup (17,\infty)$\\
F.$x \in (1,8) \cup (17,\infty)$\\
G.$x \in [1,8) \cup (17,\infty)$\\
H.$x \in (1,8] \cup (17,\infty)$
\testStop
\kluczStart
A
\kluczStop



\zadStart{Zadanie z Wikieł Z 1.62 a) moja wersja nr 103}

Rozwiązać nierówności $(x-1)(x-8)(x-18)\ge0$.
\zadStop
\rozwStart{Patryk Wirkus}{}
Miejsca zerowe naszego wielomianu to: $1, 8, 18$.\\
Wielomian jest stopnia nieparzystego, ponadto znak współczynnika przy\linebreak najwyższej potędze x jest dodatni.\\ W związku z tym wykres wielomianu zaczyna się od lewej strony poniżej osi OX. A więc $$x \in [1,8] \cup [18,\infty).$$
\rozwStop
\odpStart
$x \in [1,8] \cup [18,\infty)$
\odpStop
\testStart
A.$x \in [1,8] \cup [18,\infty)$\\
B.$x \in (1,8) \cup [18,\infty)$\\
C.$x \in (1,8] \cup [18,\infty)$\\
D.$x \in [1,8) \cup [18,\infty)$\\
E.$x \in [1,8] \cup (18,\infty)$\\
F.$x \in (1,8) \cup (18,\infty)$\\
G.$x \in [1,8) \cup (18,\infty)$\\
H.$x \in (1,8] \cup (18,\infty)$
\testStop
\kluczStart
A
\kluczStop



\zadStart{Zadanie z Wikieł Z 1.62 a) moja wersja nr 104}

Rozwiązać nierówności $(x-1)(x-8)(x-19)\ge0$.
\zadStop
\rozwStart{Patryk Wirkus}{}
Miejsca zerowe naszego wielomianu to: $1, 8, 19$.\\
Wielomian jest stopnia nieparzystego, ponadto znak współczynnika przy\linebreak najwyższej potędze x jest dodatni.\\ W związku z tym wykres wielomianu zaczyna się od lewej strony poniżej osi OX. A więc $$x \in [1,8] \cup [19,\infty).$$
\rozwStop
\odpStart
$x \in [1,8] \cup [19,\infty)$
\odpStop
\testStart
A.$x \in [1,8] \cup [19,\infty)$\\
B.$x \in (1,8) \cup [19,\infty)$\\
C.$x \in (1,8] \cup [19,\infty)$\\
D.$x \in [1,8) \cup [19,\infty)$\\
E.$x \in [1,8] \cup (19,\infty)$\\
F.$x \in (1,8) \cup (19,\infty)$\\
G.$x \in [1,8) \cup (19,\infty)$\\
H.$x \in (1,8] \cup (19,\infty)$
\testStop
\kluczStart
A
\kluczStop



\zadStart{Zadanie z Wikieł Z 1.62 a) moja wersja nr 105}

Rozwiązać nierówności $(x-1)(x-8)(x-20)\ge0$.
\zadStop
\rozwStart{Patryk Wirkus}{}
Miejsca zerowe naszego wielomianu to: $1, 8, 20$.\\
Wielomian jest stopnia nieparzystego, ponadto znak współczynnika przy\linebreak najwyższej potędze x jest dodatni.\\ W związku z tym wykres wielomianu zaczyna się od lewej strony poniżej osi OX. A więc $$x \in [1,8] \cup [20,\infty).$$
\rozwStop
\odpStart
$x \in [1,8] \cup [20,\infty)$
\odpStop
\testStart
A.$x \in [1,8] \cup [20,\infty)$\\
B.$x \in (1,8) \cup [20,\infty)$\\
C.$x \in (1,8] \cup [20,\infty)$\\
D.$x \in [1,8) \cup [20,\infty)$\\
E.$x \in [1,8] \cup (20,\infty)$\\
F.$x \in (1,8) \cup (20,\infty)$\\
G.$x \in [1,8) \cup (20,\infty)$\\
H.$x \in (1,8] \cup (20,\infty)$
\testStop
\kluczStart
A
\kluczStop



\zadStart{Zadanie z Wikieł Z 1.62 a) moja wersja nr 106}

Rozwiązać nierówności $(x-1)(x-9)(x-10)\ge0$.
\zadStop
\rozwStart{Patryk Wirkus}{}
Miejsca zerowe naszego wielomianu to: $1, 9, 10$.\\
Wielomian jest stopnia nieparzystego, ponadto znak współczynnika przy\linebreak najwyższej potędze x jest dodatni.\\ W związku z tym wykres wielomianu zaczyna się od lewej strony poniżej osi OX. A więc $$x \in [1,9] \cup [10,\infty).$$
\rozwStop
\odpStart
$x \in [1,9] \cup [10,\infty)$
\odpStop
\testStart
A.$x \in [1,9] \cup [10,\infty)$\\
B.$x \in (1,9) \cup [10,\infty)$\\
C.$x \in (1,9] \cup [10,\infty)$\\
D.$x \in [1,9) \cup [10,\infty)$\\
E.$x \in [1,9] \cup (10,\infty)$\\
F.$x \in (1,9) \cup (10,\infty)$\\
G.$x \in [1,9) \cup (10,\infty)$\\
H.$x \in (1,9] \cup (10,\infty)$
\testStop
\kluczStart
A
\kluczStop



\zadStart{Zadanie z Wikieł Z 1.62 a) moja wersja nr 107}

Rozwiązać nierówności $(x-1)(x-9)(x-11)\ge0$.
\zadStop
\rozwStart{Patryk Wirkus}{}
Miejsca zerowe naszego wielomianu to: $1, 9, 11$.\\
Wielomian jest stopnia nieparzystego, ponadto znak współczynnika przy\linebreak najwyższej potędze x jest dodatni.\\ W związku z tym wykres wielomianu zaczyna się od lewej strony poniżej osi OX. A więc $$x \in [1,9] \cup [11,\infty).$$
\rozwStop
\odpStart
$x \in [1,9] \cup [11,\infty)$
\odpStop
\testStart
A.$x \in [1,9] \cup [11,\infty)$\\
B.$x \in (1,9) \cup [11,\infty)$\\
C.$x \in (1,9] \cup [11,\infty)$\\
D.$x \in [1,9) \cup [11,\infty)$\\
E.$x \in [1,9] \cup (11,\infty)$\\
F.$x \in (1,9) \cup (11,\infty)$\\
G.$x \in [1,9) \cup (11,\infty)$\\
H.$x \in (1,9] \cup (11,\infty)$
\testStop
\kluczStart
A
\kluczStop



\zadStart{Zadanie z Wikieł Z 1.62 a) moja wersja nr 108}

Rozwiązać nierówności $(x-1)(x-9)(x-12)\ge0$.
\zadStop
\rozwStart{Patryk Wirkus}{}
Miejsca zerowe naszego wielomianu to: $1, 9, 12$.\\
Wielomian jest stopnia nieparzystego, ponadto znak współczynnika przy\linebreak najwyższej potędze x jest dodatni.\\ W związku z tym wykres wielomianu zaczyna się od lewej strony poniżej osi OX. A więc $$x \in [1,9] \cup [12,\infty).$$
\rozwStop
\odpStart
$x \in [1,9] \cup [12,\infty)$
\odpStop
\testStart
A.$x \in [1,9] \cup [12,\infty)$\\
B.$x \in (1,9) \cup [12,\infty)$\\
C.$x \in (1,9] \cup [12,\infty)$\\
D.$x \in [1,9) \cup [12,\infty)$\\
E.$x \in [1,9] \cup (12,\infty)$\\
F.$x \in (1,9) \cup (12,\infty)$\\
G.$x \in [1,9) \cup (12,\infty)$\\
H.$x \in (1,9] \cup (12,\infty)$
\testStop
\kluczStart
A
\kluczStop



\zadStart{Zadanie z Wikieł Z 1.62 a) moja wersja nr 109}

Rozwiązać nierówności $(x-1)(x-9)(x-13)\ge0$.
\zadStop
\rozwStart{Patryk Wirkus}{}
Miejsca zerowe naszego wielomianu to: $1, 9, 13$.\\
Wielomian jest stopnia nieparzystego, ponadto znak współczynnika przy\linebreak najwyższej potędze x jest dodatni.\\ W związku z tym wykres wielomianu zaczyna się od lewej strony poniżej osi OX. A więc $$x \in [1,9] \cup [13,\infty).$$
\rozwStop
\odpStart
$x \in [1,9] \cup [13,\infty)$
\odpStop
\testStart
A.$x \in [1,9] \cup [13,\infty)$\\
B.$x \in (1,9) \cup [13,\infty)$\\
C.$x \in (1,9] \cup [13,\infty)$\\
D.$x \in [1,9) \cup [13,\infty)$\\
E.$x \in [1,9] \cup (13,\infty)$\\
F.$x \in (1,9) \cup (13,\infty)$\\
G.$x \in [1,9) \cup (13,\infty)$\\
H.$x \in (1,9] \cup (13,\infty)$
\testStop
\kluczStart
A
\kluczStop



\zadStart{Zadanie z Wikieł Z 1.62 a) moja wersja nr 110}

Rozwiązać nierówności $(x-1)(x-9)(x-14)\ge0$.
\zadStop
\rozwStart{Patryk Wirkus}{}
Miejsca zerowe naszego wielomianu to: $1, 9, 14$.\\
Wielomian jest stopnia nieparzystego, ponadto znak współczynnika przy\linebreak najwyższej potędze x jest dodatni.\\ W związku z tym wykres wielomianu zaczyna się od lewej strony poniżej osi OX. A więc $$x \in [1,9] \cup [14,\infty).$$
\rozwStop
\odpStart
$x \in [1,9] \cup [14,\infty)$
\odpStop
\testStart
A.$x \in [1,9] \cup [14,\infty)$\\
B.$x \in (1,9) \cup [14,\infty)$\\
C.$x \in (1,9] \cup [14,\infty)$\\
D.$x \in [1,9) \cup [14,\infty)$\\
E.$x \in [1,9] \cup (14,\infty)$\\
F.$x \in (1,9) \cup (14,\infty)$\\
G.$x \in [1,9) \cup (14,\infty)$\\
H.$x \in (1,9] \cup (14,\infty)$
\testStop
\kluczStart
A
\kluczStop



\zadStart{Zadanie z Wikieł Z 1.62 a) moja wersja nr 111}

Rozwiązać nierówności $(x-1)(x-9)(x-15)\ge0$.
\zadStop
\rozwStart{Patryk Wirkus}{}
Miejsca zerowe naszego wielomianu to: $1, 9, 15$.\\
Wielomian jest stopnia nieparzystego, ponadto znak współczynnika przy\linebreak najwyższej potędze x jest dodatni.\\ W związku z tym wykres wielomianu zaczyna się od lewej strony poniżej osi OX. A więc $$x \in [1,9] \cup [15,\infty).$$
\rozwStop
\odpStart
$x \in [1,9] \cup [15,\infty)$
\odpStop
\testStart
A.$x \in [1,9] \cup [15,\infty)$\\
B.$x \in (1,9) \cup [15,\infty)$\\
C.$x \in (1,9] \cup [15,\infty)$\\
D.$x \in [1,9) \cup [15,\infty)$\\
E.$x \in [1,9] \cup (15,\infty)$\\
F.$x \in (1,9) \cup (15,\infty)$\\
G.$x \in [1,9) \cup (15,\infty)$\\
H.$x \in (1,9] \cup (15,\infty)$
\testStop
\kluczStart
A
\kluczStop



\zadStart{Zadanie z Wikieł Z 1.62 a) moja wersja nr 112}

Rozwiązać nierówności $(x-1)(x-9)(x-16)\ge0$.
\zadStop
\rozwStart{Patryk Wirkus}{}
Miejsca zerowe naszego wielomianu to: $1, 9, 16$.\\
Wielomian jest stopnia nieparzystego, ponadto znak współczynnika przy\linebreak najwyższej potędze x jest dodatni.\\ W związku z tym wykres wielomianu zaczyna się od lewej strony poniżej osi OX. A więc $$x \in [1,9] \cup [16,\infty).$$
\rozwStop
\odpStart
$x \in [1,9] \cup [16,\infty)$
\odpStop
\testStart
A.$x \in [1,9] \cup [16,\infty)$\\
B.$x \in (1,9) \cup [16,\infty)$\\
C.$x \in (1,9] \cup [16,\infty)$\\
D.$x \in [1,9) \cup [16,\infty)$\\
E.$x \in [1,9] \cup (16,\infty)$\\
F.$x \in (1,9) \cup (16,\infty)$\\
G.$x \in [1,9) \cup (16,\infty)$\\
H.$x \in (1,9] \cup (16,\infty)$
\testStop
\kluczStart
A
\kluczStop



\zadStart{Zadanie z Wikieł Z 1.62 a) moja wersja nr 113}

Rozwiązać nierówności $(x-1)(x-9)(x-17)\ge0$.
\zadStop
\rozwStart{Patryk Wirkus}{}
Miejsca zerowe naszego wielomianu to: $1, 9, 17$.\\
Wielomian jest stopnia nieparzystego, ponadto znak współczynnika przy\linebreak najwyższej potędze x jest dodatni.\\ W związku z tym wykres wielomianu zaczyna się od lewej strony poniżej osi OX. A więc $$x \in [1,9] \cup [17,\infty).$$
\rozwStop
\odpStart
$x \in [1,9] \cup [17,\infty)$
\odpStop
\testStart
A.$x \in [1,9] \cup [17,\infty)$\\
B.$x \in (1,9) \cup [17,\infty)$\\
C.$x \in (1,9] \cup [17,\infty)$\\
D.$x \in [1,9) \cup [17,\infty)$\\
E.$x \in [1,9] \cup (17,\infty)$\\
F.$x \in (1,9) \cup (17,\infty)$\\
G.$x \in [1,9) \cup (17,\infty)$\\
H.$x \in (1,9] \cup (17,\infty)$
\testStop
\kluczStart
A
\kluczStop



\zadStart{Zadanie z Wikieł Z 1.62 a) moja wersja nr 114}

Rozwiązać nierówności $(x-1)(x-9)(x-18)\ge0$.
\zadStop
\rozwStart{Patryk Wirkus}{}
Miejsca zerowe naszego wielomianu to: $1, 9, 18$.\\
Wielomian jest stopnia nieparzystego, ponadto znak współczynnika przy\linebreak najwyższej potędze x jest dodatni.\\ W związku z tym wykres wielomianu zaczyna się od lewej strony poniżej osi OX. A więc $$x \in [1,9] \cup [18,\infty).$$
\rozwStop
\odpStart
$x \in [1,9] \cup [18,\infty)$
\odpStop
\testStart
A.$x \in [1,9] \cup [18,\infty)$\\
B.$x \in (1,9) \cup [18,\infty)$\\
C.$x \in (1,9] \cup [18,\infty)$\\
D.$x \in [1,9) \cup [18,\infty)$\\
E.$x \in [1,9] \cup (18,\infty)$\\
F.$x \in (1,9) \cup (18,\infty)$\\
G.$x \in [1,9) \cup (18,\infty)$\\
H.$x \in (1,9] \cup (18,\infty)$
\testStop
\kluczStart
A
\kluczStop



\zadStart{Zadanie z Wikieł Z 1.62 a) moja wersja nr 115}

Rozwiązać nierówności $(x-1)(x-9)(x-19)\ge0$.
\zadStop
\rozwStart{Patryk Wirkus}{}
Miejsca zerowe naszego wielomianu to: $1, 9, 19$.\\
Wielomian jest stopnia nieparzystego, ponadto znak współczynnika przy\linebreak najwyższej potędze x jest dodatni.\\ W związku z tym wykres wielomianu zaczyna się od lewej strony poniżej osi OX. A więc $$x \in [1,9] \cup [19,\infty).$$
\rozwStop
\odpStart
$x \in [1,9] \cup [19,\infty)$
\odpStop
\testStart
A.$x \in [1,9] \cup [19,\infty)$\\
B.$x \in (1,9) \cup [19,\infty)$\\
C.$x \in (1,9] \cup [19,\infty)$\\
D.$x \in [1,9) \cup [19,\infty)$\\
E.$x \in [1,9] \cup (19,\infty)$\\
F.$x \in (1,9) \cup (19,\infty)$\\
G.$x \in [1,9) \cup (19,\infty)$\\
H.$x \in (1,9] \cup (19,\infty)$
\testStop
\kluczStart
A
\kluczStop



\zadStart{Zadanie z Wikieł Z 1.62 a) moja wersja nr 116}

Rozwiązać nierówności $(x-1)(x-9)(x-20)\ge0$.
\zadStop
\rozwStart{Patryk Wirkus}{}
Miejsca zerowe naszego wielomianu to: $1, 9, 20$.\\
Wielomian jest stopnia nieparzystego, ponadto znak współczynnika przy\linebreak najwyższej potędze x jest dodatni.\\ W związku z tym wykres wielomianu zaczyna się od lewej strony poniżej osi OX. A więc $$x \in [1,9] \cup [20,\infty).$$
\rozwStop
\odpStart
$x \in [1,9] \cup [20,\infty)$
\odpStop
\testStart
A.$x \in [1,9] \cup [20,\infty)$\\
B.$x \in (1,9) \cup [20,\infty)$\\
C.$x \in (1,9] \cup [20,\infty)$\\
D.$x \in [1,9) \cup [20,\infty)$\\
E.$x \in [1,9] \cup (20,\infty)$\\
F.$x \in (1,9) \cup (20,\infty)$\\
G.$x \in [1,9) \cup (20,\infty)$\\
H.$x \in (1,9] \cup (20,\infty)$
\testStop
\kluczStart
A
\kluczStop



\zadStart{Zadanie z Wikieł Z 1.62 a) moja wersja nr 117}

Rozwiązać nierówności $(x-1)(x-10)(x-11)\ge0$.
\zadStop
\rozwStart{Patryk Wirkus}{}
Miejsca zerowe naszego wielomianu to: $1, 10, 11$.\\
Wielomian jest stopnia nieparzystego, ponadto znak współczynnika przy\linebreak najwyższej potędze x jest dodatni.\\ W związku z tym wykres wielomianu zaczyna się od lewej strony poniżej osi OX. A więc $$x \in [1,10] \cup [11,\infty).$$
\rozwStop
\odpStart
$x \in [1,10] \cup [11,\infty)$
\odpStop
\testStart
A.$x \in [1,10] \cup [11,\infty)$\\
B.$x \in (1,10) \cup [11,\infty)$\\
C.$x \in (1,10] \cup [11,\infty)$\\
D.$x \in [1,10) \cup [11,\infty)$\\
E.$x \in [1,10] \cup (11,\infty)$\\
F.$x \in (1,10) \cup (11,\infty)$\\
G.$x \in [1,10) \cup (11,\infty)$\\
H.$x \in (1,10] \cup (11,\infty)$
\testStop
\kluczStart
A
\kluczStop



\zadStart{Zadanie z Wikieł Z 1.62 a) moja wersja nr 118}

Rozwiązać nierówności $(x-1)(x-10)(x-12)\ge0$.
\zadStop
\rozwStart{Patryk Wirkus}{}
Miejsca zerowe naszego wielomianu to: $1, 10, 12$.\\
Wielomian jest stopnia nieparzystego, ponadto znak współczynnika przy\linebreak najwyższej potędze x jest dodatni.\\ W związku z tym wykres wielomianu zaczyna się od lewej strony poniżej osi OX. A więc $$x \in [1,10] \cup [12,\infty).$$
\rozwStop
\odpStart
$x \in [1,10] \cup [12,\infty)$
\odpStop
\testStart
A.$x \in [1,10] \cup [12,\infty)$\\
B.$x \in (1,10) \cup [12,\infty)$\\
C.$x \in (1,10] \cup [12,\infty)$\\
D.$x \in [1,10) \cup [12,\infty)$\\
E.$x \in [1,10] \cup (12,\infty)$\\
F.$x \in (1,10) \cup (12,\infty)$\\
G.$x \in [1,10) \cup (12,\infty)$\\
H.$x \in (1,10] \cup (12,\infty)$
\testStop
\kluczStart
A
\kluczStop



\zadStart{Zadanie z Wikieł Z 1.62 a) moja wersja nr 119}

Rozwiązać nierówności $(x-1)(x-10)(x-13)\ge0$.
\zadStop
\rozwStart{Patryk Wirkus}{}
Miejsca zerowe naszego wielomianu to: $1, 10, 13$.\\
Wielomian jest stopnia nieparzystego, ponadto znak współczynnika przy\linebreak najwyższej potędze x jest dodatni.\\ W związku z tym wykres wielomianu zaczyna się od lewej strony poniżej osi OX. A więc $$x \in [1,10] \cup [13,\infty).$$
\rozwStop
\odpStart
$x \in [1,10] \cup [13,\infty)$
\odpStop
\testStart
A.$x \in [1,10] \cup [13,\infty)$\\
B.$x \in (1,10) \cup [13,\infty)$\\
C.$x \in (1,10] \cup [13,\infty)$\\
D.$x \in [1,10) \cup [13,\infty)$\\
E.$x \in [1,10] \cup (13,\infty)$\\
F.$x \in (1,10) \cup (13,\infty)$\\
G.$x \in [1,10) \cup (13,\infty)$\\
H.$x \in (1,10] \cup (13,\infty)$
\testStop
\kluczStart
A
\kluczStop



\zadStart{Zadanie z Wikieł Z 1.62 a) moja wersja nr 120}

Rozwiązać nierówności $(x-1)(x-10)(x-14)\ge0$.
\zadStop
\rozwStart{Patryk Wirkus}{}
Miejsca zerowe naszego wielomianu to: $1, 10, 14$.\\
Wielomian jest stopnia nieparzystego, ponadto znak współczynnika przy\linebreak najwyższej potędze x jest dodatni.\\ W związku z tym wykres wielomianu zaczyna się od lewej strony poniżej osi OX. A więc $$x \in [1,10] \cup [14,\infty).$$
\rozwStop
\odpStart
$x \in [1,10] \cup [14,\infty)$
\odpStop
\testStart
A.$x \in [1,10] \cup [14,\infty)$\\
B.$x \in (1,10) \cup [14,\infty)$\\
C.$x \in (1,10] \cup [14,\infty)$\\
D.$x \in [1,10) \cup [14,\infty)$\\
E.$x \in [1,10] \cup (14,\infty)$\\
F.$x \in (1,10) \cup (14,\infty)$\\
G.$x \in [1,10) \cup (14,\infty)$\\
H.$x \in (1,10] \cup (14,\infty)$
\testStop
\kluczStart
A
\kluczStop



\zadStart{Zadanie z Wikieł Z 1.62 a) moja wersja nr 121}

Rozwiązać nierówności $(x-1)(x-10)(x-15)\ge0$.
\zadStop
\rozwStart{Patryk Wirkus}{}
Miejsca zerowe naszego wielomianu to: $1, 10, 15$.\\
Wielomian jest stopnia nieparzystego, ponadto znak współczynnika przy\linebreak najwyższej potędze x jest dodatni.\\ W związku z tym wykres wielomianu zaczyna się od lewej strony poniżej osi OX. A więc $$x \in [1,10] \cup [15,\infty).$$
\rozwStop
\odpStart
$x \in [1,10] \cup [15,\infty)$
\odpStop
\testStart
A.$x \in [1,10] \cup [15,\infty)$\\
B.$x \in (1,10) \cup [15,\infty)$\\
C.$x \in (1,10] \cup [15,\infty)$\\
D.$x \in [1,10) \cup [15,\infty)$\\
E.$x \in [1,10] \cup (15,\infty)$\\
F.$x \in (1,10) \cup (15,\infty)$\\
G.$x \in [1,10) \cup (15,\infty)$\\
H.$x \in (1,10] \cup (15,\infty)$
\testStop
\kluczStart
A
\kluczStop



\zadStart{Zadanie z Wikieł Z 1.62 a) moja wersja nr 122}

Rozwiązać nierówności $(x-1)(x-10)(x-16)\ge0$.
\zadStop
\rozwStart{Patryk Wirkus}{}
Miejsca zerowe naszego wielomianu to: $1, 10, 16$.\\
Wielomian jest stopnia nieparzystego, ponadto znak współczynnika przy\linebreak najwyższej potędze x jest dodatni.\\ W związku z tym wykres wielomianu zaczyna się od lewej strony poniżej osi OX. A więc $$x \in [1,10] \cup [16,\infty).$$
\rozwStop
\odpStart
$x \in [1,10] \cup [16,\infty)$
\odpStop
\testStart
A.$x \in [1,10] \cup [16,\infty)$\\
B.$x \in (1,10) \cup [16,\infty)$\\
C.$x \in (1,10] \cup [16,\infty)$\\
D.$x \in [1,10) \cup [16,\infty)$\\
E.$x \in [1,10] \cup (16,\infty)$\\
F.$x \in (1,10) \cup (16,\infty)$\\
G.$x \in [1,10) \cup (16,\infty)$\\
H.$x \in (1,10] \cup (16,\infty)$
\testStop
\kluczStart
A
\kluczStop



\zadStart{Zadanie z Wikieł Z 1.62 a) moja wersja nr 123}

Rozwiązać nierówności $(x-1)(x-10)(x-17)\ge0$.
\zadStop
\rozwStart{Patryk Wirkus}{}
Miejsca zerowe naszego wielomianu to: $1, 10, 17$.\\
Wielomian jest stopnia nieparzystego, ponadto znak współczynnika przy\linebreak najwyższej potędze x jest dodatni.\\ W związku z tym wykres wielomianu zaczyna się od lewej strony poniżej osi OX. A więc $$x \in [1,10] \cup [17,\infty).$$
\rozwStop
\odpStart
$x \in [1,10] \cup [17,\infty)$
\odpStop
\testStart
A.$x \in [1,10] \cup [17,\infty)$\\
B.$x \in (1,10) \cup [17,\infty)$\\
C.$x \in (1,10] \cup [17,\infty)$\\
D.$x \in [1,10) \cup [17,\infty)$\\
E.$x \in [1,10] \cup (17,\infty)$\\
F.$x \in (1,10) \cup (17,\infty)$\\
G.$x \in [1,10) \cup (17,\infty)$\\
H.$x \in (1,10] \cup (17,\infty)$
\testStop
\kluczStart
A
\kluczStop



\zadStart{Zadanie z Wikieł Z 1.62 a) moja wersja nr 124}

Rozwiązać nierówności $(x-1)(x-10)(x-18)\ge0$.
\zadStop
\rozwStart{Patryk Wirkus}{}
Miejsca zerowe naszego wielomianu to: $1, 10, 18$.\\
Wielomian jest stopnia nieparzystego, ponadto znak współczynnika przy\linebreak najwyższej potędze x jest dodatni.\\ W związku z tym wykres wielomianu zaczyna się od lewej strony poniżej osi OX. A więc $$x \in [1,10] \cup [18,\infty).$$
\rozwStop
\odpStart
$x \in [1,10] \cup [18,\infty)$
\odpStop
\testStart
A.$x \in [1,10] \cup [18,\infty)$\\
B.$x \in (1,10) \cup [18,\infty)$\\
C.$x \in (1,10] \cup [18,\infty)$\\
D.$x \in [1,10) \cup [18,\infty)$\\
E.$x \in [1,10] \cup (18,\infty)$\\
F.$x \in (1,10) \cup (18,\infty)$\\
G.$x \in [1,10) \cup (18,\infty)$\\
H.$x \in (1,10] \cup (18,\infty)$
\testStop
\kluczStart
A
\kluczStop



\zadStart{Zadanie z Wikieł Z 1.62 a) moja wersja nr 125}

Rozwiązać nierówności $(x-1)(x-10)(x-19)\ge0$.
\zadStop
\rozwStart{Patryk Wirkus}{}
Miejsca zerowe naszego wielomianu to: $1, 10, 19$.\\
Wielomian jest stopnia nieparzystego, ponadto znak współczynnika przy\linebreak najwyższej potędze x jest dodatni.\\ W związku z tym wykres wielomianu zaczyna się od lewej strony poniżej osi OX. A więc $$x \in [1,10] \cup [19,\infty).$$
\rozwStop
\odpStart
$x \in [1,10] \cup [19,\infty)$
\odpStop
\testStart
A.$x \in [1,10] \cup [19,\infty)$\\
B.$x \in (1,10) \cup [19,\infty)$\\
C.$x \in (1,10] \cup [19,\infty)$\\
D.$x \in [1,10) \cup [19,\infty)$\\
E.$x \in [1,10] \cup (19,\infty)$\\
F.$x \in (1,10) \cup (19,\infty)$\\
G.$x \in [1,10) \cup (19,\infty)$\\
H.$x \in (1,10] \cup (19,\infty)$
\testStop
\kluczStart
A
\kluczStop



\zadStart{Zadanie z Wikieł Z 1.62 a) moja wersja nr 126}

Rozwiązać nierówności $(x-1)(x-10)(x-20)\ge0$.
\zadStop
\rozwStart{Patryk Wirkus}{}
Miejsca zerowe naszego wielomianu to: $1, 10, 20$.\\
Wielomian jest stopnia nieparzystego, ponadto znak współczynnika przy\linebreak najwyższej potędze x jest dodatni.\\ W związku z tym wykres wielomianu zaczyna się od lewej strony poniżej osi OX. A więc $$x \in [1,10] \cup [20,\infty).$$
\rozwStop
\odpStart
$x \in [1,10] \cup [20,\infty)$
\odpStop
\testStart
A.$x \in [1,10] \cup [20,\infty)$\\
B.$x \in (1,10) \cup [20,\infty)$\\
C.$x \in (1,10] \cup [20,\infty)$\\
D.$x \in [1,10) \cup [20,\infty)$\\
E.$x \in [1,10] \cup (20,\infty)$\\
F.$x \in (1,10) \cup (20,\infty)$\\
G.$x \in [1,10) \cup (20,\infty)$\\
H.$x \in (1,10] \cup (20,\infty)$
\testStop
\kluczStart
A
\kluczStop



\zadStart{Zadanie z Wikieł Z 1.62 a) moja wersja nr 127}

Rozwiązać nierówności $(x-1)(x-11)(x-12)\ge0$.
\zadStop
\rozwStart{Patryk Wirkus}{}
Miejsca zerowe naszego wielomianu to: $1, 11, 12$.\\
Wielomian jest stopnia nieparzystego, ponadto znak współczynnika przy\linebreak najwyższej potędze x jest dodatni.\\ W związku z tym wykres wielomianu zaczyna się od lewej strony poniżej osi OX. A więc $$x \in [1,11] \cup [12,\infty).$$
\rozwStop
\odpStart
$x \in [1,11] \cup [12,\infty)$
\odpStop
\testStart
A.$x \in [1,11] \cup [12,\infty)$\\
B.$x \in (1,11) \cup [12,\infty)$\\
C.$x \in (1,11] \cup [12,\infty)$\\
D.$x \in [1,11) \cup [12,\infty)$\\
E.$x \in [1,11] \cup (12,\infty)$\\
F.$x \in (1,11) \cup (12,\infty)$\\
G.$x \in [1,11) \cup (12,\infty)$\\
H.$x \in (1,11] \cup (12,\infty)$
\testStop
\kluczStart
A
\kluczStop



\zadStart{Zadanie z Wikieł Z 1.62 a) moja wersja nr 128}

Rozwiązać nierówności $(x-1)(x-11)(x-13)\ge0$.
\zadStop
\rozwStart{Patryk Wirkus}{}
Miejsca zerowe naszego wielomianu to: $1, 11, 13$.\\
Wielomian jest stopnia nieparzystego, ponadto znak współczynnika przy\linebreak najwyższej potędze x jest dodatni.\\ W związku z tym wykres wielomianu zaczyna się od lewej strony poniżej osi OX. A więc $$x \in [1,11] \cup [13,\infty).$$
\rozwStop
\odpStart
$x \in [1,11] \cup [13,\infty)$
\odpStop
\testStart
A.$x \in [1,11] \cup [13,\infty)$\\
B.$x \in (1,11) \cup [13,\infty)$\\
C.$x \in (1,11] \cup [13,\infty)$\\
D.$x \in [1,11) \cup [13,\infty)$\\
E.$x \in [1,11] \cup (13,\infty)$\\
F.$x \in (1,11) \cup (13,\infty)$\\
G.$x \in [1,11) \cup (13,\infty)$\\
H.$x \in (1,11] \cup (13,\infty)$
\testStop
\kluczStart
A
\kluczStop



\zadStart{Zadanie z Wikieł Z 1.62 a) moja wersja nr 129}

Rozwiązać nierówności $(x-1)(x-11)(x-14)\ge0$.
\zadStop
\rozwStart{Patryk Wirkus}{}
Miejsca zerowe naszego wielomianu to: $1, 11, 14$.\\
Wielomian jest stopnia nieparzystego, ponadto znak współczynnika przy\linebreak najwyższej potędze x jest dodatni.\\ W związku z tym wykres wielomianu zaczyna się od lewej strony poniżej osi OX. A więc $$x \in [1,11] \cup [14,\infty).$$
\rozwStop
\odpStart
$x \in [1,11] \cup [14,\infty)$
\odpStop
\testStart
A.$x \in [1,11] \cup [14,\infty)$\\
B.$x \in (1,11) \cup [14,\infty)$\\
C.$x \in (1,11] \cup [14,\infty)$\\
D.$x \in [1,11) \cup [14,\infty)$\\
E.$x \in [1,11] \cup (14,\infty)$\\
F.$x \in (1,11) \cup (14,\infty)$\\
G.$x \in [1,11) \cup (14,\infty)$\\
H.$x \in (1,11] \cup (14,\infty)$
\testStop
\kluczStart
A
\kluczStop



\zadStart{Zadanie z Wikieł Z 1.62 a) moja wersja nr 130}

Rozwiązać nierówności $(x-1)(x-11)(x-15)\ge0$.
\zadStop
\rozwStart{Patryk Wirkus}{}
Miejsca zerowe naszego wielomianu to: $1, 11, 15$.\\
Wielomian jest stopnia nieparzystego, ponadto znak współczynnika przy\linebreak najwyższej potędze x jest dodatni.\\ W związku z tym wykres wielomianu zaczyna się od lewej strony poniżej osi OX. A więc $$x \in [1,11] \cup [15,\infty).$$
\rozwStop
\odpStart
$x \in [1,11] \cup [15,\infty)$
\odpStop
\testStart
A.$x \in [1,11] \cup [15,\infty)$\\
B.$x \in (1,11) \cup [15,\infty)$\\
C.$x \in (1,11] \cup [15,\infty)$\\
D.$x \in [1,11) \cup [15,\infty)$\\
E.$x \in [1,11] \cup (15,\infty)$\\
F.$x \in (1,11) \cup (15,\infty)$\\
G.$x \in [1,11) \cup (15,\infty)$\\
H.$x \in (1,11] \cup (15,\infty)$
\testStop
\kluczStart
A
\kluczStop



\zadStart{Zadanie z Wikieł Z 1.62 a) moja wersja nr 131}

Rozwiązać nierówności $(x-1)(x-11)(x-16)\ge0$.
\zadStop
\rozwStart{Patryk Wirkus}{}
Miejsca zerowe naszego wielomianu to: $1, 11, 16$.\\
Wielomian jest stopnia nieparzystego, ponadto znak współczynnika przy\linebreak najwyższej potędze x jest dodatni.\\ W związku z tym wykres wielomianu zaczyna się od lewej strony poniżej osi OX. A więc $$x \in [1,11] \cup [16,\infty).$$
\rozwStop
\odpStart
$x \in [1,11] \cup [16,\infty)$
\odpStop
\testStart
A.$x \in [1,11] \cup [16,\infty)$\\
B.$x \in (1,11) \cup [16,\infty)$\\
C.$x \in (1,11] \cup [16,\infty)$\\
D.$x \in [1,11) \cup [16,\infty)$\\
E.$x \in [1,11] \cup (16,\infty)$\\
F.$x \in (1,11) \cup (16,\infty)$\\
G.$x \in [1,11) \cup (16,\infty)$\\
H.$x \in (1,11] \cup (16,\infty)$
\testStop
\kluczStart
A
\kluczStop



\zadStart{Zadanie z Wikieł Z 1.62 a) moja wersja nr 132}

Rozwiązać nierówności $(x-1)(x-11)(x-17)\ge0$.
\zadStop
\rozwStart{Patryk Wirkus}{}
Miejsca zerowe naszego wielomianu to: $1, 11, 17$.\\
Wielomian jest stopnia nieparzystego, ponadto znak współczynnika przy\linebreak najwyższej potędze x jest dodatni.\\ W związku z tym wykres wielomianu zaczyna się od lewej strony poniżej osi OX. A więc $$x \in [1,11] \cup [17,\infty).$$
\rozwStop
\odpStart
$x \in [1,11] \cup [17,\infty)$
\odpStop
\testStart
A.$x \in [1,11] \cup [17,\infty)$\\
B.$x \in (1,11) \cup [17,\infty)$\\
C.$x \in (1,11] \cup [17,\infty)$\\
D.$x \in [1,11) \cup [17,\infty)$\\
E.$x \in [1,11] \cup (17,\infty)$\\
F.$x \in (1,11) \cup (17,\infty)$\\
G.$x \in [1,11) \cup (17,\infty)$\\
H.$x \in (1,11] \cup (17,\infty)$
\testStop
\kluczStart
A
\kluczStop



\zadStart{Zadanie z Wikieł Z 1.62 a) moja wersja nr 133}

Rozwiązać nierówności $(x-1)(x-11)(x-18)\ge0$.
\zadStop
\rozwStart{Patryk Wirkus}{}
Miejsca zerowe naszego wielomianu to: $1, 11, 18$.\\
Wielomian jest stopnia nieparzystego, ponadto znak współczynnika przy\linebreak najwyższej potędze x jest dodatni.\\ W związku z tym wykres wielomianu zaczyna się od lewej strony poniżej osi OX. A więc $$x \in [1,11] \cup [18,\infty).$$
\rozwStop
\odpStart
$x \in [1,11] \cup [18,\infty)$
\odpStop
\testStart
A.$x \in [1,11] \cup [18,\infty)$\\
B.$x \in (1,11) \cup [18,\infty)$\\
C.$x \in (1,11] \cup [18,\infty)$\\
D.$x \in [1,11) \cup [18,\infty)$\\
E.$x \in [1,11] \cup (18,\infty)$\\
F.$x \in (1,11) \cup (18,\infty)$\\
G.$x \in [1,11) \cup (18,\infty)$\\
H.$x \in (1,11] \cup (18,\infty)$
\testStop
\kluczStart
A
\kluczStop



\zadStart{Zadanie z Wikieł Z 1.62 a) moja wersja nr 134}

Rozwiązać nierówności $(x-1)(x-11)(x-19)\ge0$.
\zadStop
\rozwStart{Patryk Wirkus}{}
Miejsca zerowe naszego wielomianu to: $1, 11, 19$.\\
Wielomian jest stopnia nieparzystego, ponadto znak współczynnika przy\linebreak najwyższej potędze x jest dodatni.\\ W związku z tym wykres wielomianu zaczyna się od lewej strony poniżej osi OX. A więc $$x \in [1,11] \cup [19,\infty).$$
\rozwStop
\odpStart
$x \in [1,11] \cup [19,\infty)$
\odpStop
\testStart
A.$x \in [1,11] \cup [19,\infty)$\\
B.$x \in (1,11) \cup [19,\infty)$\\
C.$x \in (1,11] \cup [19,\infty)$\\
D.$x \in [1,11) \cup [19,\infty)$\\
E.$x \in [1,11] \cup (19,\infty)$\\
F.$x \in (1,11) \cup (19,\infty)$\\
G.$x \in [1,11) \cup (19,\infty)$\\
H.$x \in (1,11] \cup (19,\infty)$
\testStop
\kluczStart
A
\kluczStop



\zadStart{Zadanie z Wikieł Z 1.62 a) moja wersja nr 135}

Rozwiązać nierówności $(x-1)(x-11)(x-20)\ge0$.
\zadStop
\rozwStart{Patryk Wirkus}{}
Miejsca zerowe naszego wielomianu to: $1, 11, 20$.\\
Wielomian jest stopnia nieparzystego, ponadto znak współczynnika przy\linebreak najwyższej potędze x jest dodatni.\\ W związku z tym wykres wielomianu zaczyna się od lewej strony poniżej osi OX. A więc $$x \in [1,11] \cup [20,\infty).$$
\rozwStop
\odpStart
$x \in [1,11] \cup [20,\infty)$
\odpStop
\testStart
A.$x \in [1,11] \cup [20,\infty)$\\
B.$x \in (1,11) \cup [20,\infty)$\\
C.$x \in (1,11] \cup [20,\infty)$\\
D.$x \in [1,11) \cup [20,\infty)$\\
E.$x \in [1,11] \cup (20,\infty)$\\
F.$x \in (1,11) \cup (20,\infty)$\\
G.$x \in [1,11) \cup (20,\infty)$\\
H.$x \in (1,11] \cup (20,\infty)$
\testStop
\kluczStart
A
\kluczStop



\zadStart{Zadanie z Wikieł Z 1.62 a) moja wersja nr 136}

Rozwiązać nierówności $(x-1)(x-12)(x-13)\ge0$.
\zadStop
\rozwStart{Patryk Wirkus}{}
Miejsca zerowe naszego wielomianu to: $1, 12, 13$.\\
Wielomian jest stopnia nieparzystego, ponadto znak współczynnika przy\linebreak najwyższej potędze x jest dodatni.\\ W związku z tym wykres wielomianu zaczyna się od lewej strony poniżej osi OX. A więc $$x \in [1,12] \cup [13,\infty).$$
\rozwStop
\odpStart
$x \in [1,12] \cup [13,\infty)$
\odpStop
\testStart
A.$x \in [1,12] \cup [13,\infty)$\\
B.$x \in (1,12) \cup [13,\infty)$\\
C.$x \in (1,12] \cup [13,\infty)$\\
D.$x \in [1,12) \cup [13,\infty)$\\
E.$x \in [1,12] \cup (13,\infty)$\\
F.$x \in (1,12) \cup (13,\infty)$\\
G.$x \in [1,12) \cup (13,\infty)$\\
H.$x \in (1,12] \cup (13,\infty)$
\testStop
\kluczStart
A
\kluczStop



\zadStart{Zadanie z Wikieł Z 1.62 a) moja wersja nr 137}

Rozwiązać nierówności $(x-1)(x-12)(x-14)\ge0$.
\zadStop
\rozwStart{Patryk Wirkus}{}
Miejsca zerowe naszego wielomianu to: $1, 12, 14$.\\
Wielomian jest stopnia nieparzystego, ponadto znak współczynnika przy\linebreak najwyższej potędze x jest dodatni.\\ W związku z tym wykres wielomianu zaczyna się od lewej strony poniżej osi OX. A więc $$x \in [1,12] \cup [14,\infty).$$
\rozwStop
\odpStart
$x \in [1,12] \cup [14,\infty)$
\odpStop
\testStart
A.$x \in [1,12] \cup [14,\infty)$\\
B.$x \in (1,12) \cup [14,\infty)$\\
C.$x \in (1,12] \cup [14,\infty)$\\
D.$x \in [1,12) \cup [14,\infty)$\\
E.$x \in [1,12] \cup (14,\infty)$\\
F.$x \in (1,12) \cup (14,\infty)$\\
G.$x \in [1,12) \cup (14,\infty)$\\
H.$x \in (1,12] \cup (14,\infty)$
\testStop
\kluczStart
A
\kluczStop



\zadStart{Zadanie z Wikieł Z 1.62 a) moja wersja nr 138}

Rozwiązać nierówności $(x-1)(x-12)(x-15)\ge0$.
\zadStop
\rozwStart{Patryk Wirkus}{}
Miejsca zerowe naszego wielomianu to: $1, 12, 15$.\\
Wielomian jest stopnia nieparzystego, ponadto znak współczynnika przy\linebreak najwyższej potędze x jest dodatni.\\ W związku z tym wykres wielomianu zaczyna się od lewej strony poniżej osi OX. A więc $$x \in [1,12] \cup [15,\infty).$$
\rozwStop
\odpStart
$x \in [1,12] \cup [15,\infty)$
\odpStop
\testStart
A.$x \in [1,12] \cup [15,\infty)$\\
B.$x \in (1,12) \cup [15,\infty)$\\
C.$x \in (1,12] \cup [15,\infty)$\\
D.$x \in [1,12) \cup [15,\infty)$\\
E.$x \in [1,12] \cup (15,\infty)$\\
F.$x \in (1,12) \cup (15,\infty)$\\
G.$x \in [1,12) \cup (15,\infty)$\\
H.$x \in (1,12] \cup (15,\infty)$
\testStop
\kluczStart
A
\kluczStop



\zadStart{Zadanie z Wikieł Z 1.62 a) moja wersja nr 139}

Rozwiązać nierówności $(x-1)(x-12)(x-16)\ge0$.
\zadStop
\rozwStart{Patryk Wirkus}{}
Miejsca zerowe naszego wielomianu to: $1, 12, 16$.\\
Wielomian jest stopnia nieparzystego, ponadto znak współczynnika przy\linebreak najwyższej potędze x jest dodatni.\\ W związku z tym wykres wielomianu zaczyna się od lewej strony poniżej osi OX. A więc $$x \in [1,12] \cup [16,\infty).$$
\rozwStop
\odpStart
$x \in [1,12] \cup [16,\infty)$
\odpStop
\testStart
A.$x \in [1,12] \cup [16,\infty)$\\
B.$x \in (1,12) \cup [16,\infty)$\\
C.$x \in (1,12] \cup [16,\infty)$\\
D.$x \in [1,12) \cup [16,\infty)$\\
E.$x \in [1,12] \cup (16,\infty)$\\
F.$x \in (1,12) \cup (16,\infty)$\\
G.$x \in [1,12) \cup (16,\infty)$\\
H.$x \in (1,12] \cup (16,\infty)$
\testStop
\kluczStart
A
\kluczStop



\zadStart{Zadanie z Wikieł Z 1.62 a) moja wersja nr 140}

Rozwiązać nierówności $(x-1)(x-12)(x-17)\ge0$.
\zadStop
\rozwStart{Patryk Wirkus}{}
Miejsca zerowe naszego wielomianu to: $1, 12, 17$.\\
Wielomian jest stopnia nieparzystego, ponadto znak współczynnika przy\linebreak najwyższej potędze x jest dodatni.\\ W związku z tym wykres wielomianu zaczyna się od lewej strony poniżej osi OX. A więc $$x \in [1,12] \cup [17,\infty).$$
\rozwStop
\odpStart
$x \in [1,12] \cup [17,\infty)$
\odpStop
\testStart
A.$x \in [1,12] \cup [17,\infty)$\\
B.$x \in (1,12) \cup [17,\infty)$\\
C.$x \in (1,12] \cup [17,\infty)$\\
D.$x \in [1,12) \cup [17,\infty)$\\
E.$x \in [1,12] \cup (17,\infty)$\\
F.$x \in (1,12) \cup (17,\infty)$\\
G.$x \in [1,12) \cup (17,\infty)$\\
H.$x \in (1,12] \cup (17,\infty)$
\testStop
\kluczStart
A
\kluczStop



\zadStart{Zadanie z Wikieł Z 1.62 a) moja wersja nr 141}

Rozwiązać nierówności $(x-1)(x-12)(x-18)\ge0$.
\zadStop
\rozwStart{Patryk Wirkus}{}
Miejsca zerowe naszego wielomianu to: $1, 12, 18$.\\
Wielomian jest stopnia nieparzystego, ponadto znak współczynnika przy\linebreak najwyższej potędze x jest dodatni.\\ W związku z tym wykres wielomianu zaczyna się od lewej strony poniżej osi OX. A więc $$x \in [1,12] \cup [18,\infty).$$
\rozwStop
\odpStart
$x \in [1,12] \cup [18,\infty)$
\odpStop
\testStart
A.$x \in [1,12] \cup [18,\infty)$\\
B.$x \in (1,12) \cup [18,\infty)$\\
C.$x \in (1,12] \cup [18,\infty)$\\
D.$x \in [1,12) \cup [18,\infty)$\\
E.$x \in [1,12] \cup (18,\infty)$\\
F.$x \in (1,12) \cup (18,\infty)$\\
G.$x \in [1,12) \cup (18,\infty)$\\
H.$x \in (1,12] \cup (18,\infty)$
\testStop
\kluczStart
A
\kluczStop



\zadStart{Zadanie z Wikieł Z 1.62 a) moja wersja nr 142}

Rozwiązać nierówności $(x-1)(x-12)(x-19)\ge0$.
\zadStop
\rozwStart{Patryk Wirkus}{}
Miejsca zerowe naszego wielomianu to: $1, 12, 19$.\\
Wielomian jest stopnia nieparzystego, ponadto znak współczynnika przy\linebreak najwyższej potędze x jest dodatni.\\ W związku z tym wykres wielomianu zaczyna się od lewej strony poniżej osi OX. A więc $$x \in [1,12] \cup [19,\infty).$$
\rozwStop
\odpStart
$x \in [1,12] \cup [19,\infty)$
\odpStop
\testStart
A.$x \in [1,12] \cup [19,\infty)$\\
B.$x \in (1,12) \cup [19,\infty)$\\
C.$x \in (1,12] \cup [19,\infty)$\\
D.$x \in [1,12) \cup [19,\infty)$\\
E.$x \in [1,12] \cup (19,\infty)$\\
F.$x \in (1,12) \cup (19,\infty)$\\
G.$x \in [1,12) \cup (19,\infty)$\\
H.$x \in (1,12] \cup (19,\infty)$
\testStop
\kluczStart
A
\kluczStop



\zadStart{Zadanie z Wikieł Z 1.62 a) moja wersja nr 143}

Rozwiązać nierówności $(x-1)(x-12)(x-20)\ge0$.
\zadStop
\rozwStart{Patryk Wirkus}{}
Miejsca zerowe naszego wielomianu to: $1, 12, 20$.\\
Wielomian jest stopnia nieparzystego, ponadto znak współczynnika przy\linebreak najwyższej potędze x jest dodatni.\\ W związku z tym wykres wielomianu zaczyna się od lewej strony poniżej osi OX. A więc $$x \in [1,12] \cup [20,\infty).$$
\rozwStop
\odpStart
$x \in [1,12] \cup [20,\infty)$
\odpStop
\testStart
A.$x \in [1,12] \cup [20,\infty)$\\
B.$x \in (1,12) \cup [20,\infty)$\\
C.$x \in (1,12] \cup [20,\infty)$\\
D.$x \in [1,12) \cup [20,\infty)$\\
E.$x \in [1,12] \cup (20,\infty)$\\
F.$x \in (1,12) \cup (20,\infty)$\\
G.$x \in [1,12) \cup (20,\infty)$\\
H.$x \in (1,12] \cup (20,\infty)$
\testStop
\kluczStart
A
\kluczStop



\zadStart{Zadanie z Wikieł Z 1.62 a) moja wersja nr 144}

Rozwiązać nierówności $(x-1)(x-13)(x-14)\ge0$.
\zadStop
\rozwStart{Patryk Wirkus}{}
Miejsca zerowe naszego wielomianu to: $1, 13, 14$.\\
Wielomian jest stopnia nieparzystego, ponadto znak współczynnika przy\linebreak najwyższej potędze x jest dodatni.\\ W związku z tym wykres wielomianu zaczyna się od lewej strony poniżej osi OX. A więc $$x \in [1,13] \cup [14,\infty).$$
\rozwStop
\odpStart
$x \in [1,13] \cup [14,\infty)$
\odpStop
\testStart
A.$x \in [1,13] \cup [14,\infty)$\\
B.$x \in (1,13) \cup [14,\infty)$\\
C.$x \in (1,13] \cup [14,\infty)$\\
D.$x \in [1,13) \cup [14,\infty)$\\
E.$x \in [1,13] \cup (14,\infty)$\\
F.$x \in (1,13) \cup (14,\infty)$\\
G.$x \in [1,13) \cup (14,\infty)$\\
H.$x \in (1,13] \cup (14,\infty)$
\testStop
\kluczStart
A
\kluczStop



\zadStart{Zadanie z Wikieł Z 1.62 a) moja wersja nr 145}

Rozwiązać nierówności $(x-1)(x-13)(x-15)\ge0$.
\zadStop
\rozwStart{Patryk Wirkus}{}
Miejsca zerowe naszego wielomianu to: $1, 13, 15$.\\
Wielomian jest stopnia nieparzystego, ponadto znak współczynnika przy\linebreak najwyższej potędze x jest dodatni.\\ W związku z tym wykres wielomianu zaczyna się od lewej strony poniżej osi OX. A więc $$x \in [1,13] \cup [15,\infty).$$
\rozwStop
\odpStart
$x \in [1,13] \cup [15,\infty)$
\odpStop
\testStart
A.$x \in [1,13] \cup [15,\infty)$\\
B.$x \in (1,13) \cup [15,\infty)$\\
C.$x \in (1,13] \cup [15,\infty)$\\
D.$x \in [1,13) \cup [15,\infty)$\\
E.$x \in [1,13] \cup (15,\infty)$\\
F.$x \in (1,13) \cup (15,\infty)$\\
G.$x \in [1,13) \cup (15,\infty)$\\
H.$x \in (1,13] \cup (15,\infty)$
\testStop
\kluczStart
A
\kluczStop



\zadStart{Zadanie z Wikieł Z 1.62 a) moja wersja nr 146}

Rozwiązać nierówności $(x-1)(x-13)(x-16)\ge0$.
\zadStop
\rozwStart{Patryk Wirkus}{}
Miejsca zerowe naszego wielomianu to: $1, 13, 16$.\\
Wielomian jest stopnia nieparzystego, ponadto znak współczynnika przy\linebreak najwyższej potędze x jest dodatni.\\ W związku z tym wykres wielomianu zaczyna się od lewej strony poniżej osi OX. A więc $$x \in [1,13] \cup [16,\infty).$$
\rozwStop
\odpStart
$x \in [1,13] \cup [16,\infty)$
\odpStop
\testStart
A.$x \in [1,13] \cup [16,\infty)$\\
B.$x \in (1,13) \cup [16,\infty)$\\
C.$x \in (1,13] \cup [16,\infty)$\\
D.$x \in [1,13) \cup [16,\infty)$\\
E.$x \in [1,13] \cup (16,\infty)$\\
F.$x \in (1,13) \cup (16,\infty)$\\
G.$x \in [1,13) \cup (16,\infty)$\\
H.$x \in (1,13] \cup (16,\infty)$
\testStop
\kluczStart
A
\kluczStop



\zadStart{Zadanie z Wikieł Z 1.62 a) moja wersja nr 147}

Rozwiązać nierówności $(x-1)(x-13)(x-17)\ge0$.
\zadStop
\rozwStart{Patryk Wirkus}{}
Miejsca zerowe naszego wielomianu to: $1, 13, 17$.\\
Wielomian jest stopnia nieparzystego, ponadto znak współczynnika przy\linebreak najwyższej potędze x jest dodatni.\\ W związku z tym wykres wielomianu zaczyna się od lewej strony poniżej osi OX. A więc $$x \in [1,13] \cup [17,\infty).$$
\rozwStop
\odpStart
$x \in [1,13] \cup [17,\infty)$
\odpStop
\testStart
A.$x \in [1,13] \cup [17,\infty)$\\
B.$x \in (1,13) \cup [17,\infty)$\\
C.$x \in (1,13] \cup [17,\infty)$\\
D.$x \in [1,13) \cup [17,\infty)$\\
E.$x \in [1,13] \cup (17,\infty)$\\
F.$x \in (1,13) \cup (17,\infty)$\\
G.$x \in [1,13) \cup (17,\infty)$\\
H.$x \in (1,13] \cup (17,\infty)$
\testStop
\kluczStart
A
\kluczStop



\zadStart{Zadanie z Wikieł Z 1.62 a) moja wersja nr 148}

Rozwiązać nierówności $(x-1)(x-13)(x-18)\ge0$.
\zadStop
\rozwStart{Patryk Wirkus}{}
Miejsca zerowe naszego wielomianu to: $1, 13, 18$.\\
Wielomian jest stopnia nieparzystego, ponadto znak współczynnika przy\linebreak najwyższej potędze x jest dodatni.\\ W związku z tym wykres wielomianu zaczyna się od lewej strony poniżej osi OX. A więc $$x \in [1,13] \cup [18,\infty).$$
\rozwStop
\odpStart
$x \in [1,13] \cup [18,\infty)$
\odpStop
\testStart
A.$x \in [1,13] \cup [18,\infty)$\\
B.$x \in (1,13) \cup [18,\infty)$\\
C.$x \in (1,13] \cup [18,\infty)$\\
D.$x \in [1,13) \cup [18,\infty)$\\
E.$x \in [1,13] \cup (18,\infty)$\\
F.$x \in (1,13) \cup (18,\infty)$\\
G.$x \in [1,13) \cup (18,\infty)$\\
H.$x \in (1,13] \cup (18,\infty)$
\testStop
\kluczStart
A
\kluczStop



\zadStart{Zadanie z Wikieł Z 1.62 a) moja wersja nr 149}

Rozwiązać nierówności $(x-1)(x-13)(x-19)\ge0$.
\zadStop
\rozwStart{Patryk Wirkus}{}
Miejsca zerowe naszego wielomianu to: $1, 13, 19$.\\
Wielomian jest stopnia nieparzystego, ponadto znak współczynnika przy\linebreak najwyższej potędze x jest dodatni.\\ W związku z tym wykres wielomianu zaczyna się od lewej strony poniżej osi OX. A więc $$x \in [1,13] \cup [19,\infty).$$
\rozwStop
\odpStart
$x \in [1,13] \cup [19,\infty)$
\odpStop
\testStart
A.$x \in [1,13] \cup [19,\infty)$\\
B.$x \in (1,13) \cup [19,\infty)$\\
C.$x \in (1,13] \cup [19,\infty)$\\
D.$x \in [1,13) \cup [19,\infty)$\\
E.$x \in [1,13] \cup (19,\infty)$\\
F.$x \in (1,13) \cup (19,\infty)$\\
G.$x \in [1,13) \cup (19,\infty)$\\
H.$x \in (1,13] \cup (19,\infty)$
\testStop
\kluczStart
A
\kluczStop



\zadStart{Zadanie z Wikieł Z 1.62 a) moja wersja nr 150}

Rozwiązać nierówności $(x-1)(x-13)(x-20)\ge0$.
\zadStop
\rozwStart{Patryk Wirkus}{}
Miejsca zerowe naszego wielomianu to: $1, 13, 20$.\\
Wielomian jest stopnia nieparzystego, ponadto znak współczynnika przy\linebreak najwyższej potędze x jest dodatni.\\ W związku z tym wykres wielomianu zaczyna się od lewej strony poniżej osi OX. A więc $$x \in [1,13] \cup [20,\infty).$$
\rozwStop
\odpStart
$x \in [1,13] \cup [20,\infty)$
\odpStop
\testStart
A.$x \in [1,13] \cup [20,\infty)$\\
B.$x \in (1,13) \cup [20,\infty)$\\
C.$x \in (1,13] \cup [20,\infty)$\\
D.$x \in [1,13) \cup [20,\infty)$\\
E.$x \in [1,13] \cup (20,\infty)$\\
F.$x \in (1,13) \cup (20,\infty)$\\
G.$x \in [1,13) \cup (20,\infty)$\\
H.$x \in (1,13] \cup (20,\infty)$
\testStop
\kluczStart
A
\kluczStop



\zadStart{Zadanie z Wikieł Z 1.62 a) moja wersja nr 151}

Rozwiązać nierówności $(x-1)(x-14)(x-15)\ge0$.
\zadStop
\rozwStart{Patryk Wirkus}{}
Miejsca zerowe naszego wielomianu to: $1, 14, 15$.\\
Wielomian jest stopnia nieparzystego, ponadto znak współczynnika przy\linebreak najwyższej potędze x jest dodatni.\\ W związku z tym wykres wielomianu zaczyna się od lewej strony poniżej osi OX. A więc $$x \in [1,14] \cup [15,\infty).$$
\rozwStop
\odpStart
$x \in [1,14] \cup [15,\infty)$
\odpStop
\testStart
A.$x \in [1,14] \cup [15,\infty)$\\
B.$x \in (1,14) \cup [15,\infty)$\\
C.$x \in (1,14] \cup [15,\infty)$\\
D.$x \in [1,14) \cup [15,\infty)$\\
E.$x \in [1,14] \cup (15,\infty)$\\
F.$x \in (1,14) \cup (15,\infty)$\\
G.$x \in [1,14) \cup (15,\infty)$\\
H.$x \in (1,14] \cup (15,\infty)$
\testStop
\kluczStart
A
\kluczStop



\zadStart{Zadanie z Wikieł Z 1.62 a) moja wersja nr 152}

Rozwiązać nierówności $(x-1)(x-14)(x-16)\ge0$.
\zadStop
\rozwStart{Patryk Wirkus}{}
Miejsca zerowe naszego wielomianu to: $1, 14, 16$.\\
Wielomian jest stopnia nieparzystego, ponadto znak współczynnika przy\linebreak najwyższej potędze x jest dodatni.\\ W związku z tym wykres wielomianu zaczyna się od lewej strony poniżej osi OX. A więc $$x \in [1,14] \cup [16,\infty).$$
\rozwStop
\odpStart
$x \in [1,14] \cup [16,\infty)$
\odpStop
\testStart
A.$x \in [1,14] \cup [16,\infty)$\\
B.$x \in (1,14) \cup [16,\infty)$\\
C.$x \in (1,14] \cup [16,\infty)$\\
D.$x \in [1,14) \cup [16,\infty)$\\
E.$x \in [1,14] \cup (16,\infty)$\\
F.$x \in (1,14) \cup (16,\infty)$\\
G.$x \in [1,14) \cup (16,\infty)$\\
H.$x \in (1,14] \cup (16,\infty)$
\testStop
\kluczStart
A
\kluczStop



\zadStart{Zadanie z Wikieł Z 1.62 a) moja wersja nr 153}

Rozwiązać nierówności $(x-1)(x-14)(x-17)\ge0$.
\zadStop
\rozwStart{Patryk Wirkus}{}
Miejsca zerowe naszego wielomianu to: $1, 14, 17$.\\
Wielomian jest stopnia nieparzystego, ponadto znak współczynnika przy\linebreak najwyższej potędze x jest dodatni.\\ W związku z tym wykres wielomianu zaczyna się od lewej strony poniżej osi OX. A więc $$x \in [1,14] \cup [17,\infty).$$
\rozwStop
\odpStart
$x \in [1,14] \cup [17,\infty)$
\odpStop
\testStart
A.$x \in [1,14] \cup [17,\infty)$\\
B.$x \in (1,14) \cup [17,\infty)$\\
C.$x \in (1,14] \cup [17,\infty)$\\
D.$x \in [1,14) \cup [17,\infty)$\\
E.$x \in [1,14] \cup (17,\infty)$\\
F.$x \in (1,14) \cup (17,\infty)$\\
G.$x \in [1,14) \cup (17,\infty)$\\
H.$x \in (1,14] \cup (17,\infty)$
\testStop
\kluczStart
A
\kluczStop



\zadStart{Zadanie z Wikieł Z 1.62 a) moja wersja nr 154}

Rozwiązać nierówności $(x-1)(x-14)(x-18)\ge0$.
\zadStop
\rozwStart{Patryk Wirkus}{}
Miejsca zerowe naszego wielomianu to: $1, 14, 18$.\\
Wielomian jest stopnia nieparzystego, ponadto znak współczynnika przy\linebreak najwyższej potędze x jest dodatni.\\ W związku z tym wykres wielomianu zaczyna się od lewej strony poniżej osi OX. A więc $$x \in [1,14] \cup [18,\infty).$$
\rozwStop
\odpStart
$x \in [1,14] \cup [18,\infty)$
\odpStop
\testStart
A.$x \in [1,14] \cup [18,\infty)$\\
B.$x \in (1,14) \cup [18,\infty)$\\
C.$x \in (1,14] \cup [18,\infty)$\\
D.$x \in [1,14) \cup [18,\infty)$\\
E.$x \in [1,14] \cup (18,\infty)$\\
F.$x \in (1,14) \cup (18,\infty)$\\
G.$x \in [1,14) \cup (18,\infty)$\\
H.$x \in (1,14] \cup (18,\infty)$
\testStop
\kluczStart
A
\kluczStop



\zadStart{Zadanie z Wikieł Z 1.62 a) moja wersja nr 155}

Rozwiązać nierówności $(x-1)(x-14)(x-19)\ge0$.
\zadStop
\rozwStart{Patryk Wirkus}{}
Miejsca zerowe naszego wielomianu to: $1, 14, 19$.\\
Wielomian jest stopnia nieparzystego, ponadto znak współczynnika przy\linebreak najwyższej potędze x jest dodatni.\\ W związku z tym wykres wielomianu zaczyna się od lewej strony poniżej osi OX. A więc $$x \in [1,14] \cup [19,\infty).$$
\rozwStop
\odpStart
$x \in [1,14] \cup [19,\infty)$
\odpStop
\testStart
A.$x \in [1,14] \cup [19,\infty)$\\
B.$x \in (1,14) \cup [19,\infty)$\\
C.$x \in (1,14] \cup [19,\infty)$\\
D.$x \in [1,14) \cup [19,\infty)$\\
E.$x \in [1,14] \cup (19,\infty)$\\
F.$x \in (1,14) \cup (19,\infty)$\\
G.$x \in [1,14) \cup (19,\infty)$\\
H.$x \in (1,14] \cup (19,\infty)$
\testStop
\kluczStart
A
\kluczStop



\zadStart{Zadanie z Wikieł Z 1.62 a) moja wersja nr 156}

Rozwiązać nierówności $(x-1)(x-14)(x-20)\ge0$.
\zadStop
\rozwStart{Patryk Wirkus}{}
Miejsca zerowe naszego wielomianu to: $1, 14, 20$.\\
Wielomian jest stopnia nieparzystego, ponadto znak współczynnika przy\linebreak najwyższej potędze x jest dodatni.\\ W związku z tym wykres wielomianu zaczyna się od lewej strony poniżej osi OX. A więc $$x \in [1,14] \cup [20,\infty).$$
\rozwStop
\odpStart
$x \in [1,14] \cup [20,\infty)$
\odpStop
\testStart
A.$x \in [1,14] \cup [20,\infty)$\\
B.$x \in (1,14) \cup [20,\infty)$\\
C.$x \in (1,14] \cup [20,\infty)$\\
D.$x \in [1,14) \cup [20,\infty)$\\
E.$x \in [1,14] \cup (20,\infty)$\\
F.$x \in (1,14) \cup (20,\infty)$\\
G.$x \in [1,14) \cup (20,\infty)$\\
H.$x \in (1,14] \cup (20,\infty)$
\testStop
\kluczStart
A
\kluczStop



\zadStart{Zadanie z Wikieł Z 1.62 a) moja wersja nr 157}

Rozwiązać nierówności $(x-1)(x-15)(x-16)\ge0$.
\zadStop
\rozwStart{Patryk Wirkus}{}
Miejsca zerowe naszego wielomianu to: $1, 15, 16$.\\
Wielomian jest stopnia nieparzystego, ponadto znak współczynnika przy\linebreak najwyższej potędze x jest dodatni.\\ W związku z tym wykres wielomianu zaczyna się od lewej strony poniżej osi OX. A więc $$x \in [1,15] \cup [16,\infty).$$
\rozwStop
\odpStart
$x \in [1,15] \cup [16,\infty)$
\odpStop
\testStart
A.$x \in [1,15] \cup [16,\infty)$\\
B.$x \in (1,15) \cup [16,\infty)$\\
C.$x \in (1,15] \cup [16,\infty)$\\
D.$x \in [1,15) \cup [16,\infty)$\\
E.$x \in [1,15] \cup (16,\infty)$\\
F.$x \in (1,15) \cup (16,\infty)$\\
G.$x \in [1,15) \cup (16,\infty)$\\
H.$x \in (1,15] \cup (16,\infty)$
\testStop
\kluczStart
A
\kluczStop



\zadStart{Zadanie z Wikieł Z 1.62 a) moja wersja nr 158}

Rozwiązać nierówności $(x-1)(x-15)(x-17)\ge0$.
\zadStop
\rozwStart{Patryk Wirkus}{}
Miejsca zerowe naszego wielomianu to: $1, 15, 17$.\\
Wielomian jest stopnia nieparzystego, ponadto znak współczynnika przy\linebreak najwyższej potędze x jest dodatni.\\ W związku z tym wykres wielomianu zaczyna się od lewej strony poniżej osi OX. A więc $$x \in [1,15] \cup [17,\infty).$$
\rozwStop
\odpStart
$x \in [1,15] \cup [17,\infty)$
\odpStop
\testStart
A.$x \in [1,15] \cup [17,\infty)$\\
B.$x \in (1,15) \cup [17,\infty)$\\
C.$x \in (1,15] \cup [17,\infty)$\\
D.$x \in [1,15) \cup [17,\infty)$\\
E.$x \in [1,15] \cup (17,\infty)$\\
F.$x \in (1,15) \cup (17,\infty)$\\
G.$x \in [1,15) \cup (17,\infty)$\\
H.$x \in (1,15] \cup (17,\infty)$
\testStop
\kluczStart
A
\kluczStop



\zadStart{Zadanie z Wikieł Z 1.62 a) moja wersja nr 159}

Rozwiązać nierówności $(x-1)(x-15)(x-18)\ge0$.
\zadStop
\rozwStart{Patryk Wirkus}{}
Miejsca zerowe naszego wielomianu to: $1, 15, 18$.\\
Wielomian jest stopnia nieparzystego, ponadto znak współczynnika przy\linebreak najwyższej potędze x jest dodatni.\\ W związku z tym wykres wielomianu zaczyna się od lewej strony poniżej osi OX. A więc $$x \in [1,15] \cup [18,\infty).$$
\rozwStop
\odpStart
$x \in [1,15] \cup [18,\infty)$
\odpStop
\testStart
A.$x \in [1,15] \cup [18,\infty)$\\
B.$x \in (1,15) \cup [18,\infty)$\\
C.$x \in (1,15] \cup [18,\infty)$\\
D.$x \in [1,15) \cup [18,\infty)$\\
E.$x \in [1,15] \cup (18,\infty)$\\
F.$x \in (1,15) \cup (18,\infty)$\\
G.$x \in [1,15) \cup (18,\infty)$\\
H.$x \in (1,15] \cup (18,\infty)$
\testStop
\kluczStart
A
\kluczStop



\zadStart{Zadanie z Wikieł Z 1.62 a) moja wersja nr 160}

Rozwiązać nierówności $(x-1)(x-15)(x-19)\ge0$.
\zadStop
\rozwStart{Patryk Wirkus}{}
Miejsca zerowe naszego wielomianu to: $1, 15, 19$.\\
Wielomian jest stopnia nieparzystego, ponadto znak współczynnika przy\linebreak najwyższej potędze x jest dodatni.\\ W związku z tym wykres wielomianu zaczyna się od lewej strony poniżej osi OX. A więc $$x \in [1,15] \cup [19,\infty).$$
\rozwStop
\odpStart
$x \in [1,15] \cup [19,\infty)$
\odpStop
\testStart
A.$x \in [1,15] \cup [19,\infty)$\\
B.$x \in (1,15) \cup [19,\infty)$\\
C.$x \in (1,15] \cup [19,\infty)$\\
D.$x \in [1,15) \cup [19,\infty)$\\
E.$x \in [1,15] \cup (19,\infty)$\\
F.$x \in (1,15) \cup (19,\infty)$\\
G.$x \in [1,15) \cup (19,\infty)$\\
H.$x \in (1,15] \cup (19,\infty)$
\testStop
\kluczStart
A
\kluczStop



\zadStart{Zadanie z Wikieł Z 1.62 a) moja wersja nr 161}

Rozwiązać nierówności $(x-1)(x-15)(x-20)\ge0$.
\zadStop
\rozwStart{Patryk Wirkus}{}
Miejsca zerowe naszego wielomianu to: $1, 15, 20$.\\
Wielomian jest stopnia nieparzystego, ponadto znak współczynnika przy\linebreak najwyższej potędze x jest dodatni.\\ W związku z tym wykres wielomianu zaczyna się od lewej strony poniżej osi OX. A więc $$x \in [1,15] \cup [20,\infty).$$
\rozwStop
\odpStart
$x \in [1,15] \cup [20,\infty)$
\odpStop
\testStart
A.$x \in [1,15] \cup [20,\infty)$\\
B.$x \in (1,15) \cup [20,\infty)$\\
C.$x \in (1,15] \cup [20,\infty)$\\
D.$x \in [1,15) \cup [20,\infty)$\\
E.$x \in [1,15] \cup (20,\infty)$\\
F.$x \in (1,15) \cup (20,\infty)$\\
G.$x \in [1,15) \cup (20,\infty)$\\
H.$x \in (1,15] \cup (20,\infty)$
\testStop
\kluczStart
A
\kluczStop



\zadStart{Zadanie z Wikieł Z 1.62 a) moja wersja nr 162}

Rozwiązać nierówności $(x-1)(x-16)(x-17)\ge0$.
\zadStop
\rozwStart{Patryk Wirkus}{}
Miejsca zerowe naszego wielomianu to: $1, 16, 17$.\\
Wielomian jest stopnia nieparzystego, ponadto znak współczynnika przy\linebreak najwyższej potędze x jest dodatni.\\ W związku z tym wykres wielomianu zaczyna się od lewej strony poniżej osi OX. A więc $$x \in [1,16] \cup [17,\infty).$$
\rozwStop
\odpStart
$x \in [1,16] \cup [17,\infty)$
\odpStop
\testStart
A.$x \in [1,16] \cup [17,\infty)$\\
B.$x \in (1,16) \cup [17,\infty)$\\
C.$x \in (1,16] \cup [17,\infty)$\\
D.$x \in [1,16) \cup [17,\infty)$\\
E.$x \in [1,16] \cup (17,\infty)$\\
F.$x \in (1,16) \cup (17,\infty)$\\
G.$x \in [1,16) \cup (17,\infty)$\\
H.$x \in (1,16] \cup (17,\infty)$
\testStop
\kluczStart
A
\kluczStop



\zadStart{Zadanie z Wikieł Z 1.62 a) moja wersja nr 163}

Rozwiązać nierówności $(x-1)(x-16)(x-18)\ge0$.
\zadStop
\rozwStart{Patryk Wirkus}{}
Miejsca zerowe naszego wielomianu to: $1, 16, 18$.\\
Wielomian jest stopnia nieparzystego, ponadto znak współczynnika przy\linebreak najwyższej potędze x jest dodatni.\\ W związku z tym wykres wielomianu zaczyna się od lewej strony poniżej osi OX. A więc $$x \in [1,16] \cup [18,\infty).$$
\rozwStop
\odpStart
$x \in [1,16] \cup [18,\infty)$
\odpStop
\testStart
A.$x \in [1,16] \cup [18,\infty)$\\
B.$x \in (1,16) \cup [18,\infty)$\\
C.$x \in (1,16] \cup [18,\infty)$\\
D.$x \in [1,16) \cup [18,\infty)$\\
E.$x \in [1,16] \cup (18,\infty)$\\
F.$x \in (1,16) \cup (18,\infty)$\\
G.$x \in [1,16) \cup (18,\infty)$\\
H.$x \in (1,16] \cup (18,\infty)$
\testStop
\kluczStart
A
\kluczStop



\zadStart{Zadanie z Wikieł Z 1.62 a) moja wersja nr 164}

Rozwiązać nierówności $(x-1)(x-16)(x-19)\ge0$.
\zadStop
\rozwStart{Patryk Wirkus}{}
Miejsca zerowe naszego wielomianu to: $1, 16, 19$.\\
Wielomian jest stopnia nieparzystego, ponadto znak współczynnika przy\linebreak najwyższej potędze x jest dodatni.\\ W związku z tym wykres wielomianu zaczyna się od lewej strony poniżej osi OX. A więc $$x \in [1,16] \cup [19,\infty).$$
\rozwStop
\odpStart
$x \in [1,16] \cup [19,\infty)$
\odpStop
\testStart
A.$x \in [1,16] \cup [19,\infty)$\\
B.$x \in (1,16) \cup [19,\infty)$\\
C.$x \in (1,16] \cup [19,\infty)$\\
D.$x \in [1,16) \cup [19,\infty)$\\
E.$x \in [1,16] \cup (19,\infty)$\\
F.$x \in (1,16) \cup (19,\infty)$\\
G.$x \in [1,16) \cup (19,\infty)$\\
H.$x \in (1,16] \cup (19,\infty)$
\testStop
\kluczStart
A
\kluczStop



\zadStart{Zadanie z Wikieł Z 1.62 a) moja wersja nr 165}

Rozwiązać nierówności $(x-1)(x-16)(x-20)\ge0$.
\zadStop
\rozwStart{Patryk Wirkus}{}
Miejsca zerowe naszego wielomianu to: $1, 16, 20$.\\
Wielomian jest stopnia nieparzystego, ponadto znak współczynnika przy\linebreak najwyższej potędze x jest dodatni.\\ W związku z tym wykres wielomianu zaczyna się od lewej strony poniżej osi OX. A więc $$x \in [1,16] \cup [20,\infty).$$
\rozwStop
\odpStart
$x \in [1,16] \cup [20,\infty)$
\odpStop
\testStart
A.$x \in [1,16] \cup [20,\infty)$\\
B.$x \in (1,16) \cup [20,\infty)$\\
C.$x \in (1,16] \cup [20,\infty)$\\
D.$x \in [1,16) \cup [20,\infty)$\\
E.$x \in [1,16] \cup (20,\infty)$\\
F.$x \in (1,16) \cup (20,\infty)$\\
G.$x \in [1,16) \cup (20,\infty)$\\
H.$x \in (1,16] \cup (20,\infty)$
\testStop
\kluczStart
A
\kluczStop



\zadStart{Zadanie z Wikieł Z 1.62 a) moja wersja nr 166}

Rozwiązać nierówności $(x-1)(x-17)(x-18)\ge0$.
\zadStop
\rozwStart{Patryk Wirkus}{}
Miejsca zerowe naszego wielomianu to: $1, 17, 18$.\\
Wielomian jest stopnia nieparzystego, ponadto znak współczynnika przy\linebreak najwyższej potędze x jest dodatni.\\ W związku z tym wykres wielomianu zaczyna się od lewej strony poniżej osi OX. A więc $$x \in [1,17] \cup [18,\infty).$$
\rozwStop
\odpStart
$x \in [1,17] \cup [18,\infty)$
\odpStop
\testStart
A.$x \in [1,17] \cup [18,\infty)$\\
B.$x \in (1,17) \cup [18,\infty)$\\
C.$x \in (1,17] \cup [18,\infty)$\\
D.$x \in [1,17) \cup [18,\infty)$\\
E.$x \in [1,17] \cup (18,\infty)$\\
F.$x \in (1,17) \cup (18,\infty)$\\
G.$x \in [1,17) \cup (18,\infty)$\\
H.$x \in (1,17] \cup (18,\infty)$
\testStop
\kluczStart
A
\kluczStop



\zadStart{Zadanie z Wikieł Z 1.62 a) moja wersja nr 167}

Rozwiązać nierówności $(x-1)(x-17)(x-19)\ge0$.
\zadStop
\rozwStart{Patryk Wirkus}{}
Miejsca zerowe naszego wielomianu to: $1, 17, 19$.\\
Wielomian jest stopnia nieparzystego, ponadto znak współczynnika przy\linebreak najwyższej potędze x jest dodatni.\\ W związku z tym wykres wielomianu zaczyna się od lewej strony poniżej osi OX. A więc $$x \in [1,17] \cup [19,\infty).$$
\rozwStop
\odpStart
$x \in [1,17] \cup [19,\infty)$
\odpStop
\testStart
A.$x \in [1,17] \cup [19,\infty)$\\
B.$x \in (1,17) \cup [19,\infty)$\\
C.$x \in (1,17] \cup [19,\infty)$\\
D.$x \in [1,17) \cup [19,\infty)$\\
E.$x \in [1,17] \cup (19,\infty)$\\
F.$x \in (1,17) \cup (19,\infty)$\\
G.$x \in [1,17) \cup (19,\infty)$\\
H.$x \in (1,17] \cup (19,\infty)$
\testStop
\kluczStart
A
\kluczStop



\zadStart{Zadanie z Wikieł Z 1.62 a) moja wersja nr 168}

Rozwiązać nierówności $(x-1)(x-17)(x-20)\ge0$.
\zadStop
\rozwStart{Patryk Wirkus}{}
Miejsca zerowe naszego wielomianu to: $1, 17, 20$.\\
Wielomian jest stopnia nieparzystego, ponadto znak współczynnika przy\linebreak najwyższej potędze x jest dodatni.\\ W związku z tym wykres wielomianu zaczyna się od lewej strony poniżej osi OX. A więc $$x \in [1,17] \cup [20,\infty).$$
\rozwStop
\odpStart
$x \in [1,17] \cup [20,\infty)$
\odpStop
\testStart
A.$x \in [1,17] \cup [20,\infty)$\\
B.$x \in (1,17) \cup [20,\infty)$\\
C.$x \in (1,17] \cup [20,\infty)$\\
D.$x \in [1,17) \cup [20,\infty)$\\
E.$x \in [1,17] \cup (20,\infty)$\\
F.$x \in (1,17) \cup (20,\infty)$\\
G.$x \in [1,17) \cup (20,\infty)$\\
H.$x \in (1,17] \cup (20,\infty)$
\testStop
\kluczStart
A
\kluczStop



\zadStart{Zadanie z Wikieł Z 1.62 a) moja wersja nr 169}

Rozwiązać nierówności $(x-1)(x-18)(x-19)\ge0$.
\zadStop
\rozwStart{Patryk Wirkus}{}
Miejsca zerowe naszego wielomianu to: $1, 18, 19$.\\
Wielomian jest stopnia nieparzystego, ponadto znak współczynnika przy\linebreak najwyższej potędze x jest dodatni.\\ W związku z tym wykres wielomianu zaczyna się od lewej strony poniżej osi OX. A więc $$x \in [1,18] \cup [19,\infty).$$
\rozwStop
\odpStart
$x \in [1,18] \cup [19,\infty)$
\odpStop
\testStart
A.$x \in [1,18] \cup [19,\infty)$\\
B.$x \in (1,18) \cup [19,\infty)$\\
C.$x \in (1,18] \cup [19,\infty)$\\
D.$x \in [1,18) \cup [19,\infty)$\\
E.$x \in [1,18] \cup (19,\infty)$\\
F.$x \in (1,18) \cup (19,\infty)$\\
G.$x \in [1,18) \cup (19,\infty)$\\
H.$x \in (1,18] \cup (19,\infty)$
\testStop
\kluczStart
A
\kluczStop



\zadStart{Zadanie z Wikieł Z 1.62 a) moja wersja nr 170}

Rozwiązać nierówności $(x-1)(x-18)(x-20)\ge0$.
\zadStop
\rozwStart{Patryk Wirkus}{}
Miejsca zerowe naszego wielomianu to: $1, 18, 20$.\\
Wielomian jest stopnia nieparzystego, ponadto znak współczynnika przy\linebreak najwyższej potędze x jest dodatni.\\ W związku z tym wykres wielomianu zaczyna się od lewej strony poniżej osi OX. A więc $$x \in [1,18] \cup [20,\infty).$$
\rozwStop
\odpStart
$x \in [1,18] \cup [20,\infty)$
\odpStop
\testStart
A.$x \in [1,18] \cup [20,\infty)$\\
B.$x \in (1,18) \cup [20,\infty)$\\
C.$x \in (1,18] \cup [20,\infty)$\\
D.$x \in [1,18) \cup [20,\infty)$\\
E.$x \in [1,18] \cup (20,\infty)$\\
F.$x \in (1,18) \cup (20,\infty)$\\
G.$x \in [1,18) \cup (20,\infty)$\\
H.$x \in (1,18] \cup (20,\infty)$
\testStop
\kluczStart
A
\kluczStop



\zadStart{Zadanie z Wikieł Z 1.62 a) moja wersja nr 171}

Rozwiązać nierówności $(x-1)(x-19)(x-20)\ge0$.
\zadStop
\rozwStart{Patryk Wirkus}{}
Miejsca zerowe naszego wielomianu to: $1, 19, 20$.\\
Wielomian jest stopnia nieparzystego, ponadto znak współczynnika przy\linebreak najwyższej potędze x jest dodatni.\\ W związku z tym wykres wielomianu zaczyna się od lewej strony poniżej osi OX. A więc $$x \in [1,19] \cup [20,\infty).$$
\rozwStop
\odpStart
$x \in [1,19] \cup [20,\infty)$
\odpStop
\testStart
A.$x \in [1,19] \cup [20,\infty)$\\
B.$x \in (1,19) \cup [20,\infty)$\\
C.$x \in (1,19] \cup [20,\infty)$\\
D.$x \in [1,19) \cup [20,\infty)$\\
E.$x \in [1,19] \cup (20,\infty)$\\
F.$x \in (1,19) \cup (20,\infty)$\\
G.$x \in [1,19) \cup (20,\infty)$\\
H.$x \in (1,19] \cup (20,\infty)$
\testStop
\kluczStart
A
\kluczStop



\zadStart{Zadanie z Wikieł Z 1.62 a) moja wersja nr 172}

Rozwiązać nierówności $(x-2)(x-3)(x-4)\ge0$.
\zadStop
\rozwStart{Patryk Wirkus}{}
Miejsca zerowe naszego wielomianu to: $2, 3, 4$.\\
Wielomian jest stopnia nieparzystego, ponadto znak współczynnika przy\linebreak najwyższej potędze x jest dodatni.\\ W związku z tym wykres wielomianu zaczyna się od lewej strony poniżej osi OX. A więc $$x \in [2,3] \cup [4,\infty).$$
\rozwStop
\odpStart
$x \in [2,3] \cup [4,\infty)$
\odpStop
\testStart
A.$x \in [2,3] \cup [4,\infty)$\\
B.$x \in (2,3) \cup [4,\infty)$\\
C.$x \in (2,3] \cup [4,\infty)$\\
D.$x \in [2,3) \cup [4,\infty)$\\
E.$x \in [2,3] \cup (4,\infty)$\\
F.$x \in (2,3) \cup (4,\infty)$\\
G.$x \in [2,3) \cup (4,\infty)$\\
H.$x \in (2,3] \cup (4,\infty)$
\testStop
\kluczStart
A
\kluczStop



\zadStart{Zadanie z Wikieł Z 1.62 a) moja wersja nr 173}

Rozwiązać nierówności $(x-2)(x-3)(x-5)\ge0$.
\zadStop
\rozwStart{Patryk Wirkus}{}
Miejsca zerowe naszego wielomianu to: $2, 3, 5$.\\
Wielomian jest stopnia nieparzystego, ponadto znak współczynnika przy\linebreak najwyższej potędze x jest dodatni.\\ W związku z tym wykres wielomianu zaczyna się od lewej strony poniżej osi OX. A więc $$x \in [2,3] \cup [5,\infty).$$
\rozwStop
\odpStart
$x \in [2,3] \cup [5,\infty)$
\odpStop
\testStart
A.$x \in [2,3] \cup [5,\infty)$\\
B.$x \in (2,3) \cup [5,\infty)$\\
C.$x \in (2,3] \cup [5,\infty)$\\
D.$x \in [2,3) \cup [5,\infty)$\\
E.$x \in [2,3] \cup (5,\infty)$\\
F.$x \in (2,3) \cup (5,\infty)$\\
G.$x \in [2,3) \cup (5,\infty)$\\
H.$x \in (2,3] \cup (5,\infty)$
\testStop
\kluczStart
A
\kluczStop



\zadStart{Zadanie z Wikieł Z 1.62 a) moja wersja nr 174}

Rozwiązać nierówności $(x-2)(x-3)(x-6)\ge0$.
\zadStop
\rozwStart{Patryk Wirkus}{}
Miejsca zerowe naszego wielomianu to: $2, 3, 6$.\\
Wielomian jest stopnia nieparzystego, ponadto znak współczynnika przy\linebreak najwyższej potędze x jest dodatni.\\ W związku z tym wykres wielomianu zaczyna się od lewej strony poniżej osi OX. A więc $$x \in [2,3] \cup [6,\infty).$$
\rozwStop
\odpStart
$x \in [2,3] \cup [6,\infty)$
\odpStop
\testStart
A.$x \in [2,3] \cup [6,\infty)$\\
B.$x \in (2,3) \cup [6,\infty)$\\
C.$x \in (2,3] \cup [6,\infty)$\\
D.$x \in [2,3) \cup [6,\infty)$\\
E.$x \in [2,3] \cup (6,\infty)$\\
F.$x \in (2,3) \cup (6,\infty)$\\
G.$x \in [2,3) \cup (6,\infty)$\\
H.$x \in (2,3] \cup (6,\infty)$
\testStop
\kluczStart
A
\kluczStop



\zadStart{Zadanie z Wikieł Z 1.62 a) moja wersja nr 175}

Rozwiązać nierówności $(x-2)(x-3)(x-7)\ge0$.
\zadStop
\rozwStart{Patryk Wirkus}{}
Miejsca zerowe naszego wielomianu to: $2, 3, 7$.\\
Wielomian jest stopnia nieparzystego, ponadto znak współczynnika przy\linebreak najwyższej potędze x jest dodatni.\\ W związku z tym wykres wielomianu zaczyna się od lewej strony poniżej osi OX. A więc $$x \in [2,3] \cup [7,\infty).$$
\rozwStop
\odpStart
$x \in [2,3] \cup [7,\infty)$
\odpStop
\testStart
A.$x \in [2,3] \cup [7,\infty)$\\
B.$x \in (2,3) \cup [7,\infty)$\\
C.$x \in (2,3] \cup [7,\infty)$\\
D.$x \in [2,3) \cup [7,\infty)$\\
E.$x \in [2,3] \cup (7,\infty)$\\
F.$x \in (2,3) \cup (7,\infty)$\\
G.$x \in [2,3) \cup (7,\infty)$\\
H.$x \in (2,3] \cup (7,\infty)$
\testStop
\kluczStart
A
\kluczStop



\zadStart{Zadanie z Wikieł Z 1.62 a) moja wersja nr 176}

Rozwiązać nierówności $(x-2)(x-3)(x-8)\ge0$.
\zadStop
\rozwStart{Patryk Wirkus}{}
Miejsca zerowe naszego wielomianu to: $2, 3, 8$.\\
Wielomian jest stopnia nieparzystego, ponadto znak współczynnika przy\linebreak najwyższej potędze x jest dodatni.\\ W związku z tym wykres wielomianu zaczyna się od lewej strony poniżej osi OX. A więc $$x \in [2,3] \cup [8,\infty).$$
\rozwStop
\odpStart
$x \in [2,3] \cup [8,\infty)$
\odpStop
\testStart
A.$x \in [2,3] \cup [8,\infty)$\\
B.$x \in (2,3) \cup [8,\infty)$\\
C.$x \in (2,3] \cup [8,\infty)$\\
D.$x \in [2,3) \cup [8,\infty)$\\
E.$x \in [2,3] \cup (8,\infty)$\\
F.$x \in (2,3) \cup (8,\infty)$\\
G.$x \in [2,3) \cup (8,\infty)$\\
H.$x \in (2,3] \cup (8,\infty)$
\testStop
\kluczStart
A
\kluczStop



\zadStart{Zadanie z Wikieł Z 1.62 a) moja wersja nr 177}

Rozwiązać nierówności $(x-2)(x-3)(x-9)\ge0$.
\zadStop
\rozwStart{Patryk Wirkus}{}
Miejsca zerowe naszego wielomianu to: $2, 3, 9$.\\
Wielomian jest stopnia nieparzystego, ponadto znak współczynnika przy\linebreak najwyższej potędze x jest dodatni.\\ W związku z tym wykres wielomianu zaczyna się od lewej strony poniżej osi OX. A więc $$x \in [2,3] \cup [9,\infty).$$
\rozwStop
\odpStart
$x \in [2,3] \cup [9,\infty)$
\odpStop
\testStart
A.$x \in [2,3] \cup [9,\infty)$\\
B.$x \in (2,3) \cup [9,\infty)$\\
C.$x \in (2,3] \cup [9,\infty)$\\
D.$x \in [2,3) \cup [9,\infty)$\\
E.$x \in [2,3] \cup (9,\infty)$\\
F.$x \in (2,3) \cup (9,\infty)$\\
G.$x \in [2,3) \cup (9,\infty)$\\
H.$x \in (2,3] \cup (9,\infty)$
\testStop
\kluczStart
A
\kluczStop



\zadStart{Zadanie z Wikieł Z 1.62 a) moja wersja nr 178}

Rozwiązać nierówności $(x-2)(x-3)(x-10)\ge0$.
\zadStop
\rozwStart{Patryk Wirkus}{}
Miejsca zerowe naszego wielomianu to: $2, 3, 10$.\\
Wielomian jest stopnia nieparzystego, ponadto znak współczynnika przy\linebreak najwyższej potędze x jest dodatni.\\ W związku z tym wykres wielomianu zaczyna się od lewej strony poniżej osi OX. A więc $$x \in [2,3] \cup [10,\infty).$$
\rozwStop
\odpStart
$x \in [2,3] \cup [10,\infty)$
\odpStop
\testStart
A.$x \in [2,3] \cup [10,\infty)$\\
B.$x \in (2,3) \cup [10,\infty)$\\
C.$x \in (2,3] \cup [10,\infty)$\\
D.$x \in [2,3) \cup [10,\infty)$\\
E.$x \in [2,3] \cup (10,\infty)$\\
F.$x \in (2,3) \cup (10,\infty)$\\
G.$x \in [2,3) \cup (10,\infty)$\\
H.$x \in (2,3] \cup (10,\infty)$
\testStop
\kluczStart
A
\kluczStop



\zadStart{Zadanie z Wikieł Z 1.62 a) moja wersja nr 179}

Rozwiązać nierówności $(x-2)(x-3)(x-11)\ge0$.
\zadStop
\rozwStart{Patryk Wirkus}{}
Miejsca zerowe naszego wielomianu to: $2, 3, 11$.\\
Wielomian jest stopnia nieparzystego, ponadto znak współczynnika przy\linebreak najwyższej potędze x jest dodatni.\\ W związku z tym wykres wielomianu zaczyna się od lewej strony poniżej osi OX. A więc $$x \in [2,3] \cup [11,\infty).$$
\rozwStop
\odpStart
$x \in [2,3] \cup [11,\infty)$
\odpStop
\testStart
A.$x \in [2,3] \cup [11,\infty)$\\
B.$x \in (2,3) \cup [11,\infty)$\\
C.$x \in (2,3] \cup [11,\infty)$\\
D.$x \in [2,3) \cup [11,\infty)$\\
E.$x \in [2,3] \cup (11,\infty)$\\
F.$x \in (2,3) \cup (11,\infty)$\\
G.$x \in [2,3) \cup (11,\infty)$\\
H.$x \in (2,3] \cup (11,\infty)$
\testStop
\kluczStart
A
\kluczStop



\zadStart{Zadanie z Wikieł Z 1.62 a) moja wersja nr 180}

Rozwiązać nierówności $(x-2)(x-3)(x-12)\ge0$.
\zadStop
\rozwStart{Patryk Wirkus}{}
Miejsca zerowe naszego wielomianu to: $2, 3, 12$.\\
Wielomian jest stopnia nieparzystego, ponadto znak współczynnika przy\linebreak najwyższej potędze x jest dodatni.\\ W związku z tym wykres wielomianu zaczyna się od lewej strony poniżej osi OX. A więc $$x \in [2,3] \cup [12,\infty).$$
\rozwStop
\odpStart
$x \in [2,3] \cup [12,\infty)$
\odpStop
\testStart
A.$x \in [2,3] \cup [12,\infty)$\\
B.$x \in (2,3) \cup [12,\infty)$\\
C.$x \in (2,3] \cup [12,\infty)$\\
D.$x \in [2,3) \cup [12,\infty)$\\
E.$x \in [2,3] \cup (12,\infty)$\\
F.$x \in (2,3) \cup (12,\infty)$\\
G.$x \in [2,3) \cup (12,\infty)$\\
H.$x \in (2,3] \cup (12,\infty)$
\testStop
\kluczStart
A
\kluczStop



\zadStart{Zadanie z Wikieł Z 1.62 a) moja wersja nr 181}

Rozwiązać nierówności $(x-2)(x-3)(x-13)\ge0$.
\zadStop
\rozwStart{Patryk Wirkus}{}
Miejsca zerowe naszego wielomianu to: $2, 3, 13$.\\
Wielomian jest stopnia nieparzystego, ponadto znak współczynnika przy\linebreak najwyższej potędze x jest dodatni.\\ W związku z tym wykres wielomianu zaczyna się od lewej strony poniżej osi OX. A więc $$x \in [2,3] \cup [13,\infty).$$
\rozwStop
\odpStart
$x \in [2,3] \cup [13,\infty)$
\odpStop
\testStart
A.$x \in [2,3] \cup [13,\infty)$\\
B.$x \in (2,3) \cup [13,\infty)$\\
C.$x \in (2,3] \cup [13,\infty)$\\
D.$x \in [2,3) \cup [13,\infty)$\\
E.$x \in [2,3] \cup (13,\infty)$\\
F.$x \in (2,3) \cup (13,\infty)$\\
G.$x \in [2,3) \cup (13,\infty)$\\
H.$x \in (2,3] \cup (13,\infty)$
\testStop
\kluczStart
A
\kluczStop



\zadStart{Zadanie z Wikieł Z 1.62 a) moja wersja nr 182}

Rozwiązać nierówności $(x-2)(x-3)(x-14)\ge0$.
\zadStop
\rozwStart{Patryk Wirkus}{}
Miejsca zerowe naszego wielomianu to: $2, 3, 14$.\\
Wielomian jest stopnia nieparzystego, ponadto znak współczynnika przy\linebreak najwyższej potędze x jest dodatni.\\ W związku z tym wykres wielomianu zaczyna się od lewej strony poniżej osi OX. A więc $$x \in [2,3] \cup [14,\infty).$$
\rozwStop
\odpStart
$x \in [2,3] \cup [14,\infty)$
\odpStop
\testStart
A.$x \in [2,3] \cup [14,\infty)$\\
B.$x \in (2,3) \cup [14,\infty)$\\
C.$x \in (2,3] \cup [14,\infty)$\\
D.$x \in [2,3) \cup [14,\infty)$\\
E.$x \in [2,3] \cup (14,\infty)$\\
F.$x \in (2,3) \cup (14,\infty)$\\
G.$x \in [2,3) \cup (14,\infty)$\\
H.$x \in (2,3] \cup (14,\infty)$
\testStop
\kluczStart
A
\kluczStop



\zadStart{Zadanie z Wikieł Z 1.62 a) moja wersja nr 183}

Rozwiązać nierówności $(x-2)(x-3)(x-15)\ge0$.
\zadStop
\rozwStart{Patryk Wirkus}{}
Miejsca zerowe naszego wielomianu to: $2, 3, 15$.\\
Wielomian jest stopnia nieparzystego, ponadto znak współczynnika przy\linebreak najwyższej potędze x jest dodatni.\\ W związku z tym wykres wielomianu zaczyna się od lewej strony poniżej osi OX. A więc $$x \in [2,3] \cup [15,\infty).$$
\rozwStop
\odpStart
$x \in [2,3] \cup [15,\infty)$
\odpStop
\testStart
A.$x \in [2,3] \cup [15,\infty)$\\
B.$x \in (2,3) \cup [15,\infty)$\\
C.$x \in (2,3] \cup [15,\infty)$\\
D.$x \in [2,3) \cup [15,\infty)$\\
E.$x \in [2,3] \cup (15,\infty)$\\
F.$x \in (2,3) \cup (15,\infty)$\\
G.$x \in [2,3) \cup (15,\infty)$\\
H.$x \in (2,3] \cup (15,\infty)$
\testStop
\kluczStart
A
\kluczStop



\zadStart{Zadanie z Wikieł Z 1.62 a) moja wersja nr 184}

Rozwiązać nierówności $(x-2)(x-3)(x-16)\ge0$.
\zadStop
\rozwStart{Patryk Wirkus}{}
Miejsca zerowe naszego wielomianu to: $2, 3, 16$.\\
Wielomian jest stopnia nieparzystego, ponadto znak współczynnika przy\linebreak najwyższej potędze x jest dodatni.\\ W związku z tym wykres wielomianu zaczyna się od lewej strony poniżej osi OX. A więc $$x \in [2,3] \cup [16,\infty).$$
\rozwStop
\odpStart
$x \in [2,3] \cup [16,\infty)$
\odpStop
\testStart
A.$x \in [2,3] \cup [16,\infty)$\\
B.$x \in (2,3) \cup [16,\infty)$\\
C.$x \in (2,3] \cup [16,\infty)$\\
D.$x \in [2,3) \cup [16,\infty)$\\
E.$x \in [2,3] \cup (16,\infty)$\\
F.$x \in (2,3) \cup (16,\infty)$\\
G.$x \in [2,3) \cup (16,\infty)$\\
H.$x \in (2,3] \cup (16,\infty)$
\testStop
\kluczStart
A
\kluczStop



\zadStart{Zadanie z Wikieł Z 1.62 a) moja wersja nr 185}

Rozwiązać nierówności $(x-2)(x-3)(x-17)\ge0$.
\zadStop
\rozwStart{Patryk Wirkus}{}
Miejsca zerowe naszego wielomianu to: $2, 3, 17$.\\
Wielomian jest stopnia nieparzystego, ponadto znak współczynnika przy\linebreak najwyższej potędze x jest dodatni.\\ W związku z tym wykres wielomianu zaczyna się od lewej strony poniżej osi OX. A więc $$x \in [2,3] \cup [17,\infty).$$
\rozwStop
\odpStart
$x \in [2,3] \cup [17,\infty)$
\odpStop
\testStart
A.$x \in [2,3] \cup [17,\infty)$\\
B.$x \in (2,3) \cup [17,\infty)$\\
C.$x \in (2,3] \cup [17,\infty)$\\
D.$x \in [2,3) \cup [17,\infty)$\\
E.$x \in [2,3] \cup (17,\infty)$\\
F.$x \in (2,3) \cup (17,\infty)$\\
G.$x \in [2,3) \cup (17,\infty)$\\
H.$x \in (2,3] \cup (17,\infty)$
\testStop
\kluczStart
A
\kluczStop



\zadStart{Zadanie z Wikieł Z 1.62 a) moja wersja nr 186}

Rozwiązać nierówności $(x-2)(x-3)(x-18)\ge0$.
\zadStop
\rozwStart{Patryk Wirkus}{}
Miejsca zerowe naszego wielomianu to: $2, 3, 18$.\\
Wielomian jest stopnia nieparzystego, ponadto znak współczynnika przy\linebreak najwyższej potędze x jest dodatni.\\ W związku z tym wykres wielomianu zaczyna się od lewej strony poniżej osi OX. A więc $$x \in [2,3] \cup [18,\infty).$$
\rozwStop
\odpStart
$x \in [2,3] \cup [18,\infty)$
\odpStop
\testStart
A.$x \in [2,3] \cup [18,\infty)$\\
B.$x \in (2,3) \cup [18,\infty)$\\
C.$x \in (2,3] \cup [18,\infty)$\\
D.$x \in [2,3) \cup [18,\infty)$\\
E.$x \in [2,3] \cup (18,\infty)$\\
F.$x \in (2,3) \cup (18,\infty)$\\
G.$x \in [2,3) \cup (18,\infty)$\\
H.$x \in (2,3] \cup (18,\infty)$
\testStop
\kluczStart
A
\kluczStop



\zadStart{Zadanie z Wikieł Z 1.62 a) moja wersja nr 187}

Rozwiązać nierówności $(x-2)(x-3)(x-19)\ge0$.
\zadStop
\rozwStart{Patryk Wirkus}{}
Miejsca zerowe naszego wielomianu to: $2, 3, 19$.\\
Wielomian jest stopnia nieparzystego, ponadto znak współczynnika przy\linebreak najwyższej potędze x jest dodatni.\\ W związku z tym wykres wielomianu zaczyna się od lewej strony poniżej osi OX. A więc $$x \in [2,3] \cup [19,\infty).$$
\rozwStop
\odpStart
$x \in [2,3] \cup [19,\infty)$
\odpStop
\testStart
A.$x \in [2,3] \cup [19,\infty)$\\
B.$x \in (2,3) \cup [19,\infty)$\\
C.$x \in (2,3] \cup [19,\infty)$\\
D.$x \in [2,3) \cup [19,\infty)$\\
E.$x \in [2,3] \cup (19,\infty)$\\
F.$x \in (2,3) \cup (19,\infty)$\\
G.$x \in [2,3) \cup (19,\infty)$\\
H.$x \in (2,3] \cup (19,\infty)$
\testStop
\kluczStart
A
\kluczStop



\zadStart{Zadanie z Wikieł Z 1.62 a) moja wersja nr 188}

Rozwiązać nierówności $(x-2)(x-3)(x-20)\ge0$.
\zadStop
\rozwStart{Patryk Wirkus}{}
Miejsca zerowe naszego wielomianu to: $2, 3, 20$.\\
Wielomian jest stopnia nieparzystego, ponadto znak współczynnika przy\linebreak najwyższej potędze x jest dodatni.\\ W związku z tym wykres wielomianu zaczyna się od lewej strony poniżej osi OX. A więc $$x \in [2,3] \cup [20,\infty).$$
\rozwStop
\odpStart
$x \in [2,3] \cup [20,\infty)$
\odpStop
\testStart
A.$x \in [2,3] \cup [20,\infty)$\\
B.$x \in (2,3) \cup [20,\infty)$\\
C.$x \in (2,3] \cup [20,\infty)$\\
D.$x \in [2,3) \cup [20,\infty)$\\
E.$x \in [2,3] \cup (20,\infty)$\\
F.$x \in (2,3) \cup (20,\infty)$\\
G.$x \in [2,3) \cup (20,\infty)$\\
H.$x \in (2,3] \cup (20,\infty)$
\testStop
\kluczStart
A
\kluczStop



\zadStart{Zadanie z Wikieł Z 1.62 a) moja wersja nr 189}

Rozwiązać nierówności $(x-2)(x-4)(x-5)\ge0$.
\zadStop
\rozwStart{Patryk Wirkus}{}
Miejsca zerowe naszego wielomianu to: $2, 4, 5$.\\
Wielomian jest stopnia nieparzystego, ponadto znak współczynnika przy\linebreak najwyższej potędze x jest dodatni.\\ W związku z tym wykres wielomianu zaczyna się od lewej strony poniżej osi OX. A więc $$x \in [2,4] \cup [5,\infty).$$
\rozwStop
\odpStart
$x \in [2,4] \cup [5,\infty)$
\odpStop
\testStart
A.$x \in [2,4] \cup [5,\infty)$\\
B.$x \in (2,4) \cup [5,\infty)$\\
C.$x \in (2,4] \cup [5,\infty)$\\
D.$x \in [2,4) \cup [5,\infty)$\\
E.$x \in [2,4] \cup (5,\infty)$\\
F.$x \in (2,4) \cup (5,\infty)$\\
G.$x \in [2,4) \cup (5,\infty)$\\
H.$x \in (2,4] \cup (5,\infty)$
\testStop
\kluczStart
A
\kluczStop



\zadStart{Zadanie z Wikieł Z 1.62 a) moja wersja nr 190}

Rozwiązać nierówności $(x-2)(x-4)(x-6)\ge0$.
\zadStop
\rozwStart{Patryk Wirkus}{}
Miejsca zerowe naszego wielomianu to: $2, 4, 6$.\\
Wielomian jest stopnia nieparzystego, ponadto znak współczynnika przy\linebreak najwyższej potędze x jest dodatni.\\ W związku z tym wykres wielomianu zaczyna się od lewej strony poniżej osi OX. A więc $$x \in [2,4] \cup [6,\infty).$$
\rozwStop
\odpStart
$x \in [2,4] \cup [6,\infty)$
\odpStop
\testStart
A.$x \in [2,4] \cup [6,\infty)$\\
B.$x \in (2,4) \cup [6,\infty)$\\
C.$x \in (2,4] \cup [6,\infty)$\\
D.$x \in [2,4) \cup [6,\infty)$\\
E.$x \in [2,4] \cup (6,\infty)$\\
F.$x \in (2,4) \cup (6,\infty)$\\
G.$x \in [2,4) \cup (6,\infty)$\\
H.$x \in (2,4] \cup (6,\infty)$
\testStop
\kluczStart
A
\kluczStop



\zadStart{Zadanie z Wikieł Z 1.62 a) moja wersja nr 191}

Rozwiązać nierówności $(x-2)(x-4)(x-7)\ge0$.
\zadStop
\rozwStart{Patryk Wirkus}{}
Miejsca zerowe naszego wielomianu to: $2, 4, 7$.\\
Wielomian jest stopnia nieparzystego, ponadto znak współczynnika przy\linebreak najwyższej potędze x jest dodatni.\\ W związku z tym wykres wielomianu zaczyna się od lewej strony poniżej osi OX. A więc $$x \in [2,4] \cup [7,\infty).$$
\rozwStop
\odpStart
$x \in [2,4] \cup [7,\infty)$
\odpStop
\testStart
A.$x \in [2,4] \cup [7,\infty)$\\
B.$x \in (2,4) \cup [7,\infty)$\\
C.$x \in (2,4] \cup [7,\infty)$\\
D.$x \in [2,4) \cup [7,\infty)$\\
E.$x \in [2,4] \cup (7,\infty)$\\
F.$x \in (2,4) \cup (7,\infty)$\\
G.$x \in [2,4) \cup (7,\infty)$\\
H.$x \in (2,4] \cup (7,\infty)$
\testStop
\kluczStart
A
\kluczStop



\zadStart{Zadanie z Wikieł Z 1.62 a) moja wersja nr 192}

Rozwiązać nierówności $(x-2)(x-4)(x-8)\ge0$.
\zadStop
\rozwStart{Patryk Wirkus}{}
Miejsca zerowe naszego wielomianu to: $2, 4, 8$.\\
Wielomian jest stopnia nieparzystego, ponadto znak współczynnika przy\linebreak najwyższej potędze x jest dodatni.\\ W związku z tym wykres wielomianu zaczyna się od lewej strony poniżej osi OX. A więc $$x \in [2,4] \cup [8,\infty).$$
\rozwStop
\odpStart
$x \in [2,4] \cup [8,\infty)$
\odpStop
\testStart
A.$x \in [2,4] \cup [8,\infty)$\\
B.$x \in (2,4) \cup [8,\infty)$\\
C.$x \in (2,4] \cup [8,\infty)$\\
D.$x \in [2,4) \cup [8,\infty)$\\
E.$x \in [2,4] \cup (8,\infty)$\\
F.$x \in (2,4) \cup (8,\infty)$\\
G.$x \in [2,4) \cup (8,\infty)$\\
H.$x \in (2,4] \cup (8,\infty)$
\testStop
\kluczStart
A
\kluczStop



\zadStart{Zadanie z Wikieł Z 1.62 a) moja wersja nr 193}

Rozwiązać nierówności $(x-2)(x-4)(x-9)\ge0$.
\zadStop
\rozwStart{Patryk Wirkus}{}
Miejsca zerowe naszego wielomianu to: $2, 4, 9$.\\
Wielomian jest stopnia nieparzystego, ponadto znak współczynnika przy\linebreak najwyższej potędze x jest dodatni.\\ W związku z tym wykres wielomianu zaczyna się od lewej strony poniżej osi OX. A więc $$x \in [2,4] \cup [9,\infty).$$
\rozwStop
\odpStart
$x \in [2,4] \cup [9,\infty)$
\odpStop
\testStart
A.$x \in [2,4] \cup [9,\infty)$\\
B.$x \in (2,4) \cup [9,\infty)$\\
C.$x \in (2,4] \cup [9,\infty)$\\
D.$x \in [2,4) \cup [9,\infty)$\\
E.$x \in [2,4] \cup (9,\infty)$\\
F.$x \in (2,4) \cup (9,\infty)$\\
G.$x \in [2,4) \cup (9,\infty)$\\
H.$x \in (2,4] \cup (9,\infty)$
\testStop
\kluczStart
A
\kluczStop



\zadStart{Zadanie z Wikieł Z 1.62 a) moja wersja nr 194}

Rozwiązać nierówności $(x-2)(x-4)(x-10)\ge0$.
\zadStop
\rozwStart{Patryk Wirkus}{}
Miejsca zerowe naszego wielomianu to: $2, 4, 10$.\\
Wielomian jest stopnia nieparzystego, ponadto znak współczynnika przy\linebreak najwyższej potędze x jest dodatni.\\ W związku z tym wykres wielomianu zaczyna się od lewej strony poniżej osi OX. A więc $$x \in [2,4] \cup [10,\infty).$$
\rozwStop
\odpStart
$x \in [2,4] \cup [10,\infty)$
\odpStop
\testStart
A.$x \in [2,4] \cup [10,\infty)$\\
B.$x \in (2,4) \cup [10,\infty)$\\
C.$x \in (2,4] \cup [10,\infty)$\\
D.$x \in [2,4) \cup [10,\infty)$\\
E.$x \in [2,4] \cup (10,\infty)$\\
F.$x \in (2,4) \cup (10,\infty)$\\
G.$x \in [2,4) \cup (10,\infty)$\\
H.$x \in (2,4] \cup (10,\infty)$
\testStop
\kluczStart
A
\kluczStop



\zadStart{Zadanie z Wikieł Z 1.62 a) moja wersja nr 195}

Rozwiązać nierówności $(x-2)(x-4)(x-11)\ge0$.
\zadStop
\rozwStart{Patryk Wirkus}{}
Miejsca zerowe naszego wielomianu to: $2, 4, 11$.\\
Wielomian jest stopnia nieparzystego, ponadto znak współczynnika przy\linebreak najwyższej potędze x jest dodatni.\\ W związku z tym wykres wielomianu zaczyna się od lewej strony poniżej osi OX. A więc $$x \in [2,4] \cup [11,\infty).$$
\rozwStop
\odpStart
$x \in [2,4] \cup [11,\infty)$
\odpStop
\testStart
A.$x \in [2,4] \cup [11,\infty)$\\
B.$x \in (2,4) \cup [11,\infty)$\\
C.$x \in (2,4] \cup [11,\infty)$\\
D.$x \in [2,4) \cup [11,\infty)$\\
E.$x \in [2,4] \cup (11,\infty)$\\
F.$x \in (2,4) \cup (11,\infty)$\\
G.$x \in [2,4) \cup (11,\infty)$\\
H.$x \in (2,4] \cup (11,\infty)$
\testStop
\kluczStart
A
\kluczStop



\zadStart{Zadanie z Wikieł Z 1.62 a) moja wersja nr 196}

Rozwiązać nierówności $(x-2)(x-4)(x-12)\ge0$.
\zadStop
\rozwStart{Patryk Wirkus}{}
Miejsca zerowe naszego wielomianu to: $2, 4, 12$.\\
Wielomian jest stopnia nieparzystego, ponadto znak współczynnika przy\linebreak najwyższej potędze x jest dodatni.\\ W związku z tym wykres wielomianu zaczyna się od lewej strony poniżej osi OX. A więc $$x \in [2,4] \cup [12,\infty).$$
\rozwStop
\odpStart
$x \in [2,4] \cup [12,\infty)$
\odpStop
\testStart
A.$x \in [2,4] \cup [12,\infty)$\\
B.$x \in (2,4) \cup [12,\infty)$\\
C.$x \in (2,4] \cup [12,\infty)$\\
D.$x \in [2,4) \cup [12,\infty)$\\
E.$x \in [2,4] \cup (12,\infty)$\\
F.$x \in (2,4) \cup (12,\infty)$\\
G.$x \in [2,4) \cup (12,\infty)$\\
H.$x \in (2,4] \cup (12,\infty)$
\testStop
\kluczStart
A
\kluczStop



\zadStart{Zadanie z Wikieł Z 1.62 a) moja wersja nr 197}

Rozwiązać nierówności $(x-2)(x-4)(x-13)\ge0$.
\zadStop
\rozwStart{Patryk Wirkus}{}
Miejsca zerowe naszego wielomianu to: $2, 4, 13$.\\
Wielomian jest stopnia nieparzystego, ponadto znak współczynnika przy\linebreak najwyższej potędze x jest dodatni.\\ W związku z tym wykres wielomianu zaczyna się od lewej strony poniżej osi OX. A więc $$x \in [2,4] \cup [13,\infty).$$
\rozwStop
\odpStart
$x \in [2,4] \cup [13,\infty)$
\odpStop
\testStart
A.$x \in [2,4] \cup [13,\infty)$\\
B.$x \in (2,4) \cup [13,\infty)$\\
C.$x \in (2,4] \cup [13,\infty)$\\
D.$x \in [2,4) \cup [13,\infty)$\\
E.$x \in [2,4] \cup (13,\infty)$\\
F.$x \in (2,4) \cup (13,\infty)$\\
G.$x \in [2,4) \cup (13,\infty)$\\
H.$x \in (2,4] \cup (13,\infty)$
\testStop
\kluczStart
A
\kluczStop



\zadStart{Zadanie z Wikieł Z 1.62 a) moja wersja nr 198}

Rozwiązać nierówności $(x-2)(x-4)(x-14)\ge0$.
\zadStop
\rozwStart{Patryk Wirkus}{}
Miejsca zerowe naszego wielomianu to: $2, 4, 14$.\\
Wielomian jest stopnia nieparzystego, ponadto znak współczynnika przy\linebreak najwyższej potędze x jest dodatni.\\ W związku z tym wykres wielomianu zaczyna się od lewej strony poniżej osi OX. A więc $$x \in [2,4] \cup [14,\infty).$$
\rozwStop
\odpStart
$x \in [2,4] \cup [14,\infty)$
\odpStop
\testStart
A.$x \in [2,4] \cup [14,\infty)$\\
B.$x \in (2,4) \cup [14,\infty)$\\
C.$x \in (2,4] \cup [14,\infty)$\\
D.$x \in [2,4) \cup [14,\infty)$\\
E.$x \in [2,4] \cup (14,\infty)$\\
F.$x \in (2,4) \cup (14,\infty)$\\
G.$x \in [2,4) \cup (14,\infty)$\\
H.$x \in (2,4] \cup (14,\infty)$
\testStop
\kluczStart
A
\kluczStop



\zadStart{Zadanie z Wikieł Z 1.62 a) moja wersja nr 199}

Rozwiązać nierówności $(x-2)(x-4)(x-15)\ge0$.
\zadStop
\rozwStart{Patryk Wirkus}{}
Miejsca zerowe naszego wielomianu to: $2, 4, 15$.\\
Wielomian jest stopnia nieparzystego, ponadto znak współczynnika przy\linebreak najwyższej potędze x jest dodatni.\\ W związku z tym wykres wielomianu zaczyna się od lewej strony poniżej osi OX. A więc $$x \in [2,4] \cup [15,\infty).$$
\rozwStop
\odpStart
$x \in [2,4] \cup [15,\infty)$
\odpStop
\testStart
A.$x \in [2,4] \cup [15,\infty)$\\
B.$x \in (2,4) \cup [15,\infty)$\\
C.$x \in (2,4] \cup [15,\infty)$\\
D.$x \in [2,4) \cup [15,\infty)$\\
E.$x \in [2,4] \cup (15,\infty)$\\
F.$x \in (2,4) \cup (15,\infty)$\\
G.$x \in [2,4) \cup (15,\infty)$\\
H.$x \in (2,4] \cup (15,\infty)$
\testStop
\kluczStart
A
\kluczStop



\zadStart{Zadanie z Wikieł Z 1.62 a) moja wersja nr 200}

Rozwiązać nierówności $(x-2)(x-4)(x-16)\ge0$.
\zadStop
\rozwStart{Patryk Wirkus}{}
Miejsca zerowe naszego wielomianu to: $2, 4, 16$.\\
Wielomian jest stopnia nieparzystego, ponadto znak współczynnika przy\linebreak najwyższej potędze x jest dodatni.\\ W związku z tym wykres wielomianu zaczyna się od lewej strony poniżej osi OX. A więc $$x \in [2,4] \cup [16,\infty).$$
\rozwStop
\odpStart
$x \in [2,4] \cup [16,\infty)$
\odpStop
\testStart
A.$x \in [2,4] \cup [16,\infty)$\\
B.$x \in (2,4) \cup [16,\infty)$\\
C.$x \in (2,4] \cup [16,\infty)$\\
D.$x \in [2,4) \cup [16,\infty)$\\
E.$x \in [2,4] \cup (16,\infty)$\\
F.$x \in (2,4) \cup (16,\infty)$\\
G.$x \in [2,4) \cup (16,\infty)$\\
H.$x \in (2,4] \cup (16,\infty)$
\testStop
\kluczStart
A
\kluczStop



\zadStart{Zadanie z Wikieł Z 1.62 a) moja wersja nr 201}

Rozwiązać nierówności $(x-2)(x-4)(x-17)\ge0$.
\zadStop
\rozwStart{Patryk Wirkus}{}
Miejsca zerowe naszego wielomianu to: $2, 4, 17$.\\
Wielomian jest stopnia nieparzystego, ponadto znak współczynnika przy\linebreak najwyższej potędze x jest dodatni.\\ W związku z tym wykres wielomianu zaczyna się od lewej strony poniżej osi OX. A więc $$x \in [2,4] \cup [17,\infty).$$
\rozwStop
\odpStart
$x \in [2,4] \cup [17,\infty)$
\odpStop
\testStart
A.$x \in [2,4] \cup [17,\infty)$\\
B.$x \in (2,4) \cup [17,\infty)$\\
C.$x \in (2,4] \cup [17,\infty)$\\
D.$x \in [2,4) \cup [17,\infty)$\\
E.$x \in [2,4] \cup (17,\infty)$\\
F.$x \in (2,4) \cup (17,\infty)$\\
G.$x \in [2,4) \cup (17,\infty)$\\
H.$x \in (2,4] \cup (17,\infty)$
\testStop
\kluczStart
A
\kluczStop



\zadStart{Zadanie z Wikieł Z 1.62 a) moja wersja nr 202}

Rozwiązać nierówności $(x-2)(x-4)(x-18)\ge0$.
\zadStop
\rozwStart{Patryk Wirkus}{}
Miejsca zerowe naszego wielomianu to: $2, 4, 18$.\\
Wielomian jest stopnia nieparzystego, ponadto znak współczynnika przy\linebreak najwyższej potędze x jest dodatni.\\ W związku z tym wykres wielomianu zaczyna się od lewej strony poniżej osi OX. A więc $$x \in [2,4] \cup [18,\infty).$$
\rozwStop
\odpStart
$x \in [2,4] \cup [18,\infty)$
\odpStop
\testStart
A.$x \in [2,4] \cup [18,\infty)$\\
B.$x \in (2,4) \cup [18,\infty)$\\
C.$x \in (2,4] \cup [18,\infty)$\\
D.$x \in [2,4) \cup [18,\infty)$\\
E.$x \in [2,4] \cup (18,\infty)$\\
F.$x \in (2,4) \cup (18,\infty)$\\
G.$x \in [2,4) \cup (18,\infty)$\\
H.$x \in (2,4] \cup (18,\infty)$
\testStop
\kluczStart
A
\kluczStop



\zadStart{Zadanie z Wikieł Z 1.62 a) moja wersja nr 203}

Rozwiązać nierówności $(x-2)(x-4)(x-19)\ge0$.
\zadStop
\rozwStart{Patryk Wirkus}{}
Miejsca zerowe naszego wielomianu to: $2, 4, 19$.\\
Wielomian jest stopnia nieparzystego, ponadto znak współczynnika przy\linebreak najwyższej potędze x jest dodatni.\\ W związku z tym wykres wielomianu zaczyna się od lewej strony poniżej osi OX. A więc $$x \in [2,4] \cup [19,\infty).$$
\rozwStop
\odpStart
$x \in [2,4] \cup [19,\infty)$
\odpStop
\testStart
A.$x \in [2,4] \cup [19,\infty)$\\
B.$x \in (2,4) \cup [19,\infty)$\\
C.$x \in (2,4] \cup [19,\infty)$\\
D.$x \in [2,4) \cup [19,\infty)$\\
E.$x \in [2,4] \cup (19,\infty)$\\
F.$x \in (2,4) \cup (19,\infty)$\\
G.$x \in [2,4) \cup (19,\infty)$\\
H.$x \in (2,4] \cup (19,\infty)$
\testStop
\kluczStart
A
\kluczStop



\zadStart{Zadanie z Wikieł Z 1.62 a) moja wersja nr 204}

Rozwiązać nierówności $(x-2)(x-4)(x-20)\ge0$.
\zadStop
\rozwStart{Patryk Wirkus}{}
Miejsca zerowe naszego wielomianu to: $2, 4, 20$.\\
Wielomian jest stopnia nieparzystego, ponadto znak współczynnika przy\linebreak najwyższej potędze x jest dodatni.\\ W związku z tym wykres wielomianu zaczyna się od lewej strony poniżej osi OX. A więc $$x \in [2,4] \cup [20,\infty).$$
\rozwStop
\odpStart
$x \in [2,4] \cup [20,\infty)$
\odpStop
\testStart
A.$x \in [2,4] \cup [20,\infty)$\\
B.$x \in (2,4) \cup [20,\infty)$\\
C.$x \in (2,4] \cup [20,\infty)$\\
D.$x \in [2,4) \cup [20,\infty)$\\
E.$x \in [2,4] \cup (20,\infty)$\\
F.$x \in (2,4) \cup (20,\infty)$\\
G.$x \in [2,4) \cup (20,\infty)$\\
H.$x \in (2,4] \cup (20,\infty)$
\testStop
\kluczStart
A
\kluczStop



\zadStart{Zadanie z Wikieł Z 1.62 a) moja wersja nr 205}

Rozwiązać nierówności $(x-2)(x-5)(x-6)\ge0$.
\zadStop
\rozwStart{Patryk Wirkus}{}
Miejsca zerowe naszego wielomianu to: $2, 5, 6$.\\
Wielomian jest stopnia nieparzystego, ponadto znak współczynnika przy\linebreak najwyższej potędze x jest dodatni.\\ W związku z tym wykres wielomianu zaczyna się od lewej strony poniżej osi OX. A więc $$x \in [2,5] \cup [6,\infty).$$
\rozwStop
\odpStart
$x \in [2,5] \cup [6,\infty)$
\odpStop
\testStart
A.$x \in [2,5] \cup [6,\infty)$\\
B.$x \in (2,5) \cup [6,\infty)$\\
C.$x \in (2,5] \cup [6,\infty)$\\
D.$x \in [2,5) \cup [6,\infty)$\\
E.$x \in [2,5] \cup (6,\infty)$\\
F.$x \in (2,5) \cup (6,\infty)$\\
G.$x \in [2,5) \cup (6,\infty)$\\
H.$x \in (2,5] \cup (6,\infty)$
\testStop
\kluczStart
A
\kluczStop



\zadStart{Zadanie z Wikieł Z 1.62 a) moja wersja nr 206}

Rozwiązać nierówności $(x-2)(x-5)(x-7)\ge0$.
\zadStop
\rozwStart{Patryk Wirkus}{}
Miejsca zerowe naszego wielomianu to: $2, 5, 7$.\\
Wielomian jest stopnia nieparzystego, ponadto znak współczynnika przy\linebreak najwyższej potędze x jest dodatni.\\ W związku z tym wykres wielomianu zaczyna się od lewej strony poniżej osi OX. A więc $$x \in [2,5] \cup [7,\infty).$$
\rozwStop
\odpStart
$x \in [2,5] \cup [7,\infty)$
\odpStop
\testStart
A.$x \in [2,5] \cup [7,\infty)$\\
B.$x \in (2,5) \cup [7,\infty)$\\
C.$x \in (2,5] \cup [7,\infty)$\\
D.$x \in [2,5) \cup [7,\infty)$\\
E.$x \in [2,5] \cup (7,\infty)$\\
F.$x \in (2,5) \cup (7,\infty)$\\
G.$x \in [2,5) \cup (7,\infty)$\\
H.$x \in (2,5] \cup (7,\infty)$
\testStop
\kluczStart
A
\kluczStop



\zadStart{Zadanie z Wikieł Z 1.62 a) moja wersja nr 207}

Rozwiązać nierówności $(x-2)(x-5)(x-8)\ge0$.
\zadStop
\rozwStart{Patryk Wirkus}{}
Miejsca zerowe naszego wielomianu to: $2, 5, 8$.\\
Wielomian jest stopnia nieparzystego, ponadto znak współczynnika przy\linebreak najwyższej potędze x jest dodatni.\\ W związku z tym wykres wielomianu zaczyna się od lewej strony poniżej osi OX. A więc $$x \in [2,5] \cup [8,\infty).$$
\rozwStop
\odpStart
$x \in [2,5] \cup [8,\infty)$
\odpStop
\testStart
A.$x \in [2,5] \cup [8,\infty)$\\
B.$x \in (2,5) \cup [8,\infty)$\\
C.$x \in (2,5] \cup [8,\infty)$\\
D.$x \in [2,5) \cup [8,\infty)$\\
E.$x \in [2,5] \cup (8,\infty)$\\
F.$x \in (2,5) \cup (8,\infty)$\\
G.$x \in [2,5) \cup (8,\infty)$\\
H.$x \in (2,5] \cup (8,\infty)$
\testStop
\kluczStart
A
\kluczStop



\zadStart{Zadanie z Wikieł Z 1.62 a) moja wersja nr 208}

Rozwiązać nierówności $(x-2)(x-5)(x-9)\ge0$.
\zadStop
\rozwStart{Patryk Wirkus}{}
Miejsca zerowe naszego wielomianu to: $2, 5, 9$.\\
Wielomian jest stopnia nieparzystego, ponadto znak współczynnika przy\linebreak najwyższej potędze x jest dodatni.\\ W związku z tym wykres wielomianu zaczyna się od lewej strony poniżej osi OX. A więc $$x \in [2,5] \cup [9,\infty).$$
\rozwStop
\odpStart
$x \in [2,5] \cup [9,\infty)$
\odpStop
\testStart
A.$x \in [2,5] \cup [9,\infty)$\\
B.$x \in (2,5) \cup [9,\infty)$\\
C.$x \in (2,5] \cup [9,\infty)$\\
D.$x \in [2,5) \cup [9,\infty)$\\
E.$x \in [2,5] \cup (9,\infty)$\\
F.$x \in (2,5) \cup (9,\infty)$\\
G.$x \in [2,5) \cup (9,\infty)$\\
H.$x \in (2,5] \cup (9,\infty)$
\testStop
\kluczStart
A
\kluczStop



\zadStart{Zadanie z Wikieł Z 1.62 a) moja wersja nr 209}

Rozwiązać nierówności $(x-2)(x-5)(x-10)\ge0$.
\zadStop
\rozwStart{Patryk Wirkus}{}
Miejsca zerowe naszego wielomianu to: $2, 5, 10$.\\
Wielomian jest stopnia nieparzystego, ponadto znak współczynnika przy\linebreak najwyższej potędze x jest dodatni.\\ W związku z tym wykres wielomianu zaczyna się od lewej strony poniżej osi OX. A więc $$x \in [2,5] \cup [10,\infty).$$
\rozwStop
\odpStart
$x \in [2,5] \cup [10,\infty)$
\odpStop
\testStart
A.$x \in [2,5] \cup [10,\infty)$\\
B.$x \in (2,5) \cup [10,\infty)$\\
C.$x \in (2,5] \cup [10,\infty)$\\
D.$x \in [2,5) \cup [10,\infty)$\\
E.$x \in [2,5] \cup (10,\infty)$\\
F.$x \in (2,5) \cup (10,\infty)$\\
G.$x \in [2,5) \cup (10,\infty)$\\
H.$x \in (2,5] \cup (10,\infty)$
\testStop
\kluczStart
A
\kluczStop



\zadStart{Zadanie z Wikieł Z 1.62 a) moja wersja nr 210}

Rozwiązać nierówności $(x-2)(x-5)(x-11)\ge0$.
\zadStop
\rozwStart{Patryk Wirkus}{}
Miejsca zerowe naszego wielomianu to: $2, 5, 11$.\\
Wielomian jest stopnia nieparzystego, ponadto znak współczynnika przy\linebreak najwyższej potędze x jest dodatni.\\ W związku z tym wykres wielomianu zaczyna się od lewej strony poniżej osi OX. A więc $$x \in [2,5] \cup [11,\infty).$$
\rozwStop
\odpStart
$x \in [2,5] \cup [11,\infty)$
\odpStop
\testStart
A.$x \in [2,5] \cup [11,\infty)$\\
B.$x \in (2,5) \cup [11,\infty)$\\
C.$x \in (2,5] \cup [11,\infty)$\\
D.$x \in [2,5) \cup [11,\infty)$\\
E.$x \in [2,5] \cup (11,\infty)$\\
F.$x \in (2,5) \cup (11,\infty)$\\
G.$x \in [2,5) \cup (11,\infty)$\\
H.$x \in (2,5] \cup (11,\infty)$
\testStop
\kluczStart
A
\kluczStop



\zadStart{Zadanie z Wikieł Z 1.62 a) moja wersja nr 211}

Rozwiązać nierówności $(x-2)(x-5)(x-12)\ge0$.
\zadStop
\rozwStart{Patryk Wirkus}{}
Miejsca zerowe naszego wielomianu to: $2, 5, 12$.\\
Wielomian jest stopnia nieparzystego, ponadto znak współczynnika przy\linebreak najwyższej potędze x jest dodatni.\\ W związku z tym wykres wielomianu zaczyna się od lewej strony poniżej osi OX. A więc $$x \in [2,5] \cup [12,\infty).$$
\rozwStop
\odpStart
$x \in [2,5] \cup [12,\infty)$
\odpStop
\testStart
A.$x \in [2,5] \cup [12,\infty)$\\
B.$x \in (2,5) \cup [12,\infty)$\\
C.$x \in (2,5] \cup [12,\infty)$\\
D.$x \in [2,5) \cup [12,\infty)$\\
E.$x \in [2,5] \cup (12,\infty)$\\
F.$x \in (2,5) \cup (12,\infty)$\\
G.$x \in [2,5) \cup (12,\infty)$\\
H.$x \in (2,5] \cup (12,\infty)$
\testStop
\kluczStart
A
\kluczStop



\zadStart{Zadanie z Wikieł Z 1.62 a) moja wersja nr 212}

Rozwiązać nierówności $(x-2)(x-5)(x-13)\ge0$.
\zadStop
\rozwStart{Patryk Wirkus}{}
Miejsca zerowe naszego wielomianu to: $2, 5, 13$.\\
Wielomian jest stopnia nieparzystego, ponadto znak współczynnika przy\linebreak najwyższej potędze x jest dodatni.\\ W związku z tym wykres wielomianu zaczyna się od lewej strony poniżej osi OX. A więc $$x \in [2,5] \cup [13,\infty).$$
\rozwStop
\odpStart
$x \in [2,5] \cup [13,\infty)$
\odpStop
\testStart
A.$x \in [2,5] \cup [13,\infty)$\\
B.$x \in (2,5) \cup [13,\infty)$\\
C.$x \in (2,5] \cup [13,\infty)$\\
D.$x \in [2,5) \cup [13,\infty)$\\
E.$x \in [2,5] \cup (13,\infty)$\\
F.$x \in (2,5) \cup (13,\infty)$\\
G.$x \in [2,5) \cup (13,\infty)$\\
H.$x \in (2,5] \cup (13,\infty)$
\testStop
\kluczStart
A
\kluczStop



\zadStart{Zadanie z Wikieł Z 1.62 a) moja wersja nr 213}

Rozwiązać nierówności $(x-2)(x-5)(x-14)\ge0$.
\zadStop
\rozwStart{Patryk Wirkus}{}
Miejsca zerowe naszego wielomianu to: $2, 5, 14$.\\
Wielomian jest stopnia nieparzystego, ponadto znak współczynnika przy\linebreak najwyższej potędze x jest dodatni.\\ W związku z tym wykres wielomianu zaczyna się od lewej strony poniżej osi OX. A więc $$x \in [2,5] \cup [14,\infty).$$
\rozwStop
\odpStart
$x \in [2,5] \cup [14,\infty)$
\odpStop
\testStart
A.$x \in [2,5] \cup [14,\infty)$\\
B.$x \in (2,5) \cup [14,\infty)$\\
C.$x \in (2,5] \cup [14,\infty)$\\
D.$x \in [2,5) \cup [14,\infty)$\\
E.$x \in [2,5] \cup (14,\infty)$\\
F.$x \in (2,5) \cup (14,\infty)$\\
G.$x \in [2,5) \cup (14,\infty)$\\
H.$x \in (2,5] \cup (14,\infty)$
\testStop
\kluczStart
A
\kluczStop



\zadStart{Zadanie z Wikieł Z 1.62 a) moja wersja nr 214}

Rozwiązać nierówności $(x-2)(x-5)(x-15)\ge0$.
\zadStop
\rozwStart{Patryk Wirkus}{}
Miejsca zerowe naszego wielomianu to: $2, 5, 15$.\\
Wielomian jest stopnia nieparzystego, ponadto znak współczynnika przy\linebreak najwyższej potędze x jest dodatni.\\ W związku z tym wykres wielomianu zaczyna się od lewej strony poniżej osi OX. A więc $$x \in [2,5] \cup [15,\infty).$$
\rozwStop
\odpStart
$x \in [2,5] \cup [15,\infty)$
\odpStop
\testStart
A.$x \in [2,5] \cup [15,\infty)$\\
B.$x \in (2,5) \cup [15,\infty)$\\
C.$x \in (2,5] \cup [15,\infty)$\\
D.$x \in [2,5) \cup [15,\infty)$\\
E.$x \in [2,5] \cup (15,\infty)$\\
F.$x \in (2,5) \cup (15,\infty)$\\
G.$x \in [2,5) \cup (15,\infty)$\\
H.$x \in (2,5] \cup (15,\infty)$
\testStop
\kluczStart
A
\kluczStop



\zadStart{Zadanie z Wikieł Z 1.62 a) moja wersja nr 215}

Rozwiązać nierówności $(x-2)(x-5)(x-16)\ge0$.
\zadStop
\rozwStart{Patryk Wirkus}{}
Miejsca zerowe naszego wielomianu to: $2, 5, 16$.\\
Wielomian jest stopnia nieparzystego, ponadto znak współczynnika przy\linebreak najwyższej potędze x jest dodatni.\\ W związku z tym wykres wielomianu zaczyna się od lewej strony poniżej osi OX. A więc $$x \in [2,5] \cup [16,\infty).$$
\rozwStop
\odpStart
$x \in [2,5] \cup [16,\infty)$
\odpStop
\testStart
A.$x \in [2,5] \cup [16,\infty)$\\
B.$x \in (2,5) \cup [16,\infty)$\\
C.$x \in (2,5] \cup [16,\infty)$\\
D.$x \in [2,5) \cup [16,\infty)$\\
E.$x \in [2,5] \cup (16,\infty)$\\
F.$x \in (2,5) \cup (16,\infty)$\\
G.$x \in [2,5) \cup (16,\infty)$\\
H.$x \in (2,5] \cup (16,\infty)$
\testStop
\kluczStart
A
\kluczStop



\zadStart{Zadanie z Wikieł Z 1.62 a) moja wersja nr 216}

Rozwiązać nierówności $(x-2)(x-5)(x-17)\ge0$.
\zadStop
\rozwStart{Patryk Wirkus}{}
Miejsca zerowe naszego wielomianu to: $2, 5, 17$.\\
Wielomian jest stopnia nieparzystego, ponadto znak współczynnika przy\linebreak najwyższej potędze x jest dodatni.\\ W związku z tym wykres wielomianu zaczyna się od lewej strony poniżej osi OX. A więc $$x \in [2,5] \cup [17,\infty).$$
\rozwStop
\odpStart
$x \in [2,5] \cup [17,\infty)$
\odpStop
\testStart
A.$x \in [2,5] \cup [17,\infty)$\\
B.$x \in (2,5) \cup [17,\infty)$\\
C.$x \in (2,5] \cup [17,\infty)$\\
D.$x \in [2,5) \cup [17,\infty)$\\
E.$x \in [2,5] \cup (17,\infty)$\\
F.$x \in (2,5) \cup (17,\infty)$\\
G.$x \in [2,5) \cup (17,\infty)$\\
H.$x \in (2,5] \cup (17,\infty)$
\testStop
\kluczStart
A
\kluczStop



\zadStart{Zadanie z Wikieł Z 1.62 a) moja wersja nr 217}

Rozwiązać nierówności $(x-2)(x-5)(x-18)\ge0$.
\zadStop
\rozwStart{Patryk Wirkus}{}
Miejsca zerowe naszego wielomianu to: $2, 5, 18$.\\
Wielomian jest stopnia nieparzystego, ponadto znak współczynnika przy\linebreak najwyższej potędze x jest dodatni.\\ W związku z tym wykres wielomianu zaczyna się od lewej strony poniżej osi OX. A więc $$x \in [2,5] \cup [18,\infty).$$
\rozwStop
\odpStart
$x \in [2,5] \cup [18,\infty)$
\odpStop
\testStart
A.$x \in [2,5] \cup [18,\infty)$\\
B.$x \in (2,5) \cup [18,\infty)$\\
C.$x \in (2,5] \cup [18,\infty)$\\
D.$x \in [2,5) \cup [18,\infty)$\\
E.$x \in [2,5] \cup (18,\infty)$\\
F.$x \in (2,5) \cup (18,\infty)$\\
G.$x \in [2,5) \cup (18,\infty)$\\
H.$x \in (2,5] \cup (18,\infty)$
\testStop
\kluczStart
A
\kluczStop



\zadStart{Zadanie z Wikieł Z 1.62 a) moja wersja nr 218}

Rozwiązać nierówności $(x-2)(x-5)(x-19)\ge0$.
\zadStop
\rozwStart{Patryk Wirkus}{}
Miejsca zerowe naszego wielomianu to: $2, 5, 19$.\\
Wielomian jest stopnia nieparzystego, ponadto znak współczynnika przy\linebreak najwyższej potędze x jest dodatni.\\ W związku z tym wykres wielomianu zaczyna się od lewej strony poniżej osi OX. A więc $$x \in [2,5] \cup [19,\infty).$$
\rozwStop
\odpStart
$x \in [2,5] \cup [19,\infty)$
\odpStop
\testStart
A.$x \in [2,5] \cup [19,\infty)$\\
B.$x \in (2,5) \cup [19,\infty)$\\
C.$x \in (2,5] \cup [19,\infty)$\\
D.$x \in [2,5) \cup [19,\infty)$\\
E.$x \in [2,5] \cup (19,\infty)$\\
F.$x \in (2,5) \cup (19,\infty)$\\
G.$x \in [2,5) \cup (19,\infty)$\\
H.$x \in (2,5] \cup (19,\infty)$
\testStop
\kluczStart
A
\kluczStop



\zadStart{Zadanie z Wikieł Z 1.62 a) moja wersja nr 219}

Rozwiązać nierówności $(x-2)(x-5)(x-20)\ge0$.
\zadStop
\rozwStart{Patryk Wirkus}{}
Miejsca zerowe naszego wielomianu to: $2, 5, 20$.\\
Wielomian jest stopnia nieparzystego, ponadto znak współczynnika przy\linebreak najwyższej potędze x jest dodatni.\\ W związku z tym wykres wielomianu zaczyna się od lewej strony poniżej osi OX. A więc $$x \in [2,5] \cup [20,\infty).$$
\rozwStop
\odpStart
$x \in [2,5] \cup [20,\infty)$
\odpStop
\testStart
A.$x \in [2,5] \cup [20,\infty)$\\
B.$x \in (2,5) \cup [20,\infty)$\\
C.$x \in (2,5] \cup [20,\infty)$\\
D.$x \in [2,5) \cup [20,\infty)$\\
E.$x \in [2,5] \cup (20,\infty)$\\
F.$x \in (2,5) \cup (20,\infty)$\\
G.$x \in [2,5) \cup (20,\infty)$\\
H.$x \in (2,5] \cup (20,\infty)$
\testStop
\kluczStart
A
\kluczStop



\zadStart{Zadanie z Wikieł Z 1.62 a) moja wersja nr 220}

Rozwiązać nierówności $(x-2)(x-6)(x-7)\ge0$.
\zadStop
\rozwStart{Patryk Wirkus}{}
Miejsca zerowe naszego wielomianu to: $2, 6, 7$.\\
Wielomian jest stopnia nieparzystego, ponadto znak współczynnika przy\linebreak najwyższej potędze x jest dodatni.\\ W związku z tym wykres wielomianu zaczyna się od lewej strony poniżej osi OX. A więc $$x \in [2,6] \cup [7,\infty).$$
\rozwStop
\odpStart
$x \in [2,6] \cup [7,\infty)$
\odpStop
\testStart
A.$x \in [2,6] \cup [7,\infty)$\\
B.$x \in (2,6) \cup [7,\infty)$\\
C.$x \in (2,6] \cup [7,\infty)$\\
D.$x \in [2,6) \cup [7,\infty)$\\
E.$x \in [2,6] \cup (7,\infty)$\\
F.$x \in (2,6) \cup (7,\infty)$\\
G.$x \in [2,6) \cup (7,\infty)$\\
H.$x \in (2,6] \cup (7,\infty)$
\testStop
\kluczStart
A
\kluczStop



\zadStart{Zadanie z Wikieł Z 1.62 a) moja wersja nr 221}

Rozwiązać nierówności $(x-2)(x-6)(x-8)\ge0$.
\zadStop
\rozwStart{Patryk Wirkus}{}
Miejsca zerowe naszego wielomianu to: $2, 6, 8$.\\
Wielomian jest stopnia nieparzystego, ponadto znak współczynnika przy\linebreak najwyższej potędze x jest dodatni.\\ W związku z tym wykres wielomianu zaczyna się od lewej strony poniżej osi OX. A więc $$x \in [2,6] \cup [8,\infty).$$
\rozwStop
\odpStart
$x \in [2,6] \cup [8,\infty)$
\odpStop
\testStart
A.$x \in [2,6] \cup [8,\infty)$\\
B.$x \in (2,6) \cup [8,\infty)$\\
C.$x \in (2,6] \cup [8,\infty)$\\
D.$x \in [2,6) \cup [8,\infty)$\\
E.$x \in [2,6] \cup (8,\infty)$\\
F.$x \in (2,6) \cup (8,\infty)$\\
G.$x \in [2,6) \cup (8,\infty)$\\
H.$x \in (2,6] \cup (8,\infty)$
\testStop
\kluczStart
A
\kluczStop



\zadStart{Zadanie z Wikieł Z 1.62 a) moja wersja nr 222}

Rozwiązać nierówności $(x-2)(x-6)(x-9)\ge0$.
\zadStop
\rozwStart{Patryk Wirkus}{}
Miejsca zerowe naszego wielomianu to: $2, 6, 9$.\\
Wielomian jest stopnia nieparzystego, ponadto znak współczynnika przy\linebreak najwyższej potędze x jest dodatni.\\ W związku z tym wykres wielomianu zaczyna się od lewej strony poniżej osi OX. A więc $$x \in [2,6] \cup [9,\infty).$$
\rozwStop
\odpStart
$x \in [2,6] \cup [9,\infty)$
\odpStop
\testStart
A.$x \in [2,6] \cup [9,\infty)$\\
B.$x \in (2,6) \cup [9,\infty)$\\
C.$x \in (2,6] \cup [9,\infty)$\\
D.$x \in [2,6) \cup [9,\infty)$\\
E.$x \in [2,6] \cup (9,\infty)$\\
F.$x \in (2,6) \cup (9,\infty)$\\
G.$x \in [2,6) \cup (9,\infty)$\\
H.$x \in (2,6] \cup (9,\infty)$
\testStop
\kluczStart
A
\kluczStop



\zadStart{Zadanie z Wikieł Z 1.62 a) moja wersja nr 223}

Rozwiązać nierówności $(x-2)(x-6)(x-10)\ge0$.
\zadStop
\rozwStart{Patryk Wirkus}{}
Miejsca zerowe naszego wielomianu to: $2, 6, 10$.\\
Wielomian jest stopnia nieparzystego, ponadto znak współczynnika przy\linebreak najwyższej potędze x jest dodatni.\\ W związku z tym wykres wielomianu zaczyna się od lewej strony poniżej osi OX. A więc $$x \in [2,6] \cup [10,\infty).$$
\rozwStop
\odpStart
$x \in [2,6] \cup [10,\infty)$
\odpStop
\testStart
A.$x \in [2,6] \cup [10,\infty)$\\
B.$x \in (2,6) \cup [10,\infty)$\\
C.$x \in (2,6] \cup [10,\infty)$\\
D.$x \in [2,6) \cup [10,\infty)$\\
E.$x \in [2,6] \cup (10,\infty)$\\
F.$x \in (2,6) \cup (10,\infty)$\\
G.$x \in [2,6) \cup (10,\infty)$\\
H.$x \in (2,6] \cup (10,\infty)$
\testStop
\kluczStart
A
\kluczStop



\zadStart{Zadanie z Wikieł Z 1.62 a) moja wersja nr 224}

Rozwiązać nierówności $(x-2)(x-6)(x-11)\ge0$.
\zadStop
\rozwStart{Patryk Wirkus}{}
Miejsca zerowe naszego wielomianu to: $2, 6, 11$.\\
Wielomian jest stopnia nieparzystego, ponadto znak współczynnika przy\linebreak najwyższej potędze x jest dodatni.\\ W związku z tym wykres wielomianu zaczyna się od lewej strony poniżej osi OX. A więc $$x \in [2,6] \cup [11,\infty).$$
\rozwStop
\odpStart
$x \in [2,6] \cup [11,\infty)$
\odpStop
\testStart
A.$x \in [2,6] \cup [11,\infty)$\\
B.$x \in (2,6) \cup [11,\infty)$\\
C.$x \in (2,6] \cup [11,\infty)$\\
D.$x \in [2,6) \cup [11,\infty)$\\
E.$x \in [2,6] \cup (11,\infty)$\\
F.$x \in (2,6) \cup (11,\infty)$\\
G.$x \in [2,6) \cup (11,\infty)$\\
H.$x \in (2,6] \cup (11,\infty)$
\testStop
\kluczStart
A
\kluczStop



\zadStart{Zadanie z Wikieł Z 1.62 a) moja wersja nr 225}

Rozwiązać nierówności $(x-2)(x-6)(x-12)\ge0$.
\zadStop
\rozwStart{Patryk Wirkus}{}
Miejsca zerowe naszego wielomianu to: $2, 6, 12$.\\
Wielomian jest stopnia nieparzystego, ponadto znak współczynnika przy\linebreak najwyższej potędze x jest dodatni.\\ W związku z tym wykres wielomianu zaczyna się od lewej strony poniżej osi OX. A więc $$x \in [2,6] \cup [12,\infty).$$
\rozwStop
\odpStart
$x \in [2,6] \cup [12,\infty)$
\odpStop
\testStart
A.$x \in [2,6] \cup [12,\infty)$\\
B.$x \in (2,6) \cup [12,\infty)$\\
C.$x \in (2,6] \cup [12,\infty)$\\
D.$x \in [2,6) \cup [12,\infty)$\\
E.$x \in [2,6] \cup (12,\infty)$\\
F.$x \in (2,6) \cup (12,\infty)$\\
G.$x \in [2,6) \cup (12,\infty)$\\
H.$x \in (2,6] \cup (12,\infty)$
\testStop
\kluczStart
A
\kluczStop



\zadStart{Zadanie z Wikieł Z 1.62 a) moja wersja nr 226}

Rozwiązać nierówności $(x-2)(x-6)(x-13)\ge0$.
\zadStop
\rozwStart{Patryk Wirkus}{}
Miejsca zerowe naszego wielomianu to: $2, 6, 13$.\\
Wielomian jest stopnia nieparzystego, ponadto znak współczynnika przy\linebreak najwyższej potędze x jest dodatni.\\ W związku z tym wykres wielomianu zaczyna się od lewej strony poniżej osi OX. A więc $$x \in [2,6] \cup [13,\infty).$$
\rozwStop
\odpStart
$x \in [2,6] \cup [13,\infty)$
\odpStop
\testStart
A.$x \in [2,6] \cup [13,\infty)$\\
B.$x \in (2,6) \cup [13,\infty)$\\
C.$x \in (2,6] \cup [13,\infty)$\\
D.$x \in [2,6) \cup [13,\infty)$\\
E.$x \in [2,6] \cup (13,\infty)$\\
F.$x \in (2,6) \cup (13,\infty)$\\
G.$x \in [2,6) \cup (13,\infty)$\\
H.$x \in (2,6] \cup (13,\infty)$
\testStop
\kluczStart
A
\kluczStop



\zadStart{Zadanie z Wikieł Z 1.62 a) moja wersja nr 227}

Rozwiązać nierówności $(x-2)(x-6)(x-14)\ge0$.
\zadStop
\rozwStart{Patryk Wirkus}{}
Miejsca zerowe naszego wielomianu to: $2, 6, 14$.\\
Wielomian jest stopnia nieparzystego, ponadto znak współczynnika przy\linebreak najwyższej potędze x jest dodatni.\\ W związku z tym wykres wielomianu zaczyna się od lewej strony poniżej osi OX. A więc $$x \in [2,6] \cup [14,\infty).$$
\rozwStop
\odpStart
$x \in [2,6] \cup [14,\infty)$
\odpStop
\testStart
A.$x \in [2,6] \cup [14,\infty)$\\
B.$x \in (2,6) \cup [14,\infty)$\\
C.$x \in (2,6] \cup [14,\infty)$\\
D.$x \in [2,6) \cup [14,\infty)$\\
E.$x \in [2,6] \cup (14,\infty)$\\
F.$x \in (2,6) \cup (14,\infty)$\\
G.$x \in [2,6) \cup (14,\infty)$\\
H.$x \in (2,6] \cup (14,\infty)$
\testStop
\kluczStart
A
\kluczStop



\zadStart{Zadanie z Wikieł Z 1.62 a) moja wersja nr 228}

Rozwiązać nierówności $(x-2)(x-6)(x-15)\ge0$.
\zadStop
\rozwStart{Patryk Wirkus}{}
Miejsca zerowe naszego wielomianu to: $2, 6, 15$.\\
Wielomian jest stopnia nieparzystego, ponadto znak współczynnika przy\linebreak najwyższej potędze x jest dodatni.\\ W związku z tym wykres wielomianu zaczyna się od lewej strony poniżej osi OX. A więc $$x \in [2,6] \cup [15,\infty).$$
\rozwStop
\odpStart
$x \in [2,6] \cup [15,\infty)$
\odpStop
\testStart
A.$x \in [2,6] \cup [15,\infty)$\\
B.$x \in (2,6) \cup [15,\infty)$\\
C.$x \in (2,6] \cup [15,\infty)$\\
D.$x \in [2,6) \cup [15,\infty)$\\
E.$x \in [2,6] \cup (15,\infty)$\\
F.$x \in (2,6) \cup (15,\infty)$\\
G.$x \in [2,6) \cup (15,\infty)$\\
H.$x \in (2,6] \cup (15,\infty)$
\testStop
\kluczStart
A
\kluczStop



\zadStart{Zadanie z Wikieł Z 1.62 a) moja wersja nr 229}

Rozwiązać nierówności $(x-2)(x-6)(x-16)\ge0$.
\zadStop
\rozwStart{Patryk Wirkus}{}
Miejsca zerowe naszego wielomianu to: $2, 6, 16$.\\
Wielomian jest stopnia nieparzystego, ponadto znak współczynnika przy\linebreak najwyższej potędze x jest dodatni.\\ W związku z tym wykres wielomianu zaczyna się od lewej strony poniżej osi OX. A więc $$x \in [2,6] \cup [16,\infty).$$
\rozwStop
\odpStart
$x \in [2,6] \cup [16,\infty)$
\odpStop
\testStart
A.$x \in [2,6] \cup [16,\infty)$\\
B.$x \in (2,6) \cup [16,\infty)$\\
C.$x \in (2,6] \cup [16,\infty)$\\
D.$x \in [2,6) \cup [16,\infty)$\\
E.$x \in [2,6] \cup (16,\infty)$\\
F.$x \in (2,6) \cup (16,\infty)$\\
G.$x \in [2,6) \cup (16,\infty)$\\
H.$x \in (2,6] \cup (16,\infty)$
\testStop
\kluczStart
A
\kluczStop



\zadStart{Zadanie z Wikieł Z 1.62 a) moja wersja nr 230}

Rozwiązać nierówności $(x-2)(x-6)(x-17)\ge0$.
\zadStop
\rozwStart{Patryk Wirkus}{}
Miejsca zerowe naszego wielomianu to: $2, 6, 17$.\\
Wielomian jest stopnia nieparzystego, ponadto znak współczynnika przy\linebreak najwyższej potędze x jest dodatni.\\ W związku z tym wykres wielomianu zaczyna się od lewej strony poniżej osi OX. A więc $$x \in [2,6] \cup [17,\infty).$$
\rozwStop
\odpStart
$x \in [2,6] \cup [17,\infty)$
\odpStop
\testStart
A.$x \in [2,6] \cup [17,\infty)$\\
B.$x \in (2,6) \cup [17,\infty)$\\
C.$x \in (2,6] \cup [17,\infty)$\\
D.$x \in [2,6) \cup [17,\infty)$\\
E.$x \in [2,6] \cup (17,\infty)$\\
F.$x \in (2,6) \cup (17,\infty)$\\
G.$x \in [2,6) \cup (17,\infty)$\\
H.$x \in (2,6] \cup (17,\infty)$
\testStop
\kluczStart
A
\kluczStop



\zadStart{Zadanie z Wikieł Z 1.62 a) moja wersja nr 231}

Rozwiązać nierówności $(x-2)(x-6)(x-18)\ge0$.
\zadStop
\rozwStart{Patryk Wirkus}{}
Miejsca zerowe naszego wielomianu to: $2, 6, 18$.\\
Wielomian jest stopnia nieparzystego, ponadto znak współczynnika przy\linebreak najwyższej potędze x jest dodatni.\\ W związku z tym wykres wielomianu zaczyna się od lewej strony poniżej osi OX. A więc $$x \in [2,6] \cup [18,\infty).$$
\rozwStop
\odpStart
$x \in [2,6] \cup [18,\infty)$
\odpStop
\testStart
A.$x \in [2,6] \cup [18,\infty)$\\
B.$x \in (2,6) \cup [18,\infty)$\\
C.$x \in (2,6] \cup [18,\infty)$\\
D.$x \in [2,6) \cup [18,\infty)$\\
E.$x \in [2,6] \cup (18,\infty)$\\
F.$x \in (2,6) \cup (18,\infty)$\\
G.$x \in [2,6) \cup (18,\infty)$\\
H.$x \in (2,6] \cup (18,\infty)$
\testStop
\kluczStart
A
\kluczStop



\zadStart{Zadanie z Wikieł Z 1.62 a) moja wersja nr 232}

Rozwiązać nierówności $(x-2)(x-6)(x-19)\ge0$.
\zadStop
\rozwStart{Patryk Wirkus}{}
Miejsca zerowe naszego wielomianu to: $2, 6, 19$.\\
Wielomian jest stopnia nieparzystego, ponadto znak współczynnika przy\linebreak najwyższej potędze x jest dodatni.\\ W związku z tym wykres wielomianu zaczyna się od lewej strony poniżej osi OX. A więc $$x \in [2,6] \cup [19,\infty).$$
\rozwStop
\odpStart
$x \in [2,6] \cup [19,\infty)$
\odpStop
\testStart
A.$x \in [2,6] \cup [19,\infty)$\\
B.$x \in (2,6) \cup [19,\infty)$\\
C.$x \in (2,6] \cup [19,\infty)$\\
D.$x \in [2,6) \cup [19,\infty)$\\
E.$x \in [2,6] \cup (19,\infty)$\\
F.$x \in (2,6) \cup (19,\infty)$\\
G.$x \in [2,6) \cup (19,\infty)$\\
H.$x \in (2,6] \cup (19,\infty)$
\testStop
\kluczStart
A
\kluczStop



\zadStart{Zadanie z Wikieł Z 1.62 a) moja wersja nr 233}

Rozwiązać nierówności $(x-2)(x-6)(x-20)\ge0$.
\zadStop
\rozwStart{Patryk Wirkus}{}
Miejsca zerowe naszego wielomianu to: $2, 6, 20$.\\
Wielomian jest stopnia nieparzystego, ponadto znak współczynnika przy\linebreak najwyższej potędze x jest dodatni.\\ W związku z tym wykres wielomianu zaczyna się od lewej strony poniżej osi OX. A więc $$x \in [2,6] \cup [20,\infty).$$
\rozwStop
\odpStart
$x \in [2,6] \cup [20,\infty)$
\odpStop
\testStart
A.$x \in [2,6] \cup [20,\infty)$\\
B.$x \in (2,6) \cup [20,\infty)$\\
C.$x \in (2,6] \cup [20,\infty)$\\
D.$x \in [2,6) \cup [20,\infty)$\\
E.$x \in [2,6] \cup (20,\infty)$\\
F.$x \in (2,6) \cup (20,\infty)$\\
G.$x \in [2,6) \cup (20,\infty)$\\
H.$x \in (2,6] \cup (20,\infty)$
\testStop
\kluczStart
A
\kluczStop



\zadStart{Zadanie z Wikieł Z 1.62 a) moja wersja nr 234}

Rozwiązać nierówności $(x-2)(x-7)(x-8)\ge0$.
\zadStop
\rozwStart{Patryk Wirkus}{}
Miejsca zerowe naszego wielomianu to: $2, 7, 8$.\\
Wielomian jest stopnia nieparzystego, ponadto znak współczynnika przy\linebreak najwyższej potędze x jest dodatni.\\ W związku z tym wykres wielomianu zaczyna się od lewej strony poniżej osi OX. A więc $$x \in [2,7] \cup [8,\infty).$$
\rozwStop
\odpStart
$x \in [2,7] \cup [8,\infty)$
\odpStop
\testStart
A.$x \in [2,7] \cup [8,\infty)$\\
B.$x \in (2,7) \cup [8,\infty)$\\
C.$x \in (2,7] \cup [8,\infty)$\\
D.$x \in [2,7) \cup [8,\infty)$\\
E.$x \in [2,7] \cup (8,\infty)$\\
F.$x \in (2,7) \cup (8,\infty)$\\
G.$x \in [2,7) \cup (8,\infty)$\\
H.$x \in (2,7] \cup (8,\infty)$
\testStop
\kluczStart
A
\kluczStop



\zadStart{Zadanie z Wikieł Z 1.62 a) moja wersja nr 235}

Rozwiązać nierówności $(x-2)(x-7)(x-9)\ge0$.
\zadStop
\rozwStart{Patryk Wirkus}{}
Miejsca zerowe naszego wielomianu to: $2, 7, 9$.\\
Wielomian jest stopnia nieparzystego, ponadto znak współczynnika przy\linebreak najwyższej potędze x jest dodatni.\\ W związku z tym wykres wielomianu zaczyna się od lewej strony poniżej osi OX. A więc $$x \in [2,7] \cup [9,\infty).$$
\rozwStop
\odpStart
$x \in [2,7] \cup [9,\infty)$
\odpStop
\testStart
A.$x \in [2,7] \cup [9,\infty)$\\
B.$x \in (2,7) \cup [9,\infty)$\\
C.$x \in (2,7] \cup [9,\infty)$\\
D.$x \in [2,7) \cup [9,\infty)$\\
E.$x \in [2,7] \cup (9,\infty)$\\
F.$x \in (2,7) \cup (9,\infty)$\\
G.$x \in [2,7) \cup (9,\infty)$\\
H.$x \in (2,7] \cup (9,\infty)$
\testStop
\kluczStart
A
\kluczStop



\zadStart{Zadanie z Wikieł Z 1.62 a) moja wersja nr 236}

Rozwiązać nierówności $(x-2)(x-7)(x-10)\ge0$.
\zadStop
\rozwStart{Patryk Wirkus}{}
Miejsca zerowe naszego wielomianu to: $2, 7, 10$.\\
Wielomian jest stopnia nieparzystego, ponadto znak współczynnika przy\linebreak najwyższej potędze x jest dodatni.\\ W związku z tym wykres wielomianu zaczyna się od lewej strony poniżej osi OX. A więc $$x \in [2,7] \cup [10,\infty).$$
\rozwStop
\odpStart
$x \in [2,7] \cup [10,\infty)$
\odpStop
\testStart
A.$x \in [2,7] \cup [10,\infty)$\\
B.$x \in (2,7) \cup [10,\infty)$\\
C.$x \in (2,7] \cup [10,\infty)$\\
D.$x \in [2,7) \cup [10,\infty)$\\
E.$x \in [2,7] \cup (10,\infty)$\\
F.$x \in (2,7) \cup (10,\infty)$\\
G.$x \in [2,7) \cup (10,\infty)$\\
H.$x \in (2,7] \cup (10,\infty)$
\testStop
\kluczStart
A
\kluczStop



\zadStart{Zadanie z Wikieł Z 1.62 a) moja wersja nr 237}

Rozwiązać nierówności $(x-2)(x-7)(x-11)\ge0$.
\zadStop
\rozwStart{Patryk Wirkus}{}
Miejsca zerowe naszego wielomianu to: $2, 7, 11$.\\
Wielomian jest stopnia nieparzystego, ponadto znak współczynnika przy\linebreak najwyższej potędze x jest dodatni.\\ W związku z tym wykres wielomianu zaczyna się od lewej strony poniżej osi OX. A więc $$x \in [2,7] \cup [11,\infty).$$
\rozwStop
\odpStart
$x \in [2,7] \cup [11,\infty)$
\odpStop
\testStart
A.$x \in [2,7] \cup [11,\infty)$\\
B.$x \in (2,7) \cup [11,\infty)$\\
C.$x \in (2,7] \cup [11,\infty)$\\
D.$x \in [2,7) \cup [11,\infty)$\\
E.$x \in [2,7] \cup (11,\infty)$\\
F.$x \in (2,7) \cup (11,\infty)$\\
G.$x \in [2,7) \cup (11,\infty)$\\
H.$x \in (2,7] \cup (11,\infty)$
\testStop
\kluczStart
A
\kluczStop



\zadStart{Zadanie z Wikieł Z 1.62 a) moja wersja nr 238}

Rozwiązać nierówności $(x-2)(x-7)(x-12)\ge0$.
\zadStop
\rozwStart{Patryk Wirkus}{}
Miejsca zerowe naszego wielomianu to: $2, 7, 12$.\\
Wielomian jest stopnia nieparzystego, ponadto znak współczynnika przy\linebreak najwyższej potędze x jest dodatni.\\ W związku z tym wykres wielomianu zaczyna się od lewej strony poniżej osi OX. A więc $$x \in [2,7] \cup [12,\infty).$$
\rozwStop
\odpStart
$x \in [2,7] \cup [12,\infty)$
\odpStop
\testStart
A.$x \in [2,7] \cup [12,\infty)$\\
B.$x \in (2,7) \cup [12,\infty)$\\
C.$x \in (2,7] \cup [12,\infty)$\\
D.$x \in [2,7) \cup [12,\infty)$\\
E.$x \in [2,7] \cup (12,\infty)$\\
F.$x \in (2,7) \cup (12,\infty)$\\
G.$x \in [2,7) \cup (12,\infty)$\\
H.$x \in (2,7] \cup (12,\infty)$
\testStop
\kluczStart
A
\kluczStop



\zadStart{Zadanie z Wikieł Z 1.62 a) moja wersja nr 239}

Rozwiązać nierówności $(x-2)(x-7)(x-13)\ge0$.
\zadStop
\rozwStart{Patryk Wirkus}{}
Miejsca zerowe naszego wielomianu to: $2, 7, 13$.\\
Wielomian jest stopnia nieparzystego, ponadto znak współczynnika przy\linebreak najwyższej potędze x jest dodatni.\\ W związku z tym wykres wielomianu zaczyna się od lewej strony poniżej osi OX. A więc $$x \in [2,7] \cup [13,\infty).$$
\rozwStop
\odpStart
$x \in [2,7] \cup [13,\infty)$
\odpStop
\testStart
A.$x \in [2,7] \cup [13,\infty)$\\
B.$x \in (2,7) \cup [13,\infty)$\\
C.$x \in (2,7] \cup [13,\infty)$\\
D.$x \in [2,7) \cup [13,\infty)$\\
E.$x \in [2,7] \cup (13,\infty)$\\
F.$x \in (2,7) \cup (13,\infty)$\\
G.$x \in [2,7) \cup (13,\infty)$\\
H.$x \in (2,7] \cup (13,\infty)$
\testStop
\kluczStart
A
\kluczStop



\zadStart{Zadanie z Wikieł Z 1.62 a) moja wersja nr 240}

Rozwiązać nierówności $(x-2)(x-7)(x-14)\ge0$.
\zadStop
\rozwStart{Patryk Wirkus}{}
Miejsca zerowe naszego wielomianu to: $2, 7, 14$.\\
Wielomian jest stopnia nieparzystego, ponadto znak współczynnika przy\linebreak najwyższej potędze x jest dodatni.\\ W związku z tym wykres wielomianu zaczyna się od lewej strony poniżej osi OX. A więc $$x \in [2,7] \cup [14,\infty).$$
\rozwStop
\odpStart
$x \in [2,7] \cup [14,\infty)$
\odpStop
\testStart
A.$x \in [2,7] \cup [14,\infty)$\\
B.$x \in (2,7) \cup [14,\infty)$\\
C.$x \in (2,7] \cup [14,\infty)$\\
D.$x \in [2,7) \cup [14,\infty)$\\
E.$x \in [2,7] \cup (14,\infty)$\\
F.$x \in (2,7) \cup (14,\infty)$\\
G.$x \in [2,7) \cup (14,\infty)$\\
H.$x \in (2,7] \cup (14,\infty)$
\testStop
\kluczStart
A
\kluczStop



\zadStart{Zadanie z Wikieł Z 1.62 a) moja wersja nr 241}

Rozwiązać nierówności $(x-2)(x-7)(x-15)\ge0$.
\zadStop
\rozwStart{Patryk Wirkus}{}
Miejsca zerowe naszego wielomianu to: $2, 7, 15$.\\
Wielomian jest stopnia nieparzystego, ponadto znak współczynnika przy\linebreak najwyższej potędze x jest dodatni.\\ W związku z tym wykres wielomianu zaczyna się od lewej strony poniżej osi OX. A więc $$x \in [2,7] \cup [15,\infty).$$
\rozwStop
\odpStart
$x \in [2,7] \cup [15,\infty)$
\odpStop
\testStart
A.$x \in [2,7] \cup [15,\infty)$\\
B.$x \in (2,7) \cup [15,\infty)$\\
C.$x \in (2,7] \cup [15,\infty)$\\
D.$x \in [2,7) \cup [15,\infty)$\\
E.$x \in [2,7] \cup (15,\infty)$\\
F.$x \in (2,7) \cup (15,\infty)$\\
G.$x \in [2,7) \cup (15,\infty)$\\
H.$x \in (2,7] \cup (15,\infty)$
\testStop
\kluczStart
A
\kluczStop



\zadStart{Zadanie z Wikieł Z 1.62 a) moja wersja nr 242}

Rozwiązać nierówności $(x-2)(x-7)(x-16)\ge0$.
\zadStop
\rozwStart{Patryk Wirkus}{}
Miejsca zerowe naszego wielomianu to: $2, 7, 16$.\\
Wielomian jest stopnia nieparzystego, ponadto znak współczynnika przy\linebreak najwyższej potędze x jest dodatni.\\ W związku z tym wykres wielomianu zaczyna się od lewej strony poniżej osi OX. A więc $$x \in [2,7] \cup [16,\infty).$$
\rozwStop
\odpStart
$x \in [2,7] \cup [16,\infty)$
\odpStop
\testStart
A.$x \in [2,7] \cup [16,\infty)$\\
B.$x \in (2,7) \cup [16,\infty)$\\
C.$x \in (2,7] \cup [16,\infty)$\\
D.$x \in [2,7) \cup [16,\infty)$\\
E.$x \in [2,7] \cup (16,\infty)$\\
F.$x \in (2,7) \cup (16,\infty)$\\
G.$x \in [2,7) \cup (16,\infty)$\\
H.$x \in (2,7] \cup (16,\infty)$
\testStop
\kluczStart
A
\kluczStop



\zadStart{Zadanie z Wikieł Z 1.62 a) moja wersja nr 243}

Rozwiązać nierówności $(x-2)(x-7)(x-17)\ge0$.
\zadStop
\rozwStart{Patryk Wirkus}{}
Miejsca zerowe naszego wielomianu to: $2, 7, 17$.\\
Wielomian jest stopnia nieparzystego, ponadto znak współczynnika przy\linebreak najwyższej potędze x jest dodatni.\\ W związku z tym wykres wielomianu zaczyna się od lewej strony poniżej osi OX. A więc $$x \in [2,7] \cup [17,\infty).$$
\rozwStop
\odpStart
$x \in [2,7] \cup [17,\infty)$
\odpStop
\testStart
A.$x \in [2,7] \cup [17,\infty)$\\
B.$x \in (2,7) \cup [17,\infty)$\\
C.$x \in (2,7] \cup [17,\infty)$\\
D.$x \in [2,7) \cup [17,\infty)$\\
E.$x \in [2,7] \cup (17,\infty)$\\
F.$x \in (2,7) \cup (17,\infty)$\\
G.$x \in [2,7) \cup (17,\infty)$\\
H.$x \in (2,7] \cup (17,\infty)$
\testStop
\kluczStart
A
\kluczStop



\zadStart{Zadanie z Wikieł Z 1.62 a) moja wersja nr 244}

Rozwiązać nierówności $(x-2)(x-7)(x-18)\ge0$.
\zadStop
\rozwStart{Patryk Wirkus}{}
Miejsca zerowe naszego wielomianu to: $2, 7, 18$.\\
Wielomian jest stopnia nieparzystego, ponadto znak współczynnika przy\linebreak najwyższej potędze x jest dodatni.\\ W związku z tym wykres wielomianu zaczyna się od lewej strony poniżej osi OX. A więc $$x \in [2,7] \cup [18,\infty).$$
\rozwStop
\odpStart
$x \in [2,7] \cup [18,\infty)$
\odpStop
\testStart
A.$x \in [2,7] \cup [18,\infty)$\\
B.$x \in (2,7) \cup [18,\infty)$\\
C.$x \in (2,7] \cup [18,\infty)$\\
D.$x \in [2,7) \cup [18,\infty)$\\
E.$x \in [2,7] \cup (18,\infty)$\\
F.$x \in (2,7) \cup (18,\infty)$\\
G.$x \in [2,7) \cup (18,\infty)$\\
H.$x \in (2,7] \cup (18,\infty)$
\testStop
\kluczStart
A
\kluczStop



\zadStart{Zadanie z Wikieł Z 1.62 a) moja wersja nr 245}

Rozwiązać nierówności $(x-2)(x-7)(x-19)\ge0$.
\zadStop
\rozwStart{Patryk Wirkus}{}
Miejsca zerowe naszego wielomianu to: $2, 7, 19$.\\
Wielomian jest stopnia nieparzystego, ponadto znak współczynnika przy\linebreak najwyższej potędze x jest dodatni.\\ W związku z tym wykres wielomianu zaczyna się od lewej strony poniżej osi OX. A więc $$x \in [2,7] \cup [19,\infty).$$
\rozwStop
\odpStart
$x \in [2,7] \cup [19,\infty)$
\odpStop
\testStart
A.$x \in [2,7] \cup [19,\infty)$\\
B.$x \in (2,7) \cup [19,\infty)$\\
C.$x \in (2,7] \cup [19,\infty)$\\
D.$x \in [2,7) \cup [19,\infty)$\\
E.$x \in [2,7] \cup (19,\infty)$\\
F.$x \in (2,7) \cup (19,\infty)$\\
G.$x \in [2,7) \cup (19,\infty)$\\
H.$x \in (2,7] \cup (19,\infty)$
\testStop
\kluczStart
A
\kluczStop



\zadStart{Zadanie z Wikieł Z 1.62 a) moja wersja nr 246}

Rozwiązać nierówności $(x-2)(x-7)(x-20)\ge0$.
\zadStop
\rozwStart{Patryk Wirkus}{}
Miejsca zerowe naszego wielomianu to: $2, 7, 20$.\\
Wielomian jest stopnia nieparzystego, ponadto znak współczynnika przy\linebreak najwyższej potędze x jest dodatni.\\ W związku z tym wykres wielomianu zaczyna się od lewej strony poniżej osi OX. A więc $$x \in [2,7] \cup [20,\infty).$$
\rozwStop
\odpStart
$x \in [2,7] \cup [20,\infty)$
\odpStop
\testStart
A.$x \in [2,7] \cup [20,\infty)$\\
B.$x \in (2,7) \cup [20,\infty)$\\
C.$x \in (2,7] \cup [20,\infty)$\\
D.$x \in [2,7) \cup [20,\infty)$\\
E.$x \in [2,7] \cup (20,\infty)$\\
F.$x \in (2,7) \cup (20,\infty)$\\
G.$x \in [2,7) \cup (20,\infty)$\\
H.$x \in (2,7] \cup (20,\infty)$
\testStop
\kluczStart
A
\kluczStop



\zadStart{Zadanie z Wikieł Z 1.62 a) moja wersja nr 247}

Rozwiązać nierówności $(x-2)(x-8)(x-9)\ge0$.
\zadStop
\rozwStart{Patryk Wirkus}{}
Miejsca zerowe naszego wielomianu to: $2, 8, 9$.\\
Wielomian jest stopnia nieparzystego, ponadto znak współczynnika przy\linebreak najwyższej potędze x jest dodatni.\\ W związku z tym wykres wielomianu zaczyna się od lewej strony poniżej osi OX. A więc $$x \in [2,8] \cup [9,\infty).$$
\rozwStop
\odpStart
$x \in [2,8] \cup [9,\infty)$
\odpStop
\testStart
A.$x \in [2,8] \cup [9,\infty)$\\
B.$x \in (2,8) \cup [9,\infty)$\\
C.$x \in (2,8] \cup [9,\infty)$\\
D.$x \in [2,8) \cup [9,\infty)$\\
E.$x \in [2,8] \cup (9,\infty)$\\
F.$x \in (2,8) \cup (9,\infty)$\\
G.$x \in [2,8) \cup (9,\infty)$\\
H.$x \in (2,8] \cup (9,\infty)$
\testStop
\kluczStart
A
\kluczStop



\zadStart{Zadanie z Wikieł Z 1.62 a) moja wersja nr 248}

Rozwiązać nierówności $(x-2)(x-8)(x-10)\ge0$.
\zadStop
\rozwStart{Patryk Wirkus}{}
Miejsca zerowe naszego wielomianu to: $2, 8, 10$.\\
Wielomian jest stopnia nieparzystego, ponadto znak współczynnika przy\linebreak najwyższej potędze x jest dodatni.\\ W związku z tym wykres wielomianu zaczyna się od lewej strony poniżej osi OX. A więc $$x \in [2,8] \cup [10,\infty).$$
\rozwStop
\odpStart
$x \in [2,8] \cup [10,\infty)$
\odpStop
\testStart
A.$x \in [2,8] \cup [10,\infty)$\\
B.$x \in (2,8) \cup [10,\infty)$\\
C.$x \in (2,8] \cup [10,\infty)$\\
D.$x \in [2,8) \cup [10,\infty)$\\
E.$x \in [2,8] \cup (10,\infty)$\\
F.$x \in (2,8) \cup (10,\infty)$\\
G.$x \in [2,8) \cup (10,\infty)$\\
H.$x \in (2,8] \cup (10,\infty)$
\testStop
\kluczStart
A
\kluczStop



\zadStart{Zadanie z Wikieł Z 1.62 a) moja wersja nr 249}

Rozwiązać nierówności $(x-2)(x-8)(x-11)\ge0$.
\zadStop
\rozwStart{Patryk Wirkus}{}
Miejsca zerowe naszego wielomianu to: $2, 8, 11$.\\
Wielomian jest stopnia nieparzystego, ponadto znak współczynnika przy\linebreak najwyższej potędze x jest dodatni.\\ W związku z tym wykres wielomianu zaczyna się od lewej strony poniżej osi OX. A więc $$x \in [2,8] \cup [11,\infty).$$
\rozwStop
\odpStart
$x \in [2,8] \cup [11,\infty)$
\odpStop
\testStart
A.$x \in [2,8] \cup [11,\infty)$\\
B.$x \in (2,8) \cup [11,\infty)$\\
C.$x \in (2,8] \cup [11,\infty)$\\
D.$x \in [2,8) \cup [11,\infty)$\\
E.$x \in [2,8] \cup (11,\infty)$\\
F.$x \in (2,8) \cup (11,\infty)$\\
G.$x \in [2,8) \cup (11,\infty)$\\
H.$x \in (2,8] \cup (11,\infty)$
\testStop
\kluczStart
A
\kluczStop



\zadStart{Zadanie z Wikieł Z 1.62 a) moja wersja nr 250}

Rozwiązać nierówności $(x-2)(x-8)(x-12)\ge0$.
\zadStop
\rozwStart{Patryk Wirkus}{}
Miejsca zerowe naszego wielomianu to: $2, 8, 12$.\\
Wielomian jest stopnia nieparzystego, ponadto znak współczynnika przy\linebreak najwyższej potędze x jest dodatni.\\ W związku z tym wykres wielomianu zaczyna się od lewej strony poniżej osi OX. A więc $$x \in [2,8] \cup [12,\infty).$$
\rozwStop
\odpStart
$x \in [2,8] \cup [12,\infty)$
\odpStop
\testStart
A.$x \in [2,8] \cup [12,\infty)$\\
B.$x \in (2,8) \cup [12,\infty)$\\
C.$x \in (2,8] \cup [12,\infty)$\\
D.$x \in [2,8) \cup [12,\infty)$\\
E.$x \in [2,8] \cup (12,\infty)$\\
F.$x \in (2,8) \cup (12,\infty)$\\
G.$x \in [2,8) \cup (12,\infty)$\\
H.$x \in (2,8] \cup (12,\infty)$
\testStop
\kluczStart
A
\kluczStop



\zadStart{Zadanie z Wikieł Z 1.62 a) moja wersja nr 251}

Rozwiązać nierówności $(x-2)(x-8)(x-13)\ge0$.
\zadStop
\rozwStart{Patryk Wirkus}{}
Miejsca zerowe naszego wielomianu to: $2, 8, 13$.\\
Wielomian jest stopnia nieparzystego, ponadto znak współczynnika przy\linebreak najwyższej potędze x jest dodatni.\\ W związku z tym wykres wielomianu zaczyna się od lewej strony poniżej osi OX. A więc $$x \in [2,8] \cup [13,\infty).$$
\rozwStop
\odpStart
$x \in [2,8] \cup [13,\infty)$
\odpStop
\testStart
A.$x \in [2,8] \cup [13,\infty)$\\
B.$x \in (2,8) \cup [13,\infty)$\\
C.$x \in (2,8] \cup [13,\infty)$\\
D.$x \in [2,8) \cup [13,\infty)$\\
E.$x \in [2,8] \cup (13,\infty)$\\
F.$x \in (2,8) \cup (13,\infty)$\\
G.$x \in [2,8) \cup (13,\infty)$\\
H.$x \in (2,8] \cup (13,\infty)$
\testStop
\kluczStart
A
\kluczStop



\zadStart{Zadanie z Wikieł Z 1.62 a) moja wersja nr 252}

Rozwiązać nierówności $(x-2)(x-8)(x-14)\ge0$.
\zadStop
\rozwStart{Patryk Wirkus}{}
Miejsca zerowe naszego wielomianu to: $2, 8, 14$.\\
Wielomian jest stopnia nieparzystego, ponadto znak współczynnika przy\linebreak najwyższej potędze x jest dodatni.\\ W związku z tym wykres wielomianu zaczyna się od lewej strony poniżej osi OX. A więc $$x \in [2,8] \cup [14,\infty).$$
\rozwStop
\odpStart
$x \in [2,8] \cup [14,\infty)$
\odpStop
\testStart
A.$x \in [2,8] \cup [14,\infty)$\\
B.$x \in (2,8) \cup [14,\infty)$\\
C.$x \in (2,8] \cup [14,\infty)$\\
D.$x \in [2,8) \cup [14,\infty)$\\
E.$x \in [2,8] \cup (14,\infty)$\\
F.$x \in (2,8) \cup (14,\infty)$\\
G.$x \in [2,8) \cup (14,\infty)$\\
H.$x \in (2,8] \cup (14,\infty)$
\testStop
\kluczStart
A
\kluczStop



\zadStart{Zadanie z Wikieł Z 1.62 a) moja wersja nr 253}

Rozwiązać nierówności $(x-2)(x-8)(x-15)\ge0$.
\zadStop
\rozwStart{Patryk Wirkus}{}
Miejsca zerowe naszego wielomianu to: $2, 8, 15$.\\
Wielomian jest stopnia nieparzystego, ponadto znak współczynnika przy\linebreak najwyższej potędze x jest dodatni.\\ W związku z tym wykres wielomianu zaczyna się od lewej strony poniżej osi OX. A więc $$x \in [2,8] \cup [15,\infty).$$
\rozwStop
\odpStart
$x \in [2,8] \cup [15,\infty)$
\odpStop
\testStart
A.$x \in [2,8] \cup [15,\infty)$\\
B.$x \in (2,8) \cup [15,\infty)$\\
C.$x \in (2,8] \cup [15,\infty)$\\
D.$x \in [2,8) \cup [15,\infty)$\\
E.$x \in [2,8] \cup (15,\infty)$\\
F.$x \in (2,8) \cup (15,\infty)$\\
G.$x \in [2,8) \cup (15,\infty)$\\
H.$x \in (2,8] \cup (15,\infty)$
\testStop
\kluczStart
A
\kluczStop



\zadStart{Zadanie z Wikieł Z 1.62 a) moja wersja nr 254}

Rozwiązać nierówności $(x-2)(x-8)(x-16)\ge0$.
\zadStop
\rozwStart{Patryk Wirkus}{}
Miejsca zerowe naszego wielomianu to: $2, 8, 16$.\\
Wielomian jest stopnia nieparzystego, ponadto znak współczynnika przy\linebreak najwyższej potędze x jest dodatni.\\ W związku z tym wykres wielomianu zaczyna się od lewej strony poniżej osi OX. A więc $$x \in [2,8] \cup [16,\infty).$$
\rozwStop
\odpStart
$x \in [2,8] \cup [16,\infty)$
\odpStop
\testStart
A.$x \in [2,8] \cup [16,\infty)$\\
B.$x \in (2,8) \cup [16,\infty)$\\
C.$x \in (2,8] \cup [16,\infty)$\\
D.$x \in [2,8) \cup [16,\infty)$\\
E.$x \in [2,8] \cup (16,\infty)$\\
F.$x \in (2,8) \cup (16,\infty)$\\
G.$x \in [2,8) \cup (16,\infty)$\\
H.$x \in (2,8] \cup (16,\infty)$
\testStop
\kluczStart
A
\kluczStop



\zadStart{Zadanie z Wikieł Z 1.62 a) moja wersja nr 255}

Rozwiązać nierówności $(x-2)(x-8)(x-17)\ge0$.
\zadStop
\rozwStart{Patryk Wirkus}{}
Miejsca zerowe naszego wielomianu to: $2, 8, 17$.\\
Wielomian jest stopnia nieparzystego, ponadto znak współczynnika przy\linebreak najwyższej potędze x jest dodatni.\\ W związku z tym wykres wielomianu zaczyna się od lewej strony poniżej osi OX. A więc $$x \in [2,8] \cup [17,\infty).$$
\rozwStop
\odpStart
$x \in [2,8] \cup [17,\infty)$
\odpStop
\testStart
A.$x \in [2,8] \cup [17,\infty)$\\
B.$x \in (2,8) \cup [17,\infty)$\\
C.$x \in (2,8] \cup [17,\infty)$\\
D.$x \in [2,8) \cup [17,\infty)$\\
E.$x \in [2,8] \cup (17,\infty)$\\
F.$x \in (2,8) \cup (17,\infty)$\\
G.$x \in [2,8) \cup (17,\infty)$\\
H.$x \in (2,8] \cup (17,\infty)$
\testStop
\kluczStart
A
\kluczStop



\zadStart{Zadanie z Wikieł Z 1.62 a) moja wersja nr 256}

Rozwiązać nierówności $(x-2)(x-8)(x-18)\ge0$.
\zadStop
\rozwStart{Patryk Wirkus}{}
Miejsca zerowe naszego wielomianu to: $2, 8, 18$.\\
Wielomian jest stopnia nieparzystego, ponadto znak współczynnika przy\linebreak najwyższej potędze x jest dodatni.\\ W związku z tym wykres wielomianu zaczyna się od lewej strony poniżej osi OX. A więc $$x \in [2,8] \cup [18,\infty).$$
\rozwStop
\odpStart
$x \in [2,8] \cup [18,\infty)$
\odpStop
\testStart
A.$x \in [2,8] \cup [18,\infty)$\\
B.$x \in (2,8) \cup [18,\infty)$\\
C.$x \in (2,8] \cup [18,\infty)$\\
D.$x \in [2,8) \cup [18,\infty)$\\
E.$x \in [2,8] \cup (18,\infty)$\\
F.$x \in (2,8) \cup (18,\infty)$\\
G.$x \in [2,8) \cup (18,\infty)$\\
H.$x \in (2,8] \cup (18,\infty)$
\testStop
\kluczStart
A
\kluczStop



\zadStart{Zadanie z Wikieł Z 1.62 a) moja wersja nr 257}

Rozwiązać nierówności $(x-2)(x-8)(x-19)\ge0$.
\zadStop
\rozwStart{Patryk Wirkus}{}
Miejsca zerowe naszego wielomianu to: $2, 8, 19$.\\
Wielomian jest stopnia nieparzystego, ponadto znak współczynnika przy\linebreak najwyższej potędze x jest dodatni.\\ W związku z tym wykres wielomianu zaczyna się od lewej strony poniżej osi OX. A więc $$x \in [2,8] \cup [19,\infty).$$
\rozwStop
\odpStart
$x \in [2,8] \cup [19,\infty)$
\odpStop
\testStart
A.$x \in [2,8] \cup [19,\infty)$\\
B.$x \in (2,8) \cup [19,\infty)$\\
C.$x \in (2,8] \cup [19,\infty)$\\
D.$x \in [2,8) \cup [19,\infty)$\\
E.$x \in [2,8] \cup (19,\infty)$\\
F.$x \in (2,8) \cup (19,\infty)$\\
G.$x \in [2,8) \cup (19,\infty)$\\
H.$x \in (2,8] \cup (19,\infty)$
\testStop
\kluczStart
A
\kluczStop



\zadStart{Zadanie z Wikieł Z 1.62 a) moja wersja nr 258}

Rozwiązać nierówności $(x-2)(x-8)(x-20)\ge0$.
\zadStop
\rozwStart{Patryk Wirkus}{}
Miejsca zerowe naszego wielomianu to: $2, 8, 20$.\\
Wielomian jest stopnia nieparzystego, ponadto znak współczynnika przy\linebreak najwyższej potędze x jest dodatni.\\ W związku z tym wykres wielomianu zaczyna się od lewej strony poniżej osi OX. A więc $$x \in [2,8] \cup [20,\infty).$$
\rozwStop
\odpStart
$x \in [2,8] \cup [20,\infty)$
\odpStop
\testStart
A.$x \in [2,8] \cup [20,\infty)$\\
B.$x \in (2,8) \cup [20,\infty)$\\
C.$x \in (2,8] \cup [20,\infty)$\\
D.$x \in [2,8) \cup [20,\infty)$\\
E.$x \in [2,8] \cup (20,\infty)$\\
F.$x \in (2,8) \cup (20,\infty)$\\
G.$x \in [2,8) \cup (20,\infty)$\\
H.$x \in (2,8] \cup (20,\infty)$
\testStop
\kluczStart
A
\kluczStop



\zadStart{Zadanie z Wikieł Z 1.62 a) moja wersja nr 259}

Rozwiązać nierówności $(x-2)(x-9)(x-10)\ge0$.
\zadStop
\rozwStart{Patryk Wirkus}{}
Miejsca zerowe naszego wielomianu to: $2, 9, 10$.\\
Wielomian jest stopnia nieparzystego, ponadto znak współczynnika przy\linebreak najwyższej potędze x jest dodatni.\\ W związku z tym wykres wielomianu zaczyna się od lewej strony poniżej osi OX. A więc $$x \in [2,9] \cup [10,\infty).$$
\rozwStop
\odpStart
$x \in [2,9] \cup [10,\infty)$
\odpStop
\testStart
A.$x \in [2,9] \cup [10,\infty)$\\
B.$x \in (2,9) \cup [10,\infty)$\\
C.$x \in (2,9] \cup [10,\infty)$\\
D.$x \in [2,9) \cup [10,\infty)$\\
E.$x \in [2,9] \cup (10,\infty)$\\
F.$x \in (2,9) \cup (10,\infty)$\\
G.$x \in [2,9) \cup (10,\infty)$\\
H.$x \in (2,9] \cup (10,\infty)$
\testStop
\kluczStart
A
\kluczStop



\zadStart{Zadanie z Wikieł Z 1.62 a) moja wersja nr 260}

Rozwiązać nierówności $(x-2)(x-9)(x-11)\ge0$.
\zadStop
\rozwStart{Patryk Wirkus}{}
Miejsca zerowe naszego wielomianu to: $2, 9, 11$.\\
Wielomian jest stopnia nieparzystego, ponadto znak współczynnika przy\linebreak najwyższej potędze x jest dodatni.\\ W związku z tym wykres wielomianu zaczyna się od lewej strony poniżej osi OX. A więc $$x \in [2,9] \cup [11,\infty).$$
\rozwStop
\odpStart
$x \in [2,9] \cup [11,\infty)$
\odpStop
\testStart
A.$x \in [2,9] \cup [11,\infty)$\\
B.$x \in (2,9) \cup [11,\infty)$\\
C.$x \in (2,9] \cup [11,\infty)$\\
D.$x \in [2,9) \cup [11,\infty)$\\
E.$x \in [2,9] \cup (11,\infty)$\\
F.$x \in (2,9) \cup (11,\infty)$\\
G.$x \in [2,9) \cup (11,\infty)$\\
H.$x \in (2,9] \cup (11,\infty)$
\testStop
\kluczStart
A
\kluczStop



\zadStart{Zadanie z Wikieł Z 1.62 a) moja wersja nr 261}

Rozwiązać nierówności $(x-2)(x-9)(x-12)\ge0$.
\zadStop
\rozwStart{Patryk Wirkus}{}
Miejsca zerowe naszego wielomianu to: $2, 9, 12$.\\
Wielomian jest stopnia nieparzystego, ponadto znak współczynnika przy\linebreak najwyższej potędze x jest dodatni.\\ W związku z tym wykres wielomianu zaczyna się od lewej strony poniżej osi OX. A więc $$x \in [2,9] \cup [12,\infty).$$
\rozwStop
\odpStart
$x \in [2,9] \cup [12,\infty)$
\odpStop
\testStart
A.$x \in [2,9] \cup [12,\infty)$\\
B.$x \in (2,9) \cup [12,\infty)$\\
C.$x \in (2,9] \cup [12,\infty)$\\
D.$x \in [2,9) \cup [12,\infty)$\\
E.$x \in [2,9] \cup (12,\infty)$\\
F.$x \in (2,9) \cup (12,\infty)$\\
G.$x \in [2,9) \cup (12,\infty)$\\
H.$x \in (2,9] \cup (12,\infty)$
\testStop
\kluczStart
A
\kluczStop



\zadStart{Zadanie z Wikieł Z 1.62 a) moja wersja nr 262}

Rozwiązać nierówności $(x-2)(x-9)(x-13)\ge0$.
\zadStop
\rozwStart{Patryk Wirkus}{}
Miejsca zerowe naszego wielomianu to: $2, 9, 13$.\\
Wielomian jest stopnia nieparzystego, ponadto znak współczynnika przy\linebreak najwyższej potędze x jest dodatni.\\ W związku z tym wykres wielomianu zaczyna się od lewej strony poniżej osi OX. A więc $$x \in [2,9] \cup [13,\infty).$$
\rozwStop
\odpStart
$x \in [2,9] \cup [13,\infty)$
\odpStop
\testStart
A.$x \in [2,9] \cup [13,\infty)$\\
B.$x \in (2,9) \cup [13,\infty)$\\
C.$x \in (2,9] \cup [13,\infty)$\\
D.$x \in [2,9) \cup [13,\infty)$\\
E.$x \in [2,9] \cup (13,\infty)$\\
F.$x \in (2,9) \cup (13,\infty)$\\
G.$x \in [2,9) \cup (13,\infty)$\\
H.$x \in (2,9] \cup (13,\infty)$
\testStop
\kluczStart
A
\kluczStop



\zadStart{Zadanie z Wikieł Z 1.62 a) moja wersja nr 263}

Rozwiązać nierówności $(x-2)(x-9)(x-14)\ge0$.
\zadStop
\rozwStart{Patryk Wirkus}{}
Miejsca zerowe naszego wielomianu to: $2, 9, 14$.\\
Wielomian jest stopnia nieparzystego, ponadto znak współczynnika przy\linebreak najwyższej potędze x jest dodatni.\\ W związku z tym wykres wielomianu zaczyna się od lewej strony poniżej osi OX. A więc $$x \in [2,9] \cup [14,\infty).$$
\rozwStop
\odpStart
$x \in [2,9] \cup [14,\infty)$
\odpStop
\testStart
A.$x \in [2,9] \cup [14,\infty)$\\
B.$x \in (2,9) \cup [14,\infty)$\\
C.$x \in (2,9] \cup [14,\infty)$\\
D.$x \in [2,9) \cup [14,\infty)$\\
E.$x \in [2,9] \cup (14,\infty)$\\
F.$x \in (2,9) \cup (14,\infty)$\\
G.$x \in [2,9) \cup (14,\infty)$\\
H.$x \in (2,9] \cup (14,\infty)$
\testStop
\kluczStart
A
\kluczStop



\zadStart{Zadanie z Wikieł Z 1.62 a) moja wersja nr 264}

Rozwiązać nierówności $(x-2)(x-9)(x-15)\ge0$.
\zadStop
\rozwStart{Patryk Wirkus}{}
Miejsca zerowe naszego wielomianu to: $2, 9, 15$.\\
Wielomian jest stopnia nieparzystego, ponadto znak współczynnika przy\linebreak najwyższej potędze x jest dodatni.\\ W związku z tym wykres wielomianu zaczyna się od lewej strony poniżej osi OX. A więc $$x \in [2,9] \cup [15,\infty).$$
\rozwStop
\odpStart
$x \in [2,9] \cup [15,\infty)$
\odpStop
\testStart
A.$x \in [2,9] \cup [15,\infty)$\\
B.$x \in (2,9) \cup [15,\infty)$\\
C.$x \in (2,9] \cup [15,\infty)$\\
D.$x \in [2,9) \cup [15,\infty)$\\
E.$x \in [2,9] \cup (15,\infty)$\\
F.$x \in (2,9) \cup (15,\infty)$\\
G.$x \in [2,9) \cup (15,\infty)$\\
H.$x \in (2,9] \cup (15,\infty)$
\testStop
\kluczStart
A
\kluczStop



\zadStart{Zadanie z Wikieł Z 1.62 a) moja wersja nr 265}

Rozwiązać nierówności $(x-2)(x-9)(x-16)\ge0$.
\zadStop
\rozwStart{Patryk Wirkus}{}
Miejsca zerowe naszego wielomianu to: $2, 9, 16$.\\
Wielomian jest stopnia nieparzystego, ponadto znak współczynnika przy\linebreak najwyższej potędze x jest dodatni.\\ W związku z tym wykres wielomianu zaczyna się od lewej strony poniżej osi OX. A więc $$x \in [2,9] \cup [16,\infty).$$
\rozwStop
\odpStart
$x \in [2,9] \cup [16,\infty)$
\odpStop
\testStart
A.$x \in [2,9] \cup [16,\infty)$\\
B.$x \in (2,9) \cup [16,\infty)$\\
C.$x \in (2,9] \cup [16,\infty)$\\
D.$x \in [2,9) \cup [16,\infty)$\\
E.$x \in [2,9] \cup (16,\infty)$\\
F.$x \in (2,9) \cup (16,\infty)$\\
G.$x \in [2,9) \cup (16,\infty)$\\
H.$x \in (2,9] \cup (16,\infty)$
\testStop
\kluczStart
A
\kluczStop



\zadStart{Zadanie z Wikieł Z 1.62 a) moja wersja nr 266}

Rozwiązać nierówności $(x-2)(x-9)(x-17)\ge0$.
\zadStop
\rozwStart{Patryk Wirkus}{}
Miejsca zerowe naszego wielomianu to: $2, 9, 17$.\\
Wielomian jest stopnia nieparzystego, ponadto znak współczynnika przy\linebreak najwyższej potędze x jest dodatni.\\ W związku z tym wykres wielomianu zaczyna się od lewej strony poniżej osi OX. A więc $$x \in [2,9] \cup [17,\infty).$$
\rozwStop
\odpStart
$x \in [2,9] \cup [17,\infty)$
\odpStop
\testStart
A.$x \in [2,9] \cup [17,\infty)$\\
B.$x \in (2,9) \cup [17,\infty)$\\
C.$x \in (2,9] \cup [17,\infty)$\\
D.$x \in [2,9) \cup [17,\infty)$\\
E.$x \in [2,9] \cup (17,\infty)$\\
F.$x \in (2,9) \cup (17,\infty)$\\
G.$x \in [2,9) \cup (17,\infty)$\\
H.$x \in (2,9] \cup (17,\infty)$
\testStop
\kluczStart
A
\kluczStop



\zadStart{Zadanie z Wikieł Z 1.62 a) moja wersja nr 267}

Rozwiązać nierówności $(x-2)(x-9)(x-18)\ge0$.
\zadStop
\rozwStart{Patryk Wirkus}{}
Miejsca zerowe naszego wielomianu to: $2, 9, 18$.\\
Wielomian jest stopnia nieparzystego, ponadto znak współczynnika przy\linebreak najwyższej potędze x jest dodatni.\\ W związku z tym wykres wielomianu zaczyna się od lewej strony poniżej osi OX. A więc $$x \in [2,9] \cup [18,\infty).$$
\rozwStop
\odpStart
$x \in [2,9] \cup [18,\infty)$
\odpStop
\testStart
A.$x \in [2,9] \cup [18,\infty)$\\
B.$x \in (2,9) \cup [18,\infty)$\\
C.$x \in (2,9] \cup [18,\infty)$\\
D.$x \in [2,9) \cup [18,\infty)$\\
E.$x \in [2,9] \cup (18,\infty)$\\
F.$x \in (2,9) \cup (18,\infty)$\\
G.$x \in [2,9) \cup (18,\infty)$\\
H.$x \in (2,9] \cup (18,\infty)$
\testStop
\kluczStart
A
\kluczStop



\zadStart{Zadanie z Wikieł Z 1.62 a) moja wersja nr 268}

Rozwiązać nierówności $(x-2)(x-9)(x-19)\ge0$.
\zadStop
\rozwStart{Patryk Wirkus}{}
Miejsca zerowe naszego wielomianu to: $2, 9, 19$.\\
Wielomian jest stopnia nieparzystego, ponadto znak współczynnika przy\linebreak najwyższej potędze x jest dodatni.\\ W związku z tym wykres wielomianu zaczyna się od lewej strony poniżej osi OX. A więc $$x \in [2,9] \cup [19,\infty).$$
\rozwStop
\odpStart
$x \in [2,9] \cup [19,\infty)$
\odpStop
\testStart
A.$x \in [2,9] \cup [19,\infty)$\\
B.$x \in (2,9) \cup [19,\infty)$\\
C.$x \in (2,9] \cup [19,\infty)$\\
D.$x \in [2,9) \cup [19,\infty)$\\
E.$x \in [2,9] \cup (19,\infty)$\\
F.$x \in (2,9) \cup (19,\infty)$\\
G.$x \in [2,9) \cup (19,\infty)$\\
H.$x \in (2,9] \cup (19,\infty)$
\testStop
\kluczStart
A
\kluczStop



\zadStart{Zadanie z Wikieł Z 1.62 a) moja wersja nr 269}

Rozwiązać nierówności $(x-2)(x-9)(x-20)\ge0$.
\zadStop
\rozwStart{Patryk Wirkus}{}
Miejsca zerowe naszego wielomianu to: $2, 9, 20$.\\
Wielomian jest stopnia nieparzystego, ponadto znak współczynnika przy\linebreak najwyższej potędze x jest dodatni.\\ W związku z tym wykres wielomianu zaczyna się od lewej strony poniżej osi OX. A więc $$x \in [2,9] \cup [20,\infty).$$
\rozwStop
\odpStart
$x \in [2,9] \cup [20,\infty)$
\odpStop
\testStart
A.$x \in [2,9] \cup [20,\infty)$\\
B.$x \in (2,9) \cup [20,\infty)$\\
C.$x \in (2,9] \cup [20,\infty)$\\
D.$x \in [2,9) \cup [20,\infty)$\\
E.$x \in [2,9] \cup (20,\infty)$\\
F.$x \in (2,9) \cup (20,\infty)$\\
G.$x \in [2,9) \cup (20,\infty)$\\
H.$x \in (2,9] \cup (20,\infty)$
\testStop
\kluczStart
A
\kluczStop



\zadStart{Zadanie z Wikieł Z 1.62 a) moja wersja nr 270}

Rozwiązać nierówności $(x-2)(x-10)(x-11)\ge0$.
\zadStop
\rozwStart{Patryk Wirkus}{}
Miejsca zerowe naszego wielomianu to: $2, 10, 11$.\\
Wielomian jest stopnia nieparzystego, ponadto znak współczynnika przy\linebreak najwyższej potędze x jest dodatni.\\ W związku z tym wykres wielomianu zaczyna się od lewej strony poniżej osi OX. A więc $$x \in [2,10] \cup [11,\infty).$$
\rozwStop
\odpStart
$x \in [2,10] \cup [11,\infty)$
\odpStop
\testStart
A.$x \in [2,10] \cup [11,\infty)$\\
B.$x \in (2,10) \cup [11,\infty)$\\
C.$x \in (2,10] \cup [11,\infty)$\\
D.$x \in [2,10) \cup [11,\infty)$\\
E.$x \in [2,10] \cup (11,\infty)$\\
F.$x \in (2,10) \cup (11,\infty)$\\
G.$x \in [2,10) \cup (11,\infty)$\\
H.$x \in (2,10] \cup (11,\infty)$
\testStop
\kluczStart
A
\kluczStop



\zadStart{Zadanie z Wikieł Z 1.62 a) moja wersja nr 271}

Rozwiązać nierówności $(x-2)(x-10)(x-12)\ge0$.
\zadStop
\rozwStart{Patryk Wirkus}{}
Miejsca zerowe naszego wielomianu to: $2, 10, 12$.\\
Wielomian jest stopnia nieparzystego, ponadto znak współczynnika przy\linebreak najwyższej potędze x jest dodatni.\\ W związku z tym wykres wielomianu zaczyna się od lewej strony poniżej osi OX. A więc $$x \in [2,10] \cup [12,\infty).$$
\rozwStop
\odpStart
$x \in [2,10] \cup [12,\infty)$
\odpStop
\testStart
A.$x \in [2,10] \cup [12,\infty)$\\
B.$x \in (2,10) \cup [12,\infty)$\\
C.$x \in (2,10] \cup [12,\infty)$\\
D.$x \in [2,10) \cup [12,\infty)$\\
E.$x \in [2,10] \cup (12,\infty)$\\
F.$x \in (2,10) \cup (12,\infty)$\\
G.$x \in [2,10) \cup (12,\infty)$\\
H.$x \in (2,10] \cup (12,\infty)$
\testStop
\kluczStart
A
\kluczStop



\zadStart{Zadanie z Wikieł Z 1.62 a) moja wersja nr 272}

Rozwiązać nierówności $(x-2)(x-10)(x-13)\ge0$.
\zadStop
\rozwStart{Patryk Wirkus}{}
Miejsca zerowe naszego wielomianu to: $2, 10, 13$.\\
Wielomian jest stopnia nieparzystego, ponadto znak współczynnika przy\linebreak najwyższej potędze x jest dodatni.\\ W związku z tym wykres wielomianu zaczyna się od lewej strony poniżej osi OX. A więc $$x \in [2,10] \cup [13,\infty).$$
\rozwStop
\odpStart
$x \in [2,10] \cup [13,\infty)$
\odpStop
\testStart
A.$x \in [2,10] \cup [13,\infty)$\\
B.$x \in (2,10) \cup [13,\infty)$\\
C.$x \in (2,10] \cup [13,\infty)$\\
D.$x \in [2,10) \cup [13,\infty)$\\
E.$x \in [2,10] \cup (13,\infty)$\\
F.$x \in (2,10) \cup (13,\infty)$\\
G.$x \in [2,10) \cup (13,\infty)$\\
H.$x \in (2,10] \cup (13,\infty)$
\testStop
\kluczStart
A
\kluczStop



\zadStart{Zadanie z Wikieł Z 1.62 a) moja wersja nr 273}

Rozwiązać nierówności $(x-2)(x-10)(x-14)\ge0$.
\zadStop
\rozwStart{Patryk Wirkus}{}
Miejsca zerowe naszego wielomianu to: $2, 10, 14$.\\
Wielomian jest stopnia nieparzystego, ponadto znak współczynnika przy\linebreak najwyższej potędze x jest dodatni.\\ W związku z tym wykres wielomianu zaczyna się od lewej strony poniżej osi OX. A więc $$x \in [2,10] \cup [14,\infty).$$
\rozwStop
\odpStart
$x \in [2,10] \cup [14,\infty)$
\odpStop
\testStart
A.$x \in [2,10] \cup [14,\infty)$\\
B.$x \in (2,10) \cup [14,\infty)$\\
C.$x \in (2,10] \cup [14,\infty)$\\
D.$x \in [2,10) \cup [14,\infty)$\\
E.$x \in [2,10] \cup (14,\infty)$\\
F.$x \in (2,10) \cup (14,\infty)$\\
G.$x \in [2,10) \cup (14,\infty)$\\
H.$x \in (2,10] \cup (14,\infty)$
\testStop
\kluczStart
A
\kluczStop



\zadStart{Zadanie z Wikieł Z 1.62 a) moja wersja nr 274}

Rozwiązać nierówności $(x-2)(x-10)(x-15)\ge0$.
\zadStop
\rozwStart{Patryk Wirkus}{}
Miejsca zerowe naszego wielomianu to: $2, 10, 15$.\\
Wielomian jest stopnia nieparzystego, ponadto znak współczynnika przy\linebreak najwyższej potędze x jest dodatni.\\ W związku z tym wykres wielomianu zaczyna się od lewej strony poniżej osi OX. A więc $$x \in [2,10] \cup [15,\infty).$$
\rozwStop
\odpStart
$x \in [2,10] \cup [15,\infty)$
\odpStop
\testStart
A.$x \in [2,10] \cup [15,\infty)$\\
B.$x \in (2,10) \cup [15,\infty)$\\
C.$x \in (2,10] \cup [15,\infty)$\\
D.$x \in [2,10) \cup [15,\infty)$\\
E.$x \in [2,10] \cup (15,\infty)$\\
F.$x \in (2,10) \cup (15,\infty)$\\
G.$x \in [2,10) \cup (15,\infty)$\\
H.$x \in (2,10] \cup (15,\infty)$
\testStop
\kluczStart
A
\kluczStop



\zadStart{Zadanie z Wikieł Z 1.62 a) moja wersja nr 275}

Rozwiązać nierówności $(x-2)(x-10)(x-16)\ge0$.
\zadStop
\rozwStart{Patryk Wirkus}{}
Miejsca zerowe naszego wielomianu to: $2, 10, 16$.\\
Wielomian jest stopnia nieparzystego, ponadto znak współczynnika przy\linebreak najwyższej potędze x jest dodatni.\\ W związku z tym wykres wielomianu zaczyna się od lewej strony poniżej osi OX. A więc $$x \in [2,10] \cup [16,\infty).$$
\rozwStop
\odpStart
$x \in [2,10] \cup [16,\infty)$
\odpStop
\testStart
A.$x \in [2,10] \cup [16,\infty)$\\
B.$x \in (2,10) \cup [16,\infty)$\\
C.$x \in (2,10] \cup [16,\infty)$\\
D.$x \in [2,10) \cup [16,\infty)$\\
E.$x \in [2,10] \cup (16,\infty)$\\
F.$x \in (2,10) \cup (16,\infty)$\\
G.$x \in [2,10) \cup (16,\infty)$\\
H.$x \in (2,10] \cup (16,\infty)$
\testStop
\kluczStart
A
\kluczStop



\zadStart{Zadanie z Wikieł Z 1.62 a) moja wersja nr 276}

Rozwiązać nierówności $(x-2)(x-10)(x-17)\ge0$.
\zadStop
\rozwStart{Patryk Wirkus}{}
Miejsca zerowe naszego wielomianu to: $2, 10, 17$.\\
Wielomian jest stopnia nieparzystego, ponadto znak współczynnika przy\linebreak najwyższej potędze x jest dodatni.\\ W związku z tym wykres wielomianu zaczyna się od lewej strony poniżej osi OX. A więc $$x \in [2,10] \cup [17,\infty).$$
\rozwStop
\odpStart
$x \in [2,10] \cup [17,\infty)$
\odpStop
\testStart
A.$x \in [2,10] \cup [17,\infty)$\\
B.$x \in (2,10) \cup [17,\infty)$\\
C.$x \in (2,10] \cup [17,\infty)$\\
D.$x \in [2,10) \cup [17,\infty)$\\
E.$x \in [2,10] \cup (17,\infty)$\\
F.$x \in (2,10) \cup (17,\infty)$\\
G.$x \in [2,10) \cup (17,\infty)$\\
H.$x \in (2,10] \cup (17,\infty)$
\testStop
\kluczStart
A
\kluczStop



\zadStart{Zadanie z Wikieł Z 1.62 a) moja wersja nr 277}

Rozwiązać nierówności $(x-2)(x-10)(x-18)\ge0$.
\zadStop
\rozwStart{Patryk Wirkus}{}
Miejsca zerowe naszego wielomianu to: $2, 10, 18$.\\
Wielomian jest stopnia nieparzystego, ponadto znak współczynnika przy\linebreak najwyższej potędze x jest dodatni.\\ W związku z tym wykres wielomianu zaczyna się od lewej strony poniżej osi OX. A więc $$x \in [2,10] \cup [18,\infty).$$
\rozwStop
\odpStart
$x \in [2,10] \cup [18,\infty)$
\odpStop
\testStart
A.$x \in [2,10] \cup [18,\infty)$\\
B.$x \in (2,10) \cup [18,\infty)$\\
C.$x \in (2,10] \cup [18,\infty)$\\
D.$x \in [2,10) \cup [18,\infty)$\\
E.$x \in [2,10] \cup (18,\infty)$\\
F.$x \in (2,10) \cup (18,\infty)$\\
G.$x \in [2,10) \cup (18,\infty)$\\
H.$x \in (2,10] \cup (18,\infty)$
\testStop
\kluczStart
A
\kluczStop



\zadStart{Zadanie z Wikieł Z 1.62 a) moja wersja nr 278}

Rozwiązać nierówności $(x-2)(x-10)(x-19)\ge0$.
\zadStop
\rozwStart{Patryk Wirkus}{}
Miejsca zerowe naszego wielomianu to: $2, 10, 19$.\\
Wielomian jest stopnia nieparzystego, ponadto znak współczynnika przy\linebreak najwyższej potędze x jest dodatni.\\ W związku z tym wykres wielomianu zaczyna się od lewej strony poniżej osi OX. A więc $$x \in [2,10] \cup [19,\infty).$$
\rozwStop
\odpStart
$x \in [2,10] \cup [19,\infty)$
\odpStop
\testStart
A.$x \in [2,10] \cup [19,\infty)$\\
B.$x \in (2,10) \cup [19,\infty)$\\
C.$x \in (2,10] \cup [19,\infty)$\\
D.$x \in [2,10) \cup [19,\infty)$\\
E.$x \in [2,10] \cup (19,\infty)$\\
F.$x \in (2,10) \cup (19,\infty)$\\
G.$x \in [2,10) \cup (19,\infty)$\\
H.$x \in (2,10] \cup (19,\infty)$
\testStop
\kluczStart
A
\kluczStop



\zadStart{Zadanie z Wikieł Z 1.62 a) moja wersja nr 279}

Rozwiązać nierówności $(x-2)(x-10)(x-20)\ge0$.
\zadStop
\rozwStart{Patryk Wirkus}{}
Miejsca zerowe naszego wielomianu to: $2, 10, 20$.\\
Wielomian jest stopnia nieparzystego, ponadto znak współczynnika przy\linebreak najwyższej potędze x jest dodatni.\\ W związku z tym wykres wielomianu zaczyna się od lewej strony poniżej osi OX. A więc $$x \in [2,10] \cup [20,\infty).$$
\rozwStop
\odpStart
$x \in [2,10] \cup [20,\infty)$
\odpStop
\testStart
A.$x \in [2,10] \cup [20,\infty)$\\
B.$x \in (2,10) \cup [20,\infty)$\\
C.$x \in (2,10] \cup [20,\infty)$\\
D.$x \in [2,10) \cup [20,\infty)$\\
E.$x \in [2,10] \cup (20,\infty)$\\
F.$x \in (2,10) \cup (20,\infty)$\\
G.$x \in [2,10) \cup (20,\infty)$\\
H.$x \in (2,10] \cup (20,\infty)$
\testStop
\kluczStart
A
\kluczStop



\zadStart{Zadanie z Wikieł Z 1.62 a) moja wersja nr 280}

Rozwiązać nierówności $(x-2)(x-11)(x-12)\ge0$.
\zadStop
\rozwStart{Patryk Wirkus}{}
Miejsca zerowe naszego wielomianu to: $2, 11, 12$.\\
Wielomian jest stopnia nieparzystego, ponadto znak współczynnika przy\linebreak najwyższej potędze x jest dodatni.\\ W związku z tym wykres wielomianu zaczyna się od lewej strony poniżej osi OX. A więc $$x \in [2,11] \cup [12,\infty).$$
\rozwStop
\odpStart
$x \in [2,11] \cup [12,\infty)$
\odpStop
\testStart
A.$x \in [2,11] \cup [12,\infty)$\\
B.$x \in (2,11) \cup [12,\infty)$\\
C.$x \in (2,11] \cup [12,\infty)$\\
D.$x \in [2,11) \cup [12,\infty)$\\
E.$x \in [2,11] \cup (12,\infty)$\\
F.$x \in (2,11) \cup (12,\infty)$\\
G.$x \in [2,11) \cup (12,\infty)$\\
H.$x \in (2,11] \cup (12,\infty)$
\testStop
\kluczStart
A
\kluczStop



\zadStart{Zadanie z Wikieł Z 1.62 a) moja wersja nr 281}

Rozwiązać nierówności $(x-2)(x-11)(x-13)\ge0$.
\zadStop
\rozwStart{Patryk Wirkus}{}
Miejsca zerowe naszego wielomianu to: $2, 11, 13$.\\
Wielomian jest stopnia nieparzystego, ponadto znak współczynnika przy\linebreak najwyższej potędze x jest dodatni.\\ W związku z tym wykres wielomianu zaczyna się od lewej strony poniżej osi OX. A więc $$x \in [2,11] \cup [13,\infty).$$
\rozwStop
\odpStart
$x \in [2,11] \cup [13,\infty)$
\odpStop
\testStart
A.$x \in [2,11] \cup [13,\infty)$\\
B.$x \in (2,11) \cup [13,\infty)$\\
C.$x \in (2,11] \cup [13,\infty)$\\
D.$x \in [2,11) \cup [13,\infty)$\\
E.$x \in [2,11] \cup (13,\infty)$\\
F.$x \in (2,11) \cup (13,\infty)$\\
G.$x \in [2,11) \cup (13,\infty)$\\
H.$x \in (2,11] \cup (13,\infty)$
\testStop
\kluczStart
A
\kluczStop



\zadStart{Zadanie z Wikieł Z 1.62 a) moja wersja nr 282}

Rozwiązać nierówności $(x-2)(x-11)(x-14)\ge0$.
\zadStop
\rozwStart{Patryk Wirkus}{}
Miejsca zerowe naszego wielomianu to: $2, 11, 14$.\\
Wielomian jest stopnia nieparzystego, ponadto znak współczynnika przy\linebreak najwyższej potędze x jest dodatni.\\ W związku z tym wykres wielomianu zaczyna się od lewej strony poniżej osi OX. A więc $$x \in [2,11] \cup [14,\infty).$$
\rozwStop
\odpStart
$x \in [2,11] \cup [14,\infty)$
\odpStop
\testStart
A.$x \in [2,11] \cup [14,\infty)$\\
B.$x \in (2,11) \cup [14,\infty)$\\
C.$x \in (2,11] \cup [14,\infty)$\\
D.$x \in [2,11) \cup [14,\infty)$\\
E.$x \in [2,11] \cup (14,\infty)$\\
F.$x \in (2,11) \cup (14,\infty)$\\
G.$x \in [2,11) \cup (14,\infty)$\\
H.$x \in (2,11] \cup (14,\infty)$
\testStop
\kluczStart
A
\kluczStop



\zadStart{Zadanie z Wikieł Z 1.62 a) moja wersja nr 283}

Rozwiązać nierówności $(x-2)(x-11)(x-15)\ge0$.
\zadStop
\rozwStart{Patryk Wirkus}{}
Miejsca zerowe naszego wielomianu to: $2, 11, 15$.\\
Wielomian jest stopnia nieparzystego, ponadto znak współczynnika przy\linebreak najwyższej potędze x jest dodatni.\\ W związku z tym wykres wielomianu zaczyna się od lewej strony poniżej osi OX. A więc $$x \in [2,11] \cup [15,\infty).$$
\rozwStop
\odpStart
$x \in [2,11] \cup [15,\infty)$
\odpStop
\testStart
A.$x \in [2,11] \cup [15,\infty)$\\
B.$x \in (2,11) \cup [15,\infty)$\\
C.$x \in (2,11] \cup [15,\infty)$\\
D.$x \in [2,11) \cup [15,\infty)$\\
E.$x \in [2,11] \cup (15,\infty)$\\
F.$x \in (2,11) \cup (15,\infty)$\\
G.$x \in [2,11) \cup (15,\infty)$\\
H.$x \in (2,11] \cup (15,\infty)$
\testStop
\kluczStart
A
\kluczStop



\zadStart{Zadanie z Wikieł Z 1.62 a) moja wersja nr 284}

Rozwiązać nierówności $(x-2)(x-11)(x-16)\ge0$.
\zadStop
\rozwStart{Patryk Wirkus}{}
Miejsca zerowe naszego wielomianu to: $2, 11, 16$.\\
Wielomian jest stopnia nieparzystego, ponadto znak współczynnika przy\linebreak najwyższej potędze x jest dodatni.\\ W związku z tym wykres wielomianu zaczyna się od lewej strony poniżej osi OX. A więc $$x \in [2,11] \cup [16,\infty).$$
\rozwStop
\odpStart
$x \in [2,11] \cup [16,\infty)$
\odpStop
\testStart
A.$x \in [2,11] \cup [16,\infty)$\\
B.$x \in (2,11) \cup [16,\infty)$\\
C.$x \in (2,11] \cup [16,\infty)$\\
D.$x \in [2,11) \cup [16,\infty)$\\
E.$x \in [2,11] \cup (16,\infty)$\\
F.$x \in (2,11) \cup (16,\infty)$\\
G.$x \in [2,11) \cup (16,\infty)$\\
H.$x \in (2,11] \cup (16,\infty)$
\testStop
\kluczStart
A
\kluczStop



\zadStart{Zadanie z Wikieł Z 1.62 a) moja wersja nr 285}

Rozwiązać nierówności $(x-2)(x-11)(x-17)\ge0$.
\zadStop
\rozwStart{Patryk Wirkus}{}
Miejsca zerowe naszego wielomianu to: $2, 11, 17$.\\
Wielomian jest stopnia nieparzystego, ponadto znak współczynnika przy\linebreak najwyższej potędze x jest dodatni.\\ W związku z tym wykres wielomianu zaczyna się od lewej strony poniżej osi OX. A więc $$x \in [2,11] \cup [17,\infty).$$
\rozwStop
\odpStart
$x \in [2,11] \cup [17,\infty)$
\odpStop
\testStart
A.$x \in [2,11] \cup [17,\infty)$\\
B.$x \in (2,11) \cup [17,\infty)$\\
C.$x \in (2,11] \cup [17,\infty)$\\
D.$x \in [2,11) \cup [17,\infty)$\\
E.$x \in [2,11] \cup (17,\infty)$\\
F.$x \in (2,11) \cup (17,\infty)$\\
G.$x \in [2,11) \cup (17,\infty)$\\
H.$x \in (2,11] \cup (17,\infty)$
\testStop
\kluczStart
A
\kluczStop



\zadStart{Zadanie z Wikieł Z 1.62 a) moja wersja nr 286}

Rozwiązać nierówności $(x-2)(x-11)(x-18)\ge0$.
\zadStop
\rozwStart{Patryk Wirkus}{}
Miejsca zerowe naszego wielomianu to: $2, 11, 18$.\\
Wielomian jest stopnia nieparzystego, ponadto znak współczynnika przy\linebreak najwyższej potędze x jest dodatni.\\ W związku z tym wykres wielomianu zaczyna się od lewej strony poniżej osi OX. A więc $$x \in [2,11] \cup [18,\infty).$$
\rozwStop
\odpStart
$x \in [2,11] \cup [18,\infty)$
\odpStop
\testStart
A.$x \in [2,11] \cup [18,\infty)$\\
B.$x \in (2,11) \cup [18,\infty)$\\
C.$x \in (2,11] \cup [18,\infty)$\\
D.$x \in [2,11) \cup [18,\infty)$\\
E.$x \in [2,11] \cup (18,\infty)$\\
F.$x \in (2,11) \cup (18,\infty)$\\
G.$x \in [2,11) \cup (18,\infty)$\\
H.$x \in (2,11] \cup (18,\infty)$
\testStop
\kluczStart
A
\kluczStop



\zadStart{Zadanie z Wikieł Z 1.62 a) moja wersja nr 287}

Rozwiązać nierówności $(x-2)(x-11)(x-19)\ge0$.
\zadStop
\rozwStart{Patryk Wirkus}{}
Miejsca zerowe naszego wielomianu to: $2, 11, 19$.\\
Wielomian jest stopnia nieparzystego, ponadto znak współczynnika przy\linebreak najwyższej potędze x jest dodatni.\\ W związku z tym wykres wielomianu zaczyna się od lewej strony poniżej osi OX. A więc $$x \in [2,11] \cup [19,\infty).$$
\rozwStop
\odpStart
$x \in [2,11] \cup [19,\infty)$
\odpStop
\testStart
A.$x \in [2,11] \cup [19,\infty)$\\
B.$x \in (2,11) \cup [19,\infty)$\\
C.$x \in (2,11] \cup [19,\infty)$\\
D.$x \in [2,11) \cup [19,\infty)$\\
E.$x \in [2,11] \cup (19,\infty)$\\
F.$x \in (2,11) \cup (19,\infty)$\\
G.$x \in [2,11) \cup (19,\infty)$\\
H.$x \in (2,11] \cup (19,\infty)$
\testStop
\kluczStart
A
\kluczStop



\zadStart{Zadanie z Wikieł Z 1.62 a) moja wersja nr 288}

Rozwiązać nierówności $(x-2)(x-11)(x-20)\ge0$.
\zadStop
\rozwStart{Patryk Wirkus}{}
Miejsca zerowe naszego wielomianu to: $2, 11, 20$.\\
Wielomian jest stopnia nieparzystego, ponadto znak współczynnika przy\linebreak najwyższej potędze x jest dodatni.\\ W związku z tym wykres wielomianu zaczyna się od lewej strony poniżej osi OX. A więc $$x \in [2,11] \cup [20,\infty).$$
\rozwStop
\odpStart
$x \in [2,11] \cup [20,\infty)$
\odpStop
\testStart
A.$x \in [2,11] \cup [20,\infty)$\\
B.$x \in (2,11) \cup [20,\infty)$\\
C.$x \in (2,11] \cup [20,\infty)$\\
D.$x \in [2,11) \cup [20,\infty)$\\
E.$x \in [2,11] \cup (20,\infty)$\\
F.$x \in (2,11) \cup (20,\infty)$\\
G.$x \in [2,11) \cup (20,\infty)$\\
H.$x \in (2,11] \cup (20,\infty)$
\testStop
\kluczStart
A
\kluczStop



\zadStart{Zadanie z Wikieł Z 1.62 a) moja wersja nr 289}

Rozwiązać nierówności $(x-2)(x-12)(x-13)\ge0$.
\zadStop
\rozwStart{Patryk Wirkus}{}
Miejsca zerowe naszego wielomianu to: $2, 12, 13$.\\
Wielomian jest stopnia nieparzystego, ponadto znak współczynnika przy\linebreak najwyższej potędze x jest dodatni.\\ W związku z tym wykres wielomianu zaczyna się od lewej strony poniżej osi OX. A więc $$x \in [2,12] \cup [13,\infty).$$
\rozwStop
\odpStart
$x \in [2,12] \cup [13,\infty)$
\odpStop
\testStart
A.$x \in [2,12] \cup [13,\infty)$\\
B.$x \in (2,12) \cup [13,\infty)$\\
C.$x \in (2,12] \cup [13,\infty)$\\
D.$x \in [2,12) \cup [13,\infty)$\\
E.$x \in [2,12] \cup (13,\infty)$\\
F.$x \in (2,12) \cup (13,\infty)$\\
G.$x \in [2,12) \cup (13,\infty)$\\
H.$x \in (2,12] \cup (13,\infty)$
\testStop
\kluczStart
A
\kluczStop



\zadStart{Zadanie z Wikieł Z 1.62 a) moja wersja nr 290}

Rozwiązać nierówności $(x-2)(x-12)(x-14)\ge0$.
\zadStop
\rozwStart{Patryk Wirkus}{}
Miejsca zerowe naszego wielomianu to: $2, 12, 14$.\\
Wielomian jest stopnia nieparzystego, ponadto znak współczynnika przy\linebreak najwyższej potędze x jest dodatni.\\ W związku z tym wykres wielomianu zaczyna się od lewej strony poniżej osi OX. A więc $$x \in [2,12] \cup [14,\infty).$$
\rozwStop
\odpStart
$x \in [2,12] \cup [14,\infty)$
\odpStop
\testStart
A.$x \in [2,12] \cup [14,\infty)$\\
B.$x \in (2,12) \cup [14,\infty)$\\
C.$x \in (2,12] \cup [14,\infty)$\\
D.$x \in [2,12) \cup [14,\infty)$\\
E.$x \in [2,12] \cup (14,\infty)$\\
F.$x \in (2,12) \cup (14,\infty)$\\
G.$x \in [2,12) \cup (14,\infty)$\\
H.$x \in (2,12] \cup (14,\infty)$
\testStop
\kluczStart
A
\kluczStop



\zadStart{Zadanie z Wikieł Z 1.62 a) moja wersja nr 291}

Rozwiązać nierówności $(x-2)(x-12)(x-15)\ge0$.
\zadStop
\rozwStart{Patryk Wirkus}{}
Miejsca zerowe naszego wielomianu to: $2, 12, 15$.\\
Wielomian jest stopnia nieparzystego, ponadto znak współczynnika przy\linebreak najwyższej potędze x jest dodatni.\\ W związku z tym wykres wielomianu zaczyna się od lewej strony poniżej osi OX. A więc $$x \in [2,12] \cup [15,\infty).$$
\rozwStop
\odpStart
$x \in [2,12] \cup [15,\infty)$
\odpStop
\testStart
A.$x \in [2,12] \cup [15,\infty)$\\
B.$x \in (2,12) \cup [15,\infty)$\\
C.$x \in (2,12] \cup [15,\infty)$\\
D.$x \in [2,12) \cup [15,\infty)$\\
E.$x \in [2,12] \cup (15,\infty)$\\
F.$x \in (2,12) \cup (15,\infty)$\\
G.$x \in [2,12) \cup (15,\infty)$\\
H.$x \in (2,12] \cup (15,\infty)$
\testStop
\kluczStart
A
\kluczStop



\zadStart{Zadanie z Wikieł Z 1.62 a) moja wersja nr 292}

Rozwiązać nierówności $(x-2)(x-12)(x-16)\ge0$.
\zadStop
\rozwStart{Patryk Wirkus}{}
Miejsca zerowe naszego wielomianu to: $2, 12, 16$.\\
Wielomian jest stopnia nieparzystego, ponadto znak współczynnika przy\linebreak najwyższej potędze x jest dodatni.\\ W związku z tym wykres wielomianu zaczyna się od lewej strony poniżej osi OX. A więc $$x \in [2,12] \cup [16,\infty).$$
\rozwStop
\odpStart
$x \in [2,12] \cup [16,\infty)$
\odpStop
\testStart
A.$x \in [2,12] \cup [16,\infty)$\\
B.$x \in (2,12) \cup [16,\infty)$\\
C.$x \in (2,12] \cup [16,\infty)$\\
D.$x \in [2,12) \cup [16,\infty)$\\
E.$x \in [2,12] \cup (16,\infty)$\\
F.$x \in (2,12) \cup (16,\infty)$\\
G.$x \in [2,12) \cup (16,\infty)$\\
H.$x \in (2,12] \cup (16,\infty)$
\testStop
\kluczStart
A
\kluczStop



\zadStart{Zadanie z Wikieł Z 1.62 a) moja wersja nr 293}

Rozwiązać nierówności $(x-2)(x-12)(x-17)\ge0$.
\zadStop
\rozwStart{Patryk Wirkus}{}
Miejsca zerowe naszego wielomianu to: $2, 12, 17$.\\
Wielomian jest stopnia nieparzystego, ponadto znak współczynnika przy\linebreak najwyższej potędze x jest dodatni.\\ W związku z tym wykres wielomianu zaczyna się od lewej strony poniżej osi OX. A więc $$x \in [2,12] \cup [17,\infty).$$
\rozwStop
\odpStart
$x \in [2,12] \cup [17,\infty)$
\odpStop
\testStart
A.$x \in [2,12] \cup [17,\infty)$\\
B.$x \in (2,12) \cup [17,\infty)$\\
C.$x \in (2,12] \cup [17,\infty)$\\
D.$x \in [2,12) \cup [17,\infty)$\\
E.$x \in [2,12] \cup (17,\infty)$\\
F.$x \in (2,12) \cup (17,\infty)$\\
G.$x \in [2,12) \cup (17,\infty)$\\
H.$x \in (2,12] \cup (17,\infty)$
\testStop
\kluczStart
A
\kluczStop



\zadStart{Zadanie z Wikieł Z 1.62 a) moja wersja nr 294}

Rozwiązać nierówności $(x-2)(x-12)(x-18)\ge0$.
\zadStop
\rozwStart{Patryk Wirkus}{}
Miejsca zerowe naszego wielomianu to: $2, 12, 18$.\\
Wielomian jest stopnia nieparzystego, ponadto znak współczynnika przy\linebreak najwyższej potędze x jest dodatni.\\ W związku z tym wykres wielomianu zaczyna się od lewej strony poniżej osi OX. A więc $$x \in [2,12] \cup [18,\infty).$$
\rozwStop
\odpStart
$x \in [2,12] \cup [18,\infty)$
\odpStop
\testStart
A.$x \in [2,12] \cup [18,\infty)$\\
B.$x \in (2,12) \cup [18,\infty)$\\
C.$x \in (2,12] \cup [18,\infty)$\\
D.$x \in [2,12) \cup [18,\infty)$\\
E.$x \in [2,12] \cup (18,\infty)$\\
F.$x \in (2,12) \cup (18,\infty)$\\
G.$x \in [2,12) \cup (18,\infty)$\\
H.$x \in (2,12] \cup (18,\infty)$
\testStop
\kluczStart
A
\kluczStop



\zadStart{Zadanie z Wikieł Z 1.62 a) moja wersja nr 295}

Rozwiązać nierówności $(x-2)(x-12)(x-19)\ge0$.
\zadStop
\rozwStart{Patryk Wirkus}{}
Miejsca zerowe naszego wielomianu to: $2, 12, 19$.\\
Wielomian jest stopnia nieparzystego, ponadto znak współczynnika przy\linebreak najwyższej potędze x jest dodatni.\\ W związku z tym wykres wielomianu zaczyna się od lewej strony poniżej osi OX. A więc $$x \in [2,12] \cup [19,\infty).$$
\rozwStop
\odpStart
$x \in [2,12] \cup [19,\infty)$
\odpStop
\testStart
A.$x \in [2,12] \cup [19,\infty)$\\
B.$x \in (2,12) \cup [19,\infty)$\\
C.$x \in (2,12] \cup [19,\infty)$\\
D.$x \in [2,12) \cup [19,\infty)$\\
E.$x \in [2,12] \cup (19,\infty)$\\
F.$x \in (2,12) \cup (19,\infty)$\\
G.$x \in [2,12) \cup (19,\infty)$\\
H.$x \in (2,12] \cup (19,\infty)$
\testStop
\kluczStart
A
\kluczStop



\zadStart{Zadanie z Wikieł Z 1.62 a) moja wersja nr 296}

Rozwiązać nierówności $(x-2)(x-12)(x-20)\ge0$.
\zadStop
\rozwStart{Patryk Wirkus}{}
Miejsca zerowe naszego wielomianu to: $2, 12, 20$.\\
Wielomian jest stopnia nieparzystego, ponadto znak współczynnika przy\linebreak najwyższej potędze x jest dodatni.\\ W związku z tym wykres wielomianu zaczyna się od lewej strony poniżej osi OX. A więc $$x \in [2,12] \cup [20,\infty).$$
\rozwStop
\odpStart
$x \in [2,12] \cup [20,\infty)$
\odpStop
\testStart
A.$x \in [2,12] \cup [20,\infty)$\\
B.$x \in (2,12) \cup [20,\infty)$\\
C.$x \in (2,12] \cup [20,\infty)$\\
D.$x \in [2,12) \cup [20,\infty)$\\
E.$x \in [2,12] \cup (20,\infty)$\\
F.$x \in (2,12) \cup (20,\infty)$\\
G.$x \in [2,12) \cup (20,\infty)$\\
H.$x \in (2,12] \cup (20,\infty)$
\testStop
\kluczStart
A
\kluczStop



\zadStart{Zadanie z Wikieł Z 1.62 a) moja wersja nr 297}

Rozwiązać nierówności $(x-2)(x-13)(x-14)\ge0$.
\zadStop
\rozwStart{Patryk Wirkus}{}
Miejsca zerowe naszego wielomianu to: $2, 13, 14$.\\
Wielomian jest stopnia nieparzystego, ponadto znak współczynnika przy\linebreak najwyższej potędze x jest dodatni.\\ W związku z tym wykres wielomianu zaczyna się od lewej strony poniżej osi OX. A więc $$x \in [2,13] \cup [14,\infty).$$
\rozwStop
\odpStart
$x \in [2,13] \cup [14,\infty)$
\odpStop
\testStart
A.$x \in [2,13] \cup [14,\infty)$\\
B.$x \in (2,13) \cup [14,\infty)$\\
C.$x \in (2,13] \cup [14,\infty)$\\
D.$x \in [2,13) \cup [14,\infty)$\\
E.$x \in [2,13] \cup (14,\infty)$\\
F.$x \in (2,13) \cup (14,\infty)$\\
G.$x \in [2,13) \cup (14,\infty)$\\
H.$x \in (2,13] \cup (14,\infty)$
\testStop
\kluczStart
A
\kluczStop



\zadStart{Zadanie z Wikieł Z 1.62 a) moja wersja nr 298}

Rozwiązać nierówności $(x-2)(x-13)(x-15)\ge0$.
\zadStop
\rozwStart{Patryk Wirkus}{}
Miejsca zerowe naszego wielomianu to: $2, 13, 15$.\\
Wielomian jest stopnia nieparzystego, ponadto znak współczynnika przy\linebreak najwyższej potędze x jest dodatni.\\ W związku z tym wykres wielomianu zaczyna się od lewej strony poniżej osi OX. A więc $$x \in [2,13] \cup [15,\infty).$$
\rozwStop
\odpStart
$x \in [2,13] \cup [15,\infty)$
\odpStop
\testStart
A.$x \in [2,13] \cup [15,\infty)$\\
B.$x \in (2,13) \cup [15,\infty)$\\
C.$x \in (2,13] \cup [15,\infty)$\\
D.$x \in [2,13) \cup [15,\infty)$\\
E.$x \in [2,13] \cup (15,\infty)$\\
F.$x \in (2,13) \cup (15,\infty)$\\
G.$x \in [2,13) \cup (15,\infty)$\\
H.$x \in (2,13] \cup (15,\infty)$
\testStop
\kluczStart
A
\kluczStop



\zadStart{Zadanie z Wikieł Z 1.62 a) moja wersja nr 299}

Rozwiązać nierówności $(x-2)(x-13)(x-16)\ge0$.
\zadStop
\rozwStart{Patryk Wirkus}{}
Miejsca zerowe naszego wielomianu to: $2, 13, 16$.\\
Wielomian jest stopnia nieparzystego, ponadto znak współczynnika przy\linebreak najwyższej potędze x jest dodatni.\\ W związku z tym wykres wielomianu zaczyna się od lewej strony poniżej osi OX. A więc $$x \in [2,13] \cup [16,\infty).$$
\rozwStop
\odpStart
$x \in [2,13] \cup [16,\infty)$
\odpStop
\testStart
A.$x \in [2,13] \cup [16,\infty)$\\
B.$x \in (2,13) \cup [16,\infty)$\\
C.$x \in (2,13] \cup [16,\infty)$\\
D.$x \in [2,13) \cup [16,\infty)$\\
E.$x \in [2,13] \cup (16,\infty)$\\
F.$x \in (2,13) \cup (16,\infty)$\\
G.$x \in [2,13) \cup (16,\infty)$\\
H.$x \in (2,13] \cup (16,\infty)$
\testStop
\kluczStart
A
\kluczStop



\zadStart{Zadanie z Wikieł Z 1.62 a) moja wersja nr 300}

Rozwiązać nierówności $(x-2)(x-13)(x-17)\ge0$.
\zadStop
\rozwStart{Patryk Wirkus}{}
Miejsca zerowe naszego wielomianu to: $2, 13, 17$.\\
Wielomian jest stopnia nieparzystego, ponadto znak współczynnika przy\linebreak najwyższej potędze x jest dodatni.\\ W związku z tym wykres wielomianu zaczyna się od lewej strony poniżej osi OX. A więc $$x \in [2,13] \cup [17,\infty).$$
\rozwStop
\odpStart
$x \in [2,13] \cup [17,\infty)$
\odpStop
\testStart
A.$x \in [2,13] \cup [17,\infty)$\\
B.$x \in (2,13) \cup [17,\infty)$\\
C.$x \in (2,13] \cup [17,\infty)$\\
D.$x \in [2,13) \cup [17,\infty)$\\
E.$x \in [2,13] \cup (17,\infty)$\\
F.$x \in (2,13) \cup (17,\infty)$\\
G.$x \in [2,13) \cup (17,\infty)$\\
H.$x \in (2,13] \cup (17,\infty)$
\testStop
\kluczStart
A
\kluczStop



\zadStart{Zadanie z Wikieł Z 1.62 a) moja wersja nr 301}

Rozwiązać nierówności $(x-2)(x-13)(x-18)\ge0$.
\zadStop
\rozwStart{Patryk Wirkus}{}
Miejsca zerowe naszego wielomianu to: $2, 13, 18$.\\
Wielomian jest stopnia nieparzystego, ponadto znak współczynnika przy\linebreak najwyższej potędze x jest dodatni.\\ W związku z tym wykres wielomianu zaczyna się od lewej strony poniżej osi OX. A więc $$x \in [2,13] \cup [18,\infty).$$
\rozwStop
\odpStart
$x \in [2,13] \cup [18,\infty)$
\odpStop
\testStart
A.$x \in [2,13] \cup [18,\infty)$\\
B.$x \in (2,13) \cup [18,\infty)$\\
C.$x \in (2,13] \cup [18,\infty)$\\
D.$x \in [2,13) \cup [18,\infty)$\\
E.$x \in [2,13] \cup (18,\infty)$\\
F.$x \in (2,13) \cup (18,\infty)$\\
G.$x \in [2,13) \cup (18,\infty)$\\
H.$x \in (2,13] \cup (18,\infty)$
\testStop
\kluczStart
A
\kluczStop



\zadStart{Zadanie z Wikieł Z 1.62 a) moja wersja nr 302}

Rozwiązać nierówności $(x-2)(x-13)(x-19)\ge0$.
\zadStop
\rozwStart{Patryk Wirkus}{}
Miejsca zerowe naszego wielomianu to: $2, 13, 19$.\\
Wielomian jest stopnia nieparzystego, ponadto znak współczynnika przy\linebreak najwyższej potędze x jest dodatni.\\ W związku z tym wykres wielomianu zaczyna się od lewej strony poniżej osi OX. A więc $$x \in [2,13] \cup [19,\infty).$$
\rozwStop
\odpStart
$x \in [2,13] \cup [19,\infty)$
\odpStop
\testStart
A.$x \in [2,13] \cup [19,\infty)$\\
B.$x \in (2,13) \cup [19,\infty)$\\
C.$x \in (2,13] \cup [19,\infty)$\\
D.$x \in [2,13) \cup [19,\infty)$\\
E.$x \in [2,13] \cup (19,\infty)$\\
F.$x \in (2,13) \cup (19,\infty)$\\
G.$x \in [2,13) \cup (19,\infty)$\\
H.$x \in (2,13] \cup (19,\infty)$
\testStop
\kluczStart
A
\kluczStop



\zadStart{Zadanie z Wikieł Z 1.62 a) moja wersja nr 303}

Rozwiązać nierówności $(x-2)(x-13)(x-20)\ge0$.
\zadStop
\rozwStart{Patryk Wirkus}{}
Miejsca zerowe naszego wielomianu to: $2, 13, 20$.\\
Wielomian jest stopnia nieparzystego, ponadto znak współczynnika przy\linebreak najwyższej potędze x jest dodatni.\\ W związku z tym wykres wielomianu zaczyna się od lewej strony poniżej osi OX. A więc $$x \in [2,13] \cup [20,\infty).$$
\rozwStop
\odpStart
$x \in [2,13] \cup [20,\infty)$
\odpStop
\testStart
A.$x \in [2,13] \cup [20,\infty)$\\
B.$x \in (2,13) \cup [20,\infty)$\\
C.$x \in (2,13] \cup [20,\infty)$\\
D.$x \in [2,13) \cup [20,\infty)$\\
E.$x \in [2,13] \cup (20,\infty)$\\
F.$x \in (2,13) \cup (20,\infty)$\\
G.$x \in [2,13) \cup (20,\infty)$\\
H.$x \in (2,13] \cup (20,\infty)$
\testStop
\kluczStart
A
\kluczStop



\zadStart{Zadanie z Wikieł Z 1.62 a) moja wersja nr 304}

Rozwiązać nierówności $(x-2)(x-14)(x-15)\ge0$.
\zadStop
\rozwStart{Patryk Wirkus}{}
Miejsca zerowe naszego wielomianu to: $2, 14, 15$.\\
Wielomian jest stopnia nieparzystego, ponadto znak współczynnika przy\linebreak najwyższej potędze x jest dodatni.\\ W związku z tym wykres wielomianu zaczyna się od lewej strony poniżej osi OX. A więc $$x \in [2,14] \cup [15,\infty).$$
\rozwStop
\odpStart
$x \in [2,14] \cup [15,\infty)$
\odpStop
\testStart
A.$x \in [2,14] \cup [15,\infty)$\\
B.$x \in (2,14) \cup [15,\infty)$\\
C.$x \in (2,14] \cup [15,\infty)$\\
D.$x \in [2,14) \cup [15,\infty)$\\
E.$x \in [2,14] \cup (15,\infty)$\\
F.$x \in (2,14) \cup (15,\infty)$\\
G.$x \in [2,14) \cup (15,\infty)$\\
H.$x \in (2,14] \cup (15,\infty)$
\testStop
\kluczStart
A
\kluczStop



\zadStart{Zadanie z Wikieł Z 1.62 a) moja wersja nr 305}

Rozwiązać nierówności $(x-2)(x-14)(x-16)\ge0$.
\zadStop
\rozwStart{Patryk Wirkus}{}
Miejsca zerowe naszego wielomianu to: $2, 14, 16$.\\
Wielomian jest stopnia nieparzystego, ponadto znak współczynnika przy\linebreak najwyższej potędze x jest dodatni.\\ W związku z tym wykres wielomianu zaczyna się od lewej strony poniżej osi OX. A więc $$x \in [2,14] \cup [16,\infty).$$
\rozwStop
\odpStart
$x \in [2,14] \cup [16,\infty)$
\odpStop
\testStart
A.$x \in [2,14] \cup [16,\infty)$\\
B.$x \in (2,14) \cup [16,\infty)$\\
C.$x \in (2,14] \cup [16,\infty)$\\
D.$x \in [2,14) \cup [16,\infty)$\\
E.$x \in [2,14] \cup (16,\infty)$\\
F.$x \in (2,14) \cup (16,\infty)$\\
G.$x \in [2,14) \cup (16,\infty)$\\
H.$x \in (2,14] \cup (16,\infty)$
\testStop
\kluczStart
A
\kluczStop



\zadStart{Zadanie z Wikieł Z 1.62 a) moja wersja nr 306}

Rozwiązać nierówności $(x-2)(x-14)(x-17)\ge0$.
\zadStop
\rozwStart{Patryk Wirkus}{}
Miejsca zerowe naszego wielomianu to: $2, 14, 17$.\\
Wielomian jest stopnia nieparzystego, ponadto znak współczynnika przy\linebreak najwyższej potędze x jest dodatni.\\ W związku z tym wykres wielomianu zaczyna się od lewej strony poniżej osi OX. A więc $$x \in [2,14] \cup [17,\infty).$$
\rozwStop
\odpStart
$x \in [2,14] \cup [17,\infty)$
\odpStop
\testStart
A.$x \in [2,14] \cup [17,\infty)$\\
B.$x \in (2,14) \cup [17,\infty)$\\
C.$x \in (2,14] \cup [17,\infty)$\\
D.$x \in [2,14) \cup [17,\infty)$\\
E.$x \in [2,14] \cup (17,\infty)$\\
F.$x \in (2,14) \cup (17,\infty)$\\
G.$x \in [2,14) \cup (17,\infty)$\\
H.$x \in (2,14] \cup (17,\infty)$
\testStop
\kluczStart
A
\kluczStop



\zadStart{Zadanie z Wikieł Z 1.62 a) moja wersja nr 307}

Rozwiązać nierówności $(x-2)(x-14)(x-18)\ge0$.
\zadStop
\rozwStart{Patryk Wirkus}{}
Miejsca zerowe naszego wielomianu to: $2, 14, 18$.\\
Wielomian jest stopnia nieparzystego, ponadto znak współczynnika przy\linebreak najwyższej potędze x jest dodatni.\\ W związku z tym wykres wielomianu zaczyna się od lewej strony poniżej osi OX. A więc $$x \in [2,14] \cup [18,\infty).$$
\rozwStop
\odpStart
$x \in [2,14] \cup [18,\infty)$
\odpStop
\testStart
A.$x \in [2,14] \cup [18,\infty)$\\
B.$x \in (2,14) \cup [18,\infty)$\\
C.$x \in (2,14] \cup [18,\infty)$\\
D.$x \in [2,14) \cup [18,\infty)$\\
E.$x \in [2,14] \cup (18,\infty)$\\
F.$x \in (2,14) \cup (18,\infty)$\\
G.$x \in [2,14) \cup (18,\infty)$\\
H.$x \in (2,14] \cup (18,\infty)$
\testStop
\kluczStart
A
\kluczStop



\zadStart{Zadanie z Wikieł Z 1.62 a) moja wersja nr 308}

Rozwiązać nierówności $(x-2)(x-14)(x-19)\ge0$.
\zadStop
\rozwStart{Patryk Wirkus}{}
Miejsca zerowe naszego wielomianu to: $2, 14, 19$.\\
Wielomian jest stopnia nieparzystego, ponadto znak współczynnika przy\linebreak najwyższej potędze x jest dodatni.\\ W związku z tym wykres wielomianu zaczyna się od lewej strony poniżej osi OX. A więc $$x \in [2,14] \cup [19,\infty).$$
\rozwStop
\odpStart
$x \in [2,14] \cup [19,\infty)$
\odpStop
\testStart
A.$x \in [2,14] \cup [19,\infty)$\\
B.$x \in (2,14) \cup [19,\infty)$\\
C.$x \in (2,14] \cup [19,\infty)$\\
D.$x \in [2,14) \cup [19,\infty)$\\
E.$x \in [2,14] \cup (19,\infty)$\\
F.$x \in (2,14) \cup (19,\infty)$\\
G.$x \in [2,14) \cup (19,\infty)$\\
H.$x \in (2,14] \cup (19,\infty)$
\testStop
\kluczStart
A
\kluczStop



\zadStart{Zadanie z Wikieł Z 1.62 a) moja wersja nr 309}

Rozwiązać nierówności $(x-2)(x-14)(x-20)\ge0$.
\zadStop
\rozwStart{Patryk Wirkus}{}
Miejsca zerowe naszego wielomianu to: $2, 14, 20$.\\
Wielomian jest stopnia nieparzystego, ponadto znak współczynnika przy\linebreak najwyższej potędze x jest dodatni.\\ W związku z tym wykres wielomianu zaczyna się od lewej strony poniżej osi OX. A więc $$x \in [2,14] \cup [20,\infty).$$
\rozwStop
\odpStart
$x \in [2,14] \cup [20,\infty)$
\odpStop
\testStart
A.$x \in [2,14] \cup [20,\infty)$\\
B.$x \in (2,14) \cup [20,\infty)$\\
C.$x \in (2,14] \cup [20,\infty)$\\
D.$x \in [2,14) \cup [20,\infty)$\\
E.$x \in [2,14] \cup (20,\infty)$\\
F.$x \in (2,14) \cup (20,\infty)$\\
G.$x \in [2,14) \cup (20,\infty)$\\
H.$x \in (2,14] \cup (20,\infty)$
\testStop
\kluczStart
A
\kluczStop



\zadStart{Zadanie z Wikieł Z 1.62 a) moja wersja nr 310}

Rozwiązać nierówności $(x-2)(x-15)(x-16)\ge0$.
\zadStop
\rozwStart{Patryk Wirkus}{}
Miejsca zerowe naszego wielomianu to: $2, 15, 16$.\\
Wielomian jest stopnia nieparzystego, ponadto znak współczynnika przy\linebreak najwyższej potędze x jest dodatni.\\ W związku z tym wykres wielomianu zaczyna się od lewej strony poniżej osi OX. A więc $$x \in [2,15] \cup [16,\infty).$$
\rozwStop
\odpStart
$x \in [2,15] \cup [16,\infty)$
\odpStop
\testStart
A.$x \in [2,15] \cup [16,\infty)$\\
B.$x \in (2,15) \cup [16,\infty)$\\
C.$x \in (2,15] \cup [16,\infty)$\\
D.$x \in [2,15) \cup [16,\infty)$\\
E.$x \in [2,15] \cup (16,\infty)$\\
F.$x \in (2,15) \cup (16,\infty)$\\
G.$x \in [2,15) \cup (16,\infty)$\\
H.$x \in (2,15] \cup (16,\infty)$
\testStop
\kluczStart
A
\kluczStop



\zadStart{Zadanie z Wikieł Z 1.62 a) moja wersja nr 311}

Rozwiązać nierówności $(x-2)(x-15)(x-17)\ge0$.
\zadStop
\rozwStart{Patryk Wirkus}{}
Miejsca zerowe naszego wielomianu to: $2, 15, 17$.\\
Wielomian jest stopnia nieparzystego, ponadto znak współczynnika przy\linebreak najwyższej potędze x jest dodatni.\\ W związku z tym wykres wielomianu zaczyna się od lewej strony poniżej osi OX. A więc $$x \in [2,15] \cup [17,\infty).$$
\rozwStop
\odpStart
$x \in [2,15] \cup [17,\infty)$
\odpStop
\testStart
A.$x \in [2,15] \cup [17,\infty)$\\
B.$x \in (2,15) \cup [17,\infty)$\\
C.$x \in (2,15] \cup [17,\infty)$\\
D.$x \in [2,15) \cup [17,\infty)$\\
E.$x \in [2,15] \cup (17,\infty)$\\
F.$x \in (2,15) \cup (17,\infty)$\\
G.$x \in [2,15) \cup (17,\infty)$\\
H.$x \in (2,15] \cup (17,\infty)$
\testStop
\kluczStart
A
\kluczStop



\zadStart{Zadanie z Wikieł Z 1.62 a) moja wersja nr 312}

Rozwiązać nierówności $(x-2)(x-15)(x-18)\ge0$.
\zadStop
\rozwStart{Patryk Wirkus}{}
Miejsca zerowe naszego wielomianu to: $2, 15, 18$.\\
Wielomian jest stopnia nieparzystego, ponadto znak współczynnika przy\linebreak najwyższej potędze x jest dodatni.\\ W związku z tym wykres wielomianu zaczyna się od lewej strony poniżej osi OX. A więc $$x \in [2,15] \cup [18,\infty).$$
\rozwStop
\odpStart
$x \in [2,15] \cup [18,\infty)$
\odpStop
\testStart
A.$x \in [2,15] \cup [18,\infty)$\\
B.$x \in (2,15) \cup [18,\infty)$\\
C.$x \in (2,15] \cup [18,\infty)$\\
D.$x \in [2,15) \cup [18,\infty)$\\
E.$x \in [2,15] \cup (18,\infty)$\\
F.$x \in (2,15) \cup (18,\infty)$\\
G.$x \in [2,15) \cup (18,\infty)$\\
H.$x \in (2,15] \cup (18,\infty)$
\testStop
\kluczStart
A
\kluczStop



\zadStart{Zadanie z Wikieł Z 1.62 a) moja wersja nr 313}

Rozwiązać nierówności $(x-2)(x-15)(x-19)\ge0$.
\zadStop
\rozwStart{Patryk Wirkus}{}
Miejsca zerowe naszego wielomianu to: $2, 15, 19$.\\
Wielomian jest stopnia nieparzystego, ponadto znak współczynnika przy\linebreak najwyższej potędze x jest dodatni.\\ W związku z tym wykres wielomianu zaczyna się od lewej strony poniżej osi OX. A więc $$x \in [2,15] \cup [19,\infty).$$
\rozwStop
\odpStart
$x \in [2,15] \cup [19,\infty)$
\odpStop
\testStart
A.$x \in [2,15] \cup [19,\infty)$\\
B.$x \in (2,15) \cup [19,\infty)$\\
C.$x \in (2,15] \cup [19,\infty)$\\
D.$x \in [2,15) \cup [19,\infty)$\\
E.$x \in [2,15] \cup (19,\infty)$\\
F.$x \in (2,15) \cup (19,\infty)$\\
G.$x \in [2,15) \cup (19,\infty)$\\
H.$x \in (2,15] \cup (19,\infty)$
\testStop
\kluczStart
A
\kluczStop



\zadStart{Zadanie z Wikieł Z 1.62 a) moja wersja nr 314}

Rozwiązać nierówności $(x-2)(x-15)(x-20)\ge0$.
\zadStop
\rozwStart{Patryk Wirkus}{}
Miejsca zerowe naszego wielomianu to: $2, 15, 20$.\\
Wielomian jest stopnia nieparzystego, ponadto znak współczynnika przy\linebreak najwyższej potędze x jest dodatni.\\ W związku z tym wykres wielomianu zaczyna się od lewej strony poniżej osi OX. A więc $$x \in [2,15] \cup [20,\infty).$$
\rozwStop
\odpStart
$x \in [2,15] \cup [20,\infty)$
\odpStop
\testStart
A.$x \in [2,15] \cup [20,\infty)$\\
B.$x \in (2,15) \cup [20,\infty)$\\
C.$x \in (2,15] \cup [20,\infty)$\\
D.$x \in [2,15) \cup [20,\infty)$\\
E.$x \in [2,15] \cup (20,\infty)$\\
F.$x \in (2,15) \cup (20,\infty)$\\
G.$x \in [2,15) \cup (20,\infty)$\\
H.$x \in (2,15] \cup (20,\infty)$
\testStop
\kluczStart
A
\kluczStop



\zadStart{Zadanie z Wikieł Z 1.62 a) moja wersja nr 315}

Rozwiązać nierówności $(x-2)(x-16)(x-17)\ge0$.
\zadStop
\rozwStart{Patryk Wirkus}{}
Miejsca zerowe naszego wielomianu to: $2, 16, 17$.\\
Wielomian jest stopnia nieparzystego, ponadto znak współczynnika przy\linebreak najwyższej potędze x jest dodatni.\\ W związku z tym wykres wielomianu zaczyna się od lewej strony poniżej osi OX. A więc $$x \in [2,16] \cup [17,\infty).$$
\rozwStop
\odpStart
$x \in [2,16] \cup [17,\infty)$
\odpStop
\testStart
A.$x \in [2,16] \cup [17,\infty)$\\
B.$x \in (2,16) \cup [17,\infty)$\\
C.$x \in (2,16] \cup [17,\infty)$\\
D.$x \in [2,16) \cup [17,\infty)$\\
E.$x \in [2,16] \cup (17,\infty)$\\
F.$x \in (2,16) \cup (17,\infty)$\\
G.$x \in [2,16) \cup (17,\infty)$\\
H.$x \in (2,16] \cup (17,\infty)$
\testStop
\kluczStart
A
\kluczStop



\zadStart{Zadanie z Wikieł Z 1.62 a) moja wersja nr 316}

Rozwiązać nierówności $(x-2)(x-16)(x-18)\ge0$.
\zadStop
\rozwStart{Patryk Wirkus}{}
Miejsca zerowe naszego wielomianu to: $2, 16, 18$.\\
Wielomian jest stopnia nieparzystego, ponadto znak współczynnika przy\linebreak najwyższej potędze x jest dodatni.\\ W związku z tym wykres wielomianu zaczyna się od lewej strony poniżej osi OX. A więc $$x \in [2,16] \cup [18,\infty).$$
\rozwStop
\odpStart
$x \in [2,16] \cup [18,\infty)$
\odpStop
\testStart
A.$x \in [2,16] \cup [18,\infty)$\\
B.$x \in (2,16) \cup [18,\infty)$\\
C.$x \in (2,16] \cup [18,\infty)$\\
D.$x \in [2,16) \cup [18,\infty)$\\
E.$x \in [2,16] \cup (18,\infty)$\\
F.$x \in (2,16) \cup (18,\infty)$\\
G.$x \in [2,16) \cup (18,\infty)$\\
H.$x \in (2,16] \cup (18,\infty)$
\testStop
\kluczStart
A
\kluczStop



\zadStart{Zadanie z Wikieł Z 1.62 a) moja wersja nr 317}

Rozwiązać nierówności $(x-2)(x-16)(x-19)\ge0$.
\zadStop
\rozwStart{Patryk Wirkus}{}
Miejsca zerowe naszego wielomianu to: $2, 16, 19$.\\
Wielomian jest stopnia nieparzystego, ponadto znak współczynnika przy\linebreak najwyższej potędze x jest dodatni.\\ W związku z tym wykres wielomianu zaczyna się od lewej strony poniżej osi OX. A więc $$x \in [2,16] \cup [19,\infty).$$
\rozwStop
\odpStart
$x \in [2,16] \cup [19,\infty)$
\odpStop
\testStart
A.$x \in [2,16] \cup [19,\infty)$\\
B.$x \in (2,16) \cup [19,\infty)$\\
C.$x \in (2,16] \cup [19,\infty)$\\
D.$x \in [2,16) \cup [19,\infty)$\\
E.$x \in [2,16] \cup (19,\infty)$\\
F.$x \in (2,16) \cup (19,\infty)$\\
G.$x \in [2,16) \cup (19,\infty)$\\
H.$x \in (2,16] \cup (19,\infty)$
\testStop
\kluczStart
A
\kluczStop



\zadStart{Zadanie z Wikieł Z 1.62 a) moja wersja nr 318}

Rozwiązać nierówności $(x-2)(x-16)(x-20)\ge0$.
\zadStop
\rozwStart{Patryk Wirkus}{}
Miejsca zerowe naszego wielomianu to: $2, 16, 20$.\\
Wielomian jest stopnia nieparzystego, ponadto znak współczynnika przy\linebreak najwyższej potędze x jest dodatni.\\ W związku z tym wykres wielomianu zaczyna się od lewej strony poniżej osi OX. A więc $$x \in [2,16] \cup [20,\infty).$$
\rozwStop
\odpStart
$x \in [2,16] \cup [20,\infty)$
\odpStop
\testStart
A.$x \in [2,16] \cup [20,\infty)$\\
B.$x \in (2,16) \cup [20,\infty)$\\
C.$x \in (2,16] \cup [20,\infty)$\\
D.$x \in [2,16) \cup [20,\infty)$\\
E.$x \in [2,16] \cup (20,\infty)$\\
F.$x \in (2,16) \cup (20,\infty)$\\
G.$x \in [2,16) \cup (20,\infty)$\\
H.$x \in (2,16] \cup (20,\infty)$
\testStop
\kluczStart
A
\kluczStop



\zadStart{Zadanie z Wikieł Z 1.62 a) moja wersja nr 319}

Rozwiązać nierówności $(x-2)(x-17)(x-18)\ge0$.
\zadStop
\rozwStart{Patryk Wirkus}{}
Miejsca zerowe naszego wielomianu to: $2, 17, 18$.\\
Wielomian jest stopnia nieparzystego, ponadto znak współczynnika przy\linebreak najwyższej potędze x jest dodatni.\\ W związku z tym wykres wielomianu zaczyna się od lewej strony poniżej osi OX. A więc $$x \in [2,17] \cup [18,\infty).$$
\rozwStop
\odpStart
$x \in [2,17] \cup [18,\infty)$
\odpStop
\testStart
A.$x \in [2,17] \cup [18,\infty)$\\
B.$x \in (2,17) \cup [18,\infty)$\\
C.$x \in (2,17] \cup [18,\infty)$\\
D.$x \in [2,17) \cup [18,\infty)$\\
E.$x \in [2,17] \cup (18,\infty)$\\
F.$x \in (2,17) \cup (18,\infty)$\\
G.$x \in [2,17) \cup (18,\infty)$\\
H.$x \in (2,17] \cup (18,\infty)$
\testStop
\kluczStart
A
\kluczStop



\zadStart{Zadanie z Wikieł Z 1.62 a) moja wersja nr 320}

Rozwiązać nierówności $(x-2)(x-17)(x-19)\ge0$.
\zadStop
\rozwStart{Patryk Wirkus}{}
Miejsca zerowe naszego wielomianu to: $2, 17, 19$.\\
Wielomian jest stopnia nieparzystego, ponadto znak współczynnika przy\linebreak najwyższej potędze x jest dodatni.\\ W związku z tym wykres wielomianu zaczyna się od lewej strony poniżej osi OX. A więc $$x \in [2,17] \cup [19,\infty).$$
\rozwStop
\odpStart
$x \in [2,17] \cup [19,\infty)$
\odpStop
\testStart
A.$x \in [2,17] \cup [19,\infty)$\\
B.$x \in (2,17) \cup [19,\infty)$\\
C.$x \in (2,17] \cup [19,\infty)$\\
D.$x \in [2,17) \cup [19,\infty)$\\
E.$x \in [2,17] \cup (19,\infty)$\\
F.$x \in (2,17) \cup (19,\infty)$\\
G.$x \in [2,17) \cup (19,\infty)$\\
H.$x \in (2,17] \cup (19,\infty)$
\testStop
\kluczStart
A
\kluczStop



\zadStart{Zadanie z Wikieł Z 1.62 a) moja wersja nr 321}

Rozwiązać nierówności $(x-2)(x-17)(x-20)\ge0$.
\zadStop
\rozwStart{Patryk Wirkus}{}
Miejsca zerowe naszego wielomianu to: $2, 17, 20$.\\
Wielomian jest stopnia nieparzystego, ponadto znak współczynnika przy\linebreak najwyższej potędze x jest dodatni.\\ W związku z tym wykres wielomianu zaczyna się od lewej strony poniżej osi OX. A więc $$x \in [2,17] \cup [20,\infty).$$
\rozwStop
\odpStart
$x \in [2,17] \cup [20,\infty)$
\odpStop
\testStart
A.$x \in [2,17] \cup [20,\infty)$\\
B.$x \in (2,17) \cup [20,\infty)$\\
C.$x \in (2,17] \cup [20,\infty)$\\
D.$x \in [2,17) \cup [20,\infty)$\\
E.$x \in [2,17] \cup (20,\infty)$\\
F.$x \in (2,17) \cup (20,\infty)$\\
G.$x \in [2,17) \cup (20,\infty)$\\
H.$x \in (2,17] \cup (20,\infty)$
\testStop
\kluczStart
A
\kluczStop



\zadStart{Zadanie z Wikieł Z 1.62 a) moja wersja nr 322}

Rozwiązać nierówności $(x-2)(x-18)(x-19)\ge0$.
\zadStop
\rozwStart{Patryk Wirkus}{}
Miejsca zerowe naszego wielomianu to: $2, 18, 19$.\\
Wielomian jest stopnia nieparzystego, ponadto znak współczynnika przy\linebreak najwyższej potędze x jest dodatni.\\ W związku z tym wykres wielomianu zaczyna się od lewej strony poniżej osi OX. A więc $$x \in [2,18] \cup [19,\infty).$$
\rozwStop
\odpStart
$x \in [2,18] \cup [19,\infty)$
\odpStop
\testStart
A.$x \in [2,18] \cup [19,\infty)$\\
B.$x \in (2,18) \cup [19,\infty)$\\
C.$x \in (2,18] \cup [19,\infty)$\\
D.$x \in [2,18) \cup [19,\infty)$\\
E.$x \in [2,18] \cup (19,\infty)$\\
F.$x \in (2,18) \cup (19,\infty)$\\
G.$x \in [2,18) \cup (19,\infty)$\\
H.$x \in (2,18] \cup (19,\infty)$
\testStop
\kluczStart
A
\kluczStop



\zadStart{Zadanie z Wikieł Z 1.62 a) moja wersja nr 323}

Rozwiązać nierówności $(x-2)(x-18)(x-20)\ge0$.
\zadStop
\rozwStart{Patryk Wirkus}{}
Miejsca zerowe naszego wielomianu to: $2, 18, 20$.\\
Wielomian jest stopnia nieparzystego, ponadto znak współczynnika przy\linebreak najwyższej potędze x jest dodatni.\\ W związku z tym wykres wielomianu zaczyna się od lewej strony poniżej osi OX. A więc $$x \in [2,18] \cup [20,\infty).$$
\rozwStop
\odpStart
$x \in [2,18] \cup [20,\infty)$
\odpStop
\testStart
A.$x \in [2,18] \cup [20,\infty)$\\
B.$x \in (2,18) \cup [20,\infty)$\\
C.$x \in (2,18] \cup [20,\infty)$\\
D.$x \in [2,18) \cup [20,\infty)$\\
E.$x \in [2,18] \cup (20,\infty)$\\
F.$x \in (2,18) \cup (20,\infty)$\\
G.$x \in [2,18) \cup (20,\infty)$\\
H.$x \in (2,18] \cup (20,\infty)$
\testStop
\kluczStart
A
\kluczStop



\zadStart{Zadanie z Wikieł Z 1.62 a) moja wersja nr 324}

Rozwiązać nierówności $(x-2)(x-19)(x-20)\ge0$.
\zadStop
\rozwStart{Patryk Wirkus}{}
Miejsca zerowe naszego wielomianu to: $2, 19, 20$.\\
Wielomian jest stopnia nieparzystego, ponadto znak współczynnika przy\linebreak najwyższej potędze x jest dodatni.\\ W związku z tym wykres wielomianu zaczyna się od lewej strony poniżej osi OX. A więc $$x \in [2,19] \cup [20,\infty).$$
\rozwStop
\odpStart
$x \in [2,19] \cup [20,\infty)$
\odpStop
\testStart
A.$x \in [2,19] \cup [20,\infty)$\\
B.$x \in (2,19) \cup [20,\infty)$\\
C.$x \in (2,19] \cup [20,\infty)$\\
D.$x \in [2,19) \cup [20,\infty)$\\
E.$x \in [2,19] \cup (20,\infty)$\\
F.$x \in (2,19) \cup (20,\infty)$\\
G.$x \in [2,19) \cup (20,\infty)$\\
H.$x \in (2,19] \cup (20,\infty)$
\testStop
\kluczStart
A
\kluczStop



\zadStart{Zadanie z Wikieł Z 1.62 a) moja wersja nr 325}

Rozwiązać nierówności $(x-3)(x-4)(x-5)\ge0$.
\zadStop
\rozwStart{Patryk Wirkus}{}
Miejsca zerowe naszego wielomianu to: $3, 4, 5$.\\
Wielomian jest stopnia nieparzystego, ponadto znak współczynnika przy\linebreak najwyższej potędze x jest dodatni.\\ W związku z tym wykres wielomianu zaczyna się od lewej strony poniżej osi OX. A więc $$x \in [3,4] \cup [5,\infty).$$
\rozwStop
\odpStart
$x \in [3,4] \cup [5,\infty)$
\odpStop
\testStart
A.$x \in [3,4] \cup [5,\infty)$\\
B.$x \in (3,4) \cup [5,\infty)$\\
C.$x \in (3,4] \cup [5,\infty)$\\
D.$x \in [3,4) \cup [5,\infty)$\\
E.$x \in [3,4] \cup (5,\infty)$\\
F.$x \in (3,4) \cup (5,\infty)$\\
G.$x \in [3,4) \cup (5,\infty)$\\
H.$x \in (3,4] \cup (5,\infty)$
\testStop
\kluczStart
A
\kluczStop



\zadStart{Zadanie z Wikieł Z 1.62 a) moja wersja nr 326}

Rozwiązać nierówności $(x-3)(x-4)(x-6)\ge0$.
\zadStop
\rozwStart{Patryk Wirkus}{}
Miejsca zerowe naszego wielomianu to: $3, 4, 6$.\\
Wielomian jest stopnia nieparzystego, ponadto znak współczynnika przy\linebreak najwyższej potędze x jest dodatni.\\ W związku z tym wykres wielomianu zaczyna się od lewej strony poniżej osi OX. A więc $$x \in [3,4] \cup [6,\infty).$$
\rozwStop
\odpStart
$x \in [3,4] \cup [6,\infty)$
\odpStop
\testStart
A.$x \in [3,4] \cup [6,\infty)$\\
B.$x \in (3,4) \cup [6,\infty)$\\
C.$x \in (3,4] \cup [6,\infty)$\\
D.$x \in [3,4) \cup [6,\infty)$\\
E.$x \in [3,4] \cup (6,\infty)$\\
F.$x \in (3,4) \cup (6,\infty)$\\
G.$x \in [3,4) \cup (6,\infty)$\\
H.$x \in (3,4] \cup (6,\infty)$
\testStop
\kluczStart
A
\kluczStop



\zadStart{Zadanie z Wikieł Z 1.62 a) moja wersja nr 327}

Rozwiązać nierówności $(x-3)(x-4)(x-7)\ge0$.
\zadStop
\rozwStart{Patryk Wirkus}{}
Miejsca zerowe naszego wielomianu to: $3, 4, 7$.\\
Wielomian jest stopnia nieparzystego, ponadto znak współczynnika przy\linebreak najwyższej potędze x jest dodatni.\\ W związku z tym wykres wielomianu zaczyna się od lewej strony poniżej osi OX. A więc $$x \in [3,4] \cup [7,\infty).$$
\rozwStop
\odpStart
$x \in [3,4] \cup [7,\infty)$
\odpStop
\testStart
A.$x \in [3,4] \cup [7,\infty)$\\
B.$x \in (3,4) \cup [7,\infty)$\\
C.$x \in (3,4] \cup [7,\infty)$\\
D.$x \in [3,4) \cup [7,\infty)$\\
E.$x \in [3,4] \cup (7,\infty)$\\
F.$x \in (3,4) \cup (7,\infty)$\\
G.$x \in [3,4) \cup (7,\infty)$\\
H.$x \in (3,4] \cup (7,\infty)$
\testStop
\kluczStart
A
\kluczStop



\zadStart{Zadanie z Wikieł Z 1.62 a) moja wersja nr 328}

Rozwiązać nierówności $(x-3)(x-4)(x-8)\ge0$.
\zadStop
\rozwStart{Patryk Wirkus}{}
Miejsca zerowe naszego wielomianu to: $3, 4, 8$.\\
Wielomian jest stopnia nieparzystego, ponadto znak współczynnika przy\linebreak najwyższej potędze x jest dodatni.\\ W związku z tym wykres wielomianu zaczyna się od lewej strony poniżej osi OX. A więc $$x \in [3,4] \cup [8,\infty).$$
\rozwStop
\odpStart
$x \in [3,4] \cup [8,\infty)$
\odpStop
\testStart
A.$x \in [3,4] \cup [8,\infty)$\\
B.$x \in (3,4) \cup [8,\infty)$\\
C.$x \in (3,4] \cup [8,\infty)$\\
D.$x \in [3,4) \cup [8,\infty)$\\
E.$x \in [3,4] \cup (8,\infty)$\\
F.$x \in (3,4) \cup (8,\infty)$\\
G.$x \in [3,4) \cup (8,\infty)$\\
H.$x \in (3,4] \cup (8,\infty)$
\testStop
\kluczStart
A
\kluczStop



\zadStart{Zadanie z Wikieł Z 1.62 a) moja wersja nr 329}

Rozwiązać nierówności $(x-3)(x-4)(x-9)\ge0$.
\zadStop
\rozwStart{Patryk Wirkus}{}
Miejsca zerowe naszego wielomianu to: $3, 4, 9$.\\
Wielomian jest stopnia nieparzystego, ponadto znak współczynnika przy\linebreak najwyższej potędze x jest dodatni.\\ W związku z tym wykres wielomianu zaczyna się od lewej strony poniżej osi OX. A więc $$x \in [3,4] \cup [9,\infty).$$
\rozwStop
\odpStart
$x \in [3,4] \cup [9,\infty)$
\odpStop
\testStart
A.$x \in [3,4] \cup [9,\infty)$\\
B.$x \in (3,4) \cup [9,\infty)$\\
C.$x \in (3,4] \cup [9,\infty)$\\
D.$x \in [3,4) \cup [9,\infty)$\\
E.$x \in [3,4] \cup (9,\infty)$\\
F.$x \in (3,4) \cup (9,\infty)$\\
G.$x \in [3,4) \cup (9,\infty)$\\
H.$x \in (3,4] \cup (9,\infty)$
\testStop
\kluczStart
A
\kluczStop



\zadStart{Zadanie z Wikieł Z 1.62 a) moja wersja nr 330}

Rozwiązać nierówności $(x-3)(x-4)(x-10)\ge0$.
\zadStop
\rozwStart{Patryk Wirkus}{}
Miejsca zerowe naszego wielomianu to: $3, 4, 10$.\\
Wielomian jest stopnia nieparzystego, ponadto znak współczynnika przy\linebreak najwyższej potędze x jest dodatni.\\ W związku z tym wykres wielomianu zaczyna się od lewej strony poniżej osi OX. A więc $$x \in [3,4] \cup [10,\infty).$$
\rozwStop
\odpStart
$x \in [3,4] \cup [10,\infty)$
\odpStop
\testStart
A.$x \in [3,4] \cup [10,\infty)$\\
B.$x \in (3,4) \cup [10,\infty)$\\
C.$x \in (3,4] \cup [10,\infty)$\\
D.$x \in [3,4) \cup [10,\infty)$\\
E.$x \in [3,4] \cup (10,\infty)$\\
F.$x \in (3,4) \cup (10,\infty)$\\
G.$x \in [3,4) \cup (10,\infty)$\\
H.$x \in (3,4] \cup (10,\infty)$
\testStop
\kluczStart
A
\kluczStop



\zadStart{Zadanie z Wikieł Z 1.62 a) moja wersja nr 331}

Rozwiązać nierówności $(x-3)(x-4)(x-11)\ge0$.
\zadStop
\rozwStart{Patryk Wirkus}{}
Miejsca zerowe naszego wielomianu to: $3, 4, 11$.\\
Wielomian jest stopnia nieparzystego, ponadto znak współczynnika przy\linebreak najwyższej potędze x jest dodatni.\\ W związku z tym wykres wielomianu zaczyna się od lewej strony poniżej osi OX. A więc $$x \in [3,4] \cup [11,\infty).$$
\rozwStop
\odpStart
$x \in [3,4] \cup [11,\infty)$
\odpStop
\testStart
A.$x \in [3,4] \cup [11,\infty)$\\
B.$x \in (3,4) \cup [11,\infty)$\\
C.$x \in (3,4] \cup [11,\infty)$\\
D.$x \in [3,4) \cup [11,\infty)$\\
E.$x \in [3,4] \cup (11,\infty)$\\
F.$x \in (3,4) \cup (11,\infty)$\\
G.$x \in [3,4) \cup (11,\infty)$\\
H.$x \in (3,4] \cup (11,\infty)$
\testStop
\kluczStart
A
\kluczStop



\zadStart{Zadanie z Wikieł Z 1.62 a) moja wersja nr 332}

Rozwiązać nierówności $(x-3)(x-4)(x-12)\ge0$.
\zadStop
\rozwStart{Patryk Wirkus}{}
Miejsca zerowe naszego wielomianu to: $3, 4, 12$.\\
Wielomian jest stopnia nieparzystego, ponadto znak współczynnika przy\linebreak najwyższej potędze x jest dodatni.\\ W związku z tym wykres wielomianu zaczyna się od lewej strony poniżej osi OX. A więc $$x \in [3,4] \cup [12,\infty).$$
\rozwStop
\odpStart
$x \in [3,4] \cup [12,\infty)$
\odpStop
\testStart
A.$x \in [3,4] \cup [12,\infty)$\\
B.$x \in (3,4) \cup [12,\infty)$\\
C.$x \in (3,4] \cup [12,\infty)$\\
D.$x \in [3,4) \cup [12,\infty)$\\
E.$x \in [3,4] \cup (12,\infty)$\\
F.$x \in (3,4) \cup (12,\infty)$\\
G.$x \in [3,4) \cup (12,\infty)$\\
H.$x \in (3,4] \cup (12,\infty)$
\testStop
\kluczStart
A
\kluczStop



\zadStart{Zadanie z Wikieł Z 1.62 a) moja wersja nr 333}

Rozwiązać nierówności $(x-3)(x-4)(x-13)\ge0$.
\zadStop
\rozwStart{Patryk Wirkus}{}
Miejsca zerowe naszego wielomianu to: $3, 4, 13$.\\
Wielomian jest stopnia nieparzystego, ponadto znak współczynnika przy\linebreak najwyższej potędze x jest dodatni.\\ W związku z tym wykres wielomianu zaczyna się od lewej strony poniżej osi OX. A więc $$x \in [3,4] \cup [13,\infty).$$
\rozwStop
\odpStart
$x \in [3,4] \cup [13,\infty)$
\odpStop
\testStart
A.$x \in [3,4] \cup [13,\infty)$\\
B.$x \in (3,4) \cup [13,\infty)$\\
C.$x \in (3,4] \cup [13,\infty)$\\
D.$x \in [3,4) \cup [13,\infty)$\\
E.$x \in [3,4] \cup (13,\infty)$\\
F.$x \in (3,4) \cup (13,\infty)$\\
G.$x \in [3,4) \cup (13,\infty)$\\
H.$x \in (3,4] \cup (13,\infty)$
\testStop
\kluczStart
A
\kluczStop



\zadStart{Zadanie z Wikieł Z 1.62 a) moja wersja nr 334}

Rozwiązać nierówności $(x-3)(x-4)(x-14)\ge0$.
\zadStop
\rozwStart{Patryk Wirkus}{}
Miejsca zerowe naszego wielomianu to: $3, 4, 14$.\\
Wielomian jest stopnia nieparzystego, ponadto znak współczynnika przy\linebreak najwyższej potędze x jest dodatni.\\ W związku z tym wykres wielomianu zaczyna się od lewej strony poniżej osi OX. A więc $$x \in [3,4] \cup [14,\infty).$$
\rozwStop
\odpStart
$x \in [3,4] \cup [14,\infty)$
\odpStop
\testStart
A.$x \in [3,4] \cup [14,\infty)$\\
B.$x \in (3,4) \cup [14,\infty)$\\
C.$x \in (3,4] \cup [14,\infty)$\\
D.$x \in [3,4) \cup [14,\infty)$\\
E.$x \in [3,4] \cup (14,\infty)$\\
F.$x \in (3,4) \cup (14,\infty)$\\
G.$x \in [3,4) \cup (14,\infty)$\\
H.$x \in (3,4] \cup (14,\infty)$
\testStop
\kluczStart
A
\kluczStop



\zadStart{Zadanie z Wikieł Z 1.62 a) moja wersja nr 335}

Rozwiązać nierówności $(x-3)(x-4)(x-15)\ge0$.
\zadStop
\rozwStart{Patryk Wirkus}{}
Miejsca zerowe naszego wielomianu to: $3, 4, 15$.\\
Wielomian jest stopnia nieparzystego, ponadto znak współczynnika przy\linebreak najwyższej potędze x jest dodatni.\\ W związku z tym wykres wielomianu zaczyna się od lewej strony poniżej osi OX. A więc $$x \in [3,4] \cup [15,\infty).$$
\rozwStop
\odpStart
$x \in [3,4] \cup [15,\infty)$
\odpStop
\testStart
A.$x \in [3,4] \cup [15,\infty)$\\
B.$x \in (3,4) \cup [15,\infty)$\\
C.$x \in (3,4] \cup [15,\infty)$\\
D.$x \in [3,4) \cup [15,\infty)$\\
E.$x \in [3,4] \cup (15,\infty)$\\
F.$x \in (3,4) \cup (15,\infty)$\\
G.$x \in [3,4) \cup (15,\infty)$\\
H.$x \in (3,4] \cup (15,\infty)$
\testStop
\kluczStart
A
\kluczStop



\zadStart{Zadanie z Wikieł Z 1.62 a) moja wersja nr 336}

Rozwiązać nierówności $(x-3)(x-4)(x-16)\ge0$.
\zadStop
\rozwStart{Patryk Wirkus}{}
Miejsca zerowe naszego wielomianu to: $3, 4, 16$.\\
Wielomian jest stopnia nieparzystego, ponadto znak współczynnika przy\linebreak najwyższej potędze x jest dodatni.\\ W związku z tym wykres wielomianu zaczyna się od lewej strony poniżej osi OX. A więc $$x \in [3,4] \cup [16,\infty).$$
\rozwStop
\odpStart
$x \in [3,4] \cup [16,\infty)$
\odpStop
\testStart
A.$x \in [3,4] \cup [16,\infty)$\\
B.$x \in (3,4) \cup [16,\infty)$\\
C.$x \in (3,4] \cup [16,\infty)$\\
D.$x \in [3,4) \cup [16,\infty)$\\
E.$x \in [3,4] \cup (16,\infty)$\\
F.$x \in (3,4) \cup (16,\infty)$\\
G.$x \in [3,4) \cup (16,\infty)$\\
H.$x \in (3,4] \cup (16,\infty)$
\testStop
\kluczStart
A
\kluczStop



\zadStart{Zadanie z Wikieł Z 1.62 a) moja wersja nr 337}

Rozwiązać nierówności $(x-3)(x-4)(x-17)\ge0$.
\zadStop
\rozwStart{Patryk Wirkus}{}
Miejsca zerowe naszego wielomianu to: $3, 4, 17$.\\
Wielomian jest stopnia nieparzystego, ponadto znak współczynnika przy\linebreak najwyższej potędze x jest dodatni.\\ W związku z tym wykres wielomianu zaczyna się od lewej strony poniżej osi OX. A więc $$x \in [3,4] \cup [17,\infty).$$
\rozwStop
\odpStart
$x \in [3,4] \cup [17,\infty)$
\odpStop
\testStart
A.$x \in [3,4] \cup [17,\infty)$\\
B.$x \in (3,4) \cup [17,\infty)$\\
C.$x \in (3,4] \cup [17,\infty)$\\
D.$x \in [3,4) \cup [17,\infty)$\\
E.$x \in [3,4] \cup (17,\infty)$\\
F.$x \in (3,4) \cup (17,\infty)$\\
G.$x \in [3,4) \cup (17,\infty)$\\
H.$x \in (3,4] \cup (17,\infty)$
\testStop
\kluczStart
A
\kluczStop



\zadStart{Zadanie z Wikieł Z 1.62 a) moja wersja nr 338}

Rozwiązać nierówności $(x-3)(x-4)(x-18)\ge0$.
\zadStop
\rozwStart{Patryk Wirkus}{}
Miejsca zerowe naszego wielomianu to: $3, 4, 18$.\\
Wielomian jest stopnia nieparzystego, ponadto znak współczynnika przy\linebreak najwyższej potędze x jest dodatni.\\ W związku z tym wykres wielomianu zaczyna się od lewej strony poniżej osi OX. A więc $$x \in [3,4] \cup [18,\infty).$$
\rozwStop
\odpStart
$x \in [3,4] \cup [18,\infty)$
\odpStop
\testStart
A.$x \in [3,4] \cup [18,\infty)$\\
B.$x \in (3,4) \cup [18,\infty)$\\
C.$x \in (3,4] \cup [18,\infty)$\\
D.$x \in [3,4) \cup [18,\infty)$\\
E.$x \in [3,4] \cup (18,\infty)$\\
F.$x \in (3,4) \cup (18,\infty)$\\
G.$x \in [3,4) \cup (18,\infty)$\\
H.$x \in (3,4] \cup (18,\infty)$
\testStop
\kluczStart
A
\kluczStop



\zadStart{Zadanie z Wikieł Z 1.62 a) moja wersja nr 339}

Rozwiązać nierówności $(x-3)(x-4)(x-19)\ge0$.
\zadStop
\rozwStart{Patryk Wirkus}{}
Miejsca zerowe naszego wielomianu to: $3, 4, 19$.\\
Wielomian jest stopnia nieparzystego, ponadto znak współczynnika przy\linebreak najwyższej potędze x jest dodatni.\\ W związku z tym wykres wielomianu zaczyna się od lewej strony poniżej osi OX. A więc $$x \in [3,4] \cup [19,\infty).$$
\rozwStop
\odpStart
$x \in [3,4] \cup [19,\infty)$
\odpStop
\testStart
A.$x \in [3,4] \cup [19,\infty)$\\
B.$x \in (3,4) \cup [19,\infty)$\\
C.$x \in (3,4] \cup [19,\infty)$\\
D.$x \in [3,4) \cup [19,\infty)$\\
E.$x \in [3,4] \cup (19,\infty)$\\
F.$x \in (3,4) \cup (19,\infty)$\\
G.$x \in [3,4) \cup (19,\infty)$\\
H.$x \in (3,4] \cup (19,\infty)$
\testStop
\kluczStart
A
\kluczStop



\zadStart{Zadanie z Wikieł Z 1.62 a) moja wersja nr 340}

Rozwiązać nierówności $(x-3)(x-4)(x-20)\ge0$.
\zadStop
\rozwStart{Patryk Wirkus}{}
Miejsca zerowe naszego wielomianu to: $3, 4, 20$.\\
Wielomian jest stopnia nieparzystego, ponadto znak współczynnika przy\linebreak najwyższej potędze x jest dodatni.\\ W związku z tym wykres wielomianu zaczyna się od lewej strony poniżej osi OX. A więc $$x \in [3,4] \cup [20,\infty).$$
\rozwStop
\odpStart
$x \in [3,4] \cup [20,\infty)$
\odpStop
\testStart
A.$x \in [3,4] \cup [20,\infty)$\\
B.$x \in (3,4) \cup [20,\infty)$\\
C.$x \in (3,4] \cup [20,\infty)$\\
D.$x \in [3,4) \cup [20,\infty)$\\
E.$x \in [3,4] \cup (20,\infty)$\\
F.$x \in (3,4) \cup (20,\infty)$\\
G.$x \in [3,4) \cup (20,\infty)$\\
H.$x \in (3,4] \cup (20,\infty)$
\testStop
\kluczStart
A
\kluczStop



\zadStart{Zadanie z Wikieł Z 1.62 a) moja wersja nr 341}

Rozwiązać nierówności $(x-3)(x-5)(x-6)\ge0$.
\zadStop
\rozwStart{Patryk Wirkus}{}
Miejsca zerowe naszego wielomianu to: $3, 5, 6$.\\
Wielomian jest stopnia nieparzystego, ponadto znak współczynnika przy\linebreak najwyższej potędze x jest dodatni.\\ W związku z tym wykres wielomianu zaczyna się od lewej strony poniżej osi OX. A więc $$x \in [3,5] \cup [6,\infty).$$
\rozwStop
\odpStart
$x \in [3,5] \cup [6,\infty)$
\odpStop
\testStart
A.$x \in [3,5] \cup [6,\infty)$\\
B.$x \in (3,5) \cup [6,\infty)$\\
C.$x \in (3,5] \cup [6,\infty)$\\
D.$x \in [3,5) \cup [6,\infty)$\\
E.$x \in [3,5] \cup (6,\infty)$\\
F.$x \in (3,5) \cup (6,\infty)$\\
G.$x \in [3,5) \cup (6,\infty)$\\
H.$x \in (3,5] \cup (6,\infty)$
\testStop
\kluczStart
A
\kluczStop



\zadStart{Zadanie z Wikieł Z 1.62 a) moja wersja nr 342}

Rozwiązać nierówności $(x-3)(x-5)(x-7)\ge0$.
\zadStop
\rozwStart{Patryk Wirkus}{}
Miejsca zerowe naszego wielomianu to: $3, 5, 7$.\\
Wielomian jest stopnia nieparzystego, ponadto znak współczynnika przy\linebreak najwyższej potędze x jest dodatni.\\ W związku z tym wykres wielomianu zaczyna się od lewej strony poniżej osi OX. A więc $$x \in [3,5] \cup [7,\infty).$$
\rozwStop
\odpStart
$x \in [3,5] \cup [7,\infty)$
\odpStop
\testStart
A.$x \in [3,5] \cup [7,\infty)$\\
B.$x \in (3,5) \cup [7,\infty)$\\
C.$x \in (3,5] \cup [7,\infty)$\\
D.$x \in [3,5) \cup [7,\infty)$\\
E.$x \in [3,5] \cup (7,\infty)$\\
F.$x \in (3,5) \cup (7,\infty)$\\
G.$x \in [3,5) \cup (7,\infty)$\\
H.$x \in (3,5] \cup (7,\infty)$
\testStop
\kluczStart
A
\kluczStop



\zadStart{Zadanie z Wikieł Z 1.62 a) moja wersja nr 343}

Rozwiązać nierówności $(x-3)(x-5)(x-8)\ge0$.
\zadStop
\rozwStart{Patryk Wirkus}{}
Miejsca zerowe naszego wielomianu to: $3, 5, 8$.\\
Wielomian jest stopnia nieparzystego, ponadto znak współczynnika przy\linebreak najwyższej potędze x jest dodatni.\\ W związku z tym wykres wielomianu zaczyna się od lewej strony poniżej osi OX. A więc $$x \in [3,5] \cup [8,\infty).$$
\rozwStop
\odpStart
$x \in [3,5] \cup [8,\infty)$
\odpStop
\testStart
A.$x \in [3,5] \cup [8,\infty)$\\
B.$x \in (3,5) \cup [8,\infty)$\\
C.$x \in (3,5] \cup [8,\infty)$\\
D.$x \in [3,5) \cup [8,\infty)$\\
E.$x \in [3,5] \cup (8,\infty)$\\
F.$x \in (3,5) \cup (8,\infty)$\\
G.$x \in [3,5) \cup (8,\infty)$\\
H.$x \in (3,5] \cup (8,\infty)$
\testStop
\kluczStart
A
\kluczStop



\zadStart{Zadanie z Wikieł Z 1.62 a) moja wersja nr 344}

Rozwiązać nierówności $(x-3)(x-5)(x-9)\ge0$.
\zadStop
\rozwStart{Patryk Wirkus}{}
Miejsca zerowe naszego wielomianu to: $3, 5, 9$.\\
Wielomian jest stopnia nieparzystego, ponadto znak współczynnika przy\linebreak najwyższej potędze x jest dodatni.\\ W związku z tym wykres wielomianu zaczyna się od lewej strony poniżej osi OX. A więc $$x \in [3,5] \cup [9,\infty).$$
\rozwStop
\odpStart
$x \in [3,5] \cup [9,\infty)$
\odpStop
\testStart
A.$x \in [3,5] \cup [9,\infty)$\\
B.$x \in (3,5) \cup [9,\infty)$\\
C.$x \in (3,5] \cup [9,\infty)$\\
D.$x \in [3,5) \cup [9,\infty)$\\
E.$x \in [3,5] \cup (9,\infty)$\\
F.$x \in (3,5) \cup (9,\infty)$\\
G.$x \in [3,5) \cup (9,\infty)$\\
H.$x \in (3,5] \cup (9,\infty)$
\testStop
\kluczStart
A
\kluczStop



\zadStart{Zadanie z Wikieł Z 1.62 a) moja wersja nr 345}

Rozwiązać nierówności $(x-3)(x-5)(x-10)\ge0$.
\zadStop
\rozwStart{Patryk Wirkus}{}
Miejsca zerowe naszego wielomianu to: $3, 5, 10$.\\
Wielomian jest stopnia nieparzystego, ponadto znak współczynnika przy\linebreak najwyższej potędze x jest dodatni.\\ W związku z tym wykres wielomianu zaczyna się od lewej strony poniżej osi OX. A więc $$x \in [3,5] \cup [10,\infty).$$
\rozwStop
\odpStart
$x \in [3,5] \cup [10,\infty)$
\odpStop
\testStart
A.$x \in [3,5] \cup [10,\infty)$\\
B.$x \in (3,5) \cup [10,\infty)$\\
C.$x \in (3,5] \cup [10,\infty)$\\
D.$x \in [3,5) \cup [10,\infty)$\\
E.$x \in [3,5] \cup (10,\infty)$\\
F.$x \in (3,5) \cup (10,\infty)$\\
G.$x \in [3,5) \cup (10,\infty)$\\
H.$x \in (3,5] \cup (10,\infty)$
\testStop
\kluczStart
A
\kluczStop



\zadStart{Zadanie z Wikieł Z 1.62 a) moja wersja nr 346}

Rozwiązać nierówności $(x-3)(x-5)(x-11)\ge0$.
\zadStop
\rozwStart{Patryk Wirkus}{}
Miejsca zerowe naszego wielomianu to: $3, 5, 11$.\\
Wielomian jest stopnia nieparzystego, ponadto znak współczynnika przy\linebreak najwyższej potędze x jest dodatni.\\ W związku z tym wykres wielomianu zaczyna się od lewej strony poniżej osi OX. A więc $$x \in [3,5] \cup [11,\infty).$$
\rozwStop
\odpStart
$x \in [3,5] \cup [11,\infty)$
\odpStop
\testStart
A.$x \in [3,5] \cup [11,\infty)$\\
B.$x \in (3,5) \cup [11,\infty)$\\
C.$x \in (3,5] \cup [11,\infty)$\\
D.$x \in [3,5) \cup [11,\infty)$\\
E.$x \in [3,5] \cup (11,\infty)$\\
F.$x \in (3,5) \cup (11,\infty)$\\
G.$x \in [3,5) \cup (11,\infty)$\\
H.$x \in (3,5] \cup (11,\infty)$
\testStop
\kluczStart
A
\kluczStop



\zadStart{Zadanie z Wikieł Z 1.62 a) moja wersja nr 347}

Rozwiązać nierówności $(x-3)(x-5)(x-12)\ge0$.
\zadStop
\rozwStart{Patryk Wirkus}{}
Miejsca zerowe naszego wielomianu to: $3, 5, 12$.\\
Wielomian jest stopnia nieparzystego, ponadto znak współczynnika przy\linebreak najwyższej potędze x jest dodatni.\\ W związku z tym wykres wielomianu zaczyna się od lewej strony poniżej osi OX. A więc $$x \in [3,5] \cup [12,\infty).$$
\rozwStop
\odpStart
$x \in [3,5] \cup [12,\infty)$
\odpStop
\testStart
A.$x \in [3,5] \cup [12,\infty)$\\
B.$x \in (3,5) \cup [12,\infty)$\\
C.$x \in (3,5] \cup [12,\infty)$\\
D.$x \in [3,5) \cup [12,\infty)$\\
E.$x \in [3,5] \cup (12,\infty)$\\
F.$x \in (3,5) \cup (12,\infty)$\\
G.$x \in [3,5) \cup (12,\infty)$\\
H.$x \in (3,5] \cup (12,\infty)$
\testStop
\kluczStart
A
\kluczStop



\zadStart{Zadanie z Wikieł Z 1.62 a) moja wersja nr 348}

Rozwiązać nierówności $(x-3)(x-5)(x-13)\ge0$.
\zadStop
\rozwStart{Patryk Wirkus}{}
Miejsca zerowe naszego wielomianu to: $3, 5, 13$.\\
Wielomian jest stopnia nieparzystego, ponadto znak współczynnika przy\linebreak najwyższej potędze x jest dodatni.\\ W związku z tym wykres wielomianu zaczyna się od lewej strony poniżej osi OX. A więc $$x \in [3,5] \cup [13,\infty).$$
\rozwStop
\odpStart
$x \in [3,5] \cup [13,\infty)$
\odpStop
\testStart
A.$x \in [3,5] \cup [13,\infty)$\\
B.$x \in (3,5) \cup [13,\infty)$\\
C.$x \in (3,5] \cup [13,\infty)$\\
D.$x \in [3,5) \cup [13,\infty)$\\
E.$x \in [3,5] \cup (13,\infty)$\\
F.$x \in (3,5) \cup (13,\infty)$\\
G.$x \in [3,5) \cup (13,\infty)$\\
H.$x \in (3,5] \cup (13,\infty)$
\testStop
\kluczStart
A
\kluczStop



\zadStart{Zadanie z Wikieł Z 1.62 a) moja wersja nr 349}

Rozwiązać nierówności $(x-3)(x-5)(x-14)\ge0$.
\zadStop
\rozwStart{Patryk Wirkus}{}
Miejsca zerowe naszego wielomianu to: $3, 5, 14$.\\
Wielomian jest stopnia nieparzystego, ponadto znak współczynnika przy\linebreak najwyższej potędze x jest dodatni.\\ W związku z tym wykres wielomianu zaczyna się od lewej strony poniżej osi OX. A więc $$x \in [3,5] \cup [14,\infty).$$
\rozwStop
\odpStart
$x \in [3,5] \cup [14,\infty)$
\odpStop
\testStart
A.$x \in [3,5] \cup [14,\infty)$\\
B.$x \in (3,5) \cup [14,\infty)$\\
C.$x \in (3,5] \cup [14,\infty)$\\
D.$x \in [3,5) \cup [14,\infty)$\\
E.$x \in [3,5] \cup (14,\infty)$\\
F.$x \in (3,5) \cup (14,\infty)$\\
G.$x \in [3,5) \cup (14,\infty)$\\
H.$x \in (3,5] \cup (14,\infty)$
\testStop
\kluczStart
A
\kluczStop



\zadStart{Zadanie z Wikieł Z 1.62 a) moja wersja nr 350}

Rozwiązać nierówności $(x-3)(x-5)(x-15)\ge0$.
\zadStop
\rozwStart{Patryk Wirkus}{}
Miejsca zerowe naszego wielomianu to: $3, 5, 15$.\\
Wielomian jest stopnia nieparzystego, ponadto znak współczynnika przy\linebreak najwyższej potędze x jest dodatni.\\ W związku z tym wykres wielomianu zaczyna się od lewej strony poniżej osi OX. A więc $$x \in [3,5] \cup [15,\infty).$$
\rozwStop
\odpStart
$x \in [3,5] \cup [15,\infty)$
\odpStop
\testStart
A.$x \in [3,5] \cup [15,\infty)$\\
B.$x \in (3,5) \cup [15,\infty)$\\
C.$x \in (3,5] \cup [15,\infty)$\\
D.$x \in [3,5) \cup [15,\infty)$\\
E.$x \in [3,5] \cup (15,\infty)$\\
F.$x \in (3,5) \cup (15,\infty)$\\
G.$x \in [3,5) \cup (15,\infty)$\\
H.$x \in (3,5] \cup (15,\infty)$
\testStop
\kluczStart
A
\kluczStop



\zadStart{Zadanie z Wikieł Z 1.62 a) moja wersja nr 351}

Rozwiązać nierówności $(x-3)(x-5)(x-16)\ge0$.
\zadStop
\rozwStart{Patryk Wirkus}{}
Miejsca zerowe naszego wielomianu to: $3, 5, 16$.\\
Wielomian jest stopnia nieparzystego, ponadto znak współczynnika przy\linebreak najwyższej potędze x jest dodatni.\\ W związku z tym wykres wielomianu zaczyna się od lewej strony poniżej osi OX. A więc $$x \in [3,5] \cup [16,\infty).$$
\rozwStop
\odpStart
$x \in [3,5] \cup [16,\infty)$
\odpStop
\testStart
A.$x \in [3,5] \cup [16,\infty)$\\
B.$x \in (3,5) \cup [16,\infty)$\\
C.$x \in (3,5] \cup [16,\infty)$\\
D.$x \in [3,5) \cup [16,\infty)$\\
E.$x \in [3,5] \cup (16,\infty)$\\
F.$x \in (3,5) \cup (16,\infty)$\\
G.$x \in [3,5) \cup (16,\infty)$\\
H.$x \in (3,5] \cup (16,\infty)$
\testStop
\kluczStart
A
\kluczStop



\zadStart{Zadanie z Wikieł Z 1.62 a) moja wersja nr 352}

Rozwiązać nierówności $(x-3)(x-5)(x-17)\ge0$.
\zadStop
\rozwStart{Patryk Wirkus}{}
Miejsca zerowe naszego wielomianu to: $3, 5, 17$.\\
Wielomian jest stopnia nieparzystego, ponadto znak współczynnika przy\linebreak najwyższej potędze x jest dodatni.\\ W związku z tym wykres wielomianu zaczyna się od lewej strony poniżej osi OX. A więc $$x \in [3,5] \cup [17,\infty).$$
\rozwStop
\odpStart
$x \in [3,5] \cup [17,\infty)$
\odpStop
\testStart
A.$x \in [3,5] \cup [17,\infty)$\\
B.$x \in (3,5) \cup [17,\infty)$\\
C.$x \in (3,5] \cup [17,\infty)$\\
D.$x \in [3,5) \cup [17,\infty)$\\
E.$x \in [3,5] \cup (17,\infty)$\\
F.$x \in (3,5) \cup (17,\infty)$\\
G.$x \in [3,5) \cup (17,\infty)$\\
H.$x \in (3,5] \cup (17,\infty)$
\testStop
\kluczStart
A
\kluczStop



\zadStart{Zadanie z Wikieł Z 1.62 a) moja wersja nr 353}

Rozwiązać nierówności $(x-3)(x-5)(x-18)\ge0$.
\zadStop
\rozwStart{Patryk Wirkus}{}
Miejsca zerowe naszego wielomianu to: $3, 5, 18$.\\
Wielomian jest stopnia nieparzystego, ponadto znak współczynnika przy\linebreak najwyższej potędze x jest dodatni.\\ W związku z tym wykres wielomianu zaczyna się od lewej strony poniżej osi OX. A więc $$x \in [3,5] \cup [18,\infty).$$
\rozwStop
\odpStart
$x \in [3,5] \cup [18,\infty)$
\odpStop
\testStart
A.$x \in [3,5] \cup [18,\infty)$\\
B.$x \in (3,5) \cup [18,\infty)$\\
C.$x \in (3,5] \cup [18,\infty)$\\
D.$x \in [3,5) \cup [18,\infty)$\\
E.$x \in [3,5] \cup (18,\infty)$\\
F.$x \in (3,5) \cup (18,\infty)$\\
G.$x \in [3,5) \cup (18,\infty)$\\
H.$x \in (3,5] \cup (18,\infty)$
\testStop
\kluczStart
A
\kluczStop



\zadStart{Zadanie z Wikieł Z 1.62 a) moja wersja nr 354}

Rozwiązać nierówności $(x-3)(x-5)(x-19)\ge0$.
\zadStop
\rozwStart{Patryk Wirkus}{}
Miejsca zerowe naszego wielomianu to: $3, 5, 19$.\\
Wielomian jest stopnia nieparzystego, ponadto znak współczynnika przy\linebreak najwyższej potędze x jest dodatni.\\ W związku z tym wykres wielomianu zaczyna się od lewej strony poniżej osi OX. A więc $$x \in [3,5] \cup [19,\infty).$$
\rozwStop
\odpStart
$x \in [3,5] \cup [19,\infty)$
\odpStop
\testStart
A.$x \in [3,5] \cup [19,\infty)$\\
B.$x \in (3,5) \cup [19,\infty)$\\
C.$x \in (3,5] \cup [19,\infty)$\\
D.$x \in [3,5) \cup [19,\infty)$\\
E.$x \in [3,5] \cup (19,\infty)$\\
F.$x \in (3,5) \cup (19,\infty)$\\
G.$x \in [3,5) \cup (19,\infty)$\\
H.$x \in (3,5] \cup (19,\infty)$
\testStop
\kluczStart
A
\kluczStop



\zadStart{Zadanie z Wikieł Z 1.62 a) moja wersja nr 355}

Rozwiązać nierówności $(x-3)(x-5)(x-20)\ge0$.
\zadStop
\rozwStart{Patryk Wirkus}{}
Miejsca zerowe naszego wielomianu to: $3, 5, 20$.\\
Wielomian jest stopnia nieparzystego, ponadto znak współczynnika przy\linebreak najwyższej potędze x jest dodatni.\\ W związku z tym wykres wielomianu zaczyna się od lewej strony poniżej osi OX. A więc $$x \in [3,5] \cup [20,\infty).$$
\rozwStop
\odpStart
$x \in [3,5] \cup [20,\infty)$
\odpStop
\testStart
A.$x \in [3,5] \cup [20,\infty)$\\
B.$x \in (3,5) \cup [20,\infty)$\\
C.$x \in (3,5] \cup [20,\infty)$\\
D.$x \in [3,5) \cup [20,\infty)$\\
E.$x \in [3,5] \cup (20,\infty)$\\
F.$x \in (3,5) \cup (20,\infty)$\\
G.$x \in [3,5) \cup (20,\infty)$\\
H.$x \in (3,5] \cup (20,\infty)$
\testStop
\kluczStart
A
\kluczStop



\zadStart{Zadanie z Wikieł Z 1.62 a) moja wersja nr 356}

Rozwiązać nierówności $(x-3)(x-6)(x-7)\ge0$.
\zadStop
\rozwStart{Patryk Wirkus}{}
Miejsca zerowe naszego wielomianu to: $3, 6, 7$.\\
Wielomian jest stopnia nieparzystego, ponadto znak współczynnika przy\linebreak najwyższej potędze x jest dodatni.\\ W związku z tym wykres wielomianu zaczyna się od lewej strony poniżej osi OX. A więc $$x \in [3,6] \cup [7,\infty).$$
\rozwStop
\odpStart
$x \in [3,6] \cup [7,\infty)$
\odpStop
\testStart
A.$x \in [3,6] \cup [7,\infty)$\\
B.$x \in (3,6) \cup [7,\infty)$\\
C.$x \in (3,6] \cup [7,\infty)$\\
D.$x \in [3,6) \cup [7,\infty)$\\
E.$x \in [3,6] \cup (7,\infty)$\\
F.$x \in (3,6) \cup (7,\infty)$\\
G.$x \in [3,6) \cup (7,\infty)$\\
H.$x \in (3,6] \cup (7,\infty)$
\testStop
\kluczStart
A
\kluczStop



\zadStart{Zadanie z Wikieł Z 1.62 a) moja wersja nr 357}

Rozwiązać nierówności $(x-3)(x-6)(x-8)\ge0$.
\zadStop
\rozwStart{Patryk Wirkus}{}
Miejsca zerowe naszego wielomianu to: $3, 6, 8$.\\
Wielomian jest stopnia nieparzystego, ponadto znak współczynnika przy\linebreak najwyższej potędze x jest dodatni.\\ W związku z tym wykres wielomianu zaczyna się od lewej strony poniżej osi OX. A więc $$x \in [3,6] \cup [8,\infty).$$
\rozwStop
\odpStart
$x \in [3,6] \cup [8,\infty)$
\odpStop
\testStart
A.$x \in [3,6] \cup [8,\infty)$\\
B.$x \in (3,6) \cup [8,\infty)$\\
C.$x \in (3,6] \cup [8,\infty)$\\
D.$x \in [3,6) \cup [8,\infty)$\\
E.$x \in [3,6] \cup (8,\infty)$\\
F.$x \in (3,6) \cup (8,\infty)$\\
G.$x \in [3,6) \cup (8,\infty)$\\
H.$x \in (3,6] \cup (8,\infty)$
\testStop
\kluczStart
A
\kluczStop



\zadStart{Zadanie z Wikieł Z 1.62 a) moja wersja nr 358}

Rozwiązać nierówności $(x-3)(x-6)(x-9)\ge0$.
\zadStop
\rozwStart{Patryk Wirkus}{}
Miejsca zerowe naszego wielomianu to: $3, 6, 9$.\\
Wielomian jest stopnia nieparzystego, ponadto znak współczynnika przy\linebreak najwyższej potędze x jest dodatni.\\ W związku z tym wykres wielomianu zaczyna się od lewej strony poniżej osi OX. A więc $$x \in [3,6] \cup [9,\infty).$$
\rozwStop
\odpStart
$x \in [3,6] \cup [9,\infty)$
\odpStop
\testStart
A.$x \in [3,6] \cup [9,\infty)$\\
B.$x \in (3,6) \cup [9,\infty)$\\
C.$x \in (3,6] \cup [9,\infty)$\\
D.$x \in [3,6) \cup [9,\infty)$\\
E.$x \in [3,6] \cup (9,\infty)$\\
F.$x \in (3,6) \cup (9,\infty)$\\
G.$x \in [3,6) \cup (9,\infty)$\\
H.$x \in (3,6] \cup (9,\infty)$
\testStop
\kluczStart
A
\kluczStop



\zadStart{Zadanie z Wikieł Z 1.62 a) moja wersja nr 359}

Rozwiązać nierówności $(x-3)(x-6)(x-10)\ge0$.
\zadStop
\rozwStart{Patryk Wirkus}{}
Miejsca zerowe naszego wielomianu to: $3, 6, 10$.\\
Wielomian jest stopnia nieparzystego, ponadto znak współczynnika przy\linebreak najwyższej potędze x jest dodatni.\\ W związku z tym wykres wielomianu zaczyna się od lewej strony poniżej osi OX. A więc $$x \in [3,6] \cup [10,\infty).$$
\rozwStop
\odpStart
$x \in [3,6] \cup [10,\infty)$
\odpStop
\testStart
A.$x \in [3,6] \cup [10,\infty)$\\
B.$x \in (3,6) \cup [10,\infty)$\\
C.$x \in (3,6] \cup [10,\infty)$\\
D.$x \in [3,6) \cup [10,\infty)$\\
E.$x \in [3,6] \cup (10,\infty)$\\
F.$x \in (3,6) \cup (10,\infty)$\\
G.$x \in [3,6) \cup (10,\infty)$\\
H.$x \in (3,6] \cup (10,\infty)$
\testStop
\kluczStart
A
\kluczStop



\zadStart{Zadanie z Wikieł Z 1.62 a) moja wersja nr 360}

Rozwiązać nierówności $(x-3)(x-6)(x-11)\ge0$.
\zadStop
\rozwStart{Patryk Wirkus}{}
Miejsca zerowe naszego wielomianu to: $3, 6, 11$.\\
Wielomian jest stopnia nieparzystego, ponadto znak współczynnika przy\linebreak najwyższej potędze x jest dodatni.\\ W związku z tym wykres wielomianu zaczyna się od lewej strony poniżej osi OX. A więc $$x \in [3,6] \cup [11,\infty).$$
\rozwStop
\odpStart
$x \in [3,6] \cup [11,\infty)$
\odpStop
\testStart
A.$x \in [3,6] \cup [11,\infty)$\\
B.$x \in (3,6) \cup [11,\infty)$\\
C.$x \in (3,6] \cup [11,\infty)$\\
D.$x \in [3,6) \cup [11,\infty)$\\
E.$x \in [3,6] \cup (11,\infty)$\\
F.$x \in (3,6) \cup (11,\infty)$\\
G.$x \in [3,6) \cup (11,\infty)$\\
H.$x \in (3,6] \cup (11,\infty)$
\testStop
\kluczStart
A
\kluczStop



\zadStart{Zadanie z Wikieł Z 1.62 a) moja wersja nr 361}

Rozwiązać nierówności $(x-3)(x-6)(x-12)\ge0$.
\zadStop
\rozwStart{Patryk Wirkus}{}
Miejsca zerowe naszego wielomianu to: $3, 6, 12$.\\
Wielomian jest stopnia nieparzystego, ponadto znak współczynnika przy\linebreak najwyższej potędze x jest dodatni.\\ W związku z tym wykres wielomianu zaczyna się od lewej strony poniżej osi OX. A więc $$x \in [3,6] \cup [12,\infty).$$
\rozwStop
\odpStart
$x \in [3,6] \cup [12,\infty)$
\odpStop
\testStart
A.$x \in [3,6] \cup [12,\infty)$\\
B.$x \in (3,6) \cup [12,\infty)$\\
C.$x \in (3,6] \cup [12,\infty)$\\
D.$x \in [3,6) \cup [12,\infty)$\\
E.$x \in [3,6] \cup (12,\infty)$\\
F.$x \in (3,6) \cup (12,\infty)$\\
G.$x \in [3,6) \cup (12,\infty)$\\
H.$x \in (3,6] \cup (12,\infty)$
\testStop
\kluczStart
A
\kluczStop



\zadStart{Zadanie z Wikieł Z 1.62 a) moja wersja nr 362}

Rozwiązać nierówności $(x-3)(x-6)(x-13)\ge0$.
\zadStop
\rozwStart{Patryk Wirkus}{}
Miejsca zerowe naszego wielomianu to: $3, 6, 13$.\\
Wielomian jest stopnia nieparzystego, ponadto znak współczynnika przy\linebreak najwyższej potędze x jest dodatni.\\ W związku z tym wykres wielomianu zaczyna się od lewej strony poniżej osi OX. A więc $$x \in [3,6] \cup [13,\infty).$$
\rozwStop
\odpStart
$x \in [3,6] \cup [13,\infty)$
\odpStop
\testStart
A.$x \in [3,6] \cup [13,\infty)$\\
B.$x \in (3,6) \cup [13,\infty)$\\
C.$x \in (3,6] \cup [13,\infty)$\\
D.$x \in [3,6) \cup [13,\infty)$\\
E.$x \in [3,6] \cup (13,\infty)$\\
F.$x \in (3,6) \cup (13,\infty)$\\
G.$x \in [3,6) \cup (13,\infty)$\\
H.$x \in (3,6] \cup (13,\infty)$
\testStop
\kluczStart
A
\kluczStop



\zadStart{Zadanie z Wikieł Z 1.62 a) moja wersja nr 363}

Rozwiązać nierówności $(x-3)(x-6)(x-14)\ge0$.
\zadStop
\rozwStart{Patryk Wirkus}{}
Miejsca zerowe naszego wielomianu to: $3, 6, 14$.\\
Wielomian jest stopnia nieparzystego, ponadto znak współczynnika przy\linebreak najwyższej potędze x jest dodatni.\\ W związku z tym wykres wielomianu zaczyna się od lewej strony poniżej osi OX. A więc $$x \in [3,6] \cup [14,\infty).$$
\rozwStop
\odpStart
$x \in [3,6] \cup [14,\infty)$
\odpStop
\testStart
A.$x \in [3,6] \cup [14,\infty)$\\
B.$x \in (3,6) \cup [14,\infty)$\\
C.$x \in (3,6] \cup [14,\infty)$\\
D.$x \in [3,6) \cup [14,\infty)$\\
E.$x \in [3,6] \cup (14,\infty)$\\
F.$x \in (3,6) \cup (14,\infty)$\\
G.$x \in [3,6) \cup (14,\infty)$\\
H.$x \in (3,6] \cup (14,\infty)$
\testStop
\kluczStart
A
\kluczStop



\zadStart{Zadanie z Wikieł Z 1.62 a) moja wersja nr 364}

Rozwiązać nierówności $(x-3)(x-6)(x-15)\ge0$.
\zadStop
\rozwStart{Patryk Wirkus}{}
Miejsca zerowe naszego wielomianu to: $3, 6, 15$.\\
Wielomian jest stopnia nieparzystego, ponadto znak współczynnika przy\linebreak najwyższej potędze x jest dodatni.\\ W związku z tym wykres wielomianu zaczyna się od lewej strony poniżej osi OX. A więc $$x \in [3,6] \cup [15,\infty).$$
\rozwStop
\odpStart
$x \in [3,6] \cup [15,\infty)$
\odpStop
\testStart
A.$x \in [3,6] \cup [15,\infty)$\\
B.$x \in (3,6) \cup [15,\infty)$\\
C.$x \in (3,6] \cup [15,\infty)$\\
D.$x \in [3,6) \cup [15,\infty)$\\
E.$x \in [3,6] \cup (15,\infty)$\\
F.$x \in (3,6) \cup (15,\infty)$\\
G.$x \in [3,6) \cup (15,\infty)$\\
H.$x \in (3,6] \cup (15,\infty)$
\testStop
\kluczStart
A
\kluczStop



\zadStart{Zadanie z Wikieł Z 1.62 a) moja wersja nr 365}

Rozwiązać nierówności $(x-3)(x-6)(x-16)\ge0$.
\zadStop
\rozwStart{Patryk Wirkus}{}
Miejsca zerowe naszego wielomianu to: $3, 6, 16$.\\
Wielomian jest stopnia nieparzystego, ponadto znak współczynnika przy\linebreak najwyższej potędze x jest dodatni.\\ W związku z tym wykres wielomianu zaczyna się od lewej strony poniżej osi OX. A więc $$x \in [3,6] \cup [16,\infty).$$
\rozwStop
\odpStart
$x \in [3,6] \cup [16,\infty)$
\odpStop
\testStart
A.$x \in [3,6] \cup [16,\infty)$\\
B.$x \in (3,6) \cup [16,\infty)$\\
C.$x \in (3,6] \cup [16,\infty)$\\
D.$x \in [3,6) \cup [16,\infty)$\\
E.$x \in [3,6] \cup (16,\infty)$\\
F.$x \in (3,6) \cup (16,\infty)$\\
G.$x \in [3,6) \cup (16,\infty)$\\
H.$x \in (3,6] \cup (16,\infty)$
\testStop
\kluczStart
A
\kluczStop



\zadStart{Zadanie z Wikieł Z 1.62 a) moja wersja nr 366}

Rozwiązać nierówności $(x-3)(x-6)(x-17)\ge0$.
\zadStop
\rozwStart{Patryk Wirkus}{}
Miejsca zerowe naszego wielomianu to: $3, 6, 17$.\\
Wielomian jest stopnia nieparzystego, ponadto znak współczynnika przy\linebreak najwyższej potędze x jest dodatni.\\ W związku z tym wykres wielomianu zaczyna się od lewej strony poniżej osi OX. A więc $$x \in [3,6] \cup [17,\infty).$$
\rozwStop
\odpStart
$x \in [3,6] \cup [17,\infty)$
\odpStop
\testStart
A.$x \in [3,6] \cup [17,\infty)$\\
B.$x \in (3,6) \cup [17,\infty)$\\
C.$x \in (3,6] \cup [17,\infty)$\\
D.$x \in [3,6) \cup [17,\infty)$\\
E.$x \in [3,6] \cup (17,\infty)$\\
F.$x \in (3,6) \cup (17,\infty)$\\
G.$x \in [3,6) \cup (17,\infty)$\\
H.$x \in (3,6] \cup (17,\infty)$
\testStop
\kluczStart
A
\kluczStop



\zadStart{Zadanie z Wikieł Z 1.62 a) moja wersja nr 367}

Rozwiązać nierówności $(x-3)(x-6)(x-18)\ge0$.
\zadStop
\rozwStart{Patryk Wirkus}{}
Miejsca zerowe naszego wielomianu to: $3, 6, 18$.\\
Wielomian jest stopnia nieparzystego, ponadto znak współczynnika przy\linebreak najwyższej potędze x jest dodatni.\\ W związku z tym wykres wielomianu zaczyna się od lewej strony poniżej osi OX. A więc $$x \in [3,6] \cup [18,\infty).$$
\rozwStop
\odpStart
$x \in [3,6] \cup [18,\infty)$
\odpStop
\testStart
A.$x \in [3,6] \cup [18,\infty)$\\
B.$x \in (3,6) \cup [18,\infty)$\\
C.$x \in (3,6] \cup [18,\infty)$\\
D.$x \in [3,6) \cup [18,\infty)$\\
E.$x \in [3,6] \cup (18,\infty)$\\
F.$x \in (3,6) \cup (18,\infty)$\\
G.$x \in [3,6) \cup (18,\infty)$\\
H.$x \in (3,6] \cup (18,\infty)$
\testStop
\kluczStart
A
\kluczStop



\zadStart{Zadanie z Wikieł Z 1.62 a) moja wersja nr 368}

Rozwiązać nierówności $(x-3)(x-6)(x-19)\ge0$.
\zadStop
\rozwStart{Patryk Wirkus}{}
Miejsca zerowe naszego wielomianu to: $3, 6, 19$.\\
Wielomian jest stopnia nieparzystego, ponadto znak współczynnika przy\linebreak najwyższej potędze x jest dodatni.\\ W związku z tym wykres wielomianu zaczyna się od lewej strony poniżej osi OX. A więc $$x \in [3,6] \cup [19,\infty).$$
\rozwStop
\odpStart
$x \in [3,6] \cup [19,\infty)$
\odpStop
\testStart
A.$x \in [3,6] \cup [19,\infty)$\\
B.$x \in (3,6) \cup [19,\infty)$\\
C.$x \in (3,6] \cup [19,\infty)$\\
D.$x \in [3,6) \cup [19,\infty)$\\
E.$x \in [3,6] \cup (19,\infty)$\\
F.$x \in (3,6) \cup (19,\infty)$\\
G.$x \in [3,6) \cup (19,\infty)$\\
H.$x \in (3,6] \cup (19,\infty)$
\testStop
\kluczStart
A
\kluczStop



\zadStart{Zadanie z Wikieł Z 1.62 a) moja wersja nr 369}

Rozwiązać nierówności $(x-3)(x-6)(x-20)\ge0$.
\zadStop
\rozwStart{Patryk Wirkus}{}
Miejsca zerowe naszego wielomianu to: $3, 6, 20$.\\
Wielomian jest stopnia nieparzystego, ponadto znak współczynnika przy\linebreak najwyższej potędze x jest dodatni.\\ W związku z tym wykres wielomianu zaczyna się od lewej strony poniżej osi OX. A więc $$x \in [3,6] \cup [20,\infty).$$
\rozwStop
\odpStart
$x \in [3,6] \cup [20,\infty)$
\odpStop
\testStart
A.$x \in [3,6] \cup [20,\infty)$\\
B.$x \in (3,6) \cup [20,\infty)$\\
C.$x \in (3,6] \cup [20,\infty)$\\
D.$x \in [3,6) \cup [20,\infty)$\\
E.$x \in [3,6] \cup (20,\infty)$\\
F.$x \in (3,6) \cup (20,\infty)$\\
G.$x \in [3,6) \cup (20,\infty)$\\
H.$x \in (3,6] \cup (20,\infty)$
\testStop
\kluczStart
A
\kluczStop



\zadStart{Zadanie z Wikieł Z 1.62 a) moja wersja nr 370}

Rozwiązać nierówności $(x-3)(x-7)(x-8)\ge0$.
\zadStop
\rozwStart{Patryk Wirkus}{}
Miejsca zerowe naszego wielomianu to: $3, 7, 8$.\\
Wielomian jest stopnia nieparzystego, ponadto znak współczynnika przy\linebreak najwyższej potędze x jest dodatni.\\ W związku z tym wykres wielomianu zaczyna się od lewej strony poniżej osi OX. A więc $$x \in [3,7] \cup [8,\infty).$$
\rozwStop
\odpStart
$x \in [3,7] \cup [8,\infty)$
\odpStop
\testStart
A.$x \in [3,7] \cup [8,\infty)$\\
B.$x \in (3,7) \cup [8,\infty)$\\
C.$x \in (3,7] \cup [8,\infty)$\\
D.$x \in [3,7) \cup [8,\infty)$\\
E.$x \in [3,7] \cup (8,\infty)$\\
F.$x \in (3,7) \cup (8,\infty)$\\
G.$x \in [3,7) \cup (8,\infty)$\\
H.$x \in (3,7] \cup (8,\infty)$
\testStop
\kluczStart
A
\kluczStop



\zadStart{Zadanie z Wikieł Z 1.62 a) moja wersja nr 371}

Rozwiązać nierówności $(x-3)(x-7)(x-9)\ge0$.
\zadStop
\rozwStart{Patryk Wirkus}{}
Miejsca zerowe naszego wielomianu to: $3, 7, 9$.\\
Wielomian jest stopnia nieparzystego, ponadto znak współczynnika przy\linebreak najwyższej potędze x jest dodatni.\\ W związku z tym wykres wielomianu zaczyna się od lewej strony poniżej osi OX. A więc $$x \in [3,7] \cup [9,\infty).$$
\rozwStop
\odpStart
$x \in [3,7] \cup [9,\infty)$
\odpStop
\testStart
A.$x \in [3,7] \cup [9,\infty)$\\
B.$x \in (3,7) \cup [9,\infty)$\\
C.$x \in (3,7] \cup [9,\infty)$\\
D.$x \in [3,7) \cup [9,\infty)$\\
E.$x \in [3,7] \cup (9,\infty)$\\
F.$x \in (3,7) \cup (9,\infty)$\\
G.$x \in [3,7) \cup (9,\infty)$\\
H.$x \in (3,7] \cup (9,\infty)$
\testStop
\kluczStart
A
\kluczStop



\zadStart{Zadanie z Wikieł Z 1.62 a) moja wersja nr 372}

Rozwiązać nierówności $(x-3)(x-7)(x-10)\ge0$.
\zadStop
\rozwStart{Patryk Wirkus}{}
Miejsca zerowe naszego wielomianu to: $3, 7, 10$.\\
Wielomian jest stopnia nieparzystego, ponadto znak współczynnika przy\linebreak najwyższej potędze x jest dodatni.\\ W związku z tym wykres wielomianu zaczyna się od lewej strony poniżej osi OX. A więc $$x \in [3,7] \cup [10,\infty).$$
\rozwStop
\odpStart
$x \in [3,7] \cup [10,\infty)$
\odpStop
\testStart
A.$x \in [3,7] \cup [10,\infty)$\\
B.$x \in (3,7) \cup [10,\infty)$\\
C.$x \in (3,7] \cup [10,\infty)$\\
D.$x \in [3,7) \cup [10,\infty)$\\
E.$x \in [3,7] \cup (10,\infty)$\\
F.$x \in (3,7) \cup (10,\infty)$\\
G.$x \in [3,7) \cup (10,\infty)$\\
H.$x \in (3,7] \cup (10,\infty)$
\testStop
\kluczStart
A
\kluczStop



\zadStart{Zadanie z Wikieł Z 1.62 a) moja wersja nr 373}

Rozwiązać nierówności $(x-3)(x-7)(x-11)\ge0$.
\zadStop
\rozwStart{Patryk Wirkus}{}
Miejsca zerowe naszego wielomianu to: $3, 7, 11$.\\
Wielomian jest stopnia nieparzystego, ponadto znak współczynnika przy\linebreak najwyższej potędze x jest dodatni.\\ W związku z tym wykres wielomianu zaczyna się od lewej strony poniżej osi OX. A więc $$x \in [3,7] \cup [11,\infty).$$
\rozwStop
\odpStart
$x \in [3,7] \cup [11,\infty)$
\odpStop
\testStart
A.$x \in [3,7] \cup [11,\infty)$\\
B.$x \in (3,7) \cup [11,\infty)$\\
C.$x \in (3,7] \cup [11,\infty)$\\
D.$x \in [3,7) \cup [11,\infty)$\\
E.$x \in [3,7] \cup (11,\infty)$\\
F.$x \in (3,7) \cup (11,\infty)$\\
G.$x \in [3,7) \cup (11,\infty)$\\
H.$x \in (3,7] \cup (11,\infty)$
\testStop
\kluczStart
A
\kluczStop



\zadStart{Zadanie z Wikieł Z 1.62 a) moja wersja nr 374}

Rozwiązać nierówności $(x-3)(x-7)(x-12)\ge0$.
\zadStop
\rozwStart{Patryk Wirkus}{}
Miejsca zerowe naszego wielomianu to: $3, 7, 12$.\\
Wielomian jest stopnia nieparzystego, ponadto znak współczynnika przy\linebreak najwyższej potędze x jest dodatni.\\ W związku z tym wykres wielomianu zaczyna się od lewej strony poniżej osi OX. A więc $$x \in [3,7] \cup [12,\infty).$$
\rozwStop
\odpStart
$x \in [3,7] \cup [12,\infty)$
\odpStop
\testStart
A.$x \in [3,7] \cup [12,\infty)$\\
B.$x \in (3,7) \cup [12,\infty)$\\
C.$x \in (3,7] \cup [12,\infty)$\\
D.$x \in [3,7) \cup [12,\infty)$\\
E.$x \in [3,7] \cup (12,\infty)$\\
F.$x \in (3,7) \cup (12,\infty)$\\
G.$x \in [3,7) \cup (12,\infty)$\\
H.$x \in (3,7] \cup (12,\infty)$
\testStop
\kluczStart
A
\kluczStop



\zadStart{Zadanie z Wikieł Z 1.62 a) moja wersja nr 375}

Rozwiązać nierówności $(x-3)(x-7)(x-13)\ge0$.
\zadStop
\rozwStart{Patryk Wirkus}{}
Miejsca zerowe naszego wielomianu to: $3, 7, 13$.\\
Wielomian jest stopnia nieparzystego, ponadto znak współczynnika przy\linebreak najwyższej potędze x jest dodatni.\\ W związku z tym wykres wielomianu zaczyna się od lewej strony poniżej osi OX. A więc $$x \in [3,7] \cup [13,\infty).$$
\rozwStop
\odpStart
$x \in [3,7] \cup [13,\infty)$
\odpStop
\testStart
A.$x \in [3,7] \cup [13,\infty)$\\
B.$x \in (3,7) \cup [13,\infty)$\\
C.$x \in (3,7] \cup [13,\infty)$\\
D.$x \in [3,7) \cup [13,\infty)$\\
E.$x \in [3,7] \cup (13,\infty)$\\
F.$x \in (3,7) \cup (13,\infty)$\\
G.$x \in [3,7) \cup (13,\infty)$\\
H.$x \in (3,7] \cup (13,\infty)$
\testStop
\kluczStart
A
\kluczStop



\zadStart{Zadanie z Wikieł Z 1.62 a) moja wersja nr 376}

Rozwiązać nierówności $(x-3)(x-7)(x-14)\ge0$.
\zadStop
\rozwStart{Patryk Wirkus}{}
Miejsca zerowe naszego wielomianu to: $3, 7, 14$.\\
Wielomian jest stopnia nieparzystego, ponadto znak współczynnika przy\linebreak najwyższej potędze x jest dodatni.\\ W związku z tym wykres wielomianu zaczyna się od lewej strony poniżej osi OX. A więc $$x \in [3,7] \cup [14,\infty).$$
\rozwStop
\odpStart
$x \in [3,7] \cup [14,\infty)$
\odpStop
\testStart
A.$x \in [3,7] \cup [14,\infty)$\\
B.$x \in (3,7) \cup [14,\infty)$\\
C.$x \in (3,7] \cup [14,\infty)$\\
D.$x \in [3,7) \cup [14,\infty)$\\
E.$x \in [3,7] \cup (14,\infty)$\\
F.$x \in (3,7) \cup (14,\infty)$\\
G.$x \in [3,7) \cup (14,\infty)$\\
H.$x \in (3,7] \cup (14,\infty)$
\testStop
\kluczStart
A
\kluczStop



\zadStart{Zadanie z Wikieł Z 1.62 a) moja wersja nr 377}

Rozwiązać nierówności $(x-3)(x-7)(x-15)\ge0$.
\zadStop
\rozwStart{Patryk Wirkus}{}
Miejsca zerowe naszego wielomianu to: $3, 7, 15$.\\
Wielomian jest stopnia nieparzystego, ponadto znak współczynnika przy\linebreak najwyższej potędze x jest dodatni.\\ W związku z tym wykres wielomianu zaczyna się od lewej strony poniżej osi OX. A więc $$x \in [3,7] \cup [15,\infty).$$
\rozwStop
\odpStart
$x \in [3,7] \cup [15,\infty)$
\odpStop
\testStart
A.$x \in [3,7] \cup [15,\infty)$\\
B.$x \in (3,7) \cup [15,\infty)$\\
C.$x \in (3,7] \cup [15,\infty)$\\
D.$x \in [3,7) \cup [15,\infty)$\\
E.$x \in [3,7] \cup (15,\infty)$\\
F.$x \in (3,7) \cup (15,\infty)$\\
G.$x \in [3,7) \cup (15,\infty)$\\
H.$x \in (3,7] \cup (15,\infty)$
\testStop
\kluczStart
A
\kluczStop



\zadStart{Zadanie z Wikieł Z 1.62 a) moja wersja nr 378}

Rozwiązać nierówności $(x-3)(x-7)(x-16)\ge0$.
\zadStop
\rozwStart{Patryk Wirkus}{}
Miejsca zerowe naszego wielomianu to: $3, 7, 16$.\\
Wielomian jest stopnia nieparzystego, ponadto znak współczynnika przy\linebreak najwyższej potędze x jest dodatni.\\ W związku z tym wykres wielomianu zaczyna się od lewej strony poniżej osi OX. A więc $$x \in [3,7] \cup [16,\infty).$$
\rozwStop
\odpStart
$x \in [3,7] \cup [16,\infty)$
\odpStop
\testStart
A.$x \in [3,7] \cup [16,\infty)$\\
B.$x \in (3,7) \cup [16,\infty)$\\
C.$x \in (3,7] \cup [16,\infty)$\\
D.$x \in [3,7) \cup [16,\infty)$\\
E.$x \in [3,7] \cup (16,\infty)$\\
F.$x \in (3,7) \cup (16,\infty)$\\
G.$x \in [3,7) \cup (16,\infty)$\\
H.$x \in (3,7] \cup (16,\infty)$
\testStop
\kluczStart
A
\kluczStop



\zadStart{Zadanie z Wikieł Z 1.62 a) moja wersja nr 379}

Rozwiązać nierówności $(x-3)(x-7)(x-17)\ge0$.
\zadStop
\rozwStart{Patryk Wirkus}{}
Miejsca zerowe naszego wielomianu to: $3, 7, 17$.\\
Wielomian jest stopnia nieparzystego, ponadto znak współczynnika przy\linebreak najwyższej potędze x jest dodatni.\\ W związku z tym wykres wielomianu zaczyna się od lewej strony poniżej osi OX. A więc $$x \in [3,7] \cup [17,\infty).$$
\rozwStop
\odpStart
$x \in [3,7] \cup [17,\infty)$
\odpStop
\testStart
A.$x \in [3,7] \cup [17,\infty)$\\
B.$x \in (3,7) \cup [17,\infty)$\\
C.$x \in (3,7] \cup [17,\infty)$\\
D.$x \in [3,7) \cup [17,\infty)$\\
E.$x \in [3,7] \cup (17,\infty)$\\
F.$x \in (3,7) \cup (17,\infty)$\\
G.$x \in [3,7) \cup (17,\infty)$\\
H.$x \in (3,7] \cup (17,\infty)$
\testStop
\kluczStart
A
\kluczStop



\zadStart{Zadanie z Wikieł Z 1.62 a) moja wersja nr 380}

Rozwiązać nierówności $(x-3)(x-7)(x-18)\ge0$.
\zadStop
\rozwStart{Patryk Wirkus}{}
Miejsca zerowe naszego wielomianu to: $3, 7, 18$.\\
Wielomian jest stopnia nieparzystego, ponadto znak współczynnika przy\linebreak najwyższej potędze x jest dodatni.\\ W związku z tym wykres wielomianu zaczyna się od lewej strony poniżej osi OX. A więc $$x \in [3,7] \cup [18,\infty).$$
\rozwStop
\odpStart
$x \in [3,7] \cup [18,\infty)$
\odpStop
\testStart
A.$x \in [3,7] \cup [18,\infty)$\\
B.$x \in (3,7) \cup [18,\infty)$\\
C.$x \in (3,7] \cup [18,\infty)$\\
D.$x \in [3,7) \cup [18,\infty)$\\
E.$x \in [3,7] \cup (18,\infty)$\\
F.$x \in (3,7) \cup (18,\infty)$\\
G.$x \in [3,7) \cup (18,\infty)$\\
H.$x \in (3,7] \cup (18,\infty)$
\testStop
\kluczStart
A
\kluczStop



\zadStart{Zadanie z Wikieł Z 1.62 a) moja wersja nr 381}

Rozwiązać nierówności $(x-3)(x-7)(x-19)\ge0$.
\zadStop
\rozwStart{Patryk Wirkus}{}
Miejsca zerowe naszego wielomianu to: $3, 7, 19$.\\
Wielomian jest stopnia nieparzystego, ponadto znak współczynnika przy\linebreak najwyższej potędze x jest dodatni.\\ W związku z tym wykres wielomianu zaczyna się od lewej strony poniżej osi OX. A więc $$x \in [3,7] \cup [19,\infty).$$
\rozwStop
\odpStart
$x \in [3,7] \cup [19,\infty)$
\odpStop
\testStart
A.$x \in [3,7] \cup [19,\infty)$\\
B.$x \in (3,7) \cup [19,\infty)$\\
C.$x \in (3,7] \cup [19,\infty)$\\
D.$x \in [3,7) \cup [19,\infty)$\\
E.$x \in [3,7] \cup (19,\infty)$\\
F.$x \in (3,7) \cup (19,\infty)$\\
G.$x \in [3,7) \cup (19,\infty)$\\
H.$x \in (3,7] \cup (19,\infty)$
\testStop
\kluczStart
A
\kluczStop



\zadStart{Zadanie z Wikieł Z 1.62 a) moja wersja nr 382}

Rozwiązać nierówności $(x-3)(x-7)(x-20)\ge0$.
\zadStop
\rozwStart{Patryk Wirkus}{}
Miejsca zerowe naszego wielomianu to: $3, 7, 20$.\\
Wielomian jest stopnia nieparzystego, ponadto znak współczynnika przy\linebreak najwyższej potędze x jest dodatni.\\ W związku z tym wykres wielomianu zaczyna się od lewej strony poniżej osi OX. A więc $$x \in [3,7] \cup [20,\infty).$$
\rozwStop
\odpStart
$x \in [3,7] \cup [20,\infty)$
\odpStop
\testStart
A.$x \in [3,7] \cup [20,\infty)$\\
B.$x \in (3,7) \cup [20,\infty)$\\
C.$x \in (3,7] \cup [20,\infty)$\\
D.$x \in [3,7) \cup [20,\infty)$\\
E.$x \in [3,7] \cup (20,\infty)$\\
F.$x \in (3,7) \cup (20,\infty)$\\
G.$x \in [3,7) \cup (20,\infty)$\\
H.$x \in (3,7] \cup (20,\infty)$
\testStop
\kluczStart
A
\kluczStop



\zadStart{Zadanie z Wikieł Z 1.62 a) moja wersja nr 383}

Rozwiązać nierówności $(x-3)(x-8)(x-9)\ge0$.
\zadStop
\rozwStart{Patryk Wirkus}{}
Miejsca zerowe naszego wielomianu to: $3, 8, 9$.\\
Wielomian jest stopnia nieparzystego, ponadto znak współczynnika przy\linebreak najwyższej potędze x jest dodatni.\\ W związku z tym wykres wielomianu zaczyna się od lewej strony poniżej osi OX. A więc $$x \in [3,8] \cup [9,\infty).$$
\rozwStop
\odpStart
$x \in [3,8] \cup [9,\infty)$
\odpStop
\testStart
A.$x \in [3,8] \cup [9,\infty)$\\
B.$x \in (3,8) \cup [9,\infty)$\\
C.$x \in (3,8] \cup [9,\infty)$\\
D.$x \in [3,8) \cup [9,\infty)$\\
E.$x \in [3,8] \cup (9,\infty)$\\
F.$x \in (3,8) \cup (9,\infty)$\\
G.$x \in [3,8) \cup (9,\infty)$\\
H.$x \in (3,8] \cup (9,\infty)$
\testStop
\kluczStart
A
\kluczStop



\zadStart{Zadanie z Wikieł Z 1.62 a) moja wersja nr 384}

Rozwiązać nierówności $(x-3)(x-8)(x-10)\ge0$.
\zadStop
\rozwStart{Patryk Wirkus}{}
Miejsca zerowe naszego wielomianu to: $3, 8, 10$.\\
Wielomian jest stopnia nieparzystego, ponadto znak współczynnika przy\linebreak najwyższej potędze x jest dodatni.\\ W związku z tym wykres wielomianu zaczyna się od lewej strony poniżej osi OX. A więc $$x \in [3,8] \cup [10,\infty).$$
\rozwStop
\odpStart
$x \in [3,8] \cup [10,\infty)$
\odpStop
\testStart
A.$x \in [3,8] \cup [10,\infty)$\\
B.$x \in (3,8) \cup [10,\infty)$\\
C.$x \in (3,8] \cup [10,\infty)$\\
D.$x \in [3,8) \cup [10,\infty)$\\
E.$x \in [3,8] \cup (10,\infty)$\\
F.$x \in (3,8) \cup (10,\infty)$\\
G.$x \in [3,8) \cup (10,\infty)$\\
H.$x \in (3,8] \cup (10,\infty)$
\testStop
\kluczStart
A
\kluczStop



\zadStart{Zadanie z Wikieł Z 1.62 a) moja wersja nr 385}

Rozwiązać nierówności $(x-3)(x-8)(x-11)\ge0$.
\zadStop
\rozwStart{Patryk Wirkus}{}
Miejsca zerowe naszego wielomianu to: $3, 8, 11$.\\
Wielomian jest stopnia nieparzystego, ponadto znak współczynnika przy\linebreak najwyższej potędze x jest dodatni.\\ W związku z tym wykres wielomianu zaczyna się od lewej strony poniżej osi OX. A więc $$x \in [3,8] \cup [11,\infty).$$
\rozwStop
\odpStart
$x \in [3,8] \cup [11,\infty)$
\odpStop
\testStart
A.$x \in [3,8] \cup [11,\infty)$\\
B.$x \in (3,8) \cup [11,\infty)$\\
C.$x \in (3,8] \cup [11,\infty)$\\
D.$x \in [3,8) \cup [11,\infty)$\\
E.$x \in [3,8] \cup (11,\infty)$\\
F.$x \in (3,8) \cup (11,\infty)$\\
G.$x \in [3,8) \cup (11,\infty)$\\
H.$x \in (3,8] \cup (11,\infty)$
\testStop
\kluczStart
A
\kluczStop



\zadStart{Zadanie z Wikieł Z 1.62 a) moja wersja nr 386}

Rozwiązać nierówności $(x-3)(x-8)(x-12)\ge0$.
\zadStop
\rozwStart{Patryk Wirkus}{}
Miejsca zerowe naszego wielomianu to: $3, 8, 12$.\\
Wielomian jest stopnia nieparzystego, ponadto znak współczynnika przy\linebreak najwyższej potędze x jest dodatni.\\ W związku z tym wykres wielomianu zaczyna się od lewej strony poniżej osi OX. A więc $$x \in [3,8] \cup [12,\infty).$$
\rozwStop
\odpStart
$x \in [3,8] \cup [12,\infty)$
\odpStop
\testStart
A.$x \in [3,8] \cup [12,\infty)$\\
B.$x \in (3,8) \cup [12,\infty)$\\
C.$x \in (3,8] \cup [12,\infty)$\\
D.$x \in [3,8) \cup [12,\infty)$\\
E.$x \in [3,8] \cup (12,\infty)$\\
F.$x \in (3,8) \cup (12,\infty)$\\
G.$x \in [3,8) \cup (12,\infty)$\\
H.$x \in (3,8] \cup (12,\infty)$
\testStop
\kluczStart
A
\kluczStop



\zadStart{Zadanie z Wikieł Z 1.62 a) moja wersja nr 387}

Rozwiązać nierówności $(x-3)(x-8)(x-13)\ge0$.
\zadStop
\rozwStart{Patryk Wirkus}{}
Miejsca zerowe naszego wielomianu to: $3, 8, 13$.\\
Wielomian jest stopnia nieparzystego, ponadto znak współczynnika przy\linebreak najwyższej potędze x jest dodatni.\\ W związku z tym wykres wielomianu zaczyna się od lewej strony poniżej osi OX. A więc $$x \in [3,8] \cup [13,\infty).$$
\rozwStop
\odpStart
$x \in [3,8] \cup [13,\infty)$
\odpStop
\testStart
A.$x \in [3,8] \cup [13,\infty)$\\
B.$x \in (3,8) \cup [13,\infty)$\\
C.$x \in (3,8] \cup [13,\infty)$\\
D.$x \in [3,8) \cup [13,\infty)$\\
E.$x \in [3,8] \cup (13,\infty)$\\
F.$x \in (3,8) \cup (13,\infty)$\\
G.$x \in [3,8) \cup (13,\infty)$\\
H.$x \in (3,8] \cup (13,\infty)$
\testStop
\kluczStart
A
\kluczStop



\zadStart{Zadanie z Wikieł Z 1.62 a) moja wersja nr 388}

Rozwiązać nierówności $(x-3)(x-8)(x-14)\ge0$.
\zadStop
\rozwStart{Patryk Wirkus}{}
Miejsca zerowe naszego wielomianu to: $3, 8, 14$.\\
Wielomian jest stopnia nieparzystego, ponadto znak współczynnika przy\linebreak najwyższej potędze x jest dodatni.\\ W związku z tym wykres wielomianu zaczyna się od lewej strony poniżej osi OX. A więc $$x \in [3,8] \cup [14,\infty).$$
\rozwStop
\odpStart
$x \in [3,8] \cup [14,\infty)$
\odpStop
\testStart
A.$x \in [3,8] \cup [14,\infty)$\\
B.$x \in (3,8) \cup [14,\infty)$\\
C.$x \in (3,8] \cup [14,\infty)$\\
D.$x \in [3,8) \cup [14,\infty)$\\
E.$x \in [3,8] \cup (14,\infty)$\\
F.$x \in (3,8) \cup (14,\infty)$\\
G.$x \in [3,8) \cup (14,\infty)$\\
H.$x \in (3,8] \cup (14,\infty)$
\testStop
\kluczStart
A
\kluczStop



\zadStart{Zadanie z Wikieł Z 1.62 a) moja wersja nr 389}

Rozwiązać nierówności $(x-3)(x-8)(x-15)\ge0$.
\zadStop
\rozwStart{Patryk Wirkus}{}
Miejsca zerowe naszego wielomianu to: $3, 8, 15$.\\
Wielomian jest stopnia nieparzystego, ponadto znak współczynnika przy\linebreak najwyższej potędze x jest dodatni.\\ W związku z tym wykres wielomianu zaczyna się od lewej strony poniżej osi OX. A więc $$x \in [3,8] \cup [15,\infty).$$
\rozwStop
\odpStart
$x \in [3,8] \cup [15,\infty)$
\odpStop
\testStart
A.$x \in [3,8] \cup [15,\infty)$\\
B.$x \in (3,8) \cup [15,\infty)$\\
C.$x \in (3,8] \cup [15,\infty)$\\
D.$x \in [3,8) \cup [15,\infty)$\\
E.$x \in [3,8] \cup (15,\infty)$\\
F.$x \in (3,8) \cup (15,\infty)$\\
G.$x \in [3,8) \cup (15,\infty)$\\
H.$x \in (3,8] \cup (15,\infty)$
\testStop
\kluczStart
A
\kluczStop



\zadStart{Zadanie z Wikieł Z 1.62 a) moja wersja nr 390}

Rozwiązać nierówności $(x-3)(x-8)(x-16)\ge0$.
\zadStop
\rozwStart{Patryk Wirkus}{}
Miejsca zerowe naszego wielomianu to: $3, 8, 16$.\\
Wielomian jest stopnia nieparzystego, ponadto znak współczynnika przy\linebreak najwyższej potędze x jest dodatni.\\ W związku z tym wykres wielomianu zaczyna się od lewej strony poniżej osi OX. A więc $$x \in [3,8] \cup [16,\infty).$$
\rozwStop
\odpStart
$x \in [3,8] \cup [16,\infty)$
\odpStop
\testStart
A.$x \in [3,8] \cup [16,\infty)$\\
B.$x \in (3,8) \cup [16,\infty)$\\
C.$x \in (3,8] \cup [16,\infty)$\\
D.$x \in [3,8) \cup [16,\infty)$\\
E.$x \in [3,8] \cup (16,\infty)$\\
F.$x \in (3,8) \cup (16,\infty)$\\
G.$x \in [3,8) \cup (16,\infty)$\\
H.$x \in (3,8] \cup (16,\infty)$
\testStop
\kluczStart
A
\kluczStop



\zadStart{Zadanie z Wikieł Z 1.62 a) moja wersja nr 391}

Rozwiązać nierówności $(x-3)(x-8)(x-17)\ge0$.
\zadStop
\rozwStart{Patryk Wirkus}{}
Miejsca zerowe naszego wielomianu to: $3, 8, 17$.\\
Wielomian jest stopnia nieparzystego, ponadto znak współczynnika przy\linebreak najwyższej potędze x jest dodatni.\\ W związku z tym wykres wielomianu zaczyna się od lewej strony poniżej osi OX. A więc $$x \in [3,8] \cup [17,\infty).$$
\rozwStop
\odpStart
$x \in [3,8] \cup [17,\infty)$
\odpStop
\testStart
A.$x \in [3,8] \cup [17,\infty)$\\
B.$x \in (3,8) \cup [17,\infty)$\\
C.$x \in (3,8] \cup [17,\infty)$\\
D.$x \in [3,8) \cup [17,\infty)$\\
E.$x \in [3,8] \cup (17,\infty)$\\
F.$x \in (3,8) \cup (17,\infty)$\\
G.$x \in [3,8) \cup (17,\infty)$\\
H.$x \in (3,8] \cup (17,\infty)$
\testStop
\kluczStart
A
\kluczStop



\zadStart{Zadanie z Wikieł Z 1.62 a) moja wersja nr 392}

Rozwiązać nierówności $(x-3)(x-8)(x-18)\ge0$.
\zadStop
\rozwStart{Patryk Wirkus}{}
Miejsca zerowe naszego wielomianu to: $3, 8, 18$.\\
Wielomian jest stopnia nieparzystego, ponadto znak współczynnika przy\linebreak najwyższej potędze x jest dodatni.\\ W związku z tym wykres wielomianu zaczyna się od lewej strony poniżej osi OX. A więc $$x \in [3,8] \cup [18,\infty).$$
\rozwStop
\odpStart
$x \in [3,8] \cup [18,\infty)$
\odpStop
\testStart
A.$x \in [3,8] \cup [18,\infty)$\\
B.$x \in (3,8) \cup [18,\infty)$\\
C.$x \in (3,8] \cup [18,\infty)$\\
D.$x \in [3,8) \cup [18,\infty)$\\
E.$x \in [3,8] \cup (18,\infty)$\\
F.$x \in (3,8) \cup (18,\infty)$\\
G.$x \in [3,8) \cup (18,\infty)$\\
H.$x \in (3,8] \cup (18,\infty)$
\testStop
\kluczStart
A
\kluczStop



\zadStart{Zadanie z Wikieł Z 1.62 a) moja wersja nr 393}

Rozwiązać nierówności $(x-3)(x-8)(x-19)\ge0$.
\zadStop
\rozwStart{Patryk Wirkus}{}
Miejsca zerowe naszego wielomianu to: $3, 8, 19$.\\
Wielomian jest stopnia nieparzystego, ponadto znak współczynnika przy\linebreak najwyższej potędze x jest dodatni.\\ W związku z tym wykres wielomianu zaczyna się od lewej strony poniżej osi OX. A więc $$x \in [3,8] \cup [19,\infty).$$
\rozwStop
\odpStart
$x \in [3,8] \cup [19,\infty)$
\odpStop
\testStart
A.$x \in [3,8] \cup [19,\infty)$\\
B.$x \in (3,8) \cup [19,\infty)$\\
C.$x \in (3,8] \cup [19,\infty)$\\
D.$x \in [3,8) \cup [19,\infty)$\\
E.$x \in [3,8] \cup (19,\infty)$\\
F.$x \in (3,8) \cup (19,\infty)$\\
G.$x \in [3,8) \cup (19,\infty)$\\
H.$x \in (3,8] \cup (19,\infty)$
\testStop
\kluczStart
A
\kluczStop



\zadStart{Zadanie z Wikieł Z 1.62 a) moja wersja nr 394}

Rozwiązać nierówności $(x-3)(x-8)(x-20)\ge0$.
\zadStop
\rozwStart{Patryk Wirkus}{}
Miejsca zerowe naszego wielomianu to: $3, 8, 20$.\\
Wielomian jest stopnia nieparzystego, ponadto znak współczynnika przy\linebreak najwyższej potędze x jest dodatni.\\ W związku z tym wykres wielomianu zaczyna się od lewej strony poniżej osi OX. A więc $$x \in [3,8] \cup [20,\infty).$$
\rozwStop
\odpStart
$x \in [3,8] \cup [20,\infty)$
\odpStop
\testStart
A.$x \in [3,8] \cup [20,\infty)$\\
B.$x \in (3,8) \cup [20,\infty)$\\
C.$x \in (3,8] \cup [20,\infty)$\\
D.$x \in [3,8) \cup [20,\infty)$\\
E.$x \in [3,8] \cup (20,\infty)$\\
F.$x \in (3,8) \cup (20,\infty)$\\
G.$x \in [3,8) \cup (20,\infty)$\\
H.$x \in (3,8] \cup (20,\infty)$
\testStop
\kluczStart
A
\kluczStop



\zadStart{Zadanie z Wikieł Z 1.62 a) moja wersja nr 395}

Rozwiązać nierówności $(x-3)(x-9)(x-10)\ge0$.
\zadStop
\rozwStart{Patryk Wirkus}{}
Miejsca zerowe naszego wielomianu to: $3, 9, 10$.\\
Wielomian jest stopnia nieparzystego, ponadto znak współczynnika przy\linebreak najwyższej potędze x jest dodatni.\\ W związku z tym wykres wielomianu zaczyna się od lewej strony poniżej osi OX. A więc $$x \in [3,9] \cup [10,\infty).$$
\rozwStop
\odpStart
$x \in [3,9] \cup [10,\infty)$
\odpStop
\testStart
A.$x \in [3,9] \cup [10,\infty)$\\
B.$x \in (3,9) \cup [10,\infty)$\\
C.$x \in (3,9] \cup [10,\infty)$\\
D.$x \in [3,9) \cup [10,\infty)$\\
E.$x \in [3,9] \cup (10,\infty)$\\
F.$x \in (3,9) \cup (10,\infty)$\\
G.$x \in [3,9) \cup (10,\infty)$\\
H.$x \in (3,9] \cup (10,\infty)$
\testStop
\kluczStart
A
\kluczStop



\zadStart{Zadanie z Wikieł Z 1.62 a) moja wersja nr 396}

Rozwiązać nierówności $(x-3)(x-9)(x-11)\ge0$.
\zadStop
\rozwStart{Patryk Wirkus}{}
Miejsca zerowe naszego wielomianu to: $3, 9, 11$.\\
Wielomian jest stopnia nieparzystego, ponadto znak współczynnika przy\linebreak najwyższej potędze x jest dodatni.\\ W związku z tym wykres wielomianu zaczyna się od lewej strony poniżej osi OX. A więc $$x \in [3,9] \cup [11,\infty).$$
\rozwStop
\odpStart
$x \in [3,9] \cup [11,\infty)$
\odpStop
\testStart
A.$x \in [3,9] \cup [11,\infty)$\\
B.$x \in (3,9) \cup [11,\infty)$\\
C.$x \in (3,9] \cup [11,\infty)$\\
D.$x \in [3,9) \cup [11,\infty)$\\
E.$x \in [3,9] \cup (11,\infty)$\\
F.$x \in (3,9) \cup (11,\infty)$\\
G.$x \in [3,9) \cup (11,\infty)$\\
H.$x \in (3,9] \cup (11,\infty)$
\testStop
\kluczStart
A
\kluczStop



\zadStart{Zadanie z Wikieł Z 1.62 a) moja wersja nr 397}

Rozwiązać nierówności $(x-3)(x-9)(x-12)\ge0$.
\zadStop
\rozwStart{Patryk Wirkus}{}
Miejsca zerowe naszego wielomianu to: $3, 9, 12$.\\
Wielomian jest stopnia nieparzystego, ponadto znak współczynnika przy\linebreak najwyższej potędze x jest dodatni.\\ W związku z tym wykres wielomianu zaczyna się od lewej strony poniżej osi OX. A więc $$x \in [3,9] \cup [12,\infty).$$
\rozwStop
\odpStart
$x \in [3,9] \cup [12,\infty)$
\odpStop
\testStart
A.$x \in [3,9] \cup [12,\infty)$\\
B.$x \in (3,9) \cup [12,\infty)$\\
C.$x \in (3,9] \cup [12,\infty)$\\
D.$x \in [3,9) \cup [12,\infty)$\\
E.$x \in [3,9] \cup (12,\infty)$\\
F.$x \in (3,9) \cup (12,\infty)$\\
G.$x \in [3,9) \cup (12,\infty)$\\
H.$x \in (3,9] \cup (12,\infty)$
\testStop
\kluczStart
A
\kluczStop



\zadStart{Zadanie z Wikieł Z 1.62 a) moja wersja nr 398}

Rozwiązać nierówności $(x-3)(x-9)(x-13)\ge0$.
\zadStop
\rozwStart{Patryk Wirkus}{}
Miejsca zerowe naszego wielomianu to: $3, 9, 13$.\\
Wielomian jest stopnia nieparzystego, ponadto znak współczynnika przy\linebreak najwyższej potędze x jest dodatni.\\ W związku z tym wykres wielomianu zaczyna się od lewej strony poniżej osi OX. A więc $$x \in [3,9] \cup [13,\infty).$$
\rozwStop
\odpStart
$x \in [3,9] \cup [13,\infty)$
\odpStop
\testStart
A.$x \in [3,9] \cup [13,\infty)$\\
B.$x \in (3,9) \cup [13,\infty)$\\
C.$x \in (3,9] \cup [13,\infty)$\\
D.$x \in [3,9) \cup [13,\infty)$\\
E.$x \in [3,9] \cup (13,\infty)$\\
F.$x \in (3,9) \cup (13,\infty)$\\
G.$x \in [3,9) \cup (13,\infty)$\\
H.$x \in (3,9] \cup (13,\infty)$
\testStop
\kluczStart
A
\kluczStop



\zadStart{Zadanie z Wikieł Z 1.62 a) moja wersja nr 399}

Rozwiązać nierówności $(x-3)(x-9)(x-14)\ge0$.
\zadStop
\rozwStart{Patryk Wirkus}{}
Miejsca zerowe naszego wielomianu to: $3, 9, 14$.\\
Wielomian jest stopnia nieparzystego, ponadto znak współczynnika przy\linebreak najwyższej potędze x jest dodatni.\\ W związku z tym wykres wielomianu zaczyna się od lewej strony poniżej osi OX. A więc $$x \in [3,9] \cup [14,\infty).$$
\rozwStop
\odpStart
$x \in [3,9] \cup [14,\infty)$
\odpStop
\testStart
A.$x \in [3,9] \cup [14,\infty)$\\
B.$x \in (3,9) \cup [14,\infty)$\\
C.$x \in (3,9] \cup [14,\infty)$\\
D.$x \in [3,9) \cup [14,\infty)$\\
E.$x \in [3,9] \cup (14,\infty)$\\
F.$x \in (3,9) \cup (14,\infty)$\\
G.$x \in [3,9) \cup (14,\infty)$\\
H.$x \in (3,9] \cup (14,\infty)$
\testStop
\kluczStart
A
\kluczStop



\zadStart{Zadanie z Wikieł Z 1.62 a) moja wersja nr 400}

Rozwiązać nierówności $(x-3)(x-9)(x-15)\ge0$.
\zadStop
\rozwStart{Patryk Wirkus}{}
Miejsca zerowe naszego wielomianu to: $3, 9, 15$.\\
Wielomian jest stopnia nieparzystego, ponadto znak współczynnika przy\linebreak najwyższej potędze x jest dodatni.\\ W związku z tym wykres wielomianu zaczyna się od lewej strony poniżej osi OX. A więc $$x \in [3,9] \cup [15,\infty).$$
\rozwStop
\odpStart
$x \in [3,9] \cup [15,\infty)$
\odpStop
\testStart
A.$x \in [3,9] \cup [15,\infty)$\\
B.$x \in (3,9) \cup [15,\infty)$\\
C.$x \in (3,9] \cup [15,\infty)$\\
D.$x \in [3,9) \cup [15,\infty)$\\
E.$x \in [3,9] \cup (15,\infty)$\\
F.$x \in (3,9) \cup (15,\infty)$\\
G.$x \in [3,9) \cup (15,\infty)$\\
H.$x \in (3,9] \cup (15,\infty)$
\testStop
\kluczStart
A
\kluczStop



\zadStart{Zadanie z Wikieł Z 1.62 a) moja wersja nr 401}

Rozwiązać nierówności $(x-3)(x-9)(x-16)\ge0$.
\zadStop
\rozwStart{Patryk Wirkus}{}
Miejsca zerowe naszego wielomianu to: $3, 9, 16$.\\
Wielomian jest stopnia nieparzystego, ponadto znak współczynnika przy\linebreak najwyższej potędze x jest dodatni.\\ W związku z tym wykres wielomianu zaczyna się od lewej strony poniżej osi OX. A więc $$x \in [3,9] \cup [16,\infty).$$
\rozwStop
\odpStart
$x \in [3,9] \cup [16,\infty)$
\odpStop
\testStart
A.$x \in [3,9] \cup [16,\infty)$\\
B.$x \in (3,9) \cup [16,\infty)$\\
C.$x \in (3,9] \cup [16,\infty)$\\
D.$x \in [3,9) \cup [16,\infty)$\\
E.$x \in [3,9] \cup (16,\infty)$\\
F.$x \in (3,9) \cup (16,\infty)$\\
G.$x \in [3,9) \cup (16,\infty)$\\
H.$x \in (3,9] \cup (16,\infty)$
\testStop
\kluczStart
A
\kluczStop



\zadStart{Zadanie z Wikieł Z 1.62 a) moja wersja nr 402}

Rozwiązać nierówności $(x-3)(x-9)(x-17)\ge0$.
\zadStop
\rozwStart{Patryk Wirkus}{}
Miejsca zerowe naszego wielomianu to: $3, 9, 17$.\\
Wielomian jest stopnia nieparzystego, ponadto znak współczynnika przy\linebreak najwyższej potędze x jest dodatni.\\ W związku z tym wykres wielomianu zaczyna się od lewej strony poniżej osi OX. A więc $$x \in [3,9] \cup [17,\infty).$$
\rozwStop
\odpStart
$x \in [3,9] \cup [17,\infty)$
\odpStop
\testStart
A.$x \in [3,9] \cup [17,\infty)$\\
B.$x \in (3,9) \cup [17,\infty)$\\
C.$x \in (3,9] \cup [17,\infty)$\\
D.$x \in [3,9) \cup [17,\infty)$\\
E.$x \in [3,9] \cup (17,\infty)$\\
F.$x \in (3,9) \cup (17,\infty)$\\
G.$x \in [3,9) \cup (17,\infty)$\\
H.$x \in (3,9] \cup (17,\infty)$
\testStop
\kluczStart
A
\kluczStop



\zadStart{Zadanie z Wikieł Z 1.62 a) moja wersja nr 403}

Rozwiązać nierówności $(x-3)(x-9)(x-18)\ge0$.
\zadStop
\rozwStart{Patryk Wirkus}{}
Miejsca zerowe naszego wielomianu to: $3, 9, 18$.\\
Wielomian jest stopnia nieparzystego, ponadto znak współczynnika przy\linebreak najwyższej potędze x jest dodatni.\\ W związku z tym wykres wielomianu zaczyna się od lewej strony poniżej osi OX. A więc $$x \in [3,9] \cup [18,\infty).$$
\rozwStop
\odpStart
$x \in [3,9] \cup [18,\infty)$
\odpStop
\testStart
A.$x \in [3,9] \cup [18,\infty)$\\
B.$x \in (3,9) \cup [18,\infty)$\\
C.$x \in (3,9] \cup [18,\infty)$\\
D.$x \in [3,9) \cup [18,\infty)$\\
E.$x \in [3,9] \cup (18,\infty)$\\
F.$x \in (3,9) \cup (18,\infty)$\\
G.$x \in [3,9) \cup (18,\infty)$\\
H.$x \in (3,9] \cup (18,\infty)$
\testStop
\kluczStart
A
\kluczStop



\zadStart{Zadanie z Wikieł Z 1.62 a) moja wersja nr 404}

Rozwiązać nierówności $(x-3)(x-9)(x-19)\ge0$.
\zadStop
\rozwStart{Patryk Wirkus}{}
Miejsca zerowe naszego wielomianu to: $3, 9, 19$.\\
Wielomian jest stopnia nieparzystego, ponadto znak współczynnika przy\linebreak najwyższej potędze x jest dodatni.\\ W związku z tym wykres wielomianu zaczyna się od lewej strony poniżej osi OX. A więc $$x \in [3,9] \cup [19,\infty).$$
\rozwStop
\odpStart
$x \in [3,9] \cup [19,\infty)$
\odpStop
\testStart
A.$x \in [3,9] \cup [19,\infty)$\\
B.$x \in (3,9) \cup [19,\infty)$\\
C.$x \in (3,9] \cup [19,\infty)$\\
D.$x \in [3,9) \cup [19,\infty)$\\
E.$x \in [3,9] \cup (19,\infty)$\\
F.$x \in (3,9) \cup (19,\infty)$\\
G.$x \in [3,9) \cup (19,\infty)$\\
H.$x \in (3,9] \cup (19,\infty)$
\testStop
\kluczStart
A
\kluczStop



\zadStart{Zadanie z Wikieł Z 1.62 a) moja wersja nr 405}

Rozwiązać nierówności $(x-3)(x-9)(x-20)\ge0$.
\zadStop
\rozwStart{Patryk Wirkus}{}
Miejsca zerowe naszego wielomianu to: $3, 9, 20$.\\
Wielomian jest stopnia nieparzystego, ponadto znak współczynnika przy\linebreak najwyższej potędze x jest dodatni.\\ W związku z tym wykres wielomianu zaczyna się od lewej strony poniżej osi OX. A więc $$x \in [3,9] \cup [20,\infty).$$
\rozwStop
\odpStart
$x \in [3,9] \cup [20,\infty)$
\odpStop
\testStart
A.$x \in [3,9] \cup [20,\infty)$\\
B.$x \in (3,9) \cup [20,\infty)$\\
C.$x \in (3,9] \cup [20,\infty)$\\
D.$x \in [3,9) \cup [20,\infty)$\\
E.$x \in [3,9] \cup (20,\infty)$\\
F.$x \in (3,9) \cup (20,\infty)$\\
G.$x \in [3,9) \cup (20,\infty)$\\
H.$x \in (3,9] \cup (20,\infty)$
\testStop
\kluczStart
A
\kluczStop



\zadStart{Zadanie z Wikieł Z 1.62 a) moja wersja nr 406}

Rozwiązać nierówności $(x-3)(x-10)(x-11)\ge0$.
\zadStop
\rozwStart{Patryk Wirkus}{}
Miejsca zerowe naszego wielomianu to: $3, 10, 11$.\\
Wielomian jest stopnia nieparzystego, ponadto znak współczynnika przy\linebreak najwyższej potędze x jest dodatni.\\ W związku z tym wykres wielomianu zaczyna się od lewej strony poniżej osi OX. A więc $$x \in [3,10] \cup [11,\infty).$$
\rozwStop
\odpStart
$x \in [3,10] \cup [11,\infty)$
\odpStop
\testStart
A.$x \in [3,10] \cup [11,\infty)$\\
B.$x \in (3,10) \cup [11,\infty)$\\
C.$x \in (3,10] \cup [11,\infty)$\\
D.$x \in [3,10) \cup [11,\infty)$\\
E.$x \in [3,10] \cup (11,\infty)$\\
F.$x \in (3,10) \cup (11,\infty)$\\
G.$x \in [3,10) \cup (11,\infty)$\\
H.$x \in (3,10] \cup (11,\infty)$
\testStop
\kluczStart
A
\kluczStop



\zadStart{Zadanie z Wikieł Z 1.62 a) moja wersja nr 407}

Rozwiązać nierówności $(x-3)(x-10)(x-12)\ge0$.
\zadStop
\rozwStart{Patryk Wirkus}{}
Miejsca zerowe naszego wielomianu to: $3, 10, 12$.\\
Wielomian jest stopnia nieparzystego, ponadto znak współczynnika przy\linebreak najwyższej potędze x jest dodatni.\\ W związku z tym wykres wielomianu zaczyna się od lewej strony poniżej osi OX. A więc $$x \in [3,10] \cup [12,\infty).$$
\rozwStop
\odpStart
$x \in [3,10] \cup [12,\infty)$
\odpStop
\testStart
A.$x \in [3,10] \cup [12,\infty)$\\
B.$x \in (3,10) \cup [12,\infty)$\\
C.$x \in (3,10] \cup [12,\infty)$\\
D.$x \in [3,10) \cup [12,\infty)$\\
E.$x \in [3,10] \cup (12,\infty)$\\
F.$x \in (3,10) \cup (12,\infty)$\\
G.$x \in [3,10) \cup (12,\infty)$\\
H.$x \in (3,10] \cup (12,\infty)$
\testStop
\kluczStart
A
\kluczStop



\zadStart{Zadanie z Wikieł Z 1.62 a) moja wersja nr 408}

Rozwiązać nierówności $(x-3)(x-10)(x-13)\ge0$.
\zadStop
\rozwStart{Patryk Wirkus}{}
Miejsca zerowe naszego wielomianu to: $3, 10, 13$.\\
Wielomian jest stopnia nieparzystego, ponadto znak współczynnika przy\linebreak najwyższej potędze x jest dodatni.\\ W związku z tym wykres wielomianu zaczyna się od lewej strony poniżej osi OX. A więc $$x \in [3,10] \cup [13,\infty).$$
\rozwStop
\odpStart
$x \in [3,10] \cup [13,\infty)$
\odpStop
\testStart
A.$x \in [3,10] \cup [13,\infty)$\\
B.$x \in (3,10) \cup [13,\infty)$\\
C.$x \in (3,10] \cup [13,\infty)$\\
D.$x \in [3,10) \cup [13,\infty)$\\
E.$x \in [3,10] \cup (13,\infty)$\\
F.$x \in (3,10) \cup (13,\infty)$\\
G.$x \in [3,10) \cup (13,\infty)$\\
H.$x \in (3,10] \cup (13,\infty)$
\testStop
\kluczStart
A
\kluczStop



\zadStart{Zadanie z Wikieł Z 1.62 a) moja wersja nr 409}

Rozwiązać nierówności $(x-3)(x-10)(x-14)\ge0$.
\zadStop
\rozwStart{Patryk Wirkus}{}
Miejsca zerowe naszego wielomianu to: $3, 10, 14$.\\
Wielomian jest stopnia nieparzystego, ponadto znak współczynnika przy\linebreak najwyższej potędze x jest dodatni.\\ W związku z tym wykres wielomianu zaczyna się od lewej strony poniżej osi OX. A więc $$x \in [3,10] \cup [14,\infty).$$
\rozwStop
\odpStart
$x \in [3,10] \cup [14,\infty)$
\odpStop
\testStart
A.$x \in [3,10] \cup [14,\infty)$\\
B.$x \in (3,10) \cup [14,\infty)$\\
C.$x \in (3,10] \cup [14,\infty)$\\
D.$x \in [3,10) \cup [14,\infty)$\\
E.$x \in [3,10] \cup (14,\infty)$\\
F.$x \in (3,10) \cup (14,\infty)$\\
G.$x \in [3,10) \cup (14,\infty)$\\
H.$x \in (3,10] \cup (14,\infty)$
\testStop
\kluczStart
A
\kluczStop



\zadStart{Zadanie z Wikieł Z 1.62 a) moja wersja nr 410}

Rozwiązać nierówności $(x-3)(x-10)(x-15)\ge0$.
\zadStop
\rozwStart{Patryk Wirkus}{}
Miejsca zerowe naszego wielomianu to: $3, 10, 15$.\\
Wielomian jest stopnia nieparzystego, ponadto znak współczynnika przy\linebreak najwyższej potędze x jest dodatni.\\ W związku z tym wykres wielomianu zaczyna się od lewej strony poniżej osi OX. A więc $$x \in [3,10] \cup [15,\infty).$$
\rozwStop
\odpStart
$x \in [3,10] \cup [15,\infty)$
\odpStop
\testStart
A.$x \in [3,10] \cup [15,\infty)$\\
B.$x \in (3,10) \cup [15,\infty)$\\
C.$x \in (3,10] \cup [15,\infty)$\\
D.$x \in [3,10) \cup [15,\infty)$\\
E.$x \in [3,10] \cup (15,\infty)$\\
F.$x \in (3,10) \cup (15,\infty)$\\
G.$x \in [3,10) \cup (15,\infty)$\\
H.$x \in (3,10] \cup (15,\infty)$
\testStop
\kluczStart
A
\kluczStop



\zadStart{Zadanie z Wikieł Z 1.62 a) moja wersja nr 411}

Rozwiązać nierówności $(x-3)(x-10)(x-16)\ge0$.
\zadStop
\rozwStart{Patryk Wirkus}{}
Miejsca zerowe naszego wielomianu to: $3, 10, 16$.\\
Wielomian jest stopnia nieparzystego, ponadto znak współczynnika przy\linebreak najwyższej potędze x jest dodatni.\\ W związku z tym wykres wielomianu zaczyna się od lewej strony poniżej osi OX. A więc $$x \in [3,10] \cup [16,\infty).$$
\rozwStop
\odpStart
$x \in [3,10] \cup [16,\infty)$
\odpStop
\testStart
A.$x \in [3,10] \cup [16,\infty)$\\
B.$x \in (3,10) \cup [16,\infty)$\\
C.$x \in (3,10] \cup [16,\infty)$\\
D.$x \in [3,10) \cup [16,\infty)$\\
E.$x \in [3,10] \cup (16,\infty)$\\
F.$x \in (3,10) \cup (16,\infty)$\\
G.$x \in [3,10) \cup (16,\infty)$\\
H.$x \in (3,10] \cup (16,\infty)$
\testStop
\kluczStart
A
\kluczStop



\zadStart{Zadanie z Wikieł Z 1.62 a) moja wersja nr 412}

Rozwiązać nierówności $(x-3)(x-10)(x-17)\ge0$.
\zadStop
\rozwStart{Patryk Wirkus}{}
Miejsca zerowe naszego wielomianu to: $3, 10, 17$.\\
Wielomian jest stopnia nieparzystego, ponadto znak współczynnika przy\linebreak najwyższej potędze x jest dodatni.\\ W związku z tym wykres wielomianu zaczyna się od lewej strony poniżej osi OX. A więc $$x \in [3,10] \cup [17,\infty).$$
\rozwStop
\odpStart
$x \in [3,10] \cup [17,\infty)$
\odpStop
\testStart
A.$x \in [3,10] \cup [17,\infty)$\\
B.$x \in (3,10) \cup [17,\infty)$\\
C.$x \in (3,10] \cup [17,\infty)$\\
D.$x \in [3,10) \cup [17,\infty)$\\
E.$x \in [3,10] \cup (17,\infty)$\\
F.$x \in (3,10) \cup (17,\infty)$\\
G.$x \in [3,10) \cup (17,\infty)$\\
H.$x \in (3,10] \cup (17,\infty)$
\testStop
\kluczStart
A
\kluczStop



\zadStart{Zadanie z Wikieł Z 1.62 a) moja wersja nr 413}

Rozwiązać nierówności $(x-3)(x-10)(x-18)\ge0$.
\zadStop
\rozwStart{Patryk Wirkus}{}
Miejsca zerowe naszego wielomianu to: $3, 10, 18$.\\
Wielomian jest stopnia nieparzystego, ponadto znak współczynnika przy\linebreak najwyższej potędze x jest dodatni.\\ W związku z tym wykres wielomianu zaczyna się od lewej strony poniżej osi OX. A więc $$x \in [3,10] \cup [18,\infty).$$
\rozwStop
\odpStart
$x \in [3,10] \cup [18,\infty)$
\odpStop
\testStart
A.$x \in [3,10] \cup [18,\infty)$\\
B.$x \in (3,10) \cup [18,\infty)$\\
C.$x \in (3,10] \cup [18,\infty)$\\
D.$x \in [3,10) \cup [18,\infty)$\\
E.$x \in [3,10] \cup (18,\infty)$\\
F.$x \in (3,10) \cup (18,\infty)$\\
G.$x \in [3,10) \cup (18,\infty)$\\
H.$x \in (3,10] \cup (18,\infty)$
\testStop
\kluczStart
A
\kluczStop



\zadStart{Zadanie z Wikieł Z 1.62 a) moja wersja nr 414}

Rozwiązać nierówności $(x-3)(x-10)(x-19)\ge0$.
\zadStop
\rozwStart{Patryk Wirkus}{}
Miejsca zerowe naszego wielomianu to: $3, 10, 19$.\\
Wielomian jest stopnia nieparzystego, ponadto znak współczynnika przy\linebreak najwyższej potędze x jest dodatni.\\ W związku z tym wykres wielomianu zaczyna się od lewej strony poniżej osi OX. A więc $$x \in [3,10] \cup [19,\infty).$$
\rozwStop
\odpStart
$x \in [3,10] \cup [19,\infty)$
\odpStop
\testStart
A.$x \in [3,10] \cup [19,\infty)$\\
B.$x \in (3,10) \cup [19,\infty)$\\
C.$x \in (3,10] \cup [19,\infty)$\\
D.$x \in [3,10) \cup [19,\infty)$\\
E.$x \in [3,10] \cup (19,\infty)$\\
F.$x \in (3,10) \cup (19,\infty)$\\
G.$x \in [3,10) \cup (19,\infty)$\\
H.$x \in (3,10] \cup (19,\infty)$
\testStop
\kluczStart
A
\kluczStop



\zadStart{Zadanie z Wikieł Z 1.62 a) moja wersja nr 415}

Rozwiązać nierówności $(x-3)(x-10)(x-20)\ge0$.
\zadStop
\rozwStart{Patryk Wirkus}{}
Miejsca zerowe naszego wielomianu to: $3, 10, 20$.\\
Wielomian jest stopnia nieparzystego, ponadto znak współczynnika przy\linebreak najwyższej potędze x jest dodatni.\\ W związku z tym wykres wielomianu zaczyna się od lewej strony poniżej osi OX. A więc $$x \in [3,10] \cup [20,\infty).$$
\rozwStop
\odpStart
$x \in [3,10] \cup [20,\infty)$
\odpStop
\testStart
A.$x \in [3,10] \cup [20,\infty)$\\
B.$x \in (3,10) \cup [20,\infty)$\\
C.$x \in (3,10] \cup [20,\infty)$\\
D.$x \in [3,10) \cup [20,\infty)$\\
E.$x \in [3,10] \cup (20,\infty)$\\
F.$x \in (3,10) \cup (20,\infty)$\\
G.$x \in [3,10) \cup (20,\infty)$\\
H.$x \in (3,10] \cup (20,\infty)$
\testStop
\kluczStart
A
\kluczStop



\zadStart{Zadanie z Wikieł Z 1.62 a) moja wersja nr 416}

Rozwiązać nierówności $(x-3)(x-11)(x-12)\ge0$.
\zadStop
\rozwStart{Patryk Wirkus}{}
Miejsca zerowe naszego wielomianu to: $3, 11, 12$.\\
Wielomian jest stopnia nieparzystego, ponadto znak współczynnika przy\linebreak najwyższej potędze x jest dodatni.\\ W związku z tym wykres wielomianu zaczyna się od lewej strony poniżej osi OX. A więc $$x \in [3,11] \cup [12,\infty).$$
\rozwStop
\odpStart
$x \in [3,11] \cup [12,\infty)$
\odpStop
\testStart
A.$x \in [3,11] \cup [12,\infty)$\\
B.$x \in (3,11) \cup [12,\infty)$\\
C.$x \in (3,11] \cup [12,\infty)$\\
D.$x \in [3,11) \cup [12,\infty)$\\
E.$x \in [3,11] \cup (12,\infty)$\\
F.$x \in (3,11) \cup (12,\infty)$\\
G.$x \in [3,11) \cup (12,\infty)$\\
H.$x \in (3,11] \cup (12,\infty)$
\testStop
\kluczStart
A
\kluczStop



\zadStart{Zadanie z Wikieł Z 1.62 a) moja wersja nr 417}

Rozwiązać nierówności $(x-3)(x-11)(x-13)\ge0$.
\zadStop
\rozwStart{Patryk Wirkus}{}
Miejsca zerowe naszego wielomianu to: $3, 11, 13$.\\
Wielomian jest stopnia nieparzystego, ponadto znak współczynnika przy\linebreak najwyższej potędze x jest dodatni.\\ W związku z tym wykres wielomianu zaczyna się od lewej strony poniżej osi OX. A więc $$x \in [3,11] \cup [13,\infty).$$
\rozwStop
\odpStart
$x \in [3,11] \cup [13,\infty)$
\odpStop
\testStart
A.$x \in [3,11] \cup [13,\infty)$\\
B.$x \in (3,11) \cup [13,\infty)$\\
C.$x \in (3,11] \cup [13,\infty)$\\
D.$x \in [3,11) \cup [13,\infty)$\\
E.$x \in [3,11] \cup (13,\infty)$\\
F.$x \in (3,11) \cup (13,\infty)$\\
G.$x \in [3,11) \cup (13,\infty)$\\
H.$x \in (3,11] \cup (13,\infty)$
\testStop
\kluczStart
A
\kluczStop



\zadStart{Zadanie z Wikieł Z 1.62 a) moja wersja nr 418}

Rozwiązać nierówności $(x-3)(x-11)(x-14)\ge0$.
\zadStop
\rozwStart{Patryk Wirkus}{}
Miejsca zerowe naszego wielomianu to: $3, 11, 14$.\\
Wielomian jest stopnia nieparzystego, ponadto znak współczynnika przy\linebreak najwyższej potędze x jest dodatni.\\ W związku z tym wykres wielomianu zaczyna się od lewej strony poniżej osi OX. A więc $$x \in [3,11] \cup [14,\infty).$$
\rozwStop
\odpStart
$x \in [3,11] \cup [14,\infty)$
\odpStop
\testStart
A.$x \in [3,11] \cup [14,\infty)$\\
B.$x \in (3,11) \cup [14,\infty)$\\
C.$x \in (3,11] \cup [14,\infty)$\\
D.$x \in [3,11) \cup [14,\infty)$\\
E.$x \in [3,11] \cup (14,\infty)$\\
F.$x \in (3,11) \cup (14,\infty)$\\
G.$x \in [3,11) \cup (14,\infty)$\\
H.$x \in (3,11] \cup (14,\infty)$
\testStop
\kluczStart
A
\kluczStop



\zadStart{Zadanie z Wikieł Z 1.62 a) moja wersja nr 419}

Rozwiązać nierówności $(x-3)(x-11)(x-15)\ge0$.
\zadStop
\rozwStart{Patryk Wirkus}{}
Miejsca zerowe naszego wielomianu to: $3, 11, 15$.\\
Wielomian jest stopnia nieparzystego, ponadto znak współczynnika przy\linebreak najwyższej potędze x jest dodatni.\\ W związku z tym wykres wielomianu zaczyna się od lewej strony poniżej osi OX. A więc $$x \in [3,11] \cup [15,\infty).$$
\rozwStop
\odpStart
$x \in [3,11] \cup [15,\infty)$
\odpStop
\testStart
A.$x \in [3,11] \cup [15,\infty)$\\
B.$x \in (3,11) \cup [15,\infty)$\\
C.$x \in (3,11] \cup [15,\infty)$\\
D.$x \in [3,11) \cup [15,\infty)$\\
E.$x \in [3,11] \cup (15,\infty)$\\
F.$x \in (3,11) \cup (15,\infty)$\\
G.$x \in [3,11) \cup (15,\infty)$\\
H.$x \in (3,11] \cup (15,\infty)$
\testStop
\kluczStart
A
\kluczStop



\zadStart{Zadanie z Wikieł Z 1.62 a) moja wersja nr 420}

Rozwiązać nierówności $(x-3)(x-11)(x-16)\ge0$.
\zadStop
\rozwStart{Patryk Wirkus}{}
Miejsca zerowe naszego wielomianu to: $3, 11, 16$.\\
Wielomian jest stopnia nieparzystego, ponadto znak współczynnika przy\linebreak najwyższej potędze x jest dodatni.\\ W związku z tym wykres wielomianu zaczyna się od lewej strony poniżej osi OX. A więc $$x \in [3,11] \cup [16,\infty).$$
\rozwStop
\odpStart
$x \in [3,11] \cup [16,\infty)$
\odpStop
\testStart
A.$x \in [3,11] \cup [16,\infty)$\\
B.$x \in (3,11) \cup [16,\infty)$\\
C.$x \in (3,11] \cup [16,\infty)$\\
D.$x \in [3,11) \cup [16,\infty)$\\
E.$x \in [3,11] \cup (16,\infty)$\\
F.$x \in (3,11) \cup (16,\infty)$\\
G.$x \in [3,11) \cup (16,\infty)$\\
H.$x \in (3,11] \cup (16,\infty)$
\testStop
\kluczStart
A
\kluczStop



\zadStart{Zadanie z Wikieł Z 1.62 a) moja wersja nr 421}

Rozwiązać nierówności $(x-3)(x-11)(x-17)\ge0$.
\zadStop
\rozwStart{Patryk Wirkus}{}
Miejsca zerowe naszego wielomianu to: $3, 11, 17$.\\
Wielomian jest stopnia nieparzystego, ponadto znak współczynnika przy\linebreak najwyższej potędze x jest dodatni.\\ W związku z tym wykres wielomianu zaczyna się od lewej strony poniżej osi OX. A więc $$x \in [3,11] \cup [17,\infty).$$
\rozwStop
\odpStart
$x \in [3,11] \cup [17,\infty)$
\odpStop
\testStart
A.$x \in [3,11] \cup [17,\infty)$\\
B.$x \in (3,11) \cup [17,\infty)$\\
C.$x \in (3,11] \cup [17,\infty)$\\
D.$x \in [3,11) \cup [17,\infty)$\\
E.$x \in [3,11] \cup (17,\infty)$\\
F.$x \in (3,11) \cup (17,\infty)$\\
G.$x \in [3,11) \cup (17,\infty)$\\
H.$x \in (3,11] \cup (17,\infty)$
\testStop
\kluczStart
A
\kluczStop



\zadStart{Zadanie z Wikieł Z 1.62 a) moja wersja nr 422}

Rozwiązać nierówności $(x-3)(x-11)(x-18)\ge0$.
\zadStop
\rozwStart{Patryk Wirkus}{}
Miejsca zerowe naszego wielomianu to: $3, 11, 18$.\\
Wielomian jest stopnia nieparzystego, ponadto znak współczynnika przy\linebreak najwyższej potędze x jest dodatni.\\ W związku z tym wykres wielomianu zaczyna się od lewej strony poniżej osi OX. A więc $$x \in [3,11] \cup [18,\infty).$$
\rozwStop
\odpStart
$x \in [3,11] \cup [18,\infty)$
\odpStop
\testStart
A.$x \in [3,11] \cup [18,\infty)$\\
B.$x \in (3,11) \cup [18,\infty)$\\
C.$x \in (3,11] \cup [18,\infty)$\\
D.$x \in [3,11) \cup [18,\infty)$\\
E.$x \in [3,11] \cup (18,\infty)$\\
F.$x \in (3,11) \cup (18,\infty)$\\
G.$x \in [3,11) \cup (18,\infty)$\\
H.$x \in (3,11] \cup (18,\infty)$
\testStop
\kluczStart
A
\kluczStop



\zadStart{Zadanie z Wikieł Z 1.62 a) moja wersja nr 423}

Rozwiązać nierówności $(x-3)(x-11)(x-19)\ge0$.
\zadStop
\rozwStart{Patryk Wirkus}{}
Miejsca zerowe naszego wielomianu to: $3, 11, 19$.\\
Wielomian jest stopnia nieparzystego, ponadto znak współczynnika przy\linebreak najwyższej potędze x jest dodatni.\\ W związku z tym wykres wielomianu zaczyna się od lewej strony poniżej osi OX. A więc $$x \in [3,11] \cup [19,\infty).$$
\rozwStop
\odpStart
$x \in [3,11] \cup [19,\infty)$
\odpStop
\testStart
A.$x \in [3,11] \cup [19,\infty)$\\
B.$x \in (3,11) \cup [19,\infty)$\\
C.$x \in (3,11] \cup [19,\infty)$\\
D.$x \in [3,11) \cup [19,\infty)$\\
E.$x \in [3,11] \cup (19,\infty)$\\
F.$x \in (3,11) \cup (19,\infty)$\\
G.$x \in [3,11) \cup (19,\infty)$\\
H.$x \in (3,11] \cup (19,\infty)$
\testStop
\kluczStart
A
\kluczStop



\zadStart{Zadanie z Wikieł Z 1.62 a) moja wersja nr 424}

Rozwiązać nierówności $(x-3)(x-11)(x-20)\ge0$.
\zadStop
\rozwStart{Patryk Wirkus}{}
Miejsca zerowe naszego wielomianu to: $3, 11, 20$.\\
Wielomian jest stopnia nieparzystego, ponadto znak współczynnika przy\linebreak najwyższej potędze x jest dodatni.\\ W związku z tym wykres wielomianu zaczyna się od lewej strony poniżej osi OX. A więc $$x \in [3,11] \cup [20,\infty).$$
\rozwStop
\odpStart
$x \in [3,11] \cup [20,\infty)$
\odpStop
\testStart
A.$x \in [3,11] \cup [20,\infty)$\\
B.$x \in (3,11) \cup [20,\infty)$\\
C.$x \in (3,11] \cup [20,\infty)$\\
D.$x \in [3,11) \cup [20,\infty)$\\
E.$x \in [3,11] \cup (20,\infty)$\\
F.$x \in (3,11) \cup (20,\infty)$\\
G.$x \in [3,11) \cup (20,\infty)$\\
H.$x \in (3,11] \cup (20,\infty)$
\testStop
\kluczStart
A
\kluczStop



\zadStart{Zadanie z Wikieł Z 1.62 a) moja wersja nr 425}

Rozwiązać nierówności $(x-3)(x-12)(x-13)\ge0$.
\zadStop
\rozwStart{Patryk Wirkus}{}
Miejsca zerowe naszego wielomianu to: $3, 12, 13$.\\
Wielomian jest stopnia nieparzystego, ponadto znak współczynnika przy\linebreak najwyższej potędze x jest dodatni.\\ W związku z tym wykres wielomianu zaczyna się od lewej strony poniżej osi OX. A więc $$x \in [3,12] \cup [13,\infty).$$
\rozwStop
\odpStart
$x \in [3,12] \cup [13,\infty)$
\odpStop
\testStart
A.$x \in [3,12] \cup [13,\infty)$\\
B.$x \in (3,12) \cup [13,\infty)$\\
C.$x \in (3,12] \cup [13,\infty)$\\
D.$x \in [3,12) \cup [13,\infty)$\\
E.$x \in [3,12] \cup (13,\infty)$\\
F.$x \in (3,12) \cup (13,\infty)$\\
G.$x \in [3,12) \cup (13,\infty)$\\
H.$x \in (3,12] \cup (13,\infty)$
\testStop
\kluczStart
A
\kluczStop



\zadStart{Zadanie z Wikieł Z 1.62 a) moja wersja nr 426}

Rozwiązać nierówności $(x-3)(x-12)(x-14)\ge0$.
\zadStop
\rozwStart{Patryk Wirkus}{}
Miejsca zerowe naszego wielomianu to: $3, 12, 14$.\\
Wielomian jest stopnia nieparzystego, ponadto znak współczynnika przy\linebreak najwyższej potędze x jest dodatni.\\ W związku z tym wykres wielomianu zaczyna się od lewej strony poniżej osi OX. A więc $$x \in [3,12] \cup [14,\infty).$$
\rozwStop
\odpStart
$x \in [3,12] \cup [14,\infty)$
\odpStop
\testStart
A.$x \in [3,12] \cup [14,\infty)$\\
B.$x \in (3,12) \cup [14,\infty)$\\
C.$x \in (3,12] \cup [14,\infty)$\\
D.$x \in [3,12) \cup [14,\infty)$\\
E.$x \in [3,12] \cup (14,\infty)$\\
F.$x \in (3,12) \cup (14,\infty)$\\
G.$x \in [3,12) \cup (14,\infty)$\\
H.$x \in (3,12] \cup (14,\infty)$
\testStop
\kluczStart
A
\kluczStop



\zadStart{Zadanie z Wikieł Z 1.62 a) moja wersja nr 427}

Rozwiązać nierówności $(x-3)(x-12)(x-15)\ge0$.
\zadStop
\rozwStart{Patryk Wirkus}{}
Miejsca zerowe naszego wielomianu to: $3, 12, 15$.\\
Wielomian jest stopnia nieparzystego, ponadto znak współczynnika przy\linebreak najwyższej potędze x jest dodatni.\\ W związku z tym wykres wielomianu zaczyna się od lewej strony poniżej osi OX. A więc $$x \in [3,12] \cup [15,\infty).$$
\rozwStop
\odpStart
$x \in [3,12] \cup [15,\infty)$
\odpStop
\testStart
A.$x \in [3,12] \cup [15,\infty)$\\
B.$x \in (3,12) \cup [15,\infty)$\\
C.$x \in (3,12] \cup [15,\infty)$\\
D.$x \in [3,12) \cup [15,\infty)$\\
E.$x \in [3,12] \cup (15,\infty)$\\
F.$x \in (3,12) \cup (15,\infty)$\\
G.$x \in [3,12) \cup (15,\infty)$\\
H.$x \in (3,12] \cup (15,\infty)$
\testStop
\kluczStart
A
\kluczStop



\zadStart{Zadanie z Wikieł Z 1.62 a) moja wersja nr 428}

Rozwiązać nierówności $(x-3)(x-12)(x-16)\ge0$.
\zadStop
\rozwStart{Patryk Wirkus}{}
Miejsca zerowe naszego wielomianu to: $3, 12, 16$.\\
Wielomian jest stopnia nieparzystego, ponadto znak współczynnika przy\linebreak najwyższej potędze x jest dodatni.\\ W związku z tym wykres wielomianu zaczyna się od lewej strony poniżej osi OX. A więc $$x \in [3,12] \cup [16,\infty).$$
\rozwStop
\odpStart
$x \in [3,12] \cup [16,\infty)$
\odpStop
\testStart
A.$x \in [3,12] \cup [16,\infty)$\\
B.$x \in (3,12) \cup [16,\infty)$\\
C.$x \in (3,12] \cup [16,\infty)$\\
D.$x \in [3,12) \cup [16,\infty)$\\
E.$x \in [3,12] \cup (16,\infty)$\\
F.$x \in (3,12) \cup (16,\infty)$\\
G.$x \in [3,12) \cup (16,\infty)$\\
H.$x \in (3,12] \cup (16,\infty)$
\testStop
\kluczStart
A
\kluczStop



\zadStart{Zadanie z Wikieł Z 1.62 a) moja wersja nr 429}

Rozwiązać nierówności $(x-3)(x-12)(x-17)\ge0$.
\zadStop
\rozwStart{Patryk Wirkus}{}
Miejsca zerowe naszego wielomianu to: $3, 12, 17$.\\
Wielomian jest stopnia nieparzystego, ponadto znak współczynnika przy\linebreak najwyższej potędze x jest dodatni.\\ W związku z tym wykres wielomianu zaczyna się od lewej strony poniżej osi OX. A więc $$x \in [3,12] \cup [17,\infty).$$
\rozwStop
\odpStart
$x \in [3,12] \cup [17,\infty)$
\odpStop
\testStart
A.$x \in [3,12] \cup [17,\infty)$\\
B.$x \in (3,12) \cup [17,\infty)$\\
C.$x \in (3,12] \cup [17,\infty)$\\
D.$x \in [3,12) \cup [17,\infty)$\\
E.$x \in [3,12] \cup (17,\infty)$\\
F.$x \in (3,12) \cup (17,\infty)$\\
G.$x \in [3,12) \cup (17,\infty)$\\
H.$x \in (3,12] \cup (17,\infty)$
\testStop
\kluczStart
A
\kluczStop



\zadStart{Zadanie z Wikieł Z 1.62 a) moja wersja nr 430}

Rozwiązać nierówności $(x-3)(x-12)(x-18)\ge0$.
\zadStop
\rozwStart{Patryk Wirkus}{}
Miejsca zerowe naszego wielomianu to: $3, 12, 18$.\\
Wielomian jest stopnia nieparzystego, ponadto znak współczynnika przy\linebreak najwyższej potędze x jest dodatni.\\ W związku z tym wykres wielomianu zaczyna się od lewej strony poniżej osi OX. A więc $$x \in [3,12] \cup [18,\infty).$$
\rozwStop
\odpStart
$x \in [3,12] \cup [18,\infty)$
\odpStop
\testStart
A.$x \in [3,12] \cup [18,\infty)$\\
B.$x \in (3,12) \cup [18,\infty)$\\
C.$x \in (3,12] \cup [18,\infty)$\\
D.$x \in [3,12) \cup [18,\infty)$\\
E.$x \in [3,12] \cup (18,\infty)$\\
F.$x \in (3,12) \cup (18,\infty)$\\
G.$x \in [3,12) \cup (18,\infty)$\\
H.$x \in (3,12] \cup (18,\infty)$
\testStop
\kluczStart
A
\kluczStop



\zadStart{Zadanie z Wikieł Z 1.62 a) moja wersja nr 431}

Rozwiązać nierówności $(x-3)(x-12)(x-19)\ge0$.
\zadStop
\rozwStart{Patryk Wirkus}{}
Miejsca zerowe naszego wielomianu to: $3, 12, 19$.\\
Wielomian jest stopnia nieparzystego, ponadto znak współczynnika przy\linebreak najwyższej potędze x jest dodatni.\\ W związku z tym wykres wielomianu zaczyna się od lewej strony poniżej osi OX. A więc $$x \in [3,12] \cup [19,\infty).$$
\rozwStop
\odpStart
$x \in [3,12] \cup [19,\infty)$
\odpStop
\testStart
A.$x \in [3,12] \cup [19,\infty)$\\
B.$x \in (3,12) \cup [19,\infty)$\\
C.$x \in (3,12] \cup [19,\infty)$\\
D.$x \in [3,12) \cup [19,\infty)$\\
E.$x \in [3,12] \cup (19,\infty)$\\
F.$x \in (3,12) \cup (19,\infty)$\\
G.$x \in [3,12) \cup (19,\infty)$\\
H.$x \in (3,12] \cup (19,\infty)$
\testStop
\kluczStart
A
\kluczStop



\zadStart{Zadanie z Wikieł Z 1.62 a) moja wersja nr 432}

Rozwiązać nierówności $(x-3)(x-12)(x-20)\ge0$.
\zadStop
\rozwStart{Patryk Wirkus}{}
Miejsca zerowe naszego wielomianu to: $3, 12, 20$.\\
Wielomian jest stopnia nieparzystego, ponadto znak współczynnika przy\linebreak najwyższej potędze x jest dodatni.\\ W związku z tym wykres wielomianu zaczyna się od lewej strony poniżej osi OX. A więc $$x \in [3,12] \cup [20,\infty).$$
\rozwStop
\odpStart
$x \in [3,12] \cup [20,\infty)$
\odpStop
\testStart
A.$x \in [3,12] \cup [20,\infty)$\\
B.$x \in (3,12) \cup [20,\infty)$\\
C.$x \in (3,12] \cup [20,\infty)$\\
D.$x \in [3,12) \cup [20,\infty)$\\
E.$x \in [3,12] \cup (20,\infty)$\\
F.$x \in (3,12) \cup (20,\infty)$\\
G.$x \in [3,12) \cup (20,\infty)$\\
H.$x \in (3,12] \cup (20,\infty)$
\testStop
\kluczStart
A
\kluczStop



\zadStart{Zadanie z Wikieł Z 1.62 a) moja wersja nr 433}

Rozwiązać nierówności $(x-3)(x-13)(x-14)\ge0$.
\zadStop
\rozwStart{Patryk Wirkus}{}
Miejsca zerowe naszego wielomianu to: $3, 13, 14$.\\
Wielomian jest stopnia nieparzystego, ponadto znak współczynnika przy\linebreak najwyższej potędze x jest dodatni.\\ W związku z tym wykres wielomianu zaczyna się od lewej strony poniżej osi OX. A więc $$x \in [3,13] \cup [14,\infty).$$
\rozwStop
\odpStart
$x \in [3,13] \cup [14,\infty)$
\odpStop
\testStart
A.$x \in [3,13] \cup [14,\infty)$\\
B.$x \in (3,13) \cup [14,\infty)$\\
C.$x \in (3,13] \cup [14,\infty)$\\
D.$x \in [3,13) \cup [14,\infty)$\\
E.$x \in [3,13] \cup (14,\infty)$\\
F.$x \in (3,13) \cup (14,\infty)$\\
G.$x \in [3,13) \cup (14,\infty)$\\
H.$x \in (3,13] \cup (14,\infty)$
\testStop
\kluczStart
A
\kluczStop



\zadStart{Zadanie z Wikieł Z 1.62 a) moja wersja nr 434}

Rozwiązać nierówności $(x-3)(x-13)(x-15)\ge0$.
\zadStop
\rozwStart{Patryk Wirkus}{}
Miejsca zerowe naszego wielomianu to: $3, 13, 15$.\\
Wielomian jest stopnia nieparzystego, ponadto znak współczynnika przy\linebreak najwyższej potędze x jest dodatni.\\ W związku z tym wykres wielomianu zaczyna się od lewej strony poniżej osi OX. A więc $$x \in [3,13] \cup [15,\infty).$$
\rozwStop
\odpStart
$x \in [3,13] \cup [15,\infty)$
\odpStop
\testStart
A.$x \in [3,13] \cup [15,\infty)$\\
B.$x \in (3,13) \cup [15,\infty)$\\
C.$x \in (3,13] \cup [15,\infty)$\\
D.$x \in [3,13) \cup [15,\infty)$\\
E.$x \in [3,13] \cup (15,\infty)$\\
F.$x \in (3,13) \cup (15,\infty)$\\
G.$x \in [3,13) \cup (15,\infty)$\\
H.$x \in (3,13] \cup (15,\infty)$
\testStop
\kluczStart
A
\kluczStop



\zadStart{Zadanie z Wikieł Z 1.62 a) moja wersja nr 435}

Rozwiązać nierówności $(x-3)(x-13)(x-16)\ge0$.
\zadStop
\rozwStart{Patryk Wirkus}{}
Miejsca zerowe naszego wielomianu to: $3, 13, 16$.\\
Wielomian jest stopnia nieparzystego, ponadto znak współczynnika przy\linebreak najwyższej potędze x jest dodatni.\\ W związku z tym wykres wielomianu zaczyna się od lewej strony poniżej osi OX. A więc $$x \in [3,13] \cup [16,\infty).$$
\rozwStop
\odpStart
$x \in [3,13] \cup [16,\infty)$
\odpStop
\testStart
A.$x \in [3,13] \cup [16,\infty)$\\
B.$x \in (3,13) \cup [16,\infty)$\\
C.$x \in (3,13] \cup [16,\infty)$\\
D.$x \in [3,13) \cup [16,\infty)$\\
E.$x \in [3,13] \cup (16,\infty)$\\
F.$x \in (3,13) \cup (16,\infty)$\\
G.$x \in [3,13) \cup (16,\infty)$\\
H.$x \in (3,13] \cup (16,\infty)$
\testStop
\kluczStart
A
\kluczStop



\zadStart{Zadanie z Wikieł Z 1.62 a) moja wersja nr 436}

Rozwiązać nierówności $(x-3)(x-13)(x-17)\ge0$.
\zadStop
\rozwStart{Patryk Wirkus}{}
Miejsca zerowe naszego wielomianu to: $3, 13, 17$.\\
Wielomian jest stopnia nieparzystego, ponadto znak współczynnika przy\linebreak najwyższej potędze x jest dodatni.\\ W związku z tym wykres wielomianu zaczyna się od lewej strony poniżej osi OX. A więc $$x \in [3,13] \cup [17,\infty).$$
\rozwStop
\odpStart
$x \in [3,13] \cup [17,\infty)$
\odpStop
\testStart
A.$x \in [3,13] \cup [17,\infty)$\\
B.$x \in (3,13) \cup [17,\infty)$\\
C.$x \in (3,13] \cup [17,\infty)$\\
D.$x \in [3,13) \cup [17,\infty)$\\
E.$x \in [3,13] \cup (17,\infty)$\\
F.$x \in (3,13) \cup (17,\infty)$\\
G.$x \in [3,13) \cup (17,\infty)$\\
H.$x \in (3,13] \cup (17,\infty)$
\testStop
\kluczStart
A
\kluczStop



\zadStart{Zadanie z Wikieł Z 1.62 a) moja wersja nr 437}

Rozwiązać nierówności $(x-3)(x-13)(x-18)\ge0$.
\zadStop
\rozwStart{Patryk Wirkus}{}
Miejsca zerowe naszego wielomianu to: $3, 13, 18$.\\
Wielomian jest stopnia nieparzystego, ponadto znak współczynnika przy\linebreak najwyższej potędze x jest dodatni.\\ W związku z tym wykres wielomianu zaczyna się od lewej strony poniżej osi OX. A więc $$x \in [3,13] \cup [18,\infty).$$
\rozwStop
\odpStart
$x \in [3,13] \cup [18,\infty)$
\odpStop
\testStart
A.$x \in [3,13] \cup [18,\infty)$\\
B.$x \in (3,13) \cup [18,\infty)$\\
C.$x \in (3,13] \cup [18,\infty)$\\
D.$x \in [3,13) \cup [18,\infty)$\\
E.$x \in [3,13] \cup (18,\infty)$\\
F.$x \in (3,13) \cup (18,\infty)$\\
G.$x \in [3,13) \cup (18,\infty)$\\
H.$x \in (3,13] \cup (18,\infty)$
\testStop
\kluczStart
A
\kluczStop



\zadStart{Zadanie z Wikieł Z 1.62 a) moja wersja nr 438}

Rozwiązać nierówności $(x-3)(x-13)(x-19)\ge0$.
\zadStop
\rozwStart{Patryk Wirkus}{}
Miejsca zerowe naszego wielomianu to: $3, 13, 19$.\\
Wielomian jest stopnia nieparzystego, ponadto znak współczynnika przy\linebreak najwyższej potędze x jest dodatni.\\ W związku z tym wykres wielomianu zaczyna się od lewej strony poniżej osi OX. A więc $$x \in [3,13] \cup [19,\infty).$$
\rozwStop
\odpStart
$x \in [3,13] \cup [19,\infty)$
\odpStop
\testStart
A.$x \in [3,13] \cup [19,\infty)$\\
B.$x \in (3,13) \cup [19,\infty)$\\
C.$x \in (3,13] \cup [19,\infty)$\\
D.$x \in [3,13) \cup [19,\infty)$\\
E.$x \in [3,13] \cup (19,\infty)$\\
F.$x \in (3,13) \cup (19,\infty)$\\
G.$x \in [3,13) \cup (19,\infty)$\\
H.$x \in (3,13] \cup (19,\infty)$
\testStop
\kluczStart
A
\kluczStop



\zadStart{Zadanie z Wikieł Z 1.62 a) moja wersja nr 439}

Rozwiązać nierówności $(x-3)(x-13)(x-20)\ge0$.
\zadStop
\rozwStart{Patryk Wirkus}{}
Miejsca zerowe naszego wielomianu to: $3, 13, 20$.\\
Wielomian jest stopnia nieparzystego, ponadto znak współczynnika przy\linebreak najwyższej potędze x jest dodatni.\\ W związku z tym wykres wielomianu zaczyna się od lewej strony poniżej osi OX. A więc $$x \in [3,13] \cup [20,\infty).$$
\rozwStop
\odpStart
$x \in [3,13] \cup [20,\infty)$
\odpStop
\testStart
A.$x \in [3,13] \cup [20,\infty)$\\
B.$x \in (3,13) \cup [20,\infty)$\\
C.$x \in (3,13] \cup [20,\infty)$\\
D.$x \in [3,13) \cup [20,\infty)$\\
E.$x \in [3,13] \cup (20,\infty)$\\
F.$x \in (3,13) \cup (20,\infty)$\\
G.$x \in [3,13) \cup (20,\infty)$\\
H.$x \in (3,13] \cup (20,\infty)$
\testStop
\kluczStart
A
\kluczStop



\zadStart{Zadanie z Wikieł Z 1.62 a) moja wersja nr 440}

Rozwiązać nierówności $(x-3)(x-14)(x-15)\ge0$.
\zadStop
\rozwStart{Patryk Wirkus}{}
Miejsca zerowe naszego wielomianu to: $3, 14, 15$.\\
Wielomian jest stopnia nieparzystego, ponadto znak współczynnika przy\linebreak najwyższej potędze x jest dodatni.\\ W związku z tym wykres wielomianu zaczyna się od lewej strony poniżej osi OX. A więc $$x \in [3,14] \cup [15,\infty).$$
\rozwStop
\odpStart
$x \in [3,14] \cup [15,\infty)$
\odpStop
\testStart
A.$x \in [3,14] \cup [15,\infty)$\\
B.$x \in (3,14) \cup [15,\infty)$\\
C.$x \in (3,14] \cup [15,\infty)$\\
D.$x \in [3,14) \cup [15,\infty)$\\
E.$x \in [3,14] \cup (15,\infty)$\\
F.$x \in (3,14) \cup (15,\infty)$\\
G.$x \in [3,14) \cup (15,\infty)$\\
H.$x \in (3,14] \cup (15,\infty)$
\testStop
\kluczStart
A
\kluczStop



\zadStart{Zadanie z Wikieł Z 1.62 a) moja wersja nr 441}

Rozwiązać nierówności $(x-3)(x-14)(x-16)\ge0$.
\zadStop
\rozwStart{Patryk Wirkus}{}
Miejsca zerowe naszego wielomianu to: $3, 14, 16$.\\
Wielomian jest stopnia nieparzystego, ponadto znak współczynnika przy\linebreak najwyższej potędze x jest dodatni.\\ W związku z tym wykres wielomianu zaczyna się od lewej strony poniżej osi OX. A więc $$x \in [3,14] \cup [16,\infty).$$
\rozwStop
\odpStart
$x \in [3,14] \cup [16,\infty)$
\odpStop
\testStart
A.$x \in [3,14] \cup [16,\infty)$\\
B.$x \in (3,14) \cup [16,\infty)$\\
C.$x \in (3,14] \cup [16,\infty)$\\
D.$x \in [3,14) \cup [16,\infty)$\\
E.$x \in [3,14] \cup (16,\infty)$\\
F.$x \in (3,14) \cup (16,\infty)$\\
G.$x \in [3,14) \cup (16,\infty)$\\
H.$x \in (3,14] \cup (16,\infty)$
\testStop
\kluczStart
A
\kluczStop



\zadStart{Zadanie z Wikieł Z 1.62 a) moja wersja nr 442}

Rozwiązać nierówności $(x-3)(x-14)(x-17)\ge0$.
\zadStop
\rozwStart{Patryk Wirkus}{}
Miejsca zerowe naszego wielomianu to: $3, 14, 17$.\\
Wielomian jest stopnia nieparzystego, ponadto znak współczynnika przy\linebreak najwyższej potędze x jest dodatni.\\ W związku z tym wykres wielomianu zaczyna się od lewej strony poniżej osi OX. A więc $$x \in [3,14] \cup [17,\infty).$$
\rozwStop
\odpStart
$x \in [3,14] \cup [17,\infty)$
\odpStop
\testStart
A.$x \in [3,14] \cup [17,\infty)$\\
B.$x \in (3,14) \cup [17,\infty)$\\
C.$x \in (3,14] \cup [17,\infty)$\\
D.$x \in [3,14) \cup [17,\infty)$\\
E.$x \in [3,14] \cup (17,\infty)$\\
F.$x \in (3,14) \cup (17,\infty)$\\
G.$x \in [3,14) \cup (17,\infty)$\\
H.$x \in (3,14] \cup (17,\infty)$
\testStop
\kluczStart
A
\kluczStop



\zadStart{Zadanie z Wikieł Z 1.62 a) moja wersja nr 443}

Rozwiązać nierówności $(x-3)(x-14)(x-18)\ge0$.
\zadStop
\rozwStart{Patryk Wirkus}{}
Miejsca zerowe naszego wielomianu to: $3, 14, 18$.\\
Wielomian jest stopnia nieparzystego, ponadto znak współczynnika przy\linebreak najwyższej potędze x jest dodatni.\\ W związku z tym wykres wielomianu zaczyna się od lewej strony poniżej osi OX. A więc $$x \in [3,14] \cup [18,\infty).$$
\rozwStop
\odpStart
$x \in [3,14] \cup [18,\infty)$
\odpStop
\testStart
A.$x \in [3,14] \cup [18,\infty)$\\
B.$x \in (3,14) \cup [18,\infty)$\\
C.$x \in (3,14] \cup [18,\infty)$\\
D.$x \in [3,14) \cup [18,\infty)$\\
E.$x \in [3,14] \cup (18,\infty)$\\
F.$x \in (3,14) \cup (18,\infty)$\\
G.$x \in [3,14) \cup (18,\infty)$\\
H.$x \in (3,14] \cup (18,\infty)$
\testStop
\kluczStart
A
\kluczStop



\zadStart{Zadanie z Wikieł Z 1.62 a) moja wersja nr 444}

Rozwiązać nierówności $(x-3)(x-14)(x-19)\ge0$.
\zadStop
\rozwStart{Patryk Wirkus}{}
Miejsca zerowe naszego wielomianu to: $3, 14, 19$.\\
Wielomian jest stopnia nieparzystego, ponadto znak współczynnika przy\linebreak najwyższej potędze x jest dodatni.\\ W związku z tym wykres wielomianu zaczyna się od lewej strony poniżej osi OX. A więc $$x \in [3,14] \cup [19,\infty).$$
\rozwStop
\odpStart
$x \in [3,14] \cup [19,\infty)$
\odpStop
\testStart
A.$x \in [3,14] \cup [19,\infty)$\\
B.$x \in (3,14) \cup [19,\infty)$\\
C.$x \in (3,14] \cup [19,\infty)$\\
D.$x \in [3,14) \cup [19,\infty)$\\
E.$x \in [3,14] \cup (19,\infty)$\\
F.$x \in (3,14) \cup (19,\infty)$\\
G.$x \in [3,14) \cup (19,\infty)$\\
H.$x \in (3,14] \cup (19,\infty)$
\testStop
\kluczStart
A
\kluczStop



\zadStart{Zadanie z Wikieł Z 1.62 a) moja wersja nr 445}

Rozwiązać nierówności $(x-3)(x-14)(x-20)\ge0$.
\zadStop
\rozwStart{Patryk Wirkus}{}
Miejsca zerowe naszego wielomianu to: $3, 14, 20$.\\
Wielomian jest stopnia nieparzystego, ponadto znak współczynnika przy\linebreak najwyższej potędze x jest dodatni.\\ W związku z tym wykres wielomianu zaczyna się od lewej strony poniżej osi OX. A więc $$x \in [3,14] \cup [20,\infty).$$
\rozwStop
\odpStart
$x \in [3,14] \cup [20,\infty)$
\odpStop
\testStart
A.$x \in [3,14] \cup [20,\infty)$\\
B.$x \in (3,14) \cup [20,\infty)$\\
C.$x \in (3,14] \cup [20,\infty)$\\
D.$x \in [3,14) \cup [20,\infty)$\\
E.$x \in [3,14] \cup (20,\infty)$\\
F.$x \in (3,14) \cup (20,\infty)$\\
G.$x \in [3,14) \cup (20,\infty)$\\
H.$x \in (3,14] \cup (20,\infty)$
\testStop
\kluczStart
A
\kluczStop



\zadStart{Zadanie z Wikieł Z 1.62 a) moja wersja nr 446}

Rozwiązać nierówności $(x-3)(x-15)(x-16)\ge0$.
\zadStop
\rozwStart{Patryk Wirkus}{}
Miejsca zerowe naszego wielomianu to: $3, 15, 16$.\\
Wielomian jest stopnia nieparzystego, ponadto znak współczynnika przy\linebreak najwyższej potędze x jest dodatni.\\ W związku z tym wykres wielomianu zaczyna się od lewej strony poniżej osi OX. A więc $$x \in [3,15] \cup [16,\infty).$$
\rozwStop
\odpStart
$x \in [3,15] \cup [16,\infty)$
\odpStop
\testStart
A.$x \in [3,15] \cup [16,\infty)$\\
B.$x \in (3,15) \cup [16,\infty)$\\
C.$x \in (3,15] \cup [16,\infty)$\\
D.$x \in [3,15) \cup [16,\infty)$\\
E.$x \in [3,15] \cup (16,\infty)$\\
F.$x \in (3,15) \cup (16,\infty)$\\
G.$x \in [3,15) \cup (16,\infty)$\\
H.$x \in (3,15] \cup (16,\infty)$
\testStop
\kluczStart
A
\kluczStop



\zadStart{Zadanie z Wikieł Z 1.62 a) moja wersja nr 447}

Rozwiązać nierówności $(x-3)(x-15)(x-17)\ge0$.
\zadStop
\rozwStart{Patryk Wirkus}{}
Miejsca zerowe naszego wielomianu to: $3, 15, 17$.\\
Wielomian jest stopnia nieparzystego, ponadto znak współczynnika przy\linebreak najwyższej potędze x jest dodatni.\\ W związku z tym wykres wielomianu zaczyna się od lewej strony poniżej osi OX. A więc $$x \in [3,15] \cup [17,\infty).$$
\rozwStop
\odpStart
$x \in [3,15] \cup [17,\infty)$
\odpStop
\testStart
A.$x \in [3,15] \cup [17,\infty)$\\
B.$x \in (3,15) \cup [17,\infty)$\\
C.$x \in (3,15] \cup [17,\infty)$\\
D.$x \in [3,15) \cup [17,\infty)$\\
E.$x \in [3,15] \cup (17,\infty)$\\
F.$x \in (3,15) \cup (17,\infty)$\\
G.$x \in [3,15) \cup (17,\infty)$\\
H.$x \in (3,15] \cup (17,\infty)$
\testStop
\kluczStart
A
\kluczStop



\zadStart{Zadanie z Wikieł Z 1.62 a) moja wersja nr 448}

Rozwiązać nierówności $(x-3)(x-15)(x-18)\ge0$.
\zadStop
\rozwStart{Patryk Wirkus}{}
Miejsca zerowe naszego wielomianu to: $3, 15, 18$.\\
Wielomian jest stopnia nieparzystego, ponadto znak współczynnika przy\linebreak najwyższej potędze x jest dodatni.\\ W związku z tym wykres wielomianu zaczyna się od lewej strony poniżej osi OX. A więc $$x \in [3,15] \cup [18,\infty).$$
\rozwStop
\odpStart
$x \in [3,15] \cup [18,\infty)$
\odpStop
\testStart
A.$x \in [3,15] \cup [18,\infty)$\\
B.$x \in (3,15) \cup [18,\infty)$\\
C.$x \in (3,15] \cup [18,\infty)$\\
D.$x \in [3,15) \cup [18,\infty)$\\
E.$x \in [3,15] \cup (18,\infty)$\\
F.$x \in (3,15) \cup (18,\infty)$\\
G.$x \in [3,15) \cup (18,\infty)$\\
H.$x \in (3,15] \cup (18,\infty)$
\testStop
\kluczStart
A
\kluczStop



\zadStart{Zadanie z Wikieł Z 1.62 a) moja wersja nr 449}

Rozwiązać nierówności $(x-3)(x-15)(x-19)\ge0$.
\zadStop
\rozwStart{Patryk Wirkus}{}
Miejsca zerowe naszego wielomianu to: $3, 15, 19$.\\
Wielomian jest stopnia nieparzystego, ponadto znak współczynnika przy\linebreak najwyższej potędze x jest dodatni.\\ W związku z tym wykres wielomianu zaczyna się od lewej strony poniżej osi OX. A więc $$x \in [3,15] \cup [19,\infty).$$
\rozwStop
\odpStart
$x \in [3,15] \cup [19,\infty)$
\odpStop
\testStart
A.$x \in [3,15] \cup [19,\infty)$\\
B.$x \in (3,15) \cup [19,\infty)$\\
C.$x \in (3,15] \cup [19,\infty)$\\
D.$x \in [3,15) \cup [19,\infty)$\\
E.$x \in [3,15] \cup (19,\infty)$\\
F.$x \in (3,15) \cup (19,\infty)$\\
G.$x \in [3,15) \cup (19,\infty)$\\
H.$x \in (3,15] \cup (19,\infty)$
\testStop
\kluczStart
A
\kluczStop



\zadStart{Zadanie z Wikieł Z 1.62 a) moja wersja nr 450}

Rozwiązać nierówności $(x-3)(x-15)(x-20)\ge0$.
\zadStop
\rozwStart{Patryk Wirkus}{}
Miejsca zerowe naszego wielomianu to: $3, 15, 20$.\\
Wielomian jest stopnia nieparzystego, ponadto znak współczynnika przy\linebreak najwyższej potędze x jest dodatni.\\ W związku z tym wykres wielomianu zaczyna się od lewej strony poniżej osi OX. A więc $$x \in [3,15] \cup [20,\infty).$$
\rozwStop
\odpStart
$x \in [3,15] \cup [20,\infty)$
\odpStop
\testStart
A.$x \in [3,15] \cup [20,\infty)$\\
B.$x \in (3,15) \cup [20,\infty)$\\
C.$x \in (3,15] \cup [20,\infty)$\\
D.$x \in [3,15) \cup [20,\infty)$\\
E.$x \in [3,15] \cup (20,\infty)$\\
F.$x \in (3,15) \cup (20,\infty)$\\
G.$x \in [3,15) \cup (20,\infty)$\\
H.$x \in (3,15] \cup (20,\infty)$
\testStop
\kluczStart
A
\kluczStop



\zadStart{Zadanie z Wikieł Z 1.62 a) moja wersja nr 451}

Rozwiązać nierówności $(x-3)(x-16)(x-17)\ge0$.
\zadStop
\rozwStart{Patryk Wirkus}{}
Miejsca zerowe naszego wielomianu to: $3, 16, 17$.\\
Wielomian jest stopnia nieparzystego, ponadto znak współczynnika przy\linebreak najwyższej potędze x jest dodatni.\\ W związku z tym wykres wielomianu zaczyna się od lewej strony poniżej osi OX. A więc $$x \in [3,16] \cup [17,\infty).$$
\rozwStop
\odpStart
$x \in [3,16] \cup [17,\infty)$
\odpStop
\testStart
A.$x \in [3,16] \cup [17,\infty)$\\
B.$x \in (3,16) \cup [17,\infty)$\\
C.$x \in (3,16] \cup [17,\infty)$\\
D.$x \in [3,16) \cup [17,\infty)$\\
E.$x \in [3,16] \cup (17,\infty)$\\
F.$x \in (3,16) \cup (17,\infty)$\\
G.$x \in [3,16) \cup (17,\infty)$\\
H.$x \in (3,16] \cup (17,\infty)$
\testStop
\kluczStart
A
\kluczStop



\zadStart{Zadanie z Wikieł Z 1.62 a) moja wersja nr 452}

Rozwiązać nierówności $(x-3)(x-16)(x-18)\ge0$.
\zadStop
\rozwStart{Patryk Wirkus}{}
Miejsca zerowe naszego wielomianu to: $3, 16, 18$.\\
Wielomian jest stopnia nieparzystego, ponadto znak współczynnika przy\linebreak najwyższej potędze x jest dodatni.\\ W związku z tym wykres wielomianu zaczyna się od lewej strony poniżej osi OX. A więc $$x \in [3,16] \cup [18,\infty).$$
\rozwStop
\odpStart
$x \in [3,16] \cup [18,\infty)$
\odpStop
\testStart
A.$x \in [3,16] \cup [18,\infty)$\\
B.$x \in (3,16) \cup [18,\infty)$\\
C.$x \in (3,16] \cup [18,\infty)$\\
D.$x \in [3,16) \cup [18,\infty)$\\
E.$x \in [3,16] \cup (18,\infty)$\\
F.$x \in (3,16) \cup (18,\infty)$\\
G.$x \in [3,16) \cup (18,\infty)$\\
H.$x \in (3,16] \cup (18,\infty)$
\testStop
\kluczStart
A
\kluczStop



\zadStart{Zadanie z Wikieł Z 1.62 a) moja wersja nr 453}

Rozwiązać nierówności $(x-3)(x-16)(x-19)\ge0$.
\zadStop
\rozwStart{Patryk Wirkus}{}
Miejsca zerowe naszego wielomianu to: $3, 16, 19$.\\
Wielomian jest stopnia nieparzystego, ponadto znak współczynnika przy\linebreak najwyższej potędze x jest dodatni.\\ W związku z tym wykres wielomianu zaczyna się od lewej strony poniżej osi OX. A więc $$x \in [3,16] \cup [19,\infty).$$
\rozwStop
\odpStart
$x \in [3,16] \cup [19,\infty)$
\odpStop
\testStart
A.$x \in [3,16] \cup [19,\infty)$\\
B.$x \in (3,16) \cup [19,\infty)$\\
C.$x \in (3,16] \cup [19,\infty)$\\
D.$x \in [3,16) \cup [19,\infty)$\\
E.$x \in [3,16] \cup (19,\infty)$\\
F.$x \in (3,16) \cup (19,\infty)$\\
G.$x \in [3,16) \cup (19,\infty)$\\
H.$x \in (3,16] \cup (19,\infty)$
\testStop
\kluczStart
A
\kluczStop



\zadStart{Zadanie z Wikieł Z 1.62 a) moja wersja nr 454}

Rozwiązać nierówności $(x-3)(x-16)(x-20)\ge0$.
\zadStop
\rozwStart{Patryk Wirkus}{}
Miejsca zerowe naszego wielomianu to: $3, 16, 20$.\\
Wielomian jest stopnia nieparzystego, ponadto znak współczynnika przy\linebreak najwyższej potędze x jest dodatni.\\ W związku z tym wykres wielomianu zaczyna się od lewej strony poniżej osi OX. A więc $$x \in [3,16] \cup [20,\infty).$$
\rozwStop
\odpStart
$x \in [3,16] \cup [20,\infty)$
\odpStop
\testStart
A.$x \in [3,16] \cup [20,\infty)$\\
B.$x \in (3,16) \cup [20,\infty)$\\
C.$x \in (3,16] \cup [20,\infty)$\\
D.$x \in [3,16) \cup [20,\infty)$\\
E.$x \in [3,16] \cup (20,\infty)$\\
F.$x \in (3,16) \cup (20,\infty)$\\
G.$x \in [3,16) \cup (20,\infty)$\\
H.$x \in (3,16] \cup (20,\infty)$
\testStop
\kluczStart
A
\kluczStop



\zadStart{Zadanie z Wikieł Z 1.62 a) moja wersja nr 455}

Rozwiązać nierówności $(x-3)(x-17)(x-18)\ge0$.
\zadStop
\rozwStart{Patryk Wirkus}{}
Miejsca zerowe naszego wielomianu to: $3, 17, 18$.\\
Wielomian jest stopnia nieparzystego, ponadto znak współczynnika przy\linebreak najwyższej potędze x jest dodatni.\\ W związku z tym wykres wielomianu zaczyna się od lewej strony poniżej osi OX. A więc $$x \in [3,17] \cup [18,\infty).$$
\rozwStop
\odpStart
$x \in [3,17] \cup [18,\infty)$
\odpStop
\testStart
A.$x \in [3,17] \cup [18,\infty)$\\
B.$x \in (3,17) \cup [18,\infty)$\\
C.$x \in (3,17] \cup [18,\infty)$\\
D.$x \in [3,17) \cup [18,\infty)$\\
E.$x \in [3,17] \cup (18,\infty)$\\
F.$x \in (3,17) \cup (18,\infty)$\\
G.$x \in [3,17) \cup (18,\infty)$\\
H.$x \in (3,17] \cup (18,\infty)$
\testStop
\kluczStart
A
\kluczStop



\zadStart{Zadanie z Wikieł Z 1.62 a) moja wersja nr 456}

Rozwiązać nierówności $(x-3)(x-17)(x-19)\ge0$.
\zadStop
\rozwStart{Patryk Wirkus}{}
Miejsca zerowe naszego wielomianu to: $3, 17, 19$.\\
Wielomian jest stopnia nieparzystego, ponadto znak współczynnika przy\linebreak najwyższej potędze x jest dodatni.\\ W związku z tym wykres wielomianu zaczyna się od lewej strony poniżej osi OX. A więc $$x \in [3,17] \cup [19,\infty).$$
\rozwStop
\odpStart
$x \in [3,17] \cup [19,\infty)$
\odpStop
\testStart
A.$x \in [3,17] \cup [19,\infty)$\\
B.$x \in (3,17) \cup [19,\infty)$\\
C.$x \in (3,17] \cup [19,\infty)$\\
D.$x \in [3,17) \cup [19,\infty)$\\
E.$x \in [3,17] \cup (19,\infty)$\\
F.$x \in (3,17) \cup (19,\infty)$\\
G.$x \in [3,17) \cup (19,\infty)$\\
H.$x \in (3,17] \cup (19,\infty)$
\testStop
\kluczStart
A
\kluczStop



\zadStart{Zadanie z Wikieł Z 1.62 a) moja wersja nr 457}

Rozwiązać nierówności $(x-3)(x-17)(x-20)\ge0$.
\zadStop
\rozwStart{Patryk Wirkus}{}
Miejsca zerowe naszego wielomianu to: $3, 17, 20$.\\
Wielomian jest stopnia nieparzystego, ponadto znak współczynnika przy\linebreak najwyższej potędze x jest dodatni.\\ W związku z tym wykres wielomianu zaczyna się od lewej strony poniżej osi OX. A więc $$x \in [3,17] \cup [20,\infty).$$
\rozwStop
\odpStart
$x \in [3,17] \cup [20,\infty)$
\odpStop
\testStart
A.$x \in [3,17] \cup [20,\infty)$\\
B.$x \in (3,17) \cup [20,\infty)$\\
C.$x \in (3,17] \cup [20,\infty)$\\
D.$x \in [3,17) \cup [20,\infty)$\\
E.$x \in [3,17] \cup (20,\infty)$\\
F.$x \in (3,17) \cup (20,\infty)$\\
G.$x \in [3,17) \cup (20,\infty)$\\
H.$x \in (3,17] \cup (20,\infty)$
\testStop
\kluczStart
A
\kluczStop



\zadStart{Zadanie z Wikieł Z 1.62 a) moja wersja nr 458}

Rozwiązać nierówności $(x-3)(x-18)(x-19)\ge0$.
\zadStop
\rozwStart{Patryk Wirkus}{}
Miejsca zerowe naszego wielomianu to: $3, 18, 19$.\\
Wielomian jest stopnia nieparzystego, ponadto znak współczynnika przy\linebreak najwyższej potędze x jest dodatni.\\ W związku z tym wykres wielomianu zaczyna się od lewej strony poniżej osi OX. A więc $$x \in [3,18] \cup [19,\infty).$$
\rozwStop
\odpStart
$x \in [3,18] \cup [19,\infty)$
\odpStop
\testStart
A.$x \in [3,18] \cup [19,\infty)$\\
B.$x \in (3,18) \cup [19,\infty)$\\
C.$x \in (3,18] \cup [19,\infty)$\\
D.$x \in [3,18) \cup [19,\infty)$\\
E.$x \in [3,18] \cup (19,\infty)$\\
F.$x \in (3,18) \cup (19,\infty)$\\
G.$x \in [3,18) \cup (19,\infty)$\\
H.$x \in (3,18] \cup (19,\infty)$
\testStop
\kluczStart
A
\kluczStop



\zadStart{Zadanie z Wikieł Z 1.62 a) moja wersja nr 459}

Rozwiązać nierówności $(x-3)(x-18)(x-20)\ge0$.
\zadStop
\rozwStart{Patryk Wirkus}{}
Miejsca zerowe naszego wielomianu to: $3, 18, 20$.\\
Wielomian jest stopnia nieparzystego, ponadto znak współczynnika przy\linebreak najwyższej potędze x jest dodatni.\\ W związku z tym wykres wielomianu zaczyna się od lewej strony poniżej osi OX. A więc $$x \in [3,18] \cup [20,\infty).$$
\rozwStop
\odpStart
$x \in [3,18] \cup [20,\infty)$
\odpStop
\testStart
A.$x \in [3,18] \cup [20,\infty)$\\
B.$x \in (3,18) \cup [20,\infty)$\\
C.$x \in (3,18] \cup [20,\infty)$\\
D.$x \in [3,18) \cup [20,\infty)$\\
E.$x \in [3,18] \cup (20,\infty)$\\
F.$x \in (3,18) \cup (20,\infty)$\\
G.$x \in [3,18) \cup (20,\infty)$\\
H.$x \in (3,18] \cup (20,\infty)$
\testStop
\kluczStart
A
\kluczStop



\zadStart{Zadanie z Wikieł Z 1.62 a) moja wersja nr 460}

Rozwiązać nierówności $(x-3)(x-19)(x-20)\ge0$.
\zadStop
\rozwStart{Patryk Wirkus}{}
Miejsca zerowe naszego wielomianu to: $3, 19, 20$.\\
Wielomian jest stopnia nieparzystego, ponadto znak współczynnika przy\linebreak najwyższej potędze x jest dodatni.\\ W związku z tym wykres wielomianu zaczyna się od lewej strony poniżej osi OX. A więc $$x \in [3,19] \cup [20,\infty).$$
\rozwStop
\odpStart
$x \in [3,19] \cup [20,\infty)$
\odpStop
\testStart
A.$x \in [3,19] \cup [20,\infty)$\\
B.$x \in (3,19) \cup [20,\infty)$\\
C.$x \in (3,19] \cup [20,\infty)$\\
D.$x \in [3,19) \cup [20,\infty)$\\
E.$x \in [3,19] \cup (20,\infty)$\\
F.$x \in (3,19) \cup (20,\infty)$\\
G.$x \in [3,19) \cup (20,\infty)$\\
H.$x \in (3,19] \cup (20,\infty)$
\testStop
\kluczStart
A
\kluczStop



\zadStart{Zadanie z Wikieł Z 1.62 a) moja wersja nr 461}

Rozwiązać nierówności $(x-4)(x-5)(x-6)\ge0$.
\zadStop
\rozwStart{Patryk Wirkus}{}
Miejsca zerowe naszego wielomianu to: $4, 5, 6$.\\
Wielomian jest stopnia nieparzystego, ponadto znak współczynnika przy\linebreak najwyższej potędze x jest dodatni.\\ W związku z tym wykres wielomianu zaczyna się od lewej strony poniżej osi OX. A więc $$x \in [4,5] \cup [6,\infty).$$
\rozwStop
\odpStart
$x \in [4,5] \cup [6,\infty)$
\odpStop
\testStart
A.$x \in [4,5] \cup [6,\infty)$\\
B.$x \in (4,5) \cup [6,\infty)$\\
C.$x \in (4,5] \cup [6,\infty)$\\
D.$x \in [4,5) \cup [6,\infty)$\\
E.$x \in [4,5] \cup (6,\infty)$\\
F.$x \in (4,5) \cup (6,\infty)$\\
G.$x \in [4,5) \cup (6,\infty)$\\
H.$x \in (4,5] \cup (6,\infty)$
\testStop
\kluczStart
A
\kluczStop



\zadStart{Zadanie z Wikieł Z 1.62 a) moja wersja nr 462}

Rozwiązać nierówności $(x-4)(x-5)(x-7)\ge0$.
\zadStop
\rozwStart{Patryk Wirkus}{}
Miejsca zerowe naszego wielomianu to: $4, 5, 7$.\\
Wielomian jest stopnia nieparzystego, ponadto znak współczynnika przy\linebreak najwyższej potędze x jest dodatni.\\ W związku z tym wykres wielomianu zaczyna się od lewej strony poniżej osi OX. A więc $$x \in [4,5] \cup [7,\infty).$$
\rozwStop
\odpStart
$x \in [4,5] \cup [7,\infty)$
\odpStop
\testStart
A.$x \in [4,5] \cup [7,\infty)$\\
B.$x \in (4,5) \cup [7,\infty)$\\
C.$x \in (4,5] \cup [7,\infty)$\\
D.$x \in [4,5) \cup [7,\infty)$\\
E.$x \in [4,5] \cup (7,\infty)$\\
F.$x \in (4,5) \cup (7,\infty)$\\
G.$x \in [4,5) \cup (7,\infty)$\\
H.$x \in (4,5] \cup (7,\infty)$
\testStop
\kluczStart
A
\kluczStop



\zadStart{Zadanie z Wikieł Z 1.62 a) moja wersja nr 463}

Rozwiązać nierówności $(x-4)(x-5)(x-8)\ge0$.
\zadStop
\rozwStart{Patryk Wirkus}{}
Miejsca zerowe naszego wielomianu to: $4, 5, 8$.\\
Wielomian jest stopnia nieparzystego, ponadto znak współczynnika przy\linebreak najwyższej potędze x jest dodatni.\\ W związku z tym wykres wielomianu zaczyna się od lewej strony poniżej osi OX. A więc $$x \in [4,5] \cup [8,\infty).$$
\rozwStop
\odpStart
$x \in [4,5] \cup [8,\infty)$
\odpStop
\testStart
A.$x \in [4,5] \cup [8,\infty)$\\
B.$x \in (4,5) \cup [8,\infty)$\\
C.$x \in (4,5] \cup [8,\infty)$\\
D.$x \in [4,5) \cup [8,\infty)$\\
E.$x \in [4,5] \cup (8,\infty)$\\
F.$x \in (4,5) \cup (8,\infty)$\\
G.$x \in [4,5) \cup (8,\infty)$\\
H.$x \in (4,5] \cup (8,\infty)$
\testStop
\kluczStart
A
\kluczStop



\zadStart{Zadanie z Wikieł Z 1.62 a) moja wersja nr 464}

Rozwiązać nierówności $(x-4)(x-5)(x-9)\ge0$.
\zadStop
\rozwStart{Patryk Wirkus}{}
Miejsca zerowe naszego wielomianu to: $4, 5, 9$.\\
Wielomian jest stopnia nieparzystego, ponadto znak współczynnika przy\linebreak najwyższej potędze x jest dodatni.\\ W związku z tym wykres wielomianu zaczyna się od lewej strony poniżej osi OX. A więc $$x \in [4,5] \cup [9,\infty).$$
\rozwStop
\odpStart
$x \in [4,5] \cup [9,\infty)$
\odpStop
\testStart
A.$x \in [4,5] \cup [9,\infty)$\\
B.$x \in (4,5) \cup [9,\infty)$\\
C.$x \in (4,5] \cup [9,\infty)$\\
D.$x \in [4,5) \cup [9,\infty)$\\
E.$x \in [4,5] \cup (9,\infty)$\\
F.$x \in (4,5) \cup (9,\infty)$\\
G.$x \in [4,5) \cup (9,\infty)$\\
H.$x \in (4,5] \cup (9,\infty)$
\testStop
\kluczStart
A
\kluczStop



\zadStart{Zadanie z Wikieł Z 1.62 a) moja wersja nr 465}

Rozwiązać nierówności $(x-4)(x-5)(x-10)\ge0$.
\zadStop
\rozwStart{Patryk Wirkus}{}
Miejsca zerowe naszego wielomianu to: $4, 5, 10$.\\
Wielomian jest stopnia nieparzystego, ponadto znak współczynnika przy\linebreak najwyższej potędze x jest dodatni.\\ W związku z tym wykres wielomianu zaczyna się od lewej strony poniżej osi OX. A więc $$x \in [4,5] \cup [10,\infty).$$
\rozwStop
\odpStart
$x \in [4,5] \cup [10,\infty)$
\odpStop
\testStart
A.$x \in [4,5] \cup [10,\infty)$\\
B.$x \in (4,5) \cup [10,\infty)$\\
C.$x \in (4,5] \cup [10,\infty)$\\
D.$x \in [4,5) \cup [10,\infty)$\\
E.$x \in [4,5] \cup (10,\infty)$\\
F.$x \in (4,5) \cup (10,\infty)$\\
G.$x \in [4,5) \cup (10,\infty)$\\
H.$x \in (4,5] \cup (10,\infty)$
\testStop
\kluczStart
A
\kluczStop



\zadStart{Zadanie z Wikieł Z 1.62 a) moja wersja nr 466}

Rozwiązać nierówności $(x-4)(x-5)(x-11)\ge0$.
\zadStop
\rozwStart{Patryk Wirkus}{}
Miejsca zerowe naszego wielomianu to: $4, 5, 11$.\\
Wielomian jest stopnia nieparzystego, ponadto znak współczynnika przy\linebreak najwyższej potędze x jest dodatni.\\ W związku z tym wykres wielomianu zaczyna się od lewej strony poniżej osi OX. A więc $$x \in [4,5] \cup [11,\infty).$$
\rozwStop
\odpStart
$x \in [4,5] \cup [11,\infty)$
\odpStop
\testStart
A.$x \in [4,5] \cup [11,\infty)$\\
B.$x \in (4,5) \cup [11,\infty)$\\
C.$x \in (4,5] \cup [11,\infty)$\\
D.$x \in [4,5) \cup [11,\infty)$\\
E.$x \in [4,5] \cup (11,\infty)$\\
F.$x \in (4,5) \cup (11,\infty)$\\
G.$x \in [4,5) \cup (11,\infty)$\\
H.$x \in (4,5] \cup (11,\infty)$
\testStop
\kluczStart
A
\kluczStop



\zadStart{Zadanie z Wikieł Z 1.62 a) moja wersja nr 467}

Rozwiązać nierówności $(x-4)(x-5)(x-12)\ge0$.
\zadStop
\rozwStart{Patryk Wirkus}{}
Miejsca zerowe naszego wielomianu to: $4, 5, 12$.\\
Wielomian jest stopnia nieparzystego, ponadto znak współczynnika przy\linebreak najwyższej potędze x jest dodatni.\\ W związku z tym wykres wielomianu zaczyna się od lewej strony poniżej osi OX. A więc $$x \in [4,5] \cup [12,\infty).$$
\rozwStop
\odpStart
$x \in [4,5] \cup [12,\infty)$
\odpStop
\testStart
A.$x \in [4,5] \cup [12,\infty)$\\
B.$x \in (4,5) \cup [12,\infty)$\\
C.$x \in (4,5] \cup [12,\infty)$\\
D.$x \in [4,5) \cup [12,\infty)$\\
E.$x \in [4,5] \cup (12,\infty)$\\
F.$x \in (4,5) \cup (12,\infty)$\\
G.$x \in [4,5) \cup (12,\infty)$\\
H.$x \in (4,5] \cup (12,\infty)$
\testStop
\kluczStart
A
\kluczStop



\zadStart{Zadanie z Wikieł Z 1.62 a) moja wersja nr 468}

Rozwiązać nierówności $(x-4)(x-5)(x-13)\ge0$.
\zadStop
\rozwStart{Patryk Wirkus}{}
Miejsca zerowe naszego wielomianu to: $4, 5, 13$.\\
Wielomian jest stopnia nieparzystego, ponadto znak współczynnika przy\linebreak najwyższej potędze x jest dodatni.\\ W związku z tym wykres wielomianu zaczyna się od lewej strony poniżej osi OX. A więc $$x \in [4,5] \cup [13,\infty).$$
\rozwStop
\odpStart
$x \in [4,5] \cup [13,\infty)$
\odpStop
\testStart
A.$x \in [4,5] \cup [13,\infty)$\\
B.$x \in (4,5) \cup [13,\infty)$\\
C.$x \in (4,5] \cup [13,\infty)$\\
D.$x \in [4,5) \cup [13,\infty)$\\
E.$x \in [4,5] \cup (13,\infty)$\\
F.$x \in (4,5) \cup (13,\infty)$\\
G.$x \in [4,5) \cup (13,\infty)$\\
H.$x \in (4,5] \cup (13,\infty)$
\testStop
\kluczStart
A
\kluczStop



\zadStart{Zadanie z Wikieł Z 1.62 a) moja wersja nr 469}

Rozwiązać nierówności $(x-4)(x-5)(x-14)\ge0$.
\zadStop
\rozwStart{Patryk Wirkus}{}
Miejsca zerowe naszego wielomianu to: $4, 5, 14$.\\
Wielomian jest stopnia nieparzystego, ponadto znak współczynnika przy\linebreak najwyższej potędze x jest dodatni.\\ W związku z tym wykres wielomianu zaczyna się od lewej strony poniżej osi OX. A więc $$x \in [4,5] \cup [14,\infty).$$
\rozwStop
\odpStart
$x \in [4,5] \cup [14,\infty)$
\odpStop
\testStart
A.$x \in [4,5] \cup [14,\infty)$\\
B.$x \in (4,5) \cup [14,\infty)$\\
C.$x \in (4,5] \cup [14,\infty)$\\
D.$x \in [4,5) \cup [14,\infty)$\\
E.$x \in [4,5] \cup (14,\infty)$\\
F.$x \in (4,5) \cup (14,\infty)$\\
G.$x \in [4,5) \cup (14,\infty)$\\
H.$x \in (4,5] \cup (14,\infty)$
\testStop
\kluczStart
A
\kluczStop



\zadStart{Zadanie z Wikieł Z 1.62 a) moja wersja nr 470}

Rozwiązać nierówności $(x-4)(x-5)(x-15)\ge0$.
\zadStop
\rozwStart{Patryk Wirkus}{}
Miejsca zerowe naszego wielomianu to: $4, 5, 15$.\\
Wielomian jest stopnia nieparzystego, ponadto znak współczynnika przy\linebreak najwyższej potędze x jest dodatni.\\ W związku z tym wykres wielomianu zaczyna się od lewej strony poniżej osi OX. A więc $$x \in [4,5] \cup [15,\infty).$$
\rozwStop
\odpStart
$x \in [4,5] \cup [15,\infty)$
\odpStop
\testStart
A.$x \in [4,5] \cup [15,\infty)$\\
B.$x \in (4,5) \cup [15,\infty)$\\
C.$x \in (4,5] \cup [15,\infty)$\\
D.$x \in [4,5) \cup [15,\infty)$\\
E.$x \in [4,5] \cup (15,\infty)$\\
F.$x \in (4,5) \cup (15,\infty)$\\
G.$x \in [4,5) \cup (15,\infty)$\\
H.$x \in (4,5] \cup (15,\infty)$
\testStop
\kluczStart
A
\kluczStop



\zadStart{Zadanie z Wikieł Z 1.62 a) moja wersja nr 471}

Rozwiązać nierówności $(x-4)(x-5)(x-16)\ge0$.
\zadStop
\rozwStart{Patryk Wirkus}{}
Miejsca zerowe naszego wielomianu to: $4, 5, 16$.\\
Wielomian jest stopnia nieparzystego, ponadto znak współczynnika przy\linebreak najwyższej potędze x jest dodatni.\\ W związku z tym wykres wielomianu zaczyna się od lewej strony poniżej osi OX. A więc $$x \in [4,5] \cup [16,\infty).$$
\rozwStop
\odpStart
$x \in [4,5] \cup [16,\infty)$
\odpStop
\testStart
A.$x \in [4,5] \cup [16,\infty)$\\
B.$x \in (4,5) \cup [16,\infty)$\\
C.$x \in (4,5] \cup [16,\infty)$\\
D.$x \in [4,5) \cup [16,\infty)$\\
E.$x \in [4,5] \cup (16,\infty)$\\
F.$x \in (4,5) \cup (16,\infty)$\\
G.$x \in [4,5) \cup (16,\infty)$\\
H.$x \in (4,5] \cup (16,\infty)$
\testStop
\kluczStart
A
\kluczStop



\zadStart{Zadanie z Wikieł Z 1.62 a) moja wersja nr 472}

Rozwiązać nierówności $(x-4)(x-5)(x-17)\ge0$.
\zadStop
\rozwStart{Patryk Wirkus}{}
Miejsca zerowe naszego wielomianu to: $4, 5, 17$.\\
Wielomian jest stopnia nieparzystego, ponadto znak współczynnika przy\linebreak najwyższej potędze x jest dodatni.\\ W związku z tym wykres wielomianu zaczyna się od lewej strony poniżej osi OX. A więc $$x \in [4,5] \cup [17,\infty).$$
\rozwStop
\odpStart
$x \in [4,5] \cup [17,\infty)$
\odpStop
\testStart
A.$x \in [4,5] \cup [17,\infty)$\\
B.$x \in (4,5) \cup [17,\infty)$\\
C.$x \in (4,5] \cup [17,\infty)$\\
D.$x \in [4,5) \cup [17,\infty)$\\
E.$x \in [4,5] \cup (17,\infty)$\\
F.$x \in (4,5) \cup (17,\infty)$\\
G.$x \in [4,5) \cup (17,\infty)$\\
H.$x \in (4,5] \cup (17,\infty)$
\testStop
\kluczStart
A
\kluczStop



\zadStart{Zadanie z Wikieł Z 1.62 a) moja wersja nr 473}

Rozwiązać nierówności $(x-4)(x-5)(x-18)\ge0$.
\zadStop
\rozwStart{Patryk Wirkus}{}
Miejsca zerowe naszego wielomianu to: $4, 5, 18$.\\
Wielomian jest stopnia nieparzystego, ponadto znak współczynnika przy\linebreak najwyższej potędze x jest dodatni.\\ W związku z tym wykres wielomianu zaczyna się od lewej strony poniżej osi OX. A więc $$x \in [4,5] \cup [18,\infty).$$
\rozwStop
\odpStart
$x \in [4,5] \cup [18,\infty)$
\odpStop
\testStart
A.$x \in [4,5] \cup [18,\infty)$\\
B.$x \in (4,5) \cup [18,\infty)$\\
C.$x \in (4,5] \cup [18,\infty)$\\
D.$x \in [4,5) \cup [18,\infty)$\\
E.$x \in [4,5] \cup (18,\infty)$\\
F.$x \in (4,5) \cup (18,\infty)$\\
G.$x \in [4,5) \cup (18,\infty)$\\
H.$x \in (4,5] \cup (18,\infty)$
\testStop
\kluczStart
A
\kluczStop



\zadStart{Zadanie z Wikieł Z 1.62 a) moja wersja nr 474}

Rozwiązać nierówności $(x-4)(x-5)(x-19)\ge0$.
\zadStop
\rozwStart{Patryk Wirkus}{}
Miejsca zerowe naszego wielomianu to: $4, 5, 19$.\\
Wielomian jest stopnia nieparzystego, ponadto znak współczynnika przy\linebreak najwyższej potędze x jest dodatni.\\ W związku z tym wykres wielomianu zaczyna się od lewej strony poniżej osi OX. A więc $$x \in [4,5] \cup [19,\infty).$$
\rozwStop
\odpStart
$x \in [4,5] \cup [19,\infty)$
\odpStop
\testStart
A.$x \in [4,5] \cup [19,\infty)$\\
B.$x \in (4,5) \cup [19,\infty)$\\
C.$x \in (4,5] \cup [19,\infty)$\\
D.$x \in [4,5) \cup [19,\infty)$\\
E.$x \in [4,5] \cup (19,\infty)$\\
F.$x \in (4,5) \cup (19,\infty)$\\
G.$x \in [4,5) \cup (19,\infty)$\\
H.$x \in (4,5] \cup (19,\infty)$
\testStop
\kluczStart
A
\kluczStop



\zadStart{Zadanie z Wikieł Z 1.62 a) moja wersja nr 475}

Rozwiązać nierówności $(x-4)(x-5)(x-20)\ge0$.
\zadStop
\rozwStart{Patryk Wirkus}{}
Miejsca zerowe naszego wielomianu to: $4, 5, 20$.\\
Wielomian jest stopnia nieparzystego, ponadto znak współczynnika przy\linebreak najwyższej potędze x jest dodatni.\\ W związku z tym wykres wielomianu zaczyna się od lewej strony poniżej osi OX. A więc $$x \in [4,5] \cup [20,\infty).$$
\rozwStop
\odpStart
$x \in [4,5] \cup [20,\infty)$
\odpStop
\testStart
A.$x \in [4,5] \cup [20,\infty)$\\
B.$x \in (4,5) \cup [20,\infty)$\\
C.$x \in (4,5] \cup [20,\infty)$\\
D.$x \in [4,5) \cup [20,\infty)$\\
E.$x \in [4,5] \cup (20,\infty)$\\
F.$x \in (4,5) \cup (20,\infty)$\\
G.$x \in [4,5) \cup (20,\infty)$\\
H.$x \in (4,5] \cup (20,\infty)$
\testStop
\kluczStart
A
\kluczStop



\zadStart{Zadanie z Wikieł Z 1.62 a) moja wersja nr 476}

Rozwiązać nierówności $(x-4)(x-6)(x-7)\ge0$.
\zadStop
\rozwStart{Patryk Wirkus}{}
Miejsca zerowe naszego wielomianu to: $4, 6, 7$.\\
Wielomian jest stopnia nieparzystego, ponadto znak współczynnika przy\linebreak najwyższej potędze x jest dodatni.\\ W związku z tym wykres wielomianu zaczyna się od lewej strony poniżej osi OX. A więc $$x \in [4,6] \cup [7,\infty).$$
\rozwStop
\odpStart
$x \in [4,6] \cup [7,\infty)$
\odpStop
\testStart
A.$x \in [4,6] \cup [7,\infty)$\\
B.$x \in (4,6) \cup [7,\infty)$\\
C.$x \in (4,6] \cup [7,\infty)$\\
D.$x \in [4,6) \cup [7,\infty)$\\
E.$x \in [4,6] \cup (7,\infty)$\\
F.$x \in (4,6) \cup (7,\infty)$\\
G.$x \in [4,6) \cup (7,\infty)$\\
H.$x \in (4,6] \cup (7,\infty)$
\testStop
\kluczStart
A
\kluczStop



\zadStart{Zadanie z Wikieł Z 1.62 a) moja wersja nr 477}

Rozwiązać nierówności $(x-4)(x-6)(x-8)\ge0$.
\zadStop
\rozwStart{Patryk Wirkus}{}
Miejsca zerowe naszego wielomianu to: $4, 6, 8$.\\
Wielomian jest stopnia nieparzystego, ponadto znak współczynnika przy\linebreak najwyższej potędze x jest dodatni.\\ W związku z tym wykres wielomianu zaczyna się od lewej strony poniżej osi OX. A więc $$x \in [4,6] \cup [8,\infty).$$
\rozwStop
\odpStart
$x \in [4,6] \cup [8,\infty)$
\odpStop
\testStart
A.$x \in [4,6] \cup [8,\infty)$\\
B.$x \in (4,6) \cup [8,\infty)$\\
C.$x \in (4,6] \cup [8,\infty)$\\
D.$x \in [4,6) \cup [8,\infty)$\\
E.$x \in [4,6] \cup (8,\infty)$\\
F.$x \in (4,6) \cup (8,\infty)$\\
G.$x \in [4,6) \cup (8,\infty)$\\
H.$x \in (4,6] \cup (8,\infty)$
\testStop
\kluczStart
A
\kluczStop



\zadStart{Zadanie z Wikieł Z 1.62 a) moja wersja nr 478}

Rozwiązać nierówności $(x-4)(x-6)(x-9)\ge0$.
\zadStop
\rozwStart{Patryk Wirkus}{}
Miejsca zerowe naszego wielomianu to: $4, 6, 9$.\\
Wielomian jest stopnia nieparzystego, ponadto znak współczynnika przy\linebreak najwyższej potędze x jest dodatni.\\ W związku z tym wykres wielomianu zaczyna się od lewej strony poniżej osi OX. A więc $$x \in [4,6] \cup [9,\infty).$$
\rozwStop
\odpStart
$x \in [4,6] \cup [9,\infty)$
\odpStop
\testStart
A.$x \in [4,6] \cup [9,\infty)$\\
B.$x \in (4,6) \cup [9,\infty)$\\
C.$x \in (4,6] \cup [9,\infty)$\\
D.$x \in [4,6) \cup [9,\infty)$\\
E.$x \in [4,6] \cup (9,\infty)$\\
F.$x \in (4,6) \cup (9,\infty)$\\
G.$x \in [4,6) \cup (9,\infty)$\\
H.$x \in (4,6] \cup (9,\infty)$
\testStop
\kluczStart
A
\kluczStop



\zadStart{Zadanie z Wikieł Z 1.62 a) moja wersja nr 479}

Rozwiązać nierówności $(x-4)(x-6)(x-10)\ge0$.
\zadStop
\rozwStart{Patryk Wirkus}{}
Miejsca zerowe naszego wielomianu to: $4, 6, 10$.\\
Wielomian jest stopnia nieparzystego, ponadto znak współczynnika przy\linebreak najwyższej potędze x jest dodatni.\\ W związku z tym wykres wielomianu zaczyna się od lewej strony poniżej osi OX. A więc $$x \in [4,6] \cup [10,\infty).$$
\rozwStop
\odpStart
$x \in [4,6] \cup [10,\infty)$
\odpStop
\testStart
A.$x \in [4,6] \cup [10,\infty)$\\
B.$x \in (4,6) \cup [10,\infty)$\\
C.$x \in (4,6] \cup [10,\infty)$\\
D.$x \in [4,6) \cup [10,\infty)$\\
E.$x \in [4,6] \cup (10,\infty)$\\
F.$x \in (4,6) \cup (10,\infty)$\\
G.$x \in [4,6) \cup (10,\infty)$\\
H.$x \in (4,6] \cup (10,\infty)$
\testStop
\kluczStart
A
\kluczStop



\zadStart{Zadanie z Wikieł Z 1.62 a) moja wersja nr 480}

Rozwiązać nierówności $(x-4)(x-6)(x-11)\ge0$.
\zadStop
\rozwStart{Patryk Wirkus}{}
Miejsca zerowe naszego wielomianu to: $4, 6, 11$.\\
Wielomian jest stopnia nieparzystego, ponadto znak współczynnika przy\linebreak najwyższej potędze x jest dodatni.\\ W związku z tym wykres wielomianu zaczyna się od lewej strony poniżej osi OX. A więc $$x \in [4,6] \cup [11,\infty).$$
\rozwStop
\odpStart
$x \in [4,6] \cup [11,\infty)$
\odpStop
\testStart
A.$x \in [4,6] \cup [11,\infty)$\\
B.$x \in (4,6) \cup [11,\infty)$\\
C.$x \in (4,6] \cup [11,\infty)$\\
D.$x \in [4,6) \cup [11,\infty)$\\
E.$x \in [4,6] \cup (11,\infty)$\\
F.$x \in (4,6) \cup (11,\infty)$\\
G.$x \in [4,6) \cup (11,\infty)$\\
H.$x \in (4,6] \cup (11,\infty)$
\testStop
\kluczStart
A
\kluczStop



\zadStart{Zadanie z Wikieł Z 1.62 a) moja wersja nr 481}

Rozwiązać nierówności $(x-4)(x-6)(x-12)\ge0$.
\zadStop
\rozwStart{Patryk Wirkus}{}
Miejsca zerowe naszego wielomianu to: $4, 6, 12$.\\
Wielomian jest stopnia nieparzystego, ponadto znak współczynnika przy\linebreak najwyższej potędze x jest dodatni.\\ W związku z tym wykres wielomianu zaczyna się od lewej strony poniżej osi OX. A więc $$x \in [4,6] \cup [12,\infty).$$
\rozwStop
\odpStart
$x \in [4,6] \cup [12,\infty)$
\odpStop
\testStart
A.$x \in [4,6] \cup [12,\infty)$\\
B.$x \in (4,6) \cup [12,\infty)$\\
C.$x \in (4,6] \cup [12,\infty)$\\
D.$x \in [4,6) \cup [12,\infty)$\\
E.$x \in [4,6] \cup (12,\infty)$\\
F.$x \in (4,6) \cup (12,\infty)$\\
G.$x \in [4,6) \cup (12,\infty)$\\
H.$x \in (4,6] \cup (12,\infty)$
\testStop
\kluczStart
A
\kluczStop



\zadStart{Zadanie z Wikieł Z 1.62 a) moja wersja nr 482}

Rozwiązać nierówności $(x-4)(x-6)(x-13)\ge0$.
\zadStop
\rozwStart{Patryk Wirkus}{}
Miejsca zerowe naszego wielomianu to: $4, 6, 13$.\\
Wielomian jest stopnia nieparzystego, ponadto znak współczynnika przy\linebreak najwyższej potędze x jest dodatni.\\ W związku z tym wykres wielomianu zaczyna się od lewej strony poniżej osi OX. A więc $$x \in [4,6] \cup [13,\infty).$$
\rozwStop
\odpStart
$x \in [4,6] \cup [13,\infty)$
\odpStop
\testStart
A.$x \in [4,6] \cup [13,\infty)$\\
B.$x \in (4,6) \cup [13,\infty)$\\
C.$x \in (4,6] \cup [13,\infty)$\\
D.$x \in [4,6) \cup [13,\infty)$\\
E.$x \in [4,6] \cup (13,\infty)$\\
F.$x \in (4,6) \cup (13,\infty)$\\
G.$x \in [4,6) \cup (13,\infty)$\\
H.$x \in (4,6] \cup (13,\infty)$
\testStop
\kluczStart
A
\kluczStop



\zadStart{Zadanie z Wikieł Z 1.62 a) moja wersja nr 483}

Rozwiązać nierówności $(x-4)(x-6)(x-14)\ge0$.
\zadStop
\rozwStart{Patryk Wirkus}{}
Miejsca zerowe naszego wielomianu to: $4, 6, 14$.\\
Wielomian jest stopnia nieparzystego, ponadto znak współczynnika przy\linebreak najwyższej potędze x jest dodatni.\\ W związku z tym wykres wielomianu zaczyna się od lewej strony poniżej osi OX. A więc $$x \in [4,6] \cup [14,\infty).$$
\rozwStop
\odpStart
$x \in [4,6] \cup [14,\infty)$
\odpStop
\testStart
A.$x \in [4,6] \cup [14,\infty)$\\
B.$x \in (4,6) \cup [14,\infty)$\\
C.$x \in (4,6] \cup [14,\infty)$\\
D.$x \in [4,6) \cup [14,\infty)$\\
E.$x \in [4,6] \cup (14,\infty)$\\
F.$x \in (4,6) \cup (14,\infty)$\\
G.$x \in [4,6) \cup (14,\infty)$\\
H.$x \in (4,6] \cup (14,\infty)$
\testStop
\kluczStart
A
\kluczStop



\zadStart{Zadanie z Wikieł Z 1.62 a) moja wersja nr 484}

Rozwiązać nierówności $(x-4)(x-6)(x-15)\ge0$.
\zadStop
\rozwStart{Patryk Wirkus}{}
Miejsca zerowe naszego wielomianu to: $4, 6, 15$.\\
Wielomian jest stopnia nieparzystego, ponadto znak współczynnika przy\linebreak najwyższej potędze x jest dodatni.\\ W związku z tym wykres wielomianu zaczyna się od lewej strony poniżej osi OX. A więc $$x \in [4,6] \cup [15,\infty).$$
\rozwStop
\odpStart
$x \in [4,6] \cup [15,\infty)$
\odpStop
\testStart
A.$x \in [4,6] \cup [15,\infty)$\\
B.$x \in (4,6) \cup [15,\infty)$\\
C.$x \in (4,6] \cup [15,\infty)$\\
D.$x \in [4,6) \cup [15,\infty)$\\
E.$x \in [4,6] \cup (15,\infty)$\\
F.$x \in (4,6) \cup (15,\infty)$\\
G.$x \in [4,6) \cup (15,\infty)$\\
H.$x \in (4,6] \cup (15,\infty)$
\testStop
\kluczStart
A
\kluczStop



\zadStart{Zadanie z Wikieł Z 1.62 a) moja wersja nr 485}

Rozwiązać nierówności $(x-4)(x-6)(x-16)\ge0$.
\zadStop
\rozwStart{Patryk Wirkus}{}
Miejsca zerowe naszego wielomianu to: $4, 6, 16$.\\
Wielomian jest stopnia nieparzystego, ponadto znak współczynnika przy\linebreak najwyższej potędze x jest dodatni.\\ W związku z tym wykres wielomianu zaczyna się od lewej strony poniżej osi OX. A więc $$x \in [4,6] \cup [16,\infty).$$
\rozwStop
\odpStart
$x \in [4,6] \cup [16,\infty)$
\odpStop
\testStart
A.$x \in [4,6] \cup [16,\infty)$\\
B.$x \in (4,6) \cup [16,\infty)$\\
C.$x \in (4,6] \cup [16,\infty)$\\
D.$x \in [4,6) \cup [16,\infty)$\\
E.$x \in [4,6] \cup (16,\infty)$\\
F.$x \in (4,6) \cup (16,\infty)$\\
G.$x \in [4,6) \cup (16,\infty)$\\
H.$x \in (4,6] \cup (16,\infty)$
\testStop
\kluczStart
A
\kluczStop



\zadStart{Zadanie z Wikieł Z 1.62 a) moja wersja nr 486}

Rozwiązać nierówności $(x-4)(x-6)(x-17)\ge0$.
\zadStop
\rozwStart{Patryk Wirkus}{}
Miejsca zerowe naszego wielomianu to: $4, 6, 17$.\\
Wielomian jest stopnia nieparzystego, ponadto znak współczynnika przy\linebreak najwyższej potędze x jest dodatni.\\ W związku z tym wykres wielomianu zaczyna się od lewej strony poniżej osi OX. A więc $$x \in [4,6] \cup [17,\infty).$$
\rozwStop
\odpStart
$x \in [4,6] \cup [17,\infty)$
\odpStop
\testStart
A.$x \in [4,6] \cup [17,\infty)$\\
B.$x \in (4,6) \cup [17,\infty)$\\
C.$x \in (4,6] \cup [17,\infty)$\\
D.$x \in [4,6) \cup [17,\infty)$\\
E.$x \in [4,6] \cup (17,\infty)$\\
F.$x \in (4,6) \cup (17,\infty)$\\
G.$x \in [4,6) \cup (17,\infty)$\\
H.$x \in (4,6] \cup (17,\infty)$
\testStop
\kluczStart
A
\kluczStop



\zadStart{Zadanie z Wikieł Z 1.62 a) moja wersja nr 487}

Rozwiązać nierówności $(x-4)(x-6)(x-18)\ge0$.
\zadStop
\rozwStart{Patryk Wirkus}{}
Miejsca zerowe naszego wielomianu to: $4, 6, 18$.\\
Wielomian jest stopnia nieparzystego, ponadto znak współczynnika przy\linebreak najwyższej potędze x jest dodatni.\\ W związku z tym wykres wielomianu zaczyna się od lewej strony poniżej osi OX. A więc $$x \in [4,6] \cup [18,\infty).$$
\rozwStop
\odpStart
$x \in [4,6] \cup [18,\infty)$
\odpStop
\testStart
A.$x \in [4,6] \cup [18,\infty)$\\
B.$x \in (4,6) \cup [18,\infty)$\\
C.$x \in (4,6] \cup [18,\infty)$\\
D.$x \in [4,6) \cup [18,\infty)$\\
E.$x \in [4,6] \cup (18,\infty)$\\
F.$x \in (4,6) \cup (18,\infty)$\\
G.$x \in [4,6) \cup (18,\infty)$\\
H.$x \in (4,6] \cup (18,\infty)$
\testStop
\kluczStart
A
\kluczStop



\zadStart{Zadanie z Wikieł Z 1.62 a) moja wersja nr 488}

Rozwiązać nierówności $(x-4)(x-6)(x-19)\ge0$.
\zadStop
\rozwStart{Patryk Wirkus}{}
Miejsca zerowe naszego wielomianu to: $4, 6, 19$.\\
Wielomian jest stopnia nieparzystego, ponadto znak współczynnika przy\linebreak najwyższej potędze x jest dodatni.\\ W związku z tym wykres wielomianu zaczyna się od lewej strony poniżej osi OX. A więc $$x \in [4,6] \cup [19,\infty).$$
\rozwStop
\odpStart
$x \in [4,6] \cup [19,\infty)$
\odpStop
\testStart
A.$x \in [4,6] \cup [19,\infty)$\\
B.$x \in (4,6) \cup [19,\infty)$\\
C.$x \in (4,6] \cup [19,\infty)$\\
D.$x \in [4,6) \cup [19,\infty)$\\
E.$x \in [4,6] \cup (19,\infty)$\\
F.$x \in (4,6) \cup (19,\infty)$\\
G.$x \in [4,6) \cup (19,\infty)$\\
H.$x \in (4,6] \cup (19,\infty)$
\testStop
\kluczStart
A
\kluczStop



\zadStart{Zadanie z Wikieł Z 1.62 a) moja wersja nr 489}

Rozwiązać nierówności $(x-4)(x-6)(x-20)\ge0$.
\zadStop
\rozwStart{Patryk Wirkus}{}
Miejsca zerowe naszego wielomianu to: $4, 6, 20$.\\
Wielomian jest stopnia nieparzystego, ponadto znak współczynnika przy\linebreak najwyższej potędze x jest dodatni.\\ W związku z tym wykres wielomianu zaczyna się od lewej strony poniżej osi OX. A więc $$x \in [4,6] \cup [20,\infty).$$
\rozwStop
\odpStart
$x \in [4,6] \cup [20,\infty)$
\odpStop
\testStart
A.$x \in [4,6] \cup [20,\infty)$\\
B.$x \in (4,6) \cup [20,\infty)$\\
C.$x \in (4,6] \cup [20,\infty)$\\
D.$x \in [4,6) \cup [20,\infty)$\\
E.$x \in [4,6] \cup (20,\infty)$\\
F.$x \in (4,6) \cup (20,\infty)$\\
G.$x \in [4,6) \cup (20,\infty)$\\
H.$x \in (4,6] \cup (20,\infty)$
\testStop
\kluczStart
A
\kluczStop



\zadStart{Zadanie z Wikieł Z 1.62 a) moja wersja nr 490}

Rozwiązać nierówności $(x-4)(x-7)(x-8)\ge0$.
\zadStop
\rozwStart{Patryk Wirkus}{}
Miejsca zerowe naszego wielomianu to: $4, 7, 8$.\\
Wielomian jest stopnia nieparzystego, ponadto znak współczynnika przy\linebreak najwyższej potędze x jest dodatni.\\ W związku z tym wykres wielomianu zaczyna się od lewej strony poniżej osi OX. A więc $$x \in [4,7] \cup [8,\infty).$$
\rozwStop
\odpStart
$x \in [4,7] \cup [8,\infty)$
\odpStop
\testStart
A.$x \in [4,7] \cup [8,\infty)$\\
B.$x \in (4,7) \cup [8,\infty)$\\
C.$x \in (4,7] \cup [8,\infty)$\\
D.$x \in [4,7) \cup [8,\infty)$\\
E.$x \in [4,7] \cup (8,\infty)$\\
F.$x \in (4,7) \cup (8,\infty)$\\
G.$x \in [4,7) \cup (8,\infty)$\\
H.$x \in (4,7] \cup (8,\infty)$
\testStop
\kluczStart
A
\kluczStop



\zadStart{Zadanie z Wikieł Z 1.62 a) moja wersja nr 491}

Rozwiązać nierówności $(x-4)(x-7)(x-9)\ge0$.
\zadStop
\rozwStart{Patryk Wirkus}{}
Miejsca zerowe naszego wielomianu to: $4, 7, 9$.\\
Wielomian jest stopnia nieparzystego, ponadto znak współczynnika przy\linebreak najwyższej potędze x jest dodatni.\\ W związku z tym wykres wielomianu zaczyna się od lewej strony poniżej osi OX. A więc $$x \in [4,7] \cup [9,\infty).$$
\rozwStop
\odpStart
$x \in [4,7] \cup [9,\infty)$
\odpStop
\testStart
A.$x \in [4,7] \cup [9,\infty)$\\
B.$x \in (4,7) \cup [9,\infty)$\\
C.$x \in (4,7] \cup [9,\infty)$\\
D.$x \in [4,7) \cup [9,\infty)$\\
E.$x \in [4,7] \cup (9,\infty)$\\
F.$x \in (4,7) \cup (9,\infty)$\\
G.$x \in [4,7) \cup (9,\infty)$\\
H.$x \in (4,7] \cup (9,\infty)$
\testStop
\kluczStart
A
\kluczStop



\zadStart{Zadanie z Wikieł Z 1.62 a) moja wersja nr 492}

Rozwiązać nierówności $(x-4)(x-7)(x-10)\ge0$.
\zadStop
\rozwStart{Patryk Wirkus}{}
Miejsca zerowe naszego wielomianu to: $4, 7, 10$.\\
Wielomian jest stopnia nieparzystego, ponadto znak współczynnika przy\linebreak najwyższej potędze x jest dodatni.\\ W związku z tym wykres wielomianu zaczyna się od lewej strony poniżej osi OX. A więc $$x \in [4,7] \cup [10,\infty).$$
\rozwStop
\odpStart
$x \in [4,7] \cup [10,\infty)$
\odpStop
\testStart
A.$x \in [4,7] \cup [10,\infty)$\\
B.$x \in (4,7) \cup [10,\infty)$\\
C.$x \in (4,7] \cup [10,\infty)$\\
D.$x \in [4,7) \cup [10,\infty)$\\
E.$x \in [4,7] \cup (10,\infty)$\\
F.$x \in (4,7) \cup (10,\infty)$\\
G.$x \in [4,7) \cup (10,\infty)$\\
H.$x \in (4,7] \cup (10,\infty)$
\testStop
\kluczStart
A
\kluczStop



\zadStart{Zadanie z Wikieł Z 1.62 a) moja wersja nr 493}

Rozwiązać nierówności $(x-4)(x-7)(x-11)\ge0$.
\zadStop
\rozwStart{Patryk Wirkus}{}
Miejsca zerowe naszego wielomianu to: $4, 7, 11$.\\
Wielomian jest stopnia nieparzystego, ponadto znak współczynnika przy\linebreak najwyższej potędze x jest dodatni.\\ W związku z tym wykres wielomianu zaczyna się od lewej strony poniżej osi OX. A więc $$x \in [4,7] \cup [11,\infty).$$
\rozwStop
\odpStart
$x \in [4,7] \cup [11,\infty)$
\odpStop
\testStart
A.$x \in [4,7] \cup [11,\infty)$\\
B.$x \in (4,7) \cup [11,\infty)$\\
C.$x \in (4,7] \cup [11,\infty)$\\
D.$x \in [4,7) \cup [11,\infty)$\\
E.$x \in [4,7] \cup (11,\infty)$\\
F.$x \in (4,7) \cup (11,\infty)$\\
G.$x \in [4,7) \cup (11,\infty)$\\
H.$x \in (4,7] \cup (11,\infty)$
\testStop
\kluczStart
A
\kluczStop



\zadStart{Zadanie z Wikieł Z 1.62 a) moja wersja nr 494}

Rozwiązać nierówności $(x-4)(x-7)(x-12)\ge0$.
\zadStop
\rozwStart{Patryk Wirkus}{}
Miejsca zerowe naszego wielomianu to: $4, 7, 12$.\\
Wielomian jest stopnia nieparzystego, ponadto znak współczynnika przy\linebreak najwyższej potędze x jest dodatni.\\ W związku z tym wykres wielomianu zaczyna się od lewej strony poniżej osi OX. A więc $$x \in [4,7] \cup [12,\infty).$$
\rozwStop
\odpStart
$x \in [4,7] \cup [12,\infty)$
\odpStop
\testStart
A.$x \in [4,7] \cup [12,\infty)$\\
B.$x \in (4,7) \cup [12,\infty)$\\
C.$x \in (4,7] \cup [12,\infty)$\\
D.$x \in [4,7) \cup [12,\infty)$\\
E.$x \in [4,7] \cup (12,\infty)$\\
F.$x \in (4,7) \cup (12,\infty)$\\
G.$x \in [4,7) \cup (12,\infty)$\\
H.$x \in (4,7] \cup (12,\infty)$
\testStop
\kluczStart
A
\kluczStop



\zadStart{Zadanie z Wikieł Z 1.62 a) moja wersja nr 495}

Rozwiązać nierówności $(x-4)(x-7)(x-13)\ge0$.
\zadStop
\rozwStart{Patryk Wirkus}{}
Miejsca zerowe naszego wielomianu to: $4, 7, 13$.\\
Wielomian jest stopnia nieparzystego, ponadto znak współczynnika przy\linebreak najwyższej potędze x jest dodatni.\\ W związku z tym wykres wielomianu zaczyna się od lewej strony poniżej osi OX. A więc $$x \in [4,7] \cup [13,\infty).$$
\rozwStop
\odpStart
$x \in [4,7] \cup [13,\infty)$
\odpStop
\testStart
A.$x \in [4,7] \cup [13,\infty)$\\
B.$x \in (4,7) \cup [13,\infty)$\\
C.$x \in (4,7] \cup [13,\infty)$\\
D.$x \in [4,7) \cup [13,\infty)$\\
E.$x \in [4,7] \cup (13,\infty)$\\
F.$x \in (4,7) \cup (13,\infty)$\\
G.$x \in [4,7) \cup (13,\infty)$\\
H.$x \in (4,7] \cup (13,\infty)$
\testStop
\kluczStart
A
\kluczStop



\zadStart{Zadanie z Wikieł Z 1.62 a) moja wersja nr 496}

Rozwiązać nierówności $(x-4)(x-7)(x-14)\ge0$.
\zadStop
\rozwStart{Patryk Wirkus}{}
Miejsca zerowe naszego wielomianu to: $4, 7, 14$.\\
Wielomian jest stopnia nieparzystego, ponadto znak współczynnika przy\linebreak najwyższej potędze x jest dodatni.\\ W związku z tym wykres wielomianu zaczyna się od lewej strony poniżej osi OX. A więc $$x \in [4,7] \cup [14,\infty).$$
\rozwStop
\odpStart
$x \in [4,7] \cup [14,\infty)$
\odpStop
\testStart
A.$x \in [4,7] \cup [14,\infty)$\\
B.$x \in (4,7) \cup [14,\infty)$\\
C.$x \in (4,7] \cup [14,\infty)$\\
D.$x \in [4,7) \cup [14,\infty)$\\
E.$x \in [4,7] \cup (14,\infty)$\\
F.$x \in (4,7) \cup (14,\infty)$\\
G.$x \in [4,7) \cup (14,\infty)$\\
H.$x \in (4,7] \cup (14,\infty)$
\testStop
\kluczStart
A
\kluczStop



\zadStart{Zadanie z Wikieł Z 1.62 a) moja wersja nr 497}

Rozwiązać nierówności $(x-4)(x-7)(x-15)\ge0$.
\zadStop
\rozwStart{Patryk Wirkus}{}
Miejsca zerowe naszego wielomianu to: $4, 7, 15$.\\
Wielomian jest stopnia nieparzystego, ponadto znak współczynnika przy\linebreak najwyższej potędze x jest dodatni.\\ W związku z tym wykres wielomianu zaczyna się od lewej strony poniżej osi OX. A więc $$x \in [4,7] \cup [15,\infty).$$
\rozwStop
\odpStart
$x \in [4,7] \cup [15,\infty)$
\odpStop
\testStart
A.$x \in [4,7] \cup [15,\infty)$\\
B.$x \in (4,7) \cup [15,\infty)$\\
C.$x \in (4,7] \cup [15,\infty)$\\
D.$x \in [4,7) \cup [15,\infty)$\\
E.$x \in [4,7] \cup (15,\infty)$\\
F.$x \in (4,7) \cup (15,\infty)$\\
G.$x \in [4,7) \cup (15,\infty)$\\
H.$x \in (4,7] \cup (15,\infty)$
\testStop
\kluczStart
A
\kluczStop



\zadStart{Zadanie z Wikieł Z 1.62 a) moja wersja nr 498}

Rozwiązać nierówności $(x-4)(x-7)(x-16)\ge0$.
\zadStop
\rozwStart{Patryk Wirkus}{}
Miejsca zerowe naszego wielomianu to: $4, 7, 16$.\\
Wielomian jest stopnia nieparzystego, ponadto znak współczynnika przy\linebreak najwyższej potędze x jest dodatni.\\ W związku z tym wykres wielomianu zaczyna się od lewej strony poniżej osi OX. A więc $$x \in [4,7] \cup [16,\infty).$$
\rozwStop
\odpStart
$x \in [4,7] \cup [16,\infty)$
\odpStop
\testStart
A.$x \in [4,7] \cup [16,\infty)$\\
B.$x \in (4,7) \cup [16,\infty)$\\
C.$x \in (4,7] \cup [16,\infty)$\\
D.$x \in [4,7) \cup [16,\infty)$\\
E.$x \in [4,7] \cup (16,\infty)$\\
F.$x \in (4,7) \cup (16,\infty)$\\
G.$x \in [4,7) \cup (16,\infty)$\\
H.$x \in (4,7] \cup (16,\infty)$
\testStop
\kluczStart
A
\kluczStop



\zadStart{Zadanie z Wikieł Z 1.62 a) moja wersja nr 499}

Rozwiązać nierówności $(x-4)(x-7)(x-17)\ge0$.
\zadStop
\rozwStart{Patryk Wirkus}{}
Miejsca zerowe naszego wielomianu to: $4, 7, 17$.\\
Wielomian jest stopnia nieparzystego, ponadto znak współczynnika przy\linebreak najwyższej potędze x jest dodatni.\\ W związku z tym wykres wielomianu zaczyna się od lewej strony poniżej osi OX. A więc $$x \in [4,7] \cup [17,\infty).$$
\rozwStop
\odpStart
$x \in [4,7] \cup [17,\infty)$
\odpStop
\testStart
A.$x \in [4,7] \cup [17,\infty)$\\
B.$x \in (4,7) \cup [17,\infty)$\\
C.$x \in (4,7] \cup [17,\infty)$\\
D.$x \in [4,7) \cup [17,\infty)$\\
E.$x \in [4,7] \cup (17,\infty)$\\
F.$x \in (4,7) \cup (17,\infty)$\\
G.$x \in [4,7) \cup (17,\infty)$\\
H.$x \in (4,7] \cup (17,\infty)$
\testStop
\kluczStart
A
\kluczStop



\zadStart{Zadanie z Wikieł Z 1.62 a) moja wersja nr 500}

Rozwiązać nierówności $(x-4)(x-7)(x-18)\ge0$.
\zadStop
\rozwStart{Patryk Wirkus}{}
Miejsca zerowe naszego wielomianu to: $4, 7, 18$.\\
Wielomian jest stopnia nieparzystego, ponadto znak współczynnika przy\linebreak najwyższej potędze x jest dodatni.\\ W związku z tym wykres wielomianu zaczyna się od lewej strony poniżej osi OX. A więc $$x \in [4,7] \cup [18,\infty).$$
\rozwStop
\odpStart
$x \in [4,7] \cup [18,\infty)$
\odpStop
\testStart
A.$x \in [4,7] \cup [18,\infty)$\\
B.$x \in (4,7) \cup [18,\infty)$\\
C.$x \in (4,7] \cup [18,\infty)$\\
D.$x \in [4,7) \cup [18,\infty)$\\
E.$x \in [4,7] \cup (18,\infty)$\\
F.$x \in (4,7) \cup (18,\infty)$\\
G.$x \in [4,7) \cup (18,\infty)$\\
H.$x \in (4,7] \cup (18,\infty)$
\testStop
\kluczStart
A
\kluczStop



\zadStart{Zadanie z Wikieł Z 1.62 a) moja wersja nr 501}

Rozwiązać nierówności $(x-4)(x-7)(x-19)\ge0$.
\zadStop
\rozwStart{Patryk Wirkus}{}
Miejsca zerowe naszego wielomianu to: $4, 7, 19$.\\
Wielomian jest stopnia nieparzystego, ponadto znak współczynnika przy\linebreak najwyższej potędze x jest dodatni.\\ W związku z tym wykres wielomianu zaczyna się od lewej strony poniżej osi OX. A więc $$x \in [4,7] \cup [19,\infty).$$
\rozwStop
\odpStart
$x \in [4,7] \cup [19,\infty)$
\odpStop
\testStart
A.$x \in [4,7] \cup [19,\infty)$\\
B.$x \in (4,7) \cup [19,\infty)$\\
C.$x \in (4,7] \cup [19,\infty)$\\
D.$x \in [4,7) \cup [19,\infty)$\\
E.$x \in [4,7] \cup (19,\infty)$\\
F.$x \in (4,7) \cup (19,\infty)$\\
G.$x \in [4,7) \cup (19,\infty)$\\
H.$x \in (4,7] \cup (19,\infty)$
\testStop
\kluczStart
A
\kluczStop



\zadStart{Zadanie z Wikieł Z 1.62 a) moja wersja nr 502}

Rozwiązać nierówności $(x-4)(x-7)(x-20)\ge0$.
\zadStop
\rozwStart{Patryk Wirkus}{}
Miejsca zerowe naszego wielomianu to: $4, 7, 20$.\\
Wielomian jest stopnia nieparzystego, ponadto znak współczynnika przy\linebreak najwyższej potędze x jest dodatni.\\ W związku z tym wykres wielomianu zaczyna się od lewej strony poniżej osi OX. A więc $$x \in [4,7] \cup [20,\infty).$$
\rozwStop
\odpStart
$x \in [4,7] \cup [20,\infty)$
\odpStop
\testStart
A.$x \in [4,7] \cup [20,\infty)$\\
B.$x \in (4,7) \cup [20,\infty)$\\
C.$x \in (4,7] \cup [20,\infty)$\\
D.$x \in [4,7) \cup [20,\infty)$\\
E.$x \in [4,7] \cup (20,\infty)$\\
F.$x \in (4,7) \cup (20,\infty)$\\
G.$x \in [4,7) \cup (20,\infty)$\\
H.$x \in (4,7] \cup (20,\infty)$
\testStop
\kluczStart
A
\kluczStop



\zadStart{Zadanie z Wikieł Z 1.62 a) moja wersja nr 503}

Rozwiązać nierówności $(x-4)(x-8)(x-9)\ge0$.
\zadStop
\rozwStart{Patryk Wirkus}{}
Miejsca zerowe naszego wielomianu to: $4, 8, 9$.\\
Wielomian jest stopnia nieparzystego, ponadto znak współczynnika przy\linebreak najwyższej potędze x jest dodatni.\\ W związku z tym wykres wielomianu zaczyna się od lewej strony poniżej osi OX. A więc $$x \in [4,8] \cup [9,\infty).$$
\rozwStop
\odpStart
$x \in [4,8] \cup [9,\infty)$
\odpStop
\testStart
A.$x \in [4,8] \cup [9,\infty)$\\
B.$x \in (4,8) \cup [9,\infty)$\\
C.$x \in (4,8] \cup [9,\infty)$\\
D.$x \in [4,8) \cup [9,\infty)$\\
E.$x \in [4,8] \cup (9,\infty)$\\
F.$x \in (4,8) \cup (9,\infty)$\\
G.$x \in [4,8) \cup (9,\infty)$\\
H.$x \in (4,8] \cup (9,\infty)$
\testStop
\kluczStart
A
\kluczStop



\zadStart{Zadanie z Wikieł Z 1.62 a) moja wersja nr 504}

Rozwiązać nierówności $(x-4)(x-8)(x-10)\ge0$.
\zadStop
\rozwStart{Patryk Wirkus}{}
Miejsca zerowe naszego wielomianu to: $4, 8, 10$.\\
Wielomian jest stopnia nieparzystego, ponadto znak współczynnika przy\linebreak najwyższej potędze x jest dodatni.\\ W związku z tym wykres wielomianu zaczyna się od lewej strony poniżej osi OX. A więc $$x \in [4,8] \cup [10,\infty).$$
\rozwStop
\odpStart
$x \in [4,8] \cup [10,\infty)$
\odpStop
\testStart
A.$x \in [4,8] \cup [10,\infty)$\\
B.$x \in (4,8) \cup [10,\infty)$\\
C.$x \in (4,8] \cup [10,\infty)$\\
D.$x \in [4,8) \cup [10,\infty)$\\
E.$x \in [4,8] \cup (10,\infty)$\\
F.$x \in (4,8) \cup (10,\infty)$\\
G.$x \in [4,8) \cup (10,\infty)$\\
H.$x \in (4,8] \cup (10,\infty)$
\testStop
\kluczStart
A
\kluczStop



\zadStart{Zadanie z Wikieł Z 1.62 a) moja wersja nr 505}

Rozwiązać nierówności $(x-4)(x-8)(x-11)\ge0$.
\zadStop
\rozwStart{Patryk Wirkus}{}
Miejsca zerowe naszego wielomianu to: $4, 8, 11$.\\
Wielomian jest stopnia nieparzystego, ponadto znak współczynnika przy\linebreak najwyższej potędze x jest dodatni.\\ W związku z tym wykres wielomianu zaczyna się od lewej strony poniżej osi OX. A więc $$x \in [4,8] \cup [11,\infty).$$
\rozwStop
\odpStart
$x \in [4,8] \cup [11,\infty)$
\odpStop
\testStart
A.$x \in [4,8] \cup [11,\infty)$\\
B.$x \in (4,8) \cup [11,\infty)$\\
C.$x \in (4,8] \cup [11,\infty)$\\
D.$x \in [4,8) \cup [11,\infty)$\\
E.$x \in [4,8] \cup (11,\infty)$\\
F.$x \in (4,8) \cup (11,\infty)$\\
G.$x \in [4,8) \cup (11,\infty)$\\
H.$x \in (4,8] \cup (11,\infty)$
\testStop
\kluczStart
A
\kluczStop



\zadStart{Zadanie z Wikieł Z 1.62 a) moja wersja nr 506}

Rozwiązać nierówności $(x-4)(x-8)(x-12)\ge0$.
\zadStop
\rozwStart{Patryk Wirkus}{}
Miejsca zerowe naszego wielomianu to: $4, 8, 12$.\\
Wielomian jest stopnia nieparzystego, ponadto znak współczynnika przy\linebreak najwyższej potędze x jest dodatni.\\ W związku z tym wykres wielomianu zaczyna się od lewej strony poniżej osi OX. A więc $$x \in [4,8] \cup [12,\infty).$$
\rozwStop
\odpStart
$x \in [4,8] \cup [12,\infty)$
\odpStop
\testStart
A.$x \in [4,8] \cup [12,\infty)$\\
B.$x \in (4,8) \cup [12,\infty)$\\
C.$x \in (4,8] \cup [12,\infty)$\\
D.$x \in [4,8) \cup [12,\infty)$\\
E.$x \in [4,8] \cup (12,\infty)$\\
F.$x \in (4,8) \cup (12,\infty)$\\
G.$x \in [4,8) \cup (12,\infty)$\\
H.$x \in (4,8] \cup (12,\infty)$
\testStop
\kluczStart
A
\kluczStop



\zadStart{Zadanie z Wikieł Z 1.62 a) moja wersja nr 507}

Rozwiązać nierówności $(x-4)(x-8)(x-13)\ge0$.
\zadStop
\rozwStart{Patryk Wirkus}{}
Miejsca zerowe naszego wielomianu to: $4, 8, 13$.\\
Wielomian jest stopnia nieparzystego, ponadto znak współczynnika przy\linebreak najwyższej potędze x jest dodatni.\\ W związku z tym wykres wielomianu zaczyna się od lewej strony poniżej osi OX. A więc $$x \in [4,8] \cup [13,\infty).$$
\rozwStop
\odpStart
$x \in [4,8] \cup [13,\infty)$
\odpStop
\testStart
A.$x \in [4,8] \cup [13,\infty)$\\
B.$x \in (4,8) \cup [13,\infty)$\\
C.$x \in (4,8] \cup [13,\infty)$\\
D.$x \in [4,8) \cup [13,\infty)$\\
E.$x \in [4,8] \cup (13,\infty)$\\
F.$x \in (4,8) \cup (13,\infty)$\\
G.$x \in [4,8) \cup (13,\infty)$\\
H.$x \in (4,8] \cup (13,\infty)$
\testStop
\kluczStart
A
\kluczStop



\zadStart{Zadanie z Wikieł Z 1.62 a) moja wersja nr 508}

Rozwiązać nierówności $(x-4)(x-8)(x-14)\ge0$.
\zadStop
\rozwStart{Patryk Wirkus}{}
Miejsca zerowe naszego wielomianu to: $4, 8, 14$.\\
Wielomian jest stopnia nieparzystego, ponadto znak współczynnika przy\linebreak najwyższej potędze x jest dodatni.\\ W związku z tym wykres wielomianu zaczyna się od lewej strony poniżej osi OX. A więc $$x \in [4,8] \cup [14,\infty).$$
\rozwStop
\odpStart
$x \in [4,8] \cup [14,\infty)$
\odpStop
\testStart
A.$x \in [4,8] \cup [14,\infty)$\\
B.$x \in (4,8) \cup [14,\infty)$\\
C.$x \in (4,8] \cup [14,\infty)$\\
D.$x \in [4,8) \cup [14,\infty)$\\
E.$x \in [4,8] \cup (14,\infty)$\\
F.$x \in (4,8) \cup (14,\infty)$\\
G.$x \in [4,8) \cup (14,\infty)$\\
H.$x \in (4,8] \cup (14,\infty)$
\testStop
\kluczStart
A
\kluczStop



\zadStart{Zadanie z Wikieł Z 1.62 a) moja wersja nr 509}

Rozwiązać nierówności $(x-4)(x-8)(x-15)\ge0$.
\zadStop
\rozwStart{Patryk Wirkus}{}
Miejsca zerowe naszego wielomianu to: $4, 8, 15$.\\
Wielomian jest stopnia nieparzystego, ponadto znak współczynnika przy\linebreak najwyższej potędze x jest dodatni.\\ W związku z tym wykres wielomianu zaczyna się od lewej strony poniżej osi OX. A więc $$x \in [4,8] \cup [15,\infty).$$
\rozwStop
\odpStart
$x \in [4,8] \cup [15,\infty)$
\odpStop
\testStart
A.$x \in [4,8] \cup [15,\infty)$\\
B.$x \in (4,8) \cup [15,\infty)$\\
C.$x \in (4,8] \cup [15,\infty)$\\
D.$x \in [4,8) \cup [15,\infty)$\\
E.$x \in [4,8] \cup (15,\infty)$\\
F.$x \in (4,8) \cup (15,\infty)$\\
G.$x \in [4,8) \cup (15,\infty)$\\
H.$x \in (4,8] \cup (15,\infty)$
\testStop
\kluczStart
A
\kluczStop



\zadStart{Zadanie z Wikieł Z 1.62 a) moja wersja nr 510}

Rozwiązać nierówności $(x-4)(x-8)(x-16)\ge0$.
\zadStop
\rozwStart{Patryk Wirkus}{}
Miejsca zerowe naszego wielomianu to: $4, 8, 16$.\\
Wielomian jest stopnia nieparzystego, ponadto znak współczynnika przy\linebreak najwyższej potędze x jest dodatni.\\ W związku z tym wykres wielomianu zaczyna się od lewej strony poniżej osi OX. A więc $$x \in [4,8] \cup [16,\infty).$$
\rozwStop
\odpStart
$x \in [4,8] \cup [16,\infty)$
\odpStop
\testStart
A.$x \in [4,8] \cup [16,\infty)$\\
B.$x \in (4,8) \cup [16,\infty)$\\
C.$x \in (4,8] \cup [16,\infty)$\\
D.$x \in [4,8) \cup [16,\infty)$\\
E.$x \in [4,8] \cup (16,\infty)$\\
F.$x \in (4,8) \cup (16,\infty)$\\
G.$x \in [4,8) \cup (16,\infty)$\\
H.$x \in (4,8] \cup (16,\infty)$
\testStop
\kluczStart
A
\kluczStop



\zadStart{Zadanie z Wikieł Z 1.62 a) moja wersja nr 511}

Rozwiązać nierówności $(x-4)(x-8)(x-17)\ge0$.
\zadStop
\rozwStart{Patryk Wirkus}{}
Miejsca zerowe naszego wielomianu to: $4, 8, 17$.\\
Wielomian jest stopnia nieparzystego, ponadto znak współczynnika przy\linebreak najwyższej potędze x jest dodatni.\\ W związku z tym wykres wielomianu zaczyna się od lewej strony poniżej osi OX. A więc $$x \in [4,8] \cup [17,\infty).$$
\rozwStop
\odpStart
$x \in [4,8] \cup [17,\infty)$
\odpStop
\testStart
A.$x \in [4,8] \cup [17,\infty)$\\
B.$x \in (4,8) \cup [17,\infty)$\\
C.$x \in (4,8] \cup [17,\infty)$\\
D.$x \in [4,8) \cup [17,\infty)$\\
E.$x \in [4,8] \cup (17,\infty)$\\
F.$x \in (4,8) \cup (17,\infty)$\\
G.$x \in [4,8) \cup (17,\infty)$\\
H.$x \in (4,8] \cup (17,\infty)$
\testStop
\kluczStart
A
\kluczStop



\zadStart{Zadanie z Wikieł Z 1.62 a) moja wersja nr 512}

Rozwiązać nierówności $(x-4)(x-8)(x-18)\ge0$.
\zadStop
\rozwStart{Patryk Wirkus}{}
Miejsca zerowe naszego wielomianu to: $4, 8, 18$.\\
Wielomian jest stopnia nieparzystego, ponadto znak współczynnika przy\linebreak najwyższej potędze x jest dodatni.\\ W związku z tym wykres wielomianu zaczyna się od lewej strony poniżej osi OX. A więc $$x \in [4,8] \cup [18,\infty).$$
\rozwStop
\odpStart
$x \in [4,8] \cup [18,\infty)$
\odpStop
\testStart
A.$x \in [4,8] \cup [18,\infty)$\\
B.$x \in (4,8) \cup [18,\infty)$\\
C.$x \in (4,8] \cup [18,\infty)$\\
D.$x \in [4,8) \cup [18,\infty)$\\
E.$x \in [4,8] \cup (18,\infty)$\\
F.$x \in (4,8) \cup (18,\infty)$\\
G.$x \in [4,8) \cup (18,\infty)$\\
H.$x \in (4,8] \cup (18,\infty)$
\testStop
\kluczStart
A
\kluczStop



\zadStart{Zadanie z Wikieł Z 1.62 a) moja wersja nr 513}

Rozwiązać nierówności $(x-4)(x-8)(x-19)\ge0$.
\zadStop
\rozwStart{Patryk Wirkus}{}
Miejsca zerowe naszego wielomianu to: $4, 8, 19$.\\
Wielomian jest stopnia nieparzystego, ponadto znak współczynnika przy\linebreak najwyższej potędze x jest dodatni.\\ W związku z tym wykres wielomianu zaczyna się od lewej strony poniżej osi OX. A więc $$x \in [4,8] \cup [19,\infty).$$
\rozwStop
\odpStart
$x \in [4,8] \cup [19,\infty)$
\odpStop
\testStart
A.$x \in [4,8] \cup [19,\infty)$\\
B.$x \in (4,8) \cup [19,\infty)$\\
C.$x \in (4,8] \cup [19,\infty)$\\
D.$x \in [4,8) \cup [19,\infty)$\\
E.$x \in [4,8] \cup (19,\infty)$\\
F.$x \in (4,8) \cup (19,\infty)$\\
G.$x \in [4,8) \cup (19,\infty)$\\
H.$x \in (4,8] \cup (19,\infty)$
\testStop
\kluczStart
A
\kluczStop



\zadStart{Zadanie z Wikieł Z 1.62 a) moja wersja nr 514}

Rozwiązać nierówności $(x-4)(x-8)(x-20)\ge0$.
\zadStop
\rozwStart{Patryk Wirkus}{}
Miejsca zerowe naszego wielomianu to: $4, 8, 20$.\\
Wielomian jest stopnia nieparzystego, ponadto znak współczynnika przy\linebreak najwyższej potędze x jest dodatni.\\ W związku z tym wykres wielomianu zaczyna się od lewej strony poniżej osi OX. A więc $$x \in [4,8] \cup [20,\infty).$$
\rozwStop
\odpStart
$x \in [4,8] \cup [20,\infty)$
\odpStop
\testStart
A.$x \in [4,8] \cup [20,\infty)$\\
B.$x \in (4,8) \cup [20,\infty)$\\
C.$x \in (4,8] \cup [20,\infty)$\\
D.$x \in [4,8) \cup [20,\infty)$\\
E.$x \in [4,8] \cup (20,\infty)$\\
F.$x \in (4,8) \cup (20,\infty)$\\
G.$x \in [4,8) \cup (20,\infty)$\\
H.$x \in (4,8] \cup (20,\infty)$
\testStop
\kluczStart
A
\kluczStop



\zadStart{Zadanie z Wikieł Z 1.62 a) moja wersja nr 515}

Rozwiązać nierówności $(x-4)(x-9)(x-10)\ge0$.
\zadStop
\rozwStart{Patryk Wirkus}{}
Miejsca zerowe naszego wielomianu to: $4, 9, 10$.\\
Wielomian jest stopnia nieparzystego, ponadto znak współczynnika przy\linebreak najwyższej potędze x jest dodatni.\\ W związku z tym wykres wielomianu zaczyna się od lewej strony poniżej osi OX. A więc $$x \in [4,9] \cup [10,\infty).$$
\rozwStop
\odpStart
$x \in [4,9] \cup [10,\infty)$
\odpStop
\testStart
A.$x \in [4,9] \cup [10,\infty)$\\
B.$x \in (4,9) \cup [10,\infty)$\\
C.$x \in (4,9] \cup [10,\infty)$\\
D.$x \in [4,9) \cup [10,\infty)$\\
E.$x \in [4,9] \cup (10,\infty)$\\
F.$x \in (4,9) \cup (10,\infty)$\\
G.$x \in [4,9) \cup (10,\infty)$\\
H.$x \in (4,9] \cup (10,\infty)$
\testStop
\kluczStart
A
\kluczStop



\zadStart{Zadanie z Wikieł Z 1.62 a) moja wersja nr 516}

Rozwiązać nierówności $(x-4)(x-9)(x-11)\ge0$.
\zadStop
\rozwStart{Patryk Wirkus}{}
Miejsca zerowe naszego wielomianu to: $4, 9, 11$.\\
Wielomian jest stopnia nieparzystego, ponadto znak współczynnika przy\linebreak najwyższej potędze x jest dodatni.\\ W związku z tym wykres wielomianu zaczyna się od lewej strony poniżej osi OX. A więc $$x \in [4,9] \cup [11,\infty).$$
\rozwStop
\odpStart
$x \in [4,9] \cup [11,\infty)$
\odpStop
\testStart
A.$x \in [4,9] \cup [11,\infty)$\\
B.$x \in (4,9) \cup [11,\infty)$\\
C.$x \in (4,9] \cup [11,\infty)$\\
D.$x \in [4,9) \cup [11,\infty)$\\
E.$x \in [4,9] \cup (11,\infty)$\\
F.$x \in (4,9) \cup (11,\infty)$\\
G.$x \in [4,9) \cup (11,\infty)$\\
H.$x \in (4,9] \cup (11,\infty)$
\testStop
\kluczStart
A
\kluczStop



\zadStart{Zadanie z Wikieł Z 1.62 a) moja wersja nr 517}

Rozwiązać nierówności $(x-4)(x-9)(x-12)\ge0$.
\zadStop
\rozwStart{Patryk Wirkus}{}
Miejsca zerowe naszego wielomianu to: $4, 9, 12$.\\
Wielomian jest stopnia nieparzystego, ponadto znak współczynnika przy\linebreak najwyższej potędze x jest dodatni.\\ W związku z tym wykres wielomianu zaczyna się od lewej strony poniżej osi OX. A więc $$x \in [4,9] \cup [12,\infty).$$
\rozwStop
\odpStart
$x \in [4,9] \cup [12,\infty)$
\odpStop
\testStart
A.$x \in [4,9] \cup [12,\infty)$\\
B.$x \in (4,9) \cup [12,\infty)$\\
C.$x \in (4,9] \cup [12,\infty)$\\
D.$x \in [4,9) \cup [12,\infty)$\\
E.$x \in [4,9] \cup (12,\infty)$\\
F.$x \in (4,9) \cup (12,\infty)$\\
G.$x \in [4,9) \cup (12,\infty)$\\
H.$x \in (4,9] \cup (12,\infty)$
\testStop
\kluczStart
A
\kluczStop



\zadStart{Zadanie z Wikieł Z 1.62 a) moja wersja nr 518}

Rozwiązać nierówności $(x-4)(x-9)(x-13)\ge0$.
\zadStop
\rozwStart{Patryk Wirkus}{}
Miejsca zerowe naszego wielomianu to: $4, 9, 13$.\\
Wielomian jest stopnia nieparzystego, ponadto znak współczynnika przy\linebreak najwyższej potędze x jest dodatni.\\ W związku z tym wykres wielomianu zaczyna się od lewej strony poniżej osi OX. A więc $$x \in [4,9] \cup [13,\infty).$$
\rozwStop
\odpStart
$x \in [4,9] \cup [13,\infty)$
\odpStop
\testStart
A.$x \in [4,9] \cup [13,\infty)$\\
B.$x \in (4,9) \cup [13,\infty)$\\
C.$x \in (4,9] \cup [13,\infty)$\\
D.$x \in [4,9) \cup [13,\infty)$\\
E.$x \in [4,9] \cup (13,\infty)$\\
F.$x \in (4,9) \cup (13,\infty)$\\
G.$x \in [4,9) \cup (13,\infty)$\\
H.$x \in (4,9] \cup (13,\infty)$
\testStop
\kluczStart
A
\kluczStop



\zadStart{Zadanie z Wikieł Z 1.62 a) moja wersja nr 519}

Rozwiązać nierówności $(x-4)(x-9)(x-14)\ge0$.
\zadStop
\rozwStart{Patryk Wirkus}{}
Miejsca zerowe naszego wielomianu to: $4, 9, 14$.\\
Wielomian jest stopnia nieparzystego, ponadto znak współczynnika przy\linebreak najwyższej potędze x jest dodatni.\\ W związku z tym wykres wielomianu zaczyna się od lewej strony poniżej osi OX. A więc $$x \in [4,9] \cup [14,\infty).$$
\rozwStop
\odpStart
$x \in [4,9] \cup [14,\infty)$
\odpStop
\testStart
A.$x \in [4,9] \cup [14,\infty)$\\
B.$x \in (4,9) \cup [14,\infty)$\\
C.$x \in (4,9] \cup [14,\infty)$\\
D.$x \in [4,9) \cup [14,\infty)$\\
E.$x \in [4,9] \cup (14,\infty)$\\
F.$x \in (4,9) \cup (14,\infty)$\\
G.$x \in [4,9) \cup (14,\infty)$\\
H.$x \in (4,9] \cup (14,\infty)$
\testStop
\kluczStart
A
\kluczStop



\zadStart{Zadanie z Wikieł Z 1.62 a) moja wersja nr 520}

Rozwiązać nierówności $(x-4)(x-9)(x-15)\ge0$.
\zadStop
\rozwStart{Patryk Wirkus}{}
Miejsca zerowe naszego wielomianu to: $4, 9, 15$.\\
Wielomian jest stopnia nieparzystego, ponadto znak współczynnika przy\linebreak najwyższej potędze x jest dodatni.\\ W związku z tym wykres wielomianu zaczyna się od lewej strony poniżej osi OX. A więc $$x \in [4,9] \cup [15,\infty).$$
\rozwStop
\odpStart
$x \in [4,9] \cup [15,\infty)$
\odpStop
\testStart
A.$x \in [4,9] \cup [15,\infty)$\\
B.$x \in (4,9) \cup [15,\infty)$\\
C.$x \in (4,9] \cup [15,\infty)$\\
D.$x \in [4,9) \cup [15,\infty)$\\
E.$x \in [4,9] \cup (15,\infty)$\\
F.$x \in (4,9) \cup (15,\infty)$\\
G.$x \in [4,9) \cup (15,\infty)$\\
H.$x \in (4,9] \cup (15,\infty)$
\testStop
\kluczStart
A
\kluczStop



\zadStart{Zadanie z Wikieł Z 1.62 a) moja wersja nr 521}

Rozwiązać nierówności $(x-4)(x-9)(x-16)\ge0$.
\zadStop
\rozwStart{Patryk Wirkus}{}
Miejsca zerowe naszego wielomianu to: $4, 9, 16$.\\
Wielomian jest stopnia nieparzystego, ponadto znak współczynnika przy\linebreak najwyższej potędze x jest dodatni.\\ W związku z tym wykres wielomianu zaczyna się od lewej strony poniżej osi OX. A więc $$x \in [4,9] \cup [16,\infty).$$
\rozwStop
\odpStart
$x \in [4,9] \cup [16,\infty)$
\odpStop
\testStart
A.$x \in [4,9] \cup [16,\infty)$\\
B.$x \in (4,9) \cup [16,\infty)$\\
C.$x \in (4,9] \cup [16,\infty)$\\
D.$x \in [4,9) \cup [16,\infty)$\\
E.$x \in [4,9] \cup (16,\infty)$\\
F.$x \in (4,9) \cup (16,\infty)$\\
G.$x \in [4,9) \cup (16,\infty)$\\
H.$x \in (4,9] \cup (16,\infty)$
\testStop
\kluczStart
A
\kluczStop



\zadStart{Zadanie z Wikieł Z 1.62 a) moja wersja nr 522}

Rozwiązać nierówności $(x-4)(x-9)(x-17)\ge0$.
\zadStop
\rozwStart{Patryk Wirkus}{}
Miejsca zerowe naszego wielomianu to: $4, 9, 17$.\\
Wielomian jest stopnia nieparzystego, ponadto znak współczynnika przy\linebreak najwyższej potędze x jest dodatni.\\ W związku z tym wykres wielomianu zaczyna się od lewej strony poniżej osi OX. A więc $$x \in [4,9] \cup [17,\infty).$$
\rozwStop
\odpStart
$x \in [4,9] \cup [17,\infty)$
\odpStop
\testStart
A.$x \in [4,9] \cup [17,\infty)$\\
B.$x \in (4,9) \cup [17,\infty)$\\
C.$x \in (4,9] \cup [17,\infty)$\\
D.$x \in [4,9) \cup [17,\infty)$\\
E.$x \in [4,9] \cup (17,\infty)$\\
F.$x \in (4,9) \cup (17,\infty)$\\
G.$x \in [4,9) \cup (17,\infty)$\\
H.$x \in (4,9] \cup (17,\infty)$
\testStop
\kluczStart
A
\kluczStop



\zadStart{Zadanie z Wikieł Z 1.62 a) moja wersja nr 523}

Rozwiązać nierówności $(x-4)(x-9)(x-18)\ge0$.
\zadStop
\rozwStart{Patryk Wirkus}{}
Miejsca zerowe naszego wielomianu to: $4, 9, 18$.\\
Wielomian jest stopnia nieparzystego, ponadto znak współczynnika przy\linebreak najwyższej potędze x jest dodatni.\\ W związku z tym wykres wielomianu zaczyna się od lewej strony poniżej osi OX. A więc $$x \in [4,9] \cup [18,\infty).$$
\rozwStop
\odpStart
$x \in [4,9] \cup [18,\infty)$
\odpStop
\testStart
A.$x \in [4,9] \cup [18,\infty)$\\
B.$x \in (4,9) \cup [18,\infty)$\\
C.$x \in (4,9] \cup [18,\infty)$\\
D.$x \in [4,9) \cup [18,\infty)$\\
E.$x \in [4,9] \cup (18,\infty)$\\
F.$x \in (4,9) \cup (18,\infty)$\\
G.$x \in [4,9) \cup (18,\infty)$\\
H.$x \in (4,9] \cup (18,\infty)$
\testStop
\kluczStart
A
\kluczStop



\zadStart{Zadanie z Wikieł Z 1.62 a) moja wersja nr 524}

Rozwiązać nierówności $(x-4)(x-9)(x-19)\ge0$.
\zadStop
\rozwStart{Patryk Wirkus}{}
Miejsca zerowe naszego wielomianu to: $4, 9, 19$.\\
Wielomian jest stopnia nieparzystego, ponadto znak współczynnika przy\linebreak najwyższej potędze x jest dodatni.\\ W związku z tym wykres wielomianu zaczyna się od lewej strony poniżej osi OX. A więc $$x \in [4,9] \cup [19,\infty).$$
\rozwStop
\odpStart
$x \in [4,9] \cup [19,\infty)$
\odpStop
\testStart
A.$x \in [4,9] \cup [19,\infty)$\\
B.$x \in (4,9) \cup [19,\infty)$\\
C.$x \in (4,9] \cup [19,\infty)$\\
D.$x \in [4,9) \cup [19,\infty)$\\
E.$x \in [4,9] \cup (19,\infty)$\\
F.$x \in (4,9) \cup (19,\infty)$\\
G.$x \in [4,9) \cup (19,\infty)$\\
H.$x \in (4,9] \cup (19,\infty)$
\testStop
\kluczStart
A
\kluczStop



\zadStart{Zadanie z Wikieł Z 1.62 a) moja wersja nr 525}

Rozwiązać nierówności $(x-4)(x-9)(x-20)\ge0$.
\zadStop
\rozwStart{Patryk Wirkus}{}
Miejsca zerowe naszego wielomianu to: $4, 9, 20$.\\
Wielomian jest stopnia nieparzystego, ponadto znak współczynnika przy\linebreak najwyższej potędze x jest dodatni.\\ W związku z tym wykres wielomianu zaczyna się od lewej strony poniżej osi OX. A więc $$x \in [4,9] \cup [20,\infty).$$
\rozwStop
\odpStart
$x \in [4,9] \cup [20,\infty)$
\odpStop
\testStart
A.$x \in [4,9] \cup [20,\infty)$\\
B.$x \in (4,9) \cup [20,\infty)$\\
C.$x \in (4,9] \cup [20,\infty)$\\
D.$x \in [4,9) \cup [20,\infty)$\\
E.$x \in [4,9] \cup (20,\infty)$\\
F.$x \in (4,9) \cup (20,\infty)$\\
G.$x \in [4,9) \cup (20,\infty)$\\
H.$x \in (4,9] \cup (20,\infty)$
\testStop
\kluczStart
A
\kluczStop



\zadStart{Zadanie z Wikieł Z 1.62 a) moja wersja nr 526}

Rozwiązać nierówności $(x-4)(x-10)(x-11)\ge0$.
\zadStop
\rozwStart{Patryk Wirkus}{}
Miejsca zerowe naszego wielomianu to: $4, 10, 11$.\\
Wielomian jest stopnia nieparzystego, ponadto znak współczynnika przy\linebreak najwyższej potędze x jest dodatni.\\ W związku z tym wykres wielomianu zaczyna się od lewej strony poniżej osi OX. A więc $$x \in [4,10] \cup [11,\infty).$$
\rozwStop
\odpStart
$x \in [4,10] \cup [11,\infty)$
\odpStop
\testStart
A.$x \in [4,10] \cup [11,\infty)$\\
B.$x \in (4,10) \cup [11,\infty)$\\
C.$x \in (4,10] \cup [11,\infty)$\\
D.$x \in [4,10) \cup [11,\infty)$\\
E.$x \in [4,10] \cup (11,\infty)$\\
F.$x \in (4,10) \cup (11,\infty)$\\
G.$x \in [4,10) \cup (11,\infty)$\\
H.$x \in (4,10] \cup (11,\infty)$
\testStop
\kluczStart
A
\kluczStop



\zadStart{Zadanie z Wikieł Z 1.62 a) moja wersja nr 527}

Rozwiązać nierówności $(x-4)(x-10)(x-12)\ge0$.
\zadStop
\rozwStart{Patryk Wirkus}{}
Miejsca zerowe naszego wielomianu to: $4, 10, 12$.\\
Wielomian jest stopnia nieparzystego, ponadto znak współczynnika przy\linebreak najwyższej potędze x jest dodatni.\\ W związku z tym wykres wielomianu zaczyna się od lewej strony poniżej osi OX. A więc $$x \in [4,10] \cup [12,\infty).$$
\rozwStop
\odpStart
$x \in [4,10] \cup [12,\infty)$
\odpStop
\testStart
A.$x \in [4,10] \cup [12,\infty)$\\
B.$x \in (4,10) \cup [12,\infty)$\\
C.$x \in (4,10] \cup [12,\infty)$\\
D.$x \in [4,10) \cup [12,\infty)$\\
E.$x \in [4,10] \cup (12,\infty)$\\
F.$x \in (4,10) \cup (12,\infty)$\\
G.$x \in [4,10) \cup (12,\infty)$\\
H.$x \in (4,10] \cup (12,\infty)$
\testStop
\kluczStart
A
\kluczStop



\zadStart{Zadanie z Wikieł Z 1.62 a) moja wersja nr 528}

Rozwiązać nierówności $(x-4)(x-10)(x-13)\ge0$.
\zadStop
\rozwStart{Patryk Wirkus}{}
Miejsca zerowe naszego wielomianu to: $4, 10, 13$.\\
Wielomian jest stopnia nieparzystego, ponadto znak współczynnika przy\linebreak najwyższej potędze x jest dodatni.\\ W związku z tym wykres wielomianu zaczyna się od lewej strony poniżej osi OX. A więc $$x \in [4,10] \cup [13,\infty).$$
\rozwStop
\odpStart
$x \in [4,10] \cup [13,\infty)$
\odpStop
\testStart
A.$x \in [4,10] \cup [13,\infty)$\\
B.$x \in (4,10) \cup [13,\infty)$\\
C.$x \in (4,10] \cup [13,\infty)$\\
D.$x \in [4,10) \cup [13,\infty)$\\
E.$x \in [4,10] \cup (13,\infty)$\\
F.$x \in (4,10) \cup (13,\infty)$\\
G.$x \in [4,10) \cup (13,\infty)$\\
H.$x \in (4,10] \cup (13,\infty)$
\testStop
\kluczStart
A
\kluczStop



\zadStart{Zadanie z Wikieł Z 1.62 a) moja wersja nr 529}

Rozwiązać nierówności $(x-4)(x-10)(x-14)\ge0$.
\zadStop
\rozwStart{Patryk Wirkus}{}
Miejsca zerowe naszego wielomianu to: $4, 10, 14$.\\
Wielomian jest stopnia nieparzystego, ponadto znak współczynnika przy\linebreak najwyższej potędze x jest dodatni.\\ W związku z tym wykres wielomianu zaczyna się od lewej strony poniżej osi OX. A więc $$x \in [4,10] \cup [14,\infty).$$
\rozwStop
\odpStart
$x \in [4,10] \cup [14,\infty)$
\odpStop
\testStart
A.$x \in [4,10] \cup [14,\infty)$\\
B.$x \in (4,10) \cup [14,\infty)$\\
C.$x \in (4,10] \cup [14,\infty)$\\
D.$x \in [4,10) \cup [14,\infty)$\\
E.$x \in [4,10] \cup (14,\infty)$\\
F.$x \in (4,10) \cup (14,\infty)$\\
G.$x \in [4,10) \cup (14,\infty)$\\
H.$x \in (4,10] \cup (14,\infty)$
\testStop
\kluczStart
A
\kluczStop



\zadStart{Zadanie z Wikieł Z 1.62 a) moja wersja nr 530}

Rozwiązać nierówności $(x-4)(x-10)(x-15)\ge0$.
\zadStop
\rozwStart{Patryk Wirkus}{}
Miejsca zerowe naszego wielomianu to: $4, 10, 15$.\\
Wielomian jest stopnia nieparzystego, ponadto znak współczynnika przy\linebreak najwyższej potędze x jest dodatni.\\ W związku z tym wykres wielomianu zaczyna się od lewej strony poniżej osi OX. A więc $$x \in [4,10] \cup [15,\infty).$$
\rozwStop
\odpStart
$x \in [4,10] \cup [15,\infty)$
\odpStop
\testStart
A.$x \in [4,10] \cup [15,\infty)$\\
B.$x \in (4,10) \cup [15,\infty)$\\
C.$x \in (4,10] \cup [15,\infty)$\\
D.$x \in [4,10) \cup [15,\infty)$\\
E.$x \in [4,10] \cup (15,\infty)$\\
F.$x \in (4,10) \cup (15,\infty)$\\
G.$x \in [4,10) \cup (15,\infty)$\\
H.$x \in (4,10] \cup (15,\infty)$
\testStop
\kluczStart
A
\kluczStop



\zadStart{Zadanie z Wikieł Z 1.62 a) moja wersja nr 531}

Rozwiązać nierówności $(x-4)(x-10)(x-16)\ge0$.
\zadStop
\rozwStart{Patryk Wirkus}{}
Miejsca zerowe naszego wielomianu to: $4, 10, 16$.\\
Wielomian jest stopnia nieparzystego, ponadto znak współczynnika przy\linebreak najwyższej potędze x jest dodatni.\\ W związku z tym wykres wielomianu zaczyna się od lewej strony poniżej osi OX. A więc $$x \in [4,10] \cup [16,\infty).$$
\rozwStop
\odpStart
$x \in [4,10] \cup [16,\infty)$
\odpStop
\testStart
A.$x \in [4,10] \cup [16,\infty)$\\
B.$x \in (4,10) \cup [16,\infty)$\\
C.$x \in (4,10] \cup [16,\infty)$\\
D.$x \in [4,10) \cup [16,\infty)$\\
E.$x \in [4,10] \cup (16,\infty)$\\
F.$x \in (4,10) \cup (16,\infty)$\\
G.$x \in [4,10) \cup (16,\infty)$\\
H.$x \in (4,10] \cup (16,\infty)$
\testStop
\kluczStart
A
\kluczStop



\zadStart{Zadanie z Wikieł Z 1.62 a) moja wersja nr 532}

Rozwiązać nierówności $(x-4)(x-10)(x-17)\ge0$.
\zadStop
\rozwStart{Patryk Wirkus}{}
Miejsca zerowe naszego wielomianu to: $4, 10, 17$.\\
Wielomian jest stopnia nieparzystego, ponadto znak współczynnika przy\linebreak najwyższej potędze x jest dodatni.\\ W związku z tym wykres wielomianu zaczyna się od lewej strony poniżej osi OX. A więc $$x \in [4,10] \cup [17,\infty).$$
\rozwStop
\odpStart
$x \in [4,10] \cup [17,\infty)$
\odpStop
\testStart
A.$x \in [4,10] \cup [17,\infty)$\\
B.$x \in (4,10) \cup [17,\infty)$\\
C.$x \in (4,10] \cup [17,\infty)$\\
D.$x \in [4,10) \cup [17,\infty)$\\
E.$x \in [4,10] \cup (17,\infty)$\\
F.$x \in (4,10) \cup (17,\infty)$\\
G.$x \in [4,10) \cup (17,\infty)$\\
H.$x \in (4,10] \cup (17,\infty)$
\testStop
\kluczStart
A
\kluczStop



\zadStart{Zadanie z Wikieł Z 1.62 a) moja wersja nr 533}

Rozwiązać nierówności $(x-4)(x-10)(x-18)\ge0$.
\zadStop
\rozwStart{Patryk Wirkus}{}
Miejsca zerowe naszego wielomianu to: $4, 10, 18$.\\
Wielomian jest stopnia nieparzystego, ponadto znak współczynnika przy\linebreak najwyższej potędze x jest dodatni.\\ W związku z tym wykres wielomianu zaczyna się od lewej strony poniżej osi OX. A więc $$x \in [4,10] \cup [18,\infty).$$
\rozwStop
\odpStart
$x \in [4,10] \cup [18,\infty)$
\odpStop
\testStart
A.$x \in [4,10] \cup [18,\infty)$\\
B.$x \in (4,10) \cup [18,\infty)$\\
C.$x \in (4,10] \cup [18,\infty)$\\
D.$x \in [4,10) \cup [18,\infty)$\\
E.$x \in [4,10] \cup (18,\infty)$\\
F.$x \in (4,10) \cup (18,\infty)$\\
G.$x \in [4,10) \cup (18,\infty)$\\
H.$x \in (4,10] \cup (18,\infty)$
\testStop
\kluczStart
A
\kluczStop



\zadStart{Zadanie z Wikieł Z 1.62 a) moja wersja nr 534}

Rozwiązać nierówności $(x-4)(x-10)(x-19)\ge0$.
\zadStop
\rozwStart{Patryk Wirkus}{}
Miejsca zerowe naszego wielomianu to: $4, 10, 19$.\\
Wielomian jest stopnia nieparzystego, ponadto znak współczynnika przy\linebreak najwyższej potędze x jest dodatni.\\ W związku z tym wykres wielomianu zaczyna się od lewej strony poniżej osi OX. A więc $$x \in [4,10] \cup [19,\infty).$$
\rozwStop
\odpStart
$x \in [4,10] \cup [19,\infty)$
\odpStop
\testStart
A.$x \in [4,10] \cup [19,\infty)$\\
B.$x \in (4,10) \cup [19,\infty)$\\
C.$x \in (4,10] \cup [19,\infty)$\\
D.$x \in [4,10) \cup [19,\infty)$\\
E.$x \in [4,10] \cup (19,\infty)$\\
F.$x \in (4,10) \cup (19,\infty)$\\
G.$x \in [4,10) \cup (19,\infty)$\\
H.$x \in (4,10] \cup (19,\infty)$
\testStop
\kluczStart
A
\kluczStop



\zadStart{Zadanie z Wikieł Z 1.62 a) moja wersja nr 535}

Rozwiązać nierówności $(x-4)(x-10)(x-20)\ge0$.
\zadStop
\rozwStart{Patryk Wirkus}{}
Miejsca zerowe naszego wielomianu to: $4, 10, 20$.\\
Wielomian jest stopnia nieparzystego, ponadto znak współczynnika przy\linebreak najwyższej potędze x jest dodatni.\\ W związku z tym wykres wielomianu zaczyna się od lewej strony poniżej osi OX. A więc $$x \in [4,10] \cup [20,\infty).$$
\rozwStop
\odpStart
$x \in [4,10] \cup [20,\infty)$
\odpStop
\testStart
A.$x \in [4,10] \cup [20,\infty)$\\
B.$x \in (4,10) \cup [20,\infty)$\\
C.$x \in (4,10] \cup [20,\infty)$\\
D.$x \in [4,10) \cup [20,\infty)$\\
E.$x \in [4,10] \cup (20,\infty)$\\
F.$x \in (4,10) \cup (20,\infty)$\\
G.$x \in [4,10) \cup (20,\infty)$\\
H.$x \in (4,10] \cup (20,\infty)$
\testStop
\kluczStart
A
\kluczStop



\zadStart{Zadanie z Wikieł Z 1.62 a) moja wersja nr 536}

Rozwiązać nierówności $(x-4)(x-11)(x-12)\ge0$.
\zadStop
\rozwStart{Patryk Wirkus}{}
Miejsca zerowe naszego wielomianu to: $4, 11, 12$.\\
Wielomian jest stopnia nieparzystego, ponadto znak współczynnika przy\linebreak najwyższej potędze x jest dodatni.\\ W związku z tym wykres wielomianu zaczyna się od lewej strony poniżej osi OX. A więc $$x \in [4,11] \cup [12,\infty).$$
\rozwStop
\odpStart
$x \in [4,11] \cup [12,\infty)$
\odpStop
\testStart
A.$x \in [4,11] \cup [12,\infty)$\\
B.$x \in (4,11) \cup [12,\infty)$\\
C.$x \in (4,11] \cup [12,\infty)$\\
D.$x \in [4,11) \cup [12,\infty)$\\
E.$x \in [4,11] \cup (12,\infty)$\\
F.$x \in (4,11) \cup (12,\infty)$\\
G.$x \in [4,11) \cup (12,\infty)$\\
H.$x \in (4,11] \cup (12,\infty)$
\testStop
\kluczStart
A
\kluczStop



\zadStart{Zadanie z Wikieł Z 1.62 a) moja wersja nr 537}

Rozwiązać nierówności $(x-4)(x-11)(x-13)\ge0$.
\zadStop
\rozwStart{Patryk Wirkus}{}
Miejsca zerowe naszego wielomianu to: $4, 11, 13$.\\
Wielomian jest stopnia nieparzystego, ponadto znak współczynnika przy\linebreak najwyższej potędze x jest dodatni.\\ W związku z tym wykres wielomianu zaczyna się od lewej strony poniżej osi OX. A więc $$x \in [4,11] \cup [13,\infty).$$
\rozwStop
\odpStart
$x \in [4,11] \cup [13,\infty)$
\odpStop
\testStart
A.$x \in [4,11] \cup [13,\infty)$\\
B.$x \in (4,11) \cup [13,\infty)$\\
C.$x \in (4,11] \cup [13,\infty)$\\
D.$x \in [4,11) \cup [13,\infty)$\\
E.$x \in [4,11] \cup (13,\infty)$\\
F.$x \in (4,11) \cup (13,\infty)$\\
G.$x \in [4,11) \cup (13,\infty)$\\
H.$x \in (4,11] \cup (13,\infty)$
\testStop
\kluczStart
A
\kluczStop



\zadStart{Zadanie z Wikieł Z 1.62 a) moja wersja nr 538}

Rozwiązać nierówności $(x-4)(x-11)(x-14)\ge0$.
\zadStop
\rozwStart{Patryk Wirkus}{}
Miejsca zerowe naszego wielomianu to: $4, 11, 14$.\\
Wielomian jest stopnia nieparzystego, ponadto znak współczynnika przy\linebreak najwyższej potędze x jest dodatni.\\ W związku z tym wykres wielomianu zaczyna się od lewej strony poniżej osi OX. A więc $$x \in [4,11] \cup [14,\infty).$$
\rozwStop
\odpStart
$x \in [4,11] \cup [14,\infty)$
\odpStop
\testStart
A.$x \in [4,11] \cup [14,\infty)$\\
B.$x \in (4,11) \cup [14,\infty)$\\
C.$x \in (4,11] \cup [14,\infty)$\\
D.$x \in [4,11) \cup [14,\infty)$\\
E.$x \in [4,11] \cup (14,\infty)$\\
F.$x \in (4,11) \cup (14,\infty)$\\
G.$x \in [4,11) \cup (14,\infty)$\\
H.$x \in (4,11] \cup (14,\infty)$
\testStop
\kluczStart
A
\kluczStop



\zadStart{Zadanie z Wikieł Z 1.62 a) moja wersja nr 539}

Rozwiązać nierówności $(x-4)(x-11)(x-15)\ge0$.
\zadStop
\rozwStart{Patryk Wirkus}{}
Miejsca zerowe naszego wielomianu to: $4, 11, 15$.\\
Wielomian jest stopnia nieparzystego, ponadto znak współczynnika przy\linebreak najwyższej potędze x jest dodatni.\\ W związku z tym wykres wielomianu zaczyna się od lewej strony poniżej osi OX. A więc $$x \in [4,11] \cup [15,\infty).$$
\rozwStop
\odpStart
$x \in [4,11] \cup [15,\infty)$
\odpStop
\testStart
A.$x \in [4,11] \cup [15,\infty)$\\
B.$x \in (4,11) \cup [15,\infty)$\\
C.$x \in (4,11] \cup [15,\infty)$\\
D.$x \in [4,11) \cup [15,\infty)$\\
E.$x \in [4,11] \cup (15,\infty)$\\
F.$x \in (4,11) \cup (15,\infty)$\\
G.$x \in [4,11) \cup (15,\infty)$\\
H.$x \in (4,11] \cup (15,\infty)$
\testStop
\kluczStart
A
\kluczStop



\zadStart{Zadanie z Wikieł Z 1.62 a) moja wersja nr 540}

Rozwiązać nierówności $(x-4)(x-11)(x-16)\ge0$.
\zadStop
\rozwStart{Patryk Wirkus}{}
Miejsca zerowe naszego wielomianu to: $4, 11, 16$.\\
Wielomian jest stopnia nieparzystego, ponadto znak współczynnika przy\linebreak najwyższej potędze x jest dodatni.\\ W związku z tym wykres wielomianu zaczyna się od lewej strony poniżej osi OX. A więc $$x \in [4,11] \cup [16,\infty).$$
\rozwStop
\odpStart
$x \in [4,11] \cup [16,\infty)$
\odpStop
\testStart
A.$x \in [4,11] \cup [16,\infty)$\\
B.$x \in (4,11) \cup [16,\infty)$\\
C.$x \in (4,11] \cup [16,\infty)$\\
D.$x \in [4,11) \cup [16,\infty)$\\
E.$x \in [4,11] \cup (16,\infty)$\\
F.$x \in (4,11) \cup (16,\infty)$\\
G.$x \in [4,11) \cup (16,\infty)$\\
H.$x \in (4,11] \cup (16,\infty)$
\testStop
\kluczStart
A
\kluczStop



\zadStart{Zadanie z Wikieł Z 1.62 a) moja wersja nr 541}

Rozwiązać nierówności $(x-4)(x-11)(x-17)\ge0$.
\zadStop
\rozwStart{Patryk Wirkus}{}
Miejsca zerowe naszego wielomianu to: $4, 11, 17$.\\
Wielomian jest stopnia nieparzystego, ponadto znak współczynnika przy\linebreak najwyższej potędze x jest dodatni.\\ W związku z tym wykres wielomianu zaczyna się od lewej strony poniżej osi OX. A więc $$x \in [4,11] \cup [17,\infty).$$
\rozwStop
\odpStart
$x \in [4,11] \cup [17,\infty)$
\odpStop
\testStart
A.$x \in [4,11] \cup [17,\infty)$\\
B.$x \in (4,11) \cup [17,\infty)$\\
C.$x \in (4,11] \cup [17,\infty)$\\
D.$x \in [4,11) \cup [17,\infty)$\\
E.$x \in [4,11] \cup (17,\infty)$\\
F.$x \in (4,11) \cup (17,\infty)$\\
G.$x \in [4,11) \cup (17,\infty)$\\
H.$x \in (4,11] \cup (17,\infty)$
\testStop
\kluczStart
A
\kluczStop



\zadStart{Zadanie z Wikieł Z 1.62 a) moja wersja nr 542}

Rozwiązać nierówności $(x-4)(x-11)(x-18)\ge0$.
\zadStop
\rozwStart{Patryk Wirkus}{}
Miejsca zerowe naszego wielomianu to: $4, 11, 18$.\\
Wielomian jest stopnia nieparzystego, ponadto znak współczynnika przy\linebreak najwyższej potędze x jest dodatni.\\ W związku z tym wykres wielomianu zaczyna się od lewej strony poniżej osi OX. A więc $$x \in [4,11] \cup [18,\infty).$$
\rozwStop
\odpStart
$x \in [4,11] \cup [18,\infty)$
\odpStop
\testStart
A.$x \in [4,11] \cup [18,\infty)$\\
B.$x \in (4,11) \cup [18,\infty)$\\
C.$x \in (4,11] \cup [18,\infty)$\\
D.$x \in [4,11) \cup [18,\infty)$\\
E.$x \in [4,11] \cup (18,\infty)$\\
F.$x \in (4,11) \cup (18,\infty)$\\
G.$x \in [4,11) \cup (18,\infty)$\\
H.$x \in (4,11] \cup (18,\infty)$
\testStop
\kluczStart
A
\kluczStop



\zadStart{Zadanie z Wikieł Z 1.62 a) moja wersja nr 543}

Rozwiązać nierówności $(x-4)(x-11)(x-19)\ge0$.
\zadStop
\rozwStart{Patryk Wirkus}{}
Miejsca zerowe naszego wielomianu to: $4, 11, 19$.\\
Wielomian jest stopnia nieparzystego, ponadto znak współczynnika przy\linebreak najwyższej potędze x jest dodatni.\\ W związku z tym wykres wielomianu zaczyna się od lewej strony poniżej osi OX. A więc $$x \in [4,11] \cup [19,\infty).$$
\rozwStop
\odpStart
$x \in [4,11] \cup [19,\infty)$
\odpStop
\testStart
A.$x \in [4,11] \cup [19,\infty)$\\
B.$x \in (4,11) \cup [19,\infty)$\\
C.$x \in (4,11] \cup [19,\infty)$\\
D.$x \in [4,11) \cup [19,\infty)$\\
E.$x \in [4,11] \cup (19,\infty)$\\
F.$x \in (4,11) \cup (19,\infty)$\\
G.$x \in [4,11) \cup (19,\infty)$\\
H.$x \in (4,11] \cup (19,\infty)$
\testStop
\kluczStart
A
\kluczStop



\zadStart{Zadanie z Wikieł Z 1.62 a) moja wersja nr 544}

Rozwiązać nierówności $(x-4)(x-11)(x-20)\ge0$.
\zadStop
\rozwStart{Patryk Wirkus}{}
Miejsca zerowe naszego wielomianu to: $4, 11, 20$.\\
Wielomian jest stopnia nieparzystego, ponadto znak współczynnika przy\linebreak najwyższej potędze x jest dodatni.\\ W związku z tym wykres wielomianu zaczyna się od lewej strony poniżej osi OX. A więc $$x \in [4,11] \cup [20,\infty).$$
\rozwStop
\odpStart
$x \in [4,11] \cup [20,\infty)$
\odpStop
\testStart
A.$x \in [4,11] \cup [20,\infty)$\\
B.$x \in (4,11) \cup [20,\infty)$\\
C.$x \in (4,11] \cup [20,\infty)$\\
D.$x \in [4,11) \cup [20,\infty)$\\
E.$x \in [4,11] \cup (20,\infty)$\\
F.$x \in (4,11) \cup (20,\infty)$\\
G.$x \in [4,11) \cup (20,\infty)$\\
H.$x \in (4,11] \cup (20,\infty)$
\testStop
\kluczStart
A
\kluczStop



\zadStart{Zadanie z Wikieł Z 1.62 a) moja wersja nr 545}

Rozwiązać nierówności $(x-4)(x-12)(x-13)\ge0$.
\zadStop
\rozwStart{Patryk Wirkus}{}
Miejsca zerowe naszego wielomianu to: $4, 12, 13$.\\
Wielomian jest stopnia nieparzystego, ponadto znak współczynnika przy\linebreak najwyższej potędze x jest dodatni.\\ W związku z tym wykres wielomianu zaczyna się od lewej strony poniżej osi OX. A więc $$x \in [4,12] \cup [13,\infty).$$
\rozwStop
\odpStart
$x \in [4,12] \cup [13,\infty)$
\odpStop
\testStart
A.$x \in [4,12] \cup [13,\infty)$\\
B.$x \in (4,12) \cup [13,\infty)$\\
C.$x \in (4,12] \cup [13,\infty)$\\
D.$x \in [4,12) \cup [13,\infty)$\\
E.$x \in [4,12] \cup (13,\infty)$\\
F.$x \in (4,12) \cup (13,\infty)$\\
G.$x \in [4,12) \cup (13,\infty)$\\
H.$x \in (4,12] \cup (13,\infty)$
\testStop
\kluczStart
A
\kluczStop



\zadStart{Zadanie z Wikieł Z 1.62 a) moja wersja nr 546}

Rozwiązać nierówności $(x-4)(x-12)(x-14)\ge0$.
\zadStop
\rozwStart{Patryk Wirkus}{}
Miejsca zerowe naszego wielomianu to: $4, 12, 14$.\\
Wielomian jest stopnia nieparzystego, ponadto znak współczynnika przy\linebreak najwyższej potędze x jest dodatni.\\ W związku z tym wykres wielomianu zaczyna się od lewej strony poniżej osi OX. A więc $$x \in [4,12] \cup [14,\infty).$$
\rozwStop
\odpStart
$x \in [4,12] \cup [14,\infty)$
\odpStop
\testStart
A.$x \in [4,12] \cup [14,\infty)$\\
B.$x \in (4,12) \cup [14,\infty)$\\
C.$x \in (4,12] \cup [14,\infty)$\\
D.$x \in [4,12) \cup [14,\infty)$\\
E.$x \in [4,12] \cup (14,\infty)$\\
F.$x \in (4,12) \cup (14,\infty)$\\
G.$x \in [4,12) \cup (14,\infty)$\\
H.$x \in (4,12] \cup (14,\infty)$
\testStop
\kluczStart
A
\kluczStop



\zadStart{Zadanie z Wikieł Z 1.62 a) moja wersja nr 547}

Rozwiązać nierówności $(x-4)(x-12)(x-15)\ge0$.
\zadStop
\rozwStart{Patryk Wirkus}{}
Miejsca zerowe naszego wielomianu to: $4, 12, 15$.\\
Wielomian jest stopnia nieparzystego, ponadto znak współczynnika przy\linebreak najwyższej potędze x jest dodatni.\\ W związku z tym wykres wielomianu zaczyna się od lewej strony poniżej osi OX. A więc $$x \in [4,12] \cup [15,\infty).$$
\rozwStop
\odpStart
$x \in [4,12] \cup [15,\infty)$
\odpStop
\testStart
A.$x \in [4,12] \cup [15,\infty)$\\
B.$x \in (4,12) \cup [15,\infty)$\\
C.$x \in (4,12] \cup [15,\infty)$\\
D.$x \in [4,12) \cup [15,\infty)$\\
E.$x \in [4,12] \cup (15,\infty)$\\
F.$x \in (4,12) \cup (15,\infty)$\\
G.$x \in [4,12) \cup (15,\infty)$\\
H.$x \in (4,12] \cup (15,\infty)$
\testStop
\kluczStart
A
\kluczStop



\zadStart{Zadanie z Wikieł Z 1.62 a) moja wersja nr 548}

Rozwiązać nierówności $(x-4)(x-12)(x-16)\ge0$.
\zadStop
\rozwStart{Patryk Wirkus}{}
Miejsca zerowe naszego wielomianu to: $4, 12, 16$.\\
Wielomian jest stopnia nieparzystego, ponadto znak współczynnika przy\linebreak najwyższej potędze x jest dodatni.\\ W związku z tym wykres wielomianu zaczyna się od lewej strony poniżej osi OX. A więc $$x \in [4,12] \cup [16,\infty).$$
\rozwStop
\odpStart
$x \in [4,12] \cup [16,\infty)$
\odpStop
\testStart
A.$x \in [4,12] \cup [16,\infty)$\\
B.$x \in (4,12) \cup [16,\infty)$\\
C.$x \in (4,12] \cup [16,\infty)$\\
D.$x \in [4,12) \cup [16,\infty)$\\
E.$x \in [4,12] \cup (16,\infty)$\\
F.$x \in (4,12) \cup (16,\infty)$\\
G.$x \in [4,12) \cup (16,\infty)$\\
H.$x \in (4,12] \cup (16,\infty)$
\testStop
\kluczStart
A
\kluczStop



\zadStart{Zadanie z Wikieł Z 1.62 a) moja wersja nr 549}

Rozwiązać nierówności $(x-4)(x-12)(x-17)\ge0$.
\zadStop
\rozwStart{Patryk Wirkus}{}
Miejsca zerowe naszego wielomianu to: $4, 12, 17$.\\
Wielomian jest stopnia nieparzystego, ponadto znak współczynnika przy\linebreak najwyższej potędze x jest dodatni.\\ W związku z tym wykres wielomianu zaczyna się od lewej strony poniżej osi OX. A więc $$x \in [4,12] \cup [17,\infty).$$
\rozwStop
\odpStart
$x \in [4,12] \cup [17,\infty)$
\odpStop
\testStart
A.$x \in [4,12] \cup [17,\infty)$\\
B.$x \in (4,12) \cup [17,\infty)$\\
C.$x \in (4,12] \cup [17,\infty)$\\
D.$x \in [4,12) \cup [17,\infty)$\\
E.$x \in [4,12] \cup (17,\infty)$\\
F.$x \in (4,12) \cup (17,\infty)$\\
G.$x \in [4,12) \cup (17,\infty)$\\
H.$x \in (4,12] \cup (17,\infty)$
\testStop
\kluczStart
A
\kluczStop



\zadStart{Zadanie z Wikieł Z 1.62 a) moja wersja nr 550}

Rozwiązać nierówności $(x-4)(x-12)(x-18)\ge0$.
\zadStop
\rozwStart{Patryk Wirkus}{}
Miejsca zerowe naszego wielomianu to: $4, 12, 18$.\\
Wielomian jest stopnia nieparzystego, ponadto znak współczynnika przy\linebreak najwyższej potędze x jest dodatni.\\ W związku z tym wykres wielomianu zaczyna się od lewej strony poniżej osi OX. A więc $$x \in [4,12] \cup [18,\infty).$$
\rozwStop
\odpStart
$x \in [4,12] \cup [18,\infty)$
\odpStop
\testStart
A.$x \in [4,12] \cup [18,\infty)$\\
B.$x \in (4,12) \cup [18,\infty)$\\
C.$x \in (4,12] \cup [18,\infty)$\\
D.$x \in [4,12) \cup [18,\infty)$\\
E.$x \in [4,12] \cup (18,\infty)$\\
F.$x \in (4,12) \cup (18,\infty)$\\
G.$x \in [4,12) \cup (18,\infty)$\\
H.$x \in (4,12] \cup (18,\infty)$
\testStop
\kluczStart
A
\kluczStop



\zadStart{Zadanie z Wikieł Z 1.62 a) moja wersja nr 551}

Rozwiązać nierówności $(x-4)(x-12)(x-19)\ge0$.
\zadStop
\rozwStart{Patryk Wirkus}{}
Miejsca zerowe naszego wielomianu to: $4, 12, 19$.\\
Wielomian jest stopnia nieparzystego, ponadto znak współczynnika przy\linebreak najwyższej potędze x jest dodatni.\\ W związku z tym wykres wielomianu zaczyna się od lewej strony poniżej osi OX. A więc $$x \in [4,12] \cup [19,\infty).$$
\rozwStop
\odpStart
$x \in [4,12] \cup [19,\infty)$
\odpStop
\testStart
A.$x \in [4,12] \cup [19,\infty)$\\
B.$x \in (4,12) \cup [19,\infty)$\\
C.$x \in (4,12] \cup [19,\infty)$\\
D.$x \in [4,12) \cup [19,\infty)$\\
E.$x \in [4,12] \cup (19,\infty)$\\
F.$x \in (4,12) \cup (19,\infty)$\\
G.$x \in [4,12) \cup (19,\infty)$\\
H.$x \in (4,12] \cup (19,\infty)$
\testStop
\kluczStart
A
\kluczStop



\zadStart{Zadanie z Wikieł Z 1.62 a) moja wersja nr 552}

Rozwiązać nierówności $(x-4)(x-12)(x-20)\ge0$.
\zadStop
\rozwStart{Patryk Wirkus}{}
Miejsca zerowe naszego wielomianu to: $4, 12, 20$.\\
Wielomian jest stopnia nieparzystego, ponadto znak współczynnika przy\linebreak najwyższej potędze x jest dodatni.\\ W związku z tym wykres wielomianu zaczyna się od lewej strony poniżej osi OX. A więc $$x \in [4,12] \cup [20,\infty).$$
\rozwStop
\odpStart
$x \in [4,12] \cup [20,\infty)$
\odpStop
\testStart
A.$x \in [4,12] \cup [20,\infty)$\\
B.$x \in (4,12) \cup [20,\infty)$\\
C.$x \in (4,12] \cup [20,\infty)$\\
D.$x \in [4,12) \cup [20,\infty)$\\
E.$x \in [4,12] \cup (20,\infty)$\\
F.$x \in (4,12) \cup (20,\infty)$\\
G.$x \in [4,12) \cup (20,\infty)$\\
H.$x \in (4,12] \cup (20,\infty)$
\testStop
\kluczStart
A
\kluczStop



\zadStart{Zadanie z Wikieł Z 1.62 a) moja wersja nr 553}

Rozwiązać nierówności $(x-4)(x-13)(x-14)\ge0$.
\zadStop
\rozwStart{Patryk Wirkus}{}
Miejsca zerowe naszego wielomianu to: $4, 13, 14$.\\
Wielomian jest stopnia nieparzystego, ponadto znak współczynnika przy\linebreak najwyższej potędze x jest dodatni.\\ W związku z tym wykres wielomianu zaczyna się od lewej strony poniżej osi OX. A więc $$x \in [4,13] \cup [14,\infty).$$
\rozwStop
\odpStart
$x \in [4,13] \cup [14,\infty)$
\odpStop
\testStart
A.$x \in [4,13] \cup [14,\infty)$\\
B.$x \in (4,13) \cup [14,\infty)$\\
C.$x \in (4,13] \cup [14,\infty)$\\
D.$x \in [4,13) \cup [14,\infty)$\\
E.$x \in [4,13] \cup (14,\infty)$\\
F.$x \in (4,13) \cup (14,\infty)$\\
G.$x \in [4,13) \cup (14,\infty)$\\
H.$x \in (4,13] \cup (14,\infty)$
\testStop
\kluczStart
A
\kluczStop



\zadStart{Zadanie z Wikieł Z 1.62 a) moja wersja nr 554}

Rozwiązać nierówności $(x-4)(x-13)(x-15)\ge0$.
\zadStop
\rozwStart{Patryk Wirkus}{}
Miejsca zerowe naszego wielomianu to: $4, 13, 15$.\\
Wielomian jest stopnia nieparzystego, ponadto znak współczynnika przy\linebreak najwyższej potędze x jest dodatni.\\ W związku z tym wykres wielomianu zaczyna się od lewej strony poniżej osi OX. A więc $$x \in [4,13] \cup [15,\infty).$$
\rozwStop
\odpStart
$x \in [4,13] \cup [15,\infty)$
\odpStop
\testStart
A.$x \in [4,13] \cup [15,\infty)$\\
B.$x \in (4,13) \cup [15,\infty)$\\
C.$x \in (4,13] \cup [15,\infty)$\\
D.$x \in [4,13) \cup [15,\infty)$\\
E.$x \in [4,13] \cup (15,\infty)$\\
F.$x \in (4,13) \cup (15,\infty)$\\
G.$x \in [4,13) \cup (15,\infty)$\\
H.$x \in (4,13] \cup (15,\infty)$
\testStop
\kluczStart
A
\kluczStop



\zadStart{Zadanie z Wikieł Z 1.62 a) moja wersja nr 555}

Rozwiązać nierówności $(x-4)(x-13)(x-16)\ge0$.
\zadStop
\rozwStart{Patryk Wirkus}{}
Miejsca zerowe naszego wielomianu to: $4, 13, 16$.\\
Wielomian jest stopnia nieparzystego, ponadto znak współczynnika przy\linebreak najwyższej potędze x jest dodatni.\\ W związku z tym wykres wielomianu zaczyna się od lewej strony poniżej osi OX. A więc $$x \in [4,13] \cup [16,\infty).$$
\rozwStop
\odpStart
$x \in [4,13] \cup [16,\infty)$
\odpStop
\testStart
A.$x \in [4,13] \cup [16,\infty)$\\
B.$x \in (4,13) \cup [16,\infty)$\\
C.$x \in (4,13] \cup [16,\infty)$\\
D.$x \in [4,13) \cup [16,\infty)$\\
E.$x \in [4,13] \cup (16,\infty)$\\
F.$x \in (4,13) \cup (16,\infty)$\\
G.$x \in [4,13) \cup (16,\infty)$\\
H.$x \in (4,13] \cup (16,\infty)$
\testStop
\kluczStart
A
\kluczStop



\zadStart{Zadanie z Wikieł Z 1.62 a) moja wersja nr 556}

Rozwiązać nierówności $(x-4)(x-13)(x-17)\ge0$.
\zadStop
\rozwStart{Patryk Wirkus}{}
Miejsca zerowe naszego wielomianu to: $4, 13, 17$.\\
Wielomian jest stopnia nieparzystego, ponadto znak współczynnika przy\linebreak najwyższej potędze x jest dodatni.\\ W związku z tym wykres wielomianu zaczyna się od lewej strony poniżej osi OX. A więc $$x \in [4,13] \cup [17,\infty).$$
\rozwStop
\odpStart
$x \in [4,13] \cup [17,\infty)$
\odpStop
\testStart
A.$x \in [4,13] \cup [17,\infty)$\\
B.$x \in (4,13) \cup [17,\infty)$\\
C.$x \in (4,13] \cup [17,\infty)$\\
D.$x \in [4,13) \cup [17,\infty)$\\
E.$x \in [4,13] \cup (17,\infty)$\\
F.$x \in (4,13) \cup (17,\infty)$\\
G.$x \in [4,13) \cup (17,\infty)$\\
H.$x \in (4,13] \cup (17,\infty)$
\testStop
\kluczStart
A
\kluczStop



\zadStart{Zadanie z Wikieł Z 1.62 a) moja wersja nr 557}

Rozwiązać nierówności $(x-4)(x-13)(x-18)\ge0$.
\zadStop
\rozwStart{Patryk Wirkus}{}
Miejsca zerowe naszego wielomianu to: $4, 13, 18$.\\
Wielomian jest stopnia nieparzystego, ponadto znak współczynnika przy\linebreak najwyższej potędze x jest dodatni.\\ W związku z tym wykres wielomianu zaczyna się od lewej strony poniżej osi OX. A więc $$x \in [4,13] \cup [18,\infty).$$
\rozwStop
\odpStart
$x \in [4,13] \cup [18,\infty)$
\odpStop
\testStart
A.$x \in [4,13] \cup [18,\infty)$\\
B.$x \in (4,13) \cup [18,\infty)$\\
C.$x \in (4,13] \cup [18,\infty)$\\
D.$x \in [4,13) \cup [18,\infty)$\\
E.$x \in [4,13] \cup (18,\infty)$\\
F.$x \in (4,13) \cup (18,\infty)$\\
G.$x \in [4,13) \cup (18,\infty)$\\
H.$x \in (4,13] \cup (18,\infty)$
\testStop
\kluczStart
A
\kluczStop



\zadStart{Zadanie z Wikieł Z 1.62 a) moja wersja nr 558}

Rozwiązać nierówności $(x-4)(x-13)(x-19)\ge0$.
\zadStop
\rozwStart{Patryk Wirkus}{}
Miejsca zerowe naszego wielomianu to: $4, 13, 19$.\\
Wielomian jest stopnia nieparzystego, ponadto znak współczynnika przy\linebreak najwyższej potędze x jest dodatni.\\ W związku z tym wykres wielomianu zaczyna się od lewej strony poniżej osi OX. A więc $$x \in [4,13] \cup [19,\infty).$$
\rozwStop
\odpStart
$x \in [4,13] \cup [19,\infty)$
\odpStop
\testStart
A.$x \in [4,13] \cup [19,\infty)$\\
B.$x \in (4,13) \cup [19,\infty)$\\
C.$x \in (4,13] \cup [19,\infty)$\\
D.$x \in [4,13) \cup [19,\infty)$\\
E.$x \in [4,13] \cup (19,\infty)$\\
F.$x \in (4,13) \cup (19,\infty)$\\
G.$x \in [4,13) \cup (19,\infty)$\\
H.$x \in (4,13] \cup (19,\infty)$
\testStop
\kluczStart
A
\kluczStop



\zadStart{Zadanie z Wikieł Z 1.62 a) moja wersja nr 559}

Rozwiązać nierówności $(x-4)(x-13)(x-20)\ge0$.
\zadStop
\rozwStart{Patryk Wirkus}{}
Miejsca zerowe naszego wielomianu to: $4, 13, 20$.\\
Wielomian jest stopnia nieparzystego, ponadto znak współczynnika przy\linebreak najwyższej potędze x jest dodatni.\\ W związku z tym wykres wielomianu zaczyna się od lewej strony poniżej osi OX. A więc $$x \in [4,13] \cup [20,\infty).$$
\rozwStop
\odpStart
$x \in [4,13] \cup [20,\infty)$
\odpStop
\testStart
A.$x \in [4,13] \cup [20,\infty)$\\
B.$x \in (4,13) \cup [20,\infty)$\\
C.$x \in (4,13] \cup [20,\infty)$\\
D.$x \in [4,13) \cup [20,\infty)$\\
E.$x \in [4,13] \cup (20,\infty)$\\
F.$x \in (4,13) \cup (20,\infty)$\\
G.$x \in [4,13) \cup (20,\infty)$\\
H.$x \in (4,13] \cup (20,\infty)$
\testStop
\kluczStart
A
\kluczStop



\zadStart{Zadanie z Wikieł Z 1.62 a) moja wersja nr 560}

Rozwiązać nierówności $(x-4)(x-14)(x-15)\ge0$.
\zadStop
\rozwStart{Patryk Wirkus}{}
Miejsca zerowe naszego wielomianu to: $4, 14, 15$.\\
Wielomian jest stopnia nieparzystego, ponadto znak współczynnika przy\linebreak najwyższej potędze x jest dodatni.\\ W związku z tym wykres wielomianu zaczyna się od lewej strony poniżej osi OX. A więc $$x \in [4,14] \cup [15,\infty).$$
\rozwStop
\odpStart
$x \in [4,14] \cup [15,\infty)$
\odpStop
\testStart
A.$x \in [4,14] \cup [15,\infty)$\\
B.$x \in (4,14) \cup [15,\infty)$\\
C.$x \in (4,14] \cup [15,\infty)$\\
D.$x \in [4,14) \cup [15,\infty)$\\
E.$x \in [4,14] \cup (15,\infty)$\\
F.$x \in (4,14) \cup (15,\infty)$\\
G.$x \in [4,14) \cup (15,\infty)$\\
H.$x \in (4,14] \cup (15,\infty)$
\testStop
\kluczStart
A
\kluczStop



\zadStart{Zadanie z Wikieł Z 1.62 a) moja wersja nr 561}

Rozwiązać nierówności $(x-4)(x-14)(x-16)\ge0$.
\zadStop
\rozwStart{Patryk Wirkus}{}
Miejsca zerowe naszego wielomianu to: $4, 14, 16$.\\
Wielomian jest stopnia nieparzystego, ponadto znak współczynnika przy\linebreak najwyższej potędze x jest dodatni.\\ W związku z tym wykres wielomianu zaczyna się od lewej strony poniżej osi OX. A więc $$x \in [4,14] \cup [16,\infty).$$
\rozwStop
\odpStart
$x \in [4,14] \cup [16,\infty)$
\odpStop
\testStart
A.$x \in [4,14] \cup [16,\infty)$\\
B.$x \in (4,14) \cup [16,\infty)$\\
C.$x \in (4,14] \cup [16,\infty)$\\
D.$x \in [4,14) \cup [16,\infty)$\\
E.$x \in [4,14] \cup (16,\infty)$\\
F.$x \in (4,14) \cup (16,\infty)$\\
G.$x \in [4,14) \cup (16,\infty)$\\
H.$x \in (4,14] \cup (16,\infty)$
\testStop
\kluczStart
A
\kluczStop



\zadStart{Zadanie z Wikieł Z 1.62 a) moja wersja nr 562}

Rozwiązać nierówności $(x-4)(x-14)(x-17)\ge0$.
\zadStop
\rozwStart{Patryk Wirkus}{}
Miejsca zerowe naszego wielomianu to: $4, 14, 17$.\\
Wielomian jest stopnia nieparzystego, ponadto znak współczynnika przy\linebreak najwyższej potędze x jest dodatni.\\ W związku z tym wykres wielomianu zaczyna się od lewej strony poniżej osi OX. A więc $$x \in [4,14] \cup [17,\infty).$$
\rozwStop
\odpStart
$x \in [4,14] \cup [17,\infty)$
\odpStop
\testStart
A.$x \in [4,14] \cup [17,\infty)$\\
B.$x \in (4,14) \cup [17,\infty)$\\
C.$x \in (4,14] \cup [17,\infty)$\\
D.$x \in [4,14) \cup [17,\infty)$\\
E.$x \in [4,14] \cup (17,\infty)$\\
F.$x \in (4,14) \cup (17,\infty)$\\
G.$x \in [4,14) \cup (17,\infty)$\\
H.$x \in (4,14] \cup (17,\infty)$
\testStop
\kluczStart
A
\kluczStop



\zadStart{Zadanie z Wikieł Z 1.62 a) moja wersja nr 563}

Rozwiązać nierówności $(x-4)(x-14)(x-18)\ge0$.
\zadStop
\rozwStart{Patryk Wirkus}{}
Miejsca zerowe naszego wielomianu to: $4, 14, 18$.\\
Wielomian jest stopnia nieparzystego, ponadto znak współczynnika przy\linebreak najwyższej potędze x jest dodatni.\\ W związku z tym wykres wielomianu zaczyna się od lewej strony poniżej osi OX. A więc $$x \in [4,14] \cup [18,\infty).$$
\rozwStop
\odpStart
$x \in [4,14] \cup [18,\infty)$
\odpStop
\testStart
A.$x \in [4,14] \cup [18,\infty)$\\
B.$x \in (4,14) \cup [18,\infty)$\\
C.$x \in (4,14] \cup [18,\infty)$\\
D.$x \in [4,14) \cup [18,\infty)$\\
E.$x \in [4,14] \cup (18,\infty)$\\
F.$x \in (4,14) \cup (18,\infty)$\\
G.$x \in [4,14) \cup (18,\infty)$\\
H.$x \in (4,14] \cup (18,\infty)$
\testStop
\kluczStart
A
\kluczStop



\zadStart{Zadanie z Wikieł Z 1.62 a) moja wersja nr 564}

Rozwiązać nierówności $(x-4)(x-14)(x-19)\ge0$.
\zadStop
\rozwStart{Patryk Wirkus}{}
Miejsca zerowe naszego wielomianu to: $4, 14, 19$.\\
Wielomian jest stopnia nieparzystego, ponadto znak współczynnika przy\linebreak najwyższej potędze x jest dodatni.\\ W związku z tym wykres wielomianu zaczyna się od lewej strony poniżej osi OX. A więc $$x \in [4,14] \cup [19,\infty).$$
\rozwStop
\odpStart
$x \in [4,14] \cup [19,\infty)$
\odpStop
\testStart
A.$x \in [4,14] \cup [19,\infty)$\\
B.$x \in (4,14) \cup [19,\infty)$\\
C.$x \in (4,14] \cup [19,\infty)$\\
D.$x \in [4,14) \cup [19,\infty)$\\
E.$x \in [4,14] \cup (19,\infty)$\\
F.$x \in (4,14) \cup (19,\infty)$\\
G.$x \in [4,14) \cup (19,\infty)$\\
H.$x \in (4,14] \cup (19,\infty)$
\testStop
\kluczStart
A
\kluczStop



\zadStart{Zadanie z Wikieł Z 1.62 a) moja wersja nr 565}

Rozwiązać nierówności $(x-4)(x-14)(x-20)\ge0$.
\zadStop
\rozwStart{Patryk Wirkus}{}
Miejsca zerowe naszego wielomianu to: $4, 14, 20$.\\
Wielomian jest stopnia nieparzystego, ponadto znak współczynnika przy\linebreak najwyższej potędze x jest dodatni.\\ W związku z tym wykres wielomianu zaczyna się od lewej strony poniżej osi OX. A więc $$x \in [4,14] \cup [20,\infty).$$
\rozwStop
\odpStart
$x \in [4,14] \cup [20,\infty)$
\odpStop
\testStart
A.$x \in [4,14] \cup [20,\infty)$\\
B.$x \in (4,14) \cup [20,\infty)$\\
C.$x \in (4,14] \cup [20,\infty)$\\
D.$x \in [4,14) \cup [20,\infty)$\\
E.$x \in [4,14] \cup (20,\infty)$\\
F.$x \in (4,14) \cup (20,\infty)$\\
G.$x \in [4,14) \cup (20,\infty)$\\
H.$x \in (4,14] \cup (20,\infty)$
\testStop
\kluczStart
A
\kluczStop



\zadStart{Zadanie z Wikieł Z 1.62 a) moja wersja nr 566}

Rozwiązać nierówności $(x-4)(x-15)(x-16)\ge0$.
\zadStop
\rozwStart{Patryk Wirkus}{}
Miejsca zerowe naszego wielomianu to: $4, 15, 16$.\\
Wielomian jest stopnia nieparzystego, ponadto znak współczynnika przy\linebreak najwyższej potędze x jest dodatni.\\ W związku z tym wykres wielomianu zaczyna się od lewej strony poniżej osi OX. A więc $$x \in [4,15] \cup [16,\infty).$$
\rozwStop
\odpStart
$x \in [4,15] \cup [16,\infty)$
\odpStop
\testStart
A.$x \in [4,15] \cup [16,\infty)$\\
B.$x \in (4,15) \cup [16,\infty)$\\
C.$x \in (4,15] \cup [16,\infty)$\\
D.$x \in [4,15) \cup [16,\infty)$\\
E.$x \in [4,15] \cup (16,\infty)$\\
F.$x \in (4,15) \cup (16,\infty)$\\
G.$x \in [4,15) \cup (16,\infty)$\\
H.$x \in (4,15] \cup (16,\infty)$
\testStop
\kluczStart
A
\kluczStop



\zadStart{Zadanie z Wikieł Z 1.62 a) moja wersja nr 567}

Rozwiązać nierówności $(x-4)(x-15)(x-17)\ge0$.
\zadStop
\rozwStart{Patryk Wirkus}{}
Miejsca zerowe naszego wielomianu to: $4, 15, 17$.\\
Wielomian jest stopnia nieparzystego, ponadto znak współczynnika przy\linebreak najwyższej potędze x jest dodatni.\\ W związku z tym wykres wielomianu zaczyna się od lewej strony poniżej osi OX. A więc $$x \in [4,15] \cup [17,\infty).$$
\rozwStop
\odpStart
$x \in [4,15] \cup [17,\infty)$
\odpStop
\testStart
A.$x \in [4,15] \cup [17,\infty)$\\
B.$x \in (4,15) \cup [17,\infty)$\\
C.$x \in (4,15] \cup [17,\infty)$\\
D.$x \in [4,15) \cup [17,\infty)$\\
E.$x \in [4,15] \cup (17,\infty)$\\
F.$x \in (4,15) \cup (17,\infty)$\\
G.$x \in [4,15) \cup (17,\infty)$\\
H.$x \in (4,15] \cup (17,\infty)$
\testStop
\kluczStart
A
\kluczStop



\zadStart{Zadanie z Wikieł Z 1.62 a) moja wersja nr 568}

Rozwiązać nierówności $(x-4)(x-15)(x-18)\ge0$.
\zadStop
\rozwStart{Patryk Wirkus}{}
Miejsca zerowe naszego wielomianu to: $4, 15, 18$.\\
Wielomian jest stopnia nieparzystego, ponadto znak współczynnika przy\linebreak najwyższej potędze x jest dodatni.\\ W związku z tym wykres wielomianu zaczyna się od lewej strony poniżej osi OX. A więc $$x \in [4,15] \cup [18,\infty).$$
\rozwStop
\odpStart
$x \in [4,15] \cup [18,\infty)$
\odpStop
\testStart
A.$x \in [4,15] \cup [18,\infty)$\\
B.$x \in (4,15) \cup [18,\infty)$\\
C.$x \in (4,15] \cup [18,\infty)$\\
D.$x \in [4,15) \cup [18,\infty)$\\
E.$x \in [4,15] \cup (18,\infty)$\\
F.$x \in (4,15) \cup (18,\infty)$\\
G.$x \in [4,15) \cup (18,\infty)$\\
H.$x \in (4,15] \cup (18,\infty)$
\testStop
\kluczStart
A
\kluczStop



\zadStart{Zadanie z Wikieł Z 1.62 a) moja wersja nr 569}

Rozwiązać nierówności $(x-4)(x-15)(x-19)\ge0$.
\zadStop
\rozwStart{Patryk Wirkus}{}
Miejsca zerowe naszego wielomianu to: $4, 15, 19$.\\
Wielomian jest stopnia nieparzystego, ponadto znak współczynnika przy\linebreak najwyższej potędze x jest dodatni.\\ W związku z tym wykres wielomianu zaczyna się od lewej strony poniżej osi OX. A więc $$x \in [4,15] \cup [19,\infty).$$
\rozwStop
\odpStart
$x \in [4,15] \cup [19,\infty)$
\odpStop
\testStart
A.$x \in [4,15] \cup [19,\infty)$\\
B.$x \in (4,15) \cup [19,\infty)$\\
C.$x \in (4,15] \cup [19,\infty)$\\
D.$x \in [4,15) \cup [19,\infty)$\\
E.$x \in [4,15] \cup (19,\infty)$\\
F.$x \in (4,15) \cup (19,\infty)$\\
G.$x \in [4,15) \cup (19,\infty)$\\
H.$x \in (4,15] \cup (19,\infty)$
\testStop
\kluczStart
A
\kluczStop



\zadStart{Zadanie z Wikieł Z 1.62 a) moja wersja nr 570}

Rozwiązać nierówności $(x-4)(x-15)(x-20)\ge0$.
\zadStop
\rozwStart{Patryk Wirkus}{}
Miejsca zerowe naszego wielomianu to: $4, 15, 20$.\\
Wielomian jest stopnia nieparzystego, ponadto znak współczynnika przy\linebreak najwyższej potędze x jest dodatni.\\ W związku z tym wykres wielomianu zaczyna się od lewej strony poniżej osi OX. A więc $$x \in [4,15] \cup [20,\infty).$$
\rozwStop
\odpStart
$x \in [4,15] \cup [20,\infty)$
\odpStop
\testStart
A.$x \in [4,15] \cup [20,\infty)$\\
B.$x \in (4,15) \cup [20,\infty)$\\
C.$x \in (4,15] \cup [20,\infty)$\\
D.$x \in [4,15) \cup [20,\infty)$\\
E.$x \in [4,15] \cup (20,\infty)$\\
F.$x \in (4,15) \cup (20,\infty)$\\
G.$x \in [4,15) \cup (20,\infty)$\\
H.$x \in (4,15] \cup (20,\infty)$
\testStop
\kluczStart
A
\kluczStop



\zadStart{Zadanie z Wikieł Z 1.62 a) moja wersja nr 571}

Rozwiązać nierówności $(x-4)(x-16)(x-17)\ge0$.
\zadStop
\rozwStart{Patryk Wirkus}{}
Miejsca zerowe naszego wielomianu to: $4, 16, 17$.\\
Wielomian jest stopnia nieparzystego, ponadto znak współczynnika przy\linebreak najwyższej potędze x jest dodatni.\\ W związku z tym wykres wielomianu zaczyna się od lewej strony poniżej osi OX. A więc $$x \in [4,16] \cup [17,\infty).$$
\rozwStop
\odpStart
$x \in [4,16] \cup [17,\infty)$
\odpStop
\testStart
A.$x \in [4,16] \cup [17,\infty)$\\
B.$x \in (4,16) \cup [17,\infty)$\\
C.$x \in (4,16] \cup [17,\infty)$\\
D.$x \in [4,16) \cup [17,\infty)$\\
E.$x \in [4,16] \cup (17,\infty)$\\
F.$x \in (4,16) \cup (17,\infty)$\\
G.$x \in [4,16) \cup (17,\infty)$\\
H.$x \in (4,16] \cup (17,\infty)$
\testStop
\kluczStart
A
\kluczStop



\zadStart{Zadanie z Wikieł Z 1.62 a) moja wersja nr 572}

Rozwiązać nierówności $(x-4)(x-16)(x-18)\ge0$.
\zadStop
\rozwStart{Patryk Wirkus}{}
Miejsca zerowe naszego wielomianu to: $4, 16, 18$.\\
Wielomian jest stopnia nieparzystego, ponadto znak współczynnika przy\linebreak najwyższej potędze x jest dodatni.\\ W związku z tym wykres wielomianu zaczyna się od lewej strony poniżej osi OX. A więc $$x \in [4,16] \cup [18,\infty).$$
\rozwStop
\odpStart
$x \in [4,16] \cup [18,\infty)$
\odpStop
\testStart
A.$x \in [4,16] \cup [18,\infty)$\\
B.$x \in (4,16) \cup [18,\infty)$\\
C.$x \in (4,16] \cup [18,\infty)$\\
D.$x \in [4,16) \cup [18,\infty)$\\
E.$x \in [4,16] \cup (18,\infty)$\\
F.$x \in (4,16) \cup (18,\infty)$\\
G.$x \in [4,16) \cup (18,\infty)$\\
H.$x \in (4,16] \cup (18,\infty)$
\testStop
\kluczStart
A
\kluczStop



\zadStart{Zadanie z Wikieł Z 1.62 a) moja wersja nr 573}

Rozwiązać nierówności $(x-4)(x-16)(x-19)\ge0$.
\zadStop
\rozwStart{Patryk Wirkus}{}
Miejsca zerowe naszego wielomianu to: $4, 16, 19$.\\
Wielomian jest stopnia nieparzystego, ponadto znak współczynnika przy\linebreak najwyższej potędze x jest dodatni.\\ W związku z tym wykres wielomianu zaczyna się od lewej strony poniżej osi OX. A więc $$x \in [4,16] \cup [19,\infty).$$
\rozwStop
\odpStart
$x \in [4,16] \cup [19,\infty)$
\odpStop
\testStart
A.$x \in [4,16] \cup [19,\infty)$\\
B.$x \in (4,16) \cup [19,\infty)$\\
C.$x \in (4,16] \cup [19,\infty)$\\
D.$x \in [4,16) \cup [19,\infty)$\\
E.$x \in [4,16] \cup (19,\infty)$\\
F.$x \in (4,16) \cup (19,\infty)$\\
G.$x \in [4,16) \cup (19,\infty)$\\
H.$x \in (4,16] \cup (19,\infty)$
\testStop
\kluczStart
A
\kluczStop



\zadStart{Zadanie z Wikieł Z 1.62 a) moja wersja nr 574}

Rozwiązać nierówności $(x-4)(x-16)(x-20)\ge0$.
\zadStop
\rozwStart{Patryk Wirkus}{}
Miejsca zerowe naszego wielomianu to: $4, 16, 20$.\\
Wielomian jest stopnia nieparzystego, ponadto znak współczynnika przy\linebreak najwyższej potędze x jest dodatni.\\ W związku z tym wykres wielomianu zaczyna się od lewej strony poniżej osi OX. A więc $$x \in [4,16] \cup [20,\infty).$$
\rozwStop
\odpStart
$x \in [4,16] \cup [20,\infty)$
\odpStop
\testStart
A.$x \in [4,16] \cup [20,\infty)$\\
B.$x \in (4,16) \cup [20,\infty)$\\
C.$x \in (4,16] \cup [20,\infty)$\\
D.$x \in [4,16) \cup [20,\infty)$\\
E.$x \in [4,16] \cup (20,\infty)$\\
F.$x \in (4,16) \cup (20,\infty)$\\
G.$x \in [4,16) \cup (20,\infty)$\\
H.$x \in (4,16] \cup (20,\infty)$
\testStop
\kluczStart
A
\kluczStop



\zadStart{Zadanie z Wikieł Z 1.62 a) moja wersja nr 575}

Rozwiązać nierówności $(x-4)(x-17)(x-18)\ge0$.
\zadStop
\rozwStart{Patryk Wirkus}{}
Miejsca zerowe naszego wielomianu to: $4, 17, 18$.\\
Wielomian jest stopnia nieparzystego, ponadto znak współczynnika przy\linebreak najwyższej potędze x jest dodatni.\\ W związku z tym wykres wielomianu zaczyna się od lewej strony poniżej osi OX. A więc $$x \in [4,17] \cup [18,\infty).$$
\rozwStop
\odpStart
$x \in [4,17] \cup [18,\infty)$
\odpStop
\testStart
A.$x \in [4,17] \cup [18,\infty)$\\
B.$x \in (4,17) \cup [18,\infty)$\\
C.$x \in (4,17] \cup [18,\infty)$\\
D.$x \in [4,17) \cup [18,\infty)$\\
E.$x \in [4,17] \cup (18,\infty)$\\
F.$x \in (4,17) \cup (18,\infty)$\\
G.$x \in [4,17) \cup (18,\infty)$\\
H.$x \in (4,17] \cup (18,\infty)$
\testStop
\kluczStart
A
\kluczStop



\zadStart{Zadanie z Wikieł Z 1.62 a) moja wersja nr 576}

Rozwiązać nierówności $(x-4)(x-17)(x-19)\ge0$.
\zadStop
\rozwStart{Patryk Wirkus}{}
Miejsca zerowe naszego wielomianu to: $4, 17, 19$.\\
Wielomian jest stopnia nieparzystego, ponadto znak współczynnika przy\linebreak najwyższej potędze x jest dodatni.\\ W związku z tym wykres wielomianu zaczyna się od lewej strony poniżej osi OX. A więc $$x \in [4,17] \cup [19,\infty).$$
\rozwStop
\odpStart
$x \in [4,17] \cup [19,\infty)$
\odpStop
\testStart
A.$x \in [4,17] \cup [19,\infty)$\\
B.$x \in (4,17) \cup [19,\infty)$\\
C.$x \in (4,17] \cup [19,\infty)$\\
D.$x \in [4,17) \cup [19,\infty)$\\
E.$x \in [4,17] \cup (19,\infty)$\\
F.$x \in (4,17) \cup (19,\infty)$\\
G.$x \in [4,17) \cup (19,\infty)$\\
H.$x \in (4,17] \cup (19,\infty)$
\testStop
\kluczStart
A
\kluczStop



\zadStart{Zadanie z Wikieł Z 1.62 a) moja wersja nr 577}

Rozwiązać nierówności $(x-4)(x-17)(x-20)\ge0$.
\zadStop
\rozwStart{Patryk Wirkus}{}
Miejsca zerowe naszego wielomianu to: $4, 17, 20$.\\
Wielomian jest stopnia nieparzystego, ponadto znak współczynnika przy\linebreak najwyższej potędze x jest dodatni.\\ W związku z tym wykres wielomianu zaczyna się od lewej strony poniżej osi OX. A więc $$x \in [4,17] \cup [20,\infty).$$
\rozwStop
\odpStart
$x \in [4,17] \cup [20,\infty)$
\odpStop
\testStart
A.$x \in [4,17] \cup [20,\infty)$\\
B.$x \in (4,17) \cup [20,\infty)$\\
C.$x \in (4,17] \cup [20,\infty)$\\
D.$x \in [4,17) \cup [20,\infty)$\\
E.$x \in [4,17] \cup (20,\infty)$\\
F.$x \in (4,17) \cup (20,\infty)$\\
G.$x \in [4,17) \cup (20,\infty)$\\
H.$x \in (4,17] \cup (20,\infty)$
\testStop
\kluczStart
A
\kluczStop



\zadStart{Zadanie z Wikieł Z 1.62 a) moja wersja nr 578}

Rozwiązać nierówności $(x-4)(x-18)(x-19)\ge0$.
\zadStop
\rozwStart{Patryk Wirkus}{}
Miejsca zerowe naszego wielomianu to: $4, 18, 19$.\\
Wielomian jest stopnia nieparzystego, ponadto znak współczynnika przy\linebreak najwyższej potędze x jest dodatni.\\ W związku z tym wykres wielomianu zaczyna się od lewej strony poniżej osi OX. A więc $$x \in [4,18] \cup [19,\infty).$$
\rozwStop
\odpStart
$x \in [4,18] \cup [19,\infty)$
\odpStop
\testStart
A.$x \in [4,18] \cup [19,\infty)$\\
B.$x \in (4,18) \cup [19,\infty)$\\
C.$x \in (4,18] \cup [19,\infty)$\\
D.$x \in [4,18) \cup [19,\infty)$\\
E.$x \in [4,18] \cup (19,\infty)$\\
F.$x \in (4,18) \cup (19,\infty)$\\
G.$x \in [4,18) \cup (19,\infty)$\\
H.$x \in (4,18] \cup (19,\infty)$
\testStop
\kluczStart
A
\kluczStop



\zadStart{Zadanie z Wikieł Z 1.62 a) moja wersja nr 579}

Rozwiązać nierówności $(x-4)(x-18)(x-20)\ge0$.
\zadStop
\rozwStart{Patryk Wirkus}{}
Miejsca zerowe naszego wielomianu to: $4, 18, 20$.\\
Wielomian jest stopnia nieparzystego, ponadto znak współczynnika przy\linebreak najwyższej potędze x jest dodatni.\\ W związku z tym wykres wielomianu zaczyna się od lewej strony poniżej osi OX. A więc $$x \in [4,18] \cup [20,\infty).$$
\rozwStop
\odpStart
$x \in [4,18] \cup [20,\infty)$
\odpStop
\testStart
A.$x \in [4,18] \cup [20,\infty)$\\
B.$x \in (4,18) \cup [20,\infty)$\\
C.$x \in (4,18] \cup [20,\infty)$\\
D.$x \in [4,18) \cup [20,\infty)$\\
E.$x \in [4,18] \cup (20,\infty)$\\
F.$x \in (4,18) \cup (20,\infty)$\\
G.$x \in [4,18) \cup (20,\infty)$\\
H.$x \in (4,18] \cup (20,\infty)$
\testStop
\kluczStart
A
\kluczStop



\zadStart{Zadanie z Wikieł Z 1.62 a) moja wersja nr 580}

Rozwiązać nierówności $(x-4)(x-19)(x-20)\ge0$.
\zadStop
\rozwStart{Patryk Wirkus}{}
Miejsca zerowe naszego wielomianu to: $4, 19, 20$.\\
Wielomian jest stopnia nieparzystego, ponadto znak współczynnika przy\linebreak najwyższej potędze x jest dodatni.\\ W związku z tym wykres wielomianu zaczyna się od lewej strony poniżej osi OX. A więc $$x \in [4,19] \cup [20,\infty).$$
\rozwStop
\odpStart
$x \in [4,19] \cup [20,\infty)$
\odpStop
\testStart
A.$x \in [4,19] \cup [20,\infty)$\\
B.$x \in (4,19) \cup [20,\infty)$\\
C.$x \in (4,19] \cup [20,\infty)$\\
D.$x \in [4,19) \cup [20,\infty)$\\
E.$x \in [4,19] \cup (20,\infty)$\\
F.$x \in (4,19) \cup (20,\infty)$\\
G.$x \in [4,19) \cup (20,\infty)$\\
H.$x \in (4,19] \cup (20,\infty)$
\testStop
\kluczStart
A
\kluczStop



\zadStart{Zadanie z Wikieł Z 1.62 a) moja wersja nr 581}

Rozwiązać nierówności $(x-5)(x-6)(x-7)\ge0$.
\zadStop
\rozwStart{Patryk Wirkus}{}
Miejsca zerowe naszego wielomianu to: $5, 6, 7$.\\
Wielomian jest stopnia nieparzystego, ponadto znak współczynnika przy\linebreak najwyższej potędze x jest dodatni.\\ W związku z tym wykres wielomianu zaczyna się od lewej strony poniżej osi OX. A więc $$x \in [5,6] \cup [7,\infty).$$
\rozwStop
\odpStart
$x \in [5,6] \cup [7,\infty)$
\odpStop
\testStart
A.$x \in [5,6] \cup [7,\infty)$\\
B.$x \in (5,6) \cup [7,\infty)$\\
C.$x \in (5,6] \cup [7,\infty)$\\
D.$x \in [5,6) \cup [7,\infty)$\\
E.$x \in [5,6] \cup (7,\infty)$\\
F.$x \in (5,6) \cup (7,\infty)$\\
G.$x \in [5,6) \cup (7,\infty)$\\
H.$x \in (5,6] \cup (7,\infty)$
\testStop
\kluczStart
A
\kluczStop



\zadStart{Zadanie z Wikieł Z 1.62 a) moja wersja nr 582}

Rozwiązać nierówności $(x-5)(x-6)(x-8)\ge0$.
\zadStop
\rozwStart{Patryk Wirkus}{}
Miejsca zerowe naszego wielomianu to: $5, 6, 8$.\\
Wielomian jest stopnia nieparzystego, ponadto znak współczynnika przy\linebreak najwyższej potędze x jest dodatni.\\ W związku z tym wykres wielomianu zaczyna się od lewej strony poniżej osi OX. A więc $$x \in [5,6] \cup [8,\infty).$$
\rozwStop
\odpStart
$x \in [5,6] \cup [8,\infty)$
\odpStop
\testStart
A.$x \in [5,6] \cup [8,\infty)$\\
B.$x \in (5,6) \cup [8,\infty)$\\
C.$x \in (5,6] \cup [8,\infty)$\\
D.$x \in [5,6) \cup [8,\infty)$\\
E.$x \in [5,6] \cup (8,\infty)$\\
F.$x \in (5,6) \cup (8,\infty)$\\
G.$x \in [5,6) \cup (8,\infty)$\\
H.$x \in (5,6] \cup (8,\infty)$
\testStop
\kluczStart
A
\kluczStop



\zadStart{Zadanie z Wikieł Z 1.62 a) moja wersja nr 583}

Rozwiązać nierówności $(x-5)(x-6)(x-9)\ge0$.
\zadStop
\rozwStart{Patryk Wirkus}{}
Miejsca zerowe naszego wielomianu to: $5, 6, 9$.\\
Wielomian jest stopnia nieparzystego, ponadto znak współczynnika przy\linebreak najwyższej potędze x jest dodatni.\\ W związku z tym wykres wielomianu zaczyna się od lewej strony poniżej osi OX. A więc $$x \in [5,6] \cup [9,\infty).$$
\rozwStop
\odpStart
$x \in [5,6] \cup [9,\infty)$
\odpStop
\testStart
A.$x \in [5,6] \cup [9,\infty)$\\
B.$x \in (5,6) \cup [9,\infty)$\\
C.$x \in (5,6] \cup [9,\infty)$\\
D.$x \in [5,6) \cup [9,\infty)$\\
E.$x \in [5,6] \cup (9,\infty)$\\
F.$x \in (5,6) \cup (9,\infty)$\\
G.$x \in [5,6) \cup (9,\infty)$\\
H.$x \in (5,6] \cup (9,\infty)$
\testStop
\kluczStart
A
\kluczStop



\zadStart{Zadanie z Wikieł Z 1.62 a) moja wersja nr 584}

Rozwiązać nierówności $(x-5)(x-6)(x-10)\ge0$.
\zadStop
\rozwStart{Patryk Wirkus}{}
Miejsca zerowe naszego wielomianu to: $5, 6, 10$.\\
Wielomian jest stopnia nieparzystego, ponadto znak współczynnika przy\linebreak najwyższej potędze x jest dodatni.\\ W związku z tym wykres wielomianu zaczyna się od lewej strony poniżej osi OX. A więc $$x \in [5,6] \cup [10,\infty).$$
\rozwStop
\odpStart
$x \in [5,6] \cup [10,\infty)$
\odpStop
\testStart
A.$x \in [5,6] \cup [10,\infty)$\\
B.$x \in (5,6) \cup [10,\infty)$\\
C.$x \in (5,6] \cup [10,\infty)$\\
D.$x \in [5,6) \cup [10,\infty)$\\
E.$x \in [5,6] \cup (10,\infty)$\\
F.$x \in (5,6) \cup (10,\infty)$\\
G.$x \in [5,6) \cup (10,\infty)$\\
H.$x \in (5,6] \cup (10,\infty)$
\testStop
\kluczStart
A
\kluczStop



\zadStart{Zadanie z Wikieł Z 1.62 a) moja wersja nr 585}

Rozwiązać nierówności $(x-5)(x-6)(x-11)\ge0$.
\zadStop
\rozwStart{Patryk Wirkus}{}
Miejsca zerowe naszego wielomianu to: $5, 6, 11$.\\
Wielomian jest stopnia nieparzystego, ponadto znak współczynnika przy\linebreak najwyższej potędze x jest dodatni.\\ W związku z tym wykres wielomianu zaczyna się od lewej strony poniżej osi OX. A więc $$x \in [5,6] \cup [11,\infty).$$
\rozwStop
\odpStart
$x \in [5,6] \cup [11,\infty)$
\odpStop
\testStart
A.$x \in [5,6] \cup [11,\infty)$\\
B.$x \in (5,6) \cup [11,\infty)$\\
C.$x \in (5,6] \cup [11,\infty)$\\
D.$x \in [5,6) \cup [11,\infty)$\\
E.$x \in [5,6] \cup (11,\infty)$\\
F.$x \in (5,6) \cup (11,\infty)$\\
G.$x \in [5,6) \cup (11,\infty)$\\
H.$x \in (5,6] \cup (11,\infty)$
\testStop
\kluczStart
A
\kluczStop



\zadStart{Zadanie z Wikieł Z 1.62 a) moja wersja nr 586}

Rozwiązać nierówności $(x-5)(x-6)(x-12)\ge0$.
\zadStop
\rozwStart{Patryk Wirkus}{}
Miejsca zerowe naszego wielomianu to: $5, 6, 12$.\\
Wielomian jest stopnia nieparzystego, ponadto znak współczynnika przy\linebreak najwyższej potędze x jest dodatni.\\ W związku z tym wykres wielomianu zaczyna się od lewej strony poniżej osi OX. A więc $$x \in [5,6] \cup [12,\infty).$$
\rozwStop
\odpStart
$x \in [5,6] \cup [12,\infty)$
\odpStop
\testStart
A.$x \in [5,6] \cup [12,\infty)$\\
B.$x \in (5,6) \cup [12,\infty)$\\
C.$x \in (5,6] \cup [12,\infty)$\\
D.$x \in [5,6) \cup [12,\infty)$\\
E.$x \in [5,6] \cup (12,\infty)$\\
F.$x \in (5,6) \cup (12,\infty)$\\
G.$x \in [5,6) \cup (12,\infty)$\\
H.$x \in (5,6] \cup (12,\infty)$
\testStop
\kluczStart
A
\kluczStop



\zadStart{Zadanie z Wikieł Z 1.62 a) moja wersja nr 587}

Rozwiązać nierówności $(x-5)(x-6)(x-13)\ge0$.
\zadStop
\rozwStart{Patryk Wirkus}{}
Miejsca zerowe naszego wielomianu to: $5, 6, 13$.\\
Wielomian jest stopnia nieparzystego, ponadto znak współczynnika przy\linebreak najwyższej potędze x jest dodatni.\\ W związku z tym wykres wielomianu zaczyna się od lewej strony poniżej osi OX. A więc $$x \in [5,6] \cup [13,\infty).$$
\rozwStop
\odpStart
$x \in [5,6] \cup [13,\infty)$
\odpStop
\testStart
A.$x \in [5,6] \cup [13,\infty)$\\
B.$x \in (5,6) \cup [13,\infty)$\\
C.$x \in (5,6] \cup [13,\infty)$\\
D.$x \in [5,6) \cup [13,\infty)$\\
E.$x \in [5,6] \cup (13,\infty)$\\
F.$x \in (5,6) \cup (13,\infty)$\\
G.$x \in [5,6) \cup (13,\infty)$\\
H.$x \in (5,6] \cup (13,\infty)$
\testStop
\kluczStart
A
\kluczStop



\zadStart{Zadanie z Wikieł Z 1.62 a) moja wersja nr 588}

Rozwiązać nierówności $(x-5)(x-6)(x-14)\ge0$.
\zadStop
\rozwStart{Patryk Wirkus}{}
Miejsca zerowe naszego wielomianu to: $5, 6, 14$.\\
Wielomian jest stopnia nieparzystego, ponadto znak współczynnika przy\linebreak najwyższej potędze x jest dodatni.\\ W związku z tym wykres wielomianu zaczyna się od lewej strony poniżej osi OX. A więc $$x \in [5,6] \cup [14,\infty).$$
\rozwStop
\odpStart
$x \in [5,6] \cup [14,\infty)$
\odpStop
\testStart
A.$x \in [5,6] \cup [14,\infty)$\\
B.$x \in (5,6) \cup [14,\infty)$\\
C.$x \in (5,6] \cup [14,\infty)$\\
D.$x \in [5,6) \cup [14,\infty)$\\
E.$x \in [5,6] \cup (14,\infty)$\\
F.$x \in (5,6) \cup (14,\infty)$\\
G.$x \in [5,6) \cup (14,\infty)$\\
H.$x \in (5,6] \cup (14,\infty)$
\testStop
\kluczStart
A
\kluczStop



\zadStart{Zadanie z Wikieł Z 1.62 a) moja wersja nr 589}

Rozwiązać nierówności $(x-5)(x-6)(x-15)\ge0$.
\zadStop
\rozwStart{Patryk Wirkus}{}
Miejsca zerowe naszego wielomianu to: $5, 6, 15$.\\
Wielomian jest stopnia nieparzystego, ponadto znak współczynnika przy\linebreak najwyższej potędze x jest dodatni.\\ W związku z tym wykres wielomianu zaczyna się od lewej strony poniżej osi OX. A więc $$x \in [5,6] \cup [15,\infty).$$
\rozwStop
\odpStart
$x \in [5,6] \cup [15,\infty)$
\odpStop
\testStart
A.$x \in [5,6] \cup [15,\infty)$\\
B.$x \in (5,6) \cup [15,\infty)$\\
C.$x \in (5,6] \cup [15,\infty)$\\
D.$x \in [5,6) \cup [15,\infty)$\\
E.$x \in [5,6] \cup (15,\infty)$\\
F.$x \in (5,6) \cup (15,\infty)$\\
G.$x \in [5,6) \cup (15,\infty)$\\
H.$x \in (5,6] \cup (15,\infty)$
\testStop
\kluczStart
A
\kluczStop



\zadStart{Zadanie z Wikieł Z 1.62 a) moja wersja nr 590}

Rozwiązać nierówności $(x-5)(x-6)(x-16)\ge0$.
\zadStop
\rozwStart{Patryk Wirkus}{}
Miejsca zerowe naszego wielomianu to: $5, 6, 16$.\\
Wielomian jest stopnia nieparzystego, ponadto znak współczynnika przy\linebreak najwyższej potędze x jest dodatni.\\ W związku z tym wykres wielomianu zaczyna się od lewej strony poniżej osi OX. A więc $$x \in [5,6] \cup [16,\infty).$$
\rozwStop
\odpStart
$x \in [5,6] \cup [16,\infty)$
\odpStop
\testStart
A.$x \in [5,6] \cup [16,\infty)$\\
B.$x \in (5,6) \cup [16,\infty)$\\
C.$x \in (5,6] \cup [16,\infty)$\\
D.$x \in [5,6) \cup [16,\infty)$\\
E.$x \in [5,6] \cup (16,\infty)$\\
F.$x \in (5,6) \cup (16,\infty)$\\
G.$x \in [5,6) \cup (16,\infty)$\\
H.$x \in (5,6] \cup (16,\infty)$
\testStop
\kluczStart
A
\kluczStop



\zadStart{Zadanie z Wikieł Z 1.62 a) moja wersja nr 591}

Rozwiązać nierówności $(x-5)(x-6)(x-17)\ge0$.
\zadStop
\rozwStart{Patryk Wirkus}{}
Miejsca zerowe naszego wielomianu to: $5, 6, 17$.\\
Wielomian jest stopnia nieparzystego, ponadto znak współczynnika przy\linebreak najwyższej potędze x jest dodatni.\\ W związku z tym wykres wielomianu zaczyna się od lewej strony poniżej osi OX. A więc $$x \in [5,6] \cup [17,\infty).$$
\rozwStop
\odpStart
$x \in [5,6] \cup [17,\infty)$
\odpStop
\testStart
A.$x \in [5,6] \cup [17,\infty)$\\
B.$x \in (5,6) \cup [17,\infty)$\\
C.$x \in (5,6] \cup [17,\infty)$\\
D.$x \in [5,6) \cup [17,\infty)$\\
E.$x \in [5,6] \cup (17,\infty)$\\
F.$x \in (5,6) \cup (17,\infty)$\\
G.$x \in [5,6) \cup (17,\infty)$\\
H.$x \in (5,6] \cup (17,\infty)$
\testStop
\kluczStart
A
\kluczStop



\zadStart{Zadanie z Wikieł Z 1.62 a) moja wersja nr 592}

Rozwiązać nierówności $(x-5)(x-6)(x-18)\ge0$.
\zadStop
\rozwStart{Patryk Wirkus}{}
Miejsca zerowe naszego wielomianu to: $5, 6, 18$.\\
Wielomian jest stopnia nieparzystego, ponadto znak współczynnika przy\linebreak najwyższej potędze x jest dodatni.\\ W związku z tym wykres wielomianu zaczyna się od lewej strony poniżej osi OX. A więc $$x \in [5,6] \cup [18,\infty).$$
\rozwStop
\odpStart
$x \in [5,6] \cup [18,\infty)$
\odpStop
\testStart
A.$x \in [5,6] \cup [18,\infty)$\\
B.$x \in (5,6) \cup [18,\infty)$\\
C.$x \in (5,6] \cup [18,\infty)$\\
D.$x \in [5,6) \cup [18,\infty)$\\
E.$x \in [5,6] \cup (18,\infty)$\\
F.$x \in (5,6) \cup (18,\infty)$\\
G.$x \in [5,6) \cup (18,\infty)$\\
H.$x \in (5,6] \cup (18,\infty)$
\testStop
\kluczStart
A
\kluczStop



\zadStart{Zadanie z Wikieł Z 1.62 a) moja wersja nr 593}

Rozwiązać nierówności $(x-5)(x-6)(x-19)\ge0$.
\zadStop
\rozwStart{Patryk Wirkus}{}
Miejsca zerowe naszego wielomianu to: $5, 6, 19$.\\
Wielomian jest stopnia nieparzystego, ponadto znak współczynnika przy\linebreak najwyższej potędze x jest dodatni.\\ W związku z tym wykres wielomianu zaczyna się od lewej strony poniżej osi OX. A więc $$x \in [5,6] \cup [19,\infty).$$
\rozwStop
\odpStart
$x \in [5,6] \cup [19,\infty)$
\odpStop
\testStart
A.$x \in [5,6] \cup [19,\infty)$\\
B.$x \in (5,6) \cup [19,\infty)$\\
C.$x \in (5,6] \cup [19,\infty)$\\
D.$x \in [5,6) \cup [19,\infty)$\\
E.$x \in [5,6] \cup (19,\infty)$\\
F.$x \in (5,6) \cup (19,\infty)$\\
G.$x \in [5,6) \cup (19,\infty)$\\
H.$x \in (5,6] \cup (19,\infty)$
\testStop
\kluczStart
A
\kluczStop



\zadStart{Zadanie z Wikieł Z 1.62 a) moja wersja nr 594}

Rozwiązać nierówności $(x-5)(x-6)(x-20)\ge0$.
\zadStop
\rozwStart{Patryk Wirkus}{}
Miejsca zerowe naszego wielomianu to: $5, 6, 20$.\\
Wielomian jest stopnia nieparzystego, ponadto znak współczynnika przy\linebreak najwyższej potędze x jest dodatni.\\ W związku z tym wykres wielomianu zaczyna się od lewej strony poniżej osi OX. A więc $$x \in [5,6] \cup [20,\infty).$$
\rozwStop
\odpStart
$x \in [5,6] \cup [20,\infty)$
\odpStop
\testStart
A.$x \in [5,6] \cup [20,\infty)$\\
B.$x \in (5,6) \cup [20,\infty)$\\
C.$x \in (5,6] \cup [20,\infty)$\\
D.$x \in [5,6) \cup [20,\infty)$\\
E.$x \in [5,6] \cup (20,\infty)$\\
F.$x \in (5,6) \cup (20,\infty)$\\
G.$x \in [5,6) \cup (20,\infty)$\\
H.$x \in (5,6] \cup (20,\infty)$
\testStop
\kluczStart
A
\kluczStop



\zadStart{Zadanie z Wikieł Z 1.62 a) moja wersja nr 595}

Rozwiązać nierówności $(x-5)(x-7)(x-8)\ge0$.
\zadStop
\rozwStart{Patryk Wirkus}{}
Miejsca zerowe naszego wielomianu to: $5, 7, 8$.\\
Wielomian jest stopnia nieparzystego, ponadto znak współczynnika przy\linebreak najwyższej potędze x jest dodatni.\\ W związku z tym wykres wielomianu zaczyna się od lewej strony poniżej osi OX. A więc $$x \in [5,7] \cup [8,\infty).$$
\rozwStop
\odpStart
$x \in [5,7] \cup [8,\infty)$
\odpStop
\testStart
A.$x \in [5,7] \cup [8,\infty)$\\
B.$x \in (5,7) \cup [8,\infty)$\\
C.$x \in (5,7] \cup [8,\infty)$\\
D.$x \in [5,7) \cup [8,\infty)$\\
E.$x \in [5,7] \cup (8,\infty)$\\
F.$x \in (5,7) \cup (8,\infty)$\\
G.$x \in [5,7) \cup (8,\infty)$\\
H.$x \in (5,7] \cup (8,\infty)$
\testStop
\kluczStart
A
\kluczStop



\zadStart{Zadanie z Wikieł Z 1.62 a) moja wersja nr 596}

Rozwiązać nierówności $(x-5)(x-7)(x-9)\ge0$.
\zadStop
\rozwStart{Patryk Wirkus}{}
Miejsca zerowe naszego wielomianu to: $5, 7, 9$.\\
Wielomian jest stopnia nieparzystego, ponadto znak współczynnika przy\linebreak najwyższej potędze x jest dodatni.\\ W związku z tym wykres wielomianu zaczyna się od lewej strony poniżej osi OX. A więc $$x \in [5,7] \cup [9,\infty).$$
\rozwStop
\odpStart
$x \in [5,7] \cup [9,\infty)$
\odpStop
\testStart
A.$x \in [5,7] \cup [9,\infty)$\\
B.$x \in (5,7) \cup [9,\infty)$\\
C.$x \in (5,7] \cup [9,\infty)$\\
D.$x \in [5,7) \cup [9,\infty)$\\
E.$x \in [5,7] \cup (9,\infty)$\\
F.$x \in (5,7) \cup (9,\infty)$\\
G.$x \in [5,7) \cup (9,\infty)$\\
H.$x \in (5,7] \cup (9,\infty)$
\testStop
\kluczStart
A
\kluczStop



\zadStart{Zadanie z Wikieł Z 1.62 a) moja wersja nr 597}

Rozwiązać nierówności $(x-5)(x-7)(x-10)\ge0$.
\zadStop
\rozwStart{Patryk Wirkus}{}
Miejsca zerowe naszego wielomianu to: $5, 7, 10$.\\
Wielomian jest stopnia nieparzystego, ponadto znak współczynnika przy\linebreak najwyższej potędze x jest dodatni.\\ W związku z tym wykres wielomianu zaczyna się od lewej strony poniżej osi OX. A więc $$x \in [5,7] \cup [10,\infty).$$
\rozwStop
\odpStart
$x \in [5,7] \cup [10,\infty)$
\odpStop
\testStart
A.$x \in [5,7] \cup [10,\infty)$\\
B.$x \in (5,7) \cup [10,\infty)$\\
C.$x \in (5,7] \cup [10,\infty)$\\
D.$x \in [5,7) \cup [10,\infty)$\\
E.$x \in [5,7] \cup (10,\infty)$\\
F.$x \in (5,7) \cup (10,\infty)$\\
G.$x \in [5,7) \cup (10,\infty)$\\
H.$x \in (5,7] \cup (10,\infty)$
\testStop
\kluczStart
A
\kluczStop



\zadStart{Zadanie z Wikieł Z 1.62 a) moja wersja nr 598}

Rozwiązać nierówności $(x-5)(x-7)(x-11)\ge0$.
\zadStop
\rozwStart{Patryk Wirkus}{}
Miejsca zerowe naszego wielomianu to: $5, 7, 11$.\\
Wielomian jest stopnia nieparzystego, ponadto znak współczynnika przy\linebreak najwyższej potędze x jest dodatni.\\ W związku z tym wykres wielomianu zaczyna się od lewej strony poniżej osi OX. A więc $$x \in [5,7] \cup [11,\infty).$$
\rozwStop
\odpStart
$x \in [5,7] \cup [11,\infty)$
\odpStop
\testStart
A.$x \in [5,7] \cup [11,\infty)$\\
B.$x \in (5,7) \cup [11,\infty)$\\
C.$x \in (5,7] \cup [11,\infty)$\\
D.$x \in [5,7) \cup [11,\infty)$\\
E.$x \in [5,7] \cup (11,\infty)$\\
F.$x \in (5,7) \cup (11,\infty)$\\
G.$x \in [5,7) \cup (11,\infty)$\\
H.$x \in (5,7] \cup (11,\infty)$
\testStop
\kluczStart
A
\kluczStop



\zadStart{Zadanie z Wikieł Z 1.62 a) moja wersja nr 599}

Rozwiązać nierówności $(x-5)(x-7)(x-12)\ge0$.
\zadStop
\rozwStart{Patryk Wirkus}{}
Miejsca zerowe naszego wielomianu to: $5, 7, 12$.\\
Wielomian jest stopnia nieparzystego, ponadto znak współczynnika przy\linebreak najwyższej potędze x jest dodatni.\\ W związku z tym wykres wielomianu zaczyna się od lewej strony poniżej osi OX. A więc $$x \in [5,7] \cup [12,\infty).$$
\rozwStop
\odpStart
$x \in [5,7] \cup [12,\infty)$
\odpStop
\testStart
A.$x \in [5,7] \cup [12,\infty)$\\
B.$x \in (5,7) \cup [12,\infty)$\\
C.$x \in (5,7] \cup [12,\infty)$\\
D.$x \in [5,7) \cup [12,\infty)$\\
E.$x \in [5,7] \cup (12,\infty)$\\
F.$x \in (5,7) \cup (12,\infty)$\\
G.$x \in [5,7) \cup (12,\infty)$\\
H.$x \in (5,7] \cup (12,\infty)$
\testStop
\kluczStart
A
\kluczStop



\zadStart{Zadanie z Wikieł Z 1.62 a) moja wersja nr 600}

Rozwiązać nierówności $(x-5)(x-7)(x-13)\ge0$.
\zadStop
\rozwStart{Patryk Wirkus}{}
Miejsca zerowe naszego wielomianu to: $5, 7, 13$.\\
Wielomian jest stopnia nieparzystego, ponadto znak współczynnika przy\linebreak najwyższej potędze x jest dodatni.\\ W związku z tym wykres wielomianu zaczyna się od lewej strony poniżej osi OX. A więc $$x \in [5,7] \cup [13,\infty).$$
\rozwStop
\odpStart
$x \in [5,7] \cup [13,\infty)$
\odpStop
\testStart
A.$x \in [5,7] \cup [13,\infty)$\\
B.$x \in (5,7) \cup [13,\infty)$\\
C.$x \in (5,7] \cup [13,\infty)$\\
D.$x \in [5,7) \cup [13,\infty)$\\
E.$x \in [5,7] \cup (13,\infty)$\\
F.$x \in (5,7) \cup (13,\infty)$\\
G.$x \in [5,7) \cup (13,\infty)$\\
H.$x \in (5,7] \cup (13,\infty)$
\testStop
\kluczStart
A
\kluczStop



\zadStart{Zadanie z Wikieł Z 1.62 a) moja wersja nr 601}

Rozwiązać nierówności $(x-5)(x-7)(x-14)\ge0$.
\zadStop
\rozwStart{Patryk Wirkus}{}
Miejsca zerowe naszego wielomianu to: $5, 7, 14$.\\
Wielomian jest stopnia nieparzystego, ponadto znak współczynnika przy\linebreak najwyższej potędze x jest dodatni.\\ W związku z tym wykres wielomianu zaczyna się od lewej strony poniżej osi OX. A więc $$x \in [5,7] \cup [14,\infty).$$
\rozwStop
\odpStart
$x \in [5,7] \cup [14,\infty)$
\odpStop
\testStart
A.$x \in [5,7] \cup [14,\infty)$\\
B.$x \in (5,7) \cup [14,\infty)$\\
C.$x \in (5,7] \cup [14,\infty)$\\
D.$x \in [5,7) \cup [14,\infty)$\\
E.$x \in [5,7] \cup (14,\infty)$\\
F.$x \in (5,7) \cup (14,\infty)$\\
G.$x \in [5,7) \cup (14,\infty)$\\
H.$x \in (5,7] \cup (14,\infty)$
\testStop
\kluczStart
A
\kluczStop



\zadStart{Zadanie z Wikieł Z 1.62 a) moja wersja nr 602}

Rozwiązać nierówności $(x-5)(x-7)(x-15)\ge0$.
\zadStop
\rozwStart{Patryk Wirkus}{}
Miejsca zerowe naszego wielomianu to: $5, 7, 15$.\\
Wielomian jest stopnia nieparzystego, ponadto znak współczynnika przy\linebreak najwyższej potędze x jest dodatni.\\ W związku z tym wykres wielomianu zaczyna się od lewej strony poniżej osi OX. A więc $$x \in [5,7] \cup [15,\infty).$$
\rozwStop
\odpStart
$x \in [5,7] \cup [15,\infty)$
\odpStop
\testStart
A.$x \in [5,7] \cup [15,\infty)$\\
B.$x \in (5,7) \cup [15,\infty)$\\
C.$x \in (5,7] \cup [15,\infty)$\\
D.$x \in [5,7) \cup [15,\infty)$\\
E.$x \in [5,7] \cup (15,\infty)$\\
F.$x \in (5,7) \cup (15,\infty)$\\
G.$x \in [5,7) \cup (15,\infty)$\\
H.$x \in (5,7] \cup (15,\infty)$
\testStop
\kluczStart
A
\kluczStop



\zadStart{Zadanie z Wikieł Z 1.62 a) moja wersja nr 603}

Rozwiązać nierówności $(x-5)(x-7)(x-16)\ge0$.
\zadStop
\rozwStart{Patryk Wirkus}{}
Miejsca zerowe naszego wielomianu to: $5, 7, 16$.\\
Wielomian jest stopnia nieparzystego, ponadto znak współczynnika przy\linebreak najwyższej potędze x jest dodatni.\\ W związku z tym wykres wielomianu zaczyna się od lewej strony poniżej osi OX. A więc $$x \in [5,7] \cup [16,\infty).$$
\rozwStop
\odpStart
$x \in [5,7] \cup [16,\infty)$
\odpStop
\testStart
A.$x \in [5,7] \cup [16,\infty)$\\
B.$x \in (5,7) \cup [16,\infty)$\\
C.$x \in (5,7] \cup [16,\infty)$\\
D.$x \in [5,7) \cup [16,\infty)$\\
E.$x \in [5,7] \cup (16,\infty)$\\
F.$x \in (5,7) \cup (16,\infty)$\\
G.$x \in [5,7) \cup (16,\infty)$\\
H.$x \in (5,7] \cup (16,\infty)$
\testStop
\kluczStart
A
\kluczStop



\zadStart{Zadanie z Wikieł Z 1.62 a) moja wersja nr 604}

Rozwiązać nierówności $(x-5)(x-7)(x-17)\ge0$.
\zadStop
\rozwStart{Patryk Wirkus}{}
Miejsca zerowe naszego wielomianu to: $5, 7, 17$.\\
Wielomian jest stopnia nieparzystego, ponadto znak współczynnika przy\linebreak najwyższej potędze x jest dodatni.\\ W związku z tym wykres wielomianu zaczyna się od lewej strony poniżej osi OX. A więc $$x \in [5,7] \cup [17,\infty).$$
\rozwStop
\odpStart
$x \in [5,7] \cup [17,\infty)$
\odpStop
\testStart
A.$x \in [5,7] \cup [17,\infty)$\\
B.$x \in (5,7) \cup [17,\infty)$\\
C.$x \in (5,7] \cup [17,\infty)$\\
D.$x \in [5,7) \cup [17,\infty)$\\
E.$x \in [5,7] \cup (17,\infty)$\\
F.$x \in (5,7) \cup (17,\infty)$\\
G.$x \in [5,7) \cup (17,\infty)$\\
H.$x \in (5,7] \cup (17,\infty)$
\testStop
\kluczStart
A
\kluczStop



\zadStart{Zadanie z Wikieł Z 1.62 a) moja wersja nr 605}

Rozwiązać nierówności $(x-5)(x-7)(x-18)\ge0$.
\zadStop
\rozwStart{Patryk Wirkus}{}
Miejsca zerowe naszego wielomianu to: $5, 7, 18$.\\
Wielomian jest stopnia nieparzystego, ponadto znak współczynnika przy\linebreak najwyższej potędze x jest dodatni.\\ W związku z tym wykres wielomianu zaczyna się od lewej strony poniżej osi OX. A więc $$x \in [5,7] \cup [18,\infty).$$
\rozwStop
\odpStart
$x \in [5,7] \cup [18,\infty)$
\odpStop
\testStart
A.$x \in [5,7] \cup [18,\infty)$\\
B.$x \in (5,7) \cup [18,\infty)$\\
C.$x \in (5,7] \cup [18,\infty)$\\
D.$x \in [5,7) \cup [18,\infty)$\\
E.$x \in [5,7] \cup (18,\infty)$\\
F.$x \in (5,7) \cup (18,\infty)$\\
G.$x \in [5,7) \cup (18,\infty)$\\
H.$x \in (5,7] \cup (18,\infty)$
\testStop
\kluczStart
A
\kluczStop



\zadStart{Zadanie z Wikieł Z 1.62 a) moja wersja nr 606}

Rozwiązać nierówności $(x-5)(x-7)(x-19)\ge0$.
\zadStop
\rozwStart{Patryk Wirkus}{}
Miejsca zerowe naszego wielomianu to: $5, 7, 19$.\\
Wielomian jest stopnia nieparzystego, ponadto znak współczynnika przy\linebreak najwyższej potędze x jest dodatni.\\ W związku z tym wykres wielomianu zaczyna się od lewej strony poniżej osi OX. A więc $$x \in [5,7] \cup [19,\infty).$$
\rozwStop
\odpStart
$x \in [5,7] \cup [19,\infty)$
\odpStop
\testStart
A.$x \in [5,7] \cup [19,\infty)$\\
B.$x \in (5,7) \cup [19,\infty)$\\
C.$x \in (5,7] \cup [19,\infty)$\\
D.$x \in [5,7) \cup [19,\infty)$\\
E.$x \in [5,7] \cup (19,\infty)$\\
F.$x \in (5,7) \cup (19,\infty)$\\
G.$x \in [5,7) \cup (19,\infty)$\\
H.$x \in (5,7] \cup (19,\infty)$
\testStop
\kluczStart
A
\kluczStop



\zadStart{Zadanie z Wikieł Z 1.62 a) moja wersja nr 607}

Rozwiązać nierówności $(x-5)(x-7)(x-20)\ge0$.
\zadStop
\rozwStart{Patryk Wirkus}{}
Miejsca zerowe naszego wielomianu to: $5, 7, 20$.\\
Wielomian jest stopnia nieparzystego, ponadto znak współczynnika przy\linebreak najwyższej potędze x jest dodatni.\\ W związku z tym wykres wielomianu zaczyna się od lewej strony poniżej osi OX. A więc $$x \in [5,7] \cup [20,\infty).$$
\rozwStop
\odpStart
$x \in [5,7] \cup [20,\infty)$
\odpStop
\testStart
A.$x \in [5,7] \cup [20,\infty)$\\
B.$x \in (5,7) \cup [20,\infty)$\\
C.$x \in (5,7] \cup [20,\infty)$\\
D.$x \in [5,7) \cup [20,\infty)$\\
E.$x \in [5,7] \cup (20,\infty)$\\
F.$x \in (5,7) \cup (20,\infty)$\\
G.$x \in [5,7) \cup (20,\infty)$\\
H.$x \in (5,7] \cup (20,\infty)$
\testStop
\kluczStart
A
\kluczStop



\zadStart{Zadanie z Wikieł Z 1.62 a) moja wersja nr 608}

Rozwiązać nierówności $(x-5)(x-8)(x-9)\ge0$.
\zadStop
\rozwStart{Patryk Wirkus}{}
Miejsca zerowe naszego wielomianu to: $5, 8, 9$.\\
Wielomian jest stopnia nieparzystego, ponadto znak współczynnika przy\linebreak najwyższej potędze x jest dodatni.\\ W związku z tym wykres wielomianu zaczyna się od lewej strony poniżej osi OX. A więc $$x \in [5,8] \cup [9,\infty).$$
\rozwStop
\odpStart
$x \in [5,8] \cup [9,\infty)$
\odpStop
\testStart
A.$x \in [5,8] \cup [9,\infty)$\\
B.$x \in (5,8) \cup [9,\infty)$\\
C.$x \in (5,8] \cup [9,\infty)$\\
D.$x \in [5,8) \cup [9,\infty)$\\
E.$x \in [5,8] \cup (9,\infty)$\\
F.$x \in (5,8) \cup (9,\infty)$\\
G.$x \in [5,8) \cup (9,\infty)$\\
H.$x \in (5,8] \cup (9,\infty)$
\testStop
\kluczStart
A
\kluczStop



\zadStart{Zadanie z Wikieł Z 1.62 a) moja wersja nr 609}

Rozwiązać nierówności $(x-5)(x-8)(x-10)\ge0$.
\zadStop
\rozwStart{Patryk Wirkus}{}
Miejsca zerowe naszego wielomianu to: $5, 8, 10$.\\
Wielomian jest stopnia nieparzystego, ponadto znak współczynnika przy\linebreak najwyższej potędze x jest dodatni.\\ W związku z tym wykres wielomianu zaczyna się od lewej strony poniżej osi OX. A więc $$x \in [5,8] \cup [10,\infty).$$
\rozwStop
\odpStart
$x \in [5,8] \cup [10,\infty)$
\odpStop
\testStart
A.$x \in [5,8] \cup [10,\infty)$\\
B.$x \in (5,8) \cup [10,\infty)$\\
C.$x \in (5,8] \cup [10,\infty)$\\
D.$x \in [5,8) \cup [10,\infty)$\\
E.$x \in [5,8] \cup (10,\infty)$\\
F.$x \in (5,8) \cup (10,\infty)$\\
G.$x \in [5,8) \cup (10,\infty)$\\
H.$x \in (5,8] \cup (10,\infty)$
\testStop
\kluczStart
A
\kluczStop



\zadStart{Zadanie z Wikieł Z 1.62 a) moja wersja nr 610}

Rozwiązać nierówności $(x-5)(x-8)(x-11)\ge0$.
\zadStop
\rozwStart{Patryk Wirkus}{}
Miejsca zerowe naszego wielomianu to: $5, 8, 11$.\\
Wielomian jest stopnia nieparzystego, ponadto znak współczynnika przy\linebreak najwyższej potędze x jest dodatni.\\ W związku z tym wykres wielomianu zaczyna się od lewej strony poniżej osi OX. A więc $$x \in [5,8] \cup [11,\infty).$$
\rozwStop
\odpStart
$x \in [5,8] \cup [11,\infty)$
\odpStop
\testStart
A.$x \in [5,8] \cup [11,\infty)$\\
B.$x \in (5,8) \cup [11,\infty)$\\
C.$x \in (5,8] \cup [11,\infty)$\\
D.$x \in [5,8) \cup [11,\infty)$\\
E.$x \in [5,8] \cup (11,\infty)$\\
F.$x \in (5,8) \cup (11,\infty)$\\
G.$x \in [5,8) \cup (11,\infty)$\\
H.$x \in (5,8] \cup (11,\infty)$
\testStop
\kluczStart
A
\kluczStop



\zadStart{Zadanie z Wikieł Z 1.62 a) moja wersja nr 611}

Rozwiązać nierówności $(x-5)(x-8)(x-12)\ge0$.
\zadStop
\rozwStart{Patryk Wirkus}{}
Miejsca zerowe naszego wielomianu to: $5, 8, 12$.\\
Wielomian jest stopnia nieparzystego, ponadto znak współczynnika przy\linebreak najwyższej potędze x jest dodatni.\\ W związku z tym wykres wielomianu zaczyna się od lewej strony poniżej osi OX. A więc $$x \in [5,8] \cup [12,\infty).$$
\rozwStop
\odpStart
$x \in [5,8] \cup [12,\infty)$
\odpStop
\testStart
A.$x \in [5,8] \cup [12,\infty)$\\
B.$x \in (5,8) \cup [12,\infty)$\\
C.$x \in (5,8] \cup [12,\infty)$\\
D.$x \in [5,8) \cup [12,\infty)$\\
E.$x \in [5,8] \cup (12,\infty)$\\
F.$x \in (5,8) \cup (12,\infty)$\\
G.$x \in [5,8) \cup (12,\infty)$\\
H.$x \in (5,8] \cup (12,\infty)$
\testStop
\kluczStart
A
\kluczStop



\zadStart{Zadanie z Wikieł Z 1.62 a) moja wersja nr 612}

Rozwiązać nierówności $(x-5)(x-8)(x-13)\ge0$.
\zadStop
\rozwStart{Patryk Wirkus}{}
Miejsca zerowe naszego wielomianu to: $5, 8, 13$.\\
Wielomian jest stopnia nieparzystego, ponadto znak współczynnika przy\linebreak najwyższej potędze x jest dodatni.\\ W związku z tym wykres wielomianu zaczyna się od lewej strony poniżej osi OX. A więc $$x \in [5,8] \cup [13,\infty).$$
\rozwStop
\odpStart
$x \in [5,8] \cup [13,\infty)$
\odpStop
\testStart
A.$x \in [5,8] \cup [13,\infty)$\\
B.$x \in (5,8) \cup [13,\infty)$\\
C.$x \in (5,8] \cup [13,\infty)$\\
D.$x \in [5,8) \cup [13,\infty)$\\
E.$x \in [5,8] \cup (13,\infty)$\\
F.$x \in (5,8) \cup (13,\infty)$\\
G.$x \in [5,8) \cup (13,\infty)$\\
H.$x \in (5,8] \cup (13,\infty)$
\testStop
\kluczStart
A
\kluczStop



\zadStart{Zadanie z Wikieł Z 1.62 a) moja wersja nr 613}

Rozwiązać nierówności $(x-5)(x-8)(x-14)\ge0$.
\zadStop
\rozwStart{Patryk Wirkus}{}
Miejsca zerowe naszego wielomianu to: $5, 8, 14$.\\
Wielomian jest stopnia nieparzystego, ponadto znak współczynnika przy\linebreak najwyższej potędze x jest dodatni.\\ W związku z tym wykres wielomianu zaczyna się od lewej strony poniżej osi OX. A więc $$x \in [5,8] \cup [14,\infty).$$
\rozwStop
\odpStart
$x \in [5,8] \cup [14,\infty)$
\odpStop
\testStart
A.$x \in [5,8] \cup [14,\infty)$\\
B.$x \in (5,8) \cup [14,\infty)$\\
C.$x \in (5,8] \cup [14,\infty)$\\
D.$x \in [5,8) \cup [14,\infty)$\\
E.$x \in [5,8] \cup (14,\infty)$\\
F.$x \in (5,8) \cup (14,\infty)$\\
G.$x \in [5,8) \cup (14,\infty)$\\
H.$x \in (5,8] \cup (14,\infty)$
\testStop
\kluczStart
A
\kluczStop



\zadStart{Zadanie z Wikieł Z 1.62 a) moja wersja nr 614}

Rozwiązać nierówności $(x-5)(x-8)(x-15)\ge0$.
\zadStop
\rozwStart{Patryk Wirkus}{}
Miejsca zerowe naszego wielomianu to: $5, 8, 15$.\\
Wielomian jest stopnia nieparzystego, ponadto znak współczynnika przy\linebreak najwyższej potędze x jest dodatni.\\ W związku z tym wykres wielomianu zaczyna się od lewej strony poniżej osi OX. A więc $$x \in [5,8] \cup [15,\infty).$$
\rozwStop
\odpStart
$x \in [5,8] \cup [15,\infty)$
\odpStop
\testStart
A.$x \in [5,8] \cup [15,\infty)$\\
B.$x \in (5,8) \cup [15,\infty)$\\
C.$x \in (5,8] \cup [15,\infty)$\\
D.$x \in [5,8) \cup [15,\infty)$\\
E.$x \in [5,8] \cup (15,\infty)$\\
F.$x \in (5,8) \cup (15,\infty)$\\
G.$x \in [5,8) \cup (15,\infty)$\\
H.$x \in (5,8] \cup (15,\infty)$
\testStop
\kluczStart
A
\kluczStop



\zadStart{Zadanie z Wikieł Z 1.62 a) moja wersja nr 615}

Rozwiązać nierówności $(x-5)(x-8)(x-16)\ge0$.
\zadStop
\rozwStart{Patryk Wirkus}{}
Miejsca zerowe naszego wielomianu to: $5, 8, 16$.\\
Wielomian jest stopnia nieparzystego, ponadto znak współczynnika przy\linebreak najwyższej potędze x jest dodatni.\\ W związku z tym wykres wielomianu zaczyna się od lewej strony poniżej osi OX. A więc $$x \in [5,8] \cup [16,\infty).$$
\rozwStop
\odpStart
$x \in [5,8] \cup [16,\infty)$
\odpStop
\testStart
A.$x \in [5,8] \cup [16,\infty)$\\
B.$x \in (5,8) \cup [16,\infty)$\\
C.$x \in (5,8] \cup [16,\infty)$\\
D.$x \in [5,8) \cup [16,\infty)$\\
E.$x \in [5,8] \cup (16,\infty)$\\
F.$x \in (5,8) \cup (16,\infty)$\\
G.$x \in [5,8) \cup (16,\infty)$\\
H.$x \in (5,8] \cup (16,\infty)$
\testStop
\kluczStart
A
\kluczStop



\zadStart{Zadanie z Wikieł Z 1.62 a) moja wersja nr 616}

Rozwiązać nierówności $(x-5)(x-8)(x-17)\ge0$.
\zadStop
\rozwStart{Patryk Wirkus}{}
Miejsca zerowe naszego wielomianu to: $5, 8, 17$.\\
Wielomian jest stopnia nieparzystego, ponadto znak współczynnika przy\linebreak najwyższej potędze x jest dodatni.\\ W związku z tym wykres wielomianu zaczyna się od lewej strony poniżej osi OX. A więc $$x \in [5,8] \cup [17,\infty).$$
\rozwStop
\odpStart
$x \in [5,8] \cup [17,\infty)$
\odpStop
\testStart
A.$x \in [5,8] \cup [17,\infty)$\\
B.$x \in (5,8) \cup [17,\infty)$\\
C.$x \in (5,8] \cup [17,\infty)$\\
D.$x \in [5,8) \cup [17,\infty)$\\
E.$x \in [5,8] \cup (17,\infty)$\\
F.$x \in (5,8) \cup (17,\infty)$\\
G.$x \in [5,8) \cup (17,\infty)$\\
H.$x \in (5,8] \cup (17,\infty)$
\testStop
\kluczStart
A
\kluczStop



\zadStart{Zadanie z Wikieł Z 1.62 a) moja wersja nr 617}

Rozwiązać nierówności $(x-5)(x-8)(x-18)\ge0$.
\zadStop
\rozwStart{Patryk Wirkus}{}
Miejsca zerowe naszego wielomianu to: $5, 8, 18$.\\
Wielomian jest stopnia nieparzystego, ponadto znak współczynnika przy\linebreak najwyższej potędze x jest dodatni.\\ W związku z tym wykres wielomianu zaczyna się od lewej strony poniżej osi OX. A więc $$x \in [5,8] \cup [18,\infty).$$
\rozwStop
\odpStart
$x \in [5,8] \cup [18,\infty)$
\odpStop
\testStart
A.$x \in [5,8] \cup [18,\infty)$\\
B.$x \in (5,8) \cup [18,\infty)$\\
C.$x \in (5,8] \cup [18,\infty)$\\
D.$x \in [5,8) \cup [18,\infty)$\\
E.$x \in [5,8] \cup (18,\infty)$\\
F.$x \in (5,8) \cup (18,\infty)$\\
G.$x \in [5,8) \cup (18,\infty)$\\
H.$x \in (5,8] \cup (18,\infty)$
\testStop
\kluczStart
A
\kluczStop



\zadStart{Zadanie z Wikieł Z 1.62 a) moja wersja nr 618}

Rozwiązać nierówności $(x-5)(x-8)(x-19)\ge0$.
\zadStop
\rozwStart{Patryk Wirkus}{}
Miejsca zerowe naszego wielomianu to: $5, 8, 19$.\\
Wielomian jest stopnia nieparzystego, ponadto znak współczynnika przy\linebreak najwyższej potędze x jest dodatni.\\ W związku z tym wykres wielomianu zaczyna się od lewej strony poniżej osi OX. A więc $$x \in [5,8] \cup [19,\infty).$$
\rozwStop
\odpStart
$x \in [5,8] \cup [19,\infty)$
\odpStop
\testStart
A.$x \in [5,8] \cup [19,\infty)$\\
B.$x \in (5,8) \cup [19,\infty)$\\
C.$x \in (5,8] \cup [19,\infty)$\\
D.$x \in [5,8) \cup [19,\infty)$\\
E.$x \in [5,8] \cup (19,\infty)$\\
F.$x \in (5,8) \cup (19,\infty)$\\
G.$x \in [5,8) \cup (19,\infty)$\\
H.$x \in (5,8] \cup (19,\infty)$
\testStop
\kluczStart
A
\kluczStop



\zadStart{Zadanie z Wikieł Z 1.62 a) moja wersja nr 619}

Rozwiązać nierówności $(x-5)(x-8)(x-20)\ge0$.
\zadStop
\rozwStart{Patryk Wirkus}{}
Miejsca zerowe naszego wielomianu to: $5, 8, 20$.\\
Wielomian jest stopnia nieparzystego, ponadto znak współczynnika przy\linebreak najwyższej potędze x jest dodatni.\\ W związku z tym wykres wielomianu zaczyna się od lewej strony poniżej osi OX. A więc $$x \in [5,8] \cup [20,\infty).$$
\rozwStop
\odpStart
$x \in [5,8] \cup [20,\infty)$
\odpStop
\testStart
A.$x \in [5,8] \cup [20,\infty)$\\
B.$x \in (5,8) \cup [20,\infty)$\\
C.$x \in (5,8] \cup [20,\infty)$\\
D.$x \in [5,8) \cup [20,\infty)$\\
E.$x \in [5,8] \cup (20,\infty)$\\
F.$x \in (5,8) \cup (20,\infty)$\\
G.$x \in [5,8) \cup (20,\infty)$\\
H.$x \in (5,8] \cup (20,\infty)$
\testStop
\kluczStart
A
\kluczStop



\zadStart{Zadanie z Wikieł Z 1.62 a) moja wersja nr 620}

Rozwiązać nierówności $(x-5)(x-9)(x-10)\ge0$.
\zadStop
\rozwStart{Patryk Wirkus}{}
Miejsca zerowe naszego wielomianu to: $5, 9, 10$.\\
Wielomian jest stopnia nieparzystego, ponadto znak współczynnika przy\linebreak najwyższej potędze x jest dodatni.\\ W związku z tym wykres wielomianu zaczyna się od lewej strony poniżej osi OX. A więc $$x \in [5,9] \cup [10,\infty).$$
\rozwStop
\odpStart
$x \in [5,9] \cup [10,\infty)$
\odpStop
\testStart
A.$x \in [5,9] \cup [10,\infty)$\\
B.$x \in (5,9) \cup [10,\infty)$\\
C.$x \in (5,9] \cup [10,\infty)$\\
D.$x \in [5,9) \cup [10,\infty)$\\
E.$x \in [5,9] \cup (10,\infty)$\\
F.$x \in (5,9) \cup (10,\infty)$\\
G.$x \in [5,9) \cup (10,\infty)$\\
H.$x \in (5,9] \cup (10,\infty)$
\testStop
\kluczStart
A
\kluczStop



\zadStart{Zadanie z Wikieł Z 1.62 a) moja wersja nr 621}

Rozwiązać nierówności $(x-5)(x-9)(x-11)\ge0$.
\zadStop
\rozwStart{Patryk Wirkus}{}
Miejsca zerowe naszego wielomianu to: $5, 9, 11$.\\
Wielomian jest stopnia nieparzystego, ponadto znak współczynnika przy\linebreak najwyższej potędze x jest dodatni.\\ W związku z tym wykres wielomianu zaczyna się od lewej strony poniżej osi OX. A więc $$x \in [5,9] \cup [11,\infty).$$
\rozwStop
\odpStart
$x \in [5,9] \cup [11,\infty)$
\odpStop
\testStart
A.$x \in [5,9] \cup [11,\infty)$\\
B.$x \in (5,9) \cup [11,\infty)$\\
C.$x \in (5,9] \cup [11,\infty)$\\
D.$x \in [5,9) \cup [11,\infty)$\\
E.$x \in [5,9] \cup (11,\infty)$\\
F.$x \in (5,9) \cup (11,\infty)$\\
G.$x \in [5,9) \cup (11,\infty)$\\
H.$x \in (5,9] \cup (11,\infty)$
\testStop
\kluczStart
A
\kluczStop



\zadStart{Zadanie z Wikieł Z 1.62 a) moja wersja nr 622}

Rozwiązać nierówności $(x-5)(x-9)(x-12)\ge0$.
\zadStop
\rozwStart{Patryk Wirkus}{}
Miejsca zerowe naszego wielomianu to: $5, 9, 12$.\\
Wielomian jest stopnia nieparzystego, ponadto znak współczynnika przy\linebreak najwyższej potędze x jest dodatni.\\ W związku z tym wykres wielomianu zaczyna się od lewej strony poniżej osi OX. A więc $$x \in [5,9] \cup [12,\infty).$$
\rozwStop
\odpStart
$x \in [5,9] \cup [12,\infty)$
\odpStop
\testStart
A.$x \in [5,9] \cup [12,\infty)$\\
B.$x \in (5,9) \cup [12,\infty)$\\
C.$x \in (5,9] \cup [12,\infty)$\\
D.$x \in [5,9) \cup [12,\infty)$\\
E.$x \in [5,9] \cup (12,\infty)$\\
F.$x \in (5,9) \cup (12,\infty)$\\
G.$x \in [5,9) \cup (12,\infty)$\\
H.$x \in (5,9] \cup (12,\infty)$
\testStop
\kluczStart
A
\kluczStop



\zadStart{Zadanie z Wikieł Z 1.62 a) moja wersja nr 623}

Rozwiązać nierówności $(x-5)(x-9)(x-13)\ge0$.
\zadStop
\rozwStart{Patryk Wirkus}{}
Miejsca zerowe naszego wielomianu to: $5, 9, 13$.\\
Wielomian jest stopnia nieparzystego, ponadto znak współczynnika przy\linebreak najwyższej potędze x jest dodatni.\\ W związku z tym wykres wielomianu zaczyna się od lewej strony poniżej osi OX. A więc $$x \in [5,9] \cup [13,\infty).$$
\rozwStop
\odpStart
$x \in [5,9] \cup [13,\infty)$
\odpStop
\testStart
A.$x \in [5,9] \cup [13,\infty)$\\
B.$x \in (5,9) \cup [13,\infty)$\\
C.$x \in (5,9] \cup [13,\infty)$\\
D.$x \in [5,9) \cup [13,\infty)$\\
E.$x \in [5,9] \cup (13,\infty)$\\
F.$x \in (5,9) \cup (13,\infty)$\\
G.$x \in [5,9) \cup (13,\infty)$\\
H.$x \in (5,9] \cup (13,\infty)$
\testStop
\kluczStart
A
\kluczStop



\zadStart{Zadanie z Wikieł Z 1.62 a) moja wersja nr 624}

Rozwiązać nierówności $(x-5)(x-9)(x-14)\ge0$.
\zadStop
\rozwStart{Patryk Wirkus}{}
Miejsca zerowe naszego wielomianu to: $5, 9, 14$.\\
Wielomian jest stopnia nieparzystego, ponadto znak współczynnika przy\linebreak najwyższej potędze x jest dodatni.\\ W związku z tym wykres wielomianu zaczyna się od lewej strony poniżej osi OX. A więc $$x \in [5,9] \cup [14,\infty).$$
\rozwStop
\odpStart
$x \in [5,9] \cup [14,\infty)$
\odpStop
\testStart
A.$x \in [5,9] \cup [14,\infty)$\\
B.$x \in (5,9) \cup [14,\infty)$\\
C.$x \in (5,9] \cup [14,\infty)$\\
D.$x \in [5,9) \cup [14,\infty)$\\
E.$x \in [5,9] \cup (14,\infty)$\\
F.$x \in (5,9) \cup (14,\infty)$\\
G.$x \in [5,9) \cup (14,\infty)$\\
H.$x \in (5,9] \cup (14,\infty)$
\testStop
\kluczStart
A
\kluczStop



\zadStart{Zadanie z Wikieł Z 1.62 a) moja wersja nr 625}

Rozwiązać nierówności $(x-5)(x-9)(x-15)\ge0$.
\zadStop
\rozwStart{Patryk Wirkus}{}
Miejsca zerowe naszego wielomianu to: $5, 9, 15$.\\
Wielomian jest stopnia nieparzystego, ponadto znak współczynnika przy\linebreak najwyższej potędze x jest dodatni.\\ W związku z tym wykres wielomianu zaczyna się od lewej strony poniżej osi OX. A więc $$x \in [5,9] \cup [15,\infty).$$
\rozwStop
\odpStart
$x \in [5,9] \cup [15,\infty)$
\odpStop
\testStart
A.$x \in [5,9] \cup [15,\infty)$\\
B.$x \in (5,9) \cup [15,\infty)$\\
C.$x \in (5,9] \cup [15,\infty)$\\
D.$x \in [5,9) \cup [15,\infty)$\\
E.$x \in [5,9] \cup (15,\infty)$\\
F.$x \in (5,9) \cup (15,\infty)$\\
G.$x \in [5,9) \cup (15,\infty)$\\
H.$x \in (5,9] \cup (15,\infty)$
\testStop
\kluczStart
A
\kluczStop



\zadStart{Zadanie z Wikieł Z 1.62 a) moja wersja nr 626}

Rozwiązać nierówności $(x-5)(x-9)(x-16)\ge0$.
\zadStop
\rozwStart{Patryk Wirkus}{}
Miejsca zerowe naszego wielomianu to: $5, 9, 16$.\\
Wielomian jest stopnia nieparzystego, ponadto znak współczynnika przy\linebreak najwyższej potędze x jest dodatni.\\ W związku z tym wykres wielomianu zaczyna się od lewej strony poniżej osi OX. A więc $$x \in [5,9] \cup [16,\infty).$$
\rozwStop
\odpStart
$x \in [5,9] \cup [16,\infty)$
\odpStop
\testStart
A.$x \in [5,9] \cup [16,\infty)$\\
B.$x \in (5,9) \cup [16,\infty)$\\
C.$x \in (5,9] \cup [16,\infty)$\\
D.$x \in [5,9) \cup [16,\infty)$\\
E.$x \in [5,9] \cup (16,\infty)$\\
F.$x \in (5,9) \cup (16,\infty)$\\
G.$x \in [5,9) \cup (16,\infty)$\\
H.$x \in (5,9] \cup (16,\infty)$
\testStop
\kluczStart
A
\kluczStop



\zadStart{Zadanie z Wikieł Z 1.62 a) moja wersja nr 627}

Rozwiązać nierówności $(x-5)(x-9)(x-17)\ge0$.
\zadStop
\rozwStart{Patryk Wirkus}{}
Miejsca zerowe naszego wielomianu to: $5, 9, 17$.\\
Wielomian jest stopnia nieparzystego, ponadto znak współczynnika przy\linebreak najwyższej potędze x jest dodatni.\\ W związku z tym wykres wielomianu zaczyna się od lewej strony poniżej osi OX. A więc $$x \in [5,9] \cup [17,\infty).$$
\rozwStop
\odpStart
$x \in [5,9] \cup [17,\infty)$
\odpStop
\testStart
A.$x \in [5,9] \cup [17,\infty)$\\
B.$x \in (5,9) \cup [17,\infty)$\\
C.$x \in (5,9] \cup [17,\infty)$\\
D.$x \in [5,9) \cup [17,\infty)$\\
E.$x \in [5,9] \cup (17,\infty)$\\
F.$x \in (5,9) \cup (17,\infty)$\\
G.$x \in [5,9) \cup (17,\infty)$\\
H.$x \in (5,9] \cup (17,\infty)$
\testStop
\kluczStart
A
\kluczStop



\zadStart{Zadanie z Wikieł Z 1.62 a) moja wersja nr 628}

Rozwiązać nierówności $(x-5)(x-9)(x-18)\ge0$.
\zadStop
\rozwStart{Patryk Wirkus}{}
Miejsca zerowe naszego wielomianu to: $5, 9, 18$.\\
Wielomian jest stopnia nieparzystego, ponadto znak współczynnika przy\linebreak najwyższej potędze x jest dodatni.\\ W związku z tym wykres wielomianu zaczyna się od lewej strony poniżej osi OX. A więc $$x \in [5,9] \cup [18,\infty).$$
\rozwStop
\odpStart
$x \in [5,9] \cup [18,\infty)$
\odpStop
\testStart
A.$x \in [5,9] \cup [18,\infty)$\\
B.$x \in (5,9) \cup [18,\infty)$\\
C.$x \in (5,9] \cup [18,\infty)$\\
D.$x \in [5,9) \cup [18,\infty)$\\
E.$x \in [5,9] \cup (18,\infty)$\\
F.$x \in (5,9) \cup (18,\infty)$\\
G.$x \in [5,9) \cup (18,\infty)$\\
H.$x \in (5,9] \cup (18,\infty)$
\testStop
\kluczStart
A
\kluczStop



\zadStart{Zadanie z Wikieł Z 1.62 a) moja wersja nr 629}

Rozwiązać nierówności $(x-5)(x-9)(x-19)\ge0$.
\zadStop
\rozwStart{Patryk Wirkus}{}
Miejsca zerowe naszego wielomianu to: $5, 9, 19$.\\
Wielomian jest stopnia nieparzystego, ponadto znak współczynnika przy\linebreak najwyższej potędze x jest dodatni.\\ W związku z tym wykres wielomianu zaczyna się od lewej strony poniżej osi OX. A więc $$x \in [5,9] \cup [19,\infty).$$
\rozwStop
\odpStart
$x \in [5,9] \cup [19,\infty)$
\odpStop
\testStart
A.$x \in [5,9] \cup [19,\infty)$\\
B.$x \in (5,9) \cup [19,\infty)$\\
C.$x \in (5,9] \cup [19,\infty)$\\
D.$x \in [5,9) \cup [19,\infty)$\\
E.$x \in [5,9] \cup (19,\infty)$\\
F.$x \in (5,9) \cup (19,\infty)$\\
G.$x \in [5,9) \cup (19,\infty)$\\
H.$x \in (5,9] \cup (19,\infty)$
\testStop
\kluczStart
A
\kluczStop



\zadStart{Zadanie z Wikieł Z 1.62 a) moja wersja nr 630}

Rozwiązać nierówności $(x-5)(x-9)(x-20)\ge0$.
\zadStop
\rozwStart{Patryk Wirkus}{}
Miejsca zerowe naszego wielomianu to: $5, 9, 20$.\\
Wielomian jest stopnia nieparzystego, ponadto znak współczynnika przy\linebreak najwyższej potędze x jest dodatni.\\ W związku z tym wykres wielomianu zaczyna się od lewej strony poniżej osi OX. A więc $$x \in [5,9] \cup [20,\infty).$$
\rozwStop
\odpStart
$x \in [5,9] \cup [20,\infty)$
\odpStop
\testStart
A.$x \in [5,9] \cup [20,\infty)$\\
B.$x \in (5,9) \cup [20,\infty)$\\
C.$x \in (5,9] \cup [20,\infty)$\\
D.$x \in [5,9) \cup [20,\infty)$\\
E.$x \in [5,9] \cup (20,\infty)$\\
F.$x \in (5,9) \cup (20,\infty)$\\
G.$x \in [5,9) \cup (20,\infty)$\\
H.$x \in (5,9] \cup (20,\infty)$
\testStop
\kluczStart
A
\kluczStop



\zadStart{Zadanie z Wikieł Z 1.62 a) moja wersja nr 631}

Rozwiązać nierówności $(x-5)(x-10)(x-11)\ge0$.
\zadStop
\rozwStart{Patryk Wirkus}{}
Miejsca zerowe naszego wielomianu to: $5, 10, 11$.\\
Wielomian jest stopnia nieparzystego, ponadto znak współczynnika przy\linebreak najwyższej potędze x jest dodatni.\\ W związku z tym wykres wielomianu zaczyna się od lewej strony poniżej osi OX. A więc $$x \in [5,10] \cup [11,\infty).$$
\rozwStop
\odpStart
$x \in [5,10] \cup [11,\infty)$
\odpStop
\testStart
A.$x \in [5,10] \cup [11,\infty)$\\
B.$x \in (5,10) \cup [11,\infty)$\\
C.$x \in (5,10] \cup [11,\infty)$\\
D.$x \in [5,10) \cup [11,\infty)$\\
E.$x \in [5,10] \cup (11,\infty)$\\
F.$x \in (5,10) \cup (11,\infty)$\\
G.$x \in [5,10) \cup (11,\infty)$\\
H.$x \in (5,10] \cup (11,\infty)$
\testStop
\kluczStart
A
\kluczStop



\zadStart{Zadanie z Wikieł Z 1.62 a) moja wersja nr 632}

Rozwiązać nierówności $(x-5)(x-10)(x-12)\ge0$.
\zadStop
\rozwStart{Patryk Wirkus}{}
Miejsca zerowe naszego wielomianu to: $5, 10, 12$.\\
Wielomian jest stopnia nieparzystego, ponadto znak współczynnika przy\linebreak najwyższej potędze x jest dodatni.\\ W związku z tym wykres wielomianu zaczyna się od lewej strony poniżej osi OX. A więc $$x \in [5,10] \cup [12,\infty).$$
\rozwStop
\odpStart
$x \in [5,10] \cup [12,\infty)$
\odpStop
\testStart
A.$x \in [5,10] \cup [12,\infty)$\\
B.$x \in (5,10) \cup [12,\infty)$\\
C.$x \in (5,10] \cup [12,\infty)$\\
D.$x \in [5,10) \cup [12,\infty)$\\
E.$x \in [5,10] \cup (12,\infty)$\\
F.$x \in (5,10) \cup (12,\infty)$\\
G.$x \in [5,10) \cup (12,\infty)$\\
H.$x \in (5,10] \cup (12,\infty)$
\testStop
\kluczStart
A
\kluczStop



\zadStart{Zadanie z Wikieł Z 1.62 a) moja wersja nr 633}

Rozwiązać nierówności $(x-5)(x-10)(x-13)\ge0$.
\zadStop
\rozwStart{Patryk Wirkus}{}
Miejsca zerowe naszego wielomianu to: $5, 10, 13$.\\
Wielomian jest stopnia nieparzystego, ponadto znak współczynnika przy\linebreak najwyższej potędze x jest dodatni.\\ W związku z tym wykres wielomianu zaczyna się od lewej strony poniżej osi OX. A więc $$x \in [5,10] \cup [13,\infty).$$
\rozwStop
\odpStart
$x \in [5,10] \cup [13,\infty)$
\odpStop
\testStart
A.$x \in [5,10] \cup [13,\infty)$\\
B.$x \in (5,10) \cup [13,\infty)$\\
C.$x \in (5,10] \cup [13,\infty)$\\
D.$x \in [5,10) \cup [13,\infty)$\\
E.$x \in [5,10] \cup (13,\infty)$\\
F.$x \in (5,10) \cup (13,\infty)$\\
G.$x \in [5,10) \cup (13,\infty)$\\
H.$x \in (5,10] \cup (13,\infty)$
\testStop
\kluczStart
A
\kluczStop



\zadStart{Zadanie z Wikieł Z 1.62 a) moja wersja nr 634}

Rozwiązać nierówności $(x-5)(x-10)(x-14)\ge0$.
\zadStop
\rozwStart{Patryk Wirkus}{}
Miejsca zerowe naszego wielomianu to: $5, 10, 14$.\\
Wielomian jest stopnia nieparzystego, ponadto znak współczynnika przy\linebreak najwyższej potędze x jest dodatni.\\ W związku z tym wykres wielomianu zaczyna się od lewej strony poniżej osi OX. A więc $$x \in [5,10] \cup [14,\infty).$$
\rozwStop
\odpStart
$x \in [5,10] \cup [14,\infty)$
\odpStop
\testStart
A.$x \in [5,10] \cup [14,\infty)$\\
B.$x \in (5,10) \cup [14,\infty)$\\
C.$x \in (5,10] \cup [14,\infty)$\\
D.$x \in [5,10) \cup [14,\infty)$\\
E.$x \in [5,10] \cup (14,\infty)$\\
F.$x \in (5,10) \cup (14,\infty)$\\
G.$x \in [5,10) \cup (14,\infty)$\\
H.$x \in (5,10] \cup (14,\infty)$
\testStop
\kluczStart
A
\kluczStop



\zadStart{Zadanie z Wikieł Z 1.62 a) moja wersja nr 635}

Rozwiązać nierówności $(x-5)(x-10)(x-15)\ge0$.
\zadStop
\rozwStart{Patryk Wirkus}{}
Miejsca zerowe naszego wielomianu to: $5, 10, 15$.\\
Wielomian jest stopnia nieparzystego, ponadto znak współczynnika przy\linebreak najwyższej potędze x jest dodatni.\\ W związku z tym wykres wielomianu zaczyna się od lewej strony poniżej osi OX. A więc $$x \in [5,10] \cup [15,\infty).$$
\rozwStop
\odpStart
$x \in [5,10] \cup [15,\infty)$
\odpStop
\testStart
A.$x \in [5,10] \cup [15,\infty)$\\
B.$x \in (5,10) \cup [15,\infty)$\\
C.$x \in (5,10] \cup [15,\infty)$\\
D.$x \in [5,10) \cup [15,\infty)$\\
E.$x \in [5,10] \cup (15,\infty)$\\
F.$x \in (5,10) \cup (15,\infty)$\\
G.$x \in [5,10) \cup (15,\infty)$\\
H.$x \in (5,10] \cup (15,\infty)$
\testStop
\kluczStart
A
\kluczStop



\zadStart{Zadanie z Wikieł Z 1.62 a) moja wersja nr 636}

Rozwiązać nierówności $(x-5)(x-10)(x-16)\ge0$.
\zadStop
\rozwStart{Patryk Wirkus}{}
Miejsca zerowe naszego wielomianu to: $5, 10, 16$.\\
Wielomian jest stopnia nieparzystego, ponadto znak współczynnika przy\linebreak najwyższej potędze x jest dodatni.\\ W związku z tym wykres wielomianu zaczyna się od lewej strony poniżej osi OX. A więc $$x \in [5,10] \cup [16,\infty).$$
\rozwStop
\odpStart
$x \in [5,10] \cup [16,\infty)$
\odpStop
\testStart
A.$x \in [5,10] \cup [16,\infty)$\\
B.$x \in (5,10) \cup [16,\infty)$\\
C.$x \in (5,10] \cup [16,\infty)$\\
D.$x \in [5,10) \cup [16,\infty)$\\
E.$x \in [5,10] \cup (16,\infty)$\\
F.$x \in (5,10) \cup (16,\infty)$\\
G.$x \in [5,10) \cup (16,\infty)$\\
H.$x \in (5,10] \cup (16,\infty)$
\testStop
\kluczStart
A
\kluczStop



\zadStart{Zadanie z Wikieł Z 1.62 a) moja wersja nr 637}

Rozwiązać nierówności $(x-5)(x-10)(x-17)\ge0$.
\zadStop
\rozwStart{Patryk Wirkus}{}
Miejsca zerowe naszego wielomianu to: $5, 10, 17$.\\
Wielomian jest stopnia nieparzystego, ponadto znak współczynnika przy\linebreak najwyższej potędze x jest dodatni.\\ W związku z tym wykres wielomianu zaczyna się od lewej strony poniżej osi OX. A więc $$x \in [5,10] \cup [17,\infty).$$
\rozwStop
\odpStart
$x \in [5,10] \cup [17,\infty)$
\odpStop
\testStart
A.$x \in [5,10] \cup [17,\infty)$\\
B.$x \in (5,10) \cup [17,\infty)$\\
C.$x \in (5,10] \cup [17,\infty)$\\
D.$x \in [5,10) \cup [17,\infty)$\\
E.$x \in [5,10] \cup (17,\infty)$\\
F.$x \in (5,10) \cup (17,\infty)$\\
G.$x \in [5,10) \cup (17,\infty)$\\
H.$x \in (5,10] \cup (17,\infty)$
\testStop
\kluczStart
A
\kluczStop



\zadStart{Zadanie z Wikieł Z 1.62 a) moja wersja nr 638}

Rozwiązać nierówności $(x-5)(x-10)(x-18)\ge0$.
\zadStop
\rozwStart{Patryk Wirkus}{}
Miejsca zerowe naszego wielomianu to: $5, 10, 18$.\\
Wielomian jest stopnia nieparzystego, ponadto znak współczynnika przy\linebreak najwyższej potędze x jest dodatni.\\ W związku z tym wykres wielomianu zaczyna się od lewej strony poniżej osi OX. A więc $$x \in [5,10] \cup [18,\infty).$$
\rozwStop
\odpStart
$x \in [5,10] \cup [18,\infty)$
\odpStop
\testStart
A.$x \in [5,10] \cup [18,\infty)$\\
B.$x \in (5,10) \cup [18,\infty)$\\
C.$x \in (5,10] \cup [18,\infty)$\\
D.$x \in [5,10) \cup [18,\infty)$\\
E.$x \in [5,10] \cup (18,\infty)$\\
F.$x \in (5,10) \cup (18,\infty)$\\
G.$x \in [5,10) \cup (18,\infty)$\\
H.$x \in (5,10] \cup (18,\infty)$
\testStop
\kluczStart
A
\kluczStop



\zadStart{Zadanie z Wikieł Z 1.62 a) moja wersja nr 639}

Rozwiązać nierówności $(x-5)(x-10)(x-19)\ge0$.
\zadStop
\rozwStart{Patryk Wirkus}{}
Miejsca zerowe naszego wielomianu to: $5, 10, 19$.\\
Wielomian jest stopnia nieparzystego, ponadto znak współczynnika przy\linebreak najwyższej potędze x jest dodatni.\\ W związku z tym wykres wielomianu zaczyna się od lewej strony poniżej osi OX. A więc $$x \in [5,10] \cup [19,\infty).$$
\rozwStop
\odpStart
$x \in [5,10] \cup [19,\infty)$
\odpStop
\testStart
A.$x \in [5,10] \cup [19,\infty)$\\
B.$x \in (5,10) \cup [19,\infty)$\\
C.$x \in (5,10] \cup [19,\infty)$\\
D.$x \in [5,10) \cup [19,\infty)$\\
E.$x \in [5,10] \cup (19,\infty)$\\
F.$x \in (5,10) \cup (19,\infty)$\\
G.$x \in [5,10) \cup (19,\infty)$\\
H.$x \in (5,10] \cup (19,\infty)$
\testStop
\kluczStart
A
\kluczStop



\zadStart{Zadanie z Wikieł Z 1.62 a) moja wersja nr 640}

Rozwiązać nierówności $(x-5)(x-10)(x-20)\ge0$.
\zadStop
\rozwStart{Patryk Wirkus}{}
Miejsca zerowe naszego wielomianu to: $5, 10, 20$.\\
Wielomian jest stopnia nieparzystego, ponadto znak współczynnika przy\linebreak najwyższej potędze x jest dodatni.\\ W związku z tym wykres wielomianu zaczyna się od lewej strony poniżej osi OX. A więc $$x \in [5,10] \cup [20,\infty).$$
\rozwStop
\odpStart
$x \in [5,10] \cup [20,\infty)$
\odpStop
\testStart
A.$x \in [5,10] \cup [20,\infty)$\\
B.$x \in (5,10) \cup [20,\infty)$\\
C.$x \in (5,10] \cup [20,\infty)$\\
D.$x \in [5,10) \cup [20,\infty)$\\
E.$x \in [5,10] \cup (20,\infty)$\\
F.$x \in (5,10) \cup (20,\infty)$\\
G.$x \in [5,10) \cup (20,\infty)$\\
H.$x \in (5,10] \cup (20,\infty)$
\testStop
\kluczStart
A
\kluczStop



\zadStart{Zadanie z Wikieł Z 1.62 a) moja wersja nr 641}

Rozwiązać nierówności $(x-5)(x-11)(x-12)\ge0$.
\zadStop
\rozwStart{Patryk Wirkus}{}
Miejsca zerowe naszego wielomianu to: $5, 11, 12$.\\
Wielomian jest stopnia nieparzystego, ponadto znak współczynnika przy\linebreak najwyższej potędze x jest dodatni.\\ W związku z tym wykres wielomianu zaczyna się od lewej strony poniżej osi OX. A więc $$x \in [5,11] \cup [12,\infty).$$
\rozwStop
\odpStart
$x \in [5,11] \cup [12,\infty)$
\odpStop
\testStart
A.$x \in [5,11] \cup [12,\infty)$\\
B.$x \in (5,11) \cup [12,\infty)$\\
C.$x \in (5,11] \cup [12,\infty)$\\
D.$x \in [5,11) \cup [12,\infty)$\\
E.$x \in [5,11] \cup (12,\infty)$\\
F.$x \in (5,11) \cup (12,\infty)$\\
G.$x \in [5,11) \cup (12,\infty)$\\
H.$x \in (5,11] \cup (12,\infty)$
\testStop
\kluczStart
A
\kluczStop



\zadStart{Zadanie z Wikieł Z 1.62 a) moja wersja nr 642}

Rozwiązać nierówności $(x-5)(x-11)(x-13)\ge0$.
\zadStop
\rozwStart{Patryk Wirkus}{}
Miejsca zerowe naszego wielomianu to: $5, 11, 13$.\\
Wielomian jest stopnia nieparzystego, ponadto znak współczynnika przy\linebreak najwyższej potędze x jest dodatni.\\ W związku z tym wykres wielomianu zaczyna się od lewej strony poniżej osi OX. A więc $$x \in [5,11] \cup [13,\infty).$$
\rozwStop
\odpStart
$x \in [5,11] \cup [13,\infty)$
\odpStop
\testStart
A.$x \in [5,11] \cup [13,\infty)$\\
B.$x \in (5,11) \cup [13,\infty)$\\
C.$x \in (5,11] \cup [13,\infty)$\\
D.$x \in [5,11) \cup [13,\infty)$\\
E.$x \in [5,11] \cup (13,\infty)$\\
F.$x \in (5,11) \cup (13,\infty)$\\
G.$x \in [5,11) \cup (13,\infty)$\\
H.$x \in (5,11] \cup (13,\infty)$
\testStop
\kluczStart
A
\kluczStop



\zadStart{Zadanie z Wikieł Z 1.62 a) moja wersja nr 643}

Rozwiązać nierówności $(x-5)(x-11)(x-14)\ge0$.
\zadStop
\rozwStart{Patryk Wirkus}{}
Miejsca zerowe naszego wielomianu to: $5, 11, 14$.\\
Wielomian jest stopnia nieparzystego, ponadto znak współczynnika przy\linebreak najwyższej potędze x jest dodatni.\\ W związku z tym wykres wielomianu zaczyna się od lewej strony poniżej osi OX. A więc $$x \in [5,11] \cup [14,\infty).$$
\rozwStop
\odpStart
$x \in [5,11] \cup [14,\infty)$
\odpStop
\testStart
A.$x \in [5,11] \cup [14,\infty)$\\
B.$x \in (5,11) \cup [14,\infty)$\\
C.$x \in (5,11] \cup [14,\infty)$\\
D.$x \in [5,11) \cup [14,\infty)$\\
E.$x \in [5,11] \cup (14,\infty)$\\
F.$x \in (5,11) \cup (14,\infty)$\\
G.$x \in [5,11) \cup (14,\infty)$\\
H.$x \in (5,11] \cup (14,\infty)$
\testStop
\kluczStart
A
\kluczStop



\zadStart{Zadanie z Wikieł Z 1.62 a) moja wersja nr 644}

Rozwiązać nierówności $(x-5)(x-11)(x-15)\ge0$.
\zadStop
\rozwStart{Patryk Wirkus}{}
Miejsca zerowe naszego wielomianu to: $5, 11, 15$.\\
Wielomian jest stopnia nieparzystego, ponadto znak współczynnika przy\linebreak najwyższej potędze x jest dodatni.\\ W związku z tym wykres wielomianu zaczyna się od lewej strony poniżej osi OX. A więc $$x \in [5,11] \cup [15,\infty).$$
\rozwStop
\odpStart
$x \in [5,11] \cup [15,\infty)$
\odpStop
\testStart
A.$x \in [5,11] \cup [15,\infty)$\\
B.$x \in (5,11) \cup [15,\infty)$\\
C.$x \in (5,11] \cup [15,\infty)$\\
D.$x \in [5,11) \cup [15,\infty)$\\
E.$x \in [5,11] \cup (15,\infty)$\\
F.$x \in (5,11) \cup (15,\infty)$\\
G.$x \in [5,11) \cup (15,\infty)$\\
H.$x \in (5,11] \cup (15,\infty)$
\testStop
\kluczStart
A
\kluczStop



\zadStart{Zadanie z Wikieł Z 1.62 a) moja wersja nr 645}

Rozwiązać nierówności $(x-5)(x-11)(x-16)\ge0$.
\zadStop
\rozwStart{Patryk Wirkus}{}
Miejsca zerowe naszego wielomianu to: $5, 11, 16$.\\
Wielomian jest stopnia nieparzystego, ponadto znak współczynnika przy\linebreak najwyższej potędze x jest dodatni.\\ W związku z tym wykres wielomianu zaczyna się od lewej strony poniżej osi OX. A więc $$x \in [5,11] \cup [16,\infty).$$
\rozwStop
\odpStart
$x \in [5,11] \cup [16,\infty)$
\odpStop
\testStart
A.$x \in [5,11] \cup [16,\infty)$\\
B.$x \in (5,11) \cup [16,\infty)$\\
C.$x \in (5,11] \cup [16,\infty)$\\
D.$x \in [5,11) \cup [16,\infty)$\\
E.$x \in [5,11] \cup (16,\infty)$\\
F.$x \in (5,11) \cup (16,\infty)$\\
G.$x \in [5,11) \cup (16,\infty)$\\
H.$x \in (5,11] \cup (16,\infty)$
\testStop
\kluczStart
A
\kluczStop



\zadStart{Zadanie z Wikieł Z 1.62 a) moja wersja nr 646}

Rozwiązać nierówności $(x-5)(x-11)(x-17)\ge0$.
\zadStop
\rozwStart{Patryk Wirkus}{}
Miejsca zerowe naszego wielomianu to: $5, 11, 17$.\\
Wielomian jest stopnia nieparzystego, ponadto znak współczynnika przy\linebreak najwyższej potędze x jest dodatni.\\ W związku z tym wykres wielomianu zaczyna się od lewej strony poniżej osi OX. A więc $$x \in [5,11] \cup [17,\infty).$$
\rozwStop
\odpStart
$x \in [5,11] \cup [17,\infty)$
\odpStop
\testStart
A.$x \in [5,11] \cup [17,\infty)$\\
B.$x \in (5,11) \cup [17,\infty)$\\
C.$x \in (5,11] \cup [17,\infty)$\\
D.$x \in [5,11) \cup [17,\infty)$\\
E.$x \in [5,11] \cup (17,\infty)$\\
F.$x \in (5,11) \cup (17,\infty)$\\
G.$x \in [5,11) \cup (17,\infty)$\\
H.$x \in (5,11] \cup (17,\infty)$
\testStop
\kluczStart
A
\kluczStop



\zadStart{Zadanie z Wikieł Z 1.62 a) moja wersja nr 647}

Rozwiązać nierówności $(x-5)(x-11)(x-18)\ge0$.
\zadStop
\rozwStart{Patryk Wirkus}{}
Miejsca zerowe naszego wielomianu to: $5, 11, 18$.\\
Wielomian jest stopnia nieparzystego, ponadto znak współczynnika przy\linebreak najwyższej potędze x jest dodatni.\\ W związku z tym wykres wielomianu zaczyna się od lewej strony poniżej osi OX. A więc $$x \in [5,11] \cup [18,\infty).$$
\rozwStop
\odpStart
$x \in [5,11] \cup [18,\infty)$
\odpStop
\testStart
A.$x \in [5,11] \cup [18,\infty)$\\
B.$x \in (5,11) \cup [18,\infty)$\\
C.$x \in (5,11] \cup [18,\infty)$\\
D.$x \in [5,11) \cup [18,\infty)$\\
E.$x \in [5,11] \cup (18,\infty)$\\
F.$x \in (5,11) \cup (18,\infty)$\\
G.$x \in [5,11) \cup (18,\infty)$\\
H.$x \in (5,11] \cup (18,\infty)$
\testStop
\kluczStart
A
\kluczStop



\zadStart{Zadanie z Wikieł Z 1.62 a) moja wersja nr 648}

Rozwiązać nierówności $(x-5)(x-11)(x-19)\ge0$.
\zadStop
\rozwStart{Patryk Wirkus}{}
Miejsca zerowe naszego wielomianu to: $5, 11, 19$.\\
Wielomian jest stopnia nieparzystego, ponadto znak współczynnika przy\linebreak najwyższej potędze x jest dodatni.\\ W związku z tym wykres wielomianu zaczyna się od lewej strony poniżej osi OX. A więc $$x \in [5,11] \cup [19,\infty).$$
\rozwStop
\odpStart
$x \in [5,11] \cup [19,\infty)$
\odpStop
\testStart
A.$x \in [5,11] \cup [19,\infty)$\\
B.$x \in (5,11) \cup [19,\infty)$\\
C.$x \in (5,11] \cup [19,\infty)$\\
D.$x \in [5,11) \cup [19,\infty)$\\
E.$x \in [5,11] \cup (19,\infty)$\\
F.$x \in (5,11) \cup (19,\infty)$\\
G.$x \in [5,11) \cup (19,\infty)$\\
H.$x \in (5,11] \cup (19,\infty)$
\testStop
\kluczStart
A
\kluczStop



\zadStart{Zadanie z Wikieł Z 1.62 a) moja wersja nr 649}

Rozwiązać nierówności $(x-5)(x-11)(x-20)\ge0$.
\zadStop
\rozwStart{Patryk Wirkus}{}
Miejsca zerowe naszego wielomianu to: $5, 11, 20$.\\
Wielomian jest stopnia nieparzystego, ponadto znak współczynnika przy\linebreak najwyższej potędze x jest dodatni.\\ W związku z tym wykres wielomianu zaczyna się od lewej strony poniżej osi OX. A więc $$x \in [5,11] \cup [20,\infty).$$
\rozwStop
\odpStart
$x \in [5,11] \cup [20,\infty)$
\odpStop
\testStart
A.$x \in [5,11] \cup [20,\infty)$\\
B.$x \in (5,11) \cup [20,\infty)$\\
C.$x \in (5,11] \cup [20,\infty)$\\
D.$x \in [5,11) \cup [20,\infty)$\\
E.$x \in [5,11] \cup (20,\infty)$\\
F.$x \in (5,11) \cup (20,\infty)$\\
G.$x \in [5,11) \cup (20,\infty)$\\
H.$x \in (5,11] \cup (20,\infty)$
\testStop
\kluczStart
A
\kluczStop



\zadStart{Zadanie z Wikieł Z 1.62 a) moja wersja nr 650}

Rozwiązać nierówności $(x-5)(x-12)(x-13)\ge0$.
\zadStop
\rozwStart{Patryk Wirkus}{}
Miejsca zerowe naszego wielomianu to: $5, 12, 13$.\\
Wielomian jest stopnia nieparzystego, ponadto znak współczynnika przy\linebreak najwyższej potędze x jest dodatni.\\ W związku z tym wykres wielomianu zaczyna się od lewej strony poniżej osi OX. A więc $$x \in [5,12] \cup [13,\infty).$$
\rozwStop
\odpStart
$x \in [5,12] \cup [13,\infty)$
\odpStop
\testStart
A.$x \in [5,12] \cup [13,\infty)$\\
B.$x \in (5,12) \cup [13,\infty)$\\
C.$x \in (5,12] \cup [13,\infty)$\\
D.$x \in [5,12) \cup [13,\infty)$\\
E.$x \in [5,12] \cup (13,\infty)$\\
F.$x \in (5,12) \cup (13,\infty)$\\
G.$x \in [5,12) \cup (13,\infty)$\\
H.$x \in (5,12] \cup (13,\infty)$
\testStop
\kluczStart
A
\kluczStop



\zadStart{Zadanie z Wikieł Z 1.62 a) moja wersja nr 651}

Rozwiązać nierówności $(x-5)(x-12)(x-14)\ge0$.
\zadStop
\rozwStart{Patryk Wirkus}{}
Miejsca zerowe naszego wielomianu to: $5, 12, 14$.\\
Wielomian jest stopnia nieparzystego, ponadto znak współczynnika przy\linebreak najwyższej potędze x jest dodatni.\\ W związku z tym wykres wielomianu zaczyna się od lewej strony poniżej osi OX. A więc $$x \in [5,12] \cup [14,\infty).$$
\rozwStop
\odpStart
$x \in [5,12] \cup [14,\infty)$
\odpStop
\testStart
A.$x \in [5,12] \cup [14,\infty)$\\
B.$x \in (5,12) \cup [14,\infty)$\\
C.$x \in (5,12] \cup [14,\infty)$\\
D.$x \in [5,12) \cup [14,\infty)$\\
E.$x \in [5,12] \cup (14,\infty)$\\
F.$x \in (5,12) \cup (14,\infty)$\\
G.$x \in [5,12) \cup (14,\infty)$\\
H.$x \in (5,12] \cup (14,\infty)$
\testStop
\kluczStart
A
\kluczStop



\zadStart{Zadanie z Wikieł Z 1.62 a) moja wersja nr 652}

Rozwiązać nierówności $(x-5)(x-12)(x-15)\ge0$.
\zadStop
\rozwStart{Patryk Wirkus}{}
Miejsca zerowe naszego wielomianu to: $5, 12, 15$.\\
Wielomian jest stopnia nieparzystego, ponadto znak współczynnika przy\linebreak najwyższej potędze x jest dodatni.\\ W związku z tym wykres wielomianu zaczyna się od lewej strony poniżej osi OX. A więc $$x \in [5,12] \cup [15,\infty).$$
\rozwStop
\odpStart
$x \in [5,12] \cup [15,\infty)$
\odpStop
\testStart
A.$x \in [5,12] \cup [15,\infty)$\\
B.$x \in (5,12) \cup [15,\infty)$\\
C.$x \in (5,12] \cup [15,\infty)$\\
D.$x \in [5,12) \cup [15,\infty)$\\
E.$x \in [5,12] \cup (15,\infty)$\\
F.$x \in (5,12) \cup (15,\infty)$\\
G.$x \in [5,12) \cup (15,\infty)$\\
H.$x \in (5,12] \cup (15,\infty)$
\testStop
\kluczStart
A
\kluczStop



\zadStart{Zadanie z Wikieł Z 1.62 a) moja wersja nr 653}

Rozwiązać nierówności $(x-5)(x-12)(x-16)\ge0$.
\zadStop
\rozwStart{Patryk Wirkus}{}
Miejsca zerowe naszego wielomianu to: $5, 12, 16$.\\
Wielomian jest stopnia nieparzystego, ponadto znak współczynnika przy\linebreak najwyższej potędze x jest dodatni.\\ W związku z tym wykres wielomianu zaczyna się od lewej strony poniżej osi OX. A więc $$x \in [5,12] \cup [16,\infty).$$
\rozwStop
\odpStart
$x \in [5,12] \cup [16,\infty)$
\odpStop
\testStart
A.$x \in [5,12] \cup [16,\infty)$\\
B.$x \in (5,12) \cup [16,\infty)$\\
C.$x \in (5,12] \cup [16,\infty)$\\
D.$x \in [5,12) \cup [16,\infty)$\\
E.$x \in [5,12] \cup (16,\infty)$\\
F.$x \in (5,12) \cup (16,\infty)$\\
G.$x \in [5,12) \cup (16,\infty)$\\
H.$x \in (5,12] \cup (16,\infty)$
\testStop
\kluczStart
A
\kluczStop



\zadStart{Zadanie z Wikieł Z 1.62 a) moja wersja nr 654}

Rozwiązać nierówności $(x-5)(x-12)(x-17)\ge0$.
\zadStop
\rozwStart{Patryk Wirkus}{}
Miejsca zerowe naszego wielomianu to: $5, 12, 17$.\\
Wielomian jest stopnia nieparzystego, ponadto znak współczynnika przy\linebreak najwyższej potędze x jest dodatni.\\ W związku z tym wykres wielomianu zaczyna się od lewej strony poniżej osi OX. A więc $$x \in [5,12] \cup [17,\infty).$$
\rozwStop
\odpStart
$x \in [5,12] \cup [17,\infty)$
\odpStop
\testStart
A.$x \in [5,12] \cup [17,\infty)$\\
B.$x \in (5,12) \cup [17,\infty)$\\
C.$x \in (5,12] \cup [17,\infty)$\\
D.$x \in [5,12) \cup [17,\infty)$\\
E.$x \in [5,12] \cup (17,\infty)$\\
F.$x \in (5,12) \cup (17,\infty)$\\
G.$x \in [5,12) \cup (17,\infty)$\\
H.$x \in (5,12] \cup (17,\infty)$
\testStop
\kluczStart
A
\kluczStop



\zadStart{Zadanie z Wikieł Z 1.62 a) moja wersja nr 655}

Rozwiązać nierówności $(x-5)(x-12)(x-18)\ge0$.
\zadStop
\rozwStart{Patryk Wirkus}{}
Miejsca zerowe naszego wielomianu to: $5, 12, 18$.\\
Wielomian jest stopnia nieparzystego, ponadto znak współczynnika przy\linebreak najwyższej potędze x jest dodatni.\\ W związku z tym wykres wielomianu zaczyna się od lewej strony poniżej osi OX. A więc $$x \in [5,12] \cup [18,\infty).$$
\rozwStop
\odpStart
$x \in [5,12] \cup [18,\infty)$
\odpStop
\testStart
A.$x \in [5,12] \cup [18,\infty)$\\
B.$x \in (5,12) \cup [18,\infty)$\\
C.$x \in (5,12] \cup [18,\infty)$\\
D.$x \in [5,12) \cup [18,\infty)$\\
E.$x \in [5,12] \cup (18,\infty)$\\
F.$x \in (5,12) \cup (18,\infty)$\\
G.$x \in [5,12) \cup (18,\infty)$\\
H.$x \in (5,12] \cup (18,\infty)$
\testStop
\kluczStart
A
\kluczStop



\zadStart{Zadanie z Wikieł Z 1.62 a) moja wersja nr 656}

Rozwiązać nierówności $(x-5)(x-12)(x-19)\ge0$.
\zadStop
\rozwStart{Patryk Wirkus}{}
Miejsca zerowe naszego wielomianu to: $5, 12, 19$.\\
Wielomian jest stopnia nieparzystego, ponadto znak współczynnika przy\linebreak najwyższej potędze x jest dodatni.\\ W związku z tym wykres wielomianu zaczyna się od lewej strony poniżej osi OX. A więc $$x \in [5,12] \cup [19,\infty).$$
\rozwStop
\odpStart
$x \in [5,12] \cup [19,\infty)$
\odpStop
\testStart
A.$x \in [5,12] \cup [19,\infty)$\\
B.$x \in (5,12) \cup [19,\infty)$\\
C.$x \in (5,12] \cup [19,\infty)$\\
D.$x \in [5,12) \cup [19,\infty)$\\
E.$x \in [5,12] \cup (19,\infty)$\\
F.$x \in (5,12) \cup (19,\infty)$\\
G.$x \in [5,12) \cup (19,\infty)$\\
H.$x \in (5,12] \cup (19,\infty)$
\testStop
\kluczStart
A
\kluczStop



\zadStart{Zadanie z Wikieł Z 1.62 a) moja wersja nr 657}

Rozwiązać nierówności $(x-5)(x-12)(x-20)\ge0$.
\zadStop
\rozwStart{Patryk Wirkus}{}
Miejsca zerowe naszego wielomianu to: $5, 12, 20$.\\
Wielomian jest stopnia nieparzystego, ponadto znak współczynnika przy\linebreak najwyższej potędze x jest dodatni.\\ W związku z tym wykres wielomianu zaczyna się od lewej strony poniżej osi OX. A więc $$x \in [5,12] \cup [20,\infty).$$
\rozwStop
\odpStart
$x \in [5,12] \cup [20,\infty)$
\odpStop
\testStart
A.$x \in [5,12] \cup [20,\infty)$\\
B.$x \in (5,12) \cup [20,\infty)$\\
C.$x \in (5,12] \cup [20,\infty)$\\
D.$x \in [5,12) \cup [20,\infty)$\\
E.$x \in [5,12] \cup (20,\infty)$\\
F.$x \in (5,12) \cup (20,\infty)$\\
G.$x \in [5,12) \cup (20,\infty)$\\
H.$x \in (5,12] \cup (20,\infty)$
\testStop
\kluczStart
A
\kluczStop



\zadStart{Zadanie z Wikieł Z 1.62 a) moja wersja nr 658}

Rozwiązać nierówności $(x-5)(x-13)(x-14)\ge0$.
\zadStop
\rozwStart{Patryk Wirkus}{}
Miejsca zerowe naszego wielomianu to: $5, 13, 14$.\\
Wielomian jest stopnia nieparzystego, ponadto znak współczynnika przy\linebreak najwyższej potędze x jest dodatni.\\ W związku z tym wykres wielomianu zaczyna się od lewej strony poniżej osi OX. A więc $$x \in [5,13] \cup [14,\infty).$$
\rozwStop
\odpStart
$x \in [5,13] \cup [14,\infty)$
\odpStop
\testStart
A.$x \in [5,13] \cup [14,\infty)$\\
B.$x \in (5,13) \cup [14,\infty)$\\
C.$x \in (5,13] \cup [14,\infty)$\\
D.$x \in [5,13) \cup [14,\infty)$\\
E.$x \in [5,13] \cup (14,\infty)$\\
F.$x \in (5,13) \cup (14,\infty)$\\
G.$x \in [5,13) \cup (14,\infty)$\\
H.$x \in (5,13] \cup (14,\infty)$
\testStop
\kluczStart
A
\kluczStop



\zadStart{Zadanie z Wikieł Z 1.62 a) moja wersja nr 659}

Rozwiązać nierówności $(x-5)(x-13)(x-15)\ge0$.
\zadStop
\rozwStart{Patryk Wirkus}{}
Miejsca zerowe naszego wielomianu to: $5, 13, 15$.\\
Wielomian jest stopnia nieparzystego, ponadto znak współczynnika przy\linebreak najwyższej potędze x jest dodatni.\\ W związku z tym wykres wielomianu zaczyna się od lewej strony poniżej osi OX. A więc $$x \in [5,13] \cup [15,\infty).$$
\rozwStop
\odpStart
$x \in [5,13] \cup [15,\infty)$
\odpStop
\testStart
A.$x \in [5,13] \cup [15,\infty)$\\
B.$x \in (5,13) \cup [15,\infty)$\\
C.$x \in (5,13] \cup [15,\infty)$\\
D.$x \in [5,13) \cup [15,\infty)$\\
E.$x \in [5,13] \cup (15,\infty)$\\
F.$x \in (5,13) \cup (15,\infty)$\\
G.$x \in [5,13) \cup (15,\infty)$\\
H.$x \in (5,13] \cup (15,\infty)$
\testStop
\kluczStart
A
\kluczStop



\zadStart{Zadanie z Wikieł Z 1.62 a) moja wersja nr 660}

Rozwiązać nierówności $(x-5)(x-13)(x-16)\ge0$.
\zadStop
\rozwStart{Patryk Wirkus}{}
Miejsca zerowe naszego wielomianu to: $5, 13, 16$.\\
Wielomian jest stopnia nieparzystego, ponadto znak współczynnika przy\linebreak najwyższej potędze x jest dodatni.\\ W związku z tym wykres wielomianu zaczyna się od lewej strony poniżej osi OX. A więc $$x \in [5,13] \cup [16,\infty).$$
\rozwStop
\odpStart
$x \in [5,13] \cup [16,\infty)$
\odpStop
\testStart
A.$x \in [5,13] \cup [16,\infty)$\\
B.$x \in (5,13) \cup [16,\infty)$\\
C.$x \in (5,13] \cup [16,\infty)$\\
D.$x \in [5,13) \cup [16,\infty)$\\
E.$x \in [5,13] \cup (16,\infty)$\\
F.$x \in (5,13) \cup (16,\infty)$\\
G.$x \in [5,13) \cup (16,\infty)$\\
H.$x \in (5,13] \cup (16,\infty)$
\testStop
\kluczStart
A
\kluczStop



\zadStart{Zadanie z Wikieł Z 1.62 a) moja wersja nr 661}

Rozwiązać nierówności $(x-5)(x-13)(x-17)\ge0$.
\zadStop
\rozwStart{Patryk Wirkus}{}
Miejsca zerowe naszego wielomianu to: $5, 13, 17$.\\
Wielomian jest stopnia nieparzystego, ponadto znak współczynnika przy\linebreak najwyższej potędze x jest dodatni.\\ W związku z tym wykres wielomianu zaczyna się od lewej strony poniżej osi OX. A więc $$x \in [5,13] \cup [17,\infty).$$
\rozwStop
\odpStart
$x \in [5,13] \cup [17,\infty)$
\odpStop
\testStart
A.$x \in [5,13] \cup [17,\infty)$\\
B.$x \in (5,13) \cup [17,\infty)$\\
C.$x \in (5,13] \cup [17,\infty)$\\
D.$x \in [5,13) \cup [17,\infty)$\\
E.$x \in [5,13] \cup (17,\infty)$\\
F.$x \in (5,13) \cup (17,\infty)$\\
G.$x \in [5,13) \cup (17,\infty)$\\
H.$x \in (5,13] \cup (17,\infty)$
\testStop
\kluczStart
A
\kluczStop



\zadStart{Zadanie z Wikieł Z 1.62 a) moja wersja nr 662}

Rozwiązać nierówności $(x-5)(x-13)(x-18)\ge0$.
\zadStop
\rozwStart{Patryk Wirkus}{}
Miejsca zerowe naszego wielomianu to: $5, 13, 18$.\\
Wielomian jest stopnia nieparzystego, ponadto znak współczynnika przy\linebreak najwyższej potędze x jest dodatni.\\ W związku z tym wykres wielomianu zaczyna się od lewej strony poniżej osi OX. A więc $$x \in [5,13] \cup [18,\infty).$$
\rozwStop
\odpStart
$x \in [5,13] \cup [18,\infty)$
\odpStop
\testStart
A.$x \in [5,13] \cup [18,\infty)$\\
B.$x \in (5,13) \cup [18,\infty)$\\
C.$x \in (5,13] \cup [18,\infty)$\\
D.$x \in [5,13) \cup [18,\infty)$\\
E.$x \in [5,13] \cup (18,\infty)$\\
F.$x \in (5,13) \cup (18,\infty)$\\
G.$x \in [5,13) \cup (18,\infty)$\\
H.$x \in (5,13] \cup (18,\infty)$
\testStop
\kluczStart
A
\kluczStop



\zadStart{Zadanie z Wikieł Z 1.62 a) moja wersja nr 663}

Rozwiązać nierówności $(x-5)(x-13)(x-19)\ge0$.
\zadStop
\rozwStart{Patryk Wirkus}{}
Miejsca zerowe naszego wielomianu to: $5, 13, 19$.\\
Wielomian jest stopnia nieparzystego, ponadto znak współczynnika przy\linebreak najwyższej potędze x jest dodatni.\\ W związku z tym wykres wielomianu zaczyna się od lewej strony poniżej osi OX. A więc $$x \in [5,13] \cup [19,\infty).$$
\rozwStop
\odpStart
$x \in [5,13] \cup [19,\infty)$
\odpStop
\testStart
A.$x \in [5,13] \cup [19,\infty)$\\
B.$x \in (5,13) \cup [19,\infty)$\\
C.$x \in (5,13] \cup [19,\infty)$\\
D.$x \in [5,13) \cup [19,\infty)$\\
E.$x \in [5,13] \cup (19,\infty)$\\
F.$x \in (5,13) \cup (19,\infty)$\\
G.$x \in [5,13) \cup (19,\infty)$\\
H.$x \in (5,13] \cup (19,\infty)$
\testStop
\kluczStart
A
\kluczStop



\zadStart{Zadanie z Wikieł Z 1.62 a) moja wersja nr 664}

Rozwiązać nierówności $(x-5)(x-13)(x-20)\ge0$.
\zadStop
\rozwStart{Patryk Wirkus}{}
Miejsca zerowe naszego wielomianu to: $5, 13, 20$.\\
Wielomian jest stopnia nieparzystego, ponadto znak współczynnika przy\linebreak najwyższej potędze x jest dodatni.\\ W związku z tym wykres wielomianu zaczyna się od lewej strony poniżej osi OX. A więc $$x \in [5,13] \cup [20,\infty).$$
\rozwStop
\odpStart
$x \in [5,13] \cup [20,\infty)$
\odpStop
\testStart
A.$x \in [5,13] \cup [20,\infty)$\\
B.$x \in (5,13) \cup [20,\infty)$\\
C.$x \in (5,13] \cup [20,\infty)$\\
D.$x \in [5,13) \cup [20,\infty)$\\
E.$x \in [5,13] \cup (20,\infty)$\\
F.$x \in (5,13) \cup (20,\infty)$\\
G.$x \in [5,13) \cup (20,\infty)$\\
H.$x \in (5,13] \cup (20,\infty)$
\testStop
\kluczStart
A
\kluczStop



\zadStart{Zadanie z Wikieł Z 1.62 a) moja wersja nr 665}

Rozwiązać nierówności $(x-5)(x-14)(x-15)\ge0$.
\zadStop
\rozwStart{Patryk Wirkus}{}
Miejsca zerowe naszego wielomianu to: $5, 14, 15$.\\
Wielomian jest stopnia nieparzystego, ponadto znak współczynnika przy\linebreak najwyższej potędze x jest dodatni.\\ W związku z tym wykres wielomianu zaczyna się od lewej strony poniżej osi OX. A więc $$x \in [5,14] \cup [15,\infty).$$
\rozwStop
\odpStart
$x \in [5,14] \cup [15,\infty)$
\odpStop
\testStart
A.$x \in [5,14] \cup [15,\infty)$\\
B.$x \in (5,14) \cup [15,\infty)$\\
C.$x \in (5,14] \cup [15,\infty)$\\
D.$x \in [5,14) \cup [15,\infty)$\\
E.$x \in [5,14] \cup (15,\infty)$\\
F.$x \in (5,14) \cup (15,\infty)$\\
G.$x \in [5,14) \cup (15,\infty)$\\
H.$x \in (5,14] \cup (15,\infty)$
\testStop
\kluczStart
A
\kluczStop



\zadStart{Zadanie z Wikieł Z 1.62 a) moja wersja nr 666}

Rozwiązać nierówności $(x-5)(x-14)(x-16)\ge0$.
\zadStop
\rozwStart{Patryk Wirkus}{}
Miejsca zerowe naszego wielomianu to: $5, 14, 16$.\\
Wielomian jest stopnia nieparzystego, ponadto znak współczynnika przy\linebreak najwyższej potędze x jest dodatni.\\ W związku z tym wykres wielomianu zaczyna się od lewej strony poniżej osi OX. A więc $$x \in [5,14] \cup [16,\infty).$$
\rozwStop
\odpStart
$x \in [5,14] \cup [16,\infty)$
\odpStop
\testStart
A.$x \in [5,14] \cup [16,\infty)$\\
B.$x \in (5,14) \cup [16,\infty)$\\
C.$x \in (5,14] \cup [16,\infty)$\\
D.$x \in [5,14) \cup [16,\infty)$\\
E.$x \in [5,14] \cup (16,\infty)$\\
F.$x \in (5,14) \cup (16,\infty)$\\
G.$x \in [5,14) \cup (16,\infty)$\\
H.$x \in (5,14] \cup (16,\infty)$
\testStop
\kluczStart
A
\kluczStop



\zadStart{Zadanie z Wikieł Z 1.62 a) moja wersja nr 667}

Rozwiązać nierówności $(x-5)(x-14)(x-17)\ge0$.
\zadStop
\rozwStart{Patryk Wirkus}{}
Miejsca zerowe naszego wielomianu to: $5, 14, 17$.\\
Wielomian jest stopnia nieparzystego, ponadto znak współczynnika przy\linebreak najwyższej potędze x jest dodatni.\\ W związku z tym wykres wielomianu zaczyna się od lewej strony poniżej osi OX. A więc $$x \in [5,14] \cup [17,\infty).$$
\rozwStop
\odpStart
$x \in [5,14] \cup [17,\infty)$
\odpStop
\testStart
A.$x \in [5,14] \cup [17,\infty)$\\
B.$x \in (5,14) \cup [17,\infty)$\\
C.$x \in (5,14] \cup [17,\infty)$\\
D.$x \in [5,14) \cup [17,\infty)$\\
E.$x \in [5,14] \cup (17,\infty)$\\
F.$x \in (5,14) \cup (17,\infty)$\\
G.$x \in [5,14) \cup (17,\infty)$\\
H.$x \in (5,14] \cup (17,\infty)$
\testStop
\kluczStart
A
\kluczStop



\zadStart{Zadanie z Wikieł Z 1.62 a) moja wersja nr 668}

Rozwiązać nierówności $(x-5)(x-14)(x-18)\ge0$.
\zadStop
\rozwStart{Patryk Wirkus}{}
Miejsca zerowe naszego wielomianu to: $5, 14, 18$.\\
Wielomian jest stopnia nieparzystego, ponadto znak współczynnika przy\linebreak najwyższej potędze x jest dodatni.\\ W związku z tym wykres wielomianu zaczyna się od lewej strony poniżej osi OX. A więc $$x \in [5,14] \cup [18,\infty).$$
\rozwStop
\odpStart
$x \in [5,14] \cup [18,\infty)$
\odpStop
\testStart
A.$x \in [5,14] \cup [18,\infty)$\\
B.$x \in (5,14) \cup [18,\infty)$\\
C.$x \in (5,14] \cup [18,\infty)$\\
D.$x \in [5,14) \cup [18,\infty)$\\
E.$x \in [5,14] \cup (18,\infty)$\\
F.$x \in (5,14) \cup (18,\infty)$\\
G.$x \in [5,14) \cup (18,\infty)$\\
H.$x \in (5,14] \cup (18,\infty)$
\testStop
\kluczStart
A
\kluczStop



\zadStart{Zadanie z Wikieł Z 1.62 a) moja wersja nr 669}

Rozwiązać nierówności $(x-5)(x-14)(x-19)\ge0$.
\zadStop
\rozwStart{Patryk Wirkus}{}
Miejsca zerowe naszego wielomianu to: $5, 14, 19$.\\
Wielomian jest stopnia nieparzystego, ponadto znak współczynnika przy\linebreak najwyższej potędze x jest dodatni.\\ W związku z tym wykres wielomianu zaczyna się od lewej strony poniżej osi OX. A więc $$x \in [5,14] \cup [19,\infty).$$
\rozwStop
\odpStart
$x \in [5,14] \cup [19,\infty)$
\odpStop
\testStart
A.$x \in [5,14] \cup [19,\infty)$\\
B.$x \in (5,14) \cup [19,\infty)$\\
C.$x \in (5,14] \cup [19,\infty)$\\
D.$x \in [5,14) \cup [19,\infty)$\\
E.$x \in [5,14] \cup (19,\infty)$\\
F.$x \in (5,14) \cup (19,\infty)$\\
G.$x \in [5,14) \cup (19,\infty)$\\
H.$x \in (5,14] \cup (19,\infty)$
\testStop
\kluczStart
A
\kluczStop



\zadStart{Zadanie z Wikieł Z 1.62 a) moja wersja nr 670}

Rozwiązać nierówności $(x-5)(x-14)(x-20)\ge0$.
\zadStop
\rozwStart{Patryk Wirkus}{}
Miejsca zerowe naszego wielomianu to: $5, 14, 20$.\\
Wielomian jest stopnia nieparzystego, ponadto znak współczynnika przy\linebreak najwyższej potędze x jest dodatni.\\ W związku z tym wykres wielomianu zaczyna się od lewej strony poniżej osi OX. A więc $$x \in [5,14] \cup [20,\infty).$$
\rozwStop
\odpStart
$x \in [5,14] \cup [20,\infty)$
\odpStop
\testStart
A.$x \in [5,14] \cup [20,\infty)$\\
B.$x \in (5,14) \cup [20,\infty)$\\
C.$x \in (5,14] \cup [20,\infty)$\\
D.$x \in [5,14) \cup [20,\infty)$\\
E.$x \in [5,14] \cup (20,\infty)$\\
F.$x \in (5,14) \cup (20,\infty)$\\
G.$x \in [5,14) \cup (20,\infty)$\\
H.$x \in (5,14] \cup (20,\infty)$
\testStop
\kluczStart
A
\kluczStop



\zadStart{Zadanie z Wikieł Z 1.62 a) moja wersja nr 671}

Rozwiązać nierówności $(x-5)(x-15)(x-16)\ge0$.
\zadStop
\rozwStart{Patryk Wirkus}{}
Miejsca zerowe naszego wielomianu to: $5, 15, 16$.\\
Wielomian jest stopnia nieparzystego, ponadto znak współczynnika przy\linebreak najwyższej potędze x jest dodatni.\\ W związku z tym wykres wielomianu zaczyna się od lewej strony poniżej osi OX. A więc $$x \in [5,15] \cup [16,\infty).$$
\rozwStop
\odpStart
$x \in [5,15] \cup [16,\infty)$
\odpStop
\testStart
A.$x \in [5,15] \cup [16,\infty)$\\
B.$x \in (5,15) \cup [16,\infty)$\\
C.$x \in (5,15] \cup [16,\infty)$\\
D.$x \in [5,15) \cup [16,\infty)$\\
E.$x \in [5,15] \cup (16,\infty)$\\
F.$x \in (5,15) \cup (16,\infty)$\\
G.$x \in [5,15) \cup (16,\infty)$\\
H.$x \in (5,15] \cup (16,\infty)$
\testStop
\kluczStart
A
\kluczStop



\zadStart{Zadanie z Wikieł Z 1.62 a) moja wersja nr 672}

Rozwiązać nierówności $(x-5)(x-15)(x-17)\ge0$.
\zadStop
\rozwStart{Patryk Wirkus}{}
Miejsca zerowe naszego wielomianu to: $5, 15, 17$.\\
Wielomian jest stopnia nieparzystego, ponadto znak współczynnika przy\linebreak najwyższej potędze x jest dodatni.\\ W związku z tym wykres wielomianu zaczyna się od lewej strony poniżej osi OX. A więc $$x \in [5,15] \cup [17,\infty).$$
\rozwStop
\odpStart
$x \in [5,15] \cup [17,\infty)$
\odpStop
\testStart
A.$x \in [5,15] \cup [17,\infty)$\\
B.$x \in (5,15) \cup [17,\infty)$\\
C.$x \in (5,15] \cup [17,\infty)$\\
D.$x \in [5,15) \cup [17,\infty)$\\
E.$x \in [5,15] \cup (17,\infty)$\\
F.$x \in (5,15) \cup (17,\infty)$\\
G.$x \in [5,15) \cup (17,\infty)$\\
H.$x \in (5,15] \cup (17,\infty)$
\testStop
\kluczStart
A
\kluczStop



\zadStart{Zadanie z Wikieł Z 1.62 a) moja wersja nr 673}

Rozwiązać nierówności $(x-5)(x-15)(x-18)\ge0$.
\zadStop
\rozwStart{Patryk Wirkus}{}
Miejsca zerowe naszego wielomianu to: $5, 15, 18$.\\
Wielomian jest stopnia nieparzystego, ponadto znak współczynnika przy\linebreak najwyższej potędze x jest dodatni.\\ W związku z tym wykres wielomianu zaczyna się od lewej strony poniżej osi OX. A więc $$x \in [5,15] \cup [18,\infty).$$
\rozwStop
\odpStart
$x \in [5,15] \cup [18,\infty)$
\odpStop
\testStart
A.$x \in [5,15] \cup [18,\infty)$\\
B.$x \in (5,15) \cup [18,\infty)$\\
C.$x \in (5,15] \cup [18,\infty)$\\
D.$x \in [5,15) \cup [18,\infty)$\\
E.$x \in [5,15] \cup (18,\infty)$\\
F.$x \in (5,15) \cup (18,\infty)$\\
G.$x \in [5,15) \cup (18,\infty)$\\
H.$x \in (5,15] \cup (18,\infty)$
\testStop
\kluczStart
A
\kluczStop



\zadStart{Zadanie z Wikieł Z 1.62 a) moja wersja nr 674}

Rozwiązać nierówności $(x-5)(x-15)(x-19)\ge0$.
\zadStop
\rozwStart{Patryk Wirkus}{}
Miejsca zerowe naszego wielomianu to: $5, 15, 19$.\\
Wielomian jest stopnia nieparzystego, ponadto znak współczynnika przy\linebreak najwyższej potędze x jest dodatni.\\ W związku z tym wykres wielomianu zaczyna się od lewej strony poniżej osi OX. A więc $$x \in [5,15] \cup [19,\infty).$$
\rozwStop
\odpStart
$x \in [5,15] \cup [19,\infty)$
\odpStop
\testStart
A.$x \in [5,15] \cup [19,\infty)$\\
B.$x \in (5,15) \cup [19,\infty)$\\
C.$x \in (5,15] \cup [19,\infty)$\\
D.$x \in [5,15) \cup [19,\infty)$\\
E.$x \in [5,15] \cup (19,\infty)$\\
F.$x \in (5,15) \cup (19,\infty)$\\
G.$x \in [5,15) \cup (19,\infty)$\\
H.$x \in (5,15] \cup (19,\infty)$
\testStop
\kluczStart
A
\kluczStop



\zadStart{Zadanie z Wikieł Z 1.62 a) moja wersja nr 675}

Rozwiązać nierówności $(x-5)(x-15)(x-20)\ge0$.
\zadStop
\rozwStart{Patryk Wirkus}{}
Miejsca zerowe naszego wielomianu to: $5, 15, 20$.\\
Wielomian jest stopnia nieparzystego, ponadto znak współczynnika przy\linebreak najwyższej potędze x jest dodatni.\\ W związku z tym wykres wielomianu zaczyna się od lewej strony poniżej osi OX. A więc $$x \in [5,15] \cup [20,\infty).$$
\rozwStop
\odpStart
$x \in [5,15] \cup [20,\infty)$
\odpStop
\testStart
A.$x \in [5,15] \cup [20,\infty)$\\
B.$x \in (5,15) \cup [20,\infty)$\\
C.$x \in (5,15] \cup [20,\infty)$\\
D.$x \in [5,15) \cup [20,\infty)$\\
E.$x \in [5,15] \cup (20,\infty)$\\
F.$x \in (5,15) \cup (20,\infty)$\\
G.$x \in [5,15) \cup (20,\infty)$\\
H.$x \in (5,15] \cup (20,\infty)$
\testStop
\kluczStart
A
\kluczStop



\zadStart{Zadanie z Wikieł Z 1.62 a) moja wersja nr 676}

Rozwiązać nierówności $(x-5)(x-16)(x-17)\ge0$.
\zadStop
\rozwStart{Patryk Wirkus}{}
Miejsca zerowe naszego wielomianu to: $5, 16, 17$.\\
Wielomian jest stopnia nieparzystego, ponadto znak współczynnika przy\linebreak najwyższej potędze x jest dodatni.\\ W związku z tym wykres wielomianu zaczyna się od lewej strony poniżej osi OX. A więc $$x \in [5,16] \cup [17,\infty).$$
\rozwStop
\odpStart
$x \in [5,16] \cup [17,\infty)$
\odpStop
\testStart
A.$x \in [5,16] \cup [17,\infty)$\\
B.$x \in (5,16) \cup [17,\infty)$\\
C.$x \in (5,16] \cup [17,\infty)$\\
D.$x \in [5,16) \cup [17,\infty)$\\
E.$x \in [5,16] \cup (17,\infty)$\\
F.$x \in (5,16) \cup (17,\infty)$\\
G.$x \in [5,16) \cup (17,\infty)$\\
H.$x \in (5,16] \cup (17,\infty)$
\testStop
\kluczStart
A
\kluczStop



\zadStart{Zadanie z Wikieł Z 1.62 a) moja wersja nr 677}

Rozwiązać nierówności $(x-5)(x-16)(x-18)\ge0$.
\zadStop
\rozwStart{Patryk Wirkus}{}
Miejsca zerowe naszego wielomianu to: $5, 16, 18$.\\
Wielomian jest stopnia nieparzystego, ponadto znak współczynnika przy\linebreak najwyższej potędze x jest dodatni.\\ W związku z tym wykres wielomianu zaczyna się od lewej strony poniżej osi OX. A więc $$x \in [5,16] \cup [18,\infty).$$
\rozwStop
\odpStart
$x \in [5,16] \cup [18,\infty)$
\odpStop
\testStart
A.$x \in [5,16] \cup [18,\infty)$\\
B.$x \in (5,16) \cup [18,\infty)$\\
C.$x \in (5,16] \cup [18,\infty)$\\
D.$x \in [5,16) \cup [18,\infty)$\\
E.$x \in [5,16] \cup (18,\infty)$\\
F.$x \in (5,16) \cup (18,\infty)$\\
G.$x \in [5,16) \cup (18,\infty)$\\
H.$x \in (5,16] \cup (18,\infty)$
\testStop
\kluczStart
A
\kluczStop



\zadStart{Zadanie z Wikieł Z 1.62 a) moja wersja nr 678}

Rozwiązać nierówności $(x-5)(x-16)(x-19)\ge0$.
\zadStop
\rozwStart{Patryk Wirkus}{}
Miejsca zerowe naszego wielomianu to: $5, 16, 19$.\\
Wielomian jest stopnia nieparzystego, ponadto znak współczynnika przy\linebreak najwyższej potędze x jest dodatni.\\ W związku z tym wykres wielomianu zaczyna się od lewej strony poniżej osi OX. A więc $$x \in [5,16] \cup [19,\infty).$$
\rozwStop
\odpStart
$x \in [5,16] \cup [19,\infty)$
\odpStop
\testStart
A.$x \in [5,16] \cup [19,\infty)$\\
B.$x \in (5,16) \cup [19,\infty)$\\
C.$x \in (5,16] \cup [19,\infty)$\\
D.$x \in [5,16) \cup [19,\infty)$\\
E.$x \in [5,16] \cup (19,\infty)$\\
F.$x \in (5,16) \cup (19,\infty)$\\
G.$x \in [5,16) \cup (19,\infty)$\\
H.$x \in (5,16] \cup (19,\infty)$
\testStop
\kluczStart
A
\kluczStop



\zadStart{Zadanie z Wikieł Z 1.62 a) moja wersja nr 679}

Rozwiązać nierówności $(x-5)(x-16)(x-20)\ge0$.
\zadStop
\rozwStart{Patryk Wirkus}{}
Miejsca zerowe naszego wielomianu to: $5, 16, 20$.\\
Wielomian jest stopnia nieparzystego, ponadto znak współczynnika przy\linebreak najwyższej potędze x jest dodatni.\\ W związku z tym wykres wielomianu zaczyna się od lewej strony poniżej osi OX. A więc $$x \in [5,16] \cup [20,\infty).$$
\rozwStop
\odpStart
$x \in [5,16] \cup [20,\infty)$
\odpStop
\testStart
A.$x \in [5,16] \cup [20,\infty)$\\
B.$x \in (5,16) \cup [20,\infty)$\\
C.$x \in (5,16] \cup [20,\infty)$\\
D.$x \in [5,16) \cup [20,\infty)$\\
E.$x \in [5,16] \cup (20,\infty)$\\
F.$x \in (5,16) \cup (20,\infty)$\\
G.$x \in [5,16) \cup (20,\infty)$\\
H.$x \in (5,16] \cup (20,\infty)$
\testStop
\kluczStart
A
\kluczStop



\zadStart{Zadanie z Wikieł Z 1.62 a) moja wersja nr 680}

Rozwiązać nierówności $(x-5)(x-17)(x-18)\ge0$.
\zadStop
\rozwStart{Patryk Wirkus}{}
Miejsca zerowe naszego wielomianu to: $5, 17, 18$.\\
Wielomian jest stopnia nieparzystego, ponadto znak współczynnika przy\linebreak najwyższej potędze x jest dodatni.\\ W związku z tym wykres wielomianu zaczyna się od lewej strony poniżej osi OX. A więc $$x \in [5,17] \cup [18,\infty).$$
\rozwStop
\odpStart
$x \in [5,17] \cup [18,\infty)$
\odpStop
\testStart
A.$x \in [5,17] \cup [18,\infty)$\\
B.$x \in (5,17) \cup [18,\infty)$\\
C.$x \in (5,17] \cup [18,\infty)$\\
D.$x \in [5,17) \cup [18,\infty)$\\
E.$x \in [5,17] \cup (18,\infty)$\\
F.$x \in (5,17) \cup (18,\infty)$\\
G.$x \in [5,17) \cup (18,\infty)$\\
H.$x \in (5,17] \cup (18,\infty)$
\testStop
\kluczStart
A
\kluczStop



\zadStart{Zadanie z Wikieł Z 1.62 a) moja wersja nr 681}

Rozwiązać nierówności $(x-5)(x-17)(x-19)\ge0$.
\zadStop
\rozwStart{Patryk Wirkus}{}
Miejsca zerowe naszego wielomianu to: $5, 17, 19$.\\
Wielomian jest stopnia nieparzystego, ponadto znak współczynnika przy\linebreak najwyższej potędze x jest dodatni.\\ W związku z tym wykres wielomianu zaczyna się od lewej strony poniżej osi OX. A więc $$x \in [5,17] \cup [19,\infty).$$
\rozwStop
\odpStart
$x \in [5,17] \cup [19,\infty)$
\odpStop
\testStart
A.$x \in [5,17] \cup [19,\infty)$\\
B.$x \in (5,17) \cup [19,\infty)$\\
C.$x \in (5,17] \cup [19,\infty)$\\
D.$x \in [5,17) \cup [19,\infty)$\\
E.$x \in [5,17] \cup (19,\infty)$\\
F.$x \in (5,17) \cup (19,\infty)$\\
G.$x \in [5,17) \cup (19,\infty)$\\
H.$x \in (5,17] \cup (19,\infty)$
\testStop
\kluczStart
A
\kluczStop



\zadStart{Zadanie z Wikieł Z 1.62 a) moja wersja nr 682}

Rozwiązać nierówności $(x-5)(x-17)(x-20)\ge0$.
\zadStop
\rozwStart{Patryk Wirkus}{}
Miejsca zerowe naszego wielomianu to: $5, 17, 20$.\\
Wielomian jest stopnia nieparzystego, ponadto znak współczynnika przy\linebreak najwyższej potędze x jest dodatni.\\ W związku z tym wykres wielomianu zaczyna się od lewej strony poniżej osi OX. A więc $$x \in [5,17] \cup [20,\infty).$$
\rozwStop
\odpStart
$x \in [5,17] \cup [20,\infty)$
\odpStop
\testStart
A.$x \in [5,17] \cup [20,\infty)$\\
B.$x \in (5,17) \cup [20,\infty)$\\
C.$x \in (5,17] \cup [20,\infty)$\\
D.$x \in [5,17) \cup [20,\infty)$\\
E.$x \in [5,17] \cup (20,\infty)$\\
F.$x \in (5,17) \cup (20,\infty)$\\
G.$x \in [5,17) \cup (20,\infty)$\\
H.$x \in (5,17] \cup (20,\infty)$
\testStop
\kluczStart
A
\kluczStop



\zadStart{Zadanie z Wikieł Z 1.62 a) moja wersja nr 683}

Rozwiązać nierówności $(x-5)(x-18)(x-19)\ge0$.
\zadStop
\rozwStart{Patryk Wirkus}{}
Miejsca zerowe naszego wielomianu to: $5, 18, 19$.\\
Wielomian jest stopnia nieparzystego, ponadto znak współczynnika przy\linebreak najwyższej potędze x jest dodatni.\\ W związku z tym wykres wielomianu zaczyna się od lewej strony poniżej osi OX. A więc $$x \in [5,18] \cup [19,\infty).$$
\rozwStop
\odpStart
$x \in [5,18] \cup [19,\infty)$
\odpStop
\testStart
A.$x \in [5,18] \cup [19,\infty)$\\
B.$x \in (5,18) \cup [19,\infty)$\\
C.$x \in (5,18] \cup [19,\infty)$\\
D.$x \in [5,18) \cup [19,\infty)$\\
E.$x \in [5,18] \cup (19,\infty)$\\
F.$x \in (5,18) \cup (19,\infty)$\\
G.$x \in [5,18) \cup (19,\infty)$\\
H.$x \in (5,18] \cup (19,\infty)$
\testStop
\kluczStart
A
\kluczStop



\zadStart{Zadanie z Wikieł Z 1.62 a) moja wersja nr 684}

Rozwiązać nierówności $(x-5)(x-18)(x-20)\ge0$.
\zadStop
\rozwStart{Patryk Wirkus}{}
Miejsca zerowe naszego wielomianu to: $5, 18, 20$.\\
Wielomian jest stopnia nieparzystego, ponadto znak współczynnika przy\linebreak najwyższej potędze x jest dodatni.\\ W związku z tym wykres wielomianu zaczyna się od lewej strony poniżej osi OX. A więc $$x \in [5,18] \cup [20,\infty).$$
\rozwStop
\odpStart
$x \in [5,18] \cup [20,\infty)$
\odpStop
\testStart
A.$x \in [5,18] \cup [20,\infty)$\\
B.$x \in (5,18) \cup [20,\infty)$\\
C.$x \in (5,18] \cup [20,\infty)$\\
D.$x \in [5,18) \cup [20,\infty)$\\
E.$x \in [5,18] \cup (20,\infty)$\\
F.$x \in (5,18) \cup (20,\infty)$\\
G.$x \in [5,18) \cup (20,\infty)$\\
H.$x \in (5,18] \cup (20,\infty)$
\testStop
\kluczStart
A
\kluczStop



\zadStart{Zadanie z Wikieł Z 1.62 a) moja wersja nr 685}

Rozwiązać nierówności $(x-5)(x-19)(x-20)\ge0$.
\zadStop
\rozwStart{Patryk Wirkus}{}
Miejsca zerowe naszego wielomianu to: $5, 19, 20$.\\
Wielomian jest stopnia nieparzystego, ponadto znak współczynnika przy\linebreak najwyższej potędze x jest dodatni.\\ W związku z tym wykres wielomianu zaczyna się od lewej strony poniżej osi OX. A więc $$x \in [5,19] \cup [20,\infty).$$
\rozwStop
\odpStart
$x \in [5,19] \cup [20,\infty)$
\odpStop
\testStart
A.$x \in [5,19] \cup [20,\infty)$\\
B.$x \in (5,19) \cup [20,\infty)$\\
C.$x \in (5,19] \cup [20,\infty)$\\
D.$x \in [5,19) \cup [20,\infty)$\\
E.$x \in [5,19] \cup (20,\infty)$\\
F.$x \in (5,19) \cup (20,\infty)$\\
G.$x \in [5,19) \cup (20,\infty)$\\
H.$x \in (5,19] \cup (20,\infty)$
\testStop
\kluczStart
A
\kluczStop



\zadStart{Zadanie z Wikieł Z 1.62 a) moja wersja nr 686}

Rozwiązać nierówności $(x-6)(x-7)(x-8)\ge0$.
\zadStop
\rozwStart{Patryk Wirkus}{}
Miejsca zerowe naszego wielomianu to: $6, 7, 8$.\\
Wielomian jest stopnia nieparzystego, ponadto znak współczynnika przy\linebreak najwyższej potędze x jest dodatni.\\ W związku z tym wykres wielomianu zaczyna się od lewej strony poniżej osi OX. A więc $$x \in [6,7] \cup [8,\infty).$$
\rozwStop
\odpStart
$x \in [6,7] \cup [8,\infty)$
\odpStop
\testStart
A.$x \in [6,7] \cup [8,\infty)$\\
B.$x \in (6,7) \cup [8,\infty)$\\
C.$x \in (6,7] \cup [8,\infty)$\\
D.$x \in [6,7) \cup [8,\infty)$\\
E.$x \in [6,7] \cup (8,\infty)$\\
F.$x \in (6,7) \cup (8,\infty)$\\
G.$x \in [6,7) \cup (8,\infty)$\\
H.$x \in (6,7] \cup (8,\infty)$
\testStop
\kluczStart
A
\kluczStop



\zadStart{Zadanie z Wikieł Z 1.62 a) moja wersja nr 687}

Rozwiązać nierówności $(x-6)(x-7)(x-9)\ge0$.
\zadStop
\rozwStart{Patryk Wirkus}{}
Miejsca zerowe naszego wielomianu to: $6, 7, 9$.\\
Wielomian jest stopnia nieparzystego, ponadto znak współczynnika przy\linebreak najwyższej potędze x jest dodatni.\\ W związku z tym wykres wielomianu zaczyna się od lewej strony poniżej osi OX. A więc $$x \in [6,7] \cup [9,\infty).$$
\rozwStop
\odpStart
$x \in [6,7] \cup [9,\infty)$
\odpStop
\testStart
A.$x \in [6,7] \cup [9,\infty)$\\
B.$x \in (6,7) \cup [9,\infty)$\\
C.$x \in (6,7] \cup [9,\infty)$\\
D.$x \in [6,7) \cup [9,\infty)$\\
E.$x \in [6,7] \cup (9,\infty)$\\
F.$x \in (6,7) \cup (9,\infty)$\\
G.$x \in [6,7) \cup (9,\infty)$\\
H.$x \in (6,7] \cup (9,\infty)$
\testStop
\kluczStart
A
\kluczStop



\zadStart{Zadanie z Wikieł Z 1.62 a) moja wersja nr 688}

Rozwiązać nierówności $(x-6)(x-7)(x-10)\ge0$.
\zadStop
\rozwStart{Patryk Wirkus}{}
Miejsca zerowe naszego wielomianu to: $6, 7, 10$.\\
Wielomian jest stopnia nieparzystego, ponadto znak współczynnika przy\linebreak najwyższej potędze x jest dodatni.\\ W związku z tym wykres wielomianu zaczyna się od lewej strony poniżej osi OX. A więc $$x \in [6,7] \cup [10,\infty).$$
\rozwStop
\odpStart
$x \in [6,7] \cup [10,\infty)$
\odpStop
\testStart
A.$x \in [6,7] \cup [10,\infty)$\\
B.$x \in (6,7) \cup [10,\infty)$\\
C.$x \in (6,7] \cup [10,\infty)$\\
D.$x \in [6,7) \cup [10,\infty)$\\
E.$x \in [6,7] \cup (10,\infty)$\\
F.$x \in (6,7) \cup (10,\infty)$\\
G.$x \in [6,7) \cup (10,\infty)$\\
H.$x \in (6,7] \cup (10,\infty)$
\testStop
\kluczStart
A
\kluczStop



\zadStart{Zadanie z Wikieł Z 1.62 a) moja wersja nr 689}

Rozwiązać nierówności $(x-6)(x-7)(x-11)\ge0$.
\zadStop
\rozwStart{Patryk Wirkus}{}
Miejsca zerowe naszego wielomianu to: $6, 7, 11$.\\
Wielomian jest stopnia nieparzystego, ponadto znak współczynnika przy\linebreak najwyższej potędze x jest dodatni.\\ W związku z tym wykres wielomianu zaczyna się od lewej strony poniżej osi OX. A więc $$x \in [6,7] \cup [11,\infty).$$
\rozwStop
\odpStart
$x \in [6,7] \cup [11,\infty)$
\odpStop
\testStart
A.$x \in [6,7] \cup [11,\infty)$\\
B.$x \in (6,7) \cup [11,\infty)$\\
C.$x \in (6,7] \cup [11,\infty)$\\
D.$x \in [6,7) \cup [11,\infty)$\\
E.$x \in [6,7] \cup (11,\infty)$\\
F.$x \in (6,7) \cup (11,\infty)$\\
G.$x \in [6,7) \cup (11,\infty)$\\
H.$x \in (6,7] \cup (11,\infty)$
\testStop
\kluczStart
A
\kluczStop



\zadStart{Zadanie z Wikieł Z 1.62 a) moja wersja nr 690}

Rozwiązać nierówności $(x-6)(x-7)(x-12)\ge0$.
\zadStop
\rozwStart{Patryk Wirkus}{}
Miejsca zerowe naszego wielomianu to: $6, 7, 12$.\\
Wielomian jest stopnia nieparzystego, ponadto znak współczynnika przy\linebreak najwyższej potędze x jest dodatni.\\ W związku z tym wykres wielomianu zaczyna się od lewej strony poniżej osi OX. A więc $$x \in [6,7] \cup [12,\infty).$$
\rozwStop
\odpStart
$x \in [6,7] \cup [12,\infty)$
\odpStop
\testStart
A.$x \in [6,7] \cup [12,\infty)$\\
B.$x \in (6,7) \cup [12,\infty)$\\
C.$x \in (6,7] \cup [12,\infty)$\\
D.$x \in [6,7) \cup [12,\infty)$\\
E.$x \in [6,7] \cup (12,\infty)$\\
F.$x \in (6,7) \cup (12,\infty)$\\
G.$x \in [6,7) \cup (12,\infty)$\\
H.$x \in (6,7] \cup (12,\infty)$
\testStop
\kluczStart
A
\kluczStop



\zadStart{Zadanie z Wikieł Z 1.62 a) moja wersja nr 691}

Rozwiązać nierówności $(x-6)(x-7)(x-13)\ge0$.
\zadStop
\rozwStart{Patryk Wirkus}{}
Miejsca zerowe naszego wielomianu to: $6, 7, 13$.\\
Wielomian jest stopnia nieparzystego, ponadto znak współczynnika przy\linebreak najwyższej potędze x jest dodatni.\\ W związku z tym wykres wielomianu zaczyna się od lewej strony poniżej osi OX. A więc $$x \in [6,7] \cup [13,\infty).$$
\rozwStop
\odpStart
$x \in [6,7] \cup [13,\infty)$
\odpStop
\testStart
A.$x \in [6,7] \cup [13,\infty)$\\
B.$x \in (6,7) \cup [13,\infty)$\\
C.$x \in (6,7] \cup [13,\infty)$\\
D.$x \in [6,7) \cup [13,\infty)$\\
E.$x \in [6,7] \cup (13,\infty)$\\
F.$x \in (6,7) \cup (13,\infty)$\\
G.$x \in [6,7) \cup (13,\infty)$\\
H.$x \in (6,7] \cup (13,\infty)$
\testStop
\kluczStart
A
\kluczStop



\zadStart{Zadanie z Wikieł Z 1.62 a) moja wersja nr 692}

Rozwiązać nierówności $(x-6)(x-7)(x-14)\ge0$.
\zadStop
\rozwStart{Patryk Wirkus}{}
Miejsca zerowe naszego wielomianu to: $6, 7, 14$.\\
Wielomian jest stopnia nieparzystego, ponadto znak współczynnika przy\linebreak najwyższej potędze x jest dodatni.\\ W związku z tym wykres wielomianu zaczyna się od lewej strony poniżej osi OX. A więc $$x \in [6,7] \cup [14,\infty).$$
\rozwStop
\odpStart
$x \in [6,7] \cup [14,\infty)$
\odpStop
\testStart
A.$x \in [6,7] \cup [14,\infty)$\\
B.$x \in (6,7) \cup [14,\infty)$\\
C.$x \in (6,7] \cup [14,\infty)$\\
D.$x \in [6,7) \cup [14,\infty)$\\
E.$x \in [6,7] \cup (14,\infty)$\\
F.$x \in (6,7) \cup (14,\infty)$\\
G.$x \in [6,7) \cup (14,\infty)$\\
H.$x \in (6,7] \cup (14,\infty)$
\testStop
\kluczStart
A
\kluczStop



\zadStart{Zadanie z Wikieł Z 1.62 a) moja wersja nr 693}

Rozwiązać nierówności $(x-6)(x-7)(x-15)\ge0$.
\zadStop
\rozwStart{Patryk Wirkus}{}
Miejsca zerowe naszego wielomianu to: $6, 7, 15$.\\
Wielomian jest stopnia nieparzystego, ponadto znak współczynnika przy\linebreak najwyższej potędze x jest dodatni.\\ W związku z tym wykres wielomianu zaczyna się od lewej strony poniżej osi OX. A więc $$x \in [6,7] \cup [15,\infty).$$
\rozwStop
\odpStart
$x \in [6,7] \cup [15,\infty)$
\odpStop
\testStart
A.$x \in [6,7] \cup [15,\infty)$\\
B.$x \in (6,7) \cup [15,\infty)$\\
C.$x \in (6,7] \cup [15,\infty)$\\
D.$x \in [6,7) \cup [15,\infty)$\\
E.$x \in [6,7] \cup (15,\infty)$\\
F.$x \in (6,7) \cup (15,\infty)$\\
G.$x \in [6,7) \cup (15,\infty)$\\
H.$x \in (6,7] \cup (15,\infty)$
\testStop
\kluczStart
A
\kluczStop



\zadStart{Zadanie z Wikieł Z 1.62 a) moja wersja nr 694}

Rozwiązać nierówności $(x-6)(x-7)(x-16)\ge0$.
\zadStop
\rozwStart{Patryk Wirkus}{}
Miejsca zerowe naszego wielomianu to: $6, 7, 16$.\\
Wielomian jest stopnia nieparzystego, ponadto znak współczynnika przy\linebreak najwyższej potędze x jest dodatni.\\ W związku z tym wykres wielomianu zaczyna się od lewej strony poniżej osi OX. A więc $$x \in [6,7] \cup [16,\infty).$$
\rozwStop
\odpStart
$x \in [6,7] \cup [16,\infty)$
\odpStop
\testStart
A.$x \in [6,7] \cup [16,\infty)$\\
B.$x \in (6,7) \cup [16,\infty)$\\
C.$x \in (6,7] \cup [16,\infty)$\\
D.$x \in [6,7) \cup [16,\infty)$\\
E.$x \in [6,7] \cup (16,\infty)$\\
F.$x \in (6,7) \cup (16,\infty)$\\
G.$x \in [6,7) \cup (16,\infty)$\\
H.$x \in (6,7] \cup (16,\infty)$
\testStop
\kluczStart
A
\kluczStop



\zadStart{Zadanie z Wikieł Z 1.62 a) moja wersja nr 695}

Rozwiązać nierówności $(x-6)(x-7)(x-17)\ge0$.
\zadStop
\rozwStart{Patryk Wirkus}{}
Miejsca zerowe naszego wielomianu to: $6, 7, 17$.\\
Wielomian jest stopnia nieparzystego, ponadto znak współczynnika przy\linebreak najwyższej potędze x jest dodatni.\\ W związku z tym wykres wielomianu zaczyna się od lewej strony poniżej osi OX. A więc $$x \in [6,7] \cup [17,\infty).$$
\rozwStop
\odpStart
$x \in [6,7] \cup [17,\infty)$
\odpStop
\testStart
A.$x \in [6,7] \cup [17,\infty)$\\
B.$x \in (6,7) \cup [17,\infty)$\\
C.$x \in (6,7] \cup [17,\infty)$\\
D.$x \in [6,7) \cup [17,\infty)$\\
E.$x \in [6,7] \cup (17,\infty)$\\
F.$x \in (6,7) \cup (17,\infty)$\\
G.$x \in [6,7) \cup (17,\infty)$\\
H.$x \in (6,7] \cup (17,\infty)$
\testStop
\kluczStart
A
\kluczStop



\zadStart{Zadanie z Wikieł Z 1.62 a) moja wersja nr 696}

Rozwiązać nierówności $(x-6)(x-7)(x-18)\ge0$.
\zadStop
\rozwStart{Patryk Wirkus}{}
Miejsca zerowe naszego wielomianu to: $6, 7, 18$.\\
Wielomian jest stopnia nieparzystego, ponadto znak współczynnika przy\linebreak najwyższej potędze x jest dodatni.\\ W związku z tym wykres wielomianu zaczyna się od lewej strony poniżej osi OX. A więc $$x \in [6,7] \cup [18,\infty).$$
\rozwStop
\odpStart
$x \in [6,7] \cup [18,\infty)$
\odpStop
\testStart
A.$x \in [6,7] \cup [18,\infty)$\\
B.$x \in (6,7) \cup [18,\infty)$\\
C.$x \in (6,7] \cup [18,\infty)$\\
D.$x \in [6,7) \cup [18,\infty)$\\
E.$x \in [6,7] \cup (18,\infty)$\\
F.$x \in (6,7) \cup (18,\infty)$\\
G.$x \in [6,7) \cup (18,\infty)$\\
H.$x \in (6,7] \cup (18,\infty)$
\testStop
\kluczStart
A
\kluczStop



\zadStart{Zadanie z Wikieł Z 1.62 a) moja wersja nr 697}

Rozwiązać nierówności $(x-6)(x-7)(x-19)\ge0$.
\zadStop
\rozwStart{Patryk Wirkus}{}
Miejsca zerowe naszego wielomianu to: $6, 7, 19$.\\
Wielomian jest stopnia nieparzystego, ponadto znak współczynnika przy\linebreak najwyższej potędze x jest dodatni.\\ W związku z tym wykres wielomianu zaczyna się od lewej strony poniżej osi OX. A więc $$x \in [6,7] \cup [19,\infty).$$
\rozwStop
\odpStart
$x \in [6,7] \cup [19,\infty)$
\odpStop
\testStart
A.$x \in [6,7] \cup [19,\infty)$\\
B.$x \in (6,7) \cup [19,\infty)$\\
C.$x \in (6,7] \cup [19,\infty)$\\
D.$x \in [6,7) \cup [19,\infty)$\\
E.$x \in [6,7] \cup (19,\infty)$\\
F.$x \in (6,7) \cup (19,\infty)$\\
G.$x \in [6,7) \cup (19,\infty)$\\
H.$x \in (6,7] \cup (19,\infty)$
\testStop
\kluczStart
A
\kluczStop



\zadStart{Zadanie z Wikieł Z 1.62 a) moja wersja nr 698}

Rozwiązać nierówności $(x-6)(x-7)(x-20)\ge0$.
\zadStop
\rozwStart{Patryk Wirkus}{}
Miejsca zerowe naszego wielomianu to: $6, 7, 20$.\\
Wielomian jest stopnia nieparzystego, ponadto znak współczynnika przy\linebreak najwyższej potędze x jest dodatni.\\ W związku z tym wykres wielomianu zaczyna się od lewej strony poniżej osi OX. A więc $$x \in [6,7] \cup [20,\infty).$$
\rozwStop
\odpStart
$x \in [6,7] \cup [20,\infty)$
\odpStop
\testStart
A.$x \in [6,7] \cup [20,\infty)$\\
B.$x \in (6,7) \cup [20,\infty)$\\
C.$x \in (6,7] \cup [20,\infty)$\\
D.$x \in [6,7) \cup [20,\infty)$\\
E.$x \in [6,7] \cup (20,\infty)$\\
F.$x \in (6,7) \cup (20,\infty)$\\
G.$x \in [6,7) \cup (20,\infty)$\\
H.$x \in (6,7] \cup (20,\infty)$
\testStop
\kluczStart
A
\kluczStop



\zadStart{Zadanie z Wikieł Z 1.62 a) moja wersja nr 699}

Rozwiązać nierówności $(x-6)(x-8)(x-9)\ge0$.
\zadStop
\rozwStart{Patryk Wirkus}{}
Miejsca zerowe naszego wielomianu to: $6, 8, 9$.\\
Wielomian jest stopnia nieparzystego, ponadto znak współczynnika przy\linebreak najwyższej potędze x jest dodatni.\\ W związku z tym wykres wielomianu zaczyna się od lewej strony poniżej osi OX. A więc $$x \in [6,8] \cup [9,\infty).$$
\rozwStop
\odpStart
$x \in [6,8] \cup [9,\infty)$
\odpStop
\testStart
A.$x \in [6,8] \cup [9,\infty)$\\
B.$x \in (6,8) \cup [9,\infty)$\\
C.$x \in (6,8] \cup [9,\infty)$\\
D.$x \in [6,8) \cup [9,\infty)$\\
E.$x \in [6,8] \cup (9,\infty)$\\
F.$x \in (6,8) \cup (9,\infty)$\\
G.$x \in [6,8) \cup (9,\infty)$\\
H.$x \in (6,8] \cup (9,\infty)$
\testStop
\kluczStart
A
\kluczStop



\zadStart{Zadanie z Wikieł Z 1.62 a) moja wersja nr 700}

Rozwiązać nierówności $(x-6)(x-8)(x-10)\ge0$.
\zadStop
\rozwStart{Patryk Wirkus}{}
Miejsca zerowe naszego wielomianu to: $6, 8, 10$.\\
Wielomian jest stopnia nieparzystego, ponadto znak współczynnika przy\linebreak najwyższej potędze x jest dodatni.\\ W związku z tym wykres wielomianu zaczyna się od lewej strony poniżej osi OX. A więc $$x \in [6,8] \cup [10,\infty).$$
\rozwStop
\odpStart
$x \in [6,8] \cup [10,\infty)$
\odpStop
\testStart
A.$x \in [6,8] \cup [10,\infty)$\\
B.$x \in (6,8) \cup [10,\infty)$\\
C.$x \in (6,8] \cup [10,\infty)$\\
D.$x \in [6,8) \cup [10,\infty)$\\
E.$x \in [6,8] \cup (10,\infty)$\\
F.$x \in (6,8) \cup (10,\infty)$\\
G.$x \in [6,8) \cup (10,\infty)$\\
H.$x \in (6,8] \cup (10,\infty)$
\testStop
\kluczStart
A
\kluczStop



\zadStart{Zadanie z Wikieł Z 1.62 a) moja wersja nr 701}

Rozwiązać nierówności $(x-6)(x-8)(x-11)\ge0$.
\zadStop
\rozwStart{Patryk Wirkus}{}
Miejsca zerowe naszego wielomianu to: $6, 8, 11$.\\
Wielomian jest stopnia nieparzystego, ponadto znak współczynnika przy\linebreak najwyższej potędze x jest dodatni.\\ W związku z tym wykres wielomianu zaczyna się od lewej strony poniżej osi OX. A więc $$x \in [6,8] \cup [11,\infty).$$
\rozwStop
\odpStart
$x \in [6,8] \cup [11,\infty)$
\odpStop
\testStart
A.$x \in [6,8] \cup [11,\infty)$\\
B.$x \in (6,8) \cup [11,\infty)$\\
C.$x \in (6,8] \cup [11,\infty)$\\
D.$x \in [6,8) \cup [11,\infty)$\\
E.$x \in [6,8] \cup (11,\infty)$\\
F.$x \in (6,8) \cup (11,\infty)$\\
G.$x \in [6,8) \cup (11,\infty)$\\
H.$x \in (6,8] \cup (11,\infty)$
\testStop
\kluczStart
A
\kluczStop



\zadStart{Zadanie z Wikieł Z 1.62 a) moja wersja nr 702}

Rozwiązać nierówności $(x-6)(x-8)(x-12)\ge0$.
\zadStop
\rozwStart{Patryk Wirkus}{}
Miejsca zerowe naszego wielomianu to: $6, 8, 12$.\\
Wielomian jest stopnia nieparzystego, ponadto znak współczynnika przy\linebreak najwyższej potędze x jest dodatni.\\ W związku z tym wykres wielomianu zaczyna się od lewej strony poniżej osi OX. A więc $$x \in [6,8] \cup [12,\infty).$$
\rozwStop
\odpStart
$x \in [6,8] \cup [12,\infty)$
\odpStop
\testStart
A.$x \in [6,8] \cup [12,\infty)$\\
B.$x \in (6,8) \cup [12,\infty)$\\
C.$x \in (6,8] \cup [12,\infty)$\\
D.$x \in [6,8) \cup [12,\infty)$\\
E.$x \in [6,8] \cup (12,\infty)$\\
F.$x \in (6,8) \cup (12,\infty)$\\
G.$x \in [6,8) \cup (12,\infty)$\\
H.$x \in (6,8] \cup (12,\infty)$
\testStop
\kluczStart
A
\kluczStop



\zadStart{Zadanie z Wikieł Z 1.62 a) moja wersja nr 703}

Rozwiązać nierówności $(x-6)(x-8)(x-13)\ge0$.
\zadStop
\rozwStart{Patryk Wirkus}{}
Miejsca zerowe naszego wielomianu to: $6, 8, 13$.\\
Wielomian jest stopnia nieparzystego, ponadto znak współczynnika przy\linebreak najwyższej potędze x jest dodatni.\\ W związku z tym wykres wielomianu zaczyna się od lewej strony poniżej osi OX. A więc $$x \in [6,8] \cup [13,\infty).$$
\rozwStop
\odpStart
$x \in [6,8] \cup [13,\infty)$
\odpStop
\testStart
A.$x \in [6,8] \cup [13,\infty)$\\
B.$x \in (6,8) \cup [13,\infty)$\\
C.$x \in (6,8] \cup [13,\infty)$\\
D.$x \in [6,8) \cup [13,\infty)$\\
E.$x \in [6,8] \cup (13,\infty)$\\
F.$x \in (6,8) \cup (13,\infty)$\\
G.$x \in [6,8) \cup (13,\infty)$\\
H.$x \in (6,8] \cup (13,\infty)$
\testStop
\kluczStart
A
\kluczStop



\zadStart{Zadanie z Wikieł Z 1.62 a) moja wersja nr 704}

Rozwiązać nierówności $(x-6)(x-8)(x-14)\ge0$.
\zadStop
\rozwStart{Patryk Wirkus}{}
Miejsca zerowe naszego wielomianu to: $6, 8, 14$.\\
Wielomian jest stopnia nieparzystego, ponadto znak współczynnika przy\linebreak najwyższej potędze x jest dodatni.\\ W związku z tym wykres wielomianu zaczyna się od lewej strony poniżej osi OX. A więc $$x \in [6,8] \cup [14,\infty).$$
\rozwStop
\odpStart
$x \in [6,8] \cup [14,\infty)$
\odpStop
\testStart
A.$x \in [6,8] \cup [14,\infty)$\\
B.$x \in (6,8) \cup [14,\infty)$\\
C.$x \in (6,8] \cup [14,\infty)$\\
D.$x \in [6,8) \cup [14,\infty)$\\
E.$x \in [6,8] \cup (14,\infty)$\\
F.$x \in (6,8) \cup (14,\infty)$\\
G.$x \in [6,8) \cup (14,\infty)$\\
H.$x \in (6,8] \cup (14,\infty)$
\testStop
\kluczStart
A
\kluczStop



\zadStart{Zadanie z Wikieł Z 1.62 a) moja wersja nr 705}

Rozwiązać nierówności $(x-6)(x-8)(x-15)\ge0$.
\zadStop
\rozwStart{Patryk Wirkus}{}
Miejsca zerowe naszego wielomianu to: $6, 8, 15$.\\
Wielomian jest stopnia nieparzystego, ponadto znak współczynnika przy\linebreak najwyższej potędze x jest dodatni.\\ W związku z tym wykres wielomianu zaczyna się od lewej strony poniżej osi OX. A więc $$x \in [6,8] \cup [15,\infty).$$
\rozwStop
\odpStart
$x \in [6,8] \cup [15,\infty)$
\odpStop
\testStart
A.$x \in [6,8] \cup [15,\infty)$\\
B.$x \in (6,8) \cup [15,\infty)$\\
C.$x \in (6,8] \cup [15,\infty)$\\
D.$x \in [6,8) \cup [15,\infty)$\\
E.$x \in [6,8] \cup (15,\infty)$\\
F.$x \in (6,8) \cup (15,\infty)$\\
G.$x \in [6,8) \cup (15,\infty)$\\
H.$x \in (6,8] \cup (15,\infty)$
\testStop
\kluczStart
A
\kluczStop



\zadStart{Zadanie z Wikieł Z 1.62 a) moja wersja nr 706}

Rozwiązać nierówności $(x-6)(x-8)(x-16)\ge0$.
\zadStop
\rozwStart{Patryk Wirkus}{}
Miejsca zerowe naszego wielomianu to: $6, 8, 16$.\\
Wielomian jest stopnia nieparzystego, ponadto znak współczynnika przy\linebreak najwyższej potędze x jest dodatni.\\ W związku z tym wykres wielomianu zaczyna się od lewej strony poniżej osi OX. A więc $$x \in [6,8] \cup [16,\infty).$$
\rozwStop
\odpStart
$x \in [6,8] \cup [16,\infty)$
\odpStop
\testStart
A.$x \in [6,8] \cup [16,\infty)$\\
B.$x \in (6,8) \cup [16,\infty)$\\
C.$x \in (6,8] \cup [16,\infty)$\\
D.$x \in [6,8) \cup [16,\infty)$\\
E.$x \in [6,8] \cup (16,\infty)$\\
F.$x \in (6,8) \cup (16,\infty)$\\
G.$x \in [6,8) \cup (16,\infty)$\\
H.$x \in (6,8] \cup (16,\infty)$
\testStop
\kluczStart
A
\kluczStop



\zadStart{Zadanie z Wikieł Z 1.62 a) moja wersja nr 707}

Rozwiązać nierówności $(x-6)(x-8)(x-17)\ge0$.
\zadStop
\rozwStart{Patryk Wirkus}{}
Miejsca zerowe naszego wielomianu to: $6, 8, 17$.\\
Wielomian jest stopnia nieparzystego, ponadto znak współczynnika przy\linebreak najwyższej potędze x jest dodatni.\\ W związku z tym wykres wielomianu zaczyna się od lewej strony poniżej osi OX. A więc $$x \in [6,8] \cup [17,\infty).$$
\rozwStop
\odpStart
$x \in [6,8] \cup [17,\infty)$
\odpStop
\testStart
A.$x \in [6,8] \cup [17,\infty)$\\
B.$x \in (6,8) \cup [17,\infty)$\\
C.$x \in (6,8] \cup [17,\infty)$\\
D.$x \in [6,8) \cup [17,\infty)$\\
E.$x \in [6,8] \cup (17,\infty)$\\
F.$x \in (6,8) \cup (17,\infty)$\\
G.$x \in [6,8) \cup (17,\infty)$\\
H.$x \in (6,8] \cup (17,\infty)$
\testStop
\kluczStart
A
\kluczStop



\zadStart{Zadanie z Wikieł Z 1.62 a) moja wersja nr 708}

Rozwiązać nierówności $(x-6)(x-8)(x-18)\ge0$.
\zadStop
\rozwStart{Patryk Wirkus}{}
Miejsca zerowe naszego wielomianu to: $6, 8, 18$.\\
Wielomian jest stopnia nieparzystego, ponadto znak współczynnika przy\linebreak najwyższej potędze x jest dodatni.\\ W związku z tym wykres wielomianu zaczyna się od lewej strony poniżej osi OX. A więc $$x \in [6,8] \cup [18,\infty).$$
\rozwStop
\odpStart
$x \in [6,8] \cup [18,\infty)$
\odpStop
\testStart
A.$x \in [6,8] \cup [18,\infty)$\\
B.$x \in (6,8) \cup [18,\infty)$\\
C.$x \in (6,8] \cup [18,\infty)$\\
D.$x \in [6,8) \cup [18,\infty)$\\
E.$x \in [6,8] \cup (18,\infty)$\\
F.$x \in (6,8) \cup (18,\infty)$\\
G.$x \in [6,8) \cup (18,\infty)$\\
H.$x \in (6,8] \cup (18,\infty)$
\testStop
\kluczStart
A
\kluczStop



\zadStart{Zadanie z Wikieł Z 1.62 a) moja wersja nr 709}

Rozwiązać nierówności $(x-6)(x-8)(x-19)\ge0$.
\zadStop
\rozwStart{Patryk Wirkus}{}
Miejsca zerowe naszego wielomianu to: $6, 8, 19$.\\
Wielomian jest stopnia nieparzystego, ponadto znak współczynnika przy\linebreak najwyższej potędze x jest dodatni.\\ W związku z tym wykres wielomianu zaczyna się od lewej strony poniżej osi OX. A więc $$x \in [6,8] \cup [19,\infty).$$
\rozwStop
\odpStart
$x \in [6,8] \cup [19,\infty)$
\odpStop
\testStart
A.$x \in [6,8] \cup [19,\infty)$\\
B.$x \in (6,8) \cup [19,\infty)$\\
C.$x \in (6,8] \cup [19,\infty)$\\
D.$x \in [6,8) \cup [19,\infty)$\\
E.$x \in [6,8] \cup (19,\infty)$\\
F.$x \in (6,8) \cup (19,\infty)$\\
G.$x \in [6,8) \cup (19,\infty)$\\
H.$x \in (6,8] \cup (19,\infty)$
\testStop
\kluczStart
A
\kluczStop



\zadStart{Zadanie z Wikieł Z 1.62 a) moja wersja nr 710}

Rozwiązać nierówności $(x-6)(x-8)(x-20)\ge0$.
\zadStop
\rozwStart{Patryk Wirkus}{}
Miejsca zerowe naszego wielomianu to: $6, 8, 20$.\\
Wielomian jest stopnia nieparzystego, ponadto znak współczynnika przy\linebreak najwyższej potędze x jest dodatni.\\ W związku z tym wykres wielomianu zaczyna się od lewej strony poniżej osi OX. A więc $$x \in [6,8] \cup [20,\infty).$$
\rozwStop
\odpStart
$x \in [6,8] \cup [20,\infty)$
\odpStop
\testStart
A.$x \in [6,8] \cup [20,\infty)$\\
B.$x \in (6,8) \cup [20,\infty)$\\
C.$x \in (6,8] \cup [20,\infty)$\\
D.$x \in [6,8) \cup [20,\infty)$\\
E.$x \in [6,8] \cup (20,\infty)$\\
F.$x \in (6,8) \cup (20,\infty)$\\
G.$x \in [6,8) \cup (20,\infty)$\\
H.$x \in (6,8] \cup (20,\infty)$
\testStop
\kluczStart
A
\kluczStop



\zadStart{Zadanie z Wikieł Z 1.62 a) moja wersja nr 711}

Rozwiązać nierówności $(x-6)(x-9)(x-10)\ge0$.
\zadStop
\rozwStart{Patryk Wirkus}{}
Miejsca zerowe naszego wielomianu to: $6, 9, 10$.\\
Wielomian jest stopnia nieparzystego, ponadto znak współczynnika przy\linebreak najwyższej potędze x jest dodatni.\\ W związku z tym wykres wielomianu zaczyna się od lewej strony poniżej osi OX. A więc $$x \in [6,9] \cup [10,\infty).$$
\rozwStop
\odpStart
$x \in [6,9] \cup [10,\infty)$
\odpStop
\testStart
A.$x \in [6,9] \cup [10,\infty)$\\
B.$x \in (6,9) \cup [10,\infty)$\\
C.$x \in (6,9] \cup [10,\infty)$\\
D.$x \in [6,9) \cup [10,\infty)$\\
E.$x \in [6,9] \cup (10,\infty)$\\
F.$x \in (6,9) \cup (10,\infty)$\\
G.$x \in [6,9) \cup (10,\infty)$\\
H.$x \in (6,9] \cup (10,\infty)$
\testStop
\kluczStart
A
\kluczStop



\zadStart{Zadanie z Wikieł Z 1.62 a) moja wersja nr 712}

Rozwiązać nierówności $(x-6)(x-9)(x-11)\ge0$.
\zadStop
\rozwStart{Patryk Wirkus}{}
Miejsca zerowe naszego wielomianu to: $6, 9, 11$.\\
Wielomian jest stopnia nieparzystego, ponadto znak współczynnika przy\linebreak najwyższej potędze x jest dodatni.\\ W związku z tym wykres wielomianu zaczyna się od lewej strony poniżej osi OX. A więc $$x \in [6,9] \cup [11,\infty).$$
\rozwStop
\odpStart
$x \in [6,9] \cup [11,\infty)$
\odpStop
\testStart
A.$x \in [6,9] \cup [11,\infty)$\\
B.$x \in (6,9) \cup [11,\infty)$\\
C.$x \in (6,9] \cup [11,\infty)$\\
D.$x \in [6,9) \cup [11,\infty)$\\
E.$x \in [6,9] \cup (11,\infty)$\\
F.$x \in (6,9) \cup (11,\infty)$\\
G.$x \in [6,9) \cup (11,\infty)$\\
H.$x \in (6,9] \cup (11,\infty)$
\testStop
\kluczStart
A
\kluczStop



\zadStart{Zadanie z Wikieł Z 1.62 a) moja wersja nr 713}

Rozwiązać nierówności $(x-6)(x-9)(x-12)\ge0$.
\zadStop
\rozwStart{Patryk Wirkus}{}
Miejsca zerowe naszego wielomianu to: $6, 9, 12$.\\
Wielomian jest stopnia nieparzystego, ponadto znak współczynnika przy\linebreak najwyższej potędze x jest dodatni.\\ W związku z tym wykres wielomianu zaczyna się od lewej strony poniżej osi OX. A więc $$x \in [6,9] \cup [12,\infty).$$
\rozwStop
\odpStart
$x \in [6,9] \cup [12,\infty)$
\odpStop
\testStart
A.$x \in [6,9] \cup [12,\infty)$\\
B.$x \in (6,9) \cup [12,\infty)$\\
C.$x \in (6,9] \cup [12,\infty)$\\
D.$x \in [6,9) \cup [12,\infty)$\\
E.$x \in [6,9] \cup (12,\infty)$\\
F.$x \in (6,9) \cup (12,\infty)$\\
G.$x \in [6,9) \cup (12,\infty)$\\
H.$x \in (6,9] \cup (12,\infty)$
\testStop
\kluczStart
A
\kluczStop



\zadStart{Zadanie z Wikieł Z 1.62 a) moja wersja nr 714}

Rozwiązać nierówności $(x-6)(x-9)(x-13)\ge0$.
\zadStop
\rozwStart{Patryk Wirkus}{}
Miejsca zerowe naszego wielomianu to: $6, 9, 13$.\\
Wielomian jest stopnia nieparzystego, ponadto znak współczynnika przy\linebreak najwyższej potędze x jest dodatni.\\ W związku z tym wykres wielomianu zaczyna się od lewej strony poniżej osi OX. A więc $$x \in [6,9] \cup [13,\infty).$$
\rozwStop
\odpStart
$x \in [6,9] \cup [13,\infty)$
\odpStop
\testStart
A.$x \in [6,9] \cup [13,\infty)$\\
B.$x \in (6,9) \cup [13,\infty)$\\
C.$x \in (6,9] \cup [13,\infty)$\\
D.$x \in [6,9) \cup [13,\infty)$\\
E.$x \in [6,9] \cup (13,\infty)$\\
F.$x \in (6,9) \cup (13,\infty)$\\
G.$x \in [6,9) \cup (13,\infty)$\\
H.$x \in (6,9] \cup (13,\infty)$
\testStop
\kluczStart
A
\kluczStop



\zadStart{Zadanie z Wikieł Z 1.62 a) moja wersja nr 715}

Rozwiązać nierówności $(x-6)(x-9)(x-14)\ge0$.
\zadStop
\rozwStart{Patryk Wirkus}{}
Miejsca zerowe naszego wielomianu to: $6, 9, 14$.\\
Wielomian jest stopnia nieparzystego, ponadto znak współczynnika przy\linebreak najwyższej potędze x jest dodatni.\\ W związku z tym wykres wielomianu zaczyna się od lewej strony poniżej osi OX. A więc $$x \in [6,9] \cup [14,\infty).$$
\rozwStop
\odpStart
$x \in [6,9] \cup [14,\infty)$
\odpStop
\testStart
A.$x \in [6,9] \cup [14,\infty)$\\
B.$x \in (6,9) \cup [14,\infty)$\\
C.$x \in (6,9] \cup [14,\infty)$\\
D.$x \in [6,9) \cup [14,\infty)$\\
E.$x \in [6,9] \cup (14,\infty)$\\
F.$x \in (6,9) \cup (14,\infty)$\\
G.$x \in [6,9) \cup (14,\infty)$\\
H.$x \in (6,9] \cup (14,\infty)$
\testStop
\kluczStart
A
\kluczStop



\zadStart{Zadanie z Wikieł Z 1.62 a) moja wersja nr 716}

Rozwiązać nierówności $(x-6)(x-9)(x-15)\ge0$.
\zadStop
\rozwStart{Patryk Wirkus}{}
Miejsca zerowe naszego wielomianu to: $6, 9, 15$.\\
Wielomian jest stopnia nieparzystego, ponadto znak współczynnika przy\linebreak najwyższej potędze x jest dodatni.\\ W związku z tym wykres wielomianu zaczyna się od lewej strony poniżej osi OX. A więc $$x \in [6,9] \cup [15,\infty).$$
\rozwStop
\odpStart
$x \in [6,9] \cup [15,\infty)$
\odpStop
\testStart
A.$x \in [6,9] \cup [15,\infty)$\\
B.$x \in (6,9) \cup [15,\infty)$\\
C.$x \in (6,9] \cup [15,\infty)$\\
D.$x \in [6,9) \cup [15,\infty)$\\
E.$x \in [6,9] \cup (15,\infty)$\\
F.$x \in (6,9) \cup (15,\infty)$\\
G.$x \in [6,9) \cup (15,\infty)$\\
H.$x \in (6,9] \cup (15,\infty)$
\testStop
\kluczStart
A
\kluczStop



\zadStart{Zadanie z Wikieł Z 1.62 a) moja wersja nr 717}

Rozwiązać nierówności $(x-6)(x-9)(x-16)\ge0$.
\zadStop
\rozwStart{Patryk Wirkus}{}
Miejsca zerowe naszego wielomianu to: $6, 9, 16$.\\
Wielomian jest stopnia nieparzystego, ponadto znak współczynnika przy\linebreak najwyższej potędze x jest dodatni.\\ W związku z tym wykres wielomianu zaczyna się od lewej strony poniżej osi OX. A więc $$x \in [6,9] \cup [16,\infty).$$
\rozwStop
\odpStart
$x \in [6,9] \cup [16,\infty)$
\odpStop
\testStart
A.$x \in [6,9] \cup [16,\infty)$\\
B.$x \in (6,9) \cup [16,\infty)$\\
C.$x \in (6,9] \cup [16,\infty)$\\
D.$x \in [6,9) \cup [16,\infty)$\\
E.$x \in [6,9] \cup (16,\infty)$\\
F.$x \in (6,9) \cup (16,\infty)$\\
G.$x \in [6,9) \cup (16,\infty)$\\
H.$x \in (6,9] \cup (16,\infty)$
\testStop
\kluczStart
A
\kluczStop



\zadStart{Zadanie z Wikieł Z 1.62 a) moja wersja nr 718}

Rozwiązać nierówności $(x-6)(x-9)(x-17)\ge0$.
\zadStop
\rozwStart{Patryk Wirkus}{}
Miejsca zerowe naszego wielomianu to: $6, 9, 17$.\\
Wielomian jest stopnia nieparzystego, ponadto znak współczynnika przy\linebreak najwyższej potędze x jest dodatni.\\ W związku z tym wykres wielomianu zaczyna się od lewej strony poniżej osi OX. A więc $$x \in [6,9] \cup [17,\infty).$$
\rozwStop
\odpStart
$x \in [6,9] \cup [17,\infty)$
\odpStop
\testStart
A.$x \in [6,9] \cup [17,\infty)$\\
B.$x \in (6,9) \cup [17,\infty)$\\
C.$x \in (6,9] \cup [17,\infty)$\\
D.$x \in [6,9) \cup [17,\infty)$\\
E.$x \in [6,9] \cup (17,\infty)$\\
F.$x \in (6,9) \cup (17,\infty)$\\
G.$x \in [6,9) \cup (17,\infty)$\\
H.$x \in (6,9] \cup (17,\infty)$
\testStop
\kluczStart
A
\kluczStop



\zadStart{Zadanie z Wikieł Z 1.62 a) moja wersja nr 719}

Rozwiązać nierówności $(x-6)(x-9)(x-18)\ge0$.
\zadStop
\rozwStart{Patryk Wirkus}{}
Miejsca zerowe naszego wielomianu to: $6, 9, 18$.\\
Wielomian jest stopnia nieparzystego, ponadto znak współczynnika przy\linebreak najwyższej potędze x jest dodatni.\\ W związku z tym wykres wielomianu zaczyna się od lewej strony poniżej osi OX. A więc $$x \in [6,9] \cup [18,\infty).$$
\rozwStop
\odpStart
$x \in [6,9] \cup [18,\infty)$
\odpStop
\testStart
A.$x \in [6,9] \cup [18,\infty)$\\
B.$x \in (6,9) \cup [18,\infty)$\\
C.$x \in (6,9] \cup [18,\infty)$\\
D.$x \in [6,9) \cup [18,\infty)$\\
E.$x \in [6,9] \cup (18,\infty)$\\
F.$x \in (6,9) \cup (18,\infty)$\\
G.$x \in [6,9) \cup (18,\infty)$\\
H.$x \in (6,9] \cup (18,\infty)$
\testStop
\kluczStart
A
\kluczStop



\zadStart{Zadanie z Wikieł Z 1.62 a) moja wersja nr 720}

Rozwiązać nierówności $(x-6)(x-9)(x-19)\ge0$.
\zadStop
\rozwStart{Patryk Wirkus}{}
Miejsca zerowe naszego wielomianu to: $6, 9, 19$.\\
Wielomian jest stopnia nieparzystego, ponadto znak współczynnika przy\linebreak najwyższej potędze x jest dodatni.\\ W związku z tym wykres wielomianu zaczyna się od lewej strony poniżej osi OX. A więc $$x \in [6,9] \cup [19,\infty).$$
\rozwStop
\odpStart
$x \in [6,9] \cup [19,\infty)$
\odpStop
\testStart
A.$x \in [6,9] \cup [19,\infty)$\\
B.$x \in (6,9) \cup [19,\infty)$\\
C.$x \in (6,9] \cup [19,\infty)$\\
D.$x \in [6,9) \cup [19,\infty)$\\
E.$x \in [6,9] \cup (19,\infty)$\\
F.$x \in (6,9) \cup (19,\infty)$\\
G.$x \in [6,9) \cup (19,\infty)$\\
H.$x \in (6,9] \cup (19,\infty)$
\testStop
\kluczStart
A
\kluczStop



\zadStart{Zadanie z Wikieł Z 1.62 a) moja wersja nr 721}

Rozwiązać nierówności $(x-6)(x-9)(x-20)\ge0$.
\zadStop
\rozwStart{Patryk Wirkus}{}
Miejsca zerowe naszego wielomianu to: $6, 9, 20$.\\
Wielomian jest stopnia nieparzystego, ponadto znak współczynnika przy\linebreak najwyższej potędze x jest dodatni.\\ W związku z tym wykres wielomianu zaczyna się od lewej strony poniżej osi OX. A więc $$x \in [6,9] \cup [20,\infty).$$
\rozwStop
\odpStart
$x \in [6,9] \cup [20,\infty)$
\odpStop
\testStart
A.$x \in [6,9] \cup [20,\infty)$\\
B.$x \in (6,9) \cup [20,\infty)$\\
C.$x \in (6,9] \cup [20,\infty)$\\
D.$x \in [6,9) \cup [20,\infty)$\\
E.$x \in [6,9] \cup (20,\infty)$\\
F.$x \in (6,9) \cup (20,\infty)$\\
G.$x \in [6,9) \cup (20,\infty)$\\
H.$x \in (6,9] \cup (20,\infty)$
\testStop
\kluczStart
A
\kluczStop



\zadStart{Zadanie z Wikieł Z 1.62 a) moja wersja nr 722}

Rozwiązać nierówności $(x-6)(x-10)(x-11)\ge0$.
\zadStop
\rozwStart{Patryk Wirkus}{}
Miejsca zerowe naszego wielomianu to: $6, 10, 11$.\\
Wielomian jest stopnia nieparzystego, ponadto znak współczynnika przy\linebreak najwyższej potędze x jest dodatni.\\ W związku z tym wykres wielomianu zaczyna się od lewej strony poniżej osi OX. A więc $$x \in [6,10] \cup [11,\infty).$$
\rozwStop
\odpStart
$x \in [6,10] \cup [11,\infty)$
\odpStop
\testStart
A.$x \in [6,10] \cup [11,\infty)$\\
B.$x \in (6,10) \cup [11,\infty)$\\
C.$x \in (6,10] \cup [11,\infty)$\\
D.$x \in [6,10) \cup [11,\infty)$\\
E.$x \in [6,10] \cup (11,\infty)$\\
F.$x \in (6,10) \cup (11,\infty)$\\
G.$x \in [6,10) \cup (11,\infty)$\\
H.$x \in (6,10] \cup (11,\infty)$
\testStop
\kluczStart
A
\kluczStop



\zadStart{Zadanie z Wikieł Z 1.62 a) moja wersja nr 723}

Rozwiązać nierówności $(x-6)(x-10)(x-12)\ge0$.
\zadStop
\rozwStart{Patryk Wirkus}{}
Miejsca zerowe naszego wielomianu to: $6, 10, 12$.\\
Wielomian jest stopnia nieparzystego, ponadto znak współczynnika przy\linebreak najwyższej potędze x jest dodatni.\\ W związku z tym wykres wielomianu zaczyna się od lewej strony poniżej osi OX. A więc $$x \in [6,10] \cup [12,\infty).$$
\rozwStop
\odpStart
$x \in [6,10] \cup [12,\infty)$
\odpStop
\testStart
A.$x \in [6,10] \cup [12,\infty)$\\
B.$x \in (6,10) \cup [12,\infty)$\\
C.$x \in (6,10] \cup [12,\infty)$\\
D.$x \in [6,10) \cup [12,\infty)$\\
E.$x \in [6,10] \cup (12,\infty)$\\
F.$x \in (6,10) \cup (12,\infty)$\\
G.$x \in [6,10) \cup (12,\infty)$\\
H.$x \in (6,10] \cup (12,\infty)$
\testStop
\kluczStart
A
\kluczStop



\zadStart{Zadanie z Wikieł Z 1.62 a) moja wersja nr 724}

Rozwiązać nierówności $(x-6)(x-10)(x-13)\ge0$.
\zadStop
\rozwStart{Patryk Wirkus}{}
Miejsca zerowe naszego wielomianu to: $6, 10, 13$.\\
Wielomian jest stopnia nieparzystego, ponadto znak współczynnika przy\linebreak najwyższej potędze x jest dodatni.\\ W związku z tym wykres wielomianu zaczyna się od lewej strony poniżej osi OX. A więc $$x \in [6,10] \cup [13,\infty).$$
\rozwStop
\odpStart
$x \in [6,10] \cup [13,\infty)$
\odpStop
\testStart
A.$x \in [6,10] \cup [13,\infty)$\\
B.$x \in (6,10) \cup [13,\infty)$\\
C.$x \in (6,10] \cup [13,\infty)$\\
D.$x \in [6,10) \cup [13,\infty)$\\
E.$x \in [6,10] \cup (13,\infty)$\\
F.$x \in (6,10) \cup (13,\infty)$\\
G.$x \in [6,10) \cup (13,\infty)$\\
H.$x \in (6,10] \cup (13,\infty)$
\testStop
\kluczStart
A
\kluczStop



\zadStart{Zadanie z Wikieł Z 1.62 a) moja wersja nr 725}

Rozwiązać nierówności $(x-6)(x-10)(x-14)\ge0$.
\zadStop
\rozwStart{Patryk Wirkus}{}
Miejsca zerowe naszego wielomianu to: $6, 10, 14$.\\
Wielomian jest stopnia nieparzystego, ponadto znak współczynnika przy\linebreak najwyższej potędze x jest dodatni.\\ W związku z tym wykres wielomianu zaczyna się od lewej strony poniżej osi OX. A więc $$x \in [6,10] \cup [14,\infty).$$
\rozwStop
\odpStart
$x \in [6,10] \cup [14,\infty)$
\odpStop
\testStart
A.$x \in [6,10] \cup [14,\infty)$\\
B.$x \in (6,10) \cup [14,\infty)$\\
C.$x \in (6,10] \cup [14,\infty)$\\
D.$x \in [6,10) \cup [14,\infty)$\\
E.$x \in [6,10] \cup (14,\infty)$\\
F.$x \in (6,10) \cup (14,\infty)$\\
G.$x \in [6,10) \cup (14,\infty)$\\
H.$x \in (6,10] \cup (14,\infty)$
\testStop
\kluczStart
A
\kluczStop



\zadStart{Zadanie z Wikieł Z 1.62 a) moja wersja nr 726}

Rozwiązać nierówności $(x-6)(x-10)(x-15)\ge0$.
\zadStop
\rozwStart{Patryk Wirkus}{}
Miejsca zerowe naszego wielomianu to: $6, 10, 15$.\\
Wielomian jest stopnia nieparzystego, ponadto znak współczynnika przy\linebreak najwyższej potędze x jest dodatni.\\ W związku z tym wykres wielomianu zaczyna się od lewej strony poniżej osi OX. A więc $$x \in [6,10] \cup [15,\infty).$$
\rozwStop
\odpStart
$x \in [6,10] \cup [15,\infty)$
\odpStop
\testStart
A.$x \in [6,10] \cup [15,\infty)$\\
B.$x \in (6,10) \cup [15,\infty)$\\
C.$x \in (6,10] \cup [15,\infty)$\\
D.$x \in [6,10) \cup [15,\infty)$\\
E.$x \in [6,10] \cup (15,\infty)$\\
F.$x \in (6,10) \cup (15,\infty)$\\
G.$x \in [6,10) \cup (15,\infty)$\\
H.$x \in (6,10] \cup (15,\infty)$
\testStop
\kluczStart
A
\kluczStop



\zadStart{Zadanie z Wikieł Z 1.62 a) moja wersja nr 727}

Rozwiązać nierówności $(x-6)(x-10)(x-16)\ge0$.
\zadStop
\rozwStart{Patryk Wirkus}{}
Miejsca zerowe naszego wielomianu to: $6, 10, 16$.\\
Wielomian jest stopnia nieparzystego, ponadto znak współczynnika przy\linebreak najwyższej potędze x jest dodatni.\\ W związku z tym wykres wielomianu zaczyna się od lewej strony poniżej osi OX. A więc $$x \in [6,10] \cup [16,\infty).$$
\rozwStop
\odpStart
$x \in [6,10] \cup [16,\infty)$
\odpStop
\testStart
A.$x \in [6,10] \cup [16,\infty)$\\
B.$x \in (6,10) \cup [16,\infty)$\\
C.$x \in (6,10] \cup [16,\infty)$\\
D.$x \in [6,10) \cup [16,\infty)$\\
E.$x \in [6,10] \cup (16,\infty)$\\
F.$x \in (6,10) \cup (16,\infty)$\\
G.$x \in [6,10) \cup (16,\infty)$\\
H.$x \in (6,10] \cup (16,\infty)$
\testStop
\kluczStart
A
\kluczStop



\zadStart{Zadanie z Wikieł Z 1.62 a) moja wersja nr 728}

Rozwiązać nierówności $(x-6)(x-10)(x-17)\ge0$.
\zadStop
\rozwStart{Patryk Wirkus}{}
Miejsca zerowe naszego wielomianu to: $6, 10, 17$.\\
Wielomian jest stopnia nieparzystego, ponadto znak współczynnika przy\linebreak najwyższej potędze x jest dodatni.\\ W związku z tym wykres wielomianu zaczyna się od lewej strony poniżej osi OX. A więc $$x \in [6,10] \cup [17,\infty).$$
\rozwStop
\odpStart
$x \in [6,10] \cup [17,\infty)$
\odpStop
\testStart
A.$x \in [6,10] \cup [17,\infty)$\\
B.$x \in (6,10) \cup [17,\infty)$\\
C.$x \in (6,10] \cup [17,\infty)$\\
D.$x \in [6,10) \cup [17,\infty)$\\
E.$x \in [6,10] \cup (17,\infty)$\\
F.$x \in (6,10) \cup (17,\infty)$\\
G.$x \in [6,10) \cup (17,\infty)$\\
H.$x \in (6,10] \cup (17,\infty)$
\testStop
\kluczStart
A
\kluczStop



\zadStart{Zadanie z Wikieł Z 1.62 a) moja wersja nr 729}

Rozwiązać nierówności $(x-6)(x-10)(x-18)\ge0$.
\zadStop
\rozwStart{Patryk Wirkus}{}
Miejsca zerowe naszego wielomianu to: $6, 10, 18$.\\
Wielomian jest stopnia nieparzystego, ponadto znak współczynnika przy\linebreak najwyższej potędze x jest dodatni.\\ W związku z tym wykres wielomianu zaczyna się od lewej strony poniżej osi OX. A więc $$x \in [6,10] \cup [18,\infty).$$
\rozwStop
\odpStart
$x \in [6,10] \cup [18,\infty)$
\odpStop
\testStart
A.$x \in [6,10] \cup [18,\infty)$\\
B.$x \in (6,10) \cup [18,\infty)$\\
C.$x \in (6,10] \cup [18,\infty)$\\
D.$x \in [6,10) \cup [18,\infty)$\\
E.$x \in [6,10] \cup (18,\infty)$\\
F.$x \in (6,10) \cup (18,\infty)$\\
G.$x \in [6,10) \cup (18,\infty)$\\
H.$x \in (6,10] \cup (18,\infty)$
\testStop
\kluczStart
A
\kluczStop



\zadStart{Zadanie z Wikieł Z 1.62 a) moja wersja nr 730}

Rozwiązać nierówności $(x-6)(x-10)(x-19)\ge0$.
\zadStop
\rozwStart{Patryk Wirkus}{}
Miejsca zerowe naszego wielomianu to: $6, 10, 19$.\\
Wielomian jest stopnia nieparzystego, ponadto znak współczynnika przy\linebreak najwyższej potędze x jest dodatni.\\ W związku z tym wykres wielomianu zaczyna się od lewej strony poniżej osi OX. A więc $$x \in [6,10] \cup [19,\infty).$$
\rozwStop
\odpStart
$x \in [6,10] \cup [19,\infty)$
\odpStop
\testStart
A.$x \in [6,10] \cup [19,\infty)$\\
B.$x \in (6,10) \cup [19,\infty)$\\
C.$x \in (6,10] \cup [19,\infty)$\\
D.$x \in [6,10) \cup [19,\infty)$\\
E.$x \in [6,10] \cup (19,\infty)$\\
F.$x \in (6,10) \cup (19,\infty)$\\
G.$x \in [6,10) \cup (19,\infty)$\\
H.$x \in (6,10] \cup (19,\infty)$
\testStop
\kluczStart
A
\kluczStop



\zadStart{Zadanie z Wikieł Z 1.62 a) moja wersja nr 731}

Rozwiązać nierówności $(x-6)(x-10)(x-20)\ge0$.
\zadStop
\rozwStart{Patryk Wirkus}{}
Miejsca zerowe naszego wielomianu to: $6, 10, 20$.\\
Wielomian jest stopnia nieparzystego, ponadto znak współczynnika przy\linebreak najwyższej potędze x jest dodatni.\\ W związku z tym wykres wielomianu zaczyna się od lewej strony poniżej osi OX. A więc $$x \in [6,10] \cup [20,\infty).$$
\rozwStop
\odpStart
$x \in [6,10] \cup [20,\infty)$
\odpStop
\testStart
A.$x \in [6,10] \cup [20,\infty)$\\
B.$x \in (6,10) \cup [20,\infty)$\\
C.$x \in (6,10] \cup [20,\infty)$\\
D.$x \in [6,10) \cup [20,\infty)$\\
E.$x \in [6,10] \cup (20,\infty)$\\
F.$x \in (6,10) \cup (20,\infty)$\\
G.$x \in [6,10) \cup (20,\infty)$\\
H.$x \in (6,10] \cup (20,\infty)$
\testStop
\kluczStart
A
\kluczStop



\zadStart{Zadanie z Wikieł Z 1.62 a) moja wersja nr 732}

Rozwiązać nierówności $(x-6)(x-11)(x-12)\ge0$.
\zadStop
\rozwStart{Patryk Wirkus}{}
Miejsca zerowe naszego wielomianu to: $6, 11, 12$.\\
Wielomian jest stopnia nieparzystego, ponadto znak współczynnika przy\linebreak najwyższej potędze x jest dodatni.\\ W związku z tym wykres wielomianu zaczyna się od lewej strony poniżej osi OX. A więc $$x \in [6,11] \cup [12,\infty).$$
\rozwStop
\odpStart
$x \in [6,11] \cup [12,\infty)$
\odpStop
\testStart
A.$x \in [6,11] \cup [12,\infty)$\\
B.$x \in (6,11) \cup [12,\infty)$\\
C.$x \in (6,11] \cup [12,\infty)$\\
D.$x \in [6,11) \cup [12,\infty)$\\
E.$x \in [6,11] \cup (12,\infty)$\\
F.$x \in (6,11) \cup (12,\infty)$\\
G.$x \in [6,11) \cup (12,\infty)$\\
H.$x \in (6,11] \cup (12,\infty)$
\testStop
\kluczStart
A
\kluczStop



\zadStart{Zadanie z Wikieł Z 1.62 a) moja wersja nr 733}

Rozwiązać nierówności $(x-6)(x-11)(x-13)\ge0$.
\zadStop
\rozwStart{Patryk Wirkus}{}
Miejsca zerowe naszego wielomianu to: $6, 11, 13$.\\
Wielomian jest stopnia nieparzystego, ponadto znak współczynnika przy\linebreak najwyższej potędze x jest dodatni.\\ W związku z tym wykres wielomianu zaczyna się od lewej strony poniżej osi OX. A więc $$x \in [6,11] \cup [13,\infty).$$
\rozwStop
\odpStart
$x \in [6,11] \cup [13,\infty)$
\odpStop
\testStart
A.$x \in [6,11] \cup [13,\infty)$\\
B.$x \in (6,11) \cup [13,\infty)$\\
C.$x \in (6,11] \cup [13,\infty)$\\
D.$x \in [6,11) \cup [13,\infty)$\\
E.$x \in [6,11] \cup (13,\infty)$\\
F.$x \in (6,11) \cup (13,\infty)$\\
G.$x \in [6,11) \cup (13,\infty)$\\
H.$x \in (6,11] \cup (13,\infty)$
\testStop
\kluczStart
A
\kluczStop



\zadStart{Zadanie z Wikieł Z 1.62 a) moja wersja nr 734}

Rozwiązać nierówności $(x-6)(x-11)(x-14)\ge0$.
\zadStop
\rozwStart{Patryk Wirkus}{}
Miejsca zerowe naszego wielomianu to: $6, 11, 14$.\\
Wielomian jest stopnia nieparzystego, ponadto znak współczynnika przy\linebreak najwyższej potędze x jest dodatni.\\ W związku z tym wykres wielomianu zaczyna się od lewej strony poniżej osi OX. A więc $$x \in [6,11] \cup [14,\infty).$$
\rozwStop
\odpStart
$x \in [6,11] \cup [14,\infty)$
\odpStop
\testStart
A.$x \in [6,11] \cup [14,\infty)$\\
B.$x \in (6,11) \cup [14,\infty)$\\
C.$x \in (6,11] \cup [14,\infty)$\\
D.$x \in [6,11) \cup [14,\infty)$\\
E.$x \in [6,11] \cup (14,\infty)$\\
F.$x \in (6,11) \cup (14,\infty)$\\
G.$x \in [6,11) \cup (14,\infty)$\\
H.$x \in (6,11] \cup (14,\infty)$
\testStop
\kluczStart
A
\kluczStop



\zadStart{Zadanie z Wikieł Z 1.62 a) moja wersja nr 735}

Rozwiązać nierówności $(x-6)(x-11)(x-15)\ge0$.
\zadStop
\rozwStart{Patryk Wirkus}{}
Miejsca zerowe naszego wielomianu to: $6, 11, 15$.\\
Wielomian jest stopnia nieparzystego, ponadto znak współczynnika przy\linebreak najwyższej potędze x jest dodatni.\\ W związku z tym wykres wielomianu zaczyna się od lewej strony poniżej osi OX. A więc $$x \in [6,11] \cup [15,\infty).$$
\rozwStop
\odpStart
$x \in [6,11] \cup [15,\infty)$
\odpStop
\testStart
A.$x \in [6,11] \cup [15,\infty)$\\
B.$x \in (6,11) \cup [15,\infty)$\\
C.$x \in (6,11] \cup [15,\infty)$\\
D.$x \in [6,11) \cup [15,\infty)$\\
E.$x \in [6,11] \cup (15,\infty)$\\
F.$x \in (6,11) \cup (15,\infty)$\\
G.$x \in [6,11) \cup (15,\infty)$\\
H.$x \in (6,11] \cup (15,\infty)$
\testStop
\kluczStart
A
\kluczStop



\zadStart{Zadanie z Wikieł Z 1.62 a) moja wersja nr 736}

Rozwiązać nierówności $(x-6)(x-11)(x-16)\ge0$.
\zadStop
\rozwStart{Patryk Wirkus}{}
Miejsca zerowe naszego wielomianu to: $6, 11, 16$.\\
Wielomian jest stopnia nieparzystego, ponadto znak współczynnika przy\linebreak najwyższej potędze x jest dodatni.\\ W związku z tym wykres wielomianu zaczyna się od lewej strony poniżej osi OX. A więc $$x \in [6,11] \cup [16,\infty).$$
\rozwStop
\odpStart
$x \in [6,11] \cup [16,\infty)$
\odpStop
\testStart
A.$x \in [6,11] \cup [16,\infty)$\\
B.$x \in (6,11) \cup [16,\infty)$\\
C.$x \in (6,11] \cup [16,\infty)$\\
D.$x \in [6,11) \cup [16,\infty)$\\
E.$x \in [6,11] \cup (16,\infty)$\\
F.$x \in (6,11) \cup (16,\infty)$\\
G.$x \in [6,11) \cup (16,\infty)$\\
H.$x \in (6,11] \cup (16,\infty)$
\testStop
\kluczStart
A
\kluczStop



\zadStart{Zadanie z Wikieł Z 1.62 a) moja wersja nr 737}

Rozwiązać nierówności $(x-6)(x-11)(x-17)\ge0$.
\zadStop
\rozwStart{Patryk Wirkus}{}
Miejsca zerowe naszego wielomianu to: $6, 11, 17$.\\
Wielomian jest stopnia nieparzystego, ponadto znak współczynnika przy\linebreak najwyższej potędze x jest dodatni.\\ W związku z tym wykres wielomianu zaczyna się od lewej strony poniżej osi OX. A więc $$x \in [6,11] \cup [17,\infty).$$
\rozwStop
\odpStart
$x \in [6,11] \cup [17,\infty)$
\odpStop
\testStart
A.$x \in [6,11] \cup [17,\infty)$\\
B.$x \in (6,11) \cup [17,\infty)$\\
C.$x \in (6,11] \cup [17,\infty)$\\
D.$x \in [6,11) \cup [17,\infty)$\\
E.$x \in [6,11] \cup (17,\infty)$\\
F.$x \in (6,11) \cup (17,\infty)$\\
G.$x \in [6,11) \cup (17,\infty)$\\
H.$x \in (6,11] \cup (17,\infty)$
\testStop
\kluczStart
A
\kluczStop



\zadStart{Zadanie z Wikieł Z 1.62 a) moja wersja nr 738}

Rozwiązać nierówności $(x-6)(x-11)(x-18)\ge0$.
\zadStop
\rozwStart{Patryk Wirkus}{}
Miejsca zerowe naszego wielomianu to: $6, 11, 18$.\\
Wielomian jest stopnia nieparzystego, ponadto znak współczynnika przy\linebreak najwyższej potędze x jest dodatni.\\ W związku z tym wykres wielomianu zaczyna się od lewej strony poniżej osi OX. A więc $$x \in [6,11] \cup [18,\infty).$$
\rozwStop
\odpStart
$x \in [6,11] \cup [18,\infty)$
\odpStop
\testStart
A.$x \in [6,11] \cup [18,\infty)$\\
B.$x \in (6,11) \cup [18,\infty)$\\
C.$x \in (6,11] \cup [18,\infty)$\\
D.$x \in [6,11) \cup [18,\infty)$\\
E.$x \in [6,11] \cup (18,\infty)$\\
F.$x \in (6,11) \cup (18,\infty)$\\
G.$x \in [6,11) \cup (18,\infty)$\\
H.$x \in (6,11] \cup (18,\infty)$
\testStop
\kluczStart
A
\kluczStop



\zadStart{Zadanie z Wikieł Z 1.62 a) moja wersja nr 739}

Rozwiązać nierówności $(x-6)(x-11)(x-19)\ge0$.
\zadStop
\rozwStart{Patryk Wirkus}{}
Miejsca zerowe naszego wielomianu to: $6, 11, 19$.\\
Wielomian jest stopnia nieparzystego, ponadto znak współczynnika przy\linebreak najwyższej potędze x jest dodatni.\\ W związku z tym wykres wielomianu zaczyna się od lewej strony poniżej osi OX. A więc $$x \in [6,11] \cup [19,\infty).$$
\rozwStop
\odpStart
$x \in [6,11] \cup [19,\infty)$
\odpStop
\testStart
A.$x \in [6,11] \cup [19,\infty)$\\
B.$x \in (6,11) \cup [19,\infty)$\\
C.$x \in (6,11] \cup [19,\infty)$\\
D.$x \in [6,11) \cup [19,\infty)$\\
E.$x \in [6,11] \cup (19,\infty)$\\
F.$x \in (6,11) \cup (19,\infty)$\\
G.$x \in [6,11) \cup (19,\infty)$\\
H.$x \in (6,11] \cup (19,\infty)$
\testStop
\kluczStart
A
\kluczStop



\zadStart{Zadanie z Wikieł Z 1.62 a) moja wersja nr 740}

Rozwiązać nierówności $(x-6)(x-11)(x-20)\ge0$.
\zadStop
\rozwStart{Patryk Wirkus}{}
Miejsca zerowe naszego wielomianu to: $6, 11, 20$.\\
Wielomian jest stopnia nieparzystego, ponadto znak współczynnika przy\linebreak najwyższej potędze x jest dodatni.\\ W związku z tym wykres wielomianu zaczyna się od lewej strony poniżej osi OX. A więc $$x \in [6,11] \cup [20,\infty).$$
\rozwStop
\odpStart
$x \in [6,11] \cup [20,\infty)$
\odpStop
\testStart
A.$x \in [6,11] \cup [20,\infty)$\\
B.$x \in (6,11) \cup [20,\infty)$\\
C.$x \in (6,11] \cup [20,\infty)$\\
D.$x \in [6,11) \cup [20,\infty)$\\
E.$x \in [6,11] \cup (20,\infty)$\\
F.$x \in (6,11) \cup (20,\infty)$\\
G.$x \in [6,11) \cup (20,\infty)$\\
H.$x \in (6,11] \cup (20,\infty)$
\testStop
\kluczStart
A
\kluczStop



\zadStart{Zadanie z Wikieł Z 1.62 a) moja wersja nr 741}

Rozwiązać nierówności $(x-6)(x-12)(x-13)\ge0$.
\zadStop
\rozwStart{Patryk Wirkus}{}
Miejsca zerowe naszego wielomianu to: $6, 12, 13$.\\
Wielomian jest stopnia nieparzystego, ponadto znak współczynnika przy\linebreak najwyższej potędze x jest dodatni.\\ W związku z tym wykres wielomianu zaczyna się od lewej strony poniżej osi OX. A więc $$x \in [6,12] \cup [13,\infty).$$
\rozwStop
\odpStart
$x \in [6,12] \cup [13,\infty)$
\odpStop
\testStart
A.$x \in [6,12] \cup [13,\infty)$\\
B.$x \in (6,12) \cup [13,\infty)$\\
C.$x \in (6,12] \cup [13,\infty)$\\
D.$x \in [6,12) \cup [13,\infty)$\\
E.$x \in [6,12] \cup (13,\infty)$\\
F.$x \in (6,12) \cup (13,\infty)$\\
G.$x \in [6,12) \cup (13,\infty)$\\
H.$x \in (6,12] \cup (13,\infty)$
\testStop
\kluczStart
A
\kluczStop



\zadStart{Zadanie z Wikieł Z 1.62 a) moja wersja nr 742}

Rozwiązać nierówności $(x-6)(x-12)(x-14)\ge0$.
\zadStop
\rozwStart{Patryk Wirkus}{}
Miejsca zerowe naszego wielomianu to: $6, 12, 14$.\\
Wielomian jest stopnia nieparzystego, ponadto znak współczynnika przy\linebreak najwyższej potędze x jest dodatni.\\ W związku z tym wykres wielomianu zaczyna się od lewej strony poniżej osi OX. A więc $$x \in [6,12] \cup [14,\infty).$$
\rozwStop
\odpStart
$x \in [6,12] \cup [14,\infty)$
\odpStop
\testStart
A.$x \in [6,12] \cup [14,\infty)$\\
B.$x \in (6,12) \cup [14,\infty)$\\
C.$x \in (6,12] \cup [14,\infty)$\\
D.$x \in [6,12) \cup [14,\infty)$\\
E.$x \in [6,12] \cup (14,\infty)$\\
F.$x \in (6,12) \cup (14,\infty)$\\
G.$x \in [6,12) \cup (14,\infty)$\\
H.$x \in (6,12] \cup (14,\infty)$
\testStop
\kluczStart
A
\kluczStop



\zadStart{Zadanie z Wikieł Z 1.62 a) moja wersja nr 743}

Rozwiązać nierówności $(x-6)(x-12)(x-15)\ge0$.
\zadStop
\rozwStart{Patryk Wirkus}{}
Miejsca zerowe naszego wielomianu to: $6, 12, 15$.\\
Wielomian jest stopnia nieparzystego, ponadto znak współczynnika przy\linebreak najwyższej potędze x jest dodatni.\\ W związku z tym wykres wielomianu zaczyna się od lewej strony poniżej osi OX. A więc $$x \in [6,12] \cup [15,\infty).$$
\rozwStop
\odpStart
$x \in [6,12] \cup [15,\infty)$
\odpStop
\testStart
A.$x \in [6,12] \cup [15,\infty)$\\
B.$x \in (6,12) \cup [15,\infty)$\\
C.$x \in (6,12] \cup [15,\infty)$\\
D.$x \in [6,12) \cup [15,\infty)$\\
E.$x \in [6,12] \cup (15,\infty)$\\
F.$x \in (6,12) \cup (15,\infty)$\\
G.$x \in [6,12) \cup (15,\infty)$\\
H.$x \in (6,12] \cup (15,\infty)$
\testStop
\kluczStart
A
\kluczStop



\zadStart{Zadanie z Wikieł Z 1.62 a) moja wersja nr 744}

Rozwiązać nierówności $(x-6)(x-12)(x-16)\ge0$.
\zadStop
\rozwStart{Patryk Wirkus}{}
Miejsca zerowe naszego wielomianu to: $6, 12, 16$.\\
Wielomian jest stopnia nieparzystego, ponadto znak współczynnika przy\linebreak najwyższej potędze x jest dodatni.\\ W związku z tym wykres wielomianu zaczyna się od lewej strony poniżej osi OX. A więc $$x \in [6,12] \cup [16,\infty).$$
\rozwStop
\odpStart
$x \in [6,12] \cup [16,\infty)$
\odpStop
\testStart
A.$x \in [6,12] \cup [16,\infty)$\\
B.$x \in (6,12) \cup [16,\infty)$\\
C.$x \in (6,12] \cup [16,\infty)$\\
D.$x \in [6,12) \cup [16,\infty)$\\
E.$x \in [6,12] \cup (16,\infty)$\\
F.$x \in (6,12) \cup (16,\infty)$\\
G.$x \in [6,12) \cup (16,\infty)$\\
H.$x \in (6,12] \cup (16,\infty)$
\testStop
\kluczStart
A
\kluczStop



\zadStart{Zadanie z Wikieł Z 1.62 a) moja wersja nr 745}

Rozwiązać nierówności $(x-6)(x-12)(x-17)\ge0$.
\zadStop
\rozwStart{Patryk Wirkus}{}
Miejsca zerowe naszego wielomianu to: $6, 12, 17$.\\
Wielomian jest stopnia nieparzystego, ponadto znak współczynnika przy\linebreak najwyższej potędze x jest dodatni.\\ W związku z tym wykres wielomianu zaczyna się od lewej strony poniżej osi OX. A więc $$x \in [6,12] \cup [17,\infty).$$
\rozwStop
\odpStart
$x \in [6,12] \cup [17,\infty)$
\odpStop
\testStart
A.$x \in [6,12] \cup [17,\infty)$\\
B.$x \in (6,12) \cup [17,\infty)$\\
C.$x \in (6,12] \cup [17,\infty)$\\
D.$x \in [6,12) \cup [17,\infty)$\\
E.$x \in [6,12] \cup (17,\infty)$\\
F.$x \in (6,12) \cup (17,\infty)$\\
G.$x \in [6,12) \cup (17,\infty)$\\
H.$x \in (6,12] \cup (17,\infty)$
\testStop
\kluczStart
A
\kluczStop



\zadStart{Zadanie z Wikieł Z 1.62 a) moja wersja nr 746}

Rozwiązać nierówności $(x-6)(x-12)(x-18)\ge0$.
\zadStop
\rozwStart{Patryk Wirkus}{}
Miejsca zerowe naszego wielomianu to: $6, 12, 18$.\\
Wielomian jest stopnia nieparzystego, ponadto znak współczynnika przy\linebreak najwyższej potędze x jest dodatni.\\ W związku z tym wykres wielomianu zaczyna się od lewej strony poniżej osi OX. A więc $$x \in [6,12] \cup [18,\infty).$$
\rozwStop
\odpStart
$x \in [6,12] \cup [18,\infty)$
\odpStop
\testStart
A.$x \in [6,12] \cup [18,\infty)$\\
B.$x \in (6,12) \cup [18,\infty)$\\
C.$x \in (6,12] \cup [18,\infty)$\\
D.$x \in [6,12) \cup [18,\infty)$\\
E.$x \in [6,12] \cup (18,\infty)$\\
F.$x \in (6,12) \cup (18,\infty)$\\
G.$x \in [6,12) \cup (18,\infty)$\\
H.$x \in (6,12] \cup (18,\infty)$
\testStop
\kluczStart
A
\kluczStop



\zadStart{Zadanie z Wikieł Z 1.62 a) moja wersja nr 747}

Rozwiązać nierówności $(x-6)(x-12)(x-19)\ge0$.
\zadStop
\rozwStart{Patryk Wirkus}{}
Miejsca zerowe naszego wielomianu to: $6, 12, 19$.\\
Wielomian jest stopnia nieparzystego, ponadto znak współczynnika przy\linebreak najwyższej potędze x jest dodatni.\\ W związku z tym wykres wielomianu zaczyna się od lewej strony poniżej osi OX. A więc $$x \in [6,12] \cup [19,\infty).$$
\rozwStop
\odpStart
$x \in [6,12] \cup [19,\infty)$
\odpStop
\testStart
A.$x \in [6,12] \cup [19,\infty)$\\
B.$x \in (6,12) \cup [19,\infty)$\\
C.$x \in (6,12] \cup [19,\infty)$\\
D.$x \in [6,12) \cup [19,\infty)$\\
E.$x \in [6,12] \cup (19,\infty)$\\
F.$x \in (6,12) \cup (19,\infty)$\\
G.$x \in [6,12) \cup (19,\infty)$\\
H.$x \in (6,12] \cup (19,\infty)$
\testStop
\kluczStart
A
\kluczStop



\zadStart{Zadanie z Wikieł Z 1.62 a) moja wersja nr 748}

Rozwiązać nierówności $(x-6)(x-12)(x-20)\ge0$.
\zadStop
\rozwStart{Patryk Wirkus}{}
Miejsca zerowe naszego wielomianu to: $6, 12, 20$.\\
Wielomian jest stopnia nieparzystego, ponadto znak współczynnika przy\linebreak najwyższej potędze x jest dodatni.\\ W związku z tym wykres wielomianu zaczyna się od lewej strony poniżej osi OX. A więc $$x \in [6,12] \cup [20,\infty).$$
\rozwStop
\odpStart
$x \in [6,12] \cup [20,\infty)$
\odpStop
\testStart
A.$x \in [6,12] \cup [20,\infty)$\\
B.$x \in (6,12) \cup [20,\infty)$\\
C.$x \in (6,12] \cup [20,\infty)$\\
D.$x \in [6,12) \cup [20,\infty)$\\
E.$x \in [6,12] \cup (20,\infty)$\\
F.$x \in (6,12) \cup (20,\infty)$\\
G.$x \in [6,12) \cup (20,\infty)$\\
H.$x \in (6,12] \cup (20,\infty)$
\testStop
\kluczStart
A
\kluczStop



\zadStart{Zadanie z Wikieł Z 1.62 a) moja wersja nr 749}

Rozwiązać nierówności $(x-6)(x-13)(x-14)\ge0$.
\zadStop
\rozwStart{Patryk Wirkus}{}
Miejsca zerowe naszego wielomianu to: $6, 13, 14$.\\
Wielomian jest stopnia nieparzystego, ponadto znak współczynnika przy\linebreak najwyższej potędze x jest dodatni.\\ W związku z tym wykres wielomianu zaczyna się od lewej strony poniżej osi OX. A więc $$x \in [6,13] \cup [14,\infty).$$
\rozwStop
\odpStart
$x \in [6,13] \cup [14,\infty)$
\odpStop
\testStart
A.$x \in [6,13] \cup [14,\infty)$\\
B.$x \in (6,13) \cup [14,\infty)$\\
C.$x \in (6,13] \cup [14,\infty)$\\
D.$x \in [6,13) \cup [14,\infty)$\\
E.$x \in [6,13] \cup (14,\infty)$\\
F.$x \in (6,13) \cup (14,\infty)$\\
G.$x \in [6,13) \cup (14,\infty)$\\
H.$x \in (6,13] \cup (14,\infty)$
\testStop
\kluczStart
A
\kluczStop



\zadStart{Zadanie z Wikieł Z 1.62 a) moja wersja nr 750}

Rozwiązać nierówności $(x-6)(x-13)(x-15)\ge0$.
\zadStop
\rozwStart{Patryk Wirkus}{}
Miejsca zerowe naszego wielomianu to: $6, 13, 15$.\\
Wielomian jest stopnia nieparzystego, ponadto znak współczynnika przy\linebreak najwyższej potędze x jest dodatni.\\ W związku z tym wykres wielomianu zaczyna się od lewej strony poniżej osi OX. A więc $$x \in [6,13] \cup [15,\infty).$$
\rozwStop
\odpStart
$x \in [6,13] \cup [15,\infty)$
\odpStop
\testStart
A.$x \in [6,13] \cup [15,\infty)$\\
B.$x \in (6,13) \cup [15,\infty)$\\
C.$x \in (6,13] \cup [15,\infty)$\\
D.$x \in [6,13) \cup [15,\infty)$\\
E.$x \in [6,13] \cup (15,\infty)$\\
F.$x \in (6,13) \cup (15,\infty)$\\
G.$x \in [6,13) \cup (15,\infty)$\\
H.$x \in (6,13] \cup (15,\infty)$
\testStop
\kluczStart
A
\kluczStop



\zadStart{Zadanie z Wikieł Z 1.62 a) moja wersja nr 751}

Rozwiązać nierówności $(x-6)(x-13)(x-16)\ge0$.
\zadStop
\rozwStart{Patryk Wirkus}{}
Miejsca zerowe naszego wielomianu to: $6, 13, 16$.\\
Wielomian jest stopnia nieparzystego, ponadto znak współczynnika przy\linebreak najwyższej potędze x jest dodatni.\\ W związku z tym wykres wielomianu zaczyna się od lewej strony poniżej osi OX. A więc $$x \in [6,13] \cup [16,\infty).$$
\rozwStop
\odpStart
$x \in [6,13] \cup [16,\infty)$
\odpStop
\testStart
A.$x \in [6,13] \cup [16,\infty)$\\
B.$x \in (6,13) \cup [16,\infty)$\\
C.$x \in (6,13] \cup [16,\infty)$\\
D.$x \in [6,13) \cup [16,\infty)$\\
E.$x \in [6,13] \cup (16,\infty)$\\
F.$x \in (6,13) \cup (16,\infty)$\\
G.$x \in [6,13) \cup (16,\infty)$\\
H.$x \in (6,13] \cup (16,\infty)$
\testStop
\kluczStart
A
\kluczStop



\zadStart{Zadanie z Wikieł Z 1.62 a) moja wersja nr 752}

Rozwiązać nierówności $(x-6)(x-13)(x-17)\ge0$.
\zadStop
\rozwStart{Patryk Wirkus}{}
Miejsca zerowe naszego wielomianu to: $6, 13, 17$.\\
Wielomian jest stopnia nieparzystego, ponadto znak współczynnika przy\linebreak najwyższej potędze x jest dodatni.\\ W związku z tym wykres wielomianu zaczyna się od lewej strony poniżej osi OX. A więc $$x \in [6,13] \cup [17,\infty).$$
\rozwStop
\odpStart
$x \in [6,13] \cup [17,\infty)$
\odpStop
\testStart
A.$x \in [6,13] \cup [17,\infty)$\\
B.$x \in (6,13) \cup [17,\infty)$\\
C.$x \in (6,13] \cup [17,\infty)$\\
D.$x \in [6,13) \cup [17,\infty)$\\
E.$x \in [6,13] \cup (17,\infty)$\\
F.$x \in (6,13) \cup (17,\infty)$\\
G.$x \in [6,13) \cup (17,\infty)$\\
H.$x \in (6,13] \cup (17,\infty)$
\testStop
\kluczStart
A
\kluczStop



\zadStart{Zadanie z Wikieł Z 1.62 a) moja wersja nr 753}

Rozwiązać nierówności $(x-6)(x-13)(x-18)\ge0$.
\zadStop
\rozwStart{Patryk Wirkus}{}
Miejsca zerowe naszego wielomianu to: $6, 13, 18$.\\
Wielomian jest stopnia nieparzystego, ponadto znak współczynnika przy\linebreak najwyższej potędze x jest dodatni.\\ W związku z tym wykres wielomianu zaczyna się od lewej strony poniżej osi OX. A więc $$x \in [6,13] \cup [18,\infty).$$
\rozwStop
\odpStart
$x \in [6,13] \cup [18,\infty)$
\odpStop
\testStart
A.$x \in [6,13] \cup [18,\infty)$\\
B.$x \in (6,13) \cup [18,\infty)$\\
C.$x \in (6,13] \cup [18,\infty)$\\
D.$x \in [6,13) \cup [18,\infty)$\\
E.$x \in [6,13] \cup (18,\infty)$\\
F.$x \in (6,13) \cup (18,\infty)$\\
G.$x \in [6,13) \cup (18,\infty)$\\
H.$x \in (6,13] \cup (18,\infty)$
\testStop
\kluczStart
A
\kluczStop



\zadStart{Zadanie z Wikieł Z 1.62 a) moja wersja nr 754}

Rozwiązać nierówności $(x-6)(x-13)(x-19)\ge0$.
\zadStop
\rozwStart{Patryk Wirkus}{}
Miejsca zerowe naszego wielomianu to: $6, 13, 19$.\\
Wielomian jest stopnia nieparzystego, ponadto znak współczynnika przy\linebreak najwyższej potędze x jest dodatni.\\ W związku z tym wykres wielomianu zaczyna się od lewej strony poniżej osi OX. A więc $$x \in [6,13] \cup [19,\infty).$$
\rozwStop
\odpStart
$x \in [6,13] \cup [19,\infty)$
\odpStop
\testStart
A.$x \in [6,13] \cup [19,\infty)$\\
B.$x \in (6,13) \cup [19,\infty)$\\
C.$x \in (6,13] \cup [19,\infty)$\\
D.$x \in [6,13) \cup [19,\infty)$\\
E.$x \in [6,13] \cup (19,\infty)$\\
F.$x \in (6,13) \cup (19,\infty)$\\
G.$x \in [6,13) \cup (19,\infty)$\\
H.$x \in (6,13] \cup (19,\infty)$
\testStop
\kluczStart
A
\kluczStop



\zadStart{Zadanie z Wikieł Z 1.62 a) moja wersja nr 755}

Rozwiązać nierówności $(x-6)(x-13)(x-20)\ge0$.
\zadStop
\rozwStart{Patryk Wirkus}{}
Miejsca zerowe naszego wielomianu to: $6, 13, 20$.\\
Wielomian jest stopnia nieparzystego, ponadto znak współczynnika przy\linebreak najwyższej potędze x jest dodatni.\\ W związku z tym wykres wielomianu zaczyna się od lewej strony poniżej osi OX. A więc $$x \in [6,13] \cup [20,\infty).$$
\rozwStop
\odpStart
$x \in [6,13] \cup [20,\infty)$
\odpStop
\testStart
A.$x \in [6,13] \cup [20,\infty)$\\
B.$x \in (6,13) \cup [20,\infty)$\\
C.$x \in (6,13] \cup [20,\infty)$\\
D.$x \in [6,13) \cup [20,\infty)$\\
E.$x \in [6,13] \cup (20,\infty)$\\
F.$x \in (6,13) \cup (20,\infty)$\\
G.$x \in [6,13) \cup (20,\infty)$\\
H.$x \in (6,13] \cup (20,\infty)$
\testStop
\kluczStart
A
\kluczStop



\zadStart{Zadanie z Wikieł Z 1.62 a) moja wersja nr 756}

Rozwiązać nierówności $(x-6)(x-14)(x-15)\ge0$.
\zadStop
\rozwStart{Patryk Wirkus}{}
Miejsca zerowe naszego wielomianu to: $6, 14, 15$.\\
Wielomian jest stopnia nieparzystego, ponadto znak współczynnika przy\linebreak najwyższej potędze x jest dodatni.\\ W związku z tym wykres wielomianu zaczyna się od lewej strony poniżej osi OX. A więc $$x \in [6,14] \cup [15,\infty).$$
\rozwStop
\odpStart
$x \in [6,14] \cup [15,\infty)$
\odpStop
\testStart
A.$x \in [6,14] \cup [15,\infty)$\\
B.$x \in (6,14) \cup [15,\infty)$\\
C.$x \in (6,14] \cup [15,\infty)$\\
D.$x \in [6,14) \cup [15,\infty)$\\
E.$x \in [6,14] \cup (15,\infty)$\\
F.$x \in (6,14) \cup (15,\infty)$\\
G.$x \in [6,14) \cup (15,\infty)$\\
H.$x \in (6,14] \cup (15,\infty)$
\testStop
\kluczStart
A
\kluczStop



\zadStart{Zadanie z Wikieł Z 1.62 a) moja wersja nr 757}

Rozwiązać nierówności $(x-6)(x-14)(x-16)\ge0$.
\zadStop
\rozwStart{Patryk Wirkus}{}
Miejsca zerowe naszego wielomianu to: $6, 14, 16$.\\
Wielomian jest stopnia nieparzystego, ponadto znak współczynnika przy\linebreak najwyższej potędze x jest dodatni.\\ W związku z tym wykres wielomianu zaczyna się od lewej strony poniżej osi OX. A więc $$x \in [6,14] \cup [16,\infty).$$
\rozwStop
\odpStart
$x \in [6,14] \cup [16,\infty)$
\odpStop
\testStart
A.$x \in [6,14] \cup [16,\infty)$\\
B.$x \in (6,14) \cup [16,\infty)$\\
C.$x \in (6,14] \cup [16,\infty)$\\
D.$x \in [6,14) \cup [16,\infty)$\\
E.$x \in [6,14] \cup (16,\infty)$\\
F.$x \in (6,14) \cup (16,\infty)$\\
G.$x \in [6,14) \cup (16,\infty)$\\
H.$x \in (6,14] \cup (16,\infty)$
\testStop
\kluczStart
A
\kluczStop



\zadStart{Zadanie z Wikieł Z 1.62 a) moja wersja nr 758}

Rozwiązać nierówności $(x-6)(x-14)(x-17)\ge0$.
\zadStop
\rozwStart{Patryk Wirkus}{}
Miejsca zerowe naszego wielomianu to: $6, 14, 17$.\\
Wielomian jest stopnia nieparzystego, ponadto znak współczynnika przy\linebreak najwyższej potędze x jest dodatni.\\ W związku z tym wykres wielomianu zaczyna się od lewej strony poniżej osi OX. A więc $$x \in [6,14] \cup [17,\infty).$$
\rozwStop
\odpStart
$x \in [6,14] \cup [17,\infty)$
\odpStop
\testStart
A.$x \in [6,14] \cup [17,\infty)$\\
B.$x \in (6,14) \cup [17,\infty)$\\
C.$x \in (6,14] \cup [17,\infty)$\\
D.$x \in [6,14) \cup [17,\infty)$\\
E.$x \in [6,14] \cup (17,\infty)$\\
F.$x \in (6,14) \cup (17,\infty)$\\
G.$x \in [6,14) \cup (17,\infty)$\\
H.$x \in (6,14] \cup (17,\infty)$
\testStop
\kluczStart
A
\kluczStop



\zadStart{Zadanie z Wikieł Z 1.62 a) moja wersja nr 759}

Rozwiązać nierówności $(x-6)(x-14)(x-18)\ge0$.
\zadStop
\rozwStart{Patryk Wirkus}{}
Miejsca zerowe naszego wielomianu to: $6, 14, 18$.\\
Wielomian jest stopnia nieparzystego, ponadto znak współczynnika przy\linebreak najwyższej potędze x jest dodatni.\\ W związku z tym wykres wielomianu zaczyna się od lewej strony poniżej osi OX. A więc $$x \in [6,14] \cup [18,\infty).$$
\rozwStop
\odpStart
$x \in [6,14] \cup [18,\infty)$
\odpStop
\testStart
A.$x \in [6,14] \cup [18,\infty)$\\
B.$x \in (6,14) \cup [18,\infty)$\\
C.$x \in (6,14] \cup [18,\infty)$\\
D.$x \in [6,14) \cup [18,\infty)$\\
E.$x \in [6,14] \cup (18,\infty)$\\
F.$x \in (6,14) \cup (18,\infty)$\\
G.$x \in [6,14) \cup (18,\infty)$\\
H.$x \in (6,14] \cup (18,\infty)$
\testStop
\kluczStart
A
\kluczStop



\zadStart{Zadanie z Wikieł Z 1.62 a) moja wersja nr 760}

Rozwiązać nierówności $(x-6)(x-14)(x-19)\ge0$.
\zadStop
\rozwStart{Patryk Wirkus}{}
Miejsca zerowe naszego wielomianu to: $6, 14, 19$.\\
Wielomian jest stopnia nieparzystego, ponadto znak współczynnika przy\linebreak najwyższej potędze x jest dodatni.\\ W związku z tym wykres wielomianu zaczyna się od lewej strony poniżej osi OX. A więc $$x \in [6,14] \cup [19,\infty).$$
\rozwStop
\odpStart
$x \in [6,14] \cup [19,\infty)$
\odpStop
\testStart
A.$x \in [6,14] \cup [19,\infty)$\\
B.$x \in (6,14) \cup [19,\infty)$\\
C.$x \in (6,14] \cup [19,\infty)$\\
D.$x \in [6,14) \cup [19,\infty)$\\
E.$x \in [6,14] \cup (19,\infty)$\\
F.$x \in (6,14) \cup (19,\infty)$\\
G.$x \in [6,14) \cup (19,\infty)$\\
H.$x \in (6,14] \cup (19,\infty)$
\testStop
\kluczStart
A
\kluczStop



\zadStart{Zadanie z Wikieł Z 1.62 a) moja wersja nr 761}

Rozwiązać nierówności $(x-6)(x-14)(x-20)\ge0$.
\zadStop
\rozwStart{Patryk Wirkus}{}
Miejsca zerowe naszego wielomianu to: $6, 14, 20$.\\
Wielomian jest stopnia nieparzystego, ponadto znak współczynnika przy\linebreak najwyższej potędze x jest dodatni.\\ W związku z tym wykres wielomianu zaczyna się od lewej strony poniżej osi OX. A więc $$x \in [6,14] \cup [20,\infty).$$
\rozwStop
\odpStart
$x \in [6,14] \cup [20,\infty)$
\odpStop
\testStart
A.$x \in [6,14] \cup [20,\infty)$\\
B.$x \in (6,14) \cup [20,\infty)$\\
C.$x \in (6,14] \cup [20,\infty)$\\
D.$x \in [6,14) \cup [20,\infty)$\\
E.$x \in [6,14] \cup (20,\infty)$\\
F.$x \in (6,14) \cup (20,\infty)$\\
G.$x \in [6,14) \cup (20,\infty)$\\
H.$x \in (6,14] \cup (20,\infty)$
\testStop
\kluczStart
A
\kluczStop



\zadStart{Zadanie z Wikieł Z 1.62 a) moja wersja nr 762}

Rozwiązać nierówności $(x-6)(x-15)(x-16)\ge0$.
\zadStop
\rozwStart{Patryk Wirkus}{}
Miejsca zerowe naszego wielomianu to: $6, 15, 16$.\\
Wielomian jest stopnia nieparzystego, ponadto znak współczynnika przy\linebreak najwyższej potędze x jest dodatni.\\ W związku z tym wykres wielomianu zaczyna się od lewej strony poniżej osi OX. A więc $$x \in [6,15] \cup [16,\infty).$$
\rozwStop
\odpStart
$x \in [6,15] \cup [16,\infty)$
\odpStop
\testStart
A.$x \in [6,15] \cup [16,\infty)$\\
B.$x \in (6,15) \cup [16,\infty)$\\
C.$x \in (6,15] \cup [16,\infty)$\\
D.$x \in [6,15) \cup [16,\infty)$\\
E.$x \in [6,15] \cup (16,\infty)$\\
F.$x \in (6,15) \cup (16,\infty)$\\
G.$x \in [6,15) \cup (16,\infty)$\\
H.$x \in (6,15] \cup (16,\infty)$
\testStop
\kluczStart
A
\kluczStop



\zadStart{Zadanie z Wikieł Z 1.62 a) moja wersja nr 763}

Rozwiązać nierówności $(x-6)(x-15)(x-17)\ge0$.
\zadStop
\rozwStart{Patryk Wirkus}{}
Miejsca zerowe naszego wielomianu to: $6, 15, 17$.\\
Wielomian jest stopnia nieparzystego, ponadto znak współczynnika przy\linebreak najwyższej potędze x jest dodatni.\\ W związku z tym wykres wielomianu zaczyna się od lewej strony poniżej osi OX. A więc $$x \in [6,15] \cup [17,\infty).$$
\rozwStop
\odpStart
$x \in [6,15] \cup [17,\infty)$
\odpStop
\testStart
A.$x \in [6,15] \cup [17,\infty)$\\
B.$x \in (6,15) \cup [17,\infty)$\\
C.$x \in (6,15] \cup [17,\infty)$\\
D.$x \in [6,15) \cup [17,\infty)$\\
E.$x \in [6,15] \cup (17,\infty)$\\
F.$x \in (6,15) \cup (17,\infty)$\\
G.$x \in [6,15) \cup (17,\infty)$\\
H.$x \in (6,15] \cup (17,\infty)$
\testStop
\kluczStart
A
\kluczStop



\zadStart{Zadanie z Wikieł Z 1.62 a) moja wersja nr 764}

Rozwiązać nierówności $(x-6)(x-15)(x-18)\ge0$.
\zadStop
\rozwStart{Patryk Wirkus}{}
Miejsca zerowe naszego wielomianu to: $6, 15, 18$.\\
Wielomian jest stopnia nieparzystego, ponadto znak współczynnika przy\linebreak najwyższej potędze x jest dodatni.\\ W związku z tym wykres wielomianu zaczyna się od lewej strony poniżej osi OX. A więc $$x \in [6,15] \cup [18,\infty).$$
\rozwStop
\odpStart
$x \in [6,15] \cup [18,\infty)$
\odpStop
\testStart
A.$x \in [6,15] \cup [18,\infty)$\\
B.$x \in (6,15) \cup [18,\infty)$\\
C.$x \in (6,15] \cup [18,\infty)$\\
D.$x \in [6,15) \cup [18,\infty)$\\
E.$x \in [6,15] \cup (18,\infty)$\\
F.$x \in (6,15) \cup (18,\infty)$\\
G.$x \in [6,15) \cup (18,\infty)$\\
H.$x \in (6,15] \cup (18,\infty)$
\testStop
\kluczStart
A
\kluczStop



\zadStart{Zadanie z Wikieł Z 1.62 a) moja wersja nr 765}

Rozwiązać nierówności $(x-6)(x-15)(x-19)\ge0$.
\zadStop
\rozwStart{Patryk Wirkus}{}
Miejsca zerowe naszego wielomianu to: $6, 15, 19$.\\
Wielomian jest stopnia nieparzystego, ponadto znak współczynnika przy\linebreak najwyższej potędze x jest dodatni.\\ W związku z tym wykres wielomianu zaczyna się od lewej strony poniżej osi OX. A więc $$x \in [6,15] \cup [19,\infty).$$
\rozwStop
\odpStart
$x \in [6,15] \cup [19,\infty)$
\odpStop
\testStart
A.$x \in [6,15] \cup [19,\infty)$\\
B.$x \in (6,15) \cup [19,\infty)$\\
C.$x \in (6,15] \cup [19,\infty)$\\
D.$x \in [6,15) \cup [19,\infty)$\\
E.$x \in [6,15] \cup (19,\infty)$\\
F.$x \in (6,15) \cup (19,\infty)$\\
G.$x \in [6,15) \cup (19,\infty)$\\
H.$x \in (6,15] \cup (19,\infty)$
\testStop
\kluczStart
A
\kluczStop



\zadStart{Zadanie z Wikieł Z 1.62 a) moja wersja nr 766}

Rozwiązać nierówności $(x-6)(x-15)(x-20)\ge0$.
\zadStop
\rozwStart{Patryk Wirkus}{}
Miejsca zerowe naszego wielomianu to: $6, 15, 20$.\\
Wielomian jest stopnia nieparzystego, ponadto znak współczynnika przy\linebreak najwyższej potędze x jest dodatni.\\ W związku z tym wykres wielomianu zaczyna się od lewej strony poniżej osi OX. A więc $$x \in [6,15] \cup [20,\infty).$$
\rozwStop
\odpStart
$x \in [6,15] \cup [20,\infty)$
\odpStop
\testStart
A.$x \in [6,15] \cup [20,\infty)$\\
B.$x \in (6,15) \cup [20,\infty)$\\
C.$x \in (6,15] \cup [20,\infty)$\\
D.$x \in [6,15) \cup [20,\infty)$\\
E.$x \in [6,15] \cup (20,\infty)$\\
F.$x \in (6,15) \cup (20,\infty)$\\
G.$x \in [6,15) \cup (20,\infty)$\\
H.$x \in (6,15] \cup (20,\infty)$
\testStop
\kluczStart
A
\kluczStop



\zadStart{Zadanie z Wikieł Z 1.62 a) moja wersja nr 767}

Rozwiązać nierówności $(x-6)(x-16)(x-17)\ge0$.
\zadStop
\rozwStart{Patryk Wirkus}{}
Miejsca zerowe naszego wielomianu to: $6, 16, 17$.\\
Wielomian jest stopnia nieparzystego, ponadto znak współczynnika przy\linebreak najwyższej potędze x jest dodatni.\\ W związku z tym wykres wielomianu zaczyna się od lewej strony poniżej osi OX. A więc $$x \in [6,16] \cup [17,\infty).$$
\rozwStop
\odpStart
$x \in [6,16] \cup [17,\infty)$
\odpStop
\testStart
A.$x \in [6,16] \cup [17,\infty)$\\
B.$x \in (6,16) \cup [17,\infty)$\\
C.$x \in (6,16] \cup [17,\infty)$\\
D.$x \in [6,16) \cup [17,\infty)$\\
E.$x \in [6,16] \cup (17,\infty)$\\
F.$x \in (6,16) \cup (17,\infty)$\\
G.$x \in [6,16) \cup (17,\infty)$\\
H.$x \in (6,16] \cup (17,\infty)$
\testStop
\kluczStart
A
\kluczStop



\zadStart{Zadanie z Wikieł Z 1.62 a) moja wersja nr 768}

Rozwiązać nierówności $(x-6)(x-16)(x-18)\ge0$.
\zadStop
\rozwStart{Patryk Wirkus}{}
Miejsca zerowe naszego wielomianu to: $6, 16, 18$.\\
Wielomian jest stopnia nieparzystego, ponadto znak współczynnika przy\linebreak najwyższej potędze x jest dodatni.\\ W związku z tym wykres wielomianu zaczyna się od lewej strony poniżej osi OX. A więc $$x \in [6,16] \cup [18,\infty).$$
\rozwStop
\odpStart
$x \in [6,16] \cup [18,\infty)$
\odpStop
\testStart
A.$x \in [6,16] \cup [18,\infty)$\\
B.$x \in (6,16) \cup [18,\infty)$\\
C.$x \in (6,16] \cup [18,\infty)$\\
D.$x \in [6,16) \cup [18,\infty)$\\
E.$x \in [6,16] \cup (18,\infty)$\\
F.$x \in (6,16) \cup (18,\infty)$\\
G.$x \in [6,16) \cup (18,\infty)$\\
H.$x \in (6,16] \cup (18,\infty)$
\testStop
\kluczStart
A
\kluczStop



\zadStart{Zadanie z Wikieł Z 1.62 a) moja wersja nr 769}

Rozwiązać nierówności $(x-6)(x-16)(x-19)\ge0$.
\zadStop
\rozwStart{Patryk Wirkus}{}
Miejsca zerowe naszego wielomianu to: $6, 16, 19$.\\
Wielomian jest stopnia nieparzystego, ponadto znak współczynnika przy\linebreak najwyższej potędze x jest dodatni.\\ W związku z tym wykres wielomianu zaczyna się od lewej strony poniżej osi OX. A więc $$x \in [6,16] \cup [19,\infty).$$
\rozwStop
\odpStart
$x \in [6,16] \cup [19,\infty)$
\odpStop
\testStart
A.$x \in [6,16] \cup [19,\infty)$\\
B.$x \in (6,16) \cup [19,\infty)$\\
C.$x \in (6,16] \cup [19,\infty)$\\
D.$x \in [6,16) \cup [19,\infty)$\\
E.$x \in [6,16] \cup (19,\infty)$\\
F.$x \in (6,16) \cup (19,\infty)$\\
G.$x \in [6,16) \cup (19,\infty)$\\
H.$x \in (6,16] \cup (19,\infty)$
\testStop
\kluczStart
A
\kluczStop



\zadStart{Zadanie z Wikieł Z 1.62 a) moja wersja nr 770}

Rozwiązać nierówności $(x-6)(x-16)(x-20)\ge0$.
\zadStop
\rozwStart{Patryk Wirkus}{}
Miejsca zerowe naszego wielomianu to: $6, 16, 20$.\\
Wielomian jest stopnia nieparzystego, ponadto znak współczynnika przy\linebreak najwyższej potędze x jest dodatni.\\ W związku z tym wykres wielomianu zaczyna się od lewej strony poniżej osi OX. A więc $$x \in [6,16] \cup [20,\infty).$$
\rozwStop
\odpStart
$x \in [6,16] \cup [20,\infty)$
\odpStop
\testStart
A.$x \in [6,16] \cup [20,\infty)$\\
B.$x \in (6,16) \cup [20,\infty)$\\
C.$x \in (6,16] \cup [20,\infty)$\\
D.$x \in [6,16) \cup [20,\infty)$\\
E.$x \in [6,16] \cup (20,\infty)$\\
F.$x \in (6,16) \cup (20,\infty)$\\
G.$x \in [6,16) \cup (20,\infty)$\\
H.$x \in (6,16] \cup (20,\infty)$
\testStop
\kluczStart
A
\kluczStop



\zadStart{Zadanie z Wikieł Z 1.62 a) moja wersja nr 771}

Rozwiązać nierówności $(x-6)(x-17)(x-18)\ge0$.
\zadStop
\rozwStart{Patryk Wirkus}{}
Miejsca zerowe naszego wielomianu to: $6, 17, 18$.\\
Wielomian jest stopnia nieparzystego, ponadto znak współczynnika przy\linebreak najwyższej potędze x jest dodatni.\\ W związku z tym wykres wielomianu zaczyna się od lewej strony poniżej osi OX. A więc $$x \in [6,17] \cup [18,\infty).$$
\rozwStop
\odpStart
$x \in [6,17] \cup [18,\infty)$
\odpStop
\testStart
A.$x \in [6,17] \cup [18,\infty)$\\
B.$x \in (6,17) \cup [18,\infty)$\\
C.$x \in (6,17] \cup [18,\infty)$\\
D.$x \in [6,17) \cup [18,\infty)$\\
E.$x \in [6,17] \cup (18,\infty)$\\
F.$x \in (6,17) \cup (18,\infty)$\\
G.$x \in [6,17) \cup (18,\infty)$\\
H.$x \in (6,17] \cup (18,\infty)$
\testStop
\kluczStart
A
\kluczStop



\zadStart{Zadanie z Wikieł Z 1.62 a) moja wersja nr 772}

Rozwiązać nierówności $(x-6)(x-17)(x-19)\ge0$.
\zadStop
\rozwStart{Patryk Wirkus}{}
Miejsca zerowe naszego wielomianu to: $6, 17, 19$.\\
Wielomian jest stopnia nieparzystego, ponadto znak współczynnika przy\linebreak najwyższej potędze x jest dodatni.\\ W związku z tym wykres wielomianu zaczyna się od lewej strony poniżej osi OX. A więc $$x \in [6,17] \cup [19,\infty).$$
\rozwStop
\odpStart
$x \in [6,17] \cup [19,\infty)$
\odpStop
\testStart
A.$x \in [6,17] \cup [19,\infty)$\\
B.$x \in (6,17) \cup [19,\infty)$\\
C.$x \in (6,17] \cup [19,\infty)$\\
D.$x \in [6,17) \cup [19,\infty)$\\
E.$x \in [6,17] \cup (19,\infty)$\\
F.$x \in (6,17) \cup (19,\infty)$\\
G.$x \in [6,17) \cup (19,\infty)$\\
H.$x \in (6,17] \cup (19,\infty)$
\testStop
\kluczStart
A
\kluczStop



\zadStart{Zadanie z Wikieł Z 1.62 a) moja wersja nr 773}

Rozwiązać nierówności $(x-6)(x-17)(x-20)\ge0$.
\zadStop
\rozwStart{Patryk Wirkus}{}
Miejsca zerowe naszego wielomianu to: $6, 17, 20$.\\
Wielomian jest stopnia nieparzystego, ponadto znak współczynnika przy\linebreak najwyższej potędze x jest dodatni.\\ W związku z tym wykres wielomianu zaczyna się od lewej strony poniżej osi OX. A więc $$x \in [6,17] \cup [20,\infty).$$
\rozwStop
\odpStart
$x \in [6,17] \cup [20,\infty)$
\odpStop
\testStart
A.$x \in [6,17] \cup [20,\infty)$\\
B.$x \in (6,17) \cup [20,\infty)$\\
C.$x \in (6,17] \cup [20,\infty)$\\
D.$x \in [6,17) \cup [20,\infty)$\\
E.$x \in [6,17] \cup (20,\infty)$\\
F.$x \in (6,17) \cup (20,\infty)$\\
G.$x \in [6,17) \cup (20,\infty)$\\
H.$x \in (6,17] \cup (20,\infty)$
\testStop
\kluczStart
A
\kluczStop



\zadStart{Zadanie z Wikieł Z 1.62 a) moja wersja nr 774}

Rozwiązać nierówności $(x-6)(x-18)(x-19)\ge0$.
\zadStop
\rozwStart{Patryk Wirkus}{}
Miejsca zerowe naszego wielomianu to: $6, 18, 19$.\\
Wielomian jest stopnia nieparzystego, ponadto znak współczynnika przy\linebreak najwyższej potędze x jest dodatni.\\ W związku z tym wykres wielomianu zaczyna się od lewej strony poniżej osi OX. A więc $$x \in [6,18] \cup [19,\infty).$$
\rozwStop
\odpStart
$x \in [6,18] \cup [19,\infty)$
\odpStop
\testStart
A.$x \in [6,18] \cup [19,\infty)$\\
B.$x \in (6,18) \cup [19,\infty)$\\
C.$x \in (6,18] \cup [19,\infty)$\\
D.$x \in [6,18) \cup [19,\infty)$\\
E.$x \in [6,18] \cup (19,\infty)$\\
F.$x \in (6,18) \cup (19,\infty)$\\
G.$x \in [6,18) \cup (19,\infty)$\\
H.$x \in (6,18] \cup (19,\infty)$
\testStop
\kluczStart
A
\kluczStop



\zadStart{Zadanie z Wikieł Z 1.62 a) moja wersja nr 775}

Rozwiązać nierówności $(x-6)(x-18)(x-20)\ge0$.
\zadStop
\rozwStart{Patryk Wirkus}{}
Miejsca zerowe naszego wielomianu to: $6, 18, 20$.\\
Wielomian jest stopnia nieparzystego, ponadto znak współczynnika przy\linebreak najwyższej potędze x jest dodatni.\\ W związku z tym wykres wielomianu zaczyna się od lewej strony poniżej osi OX. A więc $$x \in [6,18] \cup [20,\infty).$$
\rozwStop
\odpStart
$x \in [6,18] \cup [20,\infty)$
\odpStop
\testStart
A.$x \in [6,18] \cup [20,\infty)$\\
B.$x \in (6,18) \cup [20,\infty)$\\
C.$x \in (6,18] \cup [20,\infty)$\\
D.$x \in [6,18) \cup [20,\infty)$\\
E.$x \in [6,18] \cup (20,\infty)$\\
F.$x \in (6,18) \cup (20,\infty)$\\
G.$x \in [6,18) \cup (20,\infty)$\\
H.$x \in (6,18] \cup (20,\infty)$
\testStop
\kluczStart
A
\kluczStop



\zadStart{Zadanie z Wikieł Z 1.62 a) moja wersja nr 776}

Rozwiązać nierówności $(x-6)(x-19)(x-20)\ge0$.
\zadStop
\rozwStart{Patryk Wirkus}{}
Miejsca zerowe naszego wielomianu to: $6, 19, 20$.\\
Wielomian jest stopnia nieparzystego, ponadto znak współczynnika przy\linebreak najwyższej potędze x jest dodatni.\\ W związku z tym wykres wielomianu zaczyna się od lewej strony poniżej osi OX. A więc $$x \in [6,19] \cup [20,\infty).$$
\rozwStop
\odpStart
$x \in [6,19] \cup [20,\infty)$
\odpStop
\testStart
A.$x \in [6,19] \cup [20,\infty)$\\
B.$x \in (6,19) \cup [20,\infty)$\\
C.$x \in (6,19] \cup [20,\infty)$\\
D.$x \in [6,19) \cup [20,\infty)$\\
E.$x \in [6,19] \cup (20,\infty)$\\
F.$x \in (6,19) \cup (20,\infty)$\\
G.$x \in [6,19) \cup (20,\infty)$\\
H.$x \in (6,19] \cup (20,\infty)$
\testStop
\kluczStart
A
\kluczStop



\zadStart{Zadanie z Wikieł Z 1.62 a) moja wersja nr 777}

Rozwiązać nierówności $(x-7)(x-8)(x-9)\ge0$.
\zadStop
\rozwStart{Patryk Wirkus}{}
Miejsca zerowe naszego wielomianu to: $7, 8, 9$.\\
Wielomian jest stopnia nieparzystego, ponadto znak współczynnika przy\linebreak najwyższej potędze x jest dodatni.\\ W związku z tym wykres wielomianu zaczyna się od lewej strony poniżej osi OX. A więc $$x \in [7,8] \cup [9,\infty).$$
\rozwStop
\odpStart
$x \in [7,8] \cup [9,\infty)$
\odpStop
\testStart
A.$x \in [7,8] \cup [9,\infty)$\\
B.$x \in (7,8) \cup [9,\infty)$\\
C.$x \in (7,8] \cup [9,\infty)$\\
D.$x \in [7,8) \cup [9,\infty)$\\
E.$x \in [7,8] \cup (9,\infty)$\\
F.$x \in (7,8) \cup (9,\infty)$\\
G.$x \in [7,8) \cup (9,\infty)$\\
H.$x \in (7,8] \cup (9,\infty)$
\testStop
\kluczStart
A
\kluczStop



\zadStart{Zadanie z Wikieł Z 1.62 a) moja wersja nr 778}

Rozwiązać nierówności $(x-7)(x-8)(x-10)\ge0$.
\zadStop
\rozwStart{Patryk Wirkus}{}
Miejsca zerowe naszego wielomianu to: $7, 8, 10$.\\
Wielomian jest stopnia nieparzystego, ponadto znak współczynnika przy\linebreak najwyższej potędze x jest dodatni.\\ W związku z tym wykres wielomianu zaczyna się od lewej strony poniżej osi OX. A więc $$x \in [7,8] \cup [10,\infty).$$
\rozwStop
\odpStart
$x \in [7,8] \cup [10,\infty)$
\odpStop
\testStart
A.$x \in [7,8] \cup [10,\infty)$\\
B.$x \in (7,8) \cup [10,\infty)$\\
C.$x \in (7,8] \cup [10,\infty)$\\
D.$x \in [7,8) \cup [10,\infty)$\\
E.$x \in [7,8] \cup (10,\infty)$\\
F.$x \in (7,8) \cup (10,\infty)$\\
G.$x \in [7,8) \cup (10,\infty)$\\
H.$x \in (7,8] \cup (10,\infty)$
\testStop
\kluczStart
A
\kluczStop



\zadStart{Zadanie z Wikieł Z 1.62 a) moja wersja nr 779}

Rozwiązać nierówności $(x-7)(x-8)(x-11)\ge0$.
\zadStop
\rozwStart{Patryk Wirkus}{}
Miejsca zerowe naszego wielomianu to: $7, 8, 11$.\\
Wielomian jest stopnia nieparzystego, ponadto znak współczynnika przy\linebreak najwyższej potędze x jest dodatni.\\ W związku z tym wykres wielomianu zaczyna się od lewej strony poniżej osi OX. A więc $$x \in [7,8] \cup [11,\infty).$$
\rozwStop
\odpStart
$x \in [7,8] \cup [11,\infty)$
\odpStop
\testStart
A.$x \in [7,8] \cup [11,\infty)$\\
B.$x \in (7,8) \cup [11,\infty)$\\
C.$x \in (7,8] \cup [11,\infty)$\\
D.$x \in [7,8) \cup [11,\infty)$\\
E.$x \in [7,8] \cup (11,\infty)$\\
F.$x \in (7,8) \cup (11,\infty)$\\
G.$x \in [7,8) \cup (11,\infty)$\\
H.$x \in (7,8] \cup (11,\infty)$
\testStop
\kluczStart
A
\kluczStop



\zadStart{Zadanie z Wikieł Z 1.62 a) moja wersja nr 780}

Rozwiązać nierówności $(x-7)(x-8)(x-12)\ge0$.
\zadStop
\rozwStart{Patryk Wirkus}{}
Miejsca zerowe naszego wielomianu to: $7, 8, 12$.\\
Wielomian jest stopnia nieparzystego, ponadto znak współczynnika przy\linebreak najwyższej potędze x jest dodatni.\\ W związku z tym wykres wielomianu zaczyna się od lewej strony poniżej osi OX. A więc $$x \in [7,8] \cup [12,\infty).$$
\rozwStop
\odpStart
$x \in [7,8] \cup [12,\infty)$
\odpStop
\testStart
A.$x \in [7,8] \cup [12,\infty)$\\
B.$x \in (7,8) \cup [12,\infty)$\\
C.$x \in (7,8] \cup [12,\infty)$\\
D.$x \in [7,8) \cup [12,\infty)$\\
E.$x \in [7,8] \cup (12,\infty)$\\
F.$x \in (7,8) \cup (12,\infty)$\\
G.$x \in [7,8) \cup (12,\infty)$\\
H.$x \in (7,8] \cup (12,\infty)$
\testStop
\kluczStart
A
\kluczStop



\zadStart{Zadanie z Wikieł Z 1.62 a) moja wersja nr 781}

Rozwiązać nierówności $(x-7)(x-8)(x-13)\ge0$.
\zadStop
\rozwStart{Patryk Wirkus}{}
Miejsca zerowe naszego wielomianu to: $7, 8, 13$.\\
Wielomian jest stopnia nieparzystego, ponadto znak współczynnika przy\linebreak najwyższej potędze x jest dodatni.\\ W związku z tym wykres wielomianu zaczyna się od lewej strony poniżej osi OX. A więc $$x \in [7,8] \cup [13,\infty).$$
\rozwStop
\odpStart
$x \in [7,8] \cup [13,\infty)$
\odpStop
\testStart
A.$x \in [7,8] \cup [13,\infty)$\\
B.$x \in (7,8) \cup [13,\infty)$\\
C.$x \in (7,8] \cup [13,\infty)$\\
D.$x \in [7,8) \cup [13,\infty)$\\
E.$x \in [7,8] \cup (13,\infty)$\\
F.$x \in (7,8) \cup (13,\infty)$\\
G.$x \in [7,8) \cup (13,\infty)$\\
H.$x \in (7,8] \cup (13,\infty)$
\testStop
\kluczStart
A
\kluczStop



\zadStart{Zadanie z Wikieł Z 1.62 a) moja wersja nr 782}

Rozwiązać nierówności $(x-7)(x-8)(x-14)\ge0$.
\zadStop
\rozwStart{Patryk Wirkus}{}
Miejsca zerowe naszego wielomianu to: $7, 8, 14$.\\
Wielomian jest stopnia nieparzystego, ponadto znak współczynnika przy\linebreak najwyższej potędze x jest dodatni.\\ W związku z tym wykres wielomianu zaczyna się od lewej strony poniżej osi OX. A więc $$x \in [7,8] \cup [14,\infty).$$
\rozwStop
\odpStart
$x \in [7,8] \cup [14,\infty)$
\odpStop
\testStart
A.$x \in [7,8] \cup [14,\infty)$\\
B.$x \in (7,8) \cup [14,\infty)$\\
C.$x \in (7,8] \cup [14,\infty)$\\
D.$x \in [7,8) \cup [14,\infty)$\\
E.$x \in [7,8] \cup (14,\infty)$\\
F.$x \in (7,8) \cup (14,\infty)$\\
G.$x \in [7,8) \cup (14,\infty)$\\
H.$x \in (7,8] \cup (14,\infty)$
\testStop
\kluczStart
A
\kluczStop



\zadStart{Zadanie z Wikieł Z 1.62 a) moja wersja nr 783}

Rozwiązać nierówności $(x-7)(x-8)(x-15)\ge0$.
\zadStop
\rozwStart{Patryk Wirkus}{}
Miejsca zerowe naszego wielomianu to: $7, 8, 15$.\\
Wielomian jest stopnia nieparzystego, ponadto znak współczynnika przy\linebreak najwyższej potędze x jest dodatni.\\ W związku z tym wykres wielomianu zaczyna się od lewej strony poniżej osi OX. A więc $$x \in [7,8] \cup [15,\infty).$$
\rozwStop
\odpStart
$x \in [7,8] \cup [15,\infty)$
\odpStop
\testStart
A.$x \in [7,8] \cup [15,\infty)$\\
B.$x \in (7,8) \cup [15,\infty)$\\
C.$x \in (7,8] \cup [15,\infty)$\\
D.$x \in [7,8) \cup [15,\infty)$\\
E.$x \in [7,8] \cup (15,\infty)$\\
F.$x \in (7,8) \cup (15,\infty)$\\
G.$x \in [7,8) \cup (15,\infty)$\\
H.$x \in (7,8] \cup (15,\infty)$
\testStop
\kluczStart
A
\kluczStop



\zadStart{Zadanie z Wikieł Z 1.62 a) moja wersja nr 784}

Rozwiązać nierówności $(x-7)(x-8)(x-16)\ge0$.
\zadStop
\rozwStart{Patryk Wirkus}{}
Miejsca zerowe naszego wielomianu to: $7, 8, 16$.\\
Wielomian jest stopnia nieparzystego, ponadto znak współczynnika przy\linebreak najwyższej potędze x jest dodatni.\\ W związku z tym wykres wielomianu zaczyna się od lewej strony poniżej osi OX. A więc $$x \in [7,8] \cup [16,\infty).$$
\rozwStop
\odpStart
$x \in [7,8] \cup [16,\infty)$
\odpStop
\testStart
A.$x \in [7,8] \cup [16,\infty)$\\
B.$x \in (7,8) \cup [16,\infty)$\\
C.$x \in (7,8] \cup [16,\infty)$\\
D.$x \in [7,8) \cup [16,\infty)$\\
E.$x \in [7,8] \cup (16,\infty)$\\
F.$x \in (7,8) \cup (16,\infty)$\\
G.$x \in [7,8) \cup (16,\infty)$\\
H.$x \in (7,8] \cup (16,\infty)$
\testStop
\kluczStart
A
\kluczStop



\zadStart{Zadanie z Wikieł Z 1.62 a) moja wersja nr 785}

Rozwiązać nierówności $(x-7)(x-8)(x-17)\ge0$.
\zadStop
\rozwStart{Patryk Wirkus}{}
Miejsca zerowe naszego wielomianu to: $7, 8, 17$.\\
Wielomian jest stopnia nieparzystego, ponadto znak współczynnika przy\linebreak najwyższej potędze x jest dodatni.\\ W związku z tym wykres wielomianu zaczyna się od lewej strony poniżej osi OX. A więc $$x \in [7,8] \cup [17,\infty).$$
\rozwStop
\odpStart
$x \in [7,8] \cup [17,\infty)$
\odpStop
\testStart
A.$x \in [7,8] \cup [17,\infty)$\\
B.$x \in (7,8) \cup [17,\infty)$\\
C.$x \in (7,8] \cup [17,\infty)$\\
D.$x \in [7,8) \cup [17,\infty)$\\
E.$x \in [7,8] \cup (17,\infty)$\\
F.$x \in (7,8) \cup (17,\infty)$\\
G.$x \in [7,8) \cup (17,\infty)$\\
H.$x \in (7,8] \cup (17,\infty)$
\testStop
\kluczStart
A
\kluczStop



\zadStart{Zadanie z Wikieł Z 1.62 a) moja wersja nr 786}

Rozwiązać nierówności $(x-7)(x-8)(x-18)\ge0$.
\zadStop
\rozwStart{Patryk Wirkus}{}
Miejsca zerowe naszego wielomianu to: $7, 8, 18$.\\
Wielomian jest stopnia nieparzystego, ponadto znak współczynnika przy\linebreak najwyższej potędze x jest dodatni.\\ W związku z tym wykres wielomianu zaczyna się od lewej strony poniżej osi OX. A więc $$x \in [7,8] \cup [18,\infty).$$
\rozwStop
\odpStart
$x \in [7,8] \cup [18,\infty)$
\odpStop
\testStart
A.$x \in [7,8] \cup [18,\infty)$\\
B.$x \in (7,8) \cup [18,\infty)$\\
C.$x \in (7,8] \cup [18,\infty)$\\
D.$x \in [7,8) \cup [18,\infty)$\\
E.$x \in [7,8] \cup (18,\infty)$\\
F.$x \in (7,8) \cup (18,\infty)$\\
G.$x \in [7,8) \cup (18,\infty)$\\
H.$x \in (7,8] \cup (18,\infty)$
\testStop
\kluczStart
A
\kluczStop



\zadStart{Zadanie z Wikieł Z 1.62 a) moja wersja nr 787}

Rozwiązać nierówności $(x-7)(x-8)(x-19)\ge0$.
\zadStop
\rozwStart{Patryk Wirkus}{}
Miejsca zerowe naszego wielomianu to: $7, 8, 19$.\\
Wielomian jest stopnia nieparzystego, ponadto znak współczynnika przy\linebreak najwyższej potędze x jest dodatni.\\ W związku z tym wykres wielomianu zaczyna się od lewej strony poniżej osi OX. A więc $$x \in [7,8] \cup [19,\infty).$$
\rozwStop
\odpStart
$x \in [7,8] \cup [19,\infty)$
\odpStop
\testStart
A.$x \in [7,8] \cup [19,\infty)$\\
B.$x \in (7,8) \cup [19,\infty)$\\
C.$x \in (7,8] \cup [19,\infty)$\\
D.$x \in [7,8) \cup [19,\infty)$\\
E.$x \in [7,8] \cup (19,\infty)$\\
F.$x \in (7,8) \cup (19,\infty)$\\
G.$x \in [7,8) \cup (19,\infty)$\\
H.$x \in (7,8] \cup (19,\infty)$
\testStop
\kluczStart
A
\kluczStop



\zadStart{Zadanie z Wikieł Z 1.62 a) moja wersja nr 788}

Rozwiązać nierówności $(x-7)(x-8)(x-20)\ge0$.
\zadStop
\rozwStart{Patryk Wirkus}{}
Miejsca zerowe naszego wielomianu to: $7, 8, 20$.\\
Wielomian jest stopnia nieparzystego, ponadto znak współczynnika przy\linebreak najwyższej potędze x jest dodatni.\\ W związku z tym wykres wielomianu zaczyna się od lewej strony poniżej osi OX. A więc $$x \in [7,8] \cup [20,\infty).$$
\rozwStop
\odpStart
$x \in [7,8] \cup [20,\infty)$
\odpStop
\testStart
A.$x \in [7,8] \cup [20,\infty)$\\
B.$x \in (7,8) \cup [20,\infty)$\\
C.$x \in (7,8] \cup [20,\infty)$\\
D.$x \in [7,8) \cup [20,\infty)$\\
E.$x \in [7,8] \cup (20,\infty)$\\
F.$x \in (7,8) \cup (20,\infty)$\\
G.$x \in [7,8) \cup (20,\infty)$\\
H.$x \in (7,8] \cup (20,\infty)$
\testStop
\kluczStart
A
\kluczStop



\zadStart{Zadanie z Wikieł Z 1.62 a) moja wersja nr 789}

Rozwiązać nierówności $(x-7)(x-9)(x-10)\ge0$.
\zadStop
\rozwStart{Patryk Wirkus}{}
Miejsca zerowe naszego wielomianu to: $7, 9, 10$.\\
Wielomian jest stopnia nieparzystego, ponadto znak współczynnika przy\linebreak najwyższej potędze x jest dodatni.\\ W związku z tym wykres wielomianu zaczyna się od lewej strony poniżej osi OX. A więc $$x \in [7,9] \cup [10,\infty).$$
\rozwStop
\odpStart
$x \in [7,9] \cup [10,\infty)$
\odpStop
\testStart
A.$x \in [7,9] \cup [10,\infty)$\\
B.$x \in (7,9) \cup [10,\infty)$\\
C.$x \in (7,9] \cup [10,\infty)$\\
D.$x \in [7,9) \cup [10,\infty)$\\
E.$x \in [7,9] \cup (10,\infty)$\\
F.$x \in (7,9) \cup (10,\infty)$\\
G.$x \in [7,9) \cup (10,\infty)$\\
H.$x \in (7,9] \cup (10,\infty)$
\testStop
\kluczStart
A
\kluczStop



\zadStart{Zadanie z Wikieł Z 1.62 a) moja wersja nr 790}

Rozwiązać nierówności $(x-7)(x-9)(x-11)\ge0$.
\zadStop
\rozwStart{Patryk Wirkus}{}
Miejsca zerowe naszego wielomianu to: $7, 9, 11$.\\
Wielomian jest stopnia nieparzystego, ponadto znak współczynnika przy\linebreak najwyższej potędze x jest dodatni.\\ W związku z tym wykres wielomianu zaczyna się od lewej strony poniżej osi OX. A więc $$x \in [7,9] \cup [11,\infty).$$
\rozwStop
\odpStart
$x \in [7,9] \cup [11,\infty)$
\odpStop
\testStart
A.$x \in [7,9] \cup [11,\infty)$\\
B.$x \in (7,9) \cup [11,\infty)$\\
C.$x \in (7,9] \cup [11,\infty)$\\
D.$x \in [7,9) \cup [11,\infty)$\\
E.$x \in [7,9] \cup (11,\infty)$\\
F.$x \in (7,9) \cup (11,\infty)$\\
G.$x \in [7,9) \cup (11,\infty)$\\
H.$x \in (7,9] \cup (11,\infty)$
\testStop
\kluczStart
A
\kluczStop



\zadStart{Zadanie z Wikieł Z 1.62 a) moja wersja nr 791}

Rozwiązać nierówności $(x-7)(x-9)(x-12)\ge0$.
\zadStop
\rozwStart{Patryk Wirkus}{}
Miejsca zerowe naszego wielomianu to: $7, 9, 12$.\\
Wielomian jest stopnia nieparzystego, ponadto znak współczynnika przy\linebreak najwyższej potędze x jest dodatni.\\ W związku z tym wykres wielomianu zaczyna się od lewej strony poniżej osi OX. A więc $$x \in [7,9] \cup [12,\infty).$$
\rozwStop
\odpStart
$x \in [7,9] \cup [12,\infty)$
\odpStop
\testStart
A.$x \in [7,9] \cup [12,\infty)$\\
B.$x \in (7,9) \cup [12,\infty)$\\
C.$x \in (7,9] \cup [12,\infty)$\\
D.$x \in [7,9) \cup [12,\infty)$\\
E.$x \in [7,9] \cup (12,\infty)$\\
F.$x \in (7,9) \cup (12,\infty)$\\
G.$x \in [7,9) \cup (12,\infty)$\\
H.$x \in (7,9] \cup (12,\infty)$
\testStop
\kluczStart
A
\kluczStop



\zadStart{Zadanie z Wikieł Z 1.62 a) moja wersja nr 792}

Rozwiązać nierówności $(x-7)(x-9)(x-13)\ge0$.
\zadStop
\rozwStart{Patryk Wirkus}{}
Miejsca zerowe naszego wielomianu to: $7, 9, 13$.\\
Wielomian jest stopnia nieparzystego, ponadto znak współczynnika przy\linebreak najwyższej potędze x jest dodatni.\\ W związku z tym wykres wielomianu zaczyna się od lewej strony poniżej osi OX. A więc $$x \in [7,9] \cup [13,\infty).$$
\rozwStop
\odpStart
$x \in [7,9] \cup [13,\infty)$
\odpStop
\testStart
A.$x \in [7,9] \cup [13,\infty)$\\
B.$x \in (7,9) \cup [13,\infty)$\\
C.$x \in (7,9] \cup [13,\infty)$\\
D.$x \in [7,9) \cup [13,\infty)$\\
E.$x \in [7,9] \cup (13,\infty)$\\
F.$x \in (7,9) \cup (13,\infty)$\\
G.$x \in [7,9) \cup (13,\infty)$\\
H.$x \in (7,9] \cup (13,\infty)$
\testStop
\kluczStart
A
\kluczStop



\zadStart{Zadanie z Wikieł Z 1.62 a) moja wersja nr 793}

Rozwiązać nierówności $(x-7)(x-9)(x-14)\ge0$.
\zadStop
\rozwStart{Patryk Wirkus}{}
Miejsca zerowe naszego wielomianu to: $7, 9, 14$.\\
Wielomian jest stopnia nieparzystego, ponadto znak współczynnika przy\linebreak najwyższej potędze x jest dodatni.\\ W związku z tym wykres wielomianu zaczyna się od lewej strony poniżej osi OX. A więc $$x \in [7,9] \cup [14,\infty).$$
\rozwStop
\odpStart
$x \in [7,9] \cup [14,\infty)$
\odpStop
\testStart
A.$x \in [7,9] \cup [14,\infty)$\\
B.$x \in (7,9) \cup [14,\infty)$\\
C.$x \in (7,9] \cup [14,\infty)$\\
D.$x \in [7,9) \cup [14,\infty)$\\
E.$x \in [7,9] \cup (14,\infty)$\\
F.$x \in (7,9) \cup (14,\infty)$\\
G.$x \in [7,9) \cup (14,\infty)$\\
H.$x \in (7,9] \cup (14,\infty)$
\testStop
\kluczStart
A
\kluczStop



\zadStart{Zadanie z Wikieł Z 1.62 a) moja wersja nr 794}

Rozwiązać nierówności $(x-7)(x-9)(x-15)\ge0$.
\zadStop
\rozwStart{Patryk Wirkus}{}
Miejsca zerowe naszego wielomianu to: $7, 9, 15$.\\
Wielomian jest stopnia nieparzystego, ponadto znak współczynnika przy\linebreak najwyższej potędze x jest dodatni.\\ W związku z tym wykres wielomianu zaczyna się od lewej strony poniżej osi OX. A więc $$x \in [7,9] \cup [15,\infty).$$
\rozwStop
\odpStart
$x \in [7,9] \cup [15,\infty)$
\odpStop
\testStart
A.$x \in [7,9] \cup [15,\infty)$\\
B.$x \in (7,9) \cup [15,\infty)$\\
C.$x \in (7,9] \cup [15,\infty)$\\
D.$x \in [7,9) \cup [15,\infty)$\\
E.$x \in [7,9] \cup (15,\infty)$\\
F.$x \in (7,9) \cup (15,\infty)$\\
G.$x \in [7,9) \cup (15,\infty)$\\
H.$x \in (7,9] \cup (15,\infty)$
\testStop
\kluczStart
A
\kluczStop



\zadStart{Zadanie z Wikieł Z 1.62 a) moja wersja nr 795}

Rozwiązać nierówności $(x-7)(x-9)(x-16)\ge0$.
\zadStop
\rozwStart{Patryk Wirkus}{}
Miejsca zerowe naszego wielomianu to: $7, 9, 16$.\\
Wielomian jest stopnia nieparzystego, ponadto znak współczynnika przy\linebreak najwyższej potędze x jest dodatni.\\ W związku z tym wykres wielomianu zaczyna się od lewej strony poniżej osi OX. A więc $$x \in [7,9] \cup [16,\infty).$$
\rozwStop
\odpStart
$x \in [7,9] \cup [16,\infty)$
\odpStop
\testStart
A.$x \in [7,9] \cup [16,\infty)$\\
B.$x \in (7,9) \cup [16,\infty)$\\
C.$x \in (7,9] \cup [16,\infty)$\\
D.$x \in [7,9) \cup [16,\infty)$\\
E.$x \in [7,9] \cup (16,\infty)$\\
F.$x \in (7,9) \cup (16,\infty)$\\
G.$x \in [7,9) \cup (16,\infty)$\\
H.$x \in (7,9] \cup (16,\infty)$
\testStop
\kluczStart
A
\kluczStop



\zadStart{Zadanie z Wikieł Z 1.62 a) moja wersja nr 796}

Rozwiązać nierówności $(x-7)(x-9)(x-17)\ge0$.
\zadStop
\rozwStart{Patryk Wirkus}{}
Miejsca zerowe naszego wielomianu to: $7, 9, 17$.\\
Wielomian jest stopnia nieparzystego, ponadto znak współczynnika przy\linebreak najwyższej potędze x jest dodatni.\\ W związku z tym wykres wielomianu zaczyna się od lewej strony poniżej osi OX. A więc $$x \in [7,9] \cup [17,\infty).$$
\rozwStop
\odpStart
$x \in [7,9] \cup [17,\infty)$
\odpStop
\testStart
A.$x \in [7,9] \cup [17,\infty)$\\
B.$x \in (7,9) \cup [17,\infty)$\\
C.$x \in (7,9] \cup [17,\infty)$\\
D.$x \in [7,9) \cup [17,\infty)$\\
E.$x \in [7,9] \cup (17,\infty)$\\
F.$x \in (7,9) \cup (17,\infty)$\\
G.$x \in [7,9) \cup (17,\infty)$\\
H.$x \in (7,9] \cup (17,\infty)$
\testStop
\kluczStart
A
\kluczStop



\zadStart{Zadanie z Wikieł Z 1.62 a) moja wersja nr 797}

Rozwiązać nierówności $(x-7)(x-9)(x-18)\ge0$.
\zadStop
\rozwStart{Patryk Wirkus}{}
Miejsca zerowe naszego wielomianu to: $7, 9, 18$.\\
Wielomian jest stopnia nieparzystego, ponadto znak współczynnika przy\linebreak najwyższej potędze x jest dodatni.\\ W związku z tym wykres wielomianu zaczyna się od lewej strony poniżej osi OX. A więc $$x \in [7,9] \cup [18,\infty).$$
\rozwStop
\odpStart
$x \in [7,9] \cup [18,\infty)$
\odpStop
\testStart
A.$x \in [7,9] \cup [18,\infty)$\\
B.$x \in (7,9) \cup [18,\infty)$\\
C.$x \in (7,9] \cup [18,\infty)$\\
D.$x \in [7,9) \cup [18,\infty)$\\
E.$x \in [7,9] \cup (18,\infty)$\\
F.$x \in (7,9) \cup (18,\infty)$\\
G.$x \in [7,9) \cup (18,\infty)$\\
H.$x \in (7,9] \cup (18,\infty)$
\testStop
\kluczStart
A
\kluczStop



\zadStart{Zadanie z Wikieł Z 1.62 a) moja wersja nr 798}

Rozwiązać nierówności $(x-7)(x-9)(x-19)\ge0$.
\zadStop
\rozwStart{Patryk Wirkus}{}
Miejsca zerowe naszego wielomianu to: $7, 9, 19$.\\
Wielomian jest stopnia nieparzystego, ponadto znak współczynnika przy\linebreak najwyższej potędze x jest dodatni.\\ W związku z tym wykres wielomianu zaczyna się od lewej strony poniżej osi OX. A więc $$x \in [7,9] \cup [19,\infty).$$
\rozwStop
\odpStart
$x \in [7,9] \cup [19,\infty)$
\odpStop
\testStart
A.$x \in [7,9] \cup [19,\infty)$\\
B.$x \in (7,9) \cup [19,\infty)$\\
C.$x \in (7,9] \cup [19,\infty)$\\
D.$x \in [7,9) \cup [19,\infty)$\\
E.$x \in [7,9] \cup (19,\infty)$\\
F.$x \in (7,9) \cup (19,\infty)$\\
G.$x \in [7,9) \cup (19,\infty)$\\
H.$x \in (7,9] \cup (19,\infty)$
\testStop
\kluczStart
A
\kluczStop



\zadStart{Zadanie z Wikieł Z 1.62 a) moja wersja nr 799}

Rozwiązać nierówności $(x-7)(x-9)(x-20)\ge0$.
\zadStop
\rozwStart{Patryk Wirkus}{}
Miejsca zerowe naszego wielomianu to: $7, 9, 20$.\\
Wielomian jest stopnia nieparzystego, ponadto znak współczynnika przy\linebreak najwyższej potędze x jest dodatni.\\ W związku z tym wykres wielomianu zaczyna się od lewej strony poniżej osi OX. A więc $$x \in [7,9] \cup [20,\infty).$$
\rozwStop
\odpStart
$x \in [7,9] \cup [20,\infty)$
\odpStop
\testStart
A.$x \in [7,9] \cup [20,\infty)$\\
B.$x \in (7,9) \cup [20,\infty)$\\
C.$x \in (7,9] \cup [20,\infty)$\\
D.$x \in [7,9) \cup [20,\infty)$\\
E.$x \in [7,9] \cup (20,\infty)$\\
F.$x \in (7,9) \cup (20,\infty)$\\
G.$x \in [7,9) \cup (20,\infty)$\\
H.$x \in (7,9] \cup (20,\infty)$
\testStop
\kluczStart
A
\kluczStop



\zadStart{Zadanie z Wikieł Z 1.62 a) moja wersja nr 800}

Rozwiązać nierówności $(x-7)(x-10)(x-11)\ge0$.
\zadStop
\rozwStart{Patryk Wirkus}{}
Miejsca zerowe naszego wielomianu to: $7, 10, 11$.\\
Wielomian jest stopnia nieparzystego, ponadto znak współczynnika przy\linebreak najwyższej potędze x jest dodatni.\\ W związku z tym wykres wielomianu zaczyna się od lewej strony poniżej osi OX. A więc $$x \in [7,10] \cup [11,\infty).$$
\rozwStop
\odpStart
$x \in [7,10] \cup [11,\infty)$
\odpStop
\testStart
A.$x \in [7,10] \cup [11,\infty)$\\
B.$x \in (7,10) \cup [11,\infty)$\\
C.$x \in (7,10] \cup [11,\infty)$\\
D.$x \in [7,10) \cup [11,\infty)$\\
E.$x \in [7,10] \cup (11,\infty)$\\
F.$x \in (7,10) \cup (11,\infty)$\\
G.$x \in [7,10) \cup (11,\infty)$\\
H.$x \in (7,10] \cup (11,\infty)$
\testStop
\kluczStart
A
\kluczStop



\zadStart{Zadanie z Wikieł Z 1.62 a) moja wersja nr 801}

Rozwiązać nierówności $(x-7)(x-10)(x-12)\ge0$.
\zadStop
\rozwStart{Patryk Wirkus}{}
Miejsca zerowe naszego wielomianu to: $7, 10, 12$.\\
Wielomian jest stopnia nieparzystego, ponadto znak współczynnika przy\linebreak najwyższej potędze x jest dodatni.\\ W związku z tym wykres wielomianu zaczyna się od lewej strony poniżej osi OX. A więc $$x \in [7,10] \cup [12,\infty).$$
\rozwStop
\odpStart
$x \in [7,10] \cup [12,\infty)$
\odpStop
\testStart
A.$x \in [7,10] \cup [12,\infty)$\\
B.$x \in (7,10) \cup [12,\infty)$\\
C.$x \in (7,10] \cup [12,\infty)$\\
D.$x \in [7,10) \cup [12,\infty)$\\
E.$x \in [7,10] \cup (12,\infty)$\\
F.$x \in (7,10) \cup (12,\infty)$\\
G.$x \in [7,10) \cup (12,\infty)$\\
H.$x \in (7,10] \cup (12,\infty)$
\testStop
\kluczStart
A
\kluczStop



\zadStart{Zadanie z Wikieł Z 1.62 a) moja wersja nr 802}

Rozwiązać nierówności $(x-7)(x-10)(x-13)\ge0$.
\zadStop
\rozwStart{Patryk Wirkus}{}
Miejsca zerowe naszego wielomianu to: $7, 10, 13$.\\
Wielomian jest stopnia nieparzystego, ponadto znak współczynnika przy\linebreak najwyższej potędze x jest dodatni.\\ W związku z tym wykres wielomianu zaczyna się od lewej strony poniżej osi OX. A więc $$x \in [7,10] \cup [13,\infty).$$
\rozwStop
\odpStart
$x \in [7,10] \cup [13,\infty)$
\odpStop
\testStart
A.$x \in [7,10] \cup [13,\infty)$\\
B.$x \in (7,10) \cup [13,\infty)$\\
C.$x \in (7,10] \cup [13,\infty)$\\
D.$x \in [7,10) \cup [13,\infty)$\\
E.$x \in [7,10] \cup (13,\infty)$\\
F.$x \in (7,10) \cup (13,\infty)$\\
G.$x \in [7,10) \cup (13,\infty)$\\
H.$x \in (7,10] \cup (13,\infty)$
\testStop
\kluczStart
A
\kluczStop



\zadStart{Zadanie z Wikieł Z 1.62 a) moja wersja nr 803}

Rozwiązać nierówności $(x-7)(x-10)(x-14)\ge0$.
\zadStop
\rozwStart{Patryk Wirkus}{}
Miejsca zerowe naszego wielomianu to: $7, 10, 14$.\\
Wielomian jest stopnia nieparzystego, ponadto znak współczynnika przy\linebreak najwyższej potędze x jest dodatni.\\ W związku z tym wykres wielomianu zaczyna się od lewej strony poniżej osi OX. A więc $$x \in [7,10] \cup [14,\infty).$$
\rozwStop
\odpStart
$x \in [7,10] \cup [14,\infty)$
\odpStop
\testStart
A.$x \in [7,10] \cup [14,\infty)$\\
B.$x \in (7,10) \cup [14,\infty)$\\
C.$x \in (7,10] \cup [14,\infty)$\\
D.$x \in [7,10) \cup [14,\infty)$\\
E.$x \in [7,10] \cup (14,\infty)$\\
F.$x \in (7,10) \cup (14,\infty)$\\
G.$x \in [7,10) \cup (14,\infty)$\\
H.$x \in (7,10] \cup (14,\infty)$
\testStop
\kluczStart
A
\kluczStop



\zadStart{Zadanie z Wikieł Z 1.62 a) moja wersja nr 804}

Rozwiązać nierówności $(x-7)(x-10)(x-15)\ge0$.
\zadStop
\rozwStart{Patryk Wirkus}{}
Miejsca zerowe naszego wielomianu to: $7, 10, 15$.\\
Wielomian jest stopnia nieparzystego, ponadto znak współczynnika przy\linebreak najwyższej potędze x jest dodatni.\\ W związku z tym wykres wielomianu zaczyna się od lewej strony poniżej osi OX. A więc $$x \in [7,10] \cup [15,\infty).$$
\rozwStop
\odpStart
$x \in [7,10] \cup [15,\infty)$
\odpStop
\testStart
A.$x \in [7,10] \cup [15,\infty)$\\
B.$x \in (7,10) \cup [15,\infty)$\\
C.$x \in (7,10] \cup [15,\infty)$\\
D.$x \in [7,10) \cup [15,\infty)$\\
E.$x \in [7,10] \cup (15,\infty)$\\
F.$x \in (7,10) \cup (15,\infty)$\\
G.$x \in [7,10) \cup (15,\infty)$\\
H.$x \in (7,10] \cup (15,\infty)$
\testStop
\kluczStart
A
\kluczStop



\zadStart{Zadanie z Wikieł Z 1.62 a) moja wersja nr 805}

Rozwiązać nierówności $(x-7)(x-10)(x-16)\ge0$.
\zadStop
\rozwStart{Patryk Wirkus}{}
Miejsca zerowe naszego wielomianu to: $7, 10, 16$.\\
Wielomian jest stopnia nieparzystego, ponadto znak współczynnika przy\linebreak najwyższej potędze x jest dodatni.\\ W związku z tym wykres wielomianu zaczyna się od lewej strony poniżej osi OX. A więc $$x \in [7,10] \cup [16,\infty).$$
\rozwStop
\odpStart
$x \in [7,10] \cup [16,\infty)$
\odpStop
\testStart
A.$x \in [7,10] \cup [16,\infty)$\\
B.$x \in (7,10) \cup [16,\infty)$\\
C.$x \in (7,10] \cup [16,\infty)$\\
D.$x \in [7,10) \cup [16,\infty)$\\
E.$x \in [7,10] \cup (16,\infty)$\\
F.$x \in (7,10) \cup (16,\infty)$\\
G.$x \in [7,10) \cup (16,\infty)$\\
H.$x \in (7,10] \cup (16,\infty)$
\testStop
\kluczStart
A
\kluczStop



\zadStart{Zadanie z Wikieł Z 1.62 a) moja wersja nr 806}

Rozwiązać nierówności $(x-7)(x-10)(x-17)\ge0$.
\zadStop
\rozwStart{Patryk Wirkus}{}
Miejsca zerowe naszego wielomianu to: $7, 10, 17$.\\
Wielomian jest stopnia nieparzystego, ponadto znak współczynnika przy\linebreak najwyższej potędze x jest dodatni.\\ W związku z tym wykres wielomianu zaczyna się od lewej strony poniżej osi OX. A więc $$x \in [7,10] \cup [17,\infty).$$
\rozwStop
\odpStart
$x \in [7,10] \cup [17,\infty)$
\odpStop
\testStart
A.$x \in [7,10] \cup [17,\infty)$\\
B.$x \in (7,10) \cup [17,\infty)$\\
C.$x \in (7,10] \cup [17,\infty)$\\
D.$x \in [7,10) \cup [17,\infty)$\\
E.$x \in [7,10] \cup (17,\infty)$\\
F.$x \in (7,10) \cup (17,\infty)$\\
G.$x \in [7,10) \cup (17,\infty)$\\
H.$x \in (7,10] \cup (17,\infty)$
\testStop
\kluczStart
A
\kluczStop



\zadStart{Zadanie z Wikieł Z 1.62 a) moja wersja nr 807}

Rozwiązać nierówności $(x-7)(x-10)(x-18)\ge0$.
\zadStop
\rozwStart{Patryk Wirkus}{}
Miejsca zerowe naszego wielomianu to: $7, 10, 18$.\\
Wielomian jest stopnia nieparzystego, ponadto znak współczynnika przy\linebreak najwyższej potędze x jest dodatni.\\ W związku z tym wykres wielomianu zaczyna się od lewej strony poniżej osi OX. A więc $$x \in [7,10] \cup [18,\infty).$$
\rozwStop
\odpStart
$x \in [7,10] \cup [18,\infty)$
\odpStop
\testStart
A.$x \in [7,10] \cup [18,\infty)$\\
B.$x \in (7,10) \cup [18,\infty)$\\
C.$x \in (7,10] \cup [18,\infty)$\\
D.$x \in [7,10) \cup [18,\infty)$\\
E.$x \in [7,10] \cup (18,\infty)$\\
F.$x \in (7,10) \cup (18,\infty)$\\
G.$x \in [7,10) \cup (18,\infty)$\\
H.$x \in (7,10] \cup (18,\infty)$
\testStop
\kluczStart
A
\kluczStop



\zadStart{Zadanie z Wikieł Z 1.62 a) moja wersja nr 808}

Rozwiązać nierówności $(x-7)(x-10)(x-19)\ge0$.
\zadStop
\rozwStart{Patryk Wirkus}{}
Miejsca zerowe naszego wielomianu to: $7, 10, 19$.\\
Wielomian jest stopnia nieparzystego, ponadto znak współczynnika przy\linebreak najwyższej potędze x jest dodatni.\\ W związku z tym wykres wielomianu zaczyna się od lewej strony poniżej osi OX. A więc $$x \in [7,10] \cup [19,\infty).$$
\rozwStop
\odpStart
$x \in [7,10] \cup [19,\infty)$
\odpStop
\testStart
A.$x \in [7,10] \cup [19,\infty)$\\
B.$x \in (7,10) \cup [19,\infty)$\\
C.$x \in (7,10] \cup [19,\infty)$\\
D.$x \in [7,10) \cup [19,\infty)$\\
E.$x \in [7,10] \cup (19,\infty)$\\
F.$x \in (7,10) \cup (19,\infty)$\\
G.$x \in [7,10) \cup (19,\infty)$\\
H.$x \in (7,10] \cup (19,\infty)$
\testStop
\kluczStart
A
\kluczStop



\zadStart{Zadanie z Wikieł Z 1.62 a) moja wersja nr 809}

Rozwiązać nierówności $(x-7)(x-10)(x-20)\ge0$.
\zadStop
\rozwStart{Patryk Wirkus}{}
Miejsca zerowe naszego wielomianu to: $7, 10, 20$.\\
Wielomian jest stopnia nieparzystego, ponadto znak współczynnika przy\linebreak najwyższej potędze x jest dodatni.\\ W związku z tym wykres wielomianu zaczyna się od lewej strony poniżej osi OX. A więc $$x \in [7,10] \cup [20,\infty).$$
\rozwStop
\odpStart
$x \in [7,10] \cup [20,\infty)$
\odpStop
\testStart
A.$x \in [7,10] \cup [20,\infty)$\\
B.$x \in (7,10) \cup [20,\infty)$\\
C.$x \in (7,10] \cup [20,\infty)$\\
D.$x \in [7,10) \cup [20,\infty)$\\
E.$x \in [7,10] \cup (20,\infty)$\\
F.$x \in (7,10) \cup (20,\infty)$\\
G.$x \in [7,10) \cup (20,\infty)$\\
H.$x \in (7,10] \cup (20,\infty)$
\testStop
\kluczStart
A
\kluczStop



\zadStart{Zadanie z Wikieł Z 1.62 a) moja wersja nr 810}

Rozwiązać nierówności $(x-7)(x-11)(x-12)\ge0$.
\zadStop
\rozwStart{Patryk Wirkus}{}
Miejsca zerowe naszego wielomianu to: $7, 11, 12$.\\
Wielomian jest stopnia nieparzystego, ponadto znak współczynnika przy\linebreak najwyższej potędze x jest dodatni.\\ W związku z tym wykres wielomianu zaczyna się od lewej strony poniżej osi OX. A więc $$x \in [7,11] \cup [12,\infty).$$
\rozwStop
\odpStart
$x \in [7,11] \cup [12,\infty)$
\odpStop
\testStart
A.$x \in [7,11] \cup [12,\infty)$\\
B.$x \in (7,11) \cup [12,\infty)$\\
C.$x \in (7,11] \cup [12,\infty)$\\
D.$x \in [7,11) \cup [12,\infty)$\\
E.$x \in [7,11] \cup (12,\infty)$\\
F.$x \in (7,11) \cup (12,\infty)$\\
G.$x \in [7,11) \cup (12,\infty)$\\
H.$x \in (7,11] \cup (12,\infty)$
\testStop
\kluczStart
A
\kluczStop



\zadStart{Zadanie z Wikieł Z 1.62 a) moja wersja nr 811}

Rozwiązać nierówności $(x-7)(x-11)(x-13)\ge0$.
\zadStop
\rozwStart{Patryk Wirkus}{}
Miejsca zerowe naszego wielomianu to: $7, 11, 13$.\\
Wielomian jest stopnia nieparzystego, ponadto znak współczynnika przy\linebreak najwyższej potędze x jest dodatni.\\ W związku z tym wykres wielomianu zaczyna się od lewej strony poniżej osi OX. A więc $$x \in [7,11] \cup [13,\infty).$$
\rozwStop
\odpStart
$x \in [7,11] \cup [13,\infty)$
\odpStop
\testStart
A.$x \in [7,11] \cup [13,\infty)$\\
B.$x \in (7,11) \cup [13,\infty)$\\
C.$x \in (7,11] \cup [13,\infty)$\\
D.$x \in [7,11) \cup [13,\infty)$\\
E.$x \in [7,11] \cup (13,\infty)$\\
F.$x \in (7,11) \cup (13,\infty)$\\
G.$x \in [7,11) \cup (13,\infty)$\\
H.$x \in (7,11] \cup (13,\infty)$
\testStop
\kluczStart
A
\kluczStop



\zadStart{Zadanie z Wikieł Z 1.62 a) moja wersja nr 812}

Rozwiązać nierówności $(x-7)(x-11)(x-14)\ge0$.
\zadStop
\rozwStart{Patryk Wirkus}{}
Miejsca zerowe naszego wielomianu to: $7, 11, 14$.\\
Wielomian jest stopnia nieparzystego, ponadto znak współczynnika przy\linebreak najwyższej potędze x jest dodatni.\\ W związku z tym wykres wielomianu zaczyna się od lewej strony poniżej osi OX. A więc $$x \in [7,11] \cup [14,\infty).$$
\rozwStop
\odpStart
$x \in [7,11] \cup [14,\infty)$
\odpStop
\testStart
A.$x \in [7,11] \cup [14,\infty)$\\
B.$x \in (7,11) \cup [14,\infty)$\\
C.$x \in (7,11] \cup [14,\infty)$\\
D.$x \in [7,11) \cup [14,\infty)$\\
E.$x \in [7,11] \cup (14,\infty)$\\
F.$x \in (7,11) \cup (14,\infty)$\\
G.$x \in [7,11) \cup (14,\infty)$\\
H.$x \in (7,11] \cup (14,\infty)$
\testStop
\kluczStart
A
\kluczStop



\zadStart{Zadanie z Wikieł Z 1.62 a) moja wersja nr 813}

Rozwiązać nierówności $(x-7)(x-11)(x-15)\ge0$.
\zadStop
\rozwStart{Patryk Wirkus}{}
Miejsca zerowe naszego wielomianu to: $7, 11, 15$.\\
Wielomian jest stopnia nieparzystego, ponadto znak współczynnika przy\linebreak najwyższej potędze x jest dodatni.\\ W związku z tym wykres wielomianu zaczyna się od lewej strony poniżej osi OX. A więc $$x \in [7,11] \cup [15,\infty).$$
\rozwStop
\odpStart
$x \in [7,11] \cup [15,\infty)$
\odpStop
\testStart
A.$x \in [7,11] \cup [15,\infty)$\\
B.$x \in (7,11) \cup [15,\infty)$\\
C.$x \in (7,11] \cup [15,\infty)$\\
D.$x \in [7,11) \cup [15,\infty)$\\
E.$x \in [7,11] \cup (15,\infty)$\\
F.$x \in (7,11) \cup (15,\infty)$\\
G.$x \in [7,11) \cup (15,\infty)$\\
H.$x \in (7,11] \cup (15,\infty)$
\testStop
\kluczStart
A
\kluczStop



\zadStart{Zadanie z Wikieł Z 1.62 a) moja wersja nr 814}

Rozwiązać nierówności $(x-7)(x-11)(x-16)\ge0$.
\zadStop
\rozwStart{Patryk Wirkus}{}
Miejsca zerowe naszego wielomianu to: $7, 11, 16$.\\
Wielomian jest stopnia nieparzystego, ponadto znak współczynnika przy\linebreak najwyższej potędze x jest dodatni.\\ W związku z tym wykres wielomianu zaczyna się od lewej strony poniżej osi OX. A więc $$x \in [7,11] \cup [16,\infty).$$
\rozwStop
\odpStart
$x \in [7,11] \cup [16,\infty)$
\odpStop
\testStart
A.$x \in [7,11] \cup [16,\infty)$\\
B.$x \in (7,11) \cup [16,\infty)$\\
C.$x \in (7,11] \cup [16,\infty)$\\
D.$x \in [7,11) \cup [16,\infty)$\\
E.$x \in [7,11] \cup (16,\infty)$\\
F.$x \in (7,11) \cup (16,\infty)$\\
G.$x \in [7,11) \cup (16,\infty)$\\
H.$x \in (7,11] \cup (16,\infty)$
\testStop
\kluczStart
A
\kluczStop



\zadStart{Zadanie z Wikieł Z 1.62 a) moja wersja nr 815}

Rozwiązać nierówności $(x-7)(x-11)(x-17)\ge0$.
\zadStop
\rozwStart{Patryk Wirkus}{}
Miejsca zerowe naszego wielomianu to: $7, 11, 17$.\\
Wielomian jest stopnia nieparzystego, ponadto znak współczynnika przy\linebreak najwyższej potędze x jest dodatni.\\ W związku z tym wykres wielomianu zaczyna się od lewej strony poniżej osi OX. A więc $$x \in [7,11] \cup [17,\infty).$$
\rozwStop
\odpStart
$x \in [7,11] \cup [17,\infty)$
\odpStop
\testStart
A.$x \in [7,11] \cup [17,\infty)$\\
B.$x \in (7,11) \cup [17,\infty)$\\
C.$x \in (7,11] \cup [17,\infty)$\\
D.$x \in [7,11) \cup [17,\infty)$\\
E.$x \in [7,11] \cup (17,\infty)$\\
F.$x \in (7,11) \cup (17,\infty)$\\
G.$x \in [7,11) \cup (17,\infty)$\\
H.$x \in (7,11] \cup (17,\infty)$
\testStop
\kluczStart
A
\kluczStop



\zadStart{Zadanie z Wikieł Z 1.62 a) moja wersja nr 816}

Rozwiązać nierówności $(x-7)(x-11)(x-18)\ge0$.
\zadStop
\rozwStart{Patryk Wirkus}{}
Miejsca zerowe naszego wielomianu to: $7, 11, 18$.\\
Wielomian jest stopnia nieparzystego, ponadto znak współczynnika przy\linebreak najwyższej potędze x jest dodatni.\\ W związku z tym wykres wielomianu zaczyna się od lewej strony poniżej osi OX. A więc $$x \in [7,11] \cup [18,\infty).$$
\rozwStop
\odpStart
$x \in [7,11] \cup [18,\infty)$
\odpStop
\testStart
A.$x \in [7,11] \cup [18,\infty)$\\
B.$x \in (7,11) \cup [18,\infty)$\\
C.$x \in (7,11] \cup [18,\infty)$\\
D.$x \in [7,11) \cup [18,\infty)$\\
E.$x \in [7,11] \cup (18,\infty)$\\
F.$x \in (7,11) \cup (18,\infty)$\\
G.$x \in [7,11) \cup (18,\infty)$\\
H.$x \in (7,11] \cup (18,\infty)$
\testStop
\kluczStart
A
\kluczStop



\zadStart{Zadanie z Wikieł Z 1.62 a) moja wersja nr 817}

Rozwiązać nierówności $(x-7)(x-11)(x-19)\ge0$.
\zadStop
\rozwStart{Patryk Wirkus}{}
Miejsca zerowe naszego wielomianu to: $7, 11, 19$.\\
Wielomian jest stopnia nieparzystego, ponadto znak współczynnika przy\linebreak najwyższej potędze x jest dodatni.\\ W związku z tym wykres wielomianu zaczyna się od lewej strony poniżej osi OX. A więc $$x \in [7,11] \cup [19,\infty).$$
\rozwStop
\odpStart
$x \in [7,11] \cup [19,\infty)$
\odpStop
\testStart
A.$x \in [7,11] \cup [19,\infty)$\\
B.$x \in (7,11) \cup [19,\infty)$\\
C.$x \in (7,11] \cup [19,\infty)$\\
D.$x \in [7,11) \cup [19,\infty)$\\
E.$x \in [7,11] \cup (19,\infty)$\\
F.$x \in (7,11) \cup (19,\infty)$\\
G.$x \in [7,11) \cup (19,\infty)$\\
H.$x \in (7,11] \cup (19,\infty)$
\testStop
\kluczStart
A
\kluczStop



\zadStart{Zadanie z Wikieł Z 1.62 a) moja wersja nr 818}

Rozwiązać nierówności $(x-7)(x-11)(x-20)\ge0$.
\zadStop
\rozwStart{Patryk Wirkus}{}
Miejsca zerowe naszego wielomianu to: $7, 11, 20$.\\
Wielomian jest stopnia nieparzystego, ponadto znak współczynnika przy\linebreak najwyższej potędze x jest dodatni.\\ W związku z tym wykres wielomianu zaczyna się od lewej strony poniżej osi OX. A więc $$x \in [7,11] \cup [20,\infty).$$
\rozwStop
\odpStart
$x \in [7,11] \cup [20,\infty)$
\odpStop
\testStart
A.$x \in [7,11] \cup [20,\infty)$\\
B.$x \in (7,11) \cup [20,\infty)$\\
C.$x \in (7,11] \cup [20,\infty)$\\
D.$x \in [7,11) \cup [20,\infty)$\\
E.$x \in [7,11] \cup (20,\infty)$\\
F.$x \in (7,11) \cup (20,\infty)$\\
G.$x \in [7,11) \cup (20,\infty)$\\
H.$x \in (7,11] \cup (20,\infty)$
\testStop
\kluczStart
A
\kluczStop



\zadStart{Zadanie z Wikieł Z 1.62 a) moja wersja nr 819}

Rozwiązać nierówności $(x-7)(x-12)(x-13)\ge0$.
\zadStop
\rozwStart{Patryk Wirkus}{}
Miejsca zerowe naszego wielomianu to: $7, 12, 13$.\\
Wielomian jest stopnia nieparzystego, ponadto znak współczynnika przy\linebreak najwyższej potędze x jest dodatni.\\ W związku z tym wykres wielomianu zaczyna się od lewej strony poniżej osi OX. A więc $$x \in [7,12] \cup [13,\infty).$$
\rozwStop
\odpStart
$x \in [7,12] \cup [13,\infty)$
\odpStop
\testStart
A.$x \in [7,12] \cup [13,\infty)$\\
B.$x \in (7,12) \cup [13,\infty)$\\
C.$x \in (7,12] \cup [13,\infty)$\\
D.$x \in [7,12) \cup [13,\infty)$\\
E.$x \in [7,12] \cup (13,\infty)$\\
F.$x \in (7,12) \cup (13,\infty)$\\
G.$x \in [7,12) \cup (13,\infty)$\\
H.$x \in (7,12] \cup (13,\infty)$
\testStop
\kluczStart
A
\kluczStop



\zadStart{Zadanie z Wikieł Z 1.62 a) moja wersja nr 820}

Rozwiązać nierówności $(x-7)(x-12)(x-14)\ge0$.
\zadStop
\rozwStart{Patryk Wirkus}{}
Miejsca zerowe naszego wielomianu to: $7, 12, 14$.\\
Wielomian jest stopnia nieparzystego, ponadto znak współczynnika przy\linebreak najwyższej potędze x jest dodatni.\\ W związku z tym wykres wielomianu zaczyna się od lewej strony poniżej osi OX. A więc $$x \in [7,12] \cup [14,\infty).$$
\rozwStop
\odpStart
$x \in [7,12] \cup [14,\infty)$
\odpStop
\testStart
A.$x \in [7,12] \cup [14,\infty)$\\
B.$x \in (7,12) \cup [14,\infty)$\\
C.$x \in (7,12] \cup [14,\infty)$\\
D.$x \in [7,12) \cup [14,\infty)$\\
E.$x \in [7,12] \cup (14,\infty)$\\
F.$x \in (7,12) \cup (14,\infty)$\\
G.$x \in [7,12) \cup (14,\infty)$\\
H.$x \in (7,12] \cup (14,\infty)$
\testStop
\kluczStart
A
\kluczStop



\zadStart{Zadanie z Wikieł Z 1.62 a) moja wersja nr 821}

Rozwiązać nierówności $(x-7)(x-12)(x-15)\ge0$.
\zadStop
\rozwStart{Patryk Wirkus}{}
Miejsca zerowe naszego wielomianu to: $7, 12, 15$.\\
Wielomian jest stopnia nieparzystego, ponadto znak współczynnika przy\linebreak najwyższej potędze x jest dodatni.\\ W związku z tym wykres wielomianu zaczyna się od lewej strony poniżej osi OX. A więc $$x \in [7,12] \cup [15,\infty).$$
\rozwStop
\odpStart
$x \in [7,12] \cup [15,\infty)$
\odpStop
\testStart
A.$x \in [7,12] \cup [15,\infty)$\\
B.$x \in (7,12) \cup [15,\infty)$\\
C.$x \in (7,12] \cup [15,\infty)$\\
D.$x \in [7,12) \cup [15,\infty)$\\
E.$x \in [7,12] \cup (15,\infty)$\\
F.$x \in (7,12) \cup (15,\infty)$\\
G.$x \in [7,12) \cup (15,\infty)$\\
H.$x \in (7,12] \cup (15,\infty)$
\testStop
\kluczStart
A
\kluczStop



\zadStart{Zadanie z Wikieł Z 1.62 a) moja wersja nr 822}

Rozwiązać nierówności $(x-7)(x-12)(x-16)\ge0$.
\zadStop
\rozwStart{Patryk Wirkus}{}
Miejsca zerowe naszego wielomianu to: $7, 12, 16$.\\
Wielomian jest stopnia nieparzystego, ponadto znak współczynnika przy\linebreak najwyższej potędze x jest dodatni.\\ W związku z tym wykres wielomianu zaczyna się od lewej strony poniżej osi OX. A więc $$x \in [7,12] \cup [16,\infty).$$
\rozwStop
\odpStart
$x \in [7,12] \cup [16,\infty)$
\odpStop
\testStart
A.$x \in [7,12] \cup [16,\infty)$\\
B.$x \in (7,12) \cup [16,\infty)$\\
C.$x \in (7,12] \cup [16,\infty)$\\
D.$x \in [7,12) \cup [16,\infty)$\\
E.$x \in [7,12] \cup (16,\infty)$\\
F.$x \in (7,12) \cup (16,\infty)$\\
G.$x \in [7,12) \cup (16,\infty)$\\
H.$x \in (7,12] \cup (16,\infty)$
\testStop
\kluczStart
A
\kluczStop



\zadStart{Zadanie z Wikieł Z 1.62 a) moja wersja nr 823}

Rozwiązać nierówności $(x-7)(x-12)(x-17)\ge0$.
\zadStop
\rozwStart{Patryk Wirkus}{}
Miejsca zerowe naszego wielomianu to: $7, 12, 17$.\\
Wielomian jest stopnia nieparzystego, ponadto znak współczynnika przy\linebreak najwyższej potędze x jest dodatni.\\ W związku z tym wykres wielomianu zaczyna się od lewej strony poniżej osi OX. A więc $$x \in [7,12] \cup [17,\infty).$$
\rozwStop
\odpStart
$x \in [7,12] \cup [17,\infty)$
\odpStop
\testStart
A.$x \in [7,12] \cup [17,\infty)$\\
B.$x \in (7,12) \cup [17,\infty)$\\
C.$x \in (7,12] \cup [17,\infty)$\\
D.$x \in [7,12) \cup [17,\infty)$\\
E.$x \in [7,12] \cup (17,\infty)$\\
F.$x \in (7,12) \cup (17,\infty)$\\
G.$x \in [7,12) \cup (17,\infty)$\\
H.$x \in (7,12] \cup (17,\infty)$
\testStop
\kluczStart
A
\kluczStop



\zadStart{Zadanie z Wikieł Z 1.62 a) moja wersja nr 824}

Rozwiązać nierówności $(x-7)(x-12)(x-18)\ge0$.
\zadStop
\rozwStart{Patryk Wirkus}{}
Miejsca zerowe naszego wielomianu to: $7, 12, 18$.\\
Wielomian jest stopnia nieparzystego, ponadto znak współczynnika przy\linebreak najwyższej potędze x jest dodatni.\\ W związku z tym wykres wielomianu zaczyna się od lewej strony poniżej osi OX. A więc $$x \in [7,12] \cup [18,\infty).$$
\rozwStop
\odpStart
$x \in [7,12] \cup [18,\infty)$
\odpStop
\testStart
A.$x \in [7,12] \cup [18,\infty)$\\
B.$x \in (7,12) \cup [18,\infty)$\\
C.$x \in (7,12] \cup [18,\infty)$\\
D.$x \in [7,12) \cup [18,\infty)$\\
E.$x \in [7,12] \cup (18,\infty)$\\
F.$x \in (7,12) \cup (18,\infty)$\\
G.$x \in [7,12) \cup (18,\infty)$\\
H.$x \in (7,12] \cup (18,\infty)$
\testStop
\kluczStart
A
\kluczStop



\zadStart{Zadanie z Wikieł Z 1.62 a) moja wersja nr 825}

Rozwiązać nierówności $(x-7)(x-12)(x-19)\ge0$.
\zadStop
\rozwStart{Patryk Wirkus}{}
Miejsca zerowe naszego wielomianu to: $7, 12, 19$.\\
Wielomian jest stopnia nieparzystego, ponadto znak współczynnika przy\linebreak najwyższej potędze x jest dodatni.\\ W związku z tym wykres wielomianu zaczyna się od lewej strony poniżej osi OX. A więc $$x \in [7,12] \cup [19,\infty).$$
\rozwStop
\odpStart
$x \in [7,12] \cup [19,\infty)$
\odpStop
\testStart
A.$x \in [7,12] \cup [19,\infty)$\\
B.$x \in (7,12) \cup [19,\infty)$\\
C.$x \in (7,12] \cup [19,\infty)$\\
D.$x \in [7,12) \cup [19,\infty)$\\
E.$x \in [7,12] \cup (19,\infty)$\\
F.$x \in (7,12) \cup (19,\infty)$\\
G.$x \in [7,12) \cup (19,\infty)$\\
H.$x \in (7,12] \cup (19,\infty)$
\testStop
\kluczStart
A
\kluczStop



\zadStart{Zadanie z Wikieł Z 1.62 a) moja wersja nr 826}

Rozwiązać nierówności $(x-7)(x-12)(x-20)\ge0$.
\zadStop
\rozwStart{Patryk Wirkus}{}
Miejsca zerowe naszego wielomianu to: $7, 12, 20$.\\
Wielomian jest stopnia nieparzystego, ponadto znak współczynnika przy\linebreak najwyższej potędze x jest dodatni.\\ W związku z tym wykres wielomianu zaczyna się od lewej strony poniżej osi OX. A więc $$x \in [7,12] \cup [20,\infty).$$
\rozwStop
\odpStart
$x \in [7,12] \cup [20,\infty)$
\odpStop
\testStart
A.$x \in [7,12] \cup [20,\infty)$\\
B.$x \in (7,12) \cup [20,\infty)$\\
C.$x \in (7,12] \cup [20,\infty)$\\
D.$x \in [7,12) \cup [20,\infty)$\\
E.$x \in [7,12] \cup (20,\infty)$\\
F.$x \in (7,12) \cup (20,\infty)$\\
G.$x \in [7,12) \cup (20,\infty)$\\
H.$x \in (7,12] \cup (20,\infty)$
\testStop
\kluczStart
A
\kluczStop



\zadStart{Zadanie z Wikieł Z 1.62 a) moja wersja nr 827}

Rozwiązać nierówności $(x-7)(x-13)(x-14)\ge0$.
\zadStop
\rozwStart{Patryk Wirkus}{}
Miejsca zerowe naszego wielomianu to: $7, 13, 14$.\\
Wielomian jest stopnia nieparzystego, ponadto znak współczynnika przy\linebreak najwyższej potędze x jest dodatni.\\ W związku z tym wykres wielomianu zaczyna się od lewej strony poniżej osi OX. A więc $$x \in [7,13] \cup [14,\infty).$$
\rozwStop
\odpStart
$x \in [7,13] \cup [14,\infty)$
\odpStop
\testStart
A.$x \in [7,13] \cup [14,\infty)$\\
B.$x \in (7,13) \cup [14,\infty)$\\
C.$x \in (7,13] \cup [14,\infty)$\\
D.$x \in [7,13) \cup [14,\infty)$\\
E.$x \in [7,13] \cup (14,\infty)$\\
F.$x \in (7,13) \cup (14,\infty)$\\
G.$x \in [7,13) \cup (14,\infty)$\\
H.$x \in (7,13] \cup (14,\infty)$
\testStop
\kluczStart
A
\kluczStop



\zadStart{Zadanie z Wikieł Z 1.62 a) moja wersja nr 828}

Rozwiązać nierówności $(x-7)(x-13)(x-15)\ge0$.
\zadStop
\rozwStart{Patryk Wirkus}{}
Miejsca zerowe naszego wielomianu to: $7, 13, 15$.\\
Wielomian jest stopnia nieparzystego, ponadto znak współczynnika przy\linebreak najwyższej potędze x jest dodatni.\\ W związku z tym wykres wielomianu zaczyna się od lewej strony poniżej osi OX. A więc $$x \in [7,13] \cup [15,\infty).$$
\rozwStop
\odpStart
$x \in [7,13] \cup [15,\infty)$
\odpStop
\testStart
A.$x \in [7,13] \cup [15,\infty)$\\
B.$x \in (7,13) \cup [15,\infty)$\\
C.$x \in (7,13] \cup [15,\infty)$\\
D.$x \in [7,13) \cup [15,\infty)$\\
E.$x \in [7,13] \cup (15,\infty)$\\
F.$x \in (7,13) \cup (15,\infty)$\\
G.$x \in [7,13) \cup (15,\infty)$\\
H.$x \in (7,13] \cup (15,\infty)$
\testStop
\kluczStart
A
\kluczStop



\zadStart{Zadanie z Wikieł Z 1.62 a) moja wersja nr 829}

Rozwiązać nierówności $(x-7)(x-13)(x-16)\ge0$.
\zadStop
\rozwStart{Patryk Wirkus}{}
Miejsca zerowe naszego wielomianu to: $7, 13, 16$.\\
Wielomian jest stopnia nieparzystego, ponadto znak współczynnika przy\linebreak najwyższej potędze x jest dodatni.\\ W związku z tym wykres wielomianu zaczyna się od lewej strony poniżej osi OX. A więc $$x \in [7,13] \cup [16,\infty).$$
\rozwStop
\odpStart
$x \in [7,13] \cup [16,\infty)$
\odpStop
\testStart
A.$x \in [7,13] \cup [16,\infty)$\\
B.$x \in (7,13) \cup [16,\infty)$\\
C.$x \in (7,13] \cup [16,\infty)$\\
D.$x \in [7,13) \cup [16,\infty)$\\
E.$x \in [7,13] \cup (16,\infty)$\\
F.$x \in (7,13) \cup (16,\infty)$\\
G.$x \in [7,13) \cup (16,\infty)$\\
H.$x \in (7,13] \cup (16,\infty)$
\testStop
\kluczStart
A
\kluczStop



\zadStart{Zadanie z Wikieł Z 1.62 a) moja wersja nr 830}

Rozwiązać nierówności $(x-7)(x-13)(x-17)\ge0$.
\zadStop
\rozwStart{Patryk Wirkus}{}
Miejsca zerowe naszego wielomianu to: $7, 13, 17$.\\
Wielomian jest stopnia nieparzystego, ponadto znak współczynnika przy\linebreak najwyższej potędze x jest dodatni.\\ W związku z tym wykres wielomianu zaczyna się od lewej strony poniżej osi OX. A więc $$x \in [7,13] \cup [17,\infty).$$
\rozwStop
\odpStart
$x \in [7,13] \cup [17,\infty)$
\odpStop
\testStart
A.$x \in [7,13] \cup [17,\infty)$\\
B.$x \in (7,13) \cup [17,\infty)$\\
C.$x \in (7,13] \cup [17,\infty)$\\
D.$x \in [7,13) \cup [17,\infty)$\\
E.$x \in [7,13] \cup (17,\infty)$\\
F.$x \in (7,13) \cup (17,\infty)$\\
G.$x \in [7,13) \cup (17,\infty)$\\
H.$x \in (7,13] \cup (17,\infty)$
\testStop
\kluczStart
A
\kluczStop



\zadStart{Zadanie z Wikieł Z 1.62 a) moja wersja nr 831}

Rozwiązać nierówności $(x-7)(x-13)(x-18)\ge0$.
\zadStop
\rozwStart{Patryk Wirkus}{}
Miejsca zerowe naszego wielomianu to: $7, 13, 18$.\\
Wielomian jest stopnia nieparzystego, ponadto znak współczynnika przy\linebreak najwyższej potędze x jest dodatni.\\ W związku z tym wykres wielomianu zaczyna się od lewej strony poniżej osi OX. A więc $$x \in [7,13] \cup [18,\infty).$$
\rozwStop
\odpStart
$x \in [7,13] \cup [18,\infty)$
\odpStop
\testStart
A.$x \in [7,13] \cup [18,\infty)$\\
B.$x \in (7,13) \cup [18,\infty)$\\
C.$x \in (7,13] \cup [18,\infty)$\\
D.$x \in [7,13) \cup [18,\infty)$\\
E.$x \in [7,13] \cup (18,\infty)$\\
F.$x \in (7,13) \cup (18,\infty)$\\
G.$x \in [7,13) \cup (18,\infty)$\\
H.$x \in (7,13] \cup (18,\infty)$
\testStop
\kluczStart
A
\kluczStop



\zadStart{Zadanie z Wikieł Z 1.62 a) moja wersja nr 832}

Rozwiązać nierówności $(x-7)(x-13)(x-19)\ge0$.
\zadStop
\rozwStart{Patryk Wirkus}{}
Miejsca zerowe naszego wielomianu to: $7, 13, 19$.\\
Wielomian jest stopnia nieparzystego, ponadto znak współczynnika przy\linebreak najwyższej potędze x jest dodatni.\\ W związku z tym wykres wielomianu zaczyna się od lewej strony poniżej osi OX. A więc $$x \in [7,13] \cup [19,\infty).$$
\rozwStop
\odpStart
$x \in [7,13] \cup [19,\infty)$
\odpStop
\testStart
A.$x \in [7,13] \cup [19,\infty)$\\
B.$x \in (7,13) \cup [19,\infty)$\\
C.$x \in (7,13] \cup [19,\infty)$\\
D.$x \in [7,13) \cup [19,\infty)$\\
E.$x \in [7,13] \cup (19,\infty)$\\
F.$x \in (7,13) \cup (19,\infty)$\\
G.$x \in [7,13) \cup (19,\infty)$\\
H.$x \in (7,13] \cup (19,\infty)$
\testStop
\kluczStart
A
\kluczStop



\zadStart{Zadanie z Wikieł Z 1.62 a) moja wersja nr 833}

Rozwiązać nierówności $(x-7)(x-13)(x-20)\ge0$.
\zadStop
\rozwStart{Patryk Wirkus}{}
Miejsca zerowe naszego wielomianu to: $7, 13, 20$.\\
Wielomian jest stopnia nieparzystego, ponadto znak współczynnika przy\linebreak najwyższej potędze x jest dodatni.\\ W związku z tym wykres wielomianu zaczyna się od lewej strony poniżej osi OX. A więc $$x \in [7,13] \cup [20,\infty).$$
\rozwStop
\odpStart
$x \in [7,13] \cup [20,\infty)$
\odpStop
\testStart
A.$x \in [7,13] \cup [20,\infty)$\\
B.$x \in (7,13) \cup [20,\infty)$\\
C.$x \in (7,13] \cup [20,\infty)$\\
D.$x \in [7,13) \cup [20,\infty)$\\
E.$x \in [7,13] \cup (20,\infty)$\\
F.$x \in (7,13) \cup (20,\infty)$\\
G.$x \in [7,13) \cup (20,\infty)$\\
H.$x \in (7,13] \cup (20,\infty)$
\testStop
\kluczStart
A
\kluczStop



\zadStart{Zadanie z Wikieł Z 1.62 a) moja wersja nr 834}

Rozwiązać nierówności $(x-7)(x-14)(x-15)\ge0$.
\zadStop
\rozwStart{Patryk Wirkus}{}
Miejsca zerowe naszego wielomianu to: $7, 14, 15$.\\
Wielomian jest stopnia nieparzystego, ponadto znak współczynnika przy\linebreak najwyższej potędze x jest dodatni.\\ W związku z tym wykres wielomianu zaczyna się od lewej strony poniżej osi OX. A więc $$x \in [7,14] \cup [15,\infty).$$
\rozwStop
\odpStart
$x \in [7,14] \cup [15,\infty)$
\odpStop
\testStart
A.$x \in [7,14] \cup [15,\infty)$\\
B.$x \in (7,14) \cup [15,\infty)$\\
C.$x \in (7,14] \cup [15,\infty)$\\
D.$x \in [7,14) \cup [15,\infty)$\\
E.$x \in [7,14] \cup (15,\infty)$\\
F.$x \in (7,14) \cup (15,\infty)$\\
G.$x \in [7,14) \cup (15,\infty)$\\
H.$x \in (7,14] \cup (15,\infty)$
\testStop
\kluczStart
A
\kluczStop



\zadStart{Zadanie z Wikieł Z 1.62 a) moja wersja nr 835}

Rozwiązać nierówności $(x-7)(x-14)(x-16)\ge0$.
\zadStop
\rozwStart{Patryk Wirkus}{}
Miejsca zerowe naszego wielomianu to: $7, 14, 16$.\\
Wielomian jest stopnia nieparzystego, ponadto znak współczynnika przy\linebreak najwyższej potędze x jest dodatni.\\ W związku z tym wykres wielomianu zaczyna się od lewej strony poniżej osi OX. A więc $$x \in [7,14] \cup [16,\infty).$$
\rozwStop
\odpStart
$x \in [7,14] \cup [16,\infty)$
\odpStop
\testStart
A.$x \in [7,14] \cup [16,\infty)$\\
B.$x \in (7,14) \cup [16,\infty)$\\
C.$x \in (7,14] \cup [16,\infty)$\\
D.$x \in [7,14) \cup [16,\infty)$\\
E.$x \in [7,14] \cup (16,\infty)$\\
F.$x \in (7,14) \cup (16,\infty)$\\
G.$x \in [7,14) \cup (16,\infty)$\\
H.$x \in (7,14] \cup (16,\infty)$
\testStop
\kluczStart
A
\kluczStop



\zadStart{Zadanie z Wikieł Z 1.62 a) moja wersja nr 836}

Rozwiązać nierówności $(x-7)(x-14)(x-17)\ge0$.
\zadStop
\rozwStart{Patryk Wirkus}{}
Miejsca zerowe naszego wielomianu to: $7, 14, 17$.\\
Wielomian jest stopnia nieparzystego, ponadto znak współczynnika przy\linebreak najwyższej potędze x jest dodatni.\\ W związku z tym wykres wielomianu zaczyna się od lewej strony poniżej osi OX. A więc $$x \in [7,14] \cup [17,\infty).$$
\rozwStop
\odpStart
$x \in [7,14] \cup [17,\infty)$
\odpStop
\testStart
A.$x \in [7,14] \cup [17,\infty)$\\
B.$x \in (7,14) \cup [17,\infty)$\\
C.$x \in (7,14] \cup [17,\infty)$\\
D.$x \in [7,14) \cup [17,\infty)$\\
E.$x \in [7,14] \cup (17,\infty)$\\
F.$x \in (7,14) \cup (17,\infty)$\\
G.$x \in [7,14) \cup (17,\infty)$\\
H.$x \in (7,14] \cup (17,\infty)$
\testStop
\kluczStart
A
\kluczStop



\zadStart{Zadanie z Wikieł Z 1.62 a) moja wersja nr 837}

Rozwiązać nierówności $(x-7)(x-14)(x-18)\ge0$.
\zadStop
\rozwStart{Patryk Wirkus}{}
Miejsca zerowe naszego wielomianu to: $7, 14, 18$.\\
Wielomian jest stopnia nieparzystego, ponadto znak współczynnika przy\linebreak najwyższej potędze x jest dodatni.\\ W związku z tym wykres wielomianu zaczyna się od lewej strony poniżej osi OX. A więc $$x \in [7,14] \cup [18,\infty).$$
\rozwStop
\odpStart
$x \in [7,14] \cup [18,\infty)$
\odpStop
\testStart
A.$x \in [7,14] \cup [18,\infty)$\\
B.$x \in (7,14) \cup [18,\infty)$\\
C.$x \in (7,14] \cup [18,\infty)$\\
D.$x \in [7,14) \cup [18,\infty)$\\
E.$x \in [7,14] \cup (18,\infty)$\\
F.$x \in (7,14) \cup (18,\infty)$\\
G.$x \in [7,14) \cup (18,\infty)$\\
H.$x \in (7,14] \cup (18,\infty)$
\testStop
\kluczStart
A
\kluczStop



\zadStart{Zadanie z Wikieł Z 1.62 a) moja wersja nr 838}

Rozwiązać nierówności $(x-7)(x-14)(x-19)\ge0$.
\zadStop
\rozwStart{Patryk Wirkus}{}
Miejsca zerowe naszego wielomianu to: $7, 14, 19$.\\
Wielomian jest stopnia nieparzystego, ponadto znak współczynnika przy\linebreak najwyższej potędze x jest dodatni.\\ W związku z tym wykres wielomianu zaczyna się od lewej strony poniżej osi OX. A więc $$x \in [7,14] \cup [19,\infty).$$
\rozwStop
\odpStart
$x \in [7,14] \cup [19,\infty)$
\odpStop
\testStart
A.$x \in [7,14] \cup [19,\infty)$\\
B.$x \in (7,14) \cup [19,\infty)$\\
C.$x \in (7,14] \cup [19,\infty)$\\
D.$x \in [7,14) \cup [19,\infty)$\\
E.$x \in [7,14] \cup (19,\infty)$\\
F.$x \in (7,14) \cup (19,\infty)$\\
G.$x \in [7,14) \cup (19,\infty)$\\
H.$x \in (7,14] \cup (19,\infty)$
\testStop
\kluczStart
A
\kluczStop



\zadStart{Zadanie z Wikieł Z 1.62 a) moja wersja nr 839}

Rozwiązać nierówności $(x-7)(x-14)(x-20)\ge0$.
\zadStop
\rozwStart{Patryk Wirkus}{}
Miejsca zerowe naszego wielomianu to: $7, 14, 20$.\\
Wielomian jest stopnia nieparzystego, ponadto znak współczynnika przy\linebreak najwyższej potędze x jest dodatni.\\ W związku z tym wykres wielomianu zaczyna się od lewej strony poniżej osi OX. A więc $$x \in [7,14] \cup [20,\infty).$$
\rozwStop
\odpStart
$x \in [7,14] \cup [20,\infty)$
\odpStop
\testStart
A.$x \in [7,14] \cup [20,\infty)$\\
B.$x \in (7,14) \cup [20,\infty)$\\
C.$x \in (7,14] \cup [20,\infty)$\\
D.$x \in [7,14) \cup [20,\infty)$\\
E.$x \in [7,14] \cup (20,\infty)$\\
F.$x \in (7,14) \cup (20,\infty)$\\
G.$x \in [7,14) \cup (20,\infty)$\\
H.$x \in (7,14] \cup (20,\infty)$
\testStop
\kluczStart
A
\kluczStop



\zadStart{Zadanie z Wikieł Z 1.62 a) moja wersja nr 840}

Rozwiązać nierówności $(x-7)(x-15)(x-16)\ge0$.
\zadStop
\rozwStart{Patryk Wirkus}{}
Miejsca zerowe naszego wielomianu to: $7, 15, 16$.\\
Wielomian jest stopnia nieparzystego, ponadto znak współczynnika przy\linebreak najwyższej potędze x jest dodatni.\\ W związku z tym wykres wielomianu zaczyna się od lewej strony poniżej osi OX. A więc $$x \in [7,15] \cup [16,\infty).$$
\rozwStop
\odpStart
$x \in [7,15] \cup [16,\infty)$
\odpStop
\testStart
A.$x \in [7,15] \cup [16,\infty)$\\
B.$x \in (7,15) \cup [16,\infty)$\\
C.$x \in (7,15] \cup [16,\infty)$\\
D.$x \in [7,15) \cup [16,\infty)$\\
E.$x \in [7,15] \cup (16,\infty)$\\
F.$x \in (7,15) \cup (16,\infty)$\\
G.$x \in [7,15) \cup (16,\infty)$\\
H.$x \in (7,15] \cup (16,\infty)$
\testStop
\kluczStart
A
\kluczStop



\zadStart{Zadanie z Wikieł Z 1.62 a) moja wersja nr 841}

Rozwiązać nierówności $(x-7)(x-15)(x-17)\ge0$.
\zadStop
\rozwStart{Patryk Wirkus}{}
Miejsca zerowe naszego wielomianu to: $7, 15, 17$.\\
Wielomian jest stopnia nieparzystego, ponadto znak współczynnika przy\linebreak najwyższej potędze x jest dodatni.\\ W związku z tym wykres wielomianu zaczyna się od lewej strony poniżej osi OX. A więc $$x \in [7,15] \cup [17,\infty).$$
\rozwStop
\odpStart
$x \in [7,15] \cup [17,\infty)$
\odpStop
\testStart
A.$x \in [7,15] \cup [17,\infty)$\\
B.$x \in (7,15) \cup [17,\infty)$\\
C.$x \in (7,15] \cup [17,\infty)$\\
D.$x \in [7,15) \cup [17,\infty)$\\
E.$x \in [7,15] \cup (17,\infty)$\\
F.$x \in (7,15) \cup (17,\infty)$\\
G.$x \in [7,15) \cup (17,\infty)$\\
H.$x \in (7,15] \cup (17,\infty)$
\testStop
\kluczStart
A
\kluczStop



\zadStart{Zadanie z Wikieł Z 1.62 a) moja wersja nr 842}

Rozwiązać nierówności $(x-7)(x-15)(x-18)\ge0$.
\zadStop
\rozwStart{Patryk Wirkus}{}
Miejsca zerowe naszego wielomianu to: $7, 15, 18$.\\
Wielomian jest stopnia nieparzystego, ponadto znak współczynnika przy\linebreak najwyższej potędze x jest dodatni.\\ W związku z tym wykres wielomianu zaczyna się od lewej strony poniżej osi OX. A więc $$x \in [7,15] \cup [18,\infty).$$
\rozwStop
\odpStart
$x \in [7,15] \cup [18,\infty)$
\odpStop
\testStart
A.$x \in [7,15] \cup [18,\infty)$\\
B.$x \in (7,15) \cup [18,\infty)$\\
C.$x \in (7,15] \cup [18,\infty)$\\
D.$x \in [7,15) \cup [18,\infty)$\\
E.$x \in [7,15] \cup (18,\infty)$\\
F.$x \in (7,15) \cup (18,\infty)$\\
G.$x \in [7,15) \cup (18,\infty)$\\
H.$x \in (7,15] \cup (18,\infty)$
\testStop
\kluczStart
A
\kluczStop



\zadStart{Zadanie z Wikieł Z 1.62 a) moja wersja nr 843}

Rozwiązać nierówności $(x-7)(x-15)(x-19)\ge0$.
\zadStop
\rozwStart{Patryk Wirkus}{}
Miejsca zerowe naszego wielomianu to: $7, 15, 19$.\\
Wielomian jest stopnia nieparzystego, ponadto znak współczynnika przy\linebreak najwyższej potędze x jest dodatni.\\ W związku z tym wykres wielomianu zaczyna się od lewej strony poniżej osi OX. A więc $$x \in [7,15] \cup [19,\infty).$$
\rozwStop
\odpStart
$x \in [7,15] \cup [19,\infty)$
\odpStop
\testStart
A.$x \in [7,15] \cup [19,\infty)$\\
B.$x \in (7,15) \cup [19,\infty)$\\
C.$x \in (7,15] \cup [19,\infty)$\\
D.$x \in [7,15) \cup [19,\infty)$\\
E.$x \in [7,15] \cup (19,\infty)$\\
F.$x \in (7,15) \cup (19,\infty)$\\
G.$x \in [7,15) \cup (19,\infty)$\\
H.$x \in (7,15] \cup (19,\infty)$
\testStop
\kluczStart
A
\kluczStop



\zadStart{Zadanie z Wikieł Z 1.62 a) moja wersja nr 844}

Rozwiązać nierówności $(x-7)(x-15)(x-20)\ge0$.
\zadStop
\rozwStart{Patryk Wirkus}{}
Miejsca zerowe naszego wielomianu to: $7, 15, 20$.\\
Wielomian jest stopnia nieparzystego, ponadto znak współczynnika przy\linebreak najwyższej potędze x jest dodatni.\\ W związku z tym wykres wielomianu zaczyna się od lewej strony poniżej osi OX. A więc $$x \in [7,15] \cup [20,\infty).$$
\rozwStop
\odpStart
$x \in [7,15] \cup [20,\infty)$
\odpStop
\testStart
A.$x \in [7,15] \cup [20,\infty)$\\
B.$x \in (7,15) \cup [20,\infty)$\\
C.$x \in (7,15] \cup [20,\infty)$\\
D.$x \in [7,15) \cup [20,\infty)$\\
E.$x \in [7,15] \cup (20,\infty)$\\
F.$x \in (7,15) \cup (20,\infty)$\\
G.$x \in [7,15) \cup (20,\infty)$\\
H.$x \in (7,15] \cup (20,\infty)$
\testStop
\kluczStart
A
\kluczStop



\zadStart{Zadanie z Wikieł Z 1.62 a) moja wersja nr 845}

Rozwiązać nierówności $(x-7)(x-16)(x-17)\ge0$.
\zadStop
\rozwStart{Patryk Wirkus}{}
Miejsca zerowe naszego wielomianu to: $7, 16, 17$.\\
Wielomian jest stopnia nieparzystego, ponadto znak współczynnika przy\linebreak najwyższej potędze x jest dodatni.\\ W związku z tym wykres wielomianu zaczyna się od lewej strony poniżej osi OX. A więc $$x \in [7,16] \cup [17,\infty).$$
\rozwStop
\odpStart
$x \in [7,16] \cup [17,\infty)$
\odpStop
\testStart
A.$x \in [7,16] \cup [17,\infty)$\\
B.$x \in (7,16) \cup [17,\infty)$\\
C.$x \in (7,16] \cup [17,\infty)$\\
D.$x \in [7,16) \cup [17,\infty)$\\
E.$x \in [7,16] \cup (17,\infty)$\\
F.$x \in (7,16) \cup (17,\infty)$\\
G.$x \in [7,16) \cup (17,\infty)$\\
H.$x \in (7,16] \cup (17,\infty)$
\testStop
\kluczStart
A
\kluczStop



\zadStart{Zadanie z Wikieł Z 1.62 a) moja wersja nr 846}

Rozwiązać nierówności $(x-7)(x-16)(x-18)\ge0$.
\zadStop
\rozwStart{Patryk Wirkus}{}
Miejsca zerowe naszego wielomianu to: $7, 16, 18$.\\
Wielomian jest stopnia nieparzystego, ponadto znak współczynnika przy\linebreak najwyższej potędze x jest dodatni.\\ W związku z tym wykres wielomianu zaczyna się od lewej strony poniżej osi OX. A więc $$x \in [7,16] \cup [18,\infty).$$
\rozwStop
\odpStart
$x \in [7,16] \cup [18,\infty)$
\odpStop
\testStart
A.$x \in [7,16] \cup [18,\infty)$\\
B.$x \in (7,16) \cup [18,\infty)$\\
C.$x \in (7,16] \cup [18,\infty)$\\
D.$x \in [7,16) \cup [18,\infty)$\\
E.$x \in [7,16] \cup (18,\infty)$\\
F.$x \in (7,16) \cup (18,\infty)$\\
G.$x \in [7,16) \cup (18,\infty)$\\
H.$x \in (7,16] \cup (18,\infty)$
\testStop
\kluczStart
A
\kluczStop



\zadStart{Zadanie z Wikieł Z 1.62 a) moja wersja nr 847}

Rozwiązać nierówności $(x-7)(x-16)(x-19)\ge0$.
\zadStop
\rozwStart{Patryk Wirkus}{}
Miejsca zerowe naszego wielomianu to: $7, 16, 19$.\\
Wielomian jest stopnia nieparzystego, ponadto znak współczynnika przy\linebreak najwyższej potędze x jest dodatni.\\ W związku z tym wykres wielomianu zaczyna się od lewej strony poniżej osi OX. A więc $$x \in [7,16] \cup [19,\infty).$$
\rozwStop
\odpStart
$x \in [7,16] \cup [19,\infty)$
\odpStop
\testStart
A.$x \in [7,16] \cup [19,\infty)$\\
B.$x \in (7,16) \cup [19,\infty)$\\
C.$x \in (7,16] \cup [19,\infty)$\\
D.$x \in [7,16) \cup [19,\infty)$\\
E.$x \in [7,16] \cup (19,\infty)$\\
F.$x \in (7,16) \cup (19,\infty)$\\
G.$x \in [7,16) \cup (19,\infty)$\\
H.$x \in (7,16] \cup (19,\infty)$
\testStop
\kluczStart
A
\kluczStop



\zadStart{Zadanie z Wikieł Z 1.62 a) moja wersja nr 848}

Rozwiązać nierówności $(x-7)(x-16)(x-20)\ge0$.
\zadStop
\rozwStart{Patryk Wirkus}{}
Miejsca zerowe naszego wielomianu to: $7, 16, 20$.\\
Wielomian jest stopnia nieparzystego, ponadto znak współczynnika przy\linebreak najwyższej potędze x jest dodatni.\\ W związku z tym wykres wielomianu zaczyna się od lewej strony poniżej osi OX. A więc $$x \in [7,16] \cup [20,\infty).$$
\rozwStop
\odpStart
$x \in [7,16] \cup [20,\infty)$
\odpStop
\testStart
A.$x \in [7,16] \cup [20,\infty)$\\
B.$x \in (7,16) \cup [20,\infty)$\\
C.$x \in (7,16] \cup [20,\infty)$\\
D.$x \in [7,16) \cup [20,\infty)$\\
E.$x \in [7,16] \cup (20,\infty)$\\
F.$x \in (7,16) \cup (20,\infty)$\\
G.$x \in [7,16) \cup (20,\infty)$\\
H.$x \in (7,16] \cup (20,\infty)$
\testStop
\kluczStart
A
\kluczStop



\zadStart{Zadanie z Wikieł Z 1.62 a) moja wersja nr 849}

Rozwiązać nierówności $(x-7)(x-17)(x-18)\ge0$.
\zadStop
\rozwStart{Patryk Wirkus}{}
Miejsca zerowe naszego wielomianu to: $7, 17, 18$.\\
Wielomian jest stopnia nieparzystego, ponadto znak współczynnika przy\linebreak najwyższej potędze x jest dodatni.\\ W związku z tym wykres wielomianu zaczyna się od lewej strony poniżej osi OX. A więc $$x \in [7,17] \cup [18,\infty).$$
\rozwStop
\odpStart
$x \in [7,17] \cup [18,\infty)$
\odpStop
\testStart
A.$x \in [7,17] \cup [18,\infty)$\\
B.$x \in (7,17) \cup [18,\infty)$\\
C.$x \in (7,17] \cup [18,\infty)$\\
D.$x \in [7,17) \cup [18,\infty)$\\
E.$x \in [7,17] \cup (18,\infty)$\\
F.$x \in (7,17) \cup (18,\infty)$\\
G.$x \in [7,17) \cup (18,\infty)$\\
H.$x \in (7,17] \cup (18,\infty)$
\testStop
\kluczStart
A
\kluczStop



\zadStart{Zadanie z Wikieł Z 1.62 a) moja wersja nr 850}

Rozwiązać nierówności $(x-7)(x-17)(x-19)\ge0$.
\zadStop
\rozwStart{Patryk Wirkus}{}
Miejsca zerowe naszego wielomianu to: $7, 17, 19$.\\
Wielomian jest stopnia nieparzystego, ponadto znak współczynnika przy\linebreak najwyższej potędze x jest dodatni.\\ W związku z tym wykres wielomianu zaczyna się od lewej strony poniżej osi OX. A więc $$x \in [7,17] \cup [19,\infty).$$
\rozwStop
\odpStart
$x \in [7,17] \cup [19,\infty)$
\odpStop
\testStart
A.$x \in [7,17] \cup [19,\infty)$\\
B.$x \in (7,17) \cup [19,\infty)$\\
C.$x \in (7,17] \cup [19,\infty)$\\
D.$x \in [7,17) \cup [19,\infty)$\\
E.$x \in [7,17] \cup (19,\infty)$\\
F.$x \in (7,17) \cup (19,\infty)$\\
G.$x \in [7,17) \cup (19,\infty)$\\
H.$x \in (7,17] \cup (19,\infty)$
\testStop
\kluczStart
A
\kluczStop



\zadStart{Zadanie z Wikieł Z 1.62 a) moja wersja nr 851}

Rozwiązać nierówności $(x-7)(x-17)(x-20)\ge0$.
\zadStop
\rozwStart{Patryk Wirkus}{}
Miejsca zerowe naszego wielomianu to: $7, 17, 20$.\\
Wielomian jest stopnia nieparzystego, ponadto znak współczynnika przy\linebreak najwyższej potędze x jest dodatni.\\ W związku z tym wykres wielomianu zaczyna się od lewej strony poniżej osi OX. A więc $$x \in [7,17] \cup [20,\infty).$$
\rozwStop
\odpStart
$x \in [7,17] \cup [20,\infty)$
\odpStop
\testStart
A.$x \in [7,17] \cup [20,\infty)$\\
B.$x \in (7,17) \cup [20,\infty)$\\
C.$x \in (7,17] \cup [20,\infty)$\\
D.$x \in [7,17) \cup [20,\infty)$\\
E.$x \in [7,17] \cup (20,\infty)$\\
F.$x \in (7,17) \cup (20,\infty)$\\
G.$x \in [7,17) \cup (20,\infty)$\\
H.$x \in (7,17] \cup (20,\infty)$
\testStop
\kluczStart
A
\kluczStop



\zadStart{Zadanie z Wikieł Z 1.62 a) moja wersja nr 852}

Rozwiązać nierówności $(x-7)(x-18)(x-19)\ge0$.
\zadStop
\rozwStart{Patryk Wirkus}{}
Miejsca zerowe naszego wielomianu to: $7, 18, 19$.\\
Wielomian jest stopnia nieparzystego, ponadto znak współczynnika przy\linebreak najwyższej potędze x jest dodatni.\\ W związku z tym wykres wielomianu zaczyna się od lewej strony poniżej osi OX. A więc $$x \in [7,18] \cup [19,\infty).$$
\rozwStop
\odpStart
$x \in [7,18] \cup [19,\infty)$
\odpStop
\testStart
A.$x \in [7,18] \cup [19,\infty)$\\
B.$x \in (7,18) \cup [19,\infty)$\\
C.$x \in (7,18] \cup [19,\infty)$\\
D.$x \in [7,18) \cup [19,\infty)$\\
E.$x \in [7,18] \cup (19,\infty)$\\
F.$x \in (7,18) \cup (19,\infty)$\\
G.$x \in [7,18) \cup (19,\infty)$\\
H.$x \in (7,18] \cup (19,\infty)$
\testStop
\kluczStart
A
\kluczStop



\zadStart{Zadanie z Wikieł Z 1.62 a) moja wersja nr 853}

Rozwiązać nierówności $(x-7)(x-18)(x-20)\ge0$.
\zadStop
\rozwStart{Patryk Wirkus}{}
Miejsca zerowe naszego wielomianu to: $7, 18, 20$.\\
Wielomian jest stopnia nieparzystego, ponadto znak współczynnika przy\linebreak najwyższej potędze x jest dodatni.\\ W związku z tym wykres wielomianu zaczyna się od lewej strony poniżej osi OX. A więc $$x \in [7,18] \cup [20,\infty).$$
\rozwStop
\odpStart
$x \in [7,18] \cup [20,\infty)$
\odpStop
\testStart
A.$x \in [7,18] \cup [20,\infty)$\\
B.$x \in (7,18) \cup [20,\infty)$\\
C.$x \in (7,18] \cup [20,\infty)$\\
D.$x \in [7,18) \cup [20,\infty)$\\
E.$x \in [7,18] \cup (20,\infty)$\\
F.$x \in (7,18) \cup (20,\infty)$\\
G.$x \in [7,18) \cup (20,\infty)$\\
H.$x \in (7,18] \cup (20,\infty)$
\testStop
\kluczStart
A
\kluczStop



\zadStart{Zadanie z Wikieł Z 1.62 a) moja wersja nr 854}

Rozwiązać nierówności $(x-7)(x-19)(x-20)\ge0$.
\zadStop
\rozwStart{Patryk Wirkus}{}
Miejsca zerowe naszego wielomianu to: $7, 19, 20$.\\
Wielomian jest stopnia nieparzystego, ponadto znak współczynnika przy\linebreak najwyższej potędze x jest dodatni.\\ W związku z tym wykres wielomianu zaczyna się od lewej strony poniżej osi OX. A więc $$x \in [7,19] \cup [20,\infty).$$
\rozwStop
\odpStart
$x \in [7,19] \cup [20,\infty)$
\odpStop
\testStart
A.$x \in [7,19] \cup [20,\infty)$\\
B.$x \in (7,19) \cup [20,\infty)$\\
C.$x \in (7,19] \cup [20,\infty)$\\
D.$x \in [7,19) \cup [20,\infty)$\\
E.$x \in [7,19] \cup (20,\infty)$\\
F.$x \in (7,19) \cup (20,\infty)$\\
G.$x \in [7,19) \cup (20,\infty)$\\
H.$x \in (7,19] \cup (20,\infty)$
\testStop
\kluczStart
A
\kluczStop



\zadStart{Zadanie z Wikieł Z 1.62 a) moja wersja nr 855}

Rozwiązać nierówności $(x-8)(x-9)(x-10)\ge0$.
\zadStop
\rozwStart{Patryk Wirkus}{}
Miejsca zerowe naszego wielomianu to: $8, 9, 10$.\\
Wielomian jest stopnia nieparzystego, ponadto znak współczynnika przy\linebreak najwyższej potędze x jest dodatni.\\ W związku z tym wykres wielomianu zaczyna się od lewej strony poniżej osi OX. A więc $$x \in [8,9] \cup [10,\infty).$$
\rozwStop
\odpStart
$x \in [8,9] \cup [10,\infty)$
\odpStop
\testStart
A.$x \in [8,9] \cup [10,\infty)$\\
B.$x \in (8,9) \cup [10,\infty)$\\
C.$x \in (8,9] \cup [10,\infty)$\\
D.$x \in [8,9) \cup [10,\infty)$\\
E.$x \in [8,9] \cup (10,\infty)$\\
F.$x \in (8,9) \cup (10,\infty)$\\
G.$x \in [8,9) \cup (10,\infty)$\\
H.$x \in (8,9] \cup (10,\infty)$
\testStop
\kluczStart
A
\kluczStop



\zadStart{Zadanie z Wikieł Z 1.62 a) moja wersja nr 856}

Rozwiązać nierówności $(x-8)(x-9)(x-11)\ge0$.
\zadStop
\rozwStart{Patryk Wirkus}{}
Miejsca zerowe naszego wielomianu to: $8, 9, 11$.\\
Wielomian jest stopnia nieparzystego, ponadto znak współczynnika przy\linebreak najwyższej potędze x jest dodatni.\\ W związku z tym wykres wielomianu zaczyna się od lewej strony poniżej osi OX. A więc $$x \in [8,9] \cup [11,\infty).$$
\rozwStop
\odpStart
$x \in [8,9] \cup [11,\infty)$
\odpStop
\testStart
A.$x \in [8,9] \cup [11,\infty)$\\
B.$x \in (8,9) \cup [11,\infty)$\\
C.$x \in (8,9] \cup [11,\infty)$\\
D.$x \in [8,9) \cup [11,\infty)$\\
E.$x \in [8,9] \cup (11,\infty)$\\
F.$x \in (8,9) \cup (11,\infty)$\\
G.$x \in [8,9) \cup (11,\infty)$\\
H.$x \in (8,9] \cup (11,\infty)$
\testStop
\kluczStart
A
\kluczStop



\zadStart{Zadanie z Wikieł Z 1.62 a) moja wersja nr 857}

Rozwiązać nierówności $(x-8)(x-9)(x-12)\ge0$.
\zadStop
\rozwStart{Patryk Wirkus}{}
Miejsca zerowe naszego wielomianu to: $8, 9, 12$.\\
Wielomian jest stopnia nieparzystego, ponadto znak współczynnika przy\linebreak najwyższej potędze x jest dodatni.\\ W związku z tym wykres wielomianu zaczyna się od lewej strony poniżej osi OX. A więc $$x \in [8,9] \cup [12,\infty).$$
\rozwStop
\odpStart
$x \in [8,9] \cup [12,\infty)$
\odpStop
\testStart
A.$x \in [8,9] \cup [12,\infty)$\\
B.$x \in (8,9) \cup [12,\infty)$\\
C.$x \in (8,9] \cup [12,\infty)$\\
D.$x \in [8,9) \cup [12,\infty)$\\
E.$x \in [8,9] \cup (12,\infty)$\\
F.$x \in (8,9) \cup (12,\infty)$\\
G.$x \in [8,9) \cup (12,\infty)$\\
H.$x \in (8,9] \cup (12,\infty)$
\testStop
\kluczStart
A
\kluczStop



\zadStart{Zadanie z Wikieł Z 1.62 a) moja wersja nr 858}

Rozwiązać nierówności $(x-8)(x-9)(x-13)\ge0$.
\zadStop
\rozwStart{Patryk Wirkus}{}
Miejsca zerowe naszego wielomianu to: $8, 9, 13$.\\
Wielomian jest stopnia nieparzystego, ponadto znak współczynnika przy\linebreak najwyższej potędze x jest dodatni.\\ W związku z tym wykres wielomianu zaczyna się od lewej strony poniżej osi OX. A więc $$x \in [8,9] \cup [13,\infty).$$
\rozwStop
\odpStart
$x \in [8,9] \cup [13,\infty)$
\odpStop
\testStart
A.$x \in [8,9] \cup [13,\infty)$\\
B.$x \in (8,9) \cup [13,\infty)$\\
C.$x \in (8,9] \cup [13,\infty)$\\
D.$x \in [8,9) \cup [13,\infty)$\\
E.$x \in [8,9] \cup (13,\infty)$\\
F.$x \in (8,9) \cup (13,\infty)$\\
G.$x \in [8,9) \cup (13,\infty)$\\
H.$x \in (8,9] \cup (13,\infty)$
\testStop
\kluczStart
A
\kluczStop



\zadStart{Zadanie z Wikieł Z 1.62 a) moja wersja nr 859}

Rozwiązać nierówności $(x-8)(x-9)(x-14)\ge0$.
\zadStop
\rozwStart{Patryk Wirkus}{}
Miejsca zerowe naszego wielomianu to: $8, 9, 14$.\\
Wielomian jest stopnia nieparzystego, ponadto znak współczynnika przy\linebreak najwyższej potędze x jest dodatni.\\ W związku z tym wykres wielomianu zaczyna się od lewej strony poniżej osi OX. A więc $$x \in [8,9] \cup [14,\infty).$$
\rozwStop
\odpStart
$x \in [8,9] \cup [14,\infty)$
\odpStop
\testStart
A.$x \in [8,9] \cup [14,\infty)$\\
B.$x \in (8,9) \cup [14,\infty)$\\
C.$x \in (8,9] \cup [14,\infty)$\\
D.$x \in [8,9) \cup [14,\infty)$\\
E.$x \in [8,9] \cup (14,\infty)$\\
F.$x \in (8,9) \cup (14,\infty)$\\
G.$x \in [8,9) \cup (14,\infty)$\\
H.$x \in (8,9] \cup (14,\infty)$
\testStop
\kluczStart
A
\kluczStop



\zadStart{Zadanie z Wikieł Z 1.62 a) moja wersja nr 860}

Rozwiązać nierówności $(x-8)(x-9)(x-15)\ge0$.
\zadStop
\rozwStart{Patryk Wirkus}{}
Miejsca zerowe naszego wielomianu to: $8, 9, 15$.\\
Wielomian jest stopnia nieparzystego, ponadto znak współczynnika przy\linebreak najwyższej potędze x jest dodatni.\\ W związku z tym wykres wielomianu zaczyna się od lewej strony poniżej osi OX. A więc $$x \in [8,9] \cup [15,\infty).$$
\rozwStop
\odpStart
$x \in [8,9] \cup [15,\infty)$
\odpStop
\testStart
A.$x \in [8,9] \cup [15,\infty)$\\
B.$x \in (8,9) \cup [15,\infty)$\\
C.$x \in (8,9] \cup [15,\infty)$\\
D.$x \in [8,9) \cup [15,\infty)$\\
E.$x \in [8,9] \cup (15,\infty)$\\
F.$x \in (8,9) \cup (15,\infty)$\\
G.$x \in [8,9) \cup (15,\infty)$\\
H.$x \in (8,9] \cup (15,\infty)$
\testStop
\kluczStart
A
\kluczStop



\zadStart{Zadanie z Wikieł Z 1.62 a) moja wersja nr 861}

Rozwiązać nierówności $(x-8)(x-9)(x-16)\ge0$.
\zadStop
\rozwStart{Patryk Wirkus}{}
Miejsca zerowe naszego wielomianu to: $8, 9, 16$.\\
Wielomian jest stopnia nieparzystego, ponadto znak współczynnika przy\linebreak najwyższej potędze x jest dodatni.\\ W związku z tym wykres wielomianu zaczyna się od lewej strony poniżej osi OX. A więc $$x \in [8,9] \cup [16,\infty).$$
\rozwStop
\odpStart
$x \in [8,9] \cup [16,\infty)$
\odpStop
\testStart
A.$x \in [8,9] \cup [16,\infty)$\\
B.$x \in (8,9) \cup [16,\infty)$\\
C.$x \in (8,9] \cup [16,\infty)$\\
D.$x \in [8,9) \cup [16,\infty)$\\
E.$x \in [8,9] \cup (16,\infty)$\\
F.$x \in (8,9) \cup (16,\infty)$\\
G.$x \in [8,9) \cup (16,\infty)$\\
H.$x \in (8,9] \cup (16,\infty)$
\testStop
\kluczStart
A
\kluczStop



\zadStart{Zadanie z Wikieł Z 1.62 a) moja wersja nr 862}

Rozwiązać nierówności $(x-8)(x-9)(x-17)\ge0$.
\zadStop
\rozwStart{Patryk Wirkus}{}
Miejsca zerowe naszego wielomianu to: $8, 9, 17$.\\
Wielomian jest stopnia nieparzystego, ponadto znak współczynnika przy\linebreak najwyższej potędze x jest dodatni.\\ W związku z tym wykres wielomianu zaczyna się od lewej strony poniżej osi OX. A więc $$x \in [8,9] \cup [17,\infty).$$
\rozwStop
\odpStart
$x \in [8,9] \cup [17,\infty)$
\odpStop
\testStart
A.$x \in [8,9] \cup [17,\infty)$\\
B.$x \in (8,9) \cup [17,\infty)$\\
C.$x \in (8,9] \cup [17,\infty)$\\
D.$x \in [8,9) \cup [17,\infty)$\\
E.$x \in [8,9] \cup (17,\infty)$\\
F.$x \in (8,9) \cup (17,\infty)$\\
G.$x \in [8,9) \cup (17,\infty)$\\
H.$x \in (8,9] \cup (17,\infty)$
\testStop
\kluczStart
A
\kluczStop



\zadStart{Zadanie z Wikieł Z 1.62 a) moja wersja nr 863}

Rozwiązać nierówności $(x-8)(x-9)(x-18)\ge0$.
\zadStop
\rozwStart{Patryk Wirkus}{}
Miejsca zerowe naszego wielomianu to: $8, 9, 18$.\\
Wielomian jest stopnia nieparzystego, ponadto znak współczynnika przy\linebreak najwyższej potędze x jest dodatni.\\ W związku z tym wykres wielomianu zaczyna się od lewej strony poniżej osi OX. A więc $$x \in [8,9] \cup [18,\infty).$$
\rozwStop
\odpStart
$x \in [8,9] \cup [18,\infty)$
\odpStop
\testStart
A.$x \in [8,9] \cup [18,\infty)$\\
B.$x \in (8,9) \cup [18,\infty)$\\
C.$x \in (8,9] \cup [18,\infty)$\\
D.$x \in [8,9) \cup [18,\infty)$\\
E.$x \in [8,9] \cup (18,\infty)$\\
F.$x \in (8,9) \cup (18,\infty)$\\
G.$x \in [8,9) \cup (18,\infty)$\\
H.$x \in (8,9] \cup (18,\infty)$
\testStop
\kluczStart
A
\kluczStop



\zadStart{Zadanie z Wikieł Z 1.62 a) moja wersja nr 864}

Rozwiązać nierówności $(x-8)(x-9)(x-19)\ge0$.
\zadStop
\rozwStart{Patryk Wirkus}{}
Miejsca zerowe naszego wielomianu to: $8, 9, 19$.\\
Wielomian jest stopnia nieparzystego, ponadto znak współczynnika przy\linebreak najwyższej potędze x jest dodatni.\\ W związku z tym wykres wielomianu zaczyna się od lewej strony poniżej osi OX. A więc $$x \in [8,9] \cup [19,\infty).$$
\rozwStop
\odpStart
$x \in [8,9] \cup [19,\infty)$
\odpStop
\testStart
A.$x \in [8,9] \cup [19,\infty)$\\
B.$x \in (8,9) \cup [19,\infty)$\\
C.$x \in (8,9] \cup [19,\infty)$\\
D.$x \in [8,9) \cup [19,\infty)$\\
E.$x \in [8,9] \cup (19,\infty)$\\
F.$x \in (8,9) \cup (19,\infty)$\\
G.$x \in [8,9) \cup (19,\infty)$\\
H.$x \in (8,9] \cup (19,\infty)$
\testStop
\kluczStart
A
\kluczStop



\zadStart{Zadanie z Wikieł Z 1.62 a) moja wersja nr 865}

Rozwiązać nierówności $(x-8)(x-9)(x-20)\ge0$.
\zadStop
\rozwStart{Patryk Wirkus}{}
Miejsca zerowe naszego wielomianu to: $8, 9, 20$.\\
Wielomian jest stopnia nieparzystego, ponadto znak współczynnika przy\linebreak najwyższej potędze x jest dodatni.\\ W związku z tym wykres wielomianu zaczyna się od lewej strony poniżej osi OX. A więc $$x \in [8,9] \cup [20,\infty).$$
\rozwStop
\odpStart
$x \in [8,9] \cup [20,\infty)$
\odpStop
\testStart
A.$x \in [8,9] \cup [20,\infty)$\\
B.$x \in (8,9) \cup [20,\infty)$\\
C.$x \in (8,9] \cup [20,\infty)$\\
D.$x \in [8,9) \cup [20,\infty)$\\
E.$x \in [8,9] \cup (20,\infty)$\\
F.$x \in (8,9) \cup (20,\infty)$\\
G.$x \in [8,9) \cup (20,\infty)$\\
H.$x \in (8,9] \cup (20,\infty)$
\testStop
\kluczStart
A
\kluczStop



\zadStart{Zadanie z Wikieł Z 1.62 a) moja wersja nr 866}

Rozwiązać nierówności $(x-8)(x-10)(x-11)\ge0$.
\zadStop
\rozwStart{Patryk Wirkus}{}
Miejsca zerowe naszego wielomianu to: $8, 10, 11$.\\
Wielomian jest stopnia nieparzystego, ponadto znak współczynnika przy\linebreak najwyższej potędze x jest dodatni.\\ W związku z tym wykres wielomianu zaczyna się od lewej strony poniżej osi OX. A więc $$x \in [8,10] \cup [11,\infty).$$
\rozwStop
\odpStart
$x \in [8,10] \cup [11,\infty)$
\odpStop
\testStart
A.$x \in [8,10] \cup [11,\infty)$\\
B.$x \in (8,10) \cup [11,\infty)$\\
C.$x \in (8,10] \cup [11,\infty)$\\
D.$x \in [8,10) \cup [11,\infty)$\\
E.$x \in [8,10] \cup (11,\infty)$\\
F.$x \in (8,10) \cup (11,\infty)$\\
G.$x \in [8,10) \cup (11,\infty)$\\
H.$x \in (8,10] \cup (11,\infty)$
\testStop
\kluczStart
A
\kluczStop



\zadStart{Zadanie z Wikieł Z 1.62 a) moja wersja nr 867}

Rozwiązać nierówności $(x-8)(x-10)(x-12)\ge0$.
\zadStop
\rozwStart{Patryk Wirkus}{}
Miejsca zerowe naszego wielomianu to: $8, 10, 12$.\\
Wielomian jest stopnia nieparzystego, ponadto znak współczynnika przy\linebreak najwyższej potędze x jest dodatni.\\ W związku z tym wykres wielomianu zaczyna się od lewej strony poniżej osi OX. A więc $$x \in [8,10] \cup [12,\infty).$$
\rozwStop
\odpStart
$x \in [8,10] \cup [12,\infty)$
\odpStop
\testStart
A.$x \in [8,10] \cup [12,\infty)$\\
B.$x \in (8,10) \cup [12,\infty)$\\
C.$x \in (8,10] \cup [12,\infty)$\\
D.$x \in [8,10) \cup [12,\infty)$\\
E.$x \in [8,10] \cup (12,\infty)$\\
F.$x \in (8,10) \cup (12,\infty)$\\
G.$x \in [8,10) \cup (12,\infty)$\\
H.$x \in (8,10] \cup (12,\infty)$
\testStop
\kluczStart
A
\kluczStop



\zadStart{Zadanie z Wikieł Z 1.62 a) moja wersja nr 868}

Rozwiązać nierówności $(x-8)(x-10)(x-13)\ge0$.
\zadStop
\rozwStart{Patryk Wirkus}{}
Miejsca zerowe naszego wielomianu to: $8, 10, 13$.\\
Wielomian jest stopnia nieparzystego, ponadto znak współczynnika przy\linebreak najwyższej potędze x jest dodatni.\\ W związku z tym wykres wielomianu zaczyna się od lewej strony poniżej osi OX. A więc $$x \in [8,10] \cup [13,\infty).$$
\rozwStop
\odpStart
$x \in [8,10] \cup [13,\infty)$
\odpStop
\testStart
A.$x \in [8,10] \cup [13,\infty)$\\
B.$x \in (8,10) \cup [13,\infty)$\\
C.$x \in (8,10] \cup [13,\infty)$\\
D.$x \in [8,10) \cup [13,\infty)$\\
E.$x \in [8,10] \cup (13,\infty)$\\
F.$x \in (8,10) \cup (13,\infty)$\\
G.$x \in [8,10) \cup (13,\infty)$\\
H.$x \in (8,10] \cup (13,\infty)$
\testStop
\kluczStart
A
\kluczStop



\zadStart{Zadanie z Wikieł Z 1.62 a) moja wersja nr 869}

Rozwiązać nierówności $(x-8)(x-10)(x-14)\ge0$.
\zadStop
\rozwStart{Patryk Wirkus}{}
Miejsca zerowe naszego wielomianu to: $8, 10, 14$.\\
Wielomian jest stopnia nieparzystego, ponadto znak współczynnika przy\linebreak najwyższej potędze x jest dodatni.\\ W związku z tym wykres wielomianu zaczyna się od lewej strony poniżej osi OX. A więc $$x \in [8,10] \cup [14,\infty).$$
\rozwStop
\odpStart
$x \in [8,10] \cup [14,\infty)$
\odpStop
\testStart
A.$x \in [8,10] \cup [14,\infty)$\\
B.$x \in (8,10) \cup [14,\infty)$\\
C.$x \in (8,10] \cup [14,\infty)$\\
D.$x \in [8,10) \cup [14,\infty)$\\
E.$x \in [8,10] \cup (14,\infty)$\\
F.$x \in (8,10) \cup (14,\infty)$\\
G.$x \in [8,10) \cup (14,\infty)$\\
H.$x \in (8,10] \cup (14,\infty)$
\testStop
\kluczStart
A
\kluczStop



\zadStart{Zadanie z Wikieł Z 1.62 a) moja wersja nr 870}

Rozwiązać nierówności $(x-8)(x-10)(x-15)\ge0$.
\zadStop
\rozwStart{Patryk Wirkus}{}
Miejsca zerowe naszego wielomianu to: $8, 10, 15$.\\
Wielomian jest stopnia nieparzystego, ponadto znak współczynnika przy\linebreak najwyższej potędze x jest dodatni.\\ W związku z tym wykres wielomianu zaczyna się od lewej strony poniżej osi OX. A więc $$x \in [8,10] \cup [15,\infty).$$
\rozwStop
\odpStart
$x \in [8,10] \cup [15,\infty)$
\odpStop
\testStart
A.$x \in [8,10] \cup [15,\infty)$\\
B.$x \in (8,10) \cup [15,\infty)$\\
C.$x \in (8,10] \cup [15,\infty)$\\
D.$x \in [8,10) \cup [15,\infty)$\\
E.$x \in [8,10] \cup (15,\infty)$\\
F.$x \in (8,10) \cup (15,\infty)$\\
G.$x \in [8,10) \cup (15,\infty)$\\
H.$x \in (8,10] \cup (15,\infty)$
\testStop
\kluczStart
A
\kluczStop



\zadStart{Zadanie z Wikieł Z 1.62 a) moja wersja nr 871}

Rozwiązać nierówności $(x-8)(x-10)(x-16)\ge0$.
\zadStop
\rozwStart{Patryk Wirkus}{}
Miejsca zerowe naszego wielomianu to: $8, 10, 16$.\\
Wielomian jest stopnia nieparzystego, ponadto znak współczynnika przy\linebreak najwyższej potędze x jest dodatni.\\ W związku z tym wykres wielomianu zaczyna się od lewej strony poniżej osi OX. A więc $$x \in [8,10] \cup [16,\infty).$$
\rozwStop
\odpStart
$x \in [8,10] \cup [16,\infty)$
\odpStop
\testStart
A.$x \in [8,10] \cup [16,\infty)$\\
B.$x \in (8,10) \cup [16,\infty)$\\
C.$x \in (8,10] \cup [16,\infty)$\\
D.$x \in [8,10) \cup [16,\infty)$\\
E.$x \in [8,10] \cup (16,\infty)$\\
F.$x \in (8,10) \cup (16,\infty)$\\
G.$x \in [8,10) \cup (16,\infty)$\\
H.$x \in (8,10] \cup (16,\infty)$
\testStop
\kluczStart
A
\kluczStop



\zadStart{Zadanie z Wikieł Z 1.62 a) moja wersja nr 872}

Rozwiązać nierówności $(x-8)(x-10)(x-17)\ge0$.
\zadStop
\rozwStart{Patryk Wirkus}{}
Miejsca zerowe naszego wielomianu to: $8, 10, 17$.\\
Wielomian jest stopnia nieparzystego, ponadto znak współczynnika przy\linebreak najwyższej potędze x jest dodatni.\\ W związku z tym wykres wielomianu zaczyna się od lewej strony poniżej osi OX. A więc $$x \in [8,10] \cup [17,\infty).$$
\rozwStop
\odpStart
$x \in [8,10] \cup [17,\infty)$
\odpStop
\testStart
A.$x \in [8,10] \cup [17,\infty)$\\
B.$x \in (8,10) \cup [17,\infty)$\\
C.$x \in (8,10] \cup [17,\infty)$\\
D.$x \in [8,10) \cup [17,\infty)$\\
E.$x \in [8,10] \cup (17,\infty)$\\
F.$x \in (8,10) \cup (17,\infty)$\\
G.$x \in [8,10) \cup (17,\infty)$\\
H.$x \in (8,10] \cup (17,\infty)$
\testStop
\kluczStart
A
\kluczStop



\zadStart{Zadanie z Wikieł Z 1.62 a) moja wersja nr 873}

Rozwiązać nierówności $(x-8)(x-10)(x-18)\ge0$.
\zadStop
\rozwStart{Patryk Wirkus}{}
Miejsca zerowe naszego wielomianu to: $8, 10, 18$.\\
Wielomian jest stopnia nieparzystego, ponadto znak współczynnika przy\linebreak najwyższej potędze x jest dodatni.\\ W związku z tym wykres wielomianu zaczyna się od lewej strony poniżej osi OX. A więc $$x \in [8,10] \cup [18,\infty).$$
\rozwStop
\odpStart
$x \in [8,10] \cup [18,\infty)$
\odpStop
\testStart
A.$x \in [8,10] \cup [18,\infty)$\\
B.$x \in (8,10) \cup [18,\infty)$\\
C.$x \in (8,10] \cup [18,\infty)$\\
D.$x \in [8,10) \cup [18,\infty)$\\
E.$x \in [8,10] \cup (18,\infty)$\\
F.$x \in (8,10) \cup (18,\infty)$\\
G.$x \in [8,10) \cup (18,\infty)$\\
H.$x \in (8,10] \cup (18,\infty)$
\testStop
\kluczStart
A
\kluczStop



\zadStart{Zadanie z Wikieł Z 1.62 a) moja wersja nr 874}

Rozwiązać nierówności $(x-8)(x-10)(x-19)\ge0$.
\zadStop
\rozwStart{Patryk Wirkus}{}
Miejsca zerowe naszego wielomianu to: $8, 10, 19$.\\
Wielomian jest stopnia nieparzystego, ponadto znak współczynnika przy\linebreak najwyższej potędze x jest dodatni.\\ W związku z tym wykres wielomianu zaczyna się od lewej strony poniżej osi OX. A więc $$x \in [8,10] \cup [19,\infty).$$
\rozwStop
\odpStart
$x \in [8,10] \cup [19,\infty)$
\odpStop
\testStart
A.$x \in [8,10] \cup [19,\infty)$\\
B.$x \in (8,10) \cup [19,\infty)$\\
C.$x \in (8,10] \cup [19,\infty)$\\
D.$x \in [8,10) \cup [19,\infty)$\\
E.$x \in [8,10] \cup (19,\infty)$\\
F.$x \in (8,10) \cup (19,\infty)$\\
G.$x \in [8,10) \cup (19,\infty)$\\
H.$x \in (8,10] \cup (19,\infty)$
\testStop
\kluczStart
A
\kluczStop



\zadStart{Zadanie z Wikieł Z 1.62 a) moja wersja nr 875}

Rozwiązać nierówności $(x-8)(x-10)(x-20)\ge0$.
\zadStop
\rozwStart{Patryk Wirkus}{}
Miejsca zerowe naszego wielomianu to: $8, 10, 20$.\\
Wielomian jest stopnia nieparzystego, ponadto znak współczynnika przy\linebreak najwyższej potędze x jest dodatni.\\ W związku z tym wykres wielomianu zaczyna się od lewej strony poniżej osi OX. A więc $$x \in [8,10] \cup [20,\infty).$$
\rozwStop
\odpStart
$x \in [8,10] \cup [20,\infty)$
\odpStop
\testStart
A.$x \in [8,10] \cup [20,\infty)$\\
B.$x \in (8,10) \cup [20,\infty)$\\
C.$x \in (8,10] \cup [20,\infty)$\\
D.$x \in [8,10) \cup [20,\infty)$\\
E.$x \in [8,10] \cup (20,\infty)$\\
F.$x \in (8,10) \cup (20,\infty)$\\
G.$x \in [8,10) \cup (20,\infty)$\\
H.$x \in (8,10] \cup (20,\infty)$
\testStop
\kluczStart
A
\kluczStop



\zadStart{Zadanie z Wikieł Z 1.62 a) moja wersja nr 876}

Rozwiązać nierówności $(x-8)(x-11)(x-12)\ge0$.
\zadStop
\rozwStart{Patryk Wirkus}{}
Miejsca zerowe naszego wielomianu to: $8, 11, 12$.\\
Wielomian jest stopnia nieparzystego, ponadto znak współczynnika przy\linebreak najwyższej potędze x jest dodatni.\\ W związku z tym wykres wielomianu zaczyna się od lewej strony poniżej osi OX. A więc $$x \in [8,11] \cup [12,\infty).$$
\rozwStop
\odpStart
$x \in [8,11] \cup [12,\infty)$
\odpStop
\testStart
A.$x \in [8,11] \cup [12,\infty)$\\
B.$x \in (8,11) \cup [12,\infty)$\\
C.$x \in (8,11] \cup [12,\infty)$\\
D.$x \in [8,11) \cup [12,\infty)$\\
E.$x \in [8,11] \cup (12,\infty)$\\
F.$x \in (8,11) \cup (12,\infty)$\\
G.$x \in [8,11) \cup (12,\infty)$\\
H.$x \in (8,11] \cup (12,\infty)$
\testStop
\kluczStart
A
\kluczStop



\zadStart{Zadanie z Wikieł Z 1.62 a) moja wersja nr 877}

Rozwiązać nierówności $(x-8)(x-11)(x-13)\ge0$.
\zadStop
\rozwStart{Patryk Wirkus}{}
Miejsca zerowe naszego wielomianu to: $8, 11, 13$.\\
Wielomian jest stopnia nieparzystego, ponadto znak współczynnika przy\linebreak najwyższej potędze x jest dodatni.\\ W związku z tym wykres wielomianu zaczyna się od lewej strony poniżej osi OX. A więc $$x \in [8,11] \cup [13,\infty).$$
\rozwStop
\odpStart
$x \in [8,11] \cup [13,\infty)$
\odpStop
\testStart
A.$x \in [8,11] \cup [13,\infty)$\\
B.$x \in (8,11) \cup [13,\infty)$\\
C.$x \in (8,11] \cup [13,\infty)$\\
D.$x \in [8,11) \cup [13,\infty)$\\
E.$x \in [8,11] \cup (13,\infty)$\\
F.$x \in (8,11) \cup (13,\infty)$\\
G.$x \in [8,11) \cup (13,\infty)$\\
H.$x \in (8,11] \cup (13,\infty)$
\testStop
\kluczStart
A
\kluczStop



\zadStart{Zadanie z Wikieł Z 1.62 a) moja wersja nr 878}

Rozwiązać nierówności $(x-8)(x-11)(x-14)\ge0$.
\zadStop
\rozwStart{Patryk Wirkus}{}
Miejsca zerowe naszego wielomianu to: $8, 11, 14$.\\
Wielomian jest stopnia nieparzystego, ponadto znak współczynnika przy\linebreak najwyższej potędze x jest dodatni.\\ W związku z tym wykres wielomianu zaczyna się od lewej strony poniżej osi OX. A więc $$x \in [8,11] \cup [14,\infty).$$
\rozwStop
\odpStart
$x \in [8,11] \cup [14,\infty)$
\odpStop
\testStart
A.$x \in [8,11] \cup [14,\infty)$\\
B.$x \in (8,11) \cup [14,\infty)$\\
C.$x \in (8,11] \cup [14,\infty)$\\
D.$x \in [8,11) \cup [14,\infty)$\\
E.$x \in [8,11] \cup (14,\infty)$\\
F.$x \in (8,11) \cup (14,\infty)$\\
G.$x \in [8,11) \cup (14,\infty)$\\
H.$x \in (8,11] \cup (14,\infty)$
\testStop
\kluczStart
A
\kluczStop



\zadStart{Zadanie z Wikieł Z 1.62 a) moja wersja nr 879}

Rozwiązać nierówności $(x-8)(x-11)(x-15)\ge0$.
\zadStop
\rozwStart{Patryk Wirkus}{}
Miejsca zerowe naszego wielomianu to: $8, 11, 15$.\\
Wielomian jest stopnia nieparzystego, ponadto znak współczynnika przy\linebreak najwyższej potędze x jest dodatni.\\ W związku z tym wykres wielomianu zaczyna się od lewej strony poniżej osi OX. A więc $$x \in [8,11] \cup [15,\infty).$$
\rozwStop
\odpStart
$x \in [8,11] \cup [15,\infty)$
\odpStop
\testStart
A.$x \in [8,11] \cup [15,\infty)$\\
B.$x \in (8,11) \cup [15,\infty)$\\
C.$x \in (8,11] \cup [15,\infty)$\\
D.$x \in [8,11) \cup [15,\infty)$\\
E.$x \in [8,11] \cup (15,\infty)$\\
F.$x \in (8,11) \cup (15,\infty)$\\
G.$x \in [8,11) \cup (15,\infty)$\\
H.$x \in (8,11] \cup (15,\infty)$
\testStop
\kluczStart
A
\kluczStop



\zadStart{Zadanie z Wikieł Z 1.62 a) moja wersja nr 880}

Rozwiązać nierówności $(x-8)(x-11)(x-16)\ge0$.
\zadStop
\rozwStart{Patryk Wirkus}{}
Miejsca zerowe naszego wielomianu to: $8, 11, 16$.\\
Wielomian jest stopnia nieparzystego, ponadto znak współczynnika przy\linebreak najwyższej potędze x jest dodatni.\\ W związku z tym wykres wielomianu zaczyna się od lewej strony poniżej osi OX. A więc $$x \in [8,11] \cup [16,\infty).$$
\rozwStop
\odpStart
$x \in [8,11] \cup [16,\infty)$
\odpStop
\testStart
A.$x \in [8,11] \cup [16,\infty)$\\
B.$x \in (8,11) \cup [16,\infty)$\\
C.$x \in (8,11] \cup [16,\infty)$\\
D.$x \in [8,11) \cup [16,\infty)$\\
E.$x \in [8,11] \cup (16,\infty)$\\
F.$x \in (8,11) \cup (16,\infty)$\\
G.$x \in [8,11) \cup (16,\infty)$\\
H.$x \in (8,11] \cup (16,\infty)$
\testStop
\kluczStart
A
\kluczStop



\zadStart{Zadanie z Wikieł Z 1.62 a) moja wersja nr 881}

Rozwiązać nierówności $(x-8)(x-11)(x-17)\ge0$.
\zadStop
\rozwStart{Patryk Wirkus}{}
Miejsca zerowe naszego wielomianu to: $8, 11, 17$.\\
Wielomian jest stopnia nieparzystego, ponadto znak współczynnika przy\linebreak najwyższej potędze x jest dodatni.\\ W związku z tym wykres wielomianu zaczyna się od lewej strony poniżej osi OX. A więc $$x \in [8,11] \cup [17,\infty).$$
\rozwStop
\odpStart
$x \in [8,11] \cup [17,\infty)$
\odpStop
\testStart
A.$x \in [8,11] \cup [17,\infty)$\\
B.$x \in (8,11) \cup [17,\infty)$\\
C.$x \in (8,11] \cup [17,\infty)$\\
D.$x \in [8,11) \cup [17,\infty)$\\
E.$x \in [8,11] \cup (17,\infty)$\\
F.$x \in (8,11) \cup (17,\infty)$\\
G.$x \in [8,11) \cup (17,\infty)$\\
H.$x \in (8,11] \cup (17,\infty)$
\testStop
\kluczStart
A
\kluczStop



\zadStart{Zadanie z Wikieł Z 1.62 a) moja wersja nr 882}

Rozwiązać nierówności $(x-8)(x-11)(x-18)\ge0$.
\zadStop
\rozwStart{Patryk Wirkus}{}
Miejsca zerowe naszego wielomianu to: $8, 11, 18$.\\
Wielomian jest stopnia nieparzystego, ponadto znak współczynnika przy\linebreak najwyższej potędze x jest dodatni.\\ W związku z tym wykres wielomianu zaczyna się od lewej strony poniżej osi OX. A więc $$x \in [8,11] \cup [18,\infty).$$
\rozwStop
\odpStart
$x \in [8,11] \cup [18,\infty)$
\odpStop
\testStart
A.$x \in [8,11] \cup [18,\infty)$\\
B.$x \in (8,11) \cup [18,\infty)$\\
C.$x \in (8,11] \cup [18,\infty)$\\
D.$x \in [8,11) \cup [18,\infty)$\\
E.$x \in [8,11] \cup (18,\infty)$\\
F.$x \in (8,11) \cup (18,\infty)$\\
G.$x \in [8,11) \cup (18,\infty)$\\
H.$x \in (8,11] \cup (18,\infty)$
\testStop
\kluczStart
A
\kluczStop



\zadStart{Zadanie z Wikieł Z 1.62 a) moja wersja nr 883}

Rozwiązać nierówności $(x-8)(x-11)(x-19)\ge0$.
\zadStop
\rozwStart{Patryk Wirkus}{}
Miejsca zerowe naszego wielomianu to: $8, 11, 19$.\\
Wielomian jest stopnia nieparzystego, ponadto znak współczynnika przy\linebreak najwyższej potędze x jest dodatni.\\ W związku z tym wykres wielomianu zaczyna się od lewej strony poniżej osi OX. A więc $$x \in [8,11] \cup [19,\infty).$$
\rozwStop
\odpStart
$x \in [8,11] \cup [19,\infty)$
\odpStop
\testStart
A.$x \in [8,11] \cup [19,\infty)$\\
B.$x \in (8,11) \cup [19,\infty)$\\
C.$x \in (8,11] \cup [19,\infty)$\\
D.$x \in [8,11) \cup [19,\infty)$\\
E.$x \in [8,11] \cup (19,\infty)$\\
F.$x \in (8,11) \cup (19,\infty)$\\
G.$x \in [8,11) \cup (19,\infty)$\\
H.$x \in (8,11] \cup (19,\infty)$
\testStop
\kluczStart
A
\kluczStop



\zadStart{Zadanie z Wikieł Z 1.62 a) moja wersja nr 884}

Rozwiązać nierówności $(x-8)(x-11)(x-20)\ge0$.
\zadStop
\rozwStart{Patryk Wirkus}{}
Miejsca zerowe naszego wielomianu to: $8, 11, 20$.\\
Wielomian jest stopnia nieparzystego, ponadto znak współczynnika przy\linebreak najwyższej potędze x jest dodatni.\\ W związku z tym wykres wielomianu zaczyna się od lewej strony poniżej osi OX. A więc $$x \in [8,11] \cup [20,\infty).$$
\rozwStop
\odpStart
$x \in [8,11] \cup [20,\infty)$
\odpStop
\testStart
A.$x \in [8,11] \cup [20,\infty)$\\
B.$x \in (8,11) \cup [20,\infty)$\\
C.$x \in (8,11] \cup [20,\infty)$\\
D.$x \in [8,11) \cup [20,\infty)$\\
E.$x \in [8,11] \cup (20,\infty)$\\
F.$x \in (8,11) \cup (20,\infty)$\\
G.$x \in [8,11) \cup (20,\infty)$\\
H.$x \in (8,11] \cup (20,\infty)$
\testStop
\kluczStart
A
\kluczStop



\zadStart{Zadanie z Wikieł Z 1.62 a) moja wersja nr 885}

Rozwiązać nierówności $(x-8)(x-12)(x-13)\ge0$.
\zadStop
\rozwStart{Patryk Wirkus}{}
Miejsca zerowe naszego wielomianu to: $8, 12, 13$.\\
Wielomian jest stopnia nieparzystego, ponadto znak współczynnika przy\linebreak najwyższej potędze x jest dodatni.\\ W związku z tym wykres wielomianu zaczyna się od lewej strony poniżej osi OX. A więc $$x \in [8,12] \cup [13,\infty).$$
\rozwStop
\odpStart
$x \in [8,12] \cup [13,\infty)$
\odpStop
\testStart
A.$x \in [8,12] \cup [13,\infty)$\\
B.$x \in (8,12) \cup [13,\infty)$\\
C.$x \in (8,12] \cup [13,\infty)$\\
D.$x \in [8,12) \cup [13,\infty)$\\
E.$x \in [8,12] \cup (13,\infty)$\\
F.$x \in (8,12) \cup (13,\infty)$\\
G.$x \in [8,12) \cup (13,\infty)$\\
H.$x \in (8,12] \cup (13,\infty)$
\testStop
\kluczStart
A
\kluczStop



\zadStart{Zadanie z Wikieł Z 1.62 a) moja wersja nr 886}

Rozwiązać nierówności $(x-8)(x-12)(x-14)\ge0$.
\zadStop
\rozwStart{Patryk Wirkus}{}
Miejsca zerowe naszego wielomianu to: $8, 12, 14$.\\
Wielomian jest stopnia nieparzystego, ponadto znak współczynnika przy\linebreak najwyższej potędze x jest dodatni.\\ W związku z tym wykres wielomianu zaczyna się od lewej strony poniżej osi OX. A więc $$x \in [8,12] \cup [14,\infty).$$
\rozwStop
\odpStart
$x \in [8,12] \cup [14,\infty)$
\odpStop
\testStart
A.$x \in [8,12] \cup [14,\infty)$\\
B.$x \in (8,12) \cup [14,\infty)$\\
C.$x \in (8,12] \cup [14,\infty)$\\
D.$x \in [8,12) \cup [14,\infty)$\\
E.$x \in [8,12] \cup (14,\infty)$\\
F.$x \in (8,12) \cup (14,\infty)$\\
G.$x \in [8,12) \cup (14,\infty)$\\
H.$x \in (8,12] \cup (14,\infty)$
\testStop
\kluczStart
A
\kluczStop



\zadStart{Zadanie z Wikieł Z 1.62 a) moja wersja nr 887}

Rozwiązać nierówności $(x-8)(x-12)(x-15)\ge0$.
\zadStop
\rozwStart{Patryk Wirkus}{}
Miejsca zerowe naszego wielomianu to: $8, 12, 15$.\\
Wielomian jest stopnia nieparzystego, ponadto znak współczynnika przy\linebreak najwyższej potędze x jest dodatni.\\ W związku z tym wykres wielomianu zaczyna się od lewej strony poniżej osi OX. A więc $$x \in [8,12] \cup [15,\infty).$$
\rozwStop
\odpStart
$x \in [8,12] \cup [15,\infty)$
\odpStop
\testStart
A.$x \in [8,12] \cup [15,\infty)$\\
B.$x \in (8,12) \cup [15,\infty)$\\
C.$x \in (8,12] \cup [15,\infty)$\\
D.$x \in [8,12) \cup [15,\infty)$\\
E.$x \in [8,12] \cup (15,\infty)$\\
F.$x \in (8,12) \cup (15,\infty)$\\
G.$x \in [8,12) \cup (15,\infty)$\\
H.$x \in (8,12] \cup (15,\infty)$
\testStop
\kluczStart
A
\kluczStop



\zadStart{Zadanie z Wikieł Z 1.62 a) moja wersja nr 888}

Rozwiązać nierówności $(x-8)(x-12)(x-16)\ge0$.
\zadStop
\rozwStart{Patryk Wirkus}{}
Miejsca zerowe naszego wielomianu to: $8, 12, 16$.\\
Wielomian jest stopnia nieparzystego, ponadto znak współczynnika przy\linebreak najwyższej potędze x jest dodatni.\\ W związku z tym wykres wielomianu zaczyna się od lewej strony poniżej osi OX. A więc $$x \in [8,12] \cup [16,\infty).$$
\rozwStop
\odpStart
$x \in [8,12] \cup [16,\infty)$
\odpStop
\testStart
A.$x \in [8,12] \cup [16,\infty)$\\
B.$x \in (8,12) \cup [16,\infty)$\\
C.$x \in (8,12] \cup [16,\infty)$\\
D.$x \in [8,12) \cup [16,\infty)$\\
E.$x \in [8,12] \cup (16,\infty)$\\
F.$x \in (8,12) \cup (16,\infty)$\\
G.$x \in [8,12) \cup (16,\infty)$\\
H.$x \in (8,12] \cup (16,\infty)$
\testStop
\kluczStart
A
\kluczStop



\zadStart{Zadanie z Wikieł Z 1.62 a) moja wersja nr 889}

Rozwiązać nierówności $(x-8)(x-12)(x-17)\ge0$.
\zadStop
\rozwStart{Patryk Wirkus}{}
Miejsca zerowe naszego wielomianu to: $8, 12, 17$.\\
Wielomian jest stopnia nieparzystego, ponadto znak współczynnika przy\linebreak najwyższej potędze x jest dodatni.\\ W związku z tym wykres wielomianu zaczyna się od lewej strony poniżej osi OX. A więc $$x \in [8,12] \cup [17,\infty).$$
\rozwStop
\odpStart
$x \in [8,12] \cup [17,\infty)$
\odpStop
\testStart
A.$x \in [8,12] \cup [17,\infty)$\\
B.$x \in (8,12) \cup [17,\infty)$\\
C.$x \in (8,12] \cup [17,\infty)$\\
D.$x \in [8,12) \cup [17,\infty)$\\
E.$x \in [8,12] \cup (17,\infty)$\\
F.$x \in (8,12) \cup (17,\infty)$\\
G.$x \in [8,12) \cup (17,\infty)$\\
H.$x \in (8,12] \cup (17,\infty)$
\testStop
\kluczStart
A
\kluczStop



\zadStart{Zadanie z Wikieł Z 1.62 a) moja wersja nr 890}

Rozwiązać nierówności $(x-8)(x-12)(x-18)\ge0$.
\zadStop
\rozwStart{Patryk Wirkus}{}
Miejsca zerowe naszego wielomianu to: $8, 12, 18$.\\
Wielomian jest stopnia nieparzystego, ponadto znak współczynnika przy\linebreak najwyższej potędze x jest dodatni.\\ W związku z tym wykres wielomianu zaczyna się od lewej strony poniżej osi OX. A więc $$x \in [8,12] \cup [18,\infty).$$
\rozwStop
\odpStart
$x \in [8,12] \cup [18,\infty)$
\odpStop
\testStart
A.$x \in [8,12] \cup [18,\infty)$\\
B.$x \in (8,12) \cup [18,\infty)$\\
C.$x \in (8,12] \cup [18,\infty)$\\
D.$x \in [8,12) \cup [18,\infty)$\\
E.$x \in [8,12] \cup (18,\infty)$\\
F.$x \in (8,12) \cup (18,\infty)$\\
G.$x \in [8,12) \cup (18,\infty)$\\
H.$x \in (8,12] \cup (18,\infty)$
\testStop
\kluczStart
A
\kluczStop



\zadStart{Zadanie z Wikieł Z 1.62 a) moja wersja nr 891}

Rozwiązać nierówności $(x-8)(x-12)(x-19)\ge0$.
\zadStop
\rozwStart{Patryk Wirkus}{}
Miejsca zerowe naszego wielomianu to: $8, 12, 19$.\\
Wielomian jest stopnia nieparzystego, ponadto znak współczynnika przy\linebreak najwyższej potędze x jest dodatni.\\ W związku z tym wykres wielomianu zaczyna się od lewej strony poniżej osi OX. A więc $$x \in [8,12] \cup [19,\infty).$$
\rozwStop
\odpStart
$x \in [8,12] \cup [19,\infty)$
\odpStop
\testStart
A.$x \in [8,12] \cup [19,\infty)$\\
B.$x \in (8,12) \cup [19,\infty)$\\
C.$x \in (8,12] \cup [19,\infty)$\\
D.$x \in [8,12) \cup [19,\infty)$\\
E.$x \in [8,12] \cup (19,\infty)$\\
F.$x \in (8,12) \cup (19,\infty)$\\
G.$x \in [8,12) \cup (19,\infty)$\\
H.$x \in (8,12] \cup (19,\infty)$
\testStop
\kluczStart
A
\kluczStop



\zadStart{Zadanie z Wikieł Z 1.62 a) moja wersja nr 892}

Rozwiązać nierówności $(x-8)(x-12)(x-20)\ge0$.
\zadStop
\rozwStart{Patryk Wirkus}{}
Miejsca zerowe naszego wielomianu to: $8, 12, 20$.\\
Wielomian jest stopnia nieparzystego, ponadto znak współczynnika przy\linebreak najwyższej potędze x jest dodatni.\\ W związku z tym wykres wielomianu zaczyna się od lewej strony poniżej osi OX. A więc $$x \in [8,12] \cup [20,\infty).$$
\rozwStop
\odpStart
$x \in [8,12] \cup [20,\infty)$
\odpStop
\testStart
A.$x \in [8,12] \cup [20,\infty)$\\
B.$x \in (8,12) \cup [20,\infty)$\\
C.$x \in (8,12] \cup [20,\infty)$\\
D.$x \in [8,12) \cup [20,\infty)$\\
E.$x \in [8,12] \cup (20,\infty)$\\
F.$x \in (8,12) \cup (20,\infty)$\\
G.$x \in [8,12) \cup (20,\infty)$\\
H.$x \in (8,12] \cup (20,\infty)$
\testStop
\kluczStart
A
\kluczStop



\zadStart{Zadanie z Wikieł Z 1.62 a) moja wersja nr 893}

Rozwiązać nierówności $(x-8)(x-13)(x-14)\ge0$.
\zadStop
\rozwStart{Patryk Wirkus}{}
Miejsca zerowe naszego wielomianu to: $8, 13, 14$.\\
Wielomian jest stopnia nieparzystego, ponadto znak współczynnika przy\linebreak najwyższej potędze x jest dodatni.\\ W związku z tym wykres wielomianu zaczyna się od lewej strony poniżej osi OX. A więc $$x \in [8,13] \cup [14,\infty).$$
\rozwStop
\odpStart
$x \in [8,13] \cup [14,\infty)$
\odpStop
\testStart
A.$x \in [8,13] \cup [14,\infty)$\\
B.$x \in (8,13) \cup [14,\infty)$\\
C.$x \in (8,13] \cup [14,\infty)$\\
D.$x \in [8,13) \cup [14,\infty)$\\
E.$x \in [8,13] \cup (14,\infty)$\\
F.$x \in (8,13) \cup (14,\infty)$\\
G.$x \in [8,13) \cup (14,\infty)$\\
H.$x \in (8,13] \cup (14,\infty)$
\testStop
\kluczStart
A
\kluczStop



\zadStart{Zadanie z Wikieł Z 1.62 a) moja wersja nr 894}

Rozwiązać nierówności $(x-8)(x-13)(x-15)\ge0$.
\zadStop
\rozwStart{Patryk Wirkus}{}
Miejsca zerowe naszego wielomianu to: $8, 13, 15$.\\
Wielomian jest stopnia nieparzystego, ponadto znak współczynnika przy\linebreak najwyższej potędze x jest dodatni.\\ W związku z tym wykres wielomianu zaczyna się od lewej strony poniżej osi OX. A więc $$x \in [8,13] \cup [15,\infty).$$
\rozwStop
\odpStart
$x \in [8,13] \cup [15,\infty)$
\odpStop
\testStart
A.$x \in [8,13] \cup [15,\infty)$\\
B.$x \in (8,13) \cup [15,\infty)$\\
C.$x \in (8,13] \cup [15,\infty)$\\
D.$x \in [8,13) \cup [15,\infty)$\\
E.$x \in [8,13] \cup (15,\infty)$\\
F.$x \in (8,13) \cup (15,\infty)$\\
G.$x \in [8,13) \cup (15,\infty)$\\
H.$x \in (8,13] \cup (15,\infty)$
\testStop
\kluczStart
A
\kluczStop



\zadStart{Zadanie z Wikieł Z 1.62 a) moja wersja nr 895}

Rozwiązać nierówności $(x-8)(x-13)(x-16)\ge0$.
\zadStop
\rozwStart{Patryk Wirkus}{}
Miejsca zerowe naszego wielomianu to: $8, 13, 16$.\\
Wielomian jest stopnia nieparzystego, ponadto znak współczynnika przy\linebreak najwyższej potędze x jest dodatni.\\ W związku z tym wykres wielomianu zaczyna się od lewej strony poniżej osi OX. A więc $$x \in [8,13] \cup [16,\infty).$$
\rozwStop
\odpStart
$x \in [8,13] \cup [16,\infty)$
\odpStop
\testStart
A.$x \in [8,13] \cup [16,\infty)$\\
B.$x \in (8,13) \cup [16,\infty)$\\
C.$x \in (8,13] \cup [16,\infty)$\\
D.$x \in [8,13) \cup [16,\infty)$\\
E.$x \in [8,13] \cup (16,\infty)$\\
F.$x \in (8,13) \cup (16,\infty)$\\
G.$x \in [8,13) \cup (16,\infty)$\\
H.$x \in (8,13] \cup (16,\infty)$
\testStop
\kluczStart
A
\kluczStop



\zadStart{Zadanie z Wikieł Z 1.62 a) moja wersja nr 896}

Rozwiązać nierówności $(x-8)(x-13)(x-17)\ge0$.
\zadStop
\rozwStart{Patryk Wirkus}{}
Miejsca zerowe naszego wielomianu to: $8, 13, 17$.\\
Wielomian jest stopnia nieparzystego, ponadto znak współczynnika przy\linebreak najwyższej potędze x jest dodatni.\\ W związku z tym wykres wielomianu zaczyna się od lewej strony poniżej osi OX. A więc $$x \in [8,13] \cup [17,\infty).$$
\rozwStop
\odpStart
$x \in [8,13] \cup [17,\infty)$
\odpStop
\testStart
A.$x \in [8,13] \cup [17,\infty)$\\
B.$x \in (8,13) \cup [17,\infty)$\\
C.$x \in (8,13] \cup [17,\infty)$\\
D.$x \in [8,13) \cup [17,\infty)$\\
E.$x \in [8,13] \cup (17,\infty)$\\
F.$x \in (8,13) \cup (17,\infty)$\\
G.$x \in [8,13) \cup (17,\infty)$\\
H.$x \in (8,13] \cup (17,\infty)$
\testStop
\kluczStart
A
\kluczStop



\zadStart{Zadanie z Wikieł Z 1.62 a) moja wersja nr 897}

Rozwiązać nierówności $(x-8)(x-13)(x-18)\ge0$.
\zadStop
\rozwStart{Patryk Wirkus}{}
Miejsca zerowe naszego wielomianu to: $8, 13, 18$.\\
Wielomian jest stopnia nieparzystego, ponadto znak współczynnika przy\linebreak najwyższej potędze x jest dodatni.\\ W związku z tym wykres wielomianu zaczyna się od lewej strony poniżej osi OX. A więc $$x \in [8,13] \cup [18,\infty).$$
\rozwStop
\odpStart
$x \in [8,13] \cup [18,\infty)$
\odpStop
\testStart
A.$x \in [8,13] \cup [18,\infty)$\\
B.$x \in (8,13) \cup [18,\infty)$\\
C.$x \in (8,13] \cup [18,\infty)$\\
D.$x \in [8,13) \cup [18,\infty)$\\
E.$x \in [8,13] \cup (18,\infty)$\\
F.$x \in (8,13) \cup (18,\infty)$\\
G.$x \in [8,13) \cup (18,\infty)$\\
H.$x \in (8,13] \cup (18,\infty)$
\testStop
\kluczStart
A
\kluczStop



\zadStart{Zadanie z Wikieł Z 1.62 a) moja wersja nr 898}

Rozwiązać nierówności $(x-8)(x-13)(x-19)\ge0$.
\zadStop
\rozwStart{Patryk Wirkus}{}
Miejsca zerowe naszego wielomianu to: $8, 13, 19$.\\
Wielomian jest stopnia nieparzystego, ponadto znak współczynnika przy\linebreak najwyższej potędze x jest dodatni.\\ W związku z tym wykres wielomianu zaczyna się od lewej strony poniżej osi OX. A więc $$x \in [8,13] \cup [19,\infty).$$
\rozwStop
\odpStart
$x \in [8,13] \cup [19,\infty)$
\odpStop
\testStart
A.$x \in [8,13] \cup [19,\infty)$\\
B.$x \in (8,13) \cup [19,\infty)$\\
C.$x \in (8,13] \cup [19,\infty)$\\
D.$x \in [8,13) \cup [19,\infty)$\\
E.$x \in [8,13] \cup (19,\infty)$\\
F.$x \in (8,13) \cup (19,\infty)$\\
G.$x \in [8,13) \cup (19,\infty)$\\
H.$x \in (8,13] \cup (19,\infty)$
\testStop
\kluczStart
A
\kluczStop



\zadStart{Zadanie z Wikieł Z 1.62 a) moja wersja nr 899}

Rozwiązać nierówności $(x-8)(x-13)(x-20)\ge0$.
\zadStop
\rozwStart{Patryk Wirkus}{}
Miejsca zerowe naszego wielomianu to: $8, 13, 20$.\\
Wielomian jest stopnia nieparzystego, ponadto znak współczynnika przy\linebreak najwyższej potędze x jest dodatni.\\ W związku z tym wykres wielomianu zaczyna się od lewej strony poniżej osi OX. A więc $$x \in [8,13] \cup [20,\infty).$$
\rozwStop
\odpStart
$x \in [8,13] \cup [20,\infty)$
\odpStop
\testStart
A.$x \in [8,13] \cup [20,\infty)$\\
B.$x \in (8,13) \cup [20,\infty)$\\
C.$x \in (8,13] \cup [20,\infty)$\\
D.$x \in [8,13) \cup [20,\infty)$\\
E.$x \in [8,13] \cup (20,\infty)$\\
F.$x \in (8,13) \cup (20,\infty)$\\
G.$x \in [8,13) \cup (20,\infty)$\\
H.$x \in (8,13] \cup (20,\infty)$
\testStop
\kluczStart
A
\kluczStop



\zadStart{Zadanie z Wikieł Z 1.62 a) moja wersja nr 900}

Rozwiązać nierówności $(x-8)(x-14)(x-15)\ge0$.
\zadStop
\rozwStart{Patryk Wirkus}{}
Miejsca zerowe naszego wielomianu to: $8, 14, 15$.\\
Wielomian jest stopnia nieparzystego, ponadto znak współczynnika przy\linebreak najwyższej potędze x jest dodatni.\\ W związku z tym wykres wielomianu zaczyna się od lewej strony poniżej osi OX. A więc $$x \in [8,14] \cup [15,\infty).$$
\rozwStop
\odpStart
$x \in [8,14] \cup [15,\infty)$
\odpStop
\testStart
A.$x \in [8,14] \cup [15,\infty)$\\
B.$x \in (8,14) \cup [15,\infty)$\\
C.$x \in (8,14] \cup [15,\infty)$\\
D.$x \in [8,14) \cup [15,\infty)$\\
E.$x \in [8,14] \cup (15,\infty)$\\
F.$x \in (8,14) \cup (15,\infty)$\\
G.$x \in [8,14) \cup (15,\infty)$\\
H.$x \in (8,14] \cup (15,\infty)$
\testStop
\kluczStart
A
\kluczStop



\zadStart{Zadanie z Wikieł Z 1.62 a) moja wersja nr 901}

Rozwiązać nierówności $(x-8)(x-14)(x-16)\ge0$.
\zadStop
\rozwStart{Patryk Wirkus}{}
Miejsca zerowe naszego wielomianu to: $8, 14, 16$.\\
Wielomian jest stopnia nieparzystego, ponadto znak współczynnika przy\linebreak najwyższej potędze x jest dodatni.\\ W związku z tym wykres wielomianu zaczyna się od lewej strony poniżej osi OX. A więc $$x \in [8,14] \cup [16,\infty).$$
\rozwStop
\odpStart
$x \in [8,14] \cup [16,\infty)$
\odpStop
\testStart
A.$x \in [8,14] \cup [16,\infty)$\\
B.$x \in (8,14) \cup [16,\infty)$\\
C.$x \in (8,14] \cup [16,\infty)$\\
D.$x \in [8,14) \cup [16,\infty)$\\
E.$x \in [8,14] \cup (16,\infty)$\\
F.$x \in (8,14) \cup (16,\infty)$\\
G.$x \in [8,14) \cup (16,\infty)$\\
H.$x \in (8,14] \cup (16,\infty)$
\testStop
\kluczStart
A
\kluczStop



\zadStart{Zadanie z Wikieł Z 1.62 a) moja wersja nr 902}

Rozwiązać nierówności $(x-8)(x-14)(x-17)\ge0$.
\zadStop
\rozwStart{Patryk Wirkus}{}
Miejsca zerowe naszego wielomianu to: $8, 14, 17$.\\
Wielomian jest stopnia nieparzystego, ponadto znak współczynnika przy\linebreak najwyższej potędze x jest dodatni.\\ W związku z tym wykres wielomianu zaczyna się od lewej strony poniżej osi OX. A więc $$x \in [8,14] \cup [17,\infty).$$
\rozwStop
\odpStart
$x \in [8,14] \cup [17,\infty)$
\odpStop
\testStart
A.$x \in [8,14] \cup [17,\infty)$\\
B.$x \in (8,14) \cup [17,\infty)$\\
C.$x \in (8,14] \cup [17,\infty)$\\
D.$x \in [8,14) \cup [17,\infty)$\\
E.$x \in [8,14] \cup (17,\infty)$\\
F.$x \in (8,14) \cup (17,\infty)$\\
G.$x \in [8,14) \cup (17,\infty)$\\
H.$x \in (8,14] \cup (17,\infty)$
\testStop
\kluczStart
A
\kluczStop



\zadStart{Zadanie z Wikieł Z 1.62 a) moja wersja nr 903}

Rozwiązać nierówności $(x-8)(x-14)(x-18)\ge0$.
\zadStop
\rozwStart{Patryk Wirkus}{}
Miejsca zerowe naszego wielomianu to: $8, 14, 18$.\\
Wielomian jest stopnia nieparzystego, ponadto znak współczynnika przy\linebreak najwyższej potędze x jest dodatni.\\ W związku z tym wykres wielomianu zaczyna się od lewej strony poniżej osi OX. A więc $$x \in [8,14] \cup [18,\infty).$$
\rozwStop
\odpStart
$x \in [8,14] \cup [18,\infty)$
\odpStop
\testStart
A.$x \in [8,14] \cup [18,\infty)$\\
B.$x \in (8,14) \cup [18,\infty)$\\
C.$x \in (8,14] \cup [18,\infty)$\\
D.$x \in [8,14) \cup [18,\infty)$\\
E.$x \in [8,14] \cup (18,\infty)$\\
F.$x \in (8,14) \cup (18,\infty)$\\
G.$x \in [8,14) \cup (18,\infty)$\\
H.$x \in (8,14] \cup (18,\infty)$
\testStop
\kluczStart
A
\kluczStop



\zadStart{Zadanie z Wikieł Z 1.62 a) moja wersja nr 904}

Rozwiązać nierówności $(x-8)(x-14)(x-19)\ge0$.
\zadStop
\rozwStart{Patryk Wirkus}{}
Miejsca zerowe naszego wielomianu to: $8, 14, 19$.\\
Wielomian jest stopnia nieparzystego, ponadto znak współczynnika przy\linebreak najwyższej potędze x jest dodatni.\\ W związku z tym wykres wielomianu zaczyna się od lewej strony poniżej osi OX. A więc $$x \in [8,14] \cup [19,\infty).$$
\rozwStop
\odpStart
$x \in [8,14] \cup [19,\infty)$
\odpStop
\testStart
A.$x \in [8,14] \cup [19,\infty)$\\
B.$x \in (8,14) \cup [19,\infty)$\\
C.$x \in (8,14] \cup [19,\infty)$\\
D.$x \in [8,14) \cup [19,\infty)$\\
E.$x \in [8,14] \cup (19,\infty)$\\
F.$x \in (8,14) \cup (19,\infty)$\\
G.$x \in [8,14) \cup (19,\infty)$\\
H.$x \in (8,14] \cup (19,\infty)$
\testStop
\kluczStart
A
\kluczStop



\zadStart{Zadanie z Wikieł Z 1.62 a) moja wersja nr 905}

Rozwiązać nierówności $(x-8)(x-14)(x-20)\ge0$.
\zadStop
\rozwStart{Patryk Wirkus}{}
Miejsca zerowe naszego wielomianu to: $8, 14, 20$.\\
Wielomian jest stopnia nieparzystego, ponadto znak współczynnika przy\linebreak najwyższej potędze x jest dodatni.\\ W związku z tym wykres wielomianu zaczyna się od lewej strony poniżej osi OX. A więc $$x \in [8,14] \cup [20,\infty).$$
\rozwStop
\odpStart
$x \in [8,14] \cup [20,\infty)$
\odpStop
\testStart
A.$x \in [8,14] \cup [20,\infty)$\\
B.$x \in (8,14) \cup [20,\infty)$\\
C.$x \in (8,14] \cup [20,\infty)$\\
D.$x \in [8,14) \cup [20,\infty)$\\
E.$x \in [8,14] \cup (20,\infty)$\\
F.$x \in (8,14) \cup (20,\infty)$\\
G.$x \in [8,14) \cup (20,\infty)$\\
H.$x \in (8,14] \cup (20,\infty)$
\testStop
\kluczStart
A
\kluczStop



\zadStart{Zadanie z Wikieł Z 1.62 a) moja wersja nr 906}

Rozwiązać nierówności $(x-8)(x-15)(x-16)\ge0$.
\zadStop
\rozwStart{Patryk Wirkus}{}
Miejsca zerowe naszego wielomianu to: $8, 15, 16$.\\
Wielomian jest stopnia nieparzystego, ponadto znak współczynnika przy\linebreak najwyższej potędze x jest dodatni.\\ W związku z tym wykres wielomianu zaczyna się od lewej strony poniżej osi OX. A więc $$x \in [8,15] \cup [16,\infty).$$
\rozwStop
\odpStart
$x \in [8,15] \cup [16,\infty)$
\odpStop
\testStart
A.$x \in [8,15] \cup [16,\infty)$\\
B.$x \in (8,15) \cup [16,\infty)$\\
C.$x \in (8,15] \cup [16,\infty)$\\
D.$x \in [8,15) \cup [16,\infty)$\\
E.$x \in [8,15] \cup (16,\infty)$\\
F.$x \in (8,15) \cup (16,\infty)$\\
G.$x \in [8,15) \cup (16,\infty)$\\
H.$x \in (8,15] \cup (16,\infty)$
\testStop
\kluczStart
A
\kluczStop



\zadStart{Zadanie z Wikieł Z 1.62 a) moja wersja nr 907}

Rozwiązać nierówności $(x-8)(x-15)(x-17)\ge0$.
\zadStop
\rozwStart{Patryk Wirkus}{}
Miejsca zerowe naszego wielomianu to: $8, 15, 17$.\\
Wielomian jest stopnia nieparzystego, ponadto znak współczynnika przy\linebreak najwyższej potędze x jest dodatni.\\ W związku z tym wykres wielomianu zaczyna się od lewej strony poniżej osi OX. A więc $$x \in [8,15] \cup [17,\infty).$$
\rozwStop
\odpStart
$x \in [8,15] \cup [17,\infty)$
\odpStop
\testStart
A.$x \in [8,15] \cup [17,\infty)$\\
B.$x \in (8,15) \cup [17,\infty)$\\
C.$x \in (8,15] \cup [17,\infty)$\\
D.$x \in [8,15) \cup [17,\infty)$\\
E.$x \in [8,15] \cup (17,\infty)$\\
F.$x \in (8,15) \cup (17,\infty)$\\
G.$x \in [8,15) \cup (17,\infty)$\\
H.$x \in (8,15] \cup (17,\infty)$
\testStop
\kluczStart
A
\kluczStop



\zadStart{Zadanie z Wikieł Z 1.62 a) moja wersja nr 908}

Rozwiązać nierówności $(x-8)(x-15)(x-18)\ge0$.
\zadStop
\rozwStart{Patryk Wirkus}{}
Miejsca zerowe naszego wielomianu to: $8, 15, 18$.\\
Wielomian jest stopnia nieparzystego, ponadto znak współczynnika przy\linebreak najwyższej potędze x jest dodatni.\\ W związku z tym wykres wielomianu zaczyna się od lewej strony poniżej osi OX. A więc $$x \in [8,15] \cup [18,\infty).$$
\rozwStop
\odpStart
$x \in [8,15] \cup [18,\infty)$
\odpStop
\testStart
A.$x \in [8,15] \cup [18,\infty)$\\
B.$x \in (8,15) \cup [18,\infty)$\\
C.$x \in (8,15] \cup [18,\infty)$\\
D.$x \in [8,15) \cup [18,\infty)$\\
E.$x \in [8,15] \cup (18,\infty)$\\
F.$x \in (8,15) \cup (18,\infty)$\\
G.$x \in [8,15) \cup (18,\infty)$\\
H.$x \in (8,15] \cup (18,\infty)$
\testStop
\kluczStart
A
\kluczStop



\zadStart{Zadanie z Wikieł Z 1.62 a) moja wersja nr 909}

Rozwiązać nierówności $(x-8)(x-15)(x-19)\ge0$.
\zadStop
\rozwStart{Patryk Wirkus}{}
Miejsca zerowe naszego wielomianu to: $8, 15, 19$.\\
Wielomian jest stopnia nieparzystego, ponadto znak współczynnika przy\linebreak najwyższej potędze x jest dodatni.\\ W związku z tym wykres wielomianu zaczyna się od lewej strony poniżej osi OX. A więc $$x \in [8,15] \cup [19,\infty).$$
\rozwStop
\odpStart
$x \in [8,15] \cup [19,\infty)$
\odpStop
\testStart
A.$x \in [8,15] \cup [19,\infty)$\\
B.$x \in (8,15) \cup [19,\infty)$\\
C.$x \in (8,15] \cup [19,\infty)$\\
D.$x \in [8,15) \cup [19,\infty)$\\
E.$x \in [8,15] \cup (19,\infty)$\\
F.$x \in (8,15) \cup (19,\infty)$\\
G.$x \in [8,15) \cup (19,\infty)$\\
H.$x \in (8,15] \cup (19,\infty)$
\testStop
\kluczStart
A
\kluczStop



\zadStart{Zadanie z Wikieł Z 1.62 a) moja wersja nr 910}

Rozwiązać nierówności $(x-8)(x-15)(x-20)\ge0$.
\zadStop
\rozwStart{Patryk Wirkus}{}
Miejsca zerowe naszego wielomianu to: $8, 15, 20$.\\
Wielomian jest stopnia nieparzystego, ponadto znak współczynnika przy\linebreak najwyższej potędze x jest dodatni.\\ W związku z tym wykres wielomianu zaczyna się od lewej strony poniżej osi OX. A więc $$x \in [8,15] \cup [20,\infty).$$
\rozwStop
\odpStart
$x \in [8,15] \cup [20,\infty)$
\odpStop
\testStart
A.$x \in [8,15] \cup [20,\infty)$\\
B.$x \in (8,15) \cup [20,\infty)$\\
C.$x \in (8,15] \cup [20,\infty)$\\
D.$x \in [8,15) \cup [20,\infty)$\\
E.$x \in [8,15] \cup (20,\infty)$\\
F.$x \in (8,15) \cup (20,\infty)$\\
G.$x \in [8,15) \cup (20,\infty)$\\
H.$x \in (8,15] \cup (20,\infty)$
\testStop
\kluczStart
A
\kluczStop



\zadStart{Zadanie z Wikieł Z 1.62 a) moja wersja nr 911}

Rozwiązać nierówności $(x-8)(x-16)(x-17)\ge0$.
\zadStop
\rozwStart{Patryk Wirkus}{}
Miejsca zerowe naszego wielomianu to: $8, 16, 17$.\\
Wielomian jest stopnia nieparzystego, ponadto znak współczynnika przy\linebreak najwyższej potędze x jest dodatni.\\ W związku z tym wykres wielomianu zaczyna się od lewej strony poniżej osi OX. A więc $$x \in [8,16] \cup [17,\infty).$$
\rozwStop
\odpStart
$x \in [8,16] \cup [17,\infty)$
\odpStop
\testStart
A.$x \in [8,16] \cup [17,\infty)$\\
B.$x \in (8,16) \cup [17,\infty)$\\
C.$x \in (8,16] \cup [17,\infty)$\\
D.$x \in [8,16) \cup [17,\infty)$\\
E.$x \in [8,16] \cup (17,\infty)$\\
F.$x \in (8,16) \cup (17,\infty)$\\
G.$x \in [8,16) \cup (17,\infty)$\\
H.$x \in (8,16] \cup (17,\infty)$
\testStop
\kluczStart
A
\kluczStop



\zadStart{Zadanie z Wikieł Z 1.62 a) moja wersja nr 912}

Rozwiązać nierówności $(x-8)(x-16)(x-18)\ge0$.
\zadStop
\rozwStart{Patryk Wirkus}{}
Miejsca zerowe naszego wielomianu to: $8, 16, 18$.\\
Wielomian jest stopnia nieparzystego, ponadto znak współczynnika przy\linebreak najwyższej potędze x jest dodatni.\\ W związku z tym wykres wielomianu zaczyna się od lewej strony poniżej osi OX. A więc $$x \in [8,16] \cup [18,\infty).$$
\rozwStop
\odpStart
$x \in [8,16] \cup [18,\infty)$
\odpStop
\testStart
A.$x \in [8,16] \cup [18,\infty)$\\
B.$x \in (8,16) \cup [18,\infty)$\\
C.$x \in (8,16] \cup [18,\infty)$\\
D.$x \in [8,16) \cup [18,\infty)$\\
E.$x \in [8,16] \cup (18,\infty)$\\
F.$x \in (8,16) \cup (18,\infty)$\\
G.$x \in [8,16) \cup (18,\infty)$\\
H.$x \in (8,16] \cup (18,\infty)$
\testStop
\kluczStart
A
\kluczStop



\zadStart{Zadanie z Wikieł Z 1.62 a) moja wersja nr 913}

Rozwiązać nierówności $(x-8)(x-16)(x-19)\ge0$.
\zadStop
\rozwStart{Patryk Wirkus}{}
Miejsca zerowe naszego wielomianu to: $8, 16, 19$.\\
Wielomian jest stopnia nieparzystego, ponadto znak współczynnika przy\linebreak najwyższej potędze x jest dodatni.\\ W związku z tym wykres wielomianu zaczyna się od lewej strony poniżej osi OX. A więc $$x \in [8,16] \cup [19,\infty).$$
\rozwStop
\odpStart
$x \in [8,16] \cup [19,\infty)$
\odpStop
\testStart
A.$x \in [8,16] \cup [19,\infty)$\\
B.$x \in (8,16) \cup [19,\infty)$\\
C.$x \in (8,16] \cup [19,\infty)$\\
D.$x \in [8,16) \cup [19,\infty)$\\
E.$x \in [8,16] \cup (19,\infty)$\\
F.$x \in (8,16) \cup (19,\infty)$\\
G.$x \in [8,16) \cup (19,\infty)$\\
H.$x \in (8,16] \cup (19,\infty)$
\testStop
\kluczStart
A
\kluczStop



\zadStart{Zadanie z Wikieł Z 1.62 a) moja wersja nr 914}

Rozwiązać nierówności $(x-8)(x-16)(x-20)\ge0$.
\zadStop
\rozwStart{Patryk Wirkus}{}
Miejsca zerowe naszego wielomianu to: $8, 16, 20$.\\
Wielomian jest stopnia nieparzystego, ponadto znak współczynnika przy\linebreak najwyższej potędze x jest dodatni.\\ W związku z tym wykres wielomianu zaczyna się od lewej strony poniżej osi OX. A więc $$x \in [8,16] \cup [20,\infty).$$
\rozwStop
\odpStart
$x \in [8,16] \cup [20,\infty)$
\odpStop
\testStart
A.$x \in [8,16] \cup [20,\infty)$\\
B.$x \in (8,16) \cup [20,\infty)$\\
C.$x \in (8,16] \cup [20,\infty)$\\
D.$x \in [8,16) \cup [20,\infty)$\\
E.$x \in [8,16] \cup (20,\infty)$\\
F.$x \in (8,16) \cup (20,\infty)$\\
G.$x \in [8,16) \cup (20,\infty)$\\
H.$x \in (8,16] \cup (20,\infty)$
\testStop
\kluczStart
A
\kluczStop



\zadStart{Zadanie z Wikieł Z 1.62 a) moja wersja nr 915}

Rozwiązać nierówności $(x-8)(x-17)(x-18)\ge0$.
\zadStop
\rozwStart{Patryk Wirkus}{}
Miejsca zerowe naszego wielomianu to: $8, 17, 18$.\\
Wielomian jest stopnia nieparzystego, ponadto znak współczynnika przy\linebreak najwyższej potędze x jest dodatni.\\ W związku z tym wykres wielomianu zaczyna się od lewej strony poniżej osi OX. A więc $$x \in [8,17] \cup [18,\infty).$$
\rozwStop
\odpStart
$x \in [8,17] \cup [18,\infty)$
\odpStop
\testStart
A.$x \in [8,17] \cup [18,\infty)$\\
B.$x \in (8,17) \cup [18,\infty)$\\
C.$x \in (8,17] \cup [18,\infty)$\\
D.$x \in [8,17) \cup [18,\infty)$\\
E.$x \in [8,17] \cup (18,\infty)$\\
F.$x \in (8,17) \cup (18,\infty)$\\
G.$x \in [8,17) \cup (18,\infty)$\\
H.$x \in (8,17] \cup (18,\infty)$
\testStop
\kluczStart
A
\kluczStop



\zadStart{Zadanie z Wikieł Z 1.62 a) moja wersja nr 916}

Rozwiązać nierówności $(x-8)(x-17)(x-19)\ge0$.
\zadStop
\rozwStart{Patryk Wirkus}{}
Miejsca zerowe naszego wielomianu to: $8, 17, 19$.\\
Wielomian jest stopnia nieparzystego, ponadto znak współczynnika przy\linebreak najwyższej potędze x jest dodatni.\\ W związku z tym wykres wielomianu zaczyna się od lewej strony poniżej osi OX. A więc $$x \in [8,17] \cup [19,\infty).$$
\rozwStop
\odpStart
$x \in [8,17] \cup [19,\infty)$
\odpStop
\testStart
A.$x \in [8,17] \cup [19,\infty)$\\
B.$x \in (8,17) \cup [19,\infty)$\\
C.$x \in (8,17] \cup [19,\infty)$\\
D.$x \in [8,17) \cup [19,\infty)$\\
E.$x \in [8,17] \cup (19,\infty)$\\
F.$x \in (8,17) \cup (19,\infty)$\\
G.$x \in [8,17) \cup (19,\infty)$\\
H.$x \in (8,17] \cup (19,\infty)$
\testStop
\kluczStart
A
\kluczStop



\zadStart{Zadanie z Wikieł Z 1.62 a) moja wersja nr 917}

Rozwiązać nierówności $(x-8)(x-17)(x-20)\ge0$.
\zadStop
\rozwStart{Patryk Wirkus}{}
Miejsca zerowe naszego wielomianu to: $8, 17, 20$.\\
Wielomian jest stopnia nieparzystego, ponadto znak współczynnika przy\linebreak najwyższej potędze x jest dodatni.\\ W związku z tym wykres wielomianu zaczyna się od lewej strony poniżej osi OX. A więc $$x \in [8,17] \cup [20,\infty).$$
\rozwStop
\odpStart
$x \in [8,17] \cup [20,\infty)$
\odpStop
\testStart
A.$x \in [8,17] \cup [20,\infty)$\\
B.$x \in (8,17) \cup [20,\infty)$\\
C.$x \in (8,17] \cup [20,\infty)$\\
D.$x \in [8,17) \cup [20,\infty)$\\
E.$x \in [8,17] \cup (20,\infty)$\\
F.$x \in (8,17) \cup (20,\infty)$\\
G.$x \in [8,17) \cup (20,\infty)$\\
H.$x \in (8,17] \cup (20,\infty)$
\testStop
\kluczStart
A
\kluczStop



\zadStart{Zadanie z Wikieł Z 1.62 a) moja wersja nr 918}

Rozwiązać nierówności $(x-8)(x-18)(x-19)\ge0$.
\zadStop
\rozwStart{Patryk Wirkus}{}
Miejsca zerowe naszego wielomianu to: $8, 18, 19$.\\
Wielomian jest stopnia nieparzystego, ponadto znak współczynnika przy\linebreak najwyższej potędze x jest dodatni.\\ W związku z tym wykres wielomianu zaczyna się od lewej strony poniżej osi OX. A więc $$x \in [8,18] \cup [19,\infty).$$
\rozwStop
\odpStart
$x \in [8,18] \cup [19,\infty)$
\odpStop
\testStart
A.$x \in [8,18] \cup [19,\infty)$\\
B.$x \in (8,18) \cup [19,\infty)$\\
C.$x \in (8,18] \cup [19,\infty)$\\
D.$x \in [8,18) \cup [19,\infty)$\\
E.$x \in [8,18] \cup (19,\infty)$\\
F.$x \in (8,18) \cup (19,\infty)$\\
G.$x \in [8,18) \cup (19,\infty)$\\
H.$x \in (8,18] \cup (19,\infty)$
\testStop
\kluczStart
A
\kluczStop



\zadStart{Zadanie z Wikieł Z 1.62 a) moja wersja nr 919}

Rozwiązać nierówności $(x-8)(x-18)(x-20)\ge0$.
\zadStop
\rozwStart{Patryk Wirkus}{}
Miejsca zerowe naszego wielomianu to: $8, 18, 20$.\\
Wielomian jest stopnia nieparzystego, ponadto znak współczynnika przy\linebreak najwyższej potędze x jest dodatni.\\ W związku z tym wykres wielomianu zaczyna się od lewej strony poniżej osi OX. A więc $$x \in [8,18] \cup [20,\infty).$$
\rozwStop
\odpStart
$x \in [8,18] \cup [20,\infty)$
\odpStop
\testStart
A.$x \in [8,18] \cup [20,\infty)$\\
B.$x \in (8,18) \cup [20,\infty)$\\
C.$x \in (8,18] \cup [20,\infty)$\\
D.$x \in [8,18) \cup [20,\infty)$\\
E.$x \in [8,18] \cup (20,\infty)$\\
F.$x \in (8,18) \cup (20,\infty)$\\
G.$x \in [8,18) \cup (20,\infty)$\\
H.$x \in (8,18] \cup (20,\infty)$
\testStop
\kluczStart
A
\kluczStop



\zadStart{Zadanie z Wikieł Z 1.62 a) moja wersja nr 920}

Rozwiązać nierówności $(x-8)(x-19)(x-20)\ge0$.
\zadStop
\rozwStart{Patryk Wirkus}{}
Miejsca zerowe naszego wielomianu to: $8, 19, 20$.\\
Wielomian jest stopnia nieparzystego, ponadto znak współczynnika przy\linebreak najwyższej potędze x jest dodatni.\\ W związku z tym wykres wielomianu zaczyna się od lewej strony poniżej osi OX. A więc $$x \in [8,19] \cup [20,\infty).$$
\rozwStop
\odpStart
$x \in [8,19] \cup [20,\infty)$
\odpStop
\testStart
A.$x \in [8,19] \cup [20,\infty)$\\
B.$x \in (8,19) \cup [20,\infty)$\\
C.$x \in (8,19] \cup [20,\infty)$\\
D.$x \in [8,19) \cup [20,\infty)$\\
E.$x \in [8,19] \cup (20,\infty)$\\
F.$x \in (8,19) \cup (20,\infty)$\\
G.$x \in [8,19) \cup (20,\infty)$\\
H.$x \in (8,19] \cup (20,\infty)$
\testStop
\kluczStart
A
\kluczStop



\zadStart{Zadanie z Wikieł Z 1.62 a) moja wersja nr 921}

Rozwiązać nierówności $(x-9)(x-10)(x-11)\ge0$.
\zadStop
\rozwStart{Patryk Wirkus}{}
Miejsca zerowe naszego wielomianu to: $9, 10, 11$.\\
Wielomian jest stopnia nieparzystego, ponadto znak współczynnika przy\linebreak najwyższej potędze x jest dodatni.\\ W związku z tym wykres wielomianu zaczyna się od lewej strony poniżej osi OX. A więc $$x \in [9,10] \cup [11,\infty).$$
\rozwStop
\odpStart
$x \in [9,10] \cup [11,\infty)$
\odpStop
\testStart
A.$x \in [9,10] \cup [11,\infty)$\\
B.$x \in (9,10) \cup [11,\infty)$\\
C.$x \in (9,10] \cup [11,\infty)$\\
D.$x \in [9,10) \cup [11,\infty)$\\
E.$x \in [9,10] \cup (11,\infty)$\\
F.$x \in (9,10) \cup (11,\infty)$\\
G.$x \in [9,10) \cup (11,\infty)$\\
H.$x \in (9,10] \cup (11,\infty)$
\testStop
\kluczStart
A
\kluczStop



\zadStart{Zadanie z Wikieł Z 1.62 a) moja wersja nr 922}

Rozwiązać nierówności $(x-9)(x-10)(x-12)\ge0$.
\zadStop
\rozwStart{Patryk Wirkus}{}
Miejsca zerowe naszego wielomianu to: $9, 10, 12$.\\
Wielomian jest stopnia nieparzystego, ponadto znak współczynnika przy\linebreak najwyższej potędze x jest dodatni.\\ W związku z tym wykres wielomianu zaczyna się od lewej strony poniżej osi OX. A więc $$x \in [9,10] \cup [12,\infty).$$
\rozwStop
\odpStart
$x \in [9,10] \cup [12,\infty)$
\odpStop
\testStart
A.$x \in [9,10] \cup [12,\infty)$\\
B.$x \in (9,10) \cup [12,\infty)$\\
C.$x \in (9,10] \cup [12,\infty)$\\
D.$x \in [9,10) \cup [12,\infty)$\\
E.$x \in [9,10] \cup (12,\infty)$\\
F.$x \in (9,10) \cup (12,\infty)$\\
G.$x \in [9,10) \cup (12,\infty)$\\
H.$x \in (9,10] \cup (12,\infty)$
\testStop
\kluczStart
A
\kluczStop



\zadStart{Zadanie z Wikieł Z 1.62 a) moja wersja nr 923}

Rozwiązać nierówności $(x-9)(x-10)(x-13)\ge0$.
\zadStop
\rozwStart{Patryk Wirkus}{}
Miejsca zerowe naszego wielomianu to: $9, 10, 13$.\\
Wielomian jest stopnia nieparzystego, ponadto znak współczynnika przy\linebreak najwyższej potędze x jest dodatni.\\ W związku z tym wykres wielomianu zaczyna się od lewej strony poniżej osi OX. A więc $$x \in [9,10] \cup [13,\infty).$$
\rozwStop
\odpStart
$x \in [9,10] \cup [13,\infty)$
\odpStop
\testStart
A.$x \in [9,10] \cup [13,\infty)$\\
B.$x \in (9,10) \cup [13,\infty)$\\
C.$x \in (9,10] \cup [13,\infty)$\\
D.$x \in [9,10) \cup [13,\infty)$\\
E.$x \in [9,10] \cup (13,\infty)$\\
F.$x \in (9,10) \cup (13,\infty)$\\
G.$x \in [9,10) \cup (13,\infty)$\\
H.$x \in (9,10] \cup (13,\infty)$
\testStop
\kluczStart
A
\kluczStop



\zadStart{Zadanie z Wikieł Z 1.62 a) moja wersja nr 924}

Rozwiązać nierówności $(x-9)(x-10)(x-14)\ge0$.
\zadStop
\rozwStart{Patryk Wirkus}{}
Miejsca zerowe naszego wielomianu to: $9, 10, 14$.\\
Wielomian jest stopnia nieparzystego, ponadto znak współczynnika przy\linebreak najwyższej potędze x jest dodatni.\\ W związku z tym wykres wielomianu zaczyna się od lewej strony poniżej osi OX. A więc $$x \in [9,10] \cup [14,\infty).$$
\rozwStop
\odpStart
$x \in [9,10] \cup [14,\infty)$
\odpStop
\testStart
A.$x \in [9,10] \cup [14,\infty)$\\
B.$x \in (9,10) \cup [14,\infty)$\\
C.$x \in (9,10] \cup [14,\infty)$\\
D.$x \in [9,10) \cup [14,\infty)$\\
E.$x \in [9,10] \cup (14,\infty)$\\
F.$x \in (9,10) \cup (14,\infty)$\\
G.$x \in [9,10) \cup (14,\infty)$\\
H.$x \in (9,10] \cup (14,\infty)$
\testStop
\kluczStart
A
\kluczStop



\zadStart{Zadanie z Wikieł Z 1.62 a) moja wersja nr 925}

Rozwiązać nierówności $(x-9)(x-10)(x-15)\ge0$.
\zadStop
\rozwStart{Patryk Wirkus}{}
Miejsca zerowe naszego wielomianu to: $9, 10, 15$.\\
Wielomian jest stopnia nieparzystego, ponadto znak współczynnika przy\linebreak najwyższej potędze x jest dodatni.\\ W związku z tym wykres wielomianu zaczyna się od lewej strony poniżej osi OX. A więc $$x \in [9,10] \cup [15,\infty).$$
\rozwStop
\odpStart
$x \in [9,10] \cup [15,\infty)$
\odpStop
\testStart
A.$x \in [9,10] \cup [15,\infty)$\\
B.$x \in (9,10) \cup [15,\infty)$\\
C.$x \in (9,10] \cup [15,\infty)$\\
D.$x \in [9,10) \cup [15,\infty)$\\
E.$x \in [9,10] \cup (15,\infty)$\\
F.$x \in (9,10) \cup (15,\infty)$\\
G.$x \in [9,10) \cup (15,\infty)$\\
H.$x \in (9,10] \cup (15,\infty)$
\testStop
\kluczStart
A
\kluczStop



\zadStart{Zadanie z Wikieł Z 1.62 a) moja wersja nr 926}

Rozwiązać nierówności $(x-9)(x-10)(x-16)\ge0$.
\zadStop
\rozwStart{Patryk Wirkus}{}
Miejsca zerowe naszego wielomianu to: $9, 10, 16$.\\
Wielomian jest stopnia nieparzystego, ponadto znak współczynnika przy\linebreak najwyższej potędze x jest dodatni.\\ W związku z tym wykres wielomianu zaczyna się od lewej strony poniżej osi OX. A więc $$x \in [9,10] \cup [16,\infty).$$
\rozwStop
\odpStart
$x \in [9,10] \cup [16,\infty)$
\odpStop
\testStart
A.$x \in [9,10] \cup [16,\infty)$\\
B.$x \in (9,10) \cup [16,\infty)$\\
C.$x \in (9,10] \cup [16,\infty)$\\
D.$x \in [9,10) \cup [16,\infty)$\\
E.$x \in [9,10] \cup (16,\infty)$\\
F.$x \in (9,10) \cup (16,\infty)$\\
G.$x \in [9,10) \cup (16,\infty)$\\
H.$x \in (9,10] \cup (16,\infty)$
\testStop
\kluczStart
A
\kluczStop



\zadStart{Zadanie z Wikieł Z 1.62 a) moja wersja nr 927}

Rozwiązać nierówności $(x-9)(x-10)(x-17)\ge0$.
\zadStop
\rozwStart{Patryk Wirkus}{}
Miejsca zerowe naszego wielomianu to: $9, 10, 17$.\\
Wielomian jest stopnia nieparzystego, ponadto znak współczynnika przy\linebreak najwyższej potędze x jest dodatni.\\ W związku z tym wykres wielomianu zaczyna się od lewej strony poniżej osi OX. A więc $$x \in [9,10] \cup [17,\infty).$$
\rozwStop
\odpStart
$x \in [9,10] \cup [17,\infty)$
\odpStop
\testStart
A.$x \in [9,10] \cup [17,\infty)$\\
B.$x \in (9,10) \cup [17,\infty)$\\
C.$x \in (9,10] \cup [17,\infty)$\\
D.$x \in [9,10) \cup [17,\infty)$\\
E.$x \in [9,10] \cup (17,\infty)$\\
F.$x \in (9,10) \cup (17,\infty)$\\
G.$x \in [9,10) \cup (17,\infty)$\\
H.$x \in (9,10] \cup (17,\infty)$
\testStop
\kluczStart
A
\kluczStop



\zadStart{Zadanie z Wikieł Z 1.62 a) moja wersja nr 928}

Rozwiązać nierówności $(x-9)(x-10)(x-18)\ge0$.
\zadStop
\rozwStart{Patryk Wirkus}{}
Miejsca zerowe naszego wielomianu to: $9, 10, 18$.\\
Wielomian jest stopnia nieparzystego, ponadto znak współczynnika przy\linebreak najwyższej potędze x jest dodatni.\\ W związku z tym wykres wielomianu zaczyna się od lewej strony poniżej osi OX. A więc $$x \in [9,10] \cup [18,\infty).$$
\rozwStop
\odpStart
$x \in [9,10] \cup [18,\infty)$
\odpStop
\testStart
A.$x \in [9,10] \cup [18,\infty)$\\
B.$x \in (9,10) \cup [18,\infty)$\\
C.$x \in (9,10] \cup [18,\infty)$\\
D.$x \in [9,10) \cup [18,\infty)$\\
E.$x \in [9,10] \cup (18,\infty)$\\
F.$x \in (9,10) \cup (18,\infty)$\\
G.$x \in [9,10) \cup (18,\infty)$\\
H.$x \in (9,10] \cup (18,\infty)$
\testStop
\kluczStart
A
\kluczStop



\zadStart{Zadanie z Wikieł Z 1.62 a) moja wersja nr 929}

Rozwiązać nierówności $(x-9)(x-10)(x-19)\ge0$.
\zadStop
\rozwStart{Patryk Wirkus}{}
Miejsca zerowe naszego wielomianu to: $9, 10, 19$.\\
Wielomian jest stopnia nieparzystego, ponadto znak współczynnika przy\linebreak najwyższej potędze x jest dodatni.\\ W związku z tym wykres wielomianu zaczyna się od lewej strony poniżej osi OX. A więc $$x \in [9,10] \cup [19,\infty).$$
\rozwStop
\odpStart
$x \in [9,10] \cup [19,\infty)$
\odpStop
\testStart
A.$x \in [9,10] \cup [19,\infty)$\\
B.$x \in (9,10) \cup [19,\infty)$\\
C.$x \in (9,10] \cup [19,\infty)$\\
D.$x \in [9,10) \cup [19,\infty)$\\
E.$x \in [9,10] \cup (19,\infty)$\\
F.$x \in (9,10) \cup (19,\infty)$\\
G.$x \in [9,10) \cup (19,\infty)$\\
H.$x \in (9,10] \cup (19,\infty)$
\testStop
\kluczStart
A
\kluczStop



\zadStart{Zadanie z Wikieł Z 1.62 a) moja wersja nr 930}

Rozwiązać nierówności $(x-9)(x-10)(x-20)\ge0$.
\zadStop
\rozwStart{Patryk Wirkus}{}
Miejsca zerowe naszego wielomianu to: $9, 10, 20$.\\
Wielomian jest stopnia nieparzystego, ponadto znak współczynnika przy\linebreak najwyższej potędze x jest dodatni.\\ W związku z tym wykres wielomianu zaczyna się od lewej strony poniżej osi OX. A więc $$x \in [9,10] \cup [20,\infty).$$
\rozwStop
\odpStart
$x \in [9,10] \cup [20,\infty)$
\odpStop
\testStart
A.$x \in [9,10] \cup [20,\infty)$\\
B.$x \in (9,10) \cup [20,\infty)$\\
C.$x \in (9,10] \cup [20,\infty)$\\
D.$x \in [9,10) \cup [20,\infty)$\\
E.$x \in [9,10] \cup (20,\infty)$\\
F.$x \in (9,10) \cup (20,\infty)$\\
G.$x \in [9,10) \cup (20,\infty)$\\
H.$x \in (9,10] \cup (20,\infty)$
\testStop
\kluczStart
A
\kluczStop



\zadStart{Zadanie z Wikieł Z 1.62 a) moja wersja nr 931}

Rozwiązać nierówności $(x-9)(x-11)(x-12)\ge0$.
\zadStop
\rozwStart{Patryk Wirkus}{}
Miejsca zerowe naszego wielomianu to: $9, 11, 12$.\\
Wielomian jest stopnia nieparzystego, ponadto znak współczynnika przy\linebreak najwyższej potędze x jest dodatni.\\ W związku z tym wykres wielomianu zaczyna się od lewej strony poniżej osi OX. A więc $$x \in [9,11] \cup [12,\infty).$$
\rozwStop
\odpStart
$x \in [9,11] \cup [12,\infty)$
\odpStop
\testStart
A.$x \in [9,11] \cup [12,\infty)$\\
B.$x \in (9,11) \cup [12,\infty)$\\
C.$x \in (9,11] \cup [12,\infty)$\\
D.$x \in [9,11) \cup [12,\infty)$\\
E.$x \in [9,11] \cup (12,\infty)$\\
F.$x \in (9,11) \cup (12,\infty)$\\
G.$x \in [9,11) \cup (12,\infty)$\\
H.$x \in (9,11] \cup (12,\infty)$
\testStop
\kluczStart
A
\kluczStop



\zadStart{Zadanie z Wikieł Z 1.62 a) moja wersja nr 932}

Rozwiązać nierówności $(x-9)(x-11)(x-13)\ge0$.
\zadStop
\rozwStart{Patryk Wirkus}{}
Miejsca zerowe naszego wielomianu to: $9, 11, 13$.\\
Wielomian jest stopnia nieparzystego, ponadto znak współczynnika przy\linebreak najwyższej potędze x jest dodatni.\\ W związku z tym wykres wielomianu zaczyna się od lewej strony poniżej osi OX. A więc $$x \in [9,11] \cup [13,\infty).$$
\rozwStop
\odpStart
$x \in [9,11] \cup [13,\infty)$
\odpStop
\testStart
A.$x \in [9,11] \cup [13,\infty)$\\
B.$x \in (9,11) \cup [13,\infty)$\\
C.$x \in (9,11] \cup [13,\infty)$\\
D.$x \in [9,11) \cup [13,\infty)$\\
E.$x \in [9,11] \cup (13,\infty)$\\
F.$x \in (9,11) \cup (13,\infty)$\\
G.$x \in [9,11) \cup (13,\infty)$\\
H.$x \in (9,11] \cup (13,\infty)$
\testStop
\kluczStart
A
\kluczStop



\zadStart{Zadanie z Wikieł Z 1.62 a) moja wersja nr 933}

Rozwiązać nierówności $(x-9)(x-11)(x-14)\ge0$.
\zadStop
\rozwStart{Patryk Wirkus}{}
Miejsca zerowe naszego wielomianu to: $9, 11, 14$.\\
Wielomian jest stopnia nieparzystego, ponadto znak współczynnika przy\linebreak najwyższej potędze x jest dodatni.\\ W związku z tym wykres wielomianu zaczyna się od lewej strony poniżej osi OX. A więc $$x \in [9,11] \cup [14,\infty).$$
\rozwStop
\odpStart
$x \in [9,11] \cup [14,\infty)$
\odpStop
\testStart
A.$x \in [9,11] \cup [14,\infty)$\\
B.$x \in (9,11) \cup [14,\infty)$\\
C.$x \in (9,11] \cup [14,\infty)$\\
D.$x \in [9,11) \cup [14,\infty)$\\
E.$x \in [9,11] \cup (14,\infty)$\\
F.$x \in (9,11) \cup (14,\infty)$\\
G.$x \in [9,11) \cup (14,\infty)$\\
H.$x \in (9,11] \cup (14,\infty)$
\testStop
\kluczStart
A
\kluczStop



\zadStart{Zadanie z Wikieł Z 1.62 a) moja wersja nr 934}

Rozwiązać nierówności $(x-9)(x-11)(x-15)\ge0$.
\zadStop
\rozwStart{Patryk Wirkus}{}
Miejsca zerowe naszego wielomianu to: $9, 11, 15$.\\
Wielomian jest stopnia nieparzystego, ponadto znak współczynnika przy\linebreak najwyższej potędze x jest dodatni.\\ W związku z tym wykres wielomianu zaczyna się od lewej strony poniżej osi OX. A więc $$x \in [9,11] \cup [15,\infty).$$
\rozwStop
\odpStart
$x \in [9,11] \cup [15,\infty)$
\odpStop
\testStart
A.$x \in [9,11] \cup [15,\infty)$\\
B.$x \in (9,11) \cup [15,\infty)$\\
C.$x \in (9,11] \cup [15,\infty)$\\
D.$x \in [9,11) \cup [15,\infty)$\\
E.$x \in [9,11] \cup (15,\infty)$\\
F.$x \in (9,11) \cup (15,\infty)$\\
G.$x \in [9,11) \cup (15,\infty)$\\
H.$x \in (9,11] \cup (15,\infty)$
\testStop
\kluczStart
A
\kluczStop



\zadStart{Zadanie z Wikieł Z 1.62 a) moja wersja nr 935}

Rozwiązać nierówności $(x-9)(x-11)(x-16)\ge0$.
\zadStop
\rozwStart{Patryk Wirkus}{}
Miejsca zerowe naszego wielomianu to: $9, 11, 16$.\\
Wielomian jest stopnia nieparzystego, ponadto znak współczynnika przy\linebreak najwyższej potędze x jest dodatni.\\ W związku z tym wykres wielomianu zaczyna się od lewej strony poniżej osi OX. A więc $$x \in [9,11] \cup [16,\infty).$$
\rozwStop
\odpStart
$x \in [9,11] \cup [16,\infty)$
\odpStop
\testStart
A.$x \in [9,11] \cup [16,\infty)$\\
B.$x \in (9,11) \cup [16,\infty)$\\
C.$x \in (9,11] \cup [16,\infty)$\\
D.$x \in [9,11) \cup [16,\infty)$\\
E.$x \in [9,11] \cup (16,\infty)$\\
F.$x \in (9,11) \cup (16,\infty)$\\
G.$x \in [9,11) \cup (16,\infty)$\\
H.$x \in (9,11] \cup (16,\infty)$
\testStop
\kluczStart
A
\kluczStop



\zadStart{Zadanie z Wikieł Z 1.62 a) moja wersja nr 936}

Rozwiązać nierówności $(x-9)(x-11)(x-17)\ge0$.
\zadStop
\rozwStart{Patryk Wirkus}{}
Miejsca zerowe naszego wielomianu to: $9, 11, 17$.\\
Wielomian jest stopnia nieparzystego, ponadto znak współczynnika przy\linebreak najwyższej potędze x jest dodatni.\\ W związku z tym wykres wielomianu zaczyna się od lewej strony poniżej osi OX. A więc $$x \in [9,11] \cup [17,\infty).$$
\rozwStop
\odpStart
$x \in [9,11] \cup [17,\infty)$
\odpStop
\testStart
A.$x \in [9,11] \cup [17,\infty)$\\
B.$x \in (9,11) \cup [17,\infty)$\\
C.$x \in (9,11] \cup [17,\infty)$\\
D.$x \in [9,11) \cup [17,\infty)$\\
E.$x \in [9,11] \cup (17,\infty)$\\
F.$x \in (9,11) \cup (17,\infty)$\\
G.$x \in [9,11) \cup (17,\infty)$\\
H.$x \in (9,11] \cup (17,\infty)$
\testStop
\kluczStart
A
\kluczStop



\zadStart{Zadanie z Wikieł Z 1.62 a) moja wersja nr 937}

Rozwiązać nierówności $(x-9)(x-11)(x-18)\ge0$.
\zadStop
\rozwStart{Patryk Wirkus}{}
Miejsca zerowe naszego wielomianu to: $9, 11, 18$.\\
Wielomian jest stopnia nieparzystego, ponadto znak współczynnika przy\linebreak najwyższej potędze x jest dodatni.\\ W związku z tym wykres wielomianu zaczyna się od lewej strony poniżej osi OX. A więc $$x \in [9,11] \cup [18,\infty).$$
\rozwStop
\odpStart
$x \in [9,11] \cup [18,\infty)$
\odpStop
\testStart
A.$x \in [9,11] \cup [18,\infty)$\\
B.$x \in (9,11) \cup [18,\infty)$\\
C.$x \in (9,11] \cup [18,\infty)$\\
D.$x \in [9,11) \cup [18,\infty)$\\
E.$x \in [9,11] \cup (18,\infty)$\\
F.$x \in (9,11) \cup (18,\infty)$\\
G.$x \in [9,11) \cup (18,\infty)$\\
H.$x \in (9,11] \cup (18,\infty)$
\testStop
\kluczStart
A
\kluczStop



\zadStart{Zadanie z Wikieł Z 1.62 a) moja wersja nr 938}

Rozwiązać nierówności $(x-9)(x-11)(x-19)\ge0$.
\zadStop
\rozwStart{Patryk Wirkus}{}
Miejsca zerowe naszego wielomianu to: $9, 11, 19$.\\
Wielomian jest stopnia nieparzystego, ponadto znak współczynnika przy\linebreak najwyższej potędze x jest dodatni.\\ W związku z tym wykres wielomianu zaczyna się od lewej strony poniżej osi OX. A więc $$x \in [9,11] \cup [19,\infty).$$
\rozwStop
\odpStart
$x \in [9,11] \cup [19,\infty)$
\odpStop
\testStart
A.$x \in [9,11] \cup [19,\infty)$\\
B.$x \in (9,11) \cup [19,\infty)$\\
C.$x \in (9,11] \cup [19,\infty)$\\
D.$x \in [9,11) \cup [19,\infty)$\\
E.$x \in [9,11] \cup (19,\infty)$\\
F.$x \in (9,11) \cup (19,\infty)$\\
G.$x \in [9,11) \cup (19,\infty)$\\
H.$x \in (9,11] \cup (19,\infty)$
\testStop
\kluczStart
A
\kluczStop



\zadStart{Zadanie z Wikieł Z 1.62 a) moja wersja nr 939}

Rozwiązać nierówności $(x-9)(x-11)(x-20)\ge0$.
\zadStop
\rozwStart{Patryk Wirkus}{}
Miejsca zerowe naszego wielomianu to: $9, 11, 20$.\\
Wielomian jest stopnia nieparzystego, ponadto znak współczynnika przy\linebreak najwyższej potędze x jest dodatni.\\ W związku z tym wykres wielomianu zaczyna się od lewej strony poniżej osi OX. A więc $$x \in [9,11] \cup [20,\infty).$$
\rozwStop
\odpStart
$x \in [9,11] \cup [20,\infty)$
\odpStop
\testStart
A.$x \in [9,11] \cup [20,\infty)$\\
B.$x \in (9,11) \cup [20,\infty)$\\
C.$x \in (9,11] \cup [20,\infty)$\\
D.$x \in [9,11) \cup [20,\infty)$\\
E.$x \in [9,11] \cup (20,\infty)$\\
F.$x \in (9,11) \cup (20,\infty)$\\
G.$x \in [9,11) \cup (20,\infty)$\\
H.$x \in (9,11] \cup (20,\infty)$
\testStop
\kluczStart
A
\kluczStop



\zadStart{Zadanie z Wikieł Z 1.62 a) moja wersja nr 940}

Rozwiązać nierówności $(x-9)(x-12)(x-13)\ge0$.
\zadStop
\rozwStart{Patryk Wirkus}{}
Miejsca zerowe naszego wielomianu to: $9, 12, 13$.\\
Wielomian jest stopnia nieparzystego, ponadto znak współczynnika przy\linebreak najwyższej potędze x jest dodatni.\\ W związku z tym wykres wielomianu zaczyna się od lewej strony poniżej osi OX. A więc $$x \in [9,12] \cup [13,\infty).$$
\rozwStop
\odpStart
$x \in [9,12] \cup [13,\infty)$
\odpStop
\testStart
A.$x \in [9,12] \cup [13,\infty)$\\
B.$x \in (9,12) \cup [13,\infty)$\\
C.$x \in (9,12] \cup [13,\infty)$\\
D.$x \in [9,12) \cup [13,\infty)$\\
E.$x \in [9,12] \cup (13,\infty)$\\
F.$x \in (9,12) \cup (13,\infty)$\\
G.$x \in [9,12) \cup (13,\infty)$\\
H.$x \in (9,12] \cup (13,\infty)$
\testStop
\kluczStart
A
\kluczStop



\zadStart{Zadanie z Wikieł Z 1.62 a) moja wersja nr 941}

Rozwiązać nierówności $(x-9)(x-12)(x-14)\ge0$.
\zadStop
\rozwStart{Patryk Wirkus}{}
Miejsca zerowe naszego wielomianu to: $9, 12, 14$.\\
Wielomian jest stopnia nieparzystego, ponadto znak współczynnika przy\linebreak najwyższej potędze x jest dodatni.\\ W związku z tym wykres wielomianu zaczyna się od lewej strony poniżej osi OX. A więc $$x \in [9,12] \cup [14,\infty).$$
\rozwStop
\odpStart
$x \in [9,12] \cup [14,\infty)$
\odpStop
\testStart
A.$x \in [9,12] \cup [14,\infty)$\\
B.$x \in (9,12) \cup [14,\infty)$\\
C.$x \in (9,12] \cup [14,\infty)$\\
D.$x \in [9,12) \cup [14,\infty)$\\
E.$x \in [9,12] \cup (14,\infty)$\\
F.$x \in (9,12) \cup (14,\infty)$\\
G.$x \in [9,12) \cup (14,\infty)$\\
H.$x \in (9,12] \cup (14,\infty)$
\testStop
\kluczStart
A
\kluczStop



\zadStart{Zadanie z Wikieł Z 1.62 a) moja wersja nr 942}

Rozwiązać nierówności $(x-9)(x-12)(x-15)\ge0$.
\zadStop
\rozwStart{Patryk Wirkus}{}
Miejsca zerowe naszego wielomianu to: $9, 12, 15$.\\
Wielomian jest stopnia nieparzystego, ponadto znak współczynnika przy\linebreak najwyższej potędze x jest dodatni.\\ W związku z tym wykres wielomianu zaczyna się od lewej strony poniżej osi OX. A więc $$x \in [9,12] \cup [15,\infty).$$
\rozwStop
\odpStart
$x \in [9,12] \cup [15,\infty)$
\odpStop
\testStart
A.$x \in [9,12] \cup [15,\infty)$\\
B.$x \in (9,12) \cup [15,\infty)$\\
C.$x \in (9,12] \cup [15,\infty)$\\
D.$x \in [9,12) \cup [15,\infty)$\\
E.$x \in [9,12] \cup (15,\infty)$\\
F.$x \in (9,12) \cup (15,\infty)$\\
G.$x \in [9,12) \cup (15,\infty)$\\
H.$x \in (9,12] \cup (15,\infty)$
\testStop
\kluczStart
A
\kluczStop



\zadStart{Zadanie z Wikieł Z 1.62 a) moja wersja nr 943}

Rozwiązać nierówności $(x-9)(x-12)(x-16)\ge0$.
\zadStop
\rozwStart{Patryk Wirkus}{}
Miejsca zerowe naszego wielomianu to: $9, 12, 16$.\\
Wielomian jest stopnia nieparzystego, ponadto znak współczynnika przy\linebreak najwyższej potędze x jest dodatni.\\ W związku z tym wykres wielomianu zaczyna się od lewej strony poniżej osi OX. A więc $$x \in [9,12] \cup [16,\infty).$$
\rozwStop
\odpStart
$x \in [9,12] \cup [16,\infty)$
\odpStop
\testStart
A.$x \in [9,12] \cup [16,\infty)$\\
B.$x \in (9,12) \cup [16,\infty)$\\
C.$x \in (9,12] \cup [16,\infty)$\\
D.$x \in [9,12) \cup [16,\infty)$\\
E.$x \in [9,12] \cup (16,\infty)$\\
F.$x \in (9,12) \cup (16,\infty)$\\
G.$x \in [9,12) \cup (16,\infty)$\\
H.$x \in (9,12] \cup (16,\infty)$
\testStop
\kluczStart
A
\kluczStop



\zadStart{Zadanie z Wikieł Z 1.62 a) moja wersja nr 944}

Rozwiązać nierówności $(x-9)(x-12)(x-17)\ge0$.
\zadStop
\rozwStart{Patryk Wirkus}{}
Miejsca zerowe naszego wielomianu to: $9, 12, 17$.\\
Wielomian jest stopnia nieparzystego, ponadto znak współczynnika przy\linebreak najwyższej potędze x jest dodatni.\\ W związku z tym wykres wielomianu zaczyna się od lewej strony poniżej osi OX. A więc $$x \in [9,12] \cup [17,\infty).$$
\rozwStop
\odpStart
$x \in [9,12] \cup [17,\infty)$
\odpStop
\testStart
A.$x \in [9,12] \cup [17,\infty)$\\
B.$x \in (9,12) \cup [17,\infty)$\\
C.$x \in (9,12] \cup [17,\infty)$\\
D.$x \in [9,12) \cup [17,\infty)$\\
E.$x \in [9,12] \cup (17,\infty)$\\
F.$x \in (9,12) \cup (17,\infty)$\\
G.$x \in [9,12) \cup (17,\infty)$\\
H.$x \in (9,12] \cup (17,\infty)$
\testStop
\kluczStart
A
\kluczStop



\zadStart{Zadanie z Wikieł Z 1.62 a) moja wersja nr 945}

Rozwiązać nierówności $(x-9)(x-12)(x-18)\ge0$.
\zadStop
\rozwStart{Patryk Wirkus}{}
Miejsca zerowe naszego wielomianu to: $9, 12, 18$.\\
Wielomian jest stopnia nieparzystego, ponadto znak współczynnika przy\linebreak najwyższej potędze x jest dodatni.\\ W związku z tym wykres wielomianu zaczyna się od lewej strony poniżej osi OX. A więc $$x \in [9,12] \cup [18,\infty).$$
\rozwStop
\odpStart
$x \in [9,12] \cup [18,\infty)$
\odpStop
\testStart
A.$x \in [9,12] \cup [18,\infty)$\\
B.$x \in (9,12) \cup [18,\infty)$\\
C.$x \in (9,12] \cup [18,\infty)$\\
D.$x \in [9,12) \cup [18,\infty)$\\
E.$x \in [9,12] \cup (18,\infty)$\\
F.$x \in (9,12) \cup (18,\infty)$\\
G.$x \in [9,12) \cup (18,\infty)$\\
H.$x \in (9,12] \cup (18,\infty)$
\testStop
\kluczStart
A
\kluczStop



\zadStart{Zadanie z Wikieł Z 1.62 a) moja wersja nr 946}

Rozwiązać nierówności $(x-9)(x-12)(x-19)\ge0$.
\zadStop
\rozwStart{Patryk Wirkus}{}
Miejsca zerowe naszego wielomianu to: $9, 12, 19$.\\
Wielomian jest stopnia nieparzystego, ponadto znak współczynnika przy\linebreak najwyższej potędze x jest dodatni.\\ W związku z tym wykres wielomianu zaczyna się od lewej strony poniżej osi OX. A więc $$x \in [9,12] \cup [19,\infty).$$
\rozwStop
\odpStart
$x \in [9,12] \cup [19,\infty)$
\odpStop
\testStart
A.$x \in [9,12] \cup [19,\infty)$\\
B.$x \in (9,12) \cup [19,\infty)$\\
C.$x \in (9,12] \cup [19,\infty)$\\
D.$x \in [9,12) \cup [19,\infty)$\\
E.$x \in [9,12] \cup (19,\infty)$\\
F.$x \in (9,12) \cup (19,\infty)$\\
G.$x \in [9,12) \cup (19,\infty)$\\
H.$x \in (9,12] \cup (19,\infty)$
\testStop
\kluczStart
A
\kluczStop



\zadStart{Zadanie z Wikieł Z 1.62 a) moja wersja nr 947}

Rozwiązać nierówności $(x-9)(x-12)(x-20)\ge0$.
\zadStop
\rozwStart{Patryk Wirkus}{}
Miejsca zerowe naszego wielomianu to: $9, 12, 20$.\\
Wielomian jest stopnia nieparzystego, ponadto znak współczynnika przy\linebreak najwyższej potędze x jest dodatni.\\ W związku z tym wykres wielomianu zaczyna się od lewej strony poniżej osi OX. A więc $$x \in [9,12] \cup [20,\infty).$$
\rozwStop
\odpStart
$x \in [9,12] \cup [20,\infty)$
\odpStop
\testStart
A.$x \in [9,12] \cup [20,\infty)$\\
B.$x \in (9,12) \cup [20,\infty)$\\
C.$x \in (9,12] \cup [20,\infty)$\\
D.$x \in [9,12) \cup [20,\infty)$\\
E.$x \in [9,12] \cup (20,\infty)$\\
F.$x \in (9,12) \cup (20,\infty)$\\
G.$x \in [9,12) \cup (20,\infty)$\\
H.$x \in (9,12] \cup (20,\infty)$
\testStop
\kluczStart
A
\kluczStop



\zadStart{Zadanie z Wikieł Z 1.62 a) moja wersja nr 948}

Rozwiązać nierówności $(x-9)(x-13)(x-14)\ge0$.
\zadStop
\rozwStart{Patryk Wirkus}{}
Miejsca zerowe naszego wielomianu to: $9, 13, 14$.\\
Wielomian jest stopnia nieparzystego, ponadto znak współczynnika przy\linebreak najwyższej potędze x jest dodatni.\\ W związku z tym wykres wielomianu zaczyna się od lewej strony poniżej osi OX. A więc $$x \in [9,13] \cup [14,\infty).$$
\rozwStop
\odpStart
$x \in [9,13] \cup [14,\infty)$
\odpStop
\testStart
A.$x \in [9,13] \cup [14,\infty)$\\
B.$x \in (9,13) \cup [14,\infty)$\\
C.$x \in (9,13] \cup [14,\infty)$\\
D.$x \in [9,13) \cup [14,\infty)$\\
E.$x \in [9,13] \cup (14,\infty)$\\
F.$x \in (9,13) \cup (14,\infty)$\\
G.$x \in [9,13) \cup (14,\infty)$\\
H.$x \in (9,13] \cup (14,\infty)$
\testStop
\kluczStart
A
\kluczStop



\zadStart{Zadanie z Wikieł Z 1.62 a) moja wersja nr 949}

Rozwiązać nierówności $(x-9)(x-13)(x-15)\ge0$.
\zadStop
\rozwStart{Patryk Wirkus}{}
Miejsca zerowe naszego wielomianu to: $9, 13, 15$.\\
Wielomian jest stopnia nieparzystego, ponadto znak współczynnika przy\linebreak najwyższej potędze x jest dodatni.\\ W związku z tym wykres wielomianu zaczyna się od lewej strony poniżej osi OX. A więc $$x \in [9,13] \cup [15,\infty).$$
\rozwStop
\odpStart
$x \in [9,13] \cup [15,\infty)$
\odpStop
\testStart
A.$x \in [9,13] \cup [15,\infty)$\\
B.$x \in (9,13) \cup [15,\infty)$\\
C.$x \in (9,13] \cup [15,\infty)$\\
D.$x \in [9,13) \cup [15,\infty)$\\
E.$x \in [9,13] \cup (15,\infty)$\\
F.$x \in (9,13) \cup (15,\infty)$\\
G.$x \in [9,13) \cup (15,\infty)$\\
H.$x \in (9,13] \cup (15,\infty)$
\testStop
\kluczStart
A
\kluczStop



\zadStart{Zadanie z Wikieł Z 1.62 a) moja wersja nr 950}

Rozwiązać nierówności $(x-9)(x-13)(x-16)\ge0$.
\zadStop
\rozwStart{Patryk Wirkus}{}
Miejsca zerowe naszego wielomianu to: $9, 13, 16$.\\
Wielomian jest stopnia nieparzystego, ponadto znak współczynnika przy\linebreak najwyższej potędze x jest dodatni.\\ W związku z tym wykres wielomianu zaczyna się od lewej strony poniżej osi OX. A więc $$x \in [9,13] \cup [16,\infty).$$
\rozwStop
\odpStart
$x \in [9,13] \cup [16,\infty)$
\odpStop
\testStart
A.$x \in [9,13] \cup [16,\infty)$\\
B.$x \in (9,13) \cup [16,\infty)$\\
C.$x \in (9,13] \cup [16,\infty)$\\
D.$x \in [9,13) \cup [16,\infty)$\\
E.$x \in [9,13] \cup (16,\infty)$\\
F.$x \in (9,13) \cup (16,\infty)$\\
G.$x \in [9,13) \cup (16,\infty)$\\
H.$x \in (9,13] \cup (16,\infty)$
\testStop
\kluczStart
A
\kluczStop



\zadStart{Zadanie z Wikieł Z 1.62 a) moja wersja nr 951}

Rozwiązać nierówności $(x-9)(x-13)(x-17)\ge0$.
\zadStop
\rozwStart{Patryk Wirkus}{}
Miejsca zerowe naszego wielomianu to: $9, 13, 17$.\\
Wielomian jest stopnia nieparzystego, ponadto znak współczynnika przy\linebreak najwyższej potędze x jest dodatni.\\ W związku z tym wykres wielomianu zaczyna się od lewej strony poniżej osi OX. A więc $$x \in [9,13] \cup [17,\infty).$$
\rozwStop
\odpStart
$x \in [9,13] \cup [17,\infty)$
\odpStop
\testStart
A.$x \in [9,13] \cup [17,\infty)$\\
B.$x \in (9,13) \cup [17,\infty)$\\
C.$x \in (9,13] \cup [17,\infty)$\\
D.$x \in [9,13) \cup [17,\infty)$\\
E.$x \in [9,13] \cup (17,\infty)$\\
F.$x \in (9,13) \cup (17,\infty)$\\
G.$x \in [9,13) \cup (17,\infty)$\\
H.$x \in (9,13] \cup (17,\infty)$
\testStop
\kluczStart
A
\kluczStop



\zadStart{Zadanie z Wikieł Z 1.62 a) moja wersja nr 952}

Rozwiązać nierówności $(x-9)(x-13)(x-18)\ge0$.
\zadStop
\rozwStart{Patryk Wirkus}{}
Miejsca zerowe naszego wielomianu to: $9, 13, 18$.\\
Wielomian jest stopnia nieparzystego, ponadto znak współczynnika przy\linebreak najwyższej potędze x jest dodatni.\\ W związku z tym wykres wielomianu zaczyna się od lewej strony poniżej osi OX. A więc $$x \in [9,13] \cup [18,\infty).$$
\rozwStop
\odpStart
$x \in [9,13] \cup [18,\infty)$
\odpStop
\testStart
A.$x \in [9,13] \cup [18,\infty)$\\
B.$x \in (9,13) \cup [18,\infty)$\\
C.$x \in (9,13] \cup [18,\infty)$\\
D.$x \in [9,13) \cup [18,\infty)$\\
E.$x \in [9,13] \cup (18,\infty)$\\
F.$x \in (9,13) \cup (18,\infty)$\\
G.$x \in [9,13) \cup (18,\infty)$\\
H.$x \in (9,13] \cup (18,\infty)$
\testStop
\kluczStart
A
\kluczStop



\zadStart{Zadanie z Wikieł Z 1.62 a) moja wersja nr 953}

Rozwiązać nierówności $(x-9)(x-13)(x-19)\ge0$.
\zadStop
\rozwStart{Patryk Wirkus}{}
Miejsca zerowe naszego wielomianu to: $9, 13, 19$.\\
Wielomian jest stopnia nieparzystego, ponadto znak współczynnika przy\linebreak najwyższej potędze x jest dodatni.\\ W związku z tym wykres wielomianu zaczyna się od lewej strony poniżej osi OX. A więc $$x \in [9,13] \cup [19,\infty).$$
\rozwStop
\odpStart
$x \in [9,13] \cup [19,\infty)$
\odpStop
\testStart
A.$x \in [9,13] \cup [19,\infty)$\\
B.$x \in (9,13) \cup [19,\infty)$\\
C.$x \in (9,13] \cup [19,\infty)$\\
D.$x \in [9,13) \cup [19,\infty)$\\
E.$x \in [9,13] \cup (19,\infty)$\\
F.$x \in (9,13) \cup (19,\infty)$\\
G.$x \in [9,13) \cup (19,\infty)$\\
H.$x \in (9,13] \cup (19,\infty)$
\testStop
\kluczStart
A
\kluczStop



\zadStart{Zadanie z Wikieł Z 1.62 a) moja wersja nr 954}

Rozwiązać nierówności $(x-9)(x-13)(x-20)\ge0$.
\zadStop
\rozwStart{Patryk Wirkus}{}
Miejsca zerowe naszego wielomianu to: $9, 13, 20$.\\
Wielomian jest stopnia nieparzystego, ponadto znak współczynnika przy\linebreak najwyższej potędze x jest dodatni.\\ W związku z tym wykres wielomianu zaczyna się od lewej strony poniżej osi OX. A więc $$x \in [9,13] \cup [20,\infty).$$
\rozwStop
\odpStart
$x \in [9,13] \cup [20,\infty)$
\odpStop
\testStart
A.$x \in [9,13] \cup [20,\infty)$\\
B.$x \in (9,13) \cup [20,\infty)$\\
C.$x \in (9,13] \cup [20,\infty)$\\
D.$x \in [9,13) \cup [20,\infty)$\\
E.$x \in [9,13] \cup (20,\infty)$\\
F.$x \in (9,13) \cup (20,\infty)$\\
G.$x \in [9,13) \cup (20,\infty)$\\
H.$x \in (9,13] \cup (20,\infty)$
\testStop
\kluczStart
A
\kluczStop



\zadStart{Zadanie z Wikieł Z 1.62 a) moja wersja nr 955}

Rozwiązać nierówności $(x-9)(x-14)(x-15)\ge0$.
\zadStop
\rozwStart{Patryk Wirkus}{}
Miejsca zerowe naszego wielomianu to: $9, 14, 15$.\\
Wielomian jest stopnia nieparzystego, ponadto znak współczynnika przy\linebreak najwyższej potędze x jest dodatni.\\ W związku z tym wykres wielomianu zaczyna się od lewej strony poniżej osi OX. A więc $$x \in [9,14] \cup [15,\infty).$$
\rozwStop
\odpStart
$x \in [9,14] \cup [15,\infty)$
\odpStop
\testStart
A.$x \in [9,14] \cup [15,\infty)$\\
B.$x \in (9,14) \cup [15,\infty)$\\
C.$x \in (9,14] \cup [15,\infty)$\\
D.$x \in [9,14) \cup [15,\infty)$\\
E.$x \in [9,14] \cup (15,\infty)$\\
F.$x \in (9,14) \cup (15,\infty)$\\
G.$x \in [9,14) \cup (15,\infty)$\\
H.$x \in (9,14] \cup (15,\infty)$
\testStop
\kluczStart
A
\kluczStop



\zadStart{Zadanie z Wikieł Z 1.62 a) moja wersja nr 956}

Rozwiązać nierówności $(x-9)(x-14)(x-16)\ge0$.
\zadStop
\rozwStart{Patryk Wirkus}{}
Miejsca zerowe naszego wielomianu to: $9, 14, 16$.\\
Wielomian jest stopnia nieparzystego, ponadto znak współczynnika przy\linebreak najwyższej potędze x jest dodatni.\\ W związku z tym wykres wielomianu zaczyna się od lewej strony poniżej osi OX. A więc $$x \in [9,14] \cup [16,\infty).$$
\rozwStop
\odpStart
$x \in [9,14] \cup [16,\infty)$
\odpStop
\testStart
A.$x \in [9,14] \cup [16,\infty)$\\
B.$x \in (9,14) \cup [16,\infty)$\\
C.$x \in (9,14] \cup [16,\infty)$\\
D.$x \in [9,14) \cup [16,\infty)$\\
E.$x \in [9,14] \cup (16,\infty)$\\
F.$x \in (9,14) \cup (16,\infty)$\\
G.$x \in [9,14) \cup (16,\infty)$\\
H.$x \in (9,14] \cup (16,\infty)$
\testStop
\kluczStart
A
\kluczStop



\zadStart{Zadanie z Wikieł Z 1.62 a) moja wersja nr 957}

Rozwiązać nierówności $(x-9)(x-14)(x-17)\ge0$.
\zadStop
\rozwStart{Patryk Wirkus}{}
Miejsca zerowe naszego wielomianu to: $9, 14, 17$.\\
Wielomian jest stopnia nieparzystego, ponadto znak współczynnika przy\linebreak najwyższej potędze x jest dodatni.\\ W związku z tym wykres wielomianu zaczyna się od lewej strony poniżej osi OX. A więc $$x \in [9,14] \cup [17,\infty).$$
\rozwStop
\odpStart
$x \in [9,14] \cup [17,\infty)$
\odpStop
\testStart
A.$x \in [9,14] \cup [17,\infty)$\\
B.$x \in (9,14) \cup [17,\infty)$\\
C.$x \in (9,14] \cup [17,\infty)$\\
D.$x \in [9,14) \cup [17,\infty)$\\
E.$x \in [9,14] \cup (17,\infty)$\\
F.$x \in (9,14) \cup (17,\infty)$\\
G.$x \in [9,14) \cup (17,\infty)$\\
H.$x \in (9,14] \cup (17,\infty)$
\testStop
\kluczStart
A
\kluczStop



\zadStart{Zadanie z Wikieł Z 1.62 a) moja wersja nr 958}

Rozwiązać nierówności $(x-9)(x-14)(x-18)\ge0$.
\zadStop
\rozwStart{Patryk Wirkus}{}
Miejsca zerowe naszego wielomianu to: $9, 14, 18$.\\
Wielomian jest stopnia nieparzystego, ponadto znak współczynnika przy\linebreak najwyższej potędze x jest dodatni.\\ W związku z tym wykres wielomianu zaczyna się od lewej strony poniżej osi OX. A więc $$x \in [9,14] \cup [18,\infty).$$
\rozwStop
\odpStart
$x \in [9,14] \cup [18,\infty)$
\odpStop
\testStart
A.$x \in [9,14] \cup [18,\infty)$\\
B.$x \in (9,14) \cup [18,\infty)$\\
C.$x \in (9,14] \cup [18,\infty)$\\
D.$x \in [9,14) \cup [18,\infty)$\\
E.$x \in [9,14] \cup (18,\infty)$\\
F.$x \in (9,14) \cup (18,\infty)$\\
G.$x \in [9,14) \cup (18,\infty)$\\
H.$x \in (9,14] \cup (18,\infty)$
\testStop
\kluczStart
A
\kluczStop



\zadStart{Zadanie z Wikieł Z 1.62 a) moja wersja nr 959}

Rozwiązać nierówności $(x-9)(x-14)(x-19)\ge0$.
\zadStop
\rozwStart{Patryk Wirkus}{}
Miejsca zerowe naszego wielomianu to: $9, 14, 19$.\\
Wielomian jest stopnia nieparzystego, ponadto znak współczynnika przy\linebreak najwyższej potędze x jest dodatni.\\ W związku z tym wykres wielomianu zaczyna się od lewej strony poniżej osi OX. A więc $$x \in [9,14] \cup [19,\infty).$$
\rozwStop
\odpStart
$x \in [9,14] \cup [19,\infty)$
\odpStop
\testStart
A.$x \in [9,14] \cup [19,\infty)$\\
B.$x \in (9,14) \cup [19,\infty)$\\
C.$x \in (9,14] \cup [19,\infty)$\\
D.$x \in [9,14) \cup [19,\infty)$\\
E.$x \in [9,14] \cup (19,\infty)$\\
F.$x \in (9,14) \cup (19,\infty)$\\
G.$x \in [9,14) \cup (19,\infty)$\\
H.$x \in (9,14] \cup (19,\infty)$
\testStop
\kluczStart
A
\kluczStop



\zadStart{Zadanie z Wikieł Z 1.62 a) moja wersja nr 960}

Rozwiązać nierówności $(x-9)(x-14)(x-20)\ge0$.
\zadStop
\rozwStart{Patryk Wirkus}{}
Miejsca zerowe naszego wielomianu to: $9, 14, 20$.\\
Wielomian jest stopnia nieparzystego, ponadto znak współczynnika przy\linebreak najwyższej potędze x jest dodatni.\\ W związku z tym wykres wielomianu zaczyna się od lewej strony poniżej osi OX. A więc $$x \in [9,14] \cup [20,\infty).$$
\rozwStop
\odpStart
$x \in [9,14] \cup [20,\infty)$
\odpStop
\testStart
A.$x \in [9,14] \cup [20,\infty)$\\
B.$x \in (9,14) \cup [20,\infty)$\\
C.$x \in (9,14] \cup [20,\infty)$\\
D.$x \in [9,14) \cup [20,\infty)$\\
E.$x \in [9,14] \cup (20,\infty)$\\
F.$x \in (9,14) \cup (20,\infty)$\\
G.$x \in [9,14) \cup (20,\infty)$\\
H.$x \in (9,14] \cup (20,\infty)$
\testStop
\kluczStart
A
\kluczStop



\zadStart{Zadanie z Wikieł Z 1.62 a) moja wersja nr 961}

Rozwiązać nierówności $(x-9)(x-15)(x-16)\ge0$.
\zadStop
\rozwStart{Patryk Wirkus}{}
Miejsca zerowe naszego wielomianu to: $9, 15, 16$.\\
Wielomian jest stopnia nieparzystego, ponadto znak współczynnika przy\linebreak najwyższej potędze x jest dodatni.\\ W związku z tym wykres wielomianu zaczyna się od lewej strony poniżej osi OX. A więc $$x \in [9,15] \cup [16,\infty).$$
\rozwStop
\odpStart
$x \in [9,15] \cup [16,\infty)$
\odpStop
\testStart
A.$x \in [9,15] \cup [16,\infty)$\\
B.$x \in (9,15) \cup [16,\infty)$\\
C.$x \in (9,15] \cup [16,\infty)$\\
D.$x \in [9,15) \cup [16,\infty)$\\
E.$x \in [9,15] \cup (16,\infty)$\\
F.$x \in (9,15) \cup (16,\infty)$\\
G.$x \in [9,15) \cup (16,\infty)$\\
H.$x \in (9,15] \cup (16,\infty)$
\testStop
\kluczStart
A
\kluczStop



\zadStart{Zadanie z Wikieł Z 1.62 a) moja wersja nr 962}

Rozwiązać nierówności $(x-9)(x-15)(x-17)\ge0$.
\zadStop
\rozwStart{Patryk Wirkus}{}
Miejsca zerowe naszego wielomianu to: $9, 15, 17$.\\
Wielomian jest stopnia nieparzystego, ponadto znak współczynnika przy\linebreak najwyższej potędze x jest dodatni.\\ W związku z tym wykres wielomianu zaczyna się od lewej strony poniżej osi OX. A więc $$x \in [9,15] \cup [17,\infty).$$
\rozwStop
\odpStart
$x \in [9,15] \cup [17,\infty)$
\odpStop
\testStart
A.$x \in [9,15] \cup [17,\infty)$\\
B.$x \in (9,15) \cup [17,\infty)$\\
C.$x \in (9,15] \cup [17,\infty)$\\
D.$x \in [9,15) \cup [17,\infty)$\\
E.$x \in [9,15] \cup (17,\infty)$\\
F.$x \in (9,15) \cup (17,\infty)$\\
G.$x \in [9,15) \cup (17,\infty)$\\
H.$x \in (9,15] \cup (17,\infty)$
\testStop
\kluczStart
A
\kluczStop



\zadStart{Zadanie z Wikieł Z 1.62 a) moja wersja nr 963}

Rozwiązać nierówności $(x-9)(x-15)(x-18)\ge0$.
\zadStop
\rozwStart{Patryk Wirkus}{}
Miejsca zerowe naszego wielomianu to: $9, 15, 18$.\\
Wielomian jest stopnia nieparzystego, ponadto znak współczynnika przy\linebreak najwyższej potędze x jest dodatni.\\ W związku z tym wykres wielomianu zaczyna się od lewej strony poniżej osi OX. A więc $$x \in [9,15] \cup [18,\infty).$$
\rozwStop
\odpStart
$x \in [9,15] \cup [18,\infty)$
\odpStop
\testStart
A.$x \in [9,15] \cup [18,\infty)$\\
B.$x \in (9,15) \cup [18,\infty)$\\
C.$x \in (9,15] \cup [18,\infty)$\\
D.$x \in [9,15) \cup [18,\infty)$\\
E.$x \in [9,15] \cup (18,\infty)$\\
F.$x \in (9,15) \cup (18,\infty)$\\
G.$x \in [9,15) \cup (18,\infty)$\\
H.$x \in (9,15] \cup (18,\infty)$
\testStop
\kluczStart
A
\kluczStop



\zadStart{Zadanie z Wikieł Z 1.62 a) moja wersja nr 964}

Rozwiązać nierówności $(x-9)(x-15)(x-19)\ge0$.
\zadStop
\rozwStart{Patryk Wirkus}{}
Miejsca zerowe naszego wielomianu to: $9, 15, 19$.\\
Wielomian jest stopnia nieparzystego, ponadto znak współczynnika przy\linebreak najwyższej potędze x jest dodatni.\\ W związku z tym wykres wielomianu zaczyna się od lewej strony poniżej osi OX. A więc $$x \in [9,15] \cup [19,\infty).$$
\rozwStop
\odpStart
$x \in [9,15] \cup [19,\infty)$
\odpStop
\testStart
A.$x \in [9,15] \cup [19,\infty)$\\
B.$x \in (9,15) \cup [19,\infty)$\\
C.$x \in (9,15] \cup [19,\infty)$\\
D.$x \in [9,15) \cup [19,\infty)$\\
E.$x \in [9,15] \cup (19,\infty)$\\
F.$x \in (9,15) \cup (19,\infty)$\\
G.$x \in [9,15) \cup (19,\infty)$\\
H.$x \in (9,15] \cup (19,\infty)$
\testStop
\kluczStart
A
\kluczStop



\zadStart{Zadanie z Wikieł Z 1.62 a) moja wersja nr 965}

Rozwiązać nierówności $(x-9)(x-15)(x-20)\ge0$.
\zadStop
\rozwStart{Patryk Wirkus}{}
Miejsca zerowe naszego wielomianu to: $9, 15, 20$.\\
Wielomian jest stopnia nieparzystego, ponadto znak współczynnika przy\linebreak najwyższej potędze x jest dodatni.\\ W związku z tym wykres wielomianu zaczyna się od lewej strony poniżej osi OX. A więc $$x \in [9,15] \cup [20,\infty).$$
\rozwStop
\odpStart
$x \in [9,15] \cup [20,\infty)$
\odpStop
\testStart
A.$x \in [9,15] \cup [20,\infty)$\\
B.$x \in (9,15) \cup [20,\infty)$\\
C.$x \in (9,15] \cup [20,\infty)$\\
D.$x \in [9,15) \cup [20,\infty)$\\
E.$x \in [9,15] \cup (20,\infty)$\\
F.$x \in (9,15) \cup (20,\infty)$\\
G.$x \in [9,15) \cup (20,\infty)$\\
H.$x \in (9,15] \cup (20,\infty)$
\testStop
\kluczStart
A
\kluczStop



\zadStart{Zadanie z Wikieł Z 1.62 a) moja wersja nr 966}

Rozwiązać nierówności $(x-9)(x-16)(x-17)\ge0$.
\zadStop
\rozwStart{Patryk Wirkus}{}
Miejsca zerowe naszego wielomianu to: $9, 16, 17$.\\
Wielomian jest stopnia nieparzystego, ponadto znak współczynnika przy\linebreak najwyższej potędze x jest dodatni.\\ W związku z tym wykres wielomianu zaczyna się od lewej strony poniżej osi OX. A więc $$x \in [9,16] \cup [17,\infty).$$
\rozwStop
\odpStart
$x \in [9,16] \cup [17,\infty)$
\odpStop
\testStart
A.$x \in [9,16] \cup [17,\infty)$\\
B.$x \in (9,16) \cup [17,\infty)$\\
C.$x \in (9,16] \cup [17,\infty)$\\
D.$x \in [9,16) \cup [17,\infty)$\\
E.$x \in [9,16] \cup (17,\infty)$\\
F.$x \in (9,16) \cup (17,\infty)$\\
G.$x \in [9,16) \cup (17,\infty)$\\
H.$x \in (9,16] \cup (17,\infty)$
\testStop
\kluczStart
A
\kluczStop



\zadStart{Zadanie z Wikieł Z 1.62 a) moja wersja nr 967}

Rozwiązać nierówności $(x-9)(x-16)(x-18)\ge0$.
\zadStop
\rozwStart{Patryk Wirkus}{}
Miejsca zerowe naszego wielomianu to: $9, 16, 18$.\\
Wielomian jest stopnia nieparzystego, ponadto znak współczynnika przy\linebreak najwyższej potędze x jest dodatni.\\ W związku z tym wykres wielomianu zaczyna się od lewej strony poniżej osi OX. A więc $$x \in [9,16] \cup [18,\infty).$$
\rozwStop
\odpStart
$x \in [9,16] \cup [18,\infty)$
\odpStop
\testStart
A.$x \in [9,16] \cup [18,\infty)$\\
B.$x \in (9,16) \cup [18,\infty)$\\
C.$x \in (9,16] \cup [18,\infty)$\\
D.$x \in [9,16) \cup [18,\infty)$\\
E.$x \in [9,16] \cup (18,\infty)$\\
F.$x \in (9,16) \cup (18,\infty)$\\
G.$x \in [9,16) \cup (18,\infty)$\\
H.$x \in (9,16] \cup (18,\infty)$
\testStop
\kluczStart
A
\kluczStop



\zadStart{Zadanie z Wikieł Z 1.62 a) moja wersja nr 968}

Rozwiązać nierówności $(x-9)(x-16)(x-19)\ge0$.
\zadStop
\rozwStart{Patryk Wirkus}{}
Miejsca zerowe naszego wielomianu to: $9, 16, 19$.\\
Wielomian jest stopnia nieparzystego, ponadto znak współczynnika przy\linebreak najwyższej potędze x jest dodatni.\\ W związku z tym wykres wielomianu zaczyna się od lewej strony poniżej osi OX. A więc $$x \in [9,16] \cup [19,\infty).$$
\rozwStop
\odpStart
$x \in [9,16] \cup [19,\infty)$
\odpStop
\testStart
A.$x \in [9,16] \cup [19,\infty)$\\
B.$x \in (9,16) \cup [19,\infty)$\\
C.$x \in (9,16] \cup [19,\infty)$\\
D.$x \in [9,16) \cup [19,\infty)$\\
E.$x \in [9,16] \cup (19,\infty)$\\
F.$x \in (9,16) \cup (19,\infty)$\\
G.$x \in [9,16) \cup (19,\infty)$\\
H.$x \in (9,16] \cup (19,\infty)$
\testStop
\kluczStart
A
\kluczStop



\zadStart{Zadanie z Wikieł Z 1.62 a) moja wersja nr 969}

Rozwiązać nierówności $(x-9)(x-16)(x-20)\ge0$.
\zadStop
\rozwStart{Patryk Wirkus}{}
Miejsca zerowe naszego wielomianu to: $9, 16, 20$.\\
Wielomian jest stopnia nieparzystego, ponadto znak współczynnika przy\linebreak najwyższej potędze x jest dodatni.\\ W związku z tym wykres wielomianu zaczyna się od lewej strony poniżej osi OX. A więc $$x \in [9,16] \cup [20,\infty).$$
\rozwStop
\odpStart
$x \in [9,16] \cup [20,\infty)$
\odpStop
\testStart
A.$x \in [9,16] \cup [20,\infty)$\\
B.$x \in (9,16) \cup [20,\infty)$\\
C.$x \in (9,16] \cup [20,\infty)$\\
D.$x \in [9,16) \cup [20,\infty)$\\
E.$x \in [9,16] \cup (20,\infty)$\\
F.$x \in (9,16) \cup (20,\infty)$\\
G.$x \in [9,16) \cup (20,\infty)$\\
H.$x \in (9,16] \cup (20,\infty)$
\testStop
\kluczStart
A
\kluczStop



\zadStart{Zadanie z Wikieł Z 1.62 a) moja wersja nr 970}

Rozwiązać nierówności $(x-9)(x-17)(x-18)\ge0$.
\zadStop
\rozwStart{Patryk Wirkus}{}
Miejsca zerowe naszego wielomianu to: $9, 17, 18$.\\
Wielomian jest stopnia nieparzystego, ponadto znak współczynnika przy\linebreak najwyższej potędze x jest dodatni.\\ W związku z tym wykres wielomianu zaczyna się od lewej strony poniżej osi OX. A więc $$x \in [9,17] \cup [18,\infty).$$
\rozwStop
\odpStart
$x \in [9,17] \cup [18,\infty)$
\odpStop
\testStart
A.$x \in [9,17] \cup [18,\infty)$\\
B.$x \in (9,17) \cup [18,\infty)$\\
C.$x \in (9,17] \cup [18,\infty)$\\
D.$x \in [9,17) \cup [18,\infty)$\\
E.$x \in [9,17] \cup (18,\infty)$\\
F.$x \in (9,17) \cup (18,\infty)$\\
G.$x \in [9,17) \cup (18,\infty)$\\
H.$x \in (9,17] \cup (18,\infty)$
\testStop
\kluczStart
A
\kluczStop



\zadStart{Zadanie z Wikieł Z 1.62 a) moja wersja nr 971}

Rozwiązać nierówności $(x-9)(x-17)(x-19)\ge0$.
\zadStop
\rozwStart{Patryk Wirkus}{}
Miejsca zerowe naszego wielomianu to: $9, 17, 19$.\\
Wielomian jest stopnia nieparzystego, ponadto znak współczynnika przy\linebreak najwyższej potędze x jest dodatni.\\ W związku z tym wykres wielomianu zaczyna się od lewej strony poniżej osi OX. A więc $$x \in [9,17] \cup [19,\infty).$$
\rozwStop
\odpStart
$x \in [9,17] \cup [19,\infty)$
\odpStop
\testStart
A.$x \in [9,17] \cup [19,\infty)$\\
B.$x \in (9,17) \cup [19,\infty)$\\
C.$x \in (9,17] \cup [19,\infty)$\\
D.$x \in [9,17) \cup [19,\infty)$\\
E.$x \in [9,17] \cup (19,\infty)$\\
F.$x \in (9,17) \cup (19,\infty)$\\
G.$x \in [9,17) \cup (19,\infty)$\\
H.$x \in (9,17] \cup (19,\infty)$
\testStop
\kluczStart
A
\kluczStop



\zadStart{Zadanie z Wikieł Z 1.62 a) moja wersja nr 972}

Rozwiązać nierówności $(x-9)(x-17)(x-20)\ge0$.
\zadStop
\rozwStart{Patryk Wirkus}{}
Miejsca zerowe naszego wielomianu to: $9, 17, 20$.\\
Wielomian jest stopnia nieparzystego, ponadto znak współczynnika przy\linebreak najwyższej potędze x jest dodatni.\\ W związku z tym wykres wielomianu zaczyna się od lewej strony poniżej osi OX. A więc $$x \in [9,17] \cup [20,\infty).$$
\rozwStop
\odpStart
$x \in [9,17] \cup [20,\infty)$
\odpStop
\testStart
A.$x \in [9,17] \cup [20,\infty)$\\
B.$x \in (9,17) \cup [20,\infty)$\\
C.$x \in (9,17] \cup [20,\infty)$\\
D.$x \in [9,17) \cup [20,\infty)$\\
E.$x \in [9,17] \cup (20,\infty)$\\
F.$x \in (9,17) \cup (20,\infty)$\\
G.$x \in [9,17) \cup (20,\infty)$\\
H.$x \in (9,17] \cup (20,\infty)$
\testStop
\kluczStart
A
\kluczStop



\zadStart{Zadanie z Wikieł Z 1.62 a) moja wersja nr 973}

Rozwiązać nierówności $(x-9)(x-18)(x-19)\ge0$.
\zadStop
\rozwStart{Patryk Wirkus}{}
Miejsca zerowe naszego wielomianu to: $9, 18, 19$.\\
Wielomian jest stopnia nieparzystego, ponadto znak współczynnika przy\linebreak najwyższej potędze x jest dodatni.\\ W związku z tym wykres wielomianu zaczyna się od lewej strony poniżej osi OX. A więc $$x \in [9,18] \cup [19,\infty).$$
\rozwStop
\odpStart
$x \in [9,18] \cup [19,\infty)$
\odpStop
\testStart
A.$x \in [9,18] \cup [19,\infty)$\\
B.$x \in (9,18) \cup [19,\infty)$\\
C.$x \in (9,18] \cup [19,\infty)$\\
D.$x \in [9,18) \cup [19,\infty)$\\
E.$x \in [9,18] \cup (19,\infty)$\\
F.$x \in (9,18) \cup (19,\infty)$\\
G.$x \in [9,18) \cup (19,\infty)$\\
H.$x \in (9,18] \cup (19,\infty)$
\testStop
\kluczStart
A
\kluczStop



\zadStart{Zadanie z Wikieł Z 1.62 a) moja wersja nr 974}

Rozwiązać nierówności $(x-9)(x-18)(x-20)\ge0$.
\zadStop
\rozwStart{Patryk Wirkus}{}
Miejsca zerowe naszego wielomianu to: $9, 18, 20$.\\
Wielomian jest stopnia nieparzystego, ponadto znak współczynnika przy\linebreak najwyższej potędze x jest dodatni.\\ W związku z tym wykres wielomianu zaczyna się od lewej strony poniżej osi OX. A więc $$x \in [9,18] \cup [20,\infty).$$
\rozwStop
\odpStart
$x \in [9,18] \cup [20,\infty)$
\odpStop
\testStart
A.$x \in [9,18] \cup [20,\infty)$\\
B.$x \in (9,18) \cup [20,\infty)$\\
C.$x \in (9,18] \cup [20,\infty)$\\
D.$x \in [9,18) \cup [20,\infty)$\\
E.$x \in [9,18] \cup (20,\infty)$\\
F.$x \in (9,18) \cup (20,\infty)$\\
G.$x \in [9,18) \cup (20,\infty)$\\
H.$x \in (9,18] \cup (20,\infty)$
\testStop
\kluczStart
A
\kluczStop



\zadStart{Zadanie z Wikieł Z 1.62 a) moja wersja nr 975}

Rozwiązać nierówności $(x-9)(x-19)(x-20)\ge0$.
\zadStop
\rozwStart{Patryk Wirkus}{}
Miejsca zerowe naszego wielomianu to: $9, 19, 20$.\\
Wielomian jest stopnia nieparzystego, ponadto znak współczynnika przy\linebreak najwyższej potędze x jest dodatni.\\ W związku z tym wykres wielomianu zaczyna się od lewej strony poniżej osi OX. A więc $$x \in [9,19] \cup [20,\infty).$$
\rozwStop
\odpStart
$x \in [9,19] \cup [20,\infty)$
\odpStop
\testStart
A.$x \in [9,19] \cup [20,\infty)$\\
B.$x \in (9,19) \cup [20,\infty)$\\
C.$x \in (9,19] \cup [20,\infty)$\\
D.$x \in [9,19) \cup [20,\infty)$\\
E.$x \in [9,19] \cup (20,\infty)$\\
F.$x \in (9,19) \cup (20,\infty)$\\
G.$x \in [9,19) \cup (20,\infty)$\\
H.$x \in (9,19] \cup (20,\infty)$
\testStop
\kluczStart
A
\kluczStop



\zadStart{Zadanie z Wikieł Z 1.62 a) moja wersja nr 976}

Rozwiązać nierówności $(x-10)(x-11)(x-12)\ge0$.
\zadStop
\rozwStart{Patryk Wirkus}{}
Miejsca zerowe naszego wielomianu to: $10, 11, 12$.\\
Wielomian jest stopnia nieparzystego, ponadto znak współczynnika przy\linebreak najwyższej potędze x jest dodatni.\\ W związku z tym wykres wielomianu zaczyna się od lewej strony poniżej osi OX. A więc $$x \in [10,11] \cup [12,\infty).$$
\rozwStop
\odpStart
$x \in [10,11] \cup [12,\infty)$
\odpStop
\testStart
A.$x \in [10,11] \cup [12,\infty)$\\
B.$x \in (10,11) \cup [12,\infty)$\\
C.$x \in (10,11] \cup [12,\infty)$\\
D.$x \in [10,11) \cup [12,\infty)$\\
E.$x \in [10,11] \cup (12,\infty)$\\
F.$x \in (10,11) \cup (12,\infty)$\\
G.$x \in [10,11) \cup (12,\infty)$\\
H.$x \in (10,11] \cup (12,\infty)$
\testStop
\kluczStart
A
\kluczStop



\zadStart{Zadanie z Wikieł Z 1.62 a) moja wersja nr 977}

Rozwiązać nierówności $(x-10)(x-11)(x-13)\ge0$.
\zadStop
\rozwStart{Patryk Wirkus}{}
Miejsca zerowe naszego wielomianu to: $10, 11, 13$.\\
Wielomian jest stopnia nieparzystego, ponadto znak współczynnika przy\linebreak najwyższej potędze x jest dodatni.\\ W związku z tym wykres wielomianu zaczyna się od lewej strony poniżej osi OX. A więc $$x \in [10,11] \cup [13,\infty).$$
\rozwStop
\odpStart
$x \in [10,11] \cup [13,\infty)$
\odpStop
\testStart
A.$x \in [10,11] \cup [13,\infty)$\\
B.$x \in (10,11) \cup [13,\infty)$\\
C.$x \in (10,11] \cup [13,\infty)$\\
D.$x \in [10,11) \cup [13,\infty)$\\
E.$x \in [10,11] \cup (13,\infty)$\\
F.$x \in (10,11) \cup (13,\infty)$\\
G.$x \in [10,11) \cup (13,\infty)$\\
H.$x \in (10,11] \cup (13,\infty)$
\testStop
\kluczStart
A
\kluczStop



\zadStart{Zadanie z Wikieł Z 1.62 a) moja wersja nr 978}

Rozwiązać nierówności $(x-10)(x-11)(x-14)\ge0$.
\zadStop
\rozwStart{Patryk Wirkus}{}
Miejsca zerowe naszego wielomianu to: $10, 11, 14$.\\
Wielomian jest stopnia nieparzystego, ponadto znak współczynnika przy\linebreak najwyższej potędze x jest dodatni.\\ W związku z tym wykres wielomianu zaczyna się od lewej strony poniżej osi OX. A więc $$x \in [10,11] \cup [14,\infty).$$
\rozwStop
\odpStart
$x \in [10,11] \cup [14,\infty)$
\odpStop
\testStart
A.$x \in [10,11] \cup [14,\infty)$\\
B.$x \in (10,11) \cup [14,\infty)$\\
C.$x \in (10,11] \cup [14,\infty)$\\
D.$x \in [10,11) \cup [14,\infty)$\\
E.$x \in [10,11] \cup (14,\infty)$\\
F.$x \in (10,11) \cup (14,\infty)$\\
G.$x \in [10,11) \cup (14,\infty)$\\
H.$x \in (10,11] \cup (14,\infty)$
\testStop
\kluczStart
A
\kluczStop



\zadStart{Zadanie z Wikieł Z 1.62 a) moja wersja nr 979}

Rozwiązać nierówności $(x-10)(x-11)(x-15)\ge0$.
\zadStop
\rozwStart{Patryk Wirkus}{}
Miejsca zerowe naszego wielomianu to: $10, 11, 15$.\\
Wielomian jest stopnia nieparzystego, ponadto znak współczynnika przy\linebreak najwyższej potędze x jest dodatni.\\ W związku z tym wykres wielomianu zaczyna się od lewej strony poniżej osi OX. A więc $$x \in [10,11] \cup [15,\infty).$$
\rozwStop
\odpStart
$x \in [10,11] \cup [15,\infty)$
\odpStop
\testStart
A.$x \in [10,11] \cup [15,\infty)$\\
B.$x \in (10,11) \cup [15,\infty)$\\
C.$x \in (10,11] \cup [15,\infty)$\\
D.$x \in [10,11) \cup [15,\infty)$\\
E.$x \in [10,11] \cup (15,\infty)$\\
F.$x \in (10,11) \cup (15,\infty)$\\
G.$x \in [10,11) \cup (15,\infty)$\\
H.$x \in (10,11] \cup (15,\infty)$
\testStop
\kluczStart
A
\kluczStop



\zadStart{Zadanie z Wikieł Z 1.62 a) moja wersja nr 980}

Rozwiązać nierówności $(x-10)(x-11)(x-16)\ge0$.
\zadStop
\rozwStart{Patryk Wirkus}{}
Miejsca zerowe naszego wielomianu to: $10, 11, 16$.\\
Wielomian jest stopnia nieparzystego, ponadto znak współczynnika przy\linebreak najwyższej potędze x jest dodatni.\\ W związku z tym wykres wielomianu zaczyna się od lewej strony poniżej osi OX. A więc $$x \in [10,11] \cup [16,\infty).$$
\rozwStop
\odpStart
$x \in [10,11] \cup [16,\infty)$
\odpStop
\testStart
A.$x \in [10,11] \cup [16,\infty)$\\
B.$x \in (10,11) \cup [16,\infty)$\\
C.$x \in (10,11] \cup [16,\infty)$\\
D.$x \in [10,11) \cup [16,\infty)$\\
E.$x \in [10,11] \cup (16,\infty)$\\
F.$x \in (10,11) \cup (16,\infty)$\\
G.$x \in [10,11) \cup (16,\infty)$\\
H.$x \in (10,11] \cup (16,\infty)$
\testStop
\kluczStart
A
\kluczStop



\zadStart{Zadanie z Wikieł Z 1.62 a) moja wersja nr 981}

Rozwiązać nierówności $(x-10)(x-11)(x-17)\ge0$.
\zadStop
\rozwStart{Patryk Wirkus}{}
Miejsca zerowe naszego wielomianu to: $10, 11, 17$.\\
Wielomian jest stopnia nieparzystego, ponadto znak współczynnika przy\linebreak najwyższej potędze x jest dodatni.\\ W związku z tym wykres wielomianu zaczyna się od lewej strony poniżej osi OX. A więc $$x \in [10,11] \cup [17,\infty).$$
\rozwStop
\odpStart
$x \in [10,11] \cup [17,\infty)$
\odpStop
\testStart
A.$x \in [10,11] \cup [17,\infty)$\\
B.$x \in (10,11) \cup [17,\infty)$\\
C.$x \in (10,11] \cup [17,\infty)$\\
D.$x \in [10,11) \cup [17,\infty)$\\
E.$x \in [10,11] \cup (17,\infty)$\\
F.$x \in (10,11) \cup (17,\infty)$\\
G.$x \in [10,11) \cup (17,\infty)$\\
H.$x \in (10,11] \cup (17,\infty)$
\testStop
\kluczStart
A
\kluczStop



\zadStart{Zadanie z Wikieł Z 1.62 a) moja wersja nr 982}

Rozwiązać nierówności $(x-10)(x-11)(x-18)\ge0$.
\zadStop
\rozwStart{Patryk Wirkus}{}
Miejsca zerowe naszego wielomianu to: $10, 11, 18$.\\
Wielomian jest stopnia nieparzystego, ponadto znak współczynnika przy\linebreak najwyższej potędze x jest dodatni.\\ W związku z tym wykres wielomianu zaczyna się od lewej strony poniżej osi OX. A więc $$x \in [10,11] \cup [18,\infty).$$
\rozwStop
\odpStart
$x \in [10,11] \cup [18,\infty)$
\odpStop
\testStart
A.$x \in [10,11] \cup [18,\infty)$\\
B.$x \in (10,11) \cup [18,\infty)$\\
C.$x \in (10,11] \cup [18,\infty)$\\
D.$x \in [10,11) \cup [18,\infty)$\\
E.$x \in [10,11] \cup (18,\infty)$\\
F.$x \in (10,11) \cup (18,\infty)$\\
G.$x \in [10,11) \cup (18,\infty)$\\
H.$x \in (10,11] \cup (18,\infty)$
\testStop
\kluczStart
A
\kluczStop



\zadStart{Zadanie z Wikieł Z 1.62 a) moja wersja nr 983}

Rozwiązać nierówności $(x-10)(x-11)(x-19)\ge0$.
\zadStop
\rozwStart{Patryk Wirkus}{}
Miejsca zerowe naszego wielomianu to: $10, 11, 19$.\\
Wielomian jest stopnia nieparzystego, ponadto znak współczynnika przy\linebreak najwyższej potędze x jest dodatni.\\ W związku z tym wykres wielomianu zaczyna się od lewej strony poniżej osi OX. A więc $$x \in [10,11] \cup [19,\infty).$$
\rozwStop
\odpStart
$x \in [10,11] \cup [19,\infty)$
\odpStop
\testStart
A.$x \in [10,11] \cup [19,\infty)$\\
B.$x \in (10,11) \cup [19,\infty)$\\
C.$x \in (10,11] \cup [19,\infty)$\\
D.$x \in [10,11) \cup [19,\infty)$\\
E.$x \in [10,11] \cup (19,\infty)$\\
F.$x \in (10,11) \cup (19,\infty)$\\
G.$x \in [10,11) \cup (19,\infty)$\\
H.$x \in (10,11] \cup (19,\infty)$
\testStop
\kluczStart
A
\kluczStop



\zadStart{Zadanie z Wikieł Z 1.62 a) moja wersja nr 984}

Rozwiązać nierówności $(x-10)(x-11)(x-20)\ge0$.
\zadStop
\rozwStart{Patryk Wirkus}{}
Miejsca zerowe naszego wielomianu to: $10, 11, 20$.\\
Wielomian jest stopnia nieparzystego, ponadto znak współczynnika przy\linebreak najwyższej potędze x jest dodatni.\\ W związku z tym wykres wielomianu zaczyna się od lewej strony poniżej osi OX. A więc $$x \in [10,11] \cup [20,\infty).$$
\rozwStop
\odpStart
$x \in [10,11] \cup [20,\infty)$
\odpStop
\testStart
A.$x \in [10,11] \cup [20,\infty)$\\
B.$x \in (10,11) \cup [20,\infty)$\\
C.$x \in (10,11] \cup [20,\infty)$\\
D.$x \in [10,11) \cup [20,\infty)$\\
E.$x \in [10,11] \cup (20,\infty)$\\
F.$x \in (10,11) \cup (20,\infty)$\\
G.$x \in [10,11) \cup (20,\infty)$\\
H.$x \in (10,11] \cup (20,\infty)$
\testStop
\kluczStart
A
\kluczStop



\zadStart{Zadanie z Wikieł Z 1.62 a) moja wersja nr 985}

Rozwiązać nierówności $(x-10)(x-12)(x-13)\ge0$.
\zadStop
\rozwStart{Patryk Wirkus}{}
Miejsca zerowe naszego wielomianu to: $10, 12, 13$.\\
Wielomian jest stopnia nieparzystego, ponadto znak współczynnika przy\linebreak najwyższej potędze x jest dodatni.\\ W związku z tym wykres wielomianu zaczyna się od lewej strony poniżej osi OX. A więc $$x \in [10,12] \cup [13,\infty).$$
\rozwStop
\odpStart
$x \in [10,12] \cup [13,\infty)$
\odpStop
\testStart
A.$x \in [10,12] \cup [13,\infty)$\\
B.$x \in (10,12) \cup [13,\infty)$\\
C.$x \in (10,12] \cup [13,\infty)$\\
D.$x \in [10,12) \cup [13,\infty)$\\
E.$x \in [10,12] \cup (13,\infty)$\\
F.$x \in (10,12) \cup (13,\infty)$\\
G.$x \in [10,12) \cup (13,\infty)$\\
H.$x \in (10,12] \cup (13,\infty)$
\testStop
\kluczStart
A
\kluczStop



\zadStart{Zadanie z Wikieł Z 1.62 a) moja wersja nr 986}

Rozwiązać nierówności $(x-10)(x-12)(x-14)\ge0$.
\zadStop
\rozwStart{Patryk Wirkus}{}
Miejsca zerowe naszego wielomianu to: $10, 12, 14$.\\
Wielomian jest stopnia nieparzystego, ponadto znak współczynnika przy\linebreak najwyższej potędze x jest dodatni.\\ W związku z tym wykres wielomianu zaczyna się od lewej strony poniżej osi OX. A więc $$x \in [10,12] \cup [14,\infty).$$
\rozwStop
\odpStart
$x \in [10,12] \cup [14,\infty)$
\odpStop
\testStart
A.$x \in [10,12] \cup [14,\infty)$\\
B.$x \in (10,12) \cup [14,\infty)$\\
C.$x \in (10,12] \cup [14,\infty)$\\
D.$x \in [10,12) \cup [14,\infty)$\\
E.$x \in [10,12] \cup (14,\infty)$\\
F.$x \in (10,12) \cup (14,\infty)$\\
G.$x \in [10,12) \cup (14,\infty)$\\
H.$x \in (10,12] \cup (14,\infty)$
\testStop
\kluczStart
A
\kluczStop



\zadStart{Zadanie z Wikieł Z 1.62 a) moja wersja nr 987}

Rozwiązać nierówności $(x-10)(x-12)(x-15)\ge0$.
\zadStop
\rozwStart{Patryk Wirkus}{}
Miejsca zerowe naszego wielomianu to: $10, 12, 15$.\\
Wielomian jest stopnia nieparzystego, ponadto znak współczynnika przy\linebreak najwyższej potędze x jest dodatni.\\ W związku z tym wykres wielomianu zaczyna się od lewej strony poniżej osi OX. A więc $$x \in [10,12] \cup [15,\infty).$$
\rozwStop
\odpStart
$x \in [10,12] \cup [15,\infty)$
\odpStop
\testStart
A.$x \in [10,12] \cup [15,\infty)$\\
B.$x \in (10,12) \cup [15,\infty)$\\
C.$x \in (10,12] \cup [15,\infty)$\\
D.$x \in [10,12) \cup [15,\infty)$\\
E.$x \in [10,12] \cup (15,\infty)$\\
F.$x \in (10,12) \cup (15,\infty)$\\
G.$x \in [10,12) \cup (15,\infty)$\\
H.$x \in (10,12] \cup (15,\infty)$
\testStop
\kluczStart
A
\kluczStop



\zadStart{Zadanie z Wikieł Z 1.62 a) moja wersja nr 988}

Rozwiązać nierówności $(x-10)(x-12)(x-16)\ge0$.
\zadStop
\rozwStart{Patryk Wirkus}{}
Miejsca zerowe naszego wielomianu to: $10, 12, 16$.\\
Wielomian jest stopnia nieparzystego, ponadto znak współczynnika przy\linebreak najwyższej potędze x jest dodatni.\\ W związku z tym wykres wielomianu zaczyna się od lewej strony poniżej osi OX. A więc $$x \in [10,12] \cup [16,\infty).$$
\rozwStop
\odpStart
$x \in [10,12] \cup [16,\infty)$
\odpStop
\testStart
A.$x \in [10,12] \cup [16,\infty)$\\
B.$x \in (10,12) \cup [16,\infty)$\\
C.$x \in (10,12] \cup [16,\infty)$\\
D.$x \in [10,12) \cup [16,\infty)$\\
E.$x \in [10,12] \cup (16,\infty)$\\
F.$x \in (10,12) \cup (16,\infty)$\\
G.$x \in [10,12) \cup (16,\infty)$\\
H.$x \in (10,12] \cup (16,\infty)$
\testStop
\kluczStart
A
\kluczStop



\zadStart{Zadanie z Wikieł Z 1.62 a) moja wersja nr 989}

Rozwiązać nierówności $(x-10)(x-12)(x-17)\ge0$.
\zadStop
\rozwStart{Patryk Wirkus}{}
Miejsca zerowe naszego wielomianu to: $10, 12, 17$.\\
Wielomian jest stopnia nieparzystego, ponadto znak współczynnika przy\linebreak najwyższej potędze x jest dodatni.\\ W związku z tym wykres wielomianu zaczyna się od lewej strony poniżej osi OX. A więc $$x \in [10,12] \cup [17,\infty).$$
\rozwStop
\odpStart
$x \in [10,12] \cup [17,\infty)$
\odpStop
\testStart
A.$x \in [10,12] \cup [17,\infty)$\\
B.$x \in (10,12) \cup [17,\infty)$\\
C.$x \in (10,12] \cup [17,\infty)$\\
D.$x \in [10,12) \cup [17,\infty)$\\
E.$x \in [10,12] \cup (17,\infty)$\\
F.$x \in (10,12) \cup (17,\infty)$\\
G.$x \in [10,12) \cup (17,\infty)$\\
H.$x \in (10,12] \cup (17,\infty)$
\testStop
\kluczStart
A
\kluczStop



\zadStart{Zadanie z Wikieł Z 1.62 a) moja wersja nr 990}

Rozwiązać nierówności $(x-10)(x-12)(x-18)\ge0$.
\zadStop
\rozwStart{Patryk Wirkus}{}
Miejsca zerowe naszego wielomianu to: $10, 12, 18$.\\
Wielomian jest stopnia nieparzystego, ponadto znak współczynnika przy\linebreak najwyższej potędze x jest dodatni.\\ W związku z tym wykres wielomianu zaczyna się od lewej strony poniżej osi OX. A więc $$x \in [10,12] \cup [18,\infty).$$
\rozwStop
\odpStart
$x \in [10,12] \cup [18,\infty)$
\odpStop
\testStart
A.$x \in [10,12] \cup [18,\infty)$\\
B.$x \in (10,12) \cup [18,\infty)$\\
C.$x \in (10,12] \cup [18,\infty)$\\
D.$x \in [10,12) \cup [18,\infty)$\\
E.$x \in [10,12] \cup (18,\infty)$\\
F.$x \in (10,12) \cup (18,\infty)$\\
G.$x \in [10,12) \cup (18,\infty)$\\
H.$x \in (10,12] \cup (18,\infty)$
\testStop
\kluczStart
A
\kluczStop



\zadStart{Zadanie z Wikieł Z 1.62 a) moja wersja nr 991}

Rozwiązać nierówności $(x-10)(x-12)(x-19)\ge0$.
\zadStop
\rozwStart{Patryk Wirkus}{}
Miejsca zerowe naszego wielomianu to: $10, 12, 19$.\\
Wielomian jest stopnia nieparzystego, ponadto znak współczynnika przy\linebreak najwyższej potędze x jest dodatni.\\ W związku z tym wykres wielomianu zaczyna się od lewej strony poniżej osi OX. A więc $$x \in [10,12] \cup [19,\infty).$$
\rozwStop
\odpStart
$x \in [10,12] \cup [19,\infty)$
\odpStop
\testStart
A.$x \in [10,12] \cup [19,\infty)$\\
B.$x \in (10,12) \cup [19,\infty)$\\
C.$x \in (10,12] \cup [19,\infty)$\\
D.$x \in [10,12) \cup [19,\infty)$\\
E.$x \in [10,12] \cup (19,\infty)$\\
F.$x \in (10,12) \cup (19,\infty)$\\
G.$x \in [10,12) \cup (19,\infty)$\\
H.$x \in (10,12] \cup (19,\infty)$
\testStop
\kluczStart
A
\kluczStop



\zadStart{Zadanie z Wikieł Z 1.62 a) moja wersja nr 992}

Rozwiązać nierówności $(x-10)(x-12)(x-20)\ge0$.
\zadStop
\rozwStart{Patryk Wirkus}{}
Miejsca zerowe naszego wielomianu to: $10, 12, 20$.\\
Wielomian jest stopnia nieparzystego, ponadto znak współczynnika przy\linebreak najwyższej potędze x jest dodatni.\\ W związku z tym wykres wielomianu zaczyna się od lewej strony poniżej osi OX. A więc $$x \in [10,12] \cup [20,\infty).$$
\rozwStop
\odpStart
$x \in [10,12] \cup [20,\infty)$
\odpStop
\testStart
A.$x \in [10,12] \cup [20,\infty)$\\
B.$x \in (10,12) \cup [20,\infty)$\\
C.$x \in (10,12] \cup [20,\infty)$\\
D.$x \in [10,12) \cup [20,\infty)$\\
E.$x \in [10,12] \cup (20,\infty)$\\
F.$x \in (10,12) \cup (20,\infty)$\\
G.$x \in [10,12) \cup (20,\infty)$\\
H.$x \in (10,12] \cup (20,\infty)$
\testStop
\kluczStart
A
\kluczStop



\zadStart{Zadanie z Wikieł Z 1.62 a) moja wersja nr 993}

Rozwiązać nierówności $(x-10)(x-13)(x-14)\ge0$.
\zadStop
\rozwStart{Patryk Wirkus}{}
Miejsca zerowe naszego wielomianu to: $10, 13, 14$.\\
Wielomian jest stopnia nieparzystego, ponadto znak współczynnika przy\linebreak najwyższej potędze x jest dodatni.\\ W związku z tym wykres wielomianu zaczyna się od lewej strony poniżej osi OX. A więc $$x \in [10,13] \cup [14,\infty).$$
\rozwStop
\odpStart
$x \in [10,13] \cup [14,\infty)$
\odpStop
\testStart
A.$x \in [10,13] \cup [14,\infty)$\\
B.$x \in (10,13) \cup [14,\infty)$\\
C.$x \in (10,13] \cup [14,\infty)$\\
D.$x \in [10,13) \cup [14,\infty)$\\
E.$x \in [10,13] \cup (14,\infty)$\\
F.$x \in (10,13) \cup (14,\infty)$\\
G.$x \in [10,13) \cup (14,\infty)$\\
H.$x \in (10,13] \cup (14,\infty)$
\testStop
\kluczStart
A
\kluczStop



\zadStart{Zadanie z Wikieł Z 1.62 a) moja wersja nr 994}

Rozwiązać nierówności $(x-10)(x-13)(x-15)\ge0$.
\zadStop
\rozwStart{Patryk Wirkus}{}
Miejsca zerowe naszego wielomianu to: $10, 13, 15$.\\
Wielomian jest stopnia nieparzystego, ponadto znak współczynnika przy\linebreak najwyższej potędze x jest dodatni.\\ W związku z tym wykres wielomianu zaczyna się od lewej strony poniżej osi OX. A więc $$x \in [10,13] \cup [15,\infty).$$
\rozwStop
\odpStart
$x \in [10,13] \cup [15,\infty)$
\odpStop
\testStart
A.$x \in [10,13] \cup [15,\infty)$\\
B.$x \in (10,13) \cup [15,\infty)$\\
C.$x \in (10,13] \cup [15,\infty)$\\
D.$x \in [10,13) \cup [15,\infty)$\\
E.$x \in [10,13] \cup (15,\infty)$\\
F.$x \in (10,13) \cup (15,\infty)$\\
G.$x \in [10,13) \cup (15,\infty)$\\
H.$x \in (10,13] \cup (15,\infty)$
\testStop
\kluczStart
A
\kluczStop



\zadStart{Zadanie z Wikieł Z 1.62 a) moja wersja nr 995}

Rozwiązać nierówności $(x-10)(x-13)(x-16)\ge0$.
\zadStop
\rozwStart{Patryk Wirkus}{}
Miejsca zerowe naszego wielomianu to: $10, 13, 16$.\\
Wielomian jest stopnia nieparzystego, ponadto znak współczynnika przy\linebreak najwyższej potędze x jest dodatni.\\ W związku z tym wykres wielomianu zaczyna się od lewej strony poniżej osi OX. A więc $$x \in [10,13] \cup [16,\infty).$$
\rozwStop
\odpStart
$x \in [10,13] \cup [16,\infty)$
\odpStop
\testStart
A.$x \in [10,13] \cup [16,\infty)$\\
B.$x \in (10,13) \cup [16,\infty)$\\
C.$x \in (10,13] \cup [16,\infty)$\\
D.$x \in [10,13) \cup [16,\infty)$\\
E.$x \in [10,13] \cup (16,\infty)$\\
F.$x \in (10,13) \cup (16,\infty)$\\
G.$x \in [10,13) \cup (16,\infty)$\\
H.$x \in (10,13] \cup (16,\infty)$
\testStop
\kluczStart
A
\kluczStop



\zadStart{Zadanie z Wikieł Z 1.62 a) moja wersja nr 996}

Rozwiązać nierówności $(x-10)(x-13)(x-17)\ge0$.
\zadStop
\rozwStart{Patryk Wirkus}{}
Miejsca zerowe naszego wielomianu to: $10, 13, 17$.\\
Wielomian jest stopnia nieparzystego, ponadto znak współczynnika przy\linebreak najwyższej potędze x jest dodatni.\\ W związku z tym wykres wielomianu zaczyna się od lewej strony poniżej osi OX. A więc $$x \in [10,13] \cup [17,\infty).$$
\rozwStop
\odpStart
$x \in [10,13] \cup [17,\infty)$
\odpStop
\testStart
A.$x \in [10,13] \cup [17,\infty)$\\
B.$x \in (10,13) \cup [17,\infty)$\\
C.$x \in (10,13] \cup [17,\infty)$\\
D.$x \in [10,13) \cup [17,\infty)$\\
E.$x \in [10,13] \cup (17,\infty)$\\
F.$x \in (10,13) \cup (17,\infty)$\\
G.$x \in [10,13) \cup (17,\infty)$\\
H.$x \in (10,13] \cup (17,\infty)$
\testStop
\kluczStart
A
\kluczStop



\zadStart{Zadanie z Wikieł Z 1.62 a) moja wersja nr 997}

Rozwiązać nierówności $(x-10)(x-13)(x-18)\ge0$.
\zadStop
\rozwStart{Patryk Wirkus}{}
Miejsca zerowe naszego wielomianu to: $10, 13, 18$.\\
Wielomian jest stopnia nieparzystego, ponadto znak współczynnika przy\linebreak najwyższej potędze x jest dodatni.\\ W związku z tym wykres wielomianu zaczyna się od lewej strony poniżej osi OX. A więc $$x \in [10,13] \cup [18,\infty).$$
\rozwStop
\odpStart
$x \in [10,13] \cup [18,\infty)$
\odpStop
\testStart
A.$x \in [10,13] \cup [18,\infty)$\\
B.$x \in (10,13) \cup [18,\infty)$\\
C.$x \in (10,13] \cup [18,\infty)$\\
D.$x \in [10,13) \cup [18,\infty)$\\
E.$x \in [10,13] \cup (18,\infty)$\\
F.$x \in (10,13) \cup (18,\infty)$\\
G.$x \in [10,13) \cup (18,\infty)$\\
H.$x \in (10,13] \cup (18,\infty)$
\testStop
\kluczStart
A
\kluczStop



\zadStart{Zadanie z Wikieł Z 1.62 a) moja wersja nr 998}

Rozwiązać nierówności $(x-10)(x-13)(x-19)\ge0$.
\zadStop
\rozwStart{Patryk Wirkus}{}
Miejsca zerowe naszego wielomianu to: $10, 13, 19$.\\
Wielomian jest stopnia nieparzystego, ponadto znak współczynnika przy\linebreak najwyższej potędze x jest dodatni.\\ W związku z tym wykres wielomianu zaczyna się od lewej strony poniżej osi OX. A więc $$x \in [10,13] \cup [19,\infty).$$
\rozwStop
\odpStart
$x \in [10,13] \cup [19,\infty)$
\odpStop
\testStart
A.$x \in [10,13] \cup [19,\infty)$\\
B.$x \in (10,13) \cup [19,\infty)$\\
C.$x \in (10,13] \cup [19,\infty)$\\
D.$x \in [10,13) \cup [19,\infty)$\\
E.$x \in [10,13] \cup (19,\infty)$\\
F.$x \in (10,13) \cup (19,\infty)$\\
G.$x \in [10,13) \cup (19,\infty)$\\
H.$x \in (10,13] \cup (19,\infty)$
\testStop
\kluczStart
A
\kluczStop



\zadStart{Zadanie z Wikieł Z 1.62 a) moja wersja nr 999}

Rozwiązać nierówności $(x-10)(x-13)(x-20)\ge0$.
\zadStop
\rozwStart{Patryk Wirkus}{}
Miejsca zerowe naszego wielomianu to: $10, 13, 20$.\\
Wielomian jest stopnia nieparzystego, ponadto znak współczynnika przy\linebreak najwyższej potędze x jest dodatni.\\ W związku z tym wykres wielomianu zaczyna się od lewej strony poniżej osi OX. A więc $$x \in [10,13] \cup [20,\infty).$$
\rozwStop
\odpStart
$x \in [10,13] \cup [20,\infty)$
\odpStop
\testStart
A.$x \in [10,13] \cup [20,\infty)$\\
B.$x \in (10,13) \cup [20,\infty)$\\
C.$x \in (10,13] \cup [20,\infty)$\\
D.$x \in [10,13) \cup [20,\infty)$\\
E.$x \in [10,13] \cup (20,\infty)$\\
F.$x \in (10,13) \cup (20,\infty)$\\
G.$x \in [10,13) \cup (20,\infty)$\\
H.$x \in (10,13] \cup (20,\infty)$
\testStop
\kluczStart
A
\kluczStop



\zadStart{Zadanie z Wikieł Z 1.62 a) moja wersja nr 1000}

Rozwiązać nierówności $(x-10)(x-14)(x-15)\ge0$.
\zadStop
\rozwStart{Patryk Wirkus}{}
Miejsca zerowe naszego wielomianu to: $10, 14, 15$.\\
Wielomian jest stopnia nieparzystego, ponadto znak współczynnika przy\linebreak najwyższej potędze x jest dodatni.\\ W związku z tym wykres wielomianu zaczyna się od lewej strony poniżej osi OX. A więc $$x \in [10,14] \cup [15,\infty).$$
\rozwStop
\odpStart
$x \in [10,14] \cup [15,\infty)$
\odpStop
\testStart
A.$x \in [10,14] \cup [15,\infty)$\\
B.$x \in (10,14) \cup [15,\infty)$\\
C.$x \in (10,14] \cup [15,\infty)$\\
D.$x \in [10,14) \cup [15,\infty)$\\
E.$x \in [10,14] \cup (15,\infty)$\\
F.$x \in (10,14) \cup (15,\infty)$\\
G.$x \in [10,14) \cup (15,\infty)$\\
H.$x \in (10,14] \cup (15,\infty)$
\testStop
\kluczStart
A
\kluczStop



\zadStart{Zadanie z Wikieł Z 1.62 a) moja wersja nr 1001}

Rozwiązać nierówności $(x-10)(x-14)(x-16)\ge0$.
\zadStop
\rozwStart{Patryk Wirkus}{}
Miejsca zerowe naszego wielomianu to: $10, 14, 16$.\\
Wielomian jest stopnia nieparzystego, ponadto znak współczynnika przy\linebreak najwyższej potędze x jest dodatni.\\ W związku z tym wykres wielomianu zaczyna się od lewej strony poniżej osi OX. A więc $$x \in [10,14] \cup [16,\infty).$$
\rozwStop
\odpStart
$x \in [10,14] \cup [16,\infty)$
\odpStop
\testStart
A.$x \in [10,14] \cup [16,\infty)$\\
B.$x \in (10,14) \cup [16,\infty)$\\
C.$x \in (10,14] \cup [16,\infty)$\\
D.$x \in [10,14) \cup [16,\infty)$\\
E.$x \in [10,14] \cup (16,\infty)$\\
F.$x \in (10,14) \cup (16,\infty)$\\
G.$x \in [10,14) \cup (16,\infty)$\\
H.$x \in (10,14] \cup (16,\infty)$
\testStop
\kluczStart
A
\kluczStop



\zadStart{Zadanie z Wikieł Z 1.62 a) moja wersja nr 1002}

Rozwiązać nierówności $(x-10)(x-14)(x-17)\ge0$.
\zadStop
\rozwStart{Patryk Wirkus}{}
Miejsca zerowe naszego wielomianu to: $10, 14, 17$.\\
Wielomian jest stopnia nieparzystego, ponadto znak współczynnika przy\linebreak najwyższej potędze x jest dodatni.\\ W związku z tym wykres wielomianu zaczyna się od lewej strony poniżej osi OX. A więc $$x \in [10,14] \cup [17,\infty).$$
\rozwStop
\odpStart
$x \in [10,14] \cup [17,\infty)$
\odpStop
\testStart
A.$x \in [10,14] \cup [17,\infty)$\\
B.$x \in (10,14) \cup [17,\infty)$\\
C.$x \in (10,14] \cup [17,\infty)$\\
D.$x \in [10,14) \cup [17,\infty)$\\
E.$x \in [10,14] \cup (17,\infty)$\\
F.$x \in (10,14) \cup (17,\infty)$\\
G.$x \in [10,14) \cup (17,\infty)$\\
H.$x \in (10,14] \cup (17,\infty)$
\testStop
\kluczStart
A
\kluczStop



\zadStart{Zadanie z Wikieł Z 1.62 a) moja wersja nr 1003}

Rozwiązać nierówności $(x-10)(x-14)(x-18)\ge0$.
\zadStop
\rozwStart{Patryk Wirkus}{}
Miejsca zerowe naszego wielomianu to: $10, 14, 18$.\\
Wielomian jest stopnia nieparzystego, ponadto znak współczynnika przy\linebreak najwyższej potędze x jest dodatni.\\ W związku z tym wykres wielomianu zaczyna się od lewej strony poniżej osi OX. A więc $$x \in [10,14] \cup [18,\infty).$$
\rozwStop
\odpStart
$x \in [10,14] \cup [18,\infty)$
\odpStop
\testStart
A.$x \in [10,14] \cup [18,\infty)$\\
B.$x \in (10,14) \cup [18,\infty)$\\
C.$x \in (10,14] \cup [18,\infty)$\\
D.$x \in [10,14) \cup [18,\infty)$\\
E.$x \in [10,14] \cup (18,\infty)$\\
F.$x \in (10,14) \cup (18,\infty)$\\
G.$x \in [10,14) \cup (18,\infty)$\\
H.$x \in (10,14] \cup (18,\infty)$
\testStop
\kluczStart
A
\kluczStop



\zadStart{Zadanie z Wikieł Z 1.62 a) moja wersja nr 1004}

Rozwiązać nierówności $(x-10)(x-14)(x-19)\ge0$.
\zadStop
\rozwStart{Patryk Wirkus}{}
Miejsca zerowe naszego wielomianu to: $10, 14, 19$.\\
Wielomian jest stopnia nieparzystego, ponadto znak współczynnika przy\linebreak najwyższej potędze x jest dodatni.\\ W związku z tym wykres wielomianu zaczyna się od lewej strony poniżej osi OX. A więc $$x \in [10,14] \cup [19,\infty).$$
\rozwStop
\odpStart
$x \in [10,14] \cup [19,\infty)$
\odpStop
\testStart
A.$x \in [10,14] \cup [19,\infty)$\\
B.$x \in (10,14) \cup [19,\infty)$\\
C.$x \in (10,14] \cup [19,\infty)$\\
D.$x \in [10,14) \cup [19,\infty)$\\
E.$x \in [10,14] \cup (19,\infty)$\\
F.$x \in (10,14) \cup (19,\infty)$\\
G.$x \in [10,14) \cup (19,\infty)$\\
H.$x \in (10,14] \cup (19,\infty)$
\testStop
\kluczStart
A
\kluczStop



\zadStart{Zadanie z Wikieł Z 1.62 a) moja wersja nr 1005}

Rozwiązać nierówności $(x-10)(x-14)(x-20)\ge0$.
\zadStop
\rozwStart{Patryk Wirkus}{}
Miejsca zerowe naszego wielomianu to: $10, 14, 20$.\\
Wielomian jest stopnia nieparzystego, ponadto znak współczynnika przy\linebreak najwyższej potędze x jest dodatni.\\ W związku z tym wykres wielomianu zaczyna się od lewej strony poniżej osi OX. A więc $$x \in [10,14] \cup [20,\infty).$$
\rozwStop
\odpStart
$x \in [10,14] \cup [20,\infty)$
\odpStop
\testStart
A.$x \in [10,14] \cup [20,\infty)$\\
B.$x \in (10,14) \cup [20,\infty)$\\
C.$x \in (10,14] \cup [20,\infty)$\\
D.$x \in [10,14) \cup [20,\infty)$\\
E.$x \in [10,14] \cup (20,\infty)$\\
F.$x \in (10,14) \cup (20,\infty)$\\
G.$x \in [10,14) \cup (20,\infty)$\\
H.$x \in (10,14] \cup (20,\infty)$
\testStop
\kluczStart
A
\kluczStop



\zadStart{Zadanie z Wikieł Z 1.62 a) moja wersja nr 1006}

Rozwiązać nierówności $(x-10)(x-15)(x-16)\ge0$.
\zadStop
\rozwStart{Patryk Wirkus}{}
Miejsca zerowe naszego wielomianu to: $10, 15, 16$.\\
Wielomian jest stopnia nieparzystego, ponadto znak współczynnika przy\linebreak najwyższej potędze x jest dodatni.\\ W związku z tym wykres wielomianu zaczyna się od lewej strony poniżej osi OX. A więc $$x \in [10,15] \cup [16,\infty).$$
\rozwStop
\odpStart
$x \in [10,15] \cup [16,\infty)$
\odpStop
\testStart
A.$x \in [10,15] \cup [16,\infty)$\\
B.$x \in (10,15) \cup [16,\infty)$\\
C.$x \in (10,15] \cup [16,\infty)$\\
D.$x \in [10,15) \cup [16,\infty)$\\
E.$x \in [10,15] \cup (16,\infty)$\\
F.$x \in (10,15) \cup (16,\infty)$\\
G.$x \in [10,15) \cup (16,\infty)$\\
H.$x \in (10,15] \cup (16,\infty)$
\testStop
\kluczStart
A
\kluczStop



\zadStart{Zadanie z Wikieł Z 1.62 a) moja wersja nr 1007}

Rozwiązać nierówności $(x-10)(x-15)(x-17)\ge0$.
\zadStop
\rozwStart{Patryk Wirkus}{}
Miejsca zerowe naszego wielomianu to: $10, 15, 17$.\\
Wielomian jest stopnia nieparzystego, ponadto znak współczynnika przy\linebreak najwyższej potędze x jest dodatni.\\ W związku z tym wykres wielomianu zaczyna się od lewej strony poniżej osi OX. A więc $$x \in [10,15] \cup [17,\infty).$$
\rozwStop
\odpStart
$x \in [10,15] \cup [17,\infty)$
\odpStop
\testStart
A.$x \in [10,15] \cup [17,\infty)$\\
B.$x \in (10,15) \cup [17,\infty)$\\
C.$x \in (10,15] \cup [17,\infty)$\\
D.$x \in [10,15) \cup [17,\infty)$\\
E.$x \in [10,15] \cup (17,\infty)$\\
F.$x \in (10,15) \cup (17,\infty)$\\
G.$x \in [10,15) \cup (17,\infty)$\\
H.$x \in (10,15] \cup (17,\infty)$
\testStop
\kluczStart
A
\kluczStop



\zadStart{Zadanie z Wikieł Z 1.62 a) moja wersja nr 1008}

Rozwiązać nierówności $(x-10)(x-15)(x-18)\ge0$.
\zadStop
\rozwStart{Patryk Wirkus}{}
Miejsca zerowe naszego wielomianu to: $10, 15, 18$.\\
Wielomian jest stopnia nieparzystego, ponadto znak współczynnika przy\linebreak najwyższej potędze x jest dodatni.\\ W związku z tym wykres wielomianu zaczyna się od lewej strony poniżej osi OX. A więc $$x \in [10,15] \cup [18,\infty).$$
\rozwStop
\odpStart
$x \in [10,15] \cup [18,\infty)$
\odpStop
\testStart
A.$x \in [10,15] \cup [18,\infty)$\\
B.$x \in (10,15) \cup [18,\infty)$\\
C.$x \in (10,15] \cup [18,\infty)$\\
D.$x \in [10,15) \cup [18,\infty)$\\
E.$x \in [10,15] \cup (18,\infty)$\\
F.$x \in (10,15) \cup (18,\infty)$\\
G.$x \in [10,15) \cup (18,\infty)$\\
H.$x \in (10,15] \cup (18,\infty)$
\testStop
\kluczStart
A
\kluczStop



\zadStart{Zadanie z Wikieł Z 1.62 a) moja wersja nr 1009}

Rozwiązać nierówności $(x-10)(x-15)(x-19)\ge0$.
\zadStop
\rozwStart{Patryk Wirkus}{}
Miejsca zerowe naszego wielomianu to: $10, 15, 19$.\\
Wielomian jest stopnia nieparzystego, ponadto znak współczynnika przy\linebreak najwyższej potędze x jest dodatni.\\ W związku z tym wykres wielomianu zaczyna się od lewej strony poniżej osi OX. A więc $$x \in [10,15] \cup [19,\infty).$$
\rozwStop
\odpStart
$x \in [10,15] \cup [19,\infty)$
\odpStop
\testStart
A.$x \in [10,15] \cup [19,\infty)$\\
B.$x \in (10,15) \cup [19,\infty)$\\
C.$x \in (10,15] \cup [19,\infty)$\\
D.$x \in [10,15) \cup [19,\infty)$\\
E.$x \in [10,15] \cup (19,\infty)$\\
F.$x \in (10,15) \cup (19,\infty)$\\
G.$x \in [10,15) \cup (19,\infty)$\\
H.$x \in (10,15] \cup (19,\infty)$
\testStop
\kluczStart
A
\kluczStop



\zadStart{Zadanie z Wikieł Z 1.62 a) moja wersja nr 1010}

Rozwiązać nierówności $(x-10)(x-15)(x-20)\ge0$.
\zadStop
\rozwStart{Patryk Wirkus}{}
Miejsca zerowe naszego wielomianu to: $10, 15, 20$.\\
Wielomian jest stopnia nieparzystego, ponadto znak współczynnika przy\linebreak najwyższej potędze x jest dodatni.\\ W związku z tym wykres wielomianu zaczyna się od lewej strony poniżej osi OX. A więc $$x \in [10,15] \cup [20,\infty).$$
\rozwStop
\odpStart
$x \in [10,15] \cup [20,\infty)$
\odpStop
\testStart
A.$x \in [10,15] \cup [20,\infty)$\\
B.$x \in (10,15) \cup [20,\infty)$\\
C.$x \in (10,15] \cup [20,\infty)$\\
D.$x \in [10,15) \cup [20,\infty)$\\
E.$x \in [10,15] \cup (20,\infty)$\\
F.$x \in (10,15) \cup (20,\infty)$\\
G.$x \in [10,15) \cup (20,\infty)$\\
H.$x \in (10,15] \cup (20,\infty)$
\testStop
\kluczStart
A
\kluczStop



\zadStart{Zadanie z Wikieł Z 1.62 a) moja wersja nr 1011}

Rozwiązać nierówności $(x-10)(x-16)(x-17)\ge0$.
\zadStop
\rozwStart{Patryk Wirkus}{}
Miejsca zerowe naszego wielomianu to: $10, 16, 17$.\\
Wielomian jest stopnia nieparzystego, ponadto znak współczynnika przy\linebreak najwyższej potędze x jest dodatni.\\ W związku z tym wykres wielomianu zaczyna się od lewej strony poniżej osi OX. A więc $$x \in [10,16] \cup [17,\infty).$$
\rozwStop
\odpStart
$x \in [10,16] \cup [17,\infty)$
\odpStop
\testStart
A.$x \in [10,16] \cup [17,\infty)$\\
B.$x \in (10,16) \cup [17,\infty)$\\
C.$x \in (10,16] \cup [17,\infty)$\\
D.$x \in [10,16) \cup [17,\infty)$\\
E.$x \in [10,16] \cup (17,\infty)$\\
F.$x \in (10,16) \cup (17,\infty)$\\
G.$x \in [10,16) \cup (17,\infty)$\\
H.$x \in (10,16] \cup (17,\infty)$
\testStop
\kluczStart
A
\kluczStop



\zadStart{Zadanie z Wikieł Z 1.62 a) moja wersja nr 1012}

Rozwiązać nierówności $(x-10)(x-16)(x-18)\ge0$.
\zadStop
\rozwStart{Patryk Wirkus}{}
Miejsca zerowe naszego wielomianu to: $10, 16, 18$.\\
Wielomian jest stopnia nieparzystego, ponadto znak współczynnika przy\linebreak najwyższej potędze x jest dodatni.\\ W związku z tym wykres wielomianu zaczyna się od lewej strony poniżej osi OX. A więc $$x \in [10,16] \cup [18,\infty).$$
\rozwStop
\odpStart
$x \in [10,16] \cup [18,\infty)$
\odpStop
\testStart
A.$x \in [10,16] \cup [18,\infty)$\\
B.$x \in (10,16) \cup [18,\infty)$\\
C.$x \in (10,16] \cup [18,\infty)$\\
D.$x \in [10,16) \cup [18,\infty)$\\
E.$x \in [10,16] \cup (18,\infty)$\\
F.$x \in (10,16) \cup (18,\infty)$\\
G.$x \in [10,16) \cup (18,\infty)$\\
H.$x \in (10,16] \cup (18,\infty)$
\testStop
\kluczStart
A
\kluczStop



\zadStart{Zadanie z Wikieł Z 1.62 a) moja wersja nr 1013}

Rozwiązać nierówności $(x-10)(x-16)(x-19)\ge0$.
\zadStop
\rozwStart{Patryk Wirkus}{}
Miejsca zerowe naszego wielomianu to: $10, 16, 19$.\\
Wielomian jest stopnia nieparzystego, ponadto znak współczynnika przy\linebreak najwyższej potędze x jest dodatni.\\ W związku z tym wykres wielomianu zaczyna się od lewej strony poniżej osi OX. A więc $$x \in [10,16] \cup [19,\infty).$$
\rozwStop
\odpStart
$x \in [10,16] \cup [19,\infty)$
\odpStop
\testStart
A.$x \in [10,16] \cup [19,\infty)$\\
B.$x \in (10,16) \cup [19,\infty)$\\
C.$x \in (10,16] \cup [19,\infty)$\\
D.$x \in [10,16) \cup [19,\infty)$\\
E.$x \in [10,16] \cup (19,\infty)$\\
F.$x \in (10,16) \cup (19,\infty)$\\
G.$x \in [10,16) \cup (19,\infty)$\\
H.$x \in (10,16] \cup (19,\infty)$
\testStop
\kluczStart
A
\kluczStop



\zadStart{Zadanie z Wikieł Z 1.62 a) moja wersja nr 1014}

Rozwiązać nierówności $(x-10)(x-16)(x-20)\ge0$.
\zadStop
\rozwStart{Patryk Wirkus}{}
Miejsca zerowe naszego wielomianu to: $10, 16, 20$.\\
Wielomian jest stopnia nieparzystego, ponadto znak współczynnika przy\linebreak najwyższej potędze x jest dodatni.\\ W związku z tym wykres wielomianu zaczyna się od lewej strony poniżej osi OX. A więc $$x \in [10,16] \cup [20,\infty).$$
\rozwStop
\odpStart
$x \in [10,16] \cup [20,\infty)$
\odpStop
\testStart
A.$x \in [10,16] \cup [20,\infty)$\\
B.$x \in (10,16) \cup [20,\infty)$\\
C.$x \in (10,16] \cup [20,\infty)$\\
D.$x \in [10,16) \cup [20,\infty)$\\
E.$x \in [10,16] \cup (20,\infty)$\\
F.$x \in (10,16) \cup (20,\infty)$\\
G.$x \in [10,16) \cup (20,\infty)$\\
H.$x \in (10,16] \cup (20,\infty)$
\testStop
\kluczStart
A
\kluczStop



\zadStart{Zadanie z Wikieł Z 1.62 a) moja wersja nr 1015}

Rozwiązać nierówności $(x-10)(x-17)(x-18)\ge0$.
\zadStop
\rozwStart{Patryk Wirkus}{}
Miejsca zerowe naszego wielomianu to: $10, 17, 18$.\\
Wielomian jest stopnia nieparzystego, ponadto znak współczynnika przy\linebreak najwyższej potędze x jest dodatni.\\ W związku z tym wykres wielomianu zaczyna się od lewej strony poniżej osi OX. A więc $$x \in [10,17] \cup [18,\infty).$$
\rozwStop
\odpStart
$x \in [10,17] \cup [18,\infty)$
\odpStop
\testStart
A.$x \in [10,17] \cup [18,\infty)$\\
B.$x \in (10,17) \cup [18,\infty)$\\
C.$x \in (10,17] \cup [18,\infty)$\\
D.$x \in [10,17) \cup [18,\infty)$\\
E.$x \in [10,17] \cup (18,\infty)$\\
F.$x \in (10,17) \cup (18,\infty)$\\
G.$x \in [10,17) \cup (18,\infty)$\\
H.$x \in (10,17] \cup (18,\infty)$
\testStop
\kluczStart
A
\kluczStop



\zadStart{Zadanie z Wikieł Z 1.62 a) moja wersja nr 1016}

Rozwiązać nierówności $(x-10)(x-17)(x-19)\ge0$.
\zadStop
\rozwStart{Patryk Wirkus}{}
Miejsca zerowe naszego wielomianu to: $10, 17, 19$.\\
Wielomian jest stopnia nieparzystego, ponadto znak współczynnika przy\linebreak najwyższej potędze x jest dodatni.\\ W związku z tym wykres wielomianu zaczyna się od lewej strony poniżej osi OX. A więc $$x \in [10,17] \cup [19,\infty).$$
\rozwStop
\odpStart
$x \in [10,17] \cup [19,\infty)$
\odpStop
\testStart
A.$x \in [10,17] \cup [19,\infty)$\\
B.$x \in (10,17) \cup [19,\infty)$\\
C.$x \in (10,17] \cup [19,\infty)$\\
D.$x \in [10,17) \cup [19,\infty)$\\
E.$x \in [10,17] \cup (19,\infty)$\\
F.$x \in (10,17) \cup (19,\infty)$\\
G.$x \in [10,17) \cup (19,\infty)$\\
H.$x \in (10,17] \cup (19,\infty)$
\testStop
\kluczStart
A
\kluczStop



\zadStart{Zadanie z Wikieł Z 1.62 a) moja wersja nr 1017}

Rozwiązać nierówności $(x-10)(x-17)(x-20)\ge0$.
\zadStop
\rozwStart{Patryk Wirkus}{}
Miejsca zerowe naszego wielomianu to: $10, 17, 20$.\\
Wielomian jest stopnia nieparzystego, ponadto znak współczynnika przy\linebreak najwyższej potędze x jest dodatni.\\ W związku z tym wykres wielomianu zaczyna się od lewej strony poniżej osi OX. A więc $$x \in [10,17] \cup [20,\infty).$$
\rozwStop
\odpStart
$x \in [10,17] \cup [20,\infty)$
\odpStop
\testStart
A.$x \in [10,17] \cup [20,\infty)$\\
B.$x \in (10,17) \cup [20,\infty)$\\
C.$x \in (10,17] \cup [20,\infty)$\\
D.$x \in [10,17) \cup [20,\infty)$\\
E.$x \in [10,17] \cup (20,\infty)$\\
F.$x \in (10,17) \cup (20,\infty)$\\
G.$x \in [10,17) \cup (20,\infty)$\\
H.$x \in (10,17] \cup (20,\infty)$
\testStop
\kluczStart
A
\kluczStop



\zadStart{Zadanie z Wikieł Z 1.62 a) moja wersja nr 1018}

Rozwiązać nierówności $(x-10)(x-18)(x-19)\ge0$.
\zadStop
\rozwStart{Patryk Wirkus}{}
Miejsca zerowe naszego wielomianu to: $10, 18, 19$.\\
Wielomian jest stopnia nieparzystego, ponadto znak współczynnika przy\linebreak najwyższej potędze x jest dodatni.\\ W związku z tym wykres wielomianu zaczyna się od lewej strony poniżej osi OX. A więc $$x \in [10,18] \cup [19,\infty).$$
\rozwStop
\odpStart
$x \in [10,18] \cup [19,\infty)$
\odpStop
\testStart
A.$x \in [10,18] \cup [19,\infty)$\\
B.$x \in (10,18) \cup [19,\infty)$\\
C.$x \in (10,18] \cup [19,\infty)$\\
D.$x \in [10,18) \cup [19,\infty)$\\
E.$x \in [10,18] \cup (19,\infty)$\\
F.$x \in (10,18) \cup (19,\infty)$\\
G.$x \in [10,18) \cup (19,\infty)$\\
H.$x \in (10,18] \cup (19,\infty)$
\testStop
\kluczStart
A
\kluczStop



\zadStart{Zadanie z Wikieł Z 1.62 a) moja wersja nr 1019}

Rozwiązać nierówności $(x-10)(x-18)(x-20)\ge0$.
\zadStop
\rozwStart{Patryk Wirkus}{}
Miejsca zerowe naszego wielomianu to: $10, 18, 20$.\\
Wielomian jest stopnia nieparzystego, ponadto znak współczynnika przy\linebreak najwyższej potędze x jest dodatni.\\ W związku z tym wykres wielomianu zaczyna się od lewej strony poniżej osi OX. A więc $$x \in [10,18] \cup [20,\infty).$$
\rozwStop
\odpStart
$x \in [10,18] \cup [20,\infty)$
\odpStop
\testStart
A.$x \in [10,18] \cup [20,\infty)$\\
B.$x \in (10,18) \cup [20,\infty)$\\
C.$x \in (10,18] \cup [20,\infty)$\\
D.$x \in [10,18) \cup [20,\infty)$\\
E.$x \in [10,18] \cup (20,\infty)$\\
F.$x \in (10,18) \cup (20,\infty)$\\
G.$x \in [10,18) \cup (20,\infty)$\\
H.$x \in (10,18] \cup (20,\infty)$
\testStop
\kluczStart
A
\kluczStop



\zadStart{Zadanie z Wikieł Z 1.62 a) moja wersja nr 1020}

Rozwiązać nierówności $(x-10)(x-19)(x-20)\ge0$.
\zadStop
\rozwStart{Patryk Wirkus}{}
Miejsca zerowe naszego wielomianu to: $10, 19, 20$.\\
Wielomian jest stopnia nieparzystego, ponadto znak współczynnika przy\linebreak najwyższej potędze x jest dodatni.\\ W związku z tym wykres wielomianu zaczyna się od lewej strony poniżej osi OX. A więc $$x \in [10,19] \cup [20,\infty).$$
\rozwStop
\odpStart
$x \in [10,19] \cup [20,\infty)$
\odpStop
\testStart
A.$x \in [10,19] \cup [20,\infty)$\\
B.$x \in (10,19) \cup [20,\infty)$\\
C.$x \in (10,19] \cup [20,\infty)$\\
D.$x \in [10,19) \cup [20,\infty)$\\
E.$x \in [10,19] \cup (20,\infty)$\\
F.$x \in (10,19) \cup (20,\infty)$\\
G.$x \in [10,19) \cup (20,\infty)$\\
H.$x \in (10,19] \cup (20,\infty)$
\testStop
\kluczStart
A
\kluczStop



\zadStart{Zadanie z Wikieł Z 1.62 a) moja wersja nr 1021}

Rozwiązać nierówności $(x-11)(x-12)(x-13)\ge0$.
\zadStop
\rozwStart{Patryk Wirkus}{}
Miejsca zerowe naszego wielomianu to: $11, 12, 13$.\\
Wielomian jest stopnia nieparzystego, ponadto znak współczynnika przy\linebreak najwyższej potędze x jest dodatni.\\ W związku z tym wykres wielomianu zaczyna się od lewej strony poniżej osi OX. A więc $$x \in [11,12] \cup [13,\infty).$$
\rozwStop
\odpStart
$x \in [11,12] \cup [13,\infty)$
\odpStop
\testStart
A.$x \in [11,12] \cup [13,\infty)$\\
B.$x \in (11,12) \cup [13,\infty)$\\
C.$x \in (11,12] \cup [13,\infty)$\\
D.$x \in [11,12) \cup [13,\infty)$\\
E.$x \in [11,12] \cup (13,\infty)$\\
F.$x \in (11,12) \cup (13,\infty)$\\
G.$x \in [11,12) \cup (13,\infty)$\\
H.$x \in (11,12] \cup (13,\infty)$
\testStop
\kluczStart
A
\kluczStop



\zadStart{Zadanie z Wikieł Z 1.62 a) moja wersja nr 1022}

Rozwiązać nierówności $(x-11)(x-12)(x-14)\ge0$.
\zadStop
\rozwStart{Patryk Wirkus}{}
Miejsca zerowe naszego wielomianu to: $11, 12, 14$.\\
Wielomian jest stopnia nieparzystego, ponadto znak współczynnika przy\linebreak najwyższej potędze x jest dodatni.\\ W związku z tym wykres wielomianu zaczyna się od lewej strony poniżej osi OX. A więc $$x \in [11,12] \cup [14,\infty).$$
\rozwStop
\odpStart
$x \in [11,12] \cup [14,\infty)$
\odpStop
\testStart
A.$x \in [11,12] \cup [14,\infty)$\\
B.$x \in (11,12) \cup [14,\infty)$\\
C.$x \in (11,12] \cup [14,\infty)$\\
D.$x \in [11,12) \cup [14,\infty)$\\
E.$x \in [11,12] \cup (14,\infty)$\\
F.$x \in (11,12) \cup (14,\infty)$\\
G.$x \in [11,12) \cup (14,\infty)$\\
H.$x \in (11,12] \cup (14,\infty)$
\testStop
\kluczStart
A
\kluczStop



\zadStart{Zadanie z Wikieł Z 1.62 a) moja wersja nr 1023}

Rozwiązać nierówności $(x-11)(x-12)(x-15)\ge0$.
\zadStop
\rozwStart{Patryk Wirkus}{}
Miejsca zerowe naszego wielomianu to: $11, 12, 15$.\\
Wielomian jest stopnia nieparzystego, ponadto znak współczynnika przy\linebreak najwyższej potędze x jest dodatni.\\ W związku z tym wykres wielomianu zaczyna się od lewej strony poniżej osi OX. A więc $$x \in [11,12] \cup [15,\infty).$$
\rozwStop
\odpStart
$x \in [11,12] \cup [15,\infty)$
\odpStop
\testStart
A.$x \in [11,12] \cup [15,\infty)$\\
B.$x \in (11,12) \cup [15,\infty)$\\
C.$x \in (11,12] \cup [15,\infty)$\\
D.$x \in [11,12) \cup [15,\infty)$\\
E.$x \in [11,12] \cup (15,\infty)$\\
F.$x \in (11,12) \cup (15,\infty)$\\
G.$x \in [11,12) \cup (15,\infty)$\\
H.$x \in (11,12] \cup (15,\infty)$
\testStop
\kluczStart
A
\kluczStop



\zadStart{Zadanie z Wikieł Z 1.62 a) moja wersja nr 1024}

Rozwiązać nierówności $(x-11)(x-12)(x-16)\ge0$.
\zadStop
\rozwStart{Patryk Wirkus}{}
Miejsca zerowe naszego wielomianu to: $11, 12, 16$.\\
Wielomian jest stopnia nieparzystego, ponadto znak współczynnika przy\linebreak najwyższej potędze x jest dodatni.\\ W związku z tym wykres wielomianu zaczyna się od lewej strony poniżej osi OX. A więc $$x \in [11,12] \cup [16,\infty).$$
\rozwStop
\odpStart
$x \in [11,12] \cup [16,\infty)$
\odpStop
\testStart
A.$x \in [11,12] \cup [16,\infty)$\\
B.$x \in (11,12) \cup [16,\infty)$\\
C.$x \in (11,12] \cup [16,\infty)$\\
D.$x \in [11,12) \cup [16,\infty)$\\
E.$x \in [11,12] \cup (16,\infty)$\\
F.$x \in (11,12) \cup (16,\infty)$\\
G.$x \in [11,12) \cup (16,\infty)$\\
H.$x \in (11,12] \cup (16,\infty)$
\testStop
\kluczStart
A
\kluczStop



\zadStart{Zadanie z Wikieł Z 1.62 a) moja wersja nr 1025}

Rozwiązać nierówności $(x-11)(x-12)(x-17)\ge0$.
\zadStop
\rozwStart{Patryk Wirkus}{}
Miejsca zerowe naszego wielomianu to: $11, 12, 17$.\\
Wielomian jest stopnia nieparzystego, ponadto znak współczynnika przy\linebreak najwyższej potędze x jest dodatni.\\ W związku z tym wykres wielomianu zaczyna się od lewej strony poniżej osi OX. A więc $$x \in [11,12] \cup [17,\infty).$$
\rozwStop
\odpStart
$x \in [11,12] \cup [17,\infty)$
\odpStop
\testStart
A.$x \in [11,12] \cup [17,\infty)$\\
B.$x \in (11,12) \cup [17,\infty)$\\
C.$x \in (11,12] \cup [17,\infty)$\\
D.$x \in [11,12) \cup [17,\infty)$\\
E.$x \in [11,12] \cup (17,\infty)$\\
F.$x \in (11,12) \cup (17,\infty)$\\
G.$x \in [11,12) \cup (17,\infty)$\\
H.$x \in (11,12] \cup (17,\infty)$
\testStop
\kluczStart
A
\kluczStop



\zadStart{Zadanie z Wikieł Z 1.62 a) moja wersja nr 1026}

Rozwiązać nierówności $(x-11)(x-12)(x-18)\ge0$.
\zadStop
\rozwStart{Patryk Wirkus}{}
Miejsca zerowe naszego wielomianu to: $11, 12, 18$.\\
Wielomian jest stopnia nieparzystego, ponadto znak współczynnika przy\linebreak najwyższej potędze x jest dodatni.\\ W związku z tym wykres wielomianu zaczyna się od lewej strony poniżej osi OX. A więc $$x \in [11,12] \cup [18,\infty).$$
\rozwStop
\odpStart
$x \in [11,12] \cup [18,\infty)$
\odpStop
\testStart
A.$x \in [11,12] \cup [18,\infty)$\\
B.$x \in (11,12) \cup [18,\infty)$\\
C.$x \in (11,12] \cup [18,\infty)$\\
D.$x \in [11,12) \cup [18,\infty)$\\
E.$x \in [11,12] \cup (18,\infty)$\\
F.$x \in (11,12) \cup (18,\infty)$\\
G.$x \in [11,12) \cup (18,\infty)$\\
H.$x \in (11,12] \cup (18,\infty)$
\testStop
\kluczStart
A
\kluczStop



\zadStart{Zadanie z Wikieł Z 1.62 a) moja wersja nr 1027}

Rozwiązać nierówności $(x-11)(x-12)(x-19)\ge0$.
\zadStop
\rozwStart{Patryk Wirkus}{}
Miejsca zerowe naszego wielomianu to: $11, 12, 19$.\\
Wielomian jest stopnia nieparzystego, ponadto znak współczynnika przy\linebreak najwyższej potędze x jest dodatni.\\ W związku z tym wykres wielomianu zaczyna się od lewej strony poniżej osi OX. A więc $$x \in [11,12] \cup [19,\infty).$$
\rozwStop
\odpStart
$x \in [11,12] \cup [19,\infty)$
\odpStop
\testStart
A.$x \in [11,12] \cup [19,\infty)$\\
B.$x \in (11,12) \cup [19,\infty)$\\
C.$x \in (11,12] \cup [19,\infty)$\\
D.$x \in [11,12) \cup [19,\infty)$\\
E.$x \in [11,12] \cup (19,\infty)$\\
F.$x \in (11,12) \cup (19,\infty)$\\
G.$x \in [11,12) \cup (19,\infty)$\\
H.$x \in (11,12] \cup (19,\infty)$
\testStop
\kluczStart
A
\kluczStop



\zadStart{Zadanie z Wikieł Z 1.62 a) moja wersja nr 1028}

Rozwiązać nierówności $(x-11)(x-12)(x-20)\ge0$.
\zadStop
\rozwStart{Patryk Wirkus}{}
Miejsca zerowe naszego wielomianu to: $11, 12, 20$.\\
Wielomian jest stopnia nieparzystego, ponadto znak współczynnika przy\linebreak najwyższej potędze x jest dodatni.\\ W związku z tym wykres wielomianu zaczyna się od lewej strony poniżej osi OX. A więc $$x \in [11,12] \cup [20,\infty).$$
\rozwStop
\odpStart
$x \in [11,12] \cup [20,\infty)$
\odpStop
\testStart
A.$x \in [11,12] \cup [20,\infty)$\\
B.$x \in (11,12) \cup [20,\infty)$\\
C.$x \in (11,12] \cup [20,\infty)$\\
D.$x \in [11,12) \cup [20,\infty)$\\
E.$x \in [11,12] \cup (20,\infty)$\\
F.$x \in (11,12) \cup (20,\infty)$\\
G.$x \in [11,12) \cup (20,\infty)$\\
H.$x \in (11,12] \cup (20,\infty)$
\testStop
\kluczStart
A
\kluczStop



\zadStart{Zadanie z Wikieł Z 1.62 a) moja wersja nr 1029}

Rozwiązać nierówności $(x-11)(x-13)(x-14)\ge0$.
\zadStop
\rozwStart{Patryk Wirkus}{}
Miejsca zerowe naszego wielomianu to: $11, 13, 14$.\\
Wielomian jest stopnia nieparzystego, ponadto znak współczynnika przy\linebreak najwyższej potędze x jest dodatni.\\ W związku z tym wykres wielomianu zaczyna się od lewej strony poniżej osi OX. A więc $$x \in [11,13] \cup [14,\infty).$$
\rozwStop
\odpStart
$x \in [11,13] \cup [14,\infty)$
\odpStop
\testStart
A.$x \in [11,13] \cup [14,\infty)$\\
B.$x \in (11,13) \cup [14,\infty)$\\
C.$x \in (11,13] \cup [14,\infty)$\\
D.$x \in [11,13) \cup [14,\infty)$\\
E.$x \in [11,13] \cup (14,\infty)$\\
F.$x \in (11,13) \cup (14,\infty)$\\
G.$x \in [11,13) \cup (14,\infty)$\\
H.$x \in (11,13] \cup (14,\infty)$
\testStop
\kluczStart
A
\kluczStop



\zadStart{Zadanie z Wikieł Z 1.62 a) moja wersja nr 1030}

Rozwiązać nierówności $(x-11)(x-13)(x-15)\ge0$.
\zadStop
\rozwStart{Patryk Wirkus}{}
Miejsca zerowe naszego wielomianu to: $11, 13, 15$.\\
Wielomian jest stopnia nieparzystego, ponadto znak współczynnika przy\linebreak najwyższej potędze x jest dodatni.\\ W związku z tym wykres wielomianu zaczyna się od lewej strony poniżej osi OX. A więc $$x \in [11,13] \cup [15,\infty).$$
\rozwStop
\odpStart
$x \in [11,13] \cup [15,\infty)$
\odpStop
\testStart
A.$x \in [11,13] \cup [15,\infty)$\\
B.$x \in (11,13) \cup [15,\infty)$\\
C.$x \in (11,13] \cup [15,\infty)$\\
D.$x \in [11,13) \cup [15,\infty)$\\
E.$x \in [11,13] \cup (15,\infty)$\\
F.$x \in (11,13) \cup (15,\infty)$\\
G.$x \in [11,13) \cup (15,\infty)$\\
H.$x \in (11,13] \cup (15,\infty)$
\testStop
\kluczStart
A
\kluczStop



\zadStart{Zadanie z Wikieł Z 1.62 a) moja wersja nr 1031}

Rozwiązać nierówności $(x-11)(x-13)(x-16)\ge0$.
\zadStop
\rozwStart{Patryk Wirkus}{}
Miejsca zerowe naszego wielomianu to: $11, 13, 16$.\\
Wielomian jest stopnia nieparzystego, ponadto znak współczynnika przy\linebreak najwyższej potędze x jest dodatni.\\ W związku z tym wykres wielomianu zaczyna się od lewej strony poniżej osi OX. A więc $$x \in [11,13] \cup [16,\infty).$$
\rozwStop
\odpStart
$x \in [11,13] \cup [16,\infty)$
\odpStop
\testStart
A.$x \in [11,13] \cup [16,\infty)$\\
B.$x \in (11,13) \cup [16,\infty)$\\
C.$x \in (11,13] \cup [16,\infty)$\\
D.$x \in [11,13) \cup [16,\infty)$\\
E.$x \in [11,13] \cup (16,\infty)$\\
F.$x \in (11,13) \cup (16,\infty)$\\
G.$x \in [11,13) \cup (16,\infty)$\\
H.$x \in (11,13] \cup (16,\infty)$
\testStop
\kluczStart
A
\kluczStop



\zadStart{Zadanie z Wikieł Z 1.62 a) moja wersja nr 1032}

Rozwiązać nierówności $(x-11)(x-13)(x-17)\ge0$.
\zadStop
\rozwStart{Patryk Wirkus}{}
Miejsca zerowe naszego wielomianu to: $11, 13, 17$.\\
Wielomian jest stopnia nieparzystego, ponadto znak współczynnika przy\linebreak najwyższej potędze x jest dodatni.\\ W związku z tym wykres wielomianu zaczyna się od lewej strony poniżej osi OX. A więc $$x \in [11,13] \cup [17,\infty).$$
\rozwStop
\odpStart
$x \in [11,13] \cup [17,\infty)$
\odpStop
\testStart
A.$x \in [11,13] \cup [17,\infty)$\\
B.$x \in (11,13) \cup [17,\infty)$\\
C.$x \in (11,13] \cup [17,\infty)$\\
D.$x \in [11,13) \cup [17,\infty)$\\
E.$x \in [11,13] \cup (17,\infty)$\\
F.$x \in (11,13) \cup (17,\infty)$\\
G.$x \in [11,13) \cup (17,\infty)$\\
H.$x \in (11,13] \cup (17,\infty)$
\testStop
\kluczStart
A
\kluczStop



\zadStart{Zadanie z Wikieł Z 1.62 a) moja wersja nr 1033}

Rozwiązać nierówności $(x-11)(x-13)(x-18)\ge0$.
\zadStop
\rozwStart{Patryk Wirkus}{}
Miejsca zerowe naszego wielomianu to: $11, 13, 18$.\\
Wielomian jest stopnia nieparzystego, ponadto znak współczynnika przy\linebreak najwyższej potędze x jest dodatni.\\ W związku z tym wykres wielomianu zaczyna się od lewej strony poniżej osi OX. A więc $$x \in [11,13] \cup [18,\infty).$$
\rozwStop
\odpStart
$x \in [11,13] \cup [18,\infty)$
\odpStop
\testStart
A.$x \in [11,13] \cup [18,\infty)$\\
B.$x \in (11,13) \cup [18,\infty)$\\
C.$x \in (11,13] \cup [18,\infty)$\\
D.$x \in [11,13) \cup [18,\infty)$\\
E.$x \in [11,13] \cup (18,\infty)$\\
F.$x \in (11,13) \cup (18,\infty)$\\
G.$x \in [11,13) \cup (18,\infty)$\\
H.$x \in (11,13] \cup (18,\infty)$
\testStop
\kluczStart
A
\kluczStop



\zadStart{Zadanie z Wikieł Z 1.62 a) moja wersja nr 1034}

Rozwiązać nierówności $(x-11)(x-13)(x-19)\ge0$.
\zadStop
\rozwStart{Patryk Wirkus}{}
Miejsca zerowe naszego wielomianu to: $11, 13, 19$.\\
Wielomian jest stopnia nieparzystego, ponadto znak współczynnika przy\linebreak najwyższej potędze x jest dodatni.\\ W związku z tym wykres wielomianu zaczyna się od lewej strony poniżej osi OX. A więc $$x \in [11,13] \cup [19,\infty).$$
\rozwStop
\odpStart
$x \in [11,13] \cup [19,\infty)$
\odpStop
\testStart
A.$x \in [11,13] \cup [19,\infty)$\\
B.$x \in (11,13) \cup [19,\infty)$\\
C.$x \in (11,13] \cup [19,\infty)$\\
D.$x \in [11,13) \cup [19,\infty)$\\
E.$x \in [11,13] \cup (19,\infty)$\\
F.$x \in (11,13) \cup (19,\infty)$\\
G.$x \in [11,13) \cup (19,\infty)$\\
H.$x \in (11,13] \cup (19,\infty)$
\testStop
\kluczStart
A
\kluczStop



\zadStart{Zadanie z Wikieł Z 1.62 a) moja wersja nr 1035}

Rozwiązać nierówności $(x-11)(x-13)(x-20)\ge0$.
\zadStop
\rozwStart{Patryk Wirkus}{}
Miejsca zerowe naszego wielomianu to: $11, 13, 20$.\\
Wielomian jest stopnia nieparzystego, ponadto znak współczynnika przy\linebreak najwyższej potędze x jest dodatni.\\ W związku z tym wykres wielomianu zaczyna się od lewej strony poniżej osi OX. A więc $$x \in [11,13] \cup [20,\infty).$$
\rozwStop
\odpStart
$x \in [11,13] \cup [20,\infty)$
\odpStop
\testStart
A.$x \in [11,13] \cup [20,\infty)$\\
B.$x \in (11,13) \cup [20,\infty)$\\
C.$x \in (11,13] \cup [20,\infty)$\\
D.$x \in [11,13) \cup [20,\infty)$\\
E.$x \in [11,13] \cup (20,\infty)$\\
F.$x \in (11,13) \cup (20,\infty)$\\
G.$x \in [11,13) \cup (20,\infty)$\\
H.$x \in (11,13] \cup (20,\infty)$
\testStop
\kluczStart
A
\kluczStop



\zadStart{Zadanie z Wikieł Z 1.62 a) moja wersja nr 1036}

Rozwiązać nierówności $(x-11)(x-14)(x-15)\ge0$.
\zadStop
\rozwStart{Patryk Wirkus}{}
Miejsca zerowe naszego wielomianu to: $11, 14, 15$.\\
Wielomian jest stopnia nieparzystego, ponadto znak współczynnika przy\linebreak najwyższej potędze x jest dodatni.\\ W związku z tym wykres wielomianu zaczyna się od lewej strony poniżej osi OX. A więc $$x \in [11,14] \cup [15,\infty).$$
\rozwStop
\odpStart
$x \in [11,14] \cup [15,\infty)$
\odpStop
\testStart
A.$x \in [11,14] \cup [15,\infty)$\\
B.$x \in (11,14) \cup [15,\infty)$\\
C.$x \in (11,14] \cup [15,\infty)$\\
D.$x \in [11,14) \cup [15,\infty)$\\
E.$x \in [11,14] \cup (15,\infty)$\\
F.$x \in (11,14) \cup (15,\infty)$\\
G.$x \in [11,14) \cup (15,\infty)$\\
H.$x \in (11,14] \cup (15,\infty)$
\testStop
\kluczStart
A
\kluczStop



\zadStart{Zadanie z Wikieł Z 1.62 a) moja wersja nr 1037}

Rozwiązać nierówności $(x-11)(x-14)(x-16)\ge0$.
\zadStop
\rozwStart{Patryk Wirkus}{}
Miejsca zerowe naszego wielomianu to: $11, 14, 16$.\\
Wielomian jest stopnia nieparzystego, ponadto znak współczynnika przy\linebreak najwyższej potędze x jest dodatni.\\ W związku z tym wykres wielomianu zaczyna się od lewej strony poniżej osi OX. A więc $$x \in [11,14] \cup [16,\infty).$$
\rozwStop
\odpStart
$x \in [11,14] \cup [16,\infty)$
\odpStop
\testStart
A.$x \in [11,14] \cup [16,\infty)$\\
B.$x \in (11,14) \cup [16,\infty)$\\
C.$x \in (11,14] \cup [16,\infty)$\\
D.$x \in [11,14) \cup [16,\infty)$\\
E.$x \in [11,14] \cup (16,\infty)$\\
F.$x \in (11,14) \cup (16,\infty)$\\
G.$x \in [11,14) \cup (16,\infty)$\\
H.$x \in (11,14] \cup (16,\infty)$
\testStop
\kluczStart
A
\kluczStop



\zadStart{Zadanie z Wikieł Z 1.62 a) moja wersja nr 1038}

Rozwiązać nierówności $(x-11)(x-14)(x-17)\ge0$.
\zadStop
\rozwStart{Patryk Wirkus}{}
Miejsca zerowe naszego wielomianu to: $11, 14, 17$.\\
Wielomian jest stopnia nieparzystego, ponadto znak współczynnika przy\linebreak najwyższej potędze x jest dodatni.\\ W związku z tym wykres wielomianu zaczyna się od lewej strony poniżej osi OX. A więc $$x \in [11,14] \cup [17,\infty).$$
\rozwStop
\odpStart
$x \in [11,14] \cup [17,\infty)$
\odpStop
\testStart
A.$x \in [11,14] \cup [17,\infty)$\\
B.$x \in (11,14) \cup [17,\infty)$\\
C.$x \in (11,14] \cup [17,\infty)$\\
D.$x \in [11,14) \cup [17,\infty)$\\
E.$x \in [11,14] \cup (17,\infty)$\\
F.$x \in (11,14) \cup (17,\infty)$\\
G.$x \in [11,14) \cup (17,\infty)$\\
H.$x \in (11,14] \cup (17,\infty)$
\testStop
\kluczStart
A
\kluczStop



\zadStart{Zadanie z Wikieł Z 1.62 a) moja wersja nr 1039}

Rozwiązać nierówności $(x-11)(x-14)(x-18)\ge0$.
\zadStop
\rozwStart{Patryk Wirkus}{}
Miejsca zerowe naszego wielomianu to: $11, 14, 18$.\\
Wielomian jest stopnia nieparzystego, ponadto znak współczynnika przy\linebreak najwyższej potędze x jest dodatni.\\ W związku z tym wykres wielomianu zaczyna się od lewej strony poniżej osi OX. A więc $$x \in [11,14] \cup [18,\infty).$$
\rozwStop
\odpStart
$x \in [11,14] \cup [18,\infty)$
\odpStop
\testStart
A.$x \in [11,14] \cup [18,\infty)$\\
B.$x \in (11,14) \cup [18,\infty)$\\
C.$x \in (11,14] \cup [18,\infty)$\\
D.$x \in [11,14) \cup [18,\infty)$\\
E.$x \in [11,14] \cup (18,\infty)$\\
F.$x \in (11,14) \cup (18,\infty)$\\
G.$x \in [11,14) \cup (18,\infty)$\\
H.$x \in (11,14] \cup (18,\infty)$
\testStop
\kluczStart
A
\kluczStop



\zadStart{Zadanie z Wikieł Z 1.62 a) moja wersja nr 1040}

Rozwiązać nierówności $(x-11)(x-14)(x-19)\ge0$.
\zadStop
\rozwStart{Patryk Wirkus}{}
Miejsca zerowe naszego wielomianu to: $11, 14, 19$.\\
Wielomian jest stopnia nieparzystego, ponadto znak współczynnika przy\linebreak najwyższej potędze x jest dodatni.\\ W związku z tym wykres wielomianu zaczyna się od lewej strony poniżej osi OX. A więc $$x \in [11,14] \cup [19,\infty).$$
\rozwStop
\odpStart
$x \in [11,14] \cup [19,\infty)$
\odpStop
\testStart
A.$x \in [11,14] \cup [19,\infty)$\\
B.$x \in (11,14) \cup [19,\infty)$\\
C.$x \in (11,14] \cup [19,\infty)$\\
D.$x \in [11,14) \cup [19,\infty)$\\
E.$x \in [11,14] \cup (19,\infty)$\\
F.$x \in (11,14) \cup (19,\infty)$\\
G.$x \in [11,14) \cup (19,\infty)$\\
H.$x \in (11,14] \cup (19,\infty)$
\testStop
\kluczStart
A
\kluczStop



\zadStart{Zadanie z Wikieł Z 1.62 a) moja wersja nr 1041}

Rozwiązać nierówności $(x-11)(x-14)(x-20)\ge0$.
\zadStop
\rozwStart{Patryk Wirkus}{}
Miejsca zerowe naszego wielomianu to: $11, 14, 20$.\\
Wielomian jest stopnia nieparzystego, ponadto znak współczynnika przy\linebreak najwyższej potędze x jest dodatni.\\ W związku z tym wykres wielomianu zaczyna się od lewej strony poniżej osi OX. A więc $$x \in [11,14] \cup [20,\infty).$$
\rozwStop
\odpStart
$x \in [11,14] \cup [20,\infty)$
\odpStop
\testStart
A.$x \in [11,14] \cup [20,\infty)$\\
B.$x \in (11,14) \cup [20,\infty)$\\
C.$x \in (11,14] \cup [20,\infty)$\\
D.$x \in [11,14) \cup [20,\infty)$\\
E.$x \in [11,14] \cup (20,\infty)$\\
F.$x \in (11,14) \cup (20,\infty)$\\
G.$x \in [11,14) \cup (20,\infty)$\\
H.$x \in (11,14] \cup (20,\infty)$
\testStop
\kluczStart
A
\kluczStop



\zadStart{Zadanie z Wikieł Z 1.62 a) moja wersja nr 1042}

Rozwiązać nierówności $(x-11)(x-15)(x-16)\ge0$.
\zadStop
\rozwStart{Patryk Wirkus}{}
Miejsca zerowe naszego wielomianu to: $11, 15, 16$.\\
Wielomian jest stopnia nieparzystego, ponadto znak współczynnika przy\linebreak najwyższej potędze x jest dodatni.\\ W związku z tym wykres wielomianu zaczyna się od lewej strony poniżej osi OX. A więc $$x \in [11,15] \cup [16,\infty).$$
\rozwStop
\odpStart
$x \in [11,15] \cup [16,\infty)$
\odpStop
\testStart
A.$x \in [11,15] \cup [16,\infty)$\\
B.$x \in (11,15) \cup [16,\infty)$\\
C.$x \in (11,15] \cup [16,\infty)$\\
D.$x \in [11,15) \cup [16,\infty)$\\
E.$x \in [11,15] \cup (16,\infty)$\\
F.$x \in (11,15) \cup (16,\infty)$\\
G.$x \in [11,15) \cup (16,\infty)$\\
H.$x \in (11,15] \cup (16,\infty)$
\testStop
\kluczStart
A
\kluczStop



\zadStart{Zadanie z Wikieł Z 1.62 a) moja wersja nr 1043}

Rozwiązać nierówności $(x-11)(x-15)(x-17)\ge0$.
\zadStop
\rozwStart{Patryk Wirkus}{}
Miejsca zerowe naszego wielomianu to: $11, 15, 17$.\\
Wielomian jest stopnia nieparzystego, ponadto znak współczynnika przy\linebreak najwyższej potędze x jest dodatni.\\ W związku z tym wykres wielomianu zaczyna się od lewej strony poniżej osi OX. A więc $$x \in [11,15] \cup [17,\infty).$$
\rozwStop
\odpStart
$x \in [11,15] \cup [17,\infty)$
\odpStop
\testStart
A.$x \in [11,15] \cup [17,\infty)$\\
B.$x \in (11,15) \cup [17,\infty)$\\
C.$x \in (11,15] \cup [17,\infty)$\\
D.$x \in [11,15) \cup [17,\infty)$\\
E.$x \in [11,15] \cup (17,\infty)$\\
F.$x \in (11,15) \cup (17,\infty)$\\
G.$x \in [11,15) \cup (17,\infty)$\\
H.$x \in (11,15] \cup (17,\infty)$
\testStop
\kluczStart
A
\kluczStop



\zadStart{Zadanie z Wikieł Z 1.62 a) moja wersja nr 1044}

Rozwiązać nierówności $(x-11)(x-15)(x-18)\ge0$.
\zadStop
\rozwStart{Patryk Wirkus}{}
Miejsca zerowe naszego wielomianu to: $11, 15, 18$.\\
Wielomian jest stopnia nieparzystego, ponadto znak współczynnika przy\linebreak najwyższej potędze x jest dodatni.\\ W związku z tym wykres wielomianu zaczyna się od lewej strony poniżej osi OX. A więc $$x \in [11,15] \cup [18,\infty).$$
\rozwStop
\odpStart
$x \in [11,15] \cup [18,\infty)$
\odpStop
\testStart
A.$x \in [11,15] \cup [18,\infty)$\\
B.$x \in (11,15) \cup [18,\infty)$\\
C.$x \in (11,15] \cup [18,\infty)$\\
D.$x \in [11,15) \cup [18,\infty)$\\
E.$x \in [11,15] \cup (18,\infty)$\\
F.$x \in (11,15) \cup (18,\infty)$\\
G.$x \in [11,15) \cup (18,\infty)$\\
H.$x \in (11,15] \cup (18,\infty)$
\testStop
\kluczStart
A
\kluczStop



\zadStart{Zadanie z Wikieł Z 1.62 a) moja wersja nr 1045}

Rozwiązać nierówności $(x-11)(x-15)(x-19)\ge0$.
\zadStop
\rozwStart{Patryk Wirkus}{}
Miejsca zerowe naszego wielomianu to: $11, 15, 19$.\\
Wielomian jest stopnia nieparzystego, ponadto znak współczynnika przy\linebreak najwyższej potędze x jest dodatni.\\ W związku z tym wykres wielomianu zaczyna się od lewej strony poniżej osi OX. A więc $$x \in [11,15] \cup [19,\infty).$$
\rozwStop
\odpStart
$x \in [11,15] \cup [19,\infty)$
\odpStop
\testStart
A.$x \in [11,15] \cup [19,\infty)$\\
B.$x \in (11,15) \cup [19,\infty)$\\
C.$x \in (11,15] \cup [19,\infty)$\\
D.$x \in [11,15) \cup [19,\infty)$\\
E.$x \in [11,15] \cup (19,\infty)$\\
F.$x \in (11,15) \cup (19,\infty)$\\
G.$x \in [11,15) \cup (19,\infty)$\\
H.$x \in (11,15] \cup (19,\infty)$
\testStop
\kluczStart
A
\kluczStop



\zadStart{Zadanie z Wikieł Z 1.62 a) moja wersja nr 1046}

Rozwiązać nierówności $(x-11)(x-15)(x-20)\ge0$.
\zadStop
\rozwStart{Patryk Wirkus}{}
Miejsca zerowe naszego wielomianu to: $11, 15, 20$.\\
Wielomian jest stopnia nieparzystego, ponadto znak współczynnika przy\linebreak najwyższej potędze x jest dodatni.\\ W związku z tym wykres wielomianu zaczyna się od lewej strony poniżej osi OX. A więc $$x \in [11,15] \cup [20,\infty).$$
\rozwStop
\odpStart
$x \in [11,15] \cup [20,\infty)$
\odpStop
\testStart
A.$x \in [11,15] \cup [20,\infty)$\\
B.$x \in (11,15) \cup [20,\infty)$\\
C.$x \in (11,15] \cup [20,\infty)$\\
D.$x \in [11,15) \cup [20,\infty)$\\
E.$x \in [11,15] \cup (20,\infty)$\\
F.$x \in (11,15) \cup (20,\infty)$\\
G.$x \in [11,15) \cup (20,\infty)$\\
H.$x \in (11,15] \cup (20,\infty)$
\testStop
\kluczStart
A
\kluczStop



\zadStart{Zadanie z Wikieł Z 1.62 a) moja wersja nr 1047}

Rozwiązać nierówności $(x-11)(x-16)(x-17)\ge0$.
\zadStop
\rozwStart{Patryk Wirkus}{}
Miejsca zerowe naszego wielomianu to: $11, 16, 17$.\\
Wielomian jest stopnia nieparzystego, ponadto znak współczynnika przy\linebreak najwyższej potędze x jest dodatni.\\ W związku z tym wykres wielomianu zaczyna się od lewej strony poniżej osi OX. A więc $$x \in [11,16] \cup [17,\infty).$$
\rozwStop
\odpStart
$x \in [11,16] \cup [17,\infty)$
\odpStop
\testStart
A.$x \in [11,16] \cup [17,\infty)$\\
B.$x \in (11,16) \cup [17,\infty)$\\
C.$x \in (11,16] \cup [17,\infty)$\\
D.$x \in [11,16) \cup [17,\infty)$\\
E.$x \in [11,16] \cup (17,\infty)$\\
F.$x \in (11,16) \cup (17,\infty)$\\
G.$x \in [11,16) \cup (17,\infty)$\\
H.$x \in (11,16] \cup (17,\infty)$
\testStop
\kluczStart
A
\kluczStop



\zadStart{Zadanie z Wikieł Z 1.62 a) moja wersja nr 1048}

Rozwiązać nierówności $(x-11)(x-16)(x-18)\ge0$.
\zadStop
\rozwStart{Patryk Wirkus}{}
Miejsca zerowe naszego wielomianu to: $11, 16, 18$.\\
Wielomian jest stopnia nieparzystego, ponadto znak współczynnika przy\linebreak najwyższej potędze x jest dodatni.\\ W związku z tym wykres wielomianu zaczyna się od lewej strony poniżej osi OX. A więc $$x \in [11,16] \cup [18,\infty).$$
\rozwStop
\odpStart
$x \in [11,16] \cup [18,\infty)$
\odpStop
\testStart
A.$x \in [11,16] \cup [18,\infty)$\\
B.$x \in (11,16) \cup [18,\infty)$\\
C.$x \in (11,16] \cup [18,\infty)$\\
D.$x \in [11,16) \cup [18,\infty)$\\
E.$x \in [11,16] \cup (18,\infty)$\\
F.$x \in (11,16) \cup (18,\infty)$\\
G.$x \in [11,16) \cup (18,\infty)$\\
H.$x \in (11,16] \cup (18,\infty)$
\testStop
\kluczStart
A
\kluczStop



\zadStart{Zadanie z Wikieł Z 1.62 a) moja wersja nr 1049}

Rozwiązać nierówności $(x-11)(x-16)(x-19)\ge0$.
\zadStop
\rozwStart{Patryk Wirkus}{}
Miejsca zerowe naszego wielomianu to: $11, 16, 19$.\\
Wielomian jest stopnia nieparzystego, ponadto znak współczynnika przy\linebreak najwyższej potędze x jest dodatni.\\ W związku z tym wykres wielomianu zaczyna się od lewej strony poniżej osi OX. A więc $$x \in [11,16] \cup [19,\infty).$$
\rozwStop
\odpStart
$x \in [11,16] \cup [19,\infty)$
\odpStop
\testStart
A.$x \in [11,16] \cup [19,\infty)$\\
B.$x \in (11,16) \cup [19,\infty)$\\
C.$x \in (11,16] \cup [19,\infty)$\\
D.$x \in [11,16) \cup [19,\infty)$\\
E.$x \in [11,16] \cup (19,\infty)$\\
F.$x \in (11,16) \cup (19,\infty)$\\
G.$x \in [11,16) \cup (19,\infty)$\\
H.$x \in (11,16] \cup (19,\infty)$
\testStop
\kluczStart
A
\kluczStop



\zadStart{Zadanie z Wikieł Z 1.62 a) moja wersja nr 1050}

Rozwiązać nierówności $(x-11)(x-16)(x-20)\ge0$.
\zadStop
\rozwStart{Patryk Wirkus}{}
Miejsca zerowe naszego wielomianu to: $11, 16, 20$.\\
Wielomian jest stopnia nieparzystego, ponadto znak współczynnika przy\linebreak najwyższej potędze x jest dodatni.\\ W związku z tym wykres wielomianu zaczyna się od lewej strony poniżej osi OX. A więc $$x \in [11,16] \cup [20,\infty).$$
\rozwStop
\odpStart
$x \in [11,16] \cup [20,\infty)$
\odpStop
\testStart
A.$x \in [11,16] \cup [20,\infty)$\\
B.$x \in (11,16) \cup [20,\infty)$\\
C.$x \in (11,16] \cup [20,\infty)$\\
D.$x \in [11,16) \cup [20,\infty)$\\
E.$x \in [11,16] \cup (20,\infty)$\\
F.$x \in (11,16) \cup (20,\infty)$\\
G.$x \in [11,16) \cup (20,\infty)$\\
H.$x \in (11,16] \cup (20,\infty)$
\testStop
\kluczStart
A
\kluczStop



\zadStart{Zadanie z Wikieł Z 1.62 a) moja wersja nr 1051}

Rozwiązać nierówności $(x-11)(x-17)(x-18)\ge0$.
\zadStop
\rozwStart{Patryk Wirkus}{}
Miejsca zerowe naszego wielomianu to: $11, 17, 18$.\\
Wielomian jest stopnia nieparzystego, ponadto znak współczynnika przy\linebreak najwyższej potędze x jest dodatni.\\ W związku z tym wykres wielomianu zaczyna się od lewej strony poniżej osi OX. A więc $$x \in [11,17] \cup [18,\infty).$$
\rozwStop
\odpStart
$x \in [11,17] \cup [18,\infty)$
\odpStop
\testStart
A.$x \in [11,17] \cup [18,\infty)$\\
B.$x \in (11,17) \cup [18,\infty)$\\
C.$x \in (11,17] \cup [18,\infty)$\\
D.$x \in [11,17) \cup [18,\infty)$\\
E.$x \in [11,17] \cup (18,\infty)$\\
F.$x \in (11,17) \cup (18,\infty)$\\
G.$x \in [11,17) \cup (18,\infty)$\\
H.$x \in (11,17] \cup (18,\infty)$
\testStop
\kluczStart
A
\kluczStop



\zadStart{Zadanie z Wikieł Z 1.62 a) moja wersja nr 1052}

Rozwiązać nierówności $(x-11)(x-17)(x-19)\ge0$.
\zadStop
\rozwStart{Patryk Wirkus}{}
Miejsca zerowe naszego wielomianu to: $11, 17, 19$.\\
Wielomian jest stopnia nieparzystego, ponadto znak współczynnika przy\linebreak najwyższej potędze x jest dodatni.\\ W związku z tym wykres wielomianu zaczyna się od lewej strony poniżej osi OX. A więc $$x \in [11,17] \cup [19,\infty).$$
\rozwStop
\odpStart
$x \in [11,17] \cup [19,\infty)$
\odpStop
\testStart
A.$x \in [11,17] \cup [19,\infty)$\\
B.$x \in (11,17) \cup [19,\infty)$\\
C.$x \in (11,17] \cup [19,\infty)$\\
D.$x \in [11,17) \cup [19,\infty)$\\
E.$x \in [11,17] \cup (19,\infty)$\\
F.$x \in (11,17) \cup (19,\infty)$\\
G.$x \in [11,17) \cup (19,\infty)$\\
H.$x \in (11,17] \cup (19,\infty)$
\testStop
\kluczStart
A
\kluczStop



\zadStart{Zadanie z Wikieł Z 1.62 a) moja wersja nr 1053}

Rozwiązać nierówności $(x-11)(x-17)(x-20)\ge0$.
\zadStop
\rozwStart{Patryk Wirkus}{}
Miejsca zerowe naszego wielomianu to: $11, 17, 20$.\\
Wielomian jest stopnia nieparzystego, ponadto znak współczynnika przy\linebreak najwyższej potędze x jest dodatni.\\ W związku z tym wykres wielomianu zaczyna się od lewej strony poniżej osi OX. A więc $$x \in [11,17] \cup [20,\infty).$$
\rozwStop
\odpStart
$x \in [11,17] \cup [20,\infty)$
\odpStop
\testStart
A.$x \in [11,17] \cup [20,\infty)$\\
B.$x \in (11,17) \cup [20,\infty)$\\
C.$x \in (11,17] \cup [20,\infty)$\\
D.$x \in [11,17) \cup [20,\infty)$\\
E.$x \in [11,17] \cup (20,\infty)$\\
F.$x \in (11,17) \cup (20,\infty)$\\
G.$x \in [11,17) \cup (20,\infty)$\\
H.$x \in (11,17] \cup (20,\infty)$
\testStop
\kluczStart
A
\kluczStop



\zadStart{Zadanie z Wikieł Z 1.62 a) moja wersja nr 1054}

Rozwiązać nierówności $(x-11)(x-18)(x-19)\ge0$.
\zadStop
\rozwStart{Patryk Wirkus}{}
Miejsca zerowe naszego wielomianu to: $11, 18, 19$.\\
Wielomian jest stopnia nieparzystego, ponadto znak współczynnika przy\linebreak najwyższej potędze x jest dodatni.\\ W związku z tym wykres wielomianu zaczyna się od lewej strony poniżej osi OX. A więc $$x \in [11,18] \cup [19,\infty).$$
\rozwStop
\odpStart
$x \in [11,18] \cup [19,\infty)$
\odpStop
\testStart
A.$x \in [11,18] \cup [19,\infty)$\\
B.$x \in (11,18) \cup [19,\infty)$\\
C.$x \in (11,18] \cup [19,\infty)$\\
D.$x \in [11,18) \cup [19,\infty)$\\
E.$x \in [11,18] \cup (19,\infty)$\\
F.$x \in (11,18) \cup (19,\infty)$\\
G.$x \in [11,18) \cup (19,\infty)$\\
H.$x \in (11,18] \cup (19,\infty)$
\testStop
\kluczStart
A
\kluczStop



\zadStart{Zadanie z Wikieł Z 1.62 a) moja wersja nr 1055}

Rozwiązać nierówności $(x-11)(x-18)(x-20)\ge0$.
\zadStop
\rozwStart{Patryk Wirkus}{}
Miejsca zerowe naszego wielomianu to: $11, 18, 20$.\\
Wielomian jest stopnia nieparzystego, ponadto znak współczynnika przy\linebreak najwyższej potędze x jest dodatni.\\ W związku z tym wykres wielomianu zaczyna się od lewej strony poniżej osi OX. A więc $$x \in [11,18] \cup [20,\infty).$$
\rozwStop
\odpStart
$x \in [11,18] \cup [20,\infty)$
\odpStop
\testStart
A.$x \in [11,18] \cup [20,\infty)$\\
B.$x \in (11,18) \cup [20,\infty)$\\
C.$x \in (11,18] \cup [20,\infty)$\\
D.$x \in [11,18) \cup [20,\infty)$\\
E.$x \in [11,18] \cup (20,\infty)$\\
F.$x \in (11,18) \cup (20,\infty)$\\
G.$x \in [11,18) \cup (20,\infty)$\\
H.$x \in (11,18] \cup (20,\infty)$
\testStop
\kluczStart
A
\kluczStop



\zadStart{Zadanie z Wikieł Z 1.62 a) moja wersja nr 1056}

Rozwiązać nierówności $(x-11)(x-19)(x-20)\ge0$.
\zadStop
\rozwStart{Patryk Wirkus}{}
Miejsca zerowe naszego wielomianu to: $11, 19, 20$.\\
Wielomian jest stopnia nieparzystego, ponadto znak współczynnika przy\linebreak najwyższej potędze x jest dodatni.\\ W związku z tym wykres wielomianu zaczyna się od lewej strony poniżej osi OX. A więc $$x \in [11,19] \cup [20,\infty).$$
\rozwStop
\odpStart
$x \in [11,19] \cup [20,\infty)$
\odpStop
\testStart
A.$x \in [11,19] \cup [20,\infty)$\\
B.$x \in (11,19) \cup [20,\infty)$\\
C.$x \in (11,19] \cup [20,\infty)$\\
D.$x \in [11,19) \cup [20,\infty)$\\
E.$x \in [11,19] \cup (20,\infty)$\\
F.$x \in (11,19) \cup (20,\infty)$\\
G.$x \in [11,19) \cup (20,\infty)$\\
H.$x \in (11,19] \cup (20,\infty)$
\testStop
\kluczStart
A
\kluczStop



\zadStart{Zadanie z Wikieł Z 1.62 a) moja wersja nr 1057}

Rozwiązać nierówności $(x-12)(x-13)(x-14)\ge0$.
\zadStop
\rozwStart{Patryk Wirkus}{}
Miejsca zerowe naszego wielomianu to: $12, 13, 14$.\\
Wielomian jest stopnia nieparzystego, ponadto znak współczynnika przy\linebreak najwyższej potędze x jest dodatni.\\ W związku z tym wykres wielomianu zaczyna się od lewej strony poniżej osi OX. A więc $$x \in [12,13] \cup [14,\infty).$$
\rozwStop
\odpStart
$x \in [12,13] \cup [14,\infty)$
\odpStop
\testStart
A.$x \in [12,13] \cup [14,\infty)$\\
B.$x \in (12,13) \cup [14,\infty)$\\
C.$x \in (12,13] \cup [14,\infty)$\\
D.$x \in [12,13) \cup [14,\infty)$\\
E.$x \in [12,13] \cup (14,\infty)$\\
F.$x \in (12,13) \cup (14,\infty)$\\
G.$x \in [12,13) \cup (14,\infty)$\\
H.$x \in (12,13] \cup (14,\infty)$
\testStop
\kluczStart
A
\kluczStop



\zadStart{Zadanie z Wikieł Z 1.62 a) moja wersja nr 1058}

Rozwiązać nierówności $(x-12)(x-13)(x-15)\ge0$.
\zadStop
\rozwStart{Patryk Wirkus}{}
Miejsca zerowe naszego wielomianu to: $12, 13, 15$.\\
Wielomian jest stopnia nieparzystego, ponadto znak współczynnika przy\linebreak najwyższej potędze x jest dodatni.\\ W związku z tym wykres wielomianu zaczyna się od lewej strony poniżej osi OX. A więc $$x \in [12,13] \cup [15,\infty).$$
\rozwStop
\odpStart
$x \in [12,13] \cup [15,\infty)$
\odpStop
\testStart
A.$x \in [12,13] \cup [15,\infty)$\\
B.$x \in (12,13) \cup [15,\infty)$\\
C.$x \in (12,13] \cup [15,\infty)$\\
D.$x \in [12,13) \cup [15,\infty)$\\
E.$x \in [12,13] \cup (15,\infty)$\\
F.$x \in (12,13) \cup (15,\infty)$\\
G.$x \in [12,13) \cup (15,\infty)$\\
H.$x \in (12,13] \cup (15,\infty)$
\testStop
\kluczStart
A
\kluczStop



\zadStart{Zadanie z Wikieł Z 1.62 a) moja wersja nr 1059}

Rozwiązać nierówności $(x-12)(x-13)(x-16)\ge0$.
\zadStop
\rozwStart{Patryk Wirkus}{}
Miejsca zerowe naszego wielomianu to: $12, 13, 16$.\\
Wielomian jest stopnia nieparzystego, ponadto znak współczynnika przy\linebreak najwyższej potędze x jest dodatni.\\ W związku z tym wykres wielomianu zaczyna się od lewej strony poniżej osi OX. A więc $$x \in [12,13] \cup [16,\infty).$$
\rozwStop
\odpStart
$x \in [12,13] \cup [16,\infty)$
\odpStop
\testStart
A.$x \in [12,13] \cup [16,\infty)$\\
B.$x \in (12,13) \cup [16,\infty)$\\
C.$x \in (12,13] \cup [16,\infty)$\\
D.$x \in [12,13) \cup [16,\infty)$\\
E.$x \in [12,13] \cup (16,\infty)$\\
F.$x \in (12,13) \cup (16,\infty)$\\
G.$x \in [12,13) \cup (16,\infty)$\\
H.$x \in (12,13] \cup (16,\infty)$
\testStop
\kluczStart
A
\kluczStop



\zadStart{Zadanie z Wikieł Z 1.62 a) moja wersja nr 1060}

Rozwiązać nierówności $(x-12)(x-13)(x-17)\ge0$.
\zadStop
\rozwStart{Patryk Wirkus}{}
Miejsca zerowe naszego wielomianu to: $12, 13, 17$.\\
Wielomian jest stopnia nieparzystego, ponadto znak współczynnika przy\linebreak najwyższej potędze x jest dodatni.\\ W związku z tym wykres wielomianu zaczyna się od lewej strony poniżej osi OX. A więc $$x \in [12,13] \cup [17,\infty).$$
\rozwStop
\odpStart
$x \in [12,13] \cup [17,\infty)$
\odpStop
\testStart
A.$x \in [12,13] \cup [17,\infty)$\\
B.$x \in (12,13) \cup [17,\infty)$\\
C.$x \in (12,13] \cup [17,\infty)$\\
D.$x \in [12,13) \cup [17,\infty)$\\
E.$x \in [12,13] \cup (17,\infty)$\\
F.$x \in (12,13) \cup (17,\infty)$\\
G.$x \in [12,13) \cup (17,\infty)$\\
H.$x \in (12,13] \cup (17,\infty)$
\testStop
\kluczStart
A
\kluczStop



\zadStart{Zadanie z Wikieł Z 1.62 a) moja wersja nr 1061}

Rozwiązać nierówności $(x-12)(x-13)(x-18)\ge0$.
\zadStop
\rozwStart{Patryk Wirkus}{}
Miejsca zerowe naszego wielomianu to: $12, 13, 18$.\\
Wielomian jest stopnia nieparzystego, ponadto znak współczynnika przy\linebreak najwyższej potędze x jest dodatni.\\ W związku z tym wykres wielomianu zaczyna się od lewej strony poniżej osi OX. A więc $$x \in [12,13] \cup [18,\infty).$$
\rozwStop
\odpStart
$x \in [12,13] \cup [18,\infty)$
\odpStop
\testStart
A.$x \in [12,13] \cup [18,\infty)$\\
B.$x \in (12,13) \cup [18,\infty)$\\
C.$x \in (12,13] \cup [18,\infty)$\\
D.$x \in [12,13) \cup [18,\infty)$\\
E.$x \in [12,13] \cup (18,\infty)$\\
F.$x \in (12,13) \cup (18,\infty)$\\
G.$x \in [12,13) \cup (18,\infty)$\\
H.$x \in (12,13] \cup (18,\infty)$
\testStop
\kluczStart
A
\kluczStop



\zadStart{Zadanie z Wikieł Z 1.62 a) moja wersja nr 1062}

Rozwiązać nierówności $(x-12)(x-13)(x-19)\ge0$.
\zadStop
\rozwStart{Patryk Wirkus}{}
Miejsca zerowe naszego wielomianu to: $12, 13, 19$.\\
Wielomian jest stopnia nieparzystego, ponadto znak współczynnika przy\linebreak najwyższej potędze x jest dodatni.\\ W związku z tym wykres wielomianu zaczyna się od lewej strony poniżej osi OX. A więc $$x \in [12,13] \cup [19,\infty).$$
\rozwStop
\odpStart
$x \in [12,13] \cup [19,\infty)$
\odpStop
\testStart
A.$x \in [12,13] \cup [19,\infty)$\\
B.$x \in (12,13) \cup [19,\infty)$\\
C.$x \in (12,13] \cup [19,\infty)$\\
D.$x \in [12,13) \cup [19,\infty)$\\
E.$x \in [12,13] \cup (19,\infty)$\\
F.$x \in (12,13) \cup (19,\infty)$\\
G.$x \in [12,13) \cup (19,\infty)$\\
H.$x \in (12,13] \cup (19,\infty)$
\testStop
\kluczStart
A
\kluczStop



\zadStart{Zadanie z Wikieł Z 1.62 a) moja wersja nr 1063}

Rozwiązać nierówności $(x-12)(x-13)(x-20)\ge0$.
\zadStop
\rozwStart{Patryk Wirkus}{}
Miejsca zerowe naszego wielomianu to: $12, 13, 20$.\\
Wielomian jest stopnia nieparzystego, ponadto znak współczynnika przy\linebreak najwyższej potędze x jest dodatni.\\ W związku z tym wykres wielomianu zaczyna się od lewej strony poniżej osi OX. A więc $$x \in [12,13] \cup [20,\infty).$$
\rozwStop
\odpStart
$x \in [12,13] \cup [20,\infty)$
\odpStop
\testStart
A.$x \in [12,13] \cup [20,\infty)$\\
B.$x \in (12,13) \cup [20,\infty)$\\
C.$x \in (12,13] \cup [20,\infty)$\\
D.$x \in [12,13) \cup [20,\infty)$\\
E.$x \in [12,13] \cup (20,\infty)$\\
F.$x \in (12,13) \cup (20,\infty)$\\
G.$x \in [12,13) \cup (20,\infty)$\\
H.$x \in (12,13] \cup (20,\infty)$
\testStop
\kluczStart
A
\kluczStop



\zadStart{Zadanie z Wikieł Z 1.62 a) moja wersja nr 1064}

Rozwiązać nierówności $(x-12)(x-14)(x-15)\ge0$.
\zadStop
\rozwStart{Patryk Wirkus}{}
Miejsca zerowe naszego wielomianu to: $12, 14, 15$.\\
Wielomian jest stopnia nieparzystego, ponadto znak współczynnika przy\linebreak najwyższej potędze x jest dodatni.\\ W związku z tym wykres wielomianu zaczyna się od lewej strony poniżej osi OX. A więc $$x \in [12,14] \cup [15,\infty).$$
\rozwStop
\odpStart
$x \in [12,14] \cup [15,\infty)$
\odpStop
\testStart
A.$x \in [12,14] \cup [15,\infty)$\\
B.$x \in (12,14) \cup [15,\infty)$\\
C.$x \in (12,14] \cup [15,\infty)$\\
D.$x \in [12,14) \cup [15,\infty)$\\
E.$x \in [12,14] \cup (15,\infty)$\\
F.$x \in (12,14) \cup (15,\infty)$\\
G.$x \in [12,14) \cup (15,\infty)$\\
H.$x \in (12,14] \cup (15,\infty)$
\testStop
\kluczStart
A
\kluczStop



\zadStart{Zadanie z Wikieł Z 1.62 a) moja wersja nr 1065}

Rozwiązać nierówności $(x-12)(x-14)(x-16)\ge0$.
\zadStop
\rozwStart{Patryk Wirkus}{}
Miejsca zerowe naszego wielomianu to: $12, 14, 16$.\\
Wielomian jest stopnia nieparzystego, ponadto znak współczynnika przy\linebreak najwyższej potędze x jest dodatni.\\ W związku z tym wykres wielomianu zaczyna się od lewej strony poniżej osi OX. A więc $$x \in [12,14] \cup [16,\infty).$$
\rozwStop
\odpStart
$x \in [12,14] \cup [16,\infty)$
\odpStop
\testStart
A.$x \in [12,14] \cup [16,\infty)$\\
B.$x \in (12,14) \cup [16,\infty)$\\
C.$x \in (12,14] \cup [16,\infty)$\\
D.$x \in [12,14) \cup [16,\infty)$\\
E.$x \in [12,14] \cup (16,\infty)$\\
F.$x \in (12,14) \cup (16,\infty)$\\
G.$x \in [12,14) \cup (16,\infty)$\\
H.$x \in (12,14] \cup (16,\infty)$
\testStop
\kluczStart
A
\kluczStop



\zadStart{Zadanie z Wikieł Z 1.62 a) moja wersja nr 1066}

Rozwiązać nierówności $(x-12)(x-14)(x-17)\ge0$.
\zadStop
\rozwStart{Patryk Wirkus}{}
Miejsca zerowe naszego wielomianu to: $12, 14, 17$.\\
Wielomian jest stopnia nieparzystego, ponadto znak współczynnika przy\linebreak najwyższej potędze x jest dodatni.\\ W związku z tym wykres wielomianu zaczyna się od lewej strony poniżej osi OX. A więc $$x \in [12,14] \cup [17,\infty).$$
\rozwStop
\odpStart
$x \in [12,14] \cup [17,\infty)$
\odpStop
\testStart
A.$x \in [12,14] \cup [17,\infty)$\\
B.$x \in (12,14) \cup [17,\infty)$\\
C.$x \in (12,14] \cup [17,\infty)$\\
D.$x \in [12,14) \cup [17,\infty)$\\
E.$x \in [12,14] \cup (17,\infty)$\\
F.$x \in (12,14) \cup (17,\infty)$\\
G.$x \in [12,14) \cup (17,\infty)$\\
H.$x \in (12,14] \cup (17,\infty)$
\testStop
\kluczStart
A
\kluczStop



\zadStart{Zadanie z Wikieł Z 1.62 a) moja wersja nr 1067}

Rozwiązać nierówności $(x-12)(x-14)(x-18)\ge0$.
\zadStop
\rozwStart{Patryk Wirkus}{}
Miejsca zerowe naszego wielomianu to: $12, 14, 18$.\\
Wielomian jest stopnia nieparzystego, ponadto znak współczynnika przy\linebreak najwyższej potędze x jest dodatni.\\ W związku z tym wykres wielomianu zaczyna się od lewej strony poniżej osi OX. A więc $$x \in [12,14] \cup [18,\infty).$$
\rozwStop
\odpStart
$x \in [12,14] \cup [18,\infty)$
\odpStop
\testStart
A.$x \in [12,14] \cup [18,\infty)$\\
B.$x \in (12,14) \cup [18,\infty)$\\
C.$x \in (12,14] \cup [18,\infty)$\\
D.$x \in [12,14) \cup [18,\infty)$\\
E.$x \in [12,14] \cup (18,\infty)$\\
F.$x \in (12,14) \cup (18,\infty)$\\
G.$x \in [12,14) \cup (18,\infty)$\\
H.$x \in (12,14] \cup (18,\infty)$
\testStop
\kluczStart
A
\kluczStop



\zadStart{Zadanie z Wikieł Z 1.62 a) moja wersja nr 1068}

Rozwiązać nierówności $(x-12)(x-14)(x-19)\ge0$.
\zadStop
\rozwStart{Patryk Wirkus}{}
Miejsca zerowe naszego wielomianu to: $12, 14, 19$.\\
Wielomian jest stopnia nieparzystego, ponadto znak współczynnika przy\linebreak najwyższej potędze x jest dodatni.\\ W związku z tym wykres wielomianu zaczyna się od lewej strony poniżej osi OX. A więc $$x \in [12,14] \cup [19,\infty).$$
\rozwStop
\odpStart
$x \in [12,14] \cup [19,\infty)$
\odpStop
\testStart
A.$x \in [12,14] \cup [19,\infty)$\\
B.$x \in (12,14) \cup [19,\infty)$\\
C.$x \in (12,14] \cup [19,\infty)$\\
D.$x \in [12,14) \cup [19,\infty)$\\
E.$x \in [12,14] \cup (19,\infty)$\\
F.$x \in (12,14) \cup (19,\infty)$\\
G.$x \in [12,14) \cup (19,\infty)$\\
H.$x \in (12,14] \cup (19,\infty)$
\testStop
\kluczStart
A
\kluczStop



\zadStart{Zadanie z Wikieł Z 1.62 a) moja wersja nr 1069}

Rozwiązać nierówności $(x-12)(x-14)(x-20)\ge0$.
\zadStop
\rozwStart{Patryk Wirkus}{}
Miejsca zerowe naszego wielomianu to: $12, 14, 20$.\\
Wielomian jest stopnia nieparzystego, ponadto znak współczynnika przy\linebreak najwyższej potędze x jest dodatni.\\ W związku z tym wykres wielomianu zaczyna się od lewej strony poniżej osi OX. A więc $$x \in [12,14] \cup [20,\infty).$$
\rozwStop
\odpStart
$x \in [12,14] \cup [20,\infty)$
\odpStop
\testStart
A.$x \in [12,14] \cup [20,\infty)$\\
B.$x \in (12,14) \cup [20,\infty)$\\
C.$x \in (12,14] \cup [20,\infty)$\\
D.$x \in [12,14) \cup [20,\infty)$\\
E.$x \in [12,14] \cup (20,\infty)$\\
F.$x \in (12,14) \cup (20,\infty)$\\
G.$x \in [12,14) \cup (20,\infty)$\\
H.$x \in (12,14] \cup (20,\infty)$
\testStop
\kluczStart
A
\kluczStop



\zadStart{Zadanie z Wikieł Z 1.62 a) moja wersja nr 1070}

Rozwiązać nierówności $(x-12)(x-15)(x-16)\ge0$.
\zadStop
\rozwStart{Patryk Wirkus}{}
Miejsca zerowe naszego wielomianu to: $12, 15, 16$.\\
Wielomian jest stopnia nieparzystego, ponadto znak współczynnika przy\linebreak najwyższej potędze x jest dodatni.\\ W związku z tym wykres wielomianu zaczyna się od lewej strony poniżej osi OX. A więc $$x \in [12,15] \cup [16,\infty).$$
\rozwStop
\odpStart
$x \in [12,15] \cup [16,\infty)$
\odpStop
\testStart
A.$x \in [12,15] \cup [16,\infty)$\\
B.$x \in (12,15) \cup [16,\infty)$\\
C.$x \in (12,15] \cup [16,\infty)$\\
D.$x \in [12,15) \cup [16,\infty)$\\
E.$x \in [12,15] \cup (16,\infty)$\\
F.$x \in (12,15) \cup (16,\infty)$\\
G.$x \in [12,15) \cup (16,\infty)$\\
H.$x \in (12,15] \cup (16,\infty)$
\testStop
\kluczStart
A
\kluczStop



\zadStart{Zadanie z Wikieł Z 1.62 a) moja wersja nr 1071}

Rozwiązać nierówności $(x-12)(x-15)(x-17)\ge0$.
\zadStop
\rozwStart{Patryk Wirkus}{}
Miejsca zerowe naszego wielomianu to: $12, 15, 17$.\\
Wielomian jest stopnia nieparzystego, ponadto znak współczynnika przy\linebreak najwyższej potędze x jest dodatni.\\ W związku z tym wykres wielomianu zaczyna się od lewej strony poniżej osi OX. A więc $$x \in [12,15] \cup [17,\infty).$$
\rozwStop
\odpStart
$x \in [12,15] \cup [17,\infty)$
\odpStop
\testStart
A.$x \in [12,15] \cup [17,\infty)$\\
B.$x \in (12,15) \cup [17,\infty)$\\
C.$x \in (12,15] \cup [17,\infty)$\\
D.$x \in [12,15) \cup [17,\infty)$\\
E.$x \in [12,15] \cup (17,\infty)$\\
F.$x \in (12,15) \cup (17,\infty)$\\
G.$x \in [12,15) \cup (17,\infty)$\\
H.$x \in (12,15] \cup (17,\infty)$
\testStop
\kluczStart
A
\kluczStop



\zadStart{Zadanie z Wikieł Z 1.62 a) moja wersja nr 1072}

Rozwiązać nierówności $(x-12)(x-15)(x-18)\ge0$.
\zadStop
\rozwStart{Patryk Wirkus}{}
Miejsca zerowe naszego wielomianu to: $12, 15, 18$.\\
Wielomian jest stopnia nieparzystego, ponadto znak współczynnika przy\linebreak najwyższej potędze x jest dodatni.\\ W związku z tym wykres wielomianu zaczyna się od lewej strony poniżej osi OX. A więc $$x \in [12,15] \cup [18,\infty).$$
\rozwStop
\odpStart
$x \in [12,15] \cup [18,\infty)$
\odpStop
\testStart
A.$x \in [12,15] \cup [18,\infty)$\\
B.$x \in (12,15) \cup [18,\infty)$\\
C.$x \in (12,15] \cup [18,\infty)$\\
D.$x \in [12,15) \cup [18,\infty)$\\
E.$x \in [12,15] \cup (18,\infty)$\\
F.$x \in (12,15) \cup (18,\infty)$\\
G.$x \in [12,15) \cup (18,\infty)$\\
H.$x \in (12,15] \cup (18,\infty)$
\testStop
\kluczStart
A
\kluczStop



\zadStart{Zadanie z Wikieł Z 1.62 a) moja wersja nr 1073}

Rozwiązać nierówności $(x-12)(x-15)(x-19)\ge0$.
\zadStop
\rozwStart{Patryk Wirkus}{}
Miejsca zerowe naszego wielomianu to: $12, 15, 19$.\\
Wielomian jest stopnia nieparzystego, ponadto znak współczynnika przy\linebreak najwyższej potędze x jest dodatni.\\ W związku z tym wykres wielomianu zaczyna się od lewej strony poniżej osi OX. A więc $$x \in [12,15] \cup [19,\infty).$$
\rozwStop
\odpStart
$x \in [12,15] \cup [19,\infty)$
\odpStop
\testStart
A.$x \in [12,15] \cup [19,\infty)$\\
B.$x \in (12,15) \cup [19,\infty)$\\
C.$x \in (12,15] \cup [19,\infty)$\\
D.$x \in [12,15) \cup [19,\infty)$\\
E.$x \in [12,15] \cup (19,\infty)$\\
F.$x \in (12,15) \cup (19,\infty)$\\
G.$x \in [12,15) \cup (19,\infty)$\\
H.$x \in (12,15] \cup (19,\infty)$
\testStop
\kluczStart
A
\kluczStop



\zadStart{Zadanie z Wikieł Z 1.62 a) moja wersja nr 1074}

Rozwiązać nierówności $(x-12)(x-15)(x-20)\ge0$.
\zadStop
\rozwStart{Patryk Wirkus}{}
Miejsca zerowe naszego wielomianu to: $12, 15, 20$.\\
Wielomian jest stopnia nieparzystego, ponadto znak współczynnika przy\linebreak najwyższej potędze x jest dodatni.\\ W związku z tym wykres wielomianu zaczyna się od lewej strony poniżej osi OX. A więc $$x \in [12,15] \cup [20,\infty).$$
\rozwStop
\odpStart
$x \in [12,15] \cup [20,\infty)$
\odpStop
\testStart
A.$x \in [12,15] \cup [20,\infty)$\\
B.$x \in (12,15) \cup [20,\infty)$\\
C.$x \in (12,15] \cup [20,\infty)$\\
D.$x \in [12,15) \cup [20,\infty)$\\
E.$x \in [12,15] \cup (20,\infty)$\\
F.$x \in (12,15) \cup (20,\infty)$\\
G.$x \in [12,15) \cup (20,\infty)$\\
H.$x \in (12,15] \cup (20,\infty)$
\testStop
\kluczStart
A
\kluczStop



\zadStart{Zadanie z Wikieł Z 1.62 a) moja wersja nr 1075}

Rozwiązać nierówności $(x-12)(x-16)(x-17)\ge0$.
\zadStop
\rozwStart{Patryk Wirkus}{}
Miejsca zerowe naszego wielomianu to: $12, 16, 17$.\\
Wielomian jest stopnia nieparzystego, ponadto znak współczynnika przy\linebreak najwyższej potędze x jest dodatni.\\ W związku z tym wykres wielomianu zaczyna się od lewej strony poniżej osi OX. A więc $$x \in [12,16] \cup [17,\infty).$$
\rozwStop
\odpStart
$x \in [12,16] \cup [17,\infty)$
\odpStop
\testStart
A.$x \in [12,16] \cup [17,\infty)$\\
B.$x \in (12,16) \cup [17,\infty)$\\
C.$x \in (12,16] \cup [17,\infty)$\\
D.$x \in [12,16) \cup [17,\infty)$\\
E.$x \in [12,16] \cup (17,\infty)$\\
F.$x \in (12,16) \cup (17,\infty)$\\
G.$x \in [12,16) \cup (17,\infty)$\\
H.$x \in (12,16] \cup (17,\infty)$
\testStop
\kluczStart
A
\kluczStop



\zadStart{Zadanie z Wikieł Z 1.62 a) moja wersja nr 1076}

Rozwiązać nierówności $(x-12)(x-16)(x-18)\ge0$.
\zadStop
\rozwStart{Patryk Wirkus}{}
Miejsca zerowe naszego wielomianu to: $12, 16, 18$.\\
Wielomian jest stopnia nieparzystego, ponadto znak współczynnika przy\linebreak najwyższej potędze x jest dodatni.\\ W związku z tym wykres wielomianu zaczyna się od lewej strony poniżej osi OX. A więc $$x \in [12,16] \cup [18,\infty).$$
\rozwStop
\odpStart
$x \in [12,16] \cup [18,\infty)$
\odpStop
\testStart
A.$x \in [12,16] \cup [18,\infty)$\\
B.$x \in (12,16) \cup [18,\infty)$\\
C.$x \in (12,16] \cup [18,\infty)$\\
D.$x \in [12,16) \cup [18,\infty)$\\
E.$x \in [12,16] \cup (18,\infty)$\\
F.$x \in (12,16) \cup (18,\infty)$\\
G.$x \in [12,16) \cup (18,\infty)$\\
H.$x \in (12,16] \cup (18,\infty)$
\testStop
\kluczStart
A
\kluczStop



\zadStart{Zadanie z Wikieł Z 1.62 a) moja wersja nr 1077}

Rozwiązać nierówności $(x-12)(x-16)(x-19)\ge0$.
\zadStop
\rozwStart{Patryk Wirkus}{}
Miejsca zerowe naszego wielomianu to: $12, 16, 19$.\\
Wielomian jest stopnia nieparzystego, ponadto znak współczynnika przy\linebreak najwyższej potędze x jest dodatni.\\ W związku z tym wykres wielomianu zaczyna się od lewej strony poniżej osi OX. A więc $$x \in [12,16] \cup [19,\infty).$$
\rozwStop
\odpStart
$x \in [12,16] \cup [19,\infty)$
\odpStop
\testStart
A.$x \in [12,16] \cup [19,\infty)$\\
B.$x \in (12,16) \cup [19,\infty)$\\
C.$x \in (12,16] \cup [19,\infty)$\\
D.$x \in [12,16) \cup [19,\infty)$\\
E.$x \in [12,16] \cup (19,\infty)$\\
F.$x \in (12,16) \cup (19,\infty)$\\
G.$x \in [12,16) \cup (19,\infty)$\\
H.$x \in (12,16] \cup (19,\infty)$
\testStop
\kluczStart
A
\kluczStop



\zadStart{Zadanie z Wikieł Z 1.62 a) moja wersja nr 1078}

Rozwiązać nierówności $(x-12)(x-16)(x-20)\ge0$.
\zadStop
\rozwStart{Patryk Wirkus}{}
Miejsca zerowe naszego wielomianu to: $12, 16, 20$.\\
Wielomian jest stopnia nieparzystego, ponadto znak współczynnika przy\linebreak najwyższej potędze x jest dodatni.\\ W związku z tym wykres wielomianu zaczyna się od lewej strony poniżej osi OX. A więc $$x \in [12,16] \cup [20,\infty).$$
\rozwStop
\odpStart
$x \in [12,16] \cup [20,\infty)$
\odpStop
\testStart
A.$x \in [12,16] \cup [20,\infty)$\\
B.$x \in (12,16) \cup [20,\infty)$\\
C.$x \in (12,16] \cup [20,\infty)$\\
D.$x \in [12,16) \cup [20,\infty)$\\
E.$x \in [12,16] \cup (20,\infty)$\\
F.$x \in (12,16) \cup (20,\infty)$\\
G.$x \in [12,16) \cup (20,\infty)$\\
H.$x \in (12,16] \cup (20,\infty)$
\testStop
\kluczStart
A
\kluczStop



\zadStart{Zadanie z Wikieł Z 1.62 a) moja wersja nr 1079}

Rozwiązać nierówności $(x-12)(x-17)(x-18)\ge0$.
\zadStop
\rozwStart{Patryk Wirkus}{}
Miejsca zerowe naszego wielomianu to: $12, 17, 18$.\\
Wielomian jest stopnia nieparzystego, ponadto znak współczynnika przy\linebreak najwyższej potędze x jest dodatni.\\ W związku z tym wykres wielomianu zaczyna się od lewej strony poniżej osi OX. A więc $$x \in [12,17] \cup [18,\infty).$$
\rozwStop
\odpStart
$x \in [12,17] \cup [18,\infty)$
\odpStop
\testStart
A.$x \in [12,17] \cup [18,\infty)$\\
B.$x \in (12,17) \cup [18,\infty)$\\
C.$x \in (12,17] \cup [18,\infty)$\\
D.$x \in [12,17) \cup [18,\infty)$\\
E.$x \in [12,17] \cup (18,\infty)$\\
F.$x \in (12,17) \cup (18,\infty)$\\
G.$x \in [12,17) \cup (18,\infty)$\\
H.$x \in (12,17] \cup (18,\infty)$
\testStop
\kluczStart
A
\kluczStop



\zadStart{Zadanie z Wikieł Z 1.62 a) moja wersja nr 1080}

Rozwiązać nierówności $(x-12)(x-17)(x-19)\ge0$.
\zadStop
\rozwStart{Patryk Wirkus}{}
Miejsca zerowe naszego wielomianu to: $12, 17, 19$.\\
Wielomian jest stopnia nieparzystego, ponadto znak współczynnika przy\linebreak najwyższej potędze x jest dodatni.\\ W związku z tym wykres wielomianu zaczyna się od lewej strony poniżej osi OX. A więc $$x \in [12,17] \cup [19,\infty).$$
\rozwStop
\odpStart
$x \in [12,17] \cup [19,\infty)$
\odpStop
\testStart
A.$x \in [12,17] \cup [19,\infty)$\\
B.$x \in (12,17) \cup [19,\infty)$\\
C.$x \in (12,17] \cup [19,\infty)$\\
D.$x \in [12,17) \cup [19,\infty)$\\
E.$x \in [12,17] \cup (19,\infty)$\\
F.$x \in (12,17) \cup (19,\infty)$\\
G.$x \in [12,17) \cup (19,\infty)$\\
H.$x \in (12,17] \cup (19,\infty)$
\testStop
\kluczStart
A
\kluczStop



\zadStart{Zadanie z Wikieł Z 1.62 a) moja wersja nr 1081}

Rozwiązać nierówności $(x-12)(x-17)(x-20)\ge0$.
\zadStop
\rozwStart{Patryk Wirkus}{}
Miejsca zerowe naszego wielomianu to: $12, 17, 20$.\\
Wielomian jest stopnia nieparzystego, ponadto znak współczynnika przy\linebreak najwyższej potędze x jest dodatni.\\ W związku z tym wykres wielomianu zaczyna się od lewej strony poniżej osi OX. A więc $$x \in [12,17] \cup [20,\infty).$$
\rozwStop
\odpStart
$x \in [12,17] \cup [20,\infty)$
\odpStop
\testStart
A.$x \in [12,17] \cup [20,\infty)$\\
B.$x \in (12,17) \cup [20,\infty)$\\
C.$x \in (12,17] \cup [20,\infty)$\\
D.$x \in [12,17) \cup [20,\infty)$\\
E.$x \in [12,17] \cup (20,\infty)$\\
F.$x \in (12,17) \cup (20,\infty)$\\
G.$x \in [12,17) \cup (20,\infty)$\\
H.$x \in (12,17] \cup (20,\infty)$
\testStop
\kluczStart
A
\kluczStop



\zadStart{Zadanie z Wikieł Z 1.62 a) moja wersja nr 1082}

Rozwiązać nierówności $(x-12)(x-18)(x-19)\ge0$.
\zadStop
\rozwStart{Patryk Wirkus}{}
Miejsca zerowe naszego wielomianu to: $12, 18, 19$.\\
Wielomian jest stopnia nieparzystego, ponadto znak współczynnika przy\linebreak najwyższej potędze x jest dodatni.\\ W związku z tym wykres wielomianu zaczyna się od lewej strony poniżej osi OX. A więc $$x \in [12,18] \cup [19,\infty).$$
\rozwStop
\odpStart
$x \in [12,18] \cup [19,\infty)$
\odpStop
\testStart
A.$x \in [12,18] \cup [19,\infty)$\\
B.$x \in (12,18) \cup [19,\infty)$\\
C.$x \in (12,18] \cup [19,\infty)$\\
D.$x \in [12,18) \cup [19,\infty)$\\
E.$x \in [12,18] \cup (19,\infty)$\\
F.$x \in (12,18) \cup (19,\infty)$\\
G.$x \in [12,18) \cup (19,\infty)$\\
H.$x \in (12,18] \cup (19,\infty)$
\testStop
\kluczStart
A
\kluczStop



\zadStart{Zadanie z Wikieł Z 1.62 a) moja wersja nr 1083}

Rozwiązać nierówności $(x-12)(x-18)(x-20)\ge0$.
\zadStop
\rozwStart{Patryk Wirkus}{}
Miejsca zerowe naszego wielomianu to: $12, 18, 20$.\\
Wielomian jest stopnia nieparzystego, ponadto znak współczynnika przy\linebreak najwyższej potędze x jest dodatni.\\ W związku z tym wykres wielomianu zaczyna się od lewej strony poniżej osi OX. A więc $$x \in [12,18] \cup [20,\infty).$$
\rozwStop
\odpStart
$x \in [12,18] \cup [20,\infty)$
\odpStop
\testStart
A.$x \in [12,18] \cup [20,\infty)$\\
B.$x \in (12,18) \cup [20,\infty)$\\
C.$x \in (12,18] \cup [20,\infty)$\\
D.$x \in [12,18) \cup [20,\infty)$\\
E.$x \in [12,18] \cup (20,\infty)$\\
F.$x \in (12,18) \cup (20,\infty)$\\
G.$x \in [12,18) \cup (20,\infty)$\\
H.$x \in (12,18] \cup (20,\infty)$
\testStop
\kluczStart
A
\kluczStop



\zadStart{Zadanie z Wikieł Z 1.62 a) moja wersja nr 1084}

Rozwiązać nierówności $(x-12)(x-19)(x-20)\ge0$.
\zadStop
\rozwStart{Patryk Wirkus}{}
Miejsca zerowe naszego wielomianu to: $12, 19, 20$.\\
Wielomian jest stopnia nieparzystego, ponadto znak współczynnika przy\linebreak najwyższej potędze x jest dodatni.\\ W związku z tym wykres wielomianu zaczyna się od lewej strony poniżej osi OX. A więc $$x \in [12,19] \cup [20,\infty).$$
\rozwStop
\odpStart
$x \in [12,19] \cup [20,\infty)$
\odpStop
\testStart
A.$x \in [12,19] \cup [20,\infty)$\\
B.$x \in (12,19) \cup [20,\infty)$\\
C.$x \in (12,19] \cup [20,\infty)$\\
D.$x \in [12,19) \cup [20,\infty)$\\
E.$x \in [12,19] \cup (20,\infty)$\\
F.$x \in (12,19) \cup (20,\infty)$\\
G.$x \in [12,19) \cup (20,\infty)$\\
H.$x \in (12,19] \cup (20,\infty)$
\testStop
\kluczStart
A
\kluczStop



\zadStart{Zadanie z Wikieł Z 1.62 a) moja wersja nr 1085}

Rozwiązać nierówności $(x-13)(x-14)(x-15)\ge0$.
\zadStop
\rozwStart{Patryk Wirkus}{}
Miejsca zerowe naszego wielomianu to: $13, 14, 15$.\\
Wielomian jest stopnia nieparzystego, ponadto znak współczynnika przy\linebreak najwyższej potędze x jest dodatni.\\ W związku z tym wykres wielomianu zaczyna się od lewej strony poniżej osi OX. A więc $$x \in [13,14] \cup [15,\infty).$$
\rozwStop
\odpStart
$x \in [13,14] \cup [15,\infty)$
\odpStop
\testStart
A.$x \in [13,14] \cup [15,\infty)$\\
B.$x \in (13,14) \cup [15,\infty)$\\
C.$x \in (13,14] \cup [15,\infty)$\\
D.$x \in [13,14) \cup [15,\infty)$\\
E.$x \in [13,14] \cup (15,\infty)$\\
F.$x \in (13,14) \cup (15,\infty)$\\
G.$x \in [13,14) \cup (15,\infty)$\\
H.$x \in (13,14] \cup (15,\infty)$
\testStop
\kluczStart
A
\kluczStop



\zadStart{Zadanie z Wikieł Z 1.62 a) moja wersja nr 1086}

Rozwiązać nierówności $(x-13)(x-14)(x-16)\ge0$.
\zadStop
\rozwStart{Patryk Wirkus}{}
Miejsca zerowe naszego wielomianu to: $13, 14, 16$.\\
Wielomian jest stopnia nieparzystego, ponadto znak współczynnika przy\linebreak najwyższej potędze x jest dodatni.\\ W związku z tym wykres wielomianu zaczyna się od lewej strony poniżej osi OX. A więc $$x \in [13,14] \cup [16,\infty).$$
\rozwStop
\odpStart
$x \in [13,14] \cup [16,\infty)$
\odpStop
\testStart
A.$x \in [13,14] \cup [16,\infty)$\\
B.$x \in (13,14) \cup [16,\infty)$\\
C.$x \in (13,14] \cup [16,\infty)$\\
D.$x \in [13,14) \cup [16,\infty)$\\
E.$x \in [13,14] \cup (16,\infty)$\\
F.$x \in (13,14) \cup (16,\infty)$\\
G.$x \in [13,14) \cup (16,\infty)$\\
H.$x \in (13,14] \cup (16,\infty)$
\testStop
\kluczStart
A
\kluczStop



\zadStart{Zadanie z Wikieł Z 1.62 a) moja wersja nr 1087}

Rozwiązać nierówności $(x-13)(x-14)(x-17)\ge0$.
\zadStop
\rozwStart{Patryk Wirkus}{}
Miejsca zerowe naszego wielomianu to: $13, 14, 17$.\\
Wielomian jest stopnia nieparzystego, ponadto znak współczynnika przy\linebreak najwyższej potędze x jest dodatni.\\ W związku z tym wykres wielomianu zaczyna się od lewej strony poniżej osi OX. A więc $$x \in [13,14] \cup [17,\infty).$$
\rozwStop
\odpStart
$x \in [13,14] \cup [17,\infty)$
\odpStop
\testStart
A.$x \in [13,14] \cup [17,\infty)$\\
B.$x \in (13,14) \cup [17,\infty)$\\
C.$x \in (13,14] \cup [17,\infty)$\\
D.$x \in [13,14) \cup [17,\infty)$\\
E.$x \in [13,14] \cup (17,\infty)$\\
F.$x \in (13,14) \cup (17,\infty)$\\
G.$x \in [13,14) \cup (17,\infty)$\\
H.$x \in (13,14] \cup (17,\infty)$
\testStop
\kluczStart
A
\kluczStop



\zadStart{Zadanie z Wikieł Z 1.62 a) moja wersja nr 1088}

Rozwiązać nierówności $(x-13)(x-14)(x-18)\ge0$.
\zadStop
\rozwStart{Patryk Wirkus}{}
Miejsca zerowe naszego wielomianu to: $13, 14, 18$.\\
Wielomian jest stopnia nieparzystego, ponadto znak współczynnika przy\linebreak najwyższej potędze x jest dodatni.\\ W związku z tym wykres wielomianu zaczyna się od lewej strony poniżej osi OX. A więc $$x \in [13,14] \cup [18,\infty).$$
\rozwStop
\odpStart
$x \in [13,14] \cup [18,\infty)$
\odpStop
\testStart
A.$x \in [13,14] \cup [18,\infty)$\\
B.$x \in (13,14) \cup [18,\infty)$\\
C.$x \in (13,14] \cup [18,\infty)$\\
D.$x \in [13,14) \cup [18,\infty)$\\
E.$x \in [13,14] \cup (18,\infty)$\\
F.$x \in (13,14) \cup (18,\infty)$\\
G.$x \in [13,14) \cup (18,\infty)$\\
H.$x \in (13,14] \cup (18,\infty)$
\testStop
\kluczStart
A
\kluczStop



\zadStart{Zadanie z Wikieł Z 1.62 a) moja wersja nr 1089}

Rozwiązać nierówności $(x-13)(x-14)(x-19)\ge0$.
\zadStop
\rozwStart{Patryk Wirkus}{}
Miejsca zerowe naszego wielomianu to: $13, 14, 19$.\\
Wielomian jest stopnia nieparzystego, ponadto znak współczynnika przy\linebreak najwyższej potędze x jest dodatni.\\ W związku z tym wykres wielomianu zaczyna się od lewej strony poniżej osi OX. A więc $$x \in [13,14] \cup [19,\infty).$$
\rozwStop
\odpStart
$x \in [13,14] \cup [19,\infty)$
\odpStop
\testStart
A.$x \in [13,14] \cup [19,\infty)$\\
B.$x \in (13,14) \cup [19,\infty)$\\
C.$x \in (13,14] \cup [19,\infty)$\\
D.$x \in [13,14) \cup [19,\infty)$\\
E.$x \in [13,14] \cup (19,\infty)$\\
F.$x \in (13,14) \cup (19,\infty)$\\
G.$x \in [13,14) \cup (19,\infty)$\\
H.$x \in (13,14] \cup (19,\infty)$
\testStop
\kluczStart
A
\kluczStop



\zadStart{Zadanie z Wikieł Z 1.62 a) moja wersja nr 1090}

Rozwiązać nierówności $(x-13)(x-14)(x-20)\ge0$.
\zadStop
\rozwStart{Patryk Wirkus}{}
Miejsca zerowe naszego wielomianu to: $13, 14, 20$.\\
Wielomian jest stopnia nieparzystego, ponadto znak współczynnika przy\linebreak najwyższej potędze x jest dodatni.\\ W związku z tym wykres wielomianu zaczyna się od lewej strony poniżej osi OX. A więc $$x \in [13,14] \cup [20,\infty).$$
\rozwStop
\odpStart
$x \in [13,14] \cup [20,\infty)$
\odpStop
\testStart
A.$x \in [13,14] \cup [20,\infty)$\\
B.$x \in (13,14) \cup [20,\infty)$\\
C.$x \in (13,14] \cup [20,\infty)$\\
D.$x \in [13,14) \cup [20,\infty)$\\
E.$x \in [13,14] \cup (20,\infty)$\\
F.$x \in (13,14) \cup (20,\infty)$\\
G.$x \in [13,14) \cup (20,\infty)$\\
H.$x \in (13,14] \cup (20,\infty)$
\testStop
\kluczStart
A
\kluczStop



\zadStart{Zadanie z Wikieł Z 1.62 a) moja wersja nr 1091}

Rozwiązać nierówności $(x-13)(x-15)(x-16)\ge0$.
\zadStop
\rozwStart{Patryk Wirkus}{}
Miejsca zerowe naszego wielomianu to: $13, 15, 16$.\\
Wielomian jest stopnia nieparzystego, ponadto znak współczynnika przy\linebreak najwyższej potędze x jest dodatni.\\ W związku z tym wykres wielomianu zaczyna się od lewej strony poniżej osi OX. A więc $$x \in [13,15] \cup [16,\infty).$$
\rozwStop
\odpStart
$x \in [13,15] \cup [16,\infty)$
\odpStop
\testStart
A.$x \in [13,15] \cup [16,\infty)$\\
B.$x \in (13,15) \cup [16,\infty)$\\
C.$x \in (13,15] \cup [16,\infty)$\\
D.$x \in [13,15) \cup [16,\infty)$\\
E.$x \in [13,15] \cup (16,\infty)$\\
F.$x \in (13,15) \cup (16,\infty)$\\
G.$x \in [13,15) \cup (16,\infty)$\\
H.$x \in (13,15] \cup (16,\infty)$
\testStop
\kluczStart
A
\kluczStop



\zadStart{Zadanie z Wikieł Z 1.62 a) moja wersja nr 1092}

Rozwiązać nierówności $(x-13)(x-15)(x-17)\ge0$.
\zadStop
\rozwStart{Patryk Wirkus}{}
Miejsca zerowe naszego wielomianu to: $13, 15, 17$.\\
Wielomian jest stopnia nieparzystego, ponadto znak współczynnika przy\linebreak najwyższej potędze x jest dodatni.\\ W związku z tym wykres wielomianu zaczyna się od lewej strony poniżej osi OX. A więc $$x \in [13,15] \cup [17,\infty).$$
\rozwStop
\odpStart
$x \in [13,15] \cup [17,\infty)$
\odpStop
\testStart
A.$x \in [13,15] \cup [17,\infty)$\\
B.$x \in (13,15) \cup [17,\infty)$\\
C.$x \in (13,15] \cup [17,\infty)$\\
D.$x \in [13,15) \cup [17,\infty)$\\
E.$x \in [13,15] \cup (17,\infty)$\\
F.$x \in (13,15) \cup (17,\infty)$\\
G.$x \in [13,15) \cup (17,\infty)$\\
H.$x \in (13,15] \cup (17,\infty)$
\testStop
\kluczStart
A
\kluczStop



\zadStart{Zadanie z Wikieł Z 1.62 a) moja wersja nr 1093}

Rozwiązać nierówności $(x-13)(x-15)(x-18)\ge0$.
\zadStop
\rozwStart{Patryk Wirkus}{}
Miejsca zerowe naszego wielomianu to: $13, 15, 18$.\\
Wielomian jest stopnia nieparzystego, ponadto znak współczynnika przy\linebreak najwyższej potędze x jest dodatni.\\ W związku z tym wykres wielomianu zaczyna się od lewej strony poniżej osi OX. A więc $$x \in [13,15] \cup [18,\infty).$$
\rozwStop
\odpStart
$x \in [13,15] \cup [18,\infty)$
\odpStop
\testStart
A.$x \in [13,15] \cup [18,\infty)$\\
B.$x \in (13,15) \cup [18,\infty)$\\
C.$x \in (13,15] \cup [18,\infty)$\\
D.$x \in [13,15) \cup [18,\infty)$\\
E.$x \in [13,15] \cup (18,\infty)$\\
F.$x \in (13,15) \cup (18,\infty)$\\
G.$x \in [13,15) \cup (18,\infty)$\\
H.$x \in (13,15] \cup (18,\infty)$
\testStop
\kluczStart
A
\kluczStop



\zadStart{Zadanie z Wikieł Z 1.62 a) moja wersja nr 1094}

Rozwiązać nierówności $(x-13)(x-15)(x-19)\ge0$.
\zadStop
\rozwStart{Patryk Wirkus}{}
Miejsca zerowe naszego wielomianu to: $13, 15, 19$.\\
Wielomian jest stopnia nieparzystego, ponadto znak współczynnika przy\linebreak najwyższej potędze x jest dodatni.\\ W związku z tym wykres wielomianu zaczyna się od lewej strony poniżej osi OX. A więc $$x \in [13,15] \cup [19,\infty).$$
\rozwStop
\odpStart
$x \in [13,15] \cup [19,\infty)$
\odpStop
\testStart
A.$x \in [13,15] \cup [19,\infty)$\\
B.$x \in (13,15) \cup [19,\infty)$\\
C.$x \in (13,15] \cup [19,\infty)$\\
D.$x \in [13,15) \cup [19,\infty)$\\
E.$x \in [13,15] \cup (19,\infty)$\\
F.$x \in (13,15) \cup (19,\infty)$\\
G.$x \in [13,15) \cup (19,\infty)$\\
H.$x \in (13,15] \cup (19,\infty)$
\testStop
\kluczStart
A
\kluczStop



\zadStart{Zadanie z Wikieł Z 1.62 a) moja wersja nr 1095}

Rozwiązać nierówności $(x-13)(x-15)(x-20)\ge0$.
\zadStop
\rozwStart{Patryk Wirkus}{}
Miejsca zerowe naszego wielomianu to: $13, 15, 20$.\\
Wielomian jest stopnia nieparzystego, ponadto znak współczynnika przy\linebreak najwyższej potędze x jest dodatni.\\ W związku z tym wykres wielomianu zaczyna się od lewej strony poniżej osi OX. A więc $$x \in [13,15] \cup [20,\infty).$$
\rozwStop
\odpStart
$x \in [13,15] \cup [20,\infty)$
\odpStop
\testStart
A.$x \in [13,15] \cup [20,\infty)$\\
B.$x \in (13,15) \cup [20,\infty)$\\
C.$x \in (13,15] \cup [20,\infty)$\\
D.$x \in [13,15) \cup [20,\infty)$\\
E.$x \in [13,15] \cup (20,\infty)$\\
F.$x \in (13,15) \cup (20,\infty)$\\
G.$x \in [13,15) \cup (20,\infty)$\\
H.$x \in (13,15] \cup (20,\infty)$
\testStop
\kluczStart
A
\kluczStop



\zadStart{Zadanie z Wikieł Z 1.62 a) moja wersja nr 1096}

Rozwiązać nierówności $(x-13)(x-16)(x-17)\ge0$.
\zadStop
\rozwStart{Patryk Wirkus}{}
Miejsca zerowe naszego wielomianu to: $13, 16, 17$.\\
Wielomian jest stopnia nieparzystego, ponadto znak współczynnika przy\linebreak najwyższej potędze x jest dodatni.\\ W związku z tym wykres wielomianu zaczyna się od lewej strony poniżej osi OX. A więc $$x \in [13,16] \cup [17,\infty).$$
\rozwStop
\odpStart
$x \in [13,16] \cup [17,\infty)$
\odpStop
\testStart
A.$x \in [13,16] \cup [17,\infty)$\\
B.$x \in (13,16) \cup [17,\infty)$\\
C.$x \in (13,16] \cup [17,\infty)$\\
D.$x \in [13,16) \cup [17,\infty)$\\
E.$x \in [13,16] \cup (17,\infty)$\\
F.$x \in (13,16) \cup (17,\infty)$\\
G.$x \in [13,16) \cup (17,\infty)$\\
H.$x \in (13,16] \cup (17,\infty)$
\testStop
\kluczStart
A
\kluczStop



\zadStart{Zadanie z Wikieł Z 1.62 a) moja wersja nr 1097}

Rozwiązać nierówności $(x-13)(x-16)(x-18)\ge0$.
\zadStop
\rozwStart{Patryk Wirkus}{}
Miejsca zerowe naszego wielomianu to: $13, 16, 18$.\\
Wielomian jest stopnia nieparzystego, ponadto znak współczynnika przy\linebreak najwyższej potędze x jest dodatni.\\ W związku z tym wykres wielomianu zaczyna się od lewej strony poniżej osi OX. A więc $$x \in [13,16] \cup [18,\infty).$$
\rozwStop
\odpStart
$x \in [13,16] \cup [18,\infty)$
\odpStop
\testStart
A.$x \in [13,16] \cup [18,\infty)$\\
B.$x \in (13,16) \cup [18,\infty)$\\
C.$x \in (13,16] \cup [18,\infty)$\\
D.$x \in [13,16) \cup [18,\infty)$\\
E.$x \in [13,16] \cup (18,\infty)$\\
F.$x \in (13,16) \cup (18,\infty)$\\
G.$x \in [13,16) \cup (18,\infty)$\\
H.$x \in (13,16] \cup (18,\infty)$
\testStop
\kluczStart
A
\kluczStop



\zadStart{Zadanie z Wikieł Z 1.62 a) moja wersja nr 1098}

Rozwiązać nierówności $(x-13)(x-16)(x-19)\ge0$.
\zadStop
\rozwStart{Patryk Wirkus}{}
Miejsca zerowe naszego wielomianu to: $13, 16, 19$.\\
Wielomian jest stopnia nieparzystego, ponadto znak współczynnika przy\linebreak najwyższej potędze x jest dodatni.\\ W związku z tym wykres wielomianu zaczyna się od lewej strony poniżej osi OX. A więc $$x \in [13,16] \cup [19,\infty).$$
\rozwStop
\odpStart
$x \in [13,16] \cup [19,\infty)$
\odpStop
\testStart
A.$x \in [13,16] \cup [19,\infty)$\\
B.$x \in (13,16) \cup [19,\infty)$\\
C.$x \in (13,16] \cup [19,\infty)$\\
D.$x \in [13,16) \cup [19,\infty)$\\
E.$x \in [13,16] \cup (19,\infty)$\\
F.$x \in (13,16) \cup (19,\infty)$\\
G.$x \in [13,16) \cup (19,\infty)$\\
H.$x \in (13,16] \cup (19,\infty)$
\testStop
\kluczStart
A
\kluczStop



\zadStart{Zadanie z Wikieł Z 1.62 a) moja wersja nr 1099}

Rozwiązać nierówności $(x-13)(x-16)(x-20)\ge0$.
\zadStop
\rozwStart{Patryk Wirkus}{}
Miejsca zerowe naszego wielomianu to: $13, 16, 20$.\\
Wielomian jest stopnia nieparzystego, ponadto znak współczynnika przy\linebreak najwyższej potędze x jest dodatni.\\ W związku z tym wykres wielomianu zaczyna się od lewej strony poniżej osi OX. A więc $$x \in [13,16] \cup [20,\infty).$$
\rozwStop
\odpStart
$x \in [13,16] \cup [20,\infty)$
\odpStop
\testStart
A.$x \in [13,16] \cup [20,\infty)$\\
B.$x \in (13,16) \cup [20,\infty)$\\
C.$x \in (13,16] \cup [20,\infty)$\\
D.$x \in [13,16) \cup [20,\infty)$\\
E.$x \in [13,16] \cup (20,\infty)$\\
F.$x \in (13,16) \cup (20,\infty)$\\
G.$x \in [13,16) \cup (20,\infty)$\\
H.$x \in (13,16] \cup (20,\infty)$
\testStop
\kluczStart
A
\kluczStop



\zadStart{Zadanie z Wikieł Z 1.62 a) moja wersja nr 1100}

Rozwiązać nierówności $(x-13)(x-17)(x-18)\ge0$.
\zadStop
\rozwStart{Patryk Wirkus}{}
Miejsca zerowe naszego wielomianu to: $13, 17, 18$.\\
Wielomian jest stopnia nieparzystego, ponadto znak współczynnika przy\linebreak najwyższej potędze x jest dodatni.\\ W związku z tym wykres wielomianu zaczyna się od lewej strony poniżej osi OX. A więc $$x \in [13,17] \cup [18,\infty).$$
\rozwStop
\odpStart
$x \in [13,17] \cup [18,\infty)$
\odpStop
\testStart
A.$x \in [13,17] \cup [18,\infty)$\\
B.$x \in (13,17) \cup [18,\infty)$\\
C.$x \in (13,17] \cup [18,\infty)$\\
D.$x \in [13,17) \cup [18,\infty)$\\
E.$x \in [13,17] \cup (18,\infty)$\\
F.$x \in (13,17) \cup (18,\infty)$\\
G.$x \in [13,17) \cup (18,\infty)$\\
H.$x \in (13,17] \cup (18,\infty)$
\testStop
\kluczStart
A
\kluczStop



\zadStart{Zadanie z Wikieł Z 1.62 a) moja wersja nr 1101}

Rozwiązać nierówności $(x-13)(x-17)(x-19)\ge0$.
\zadStop
\rozwStart{Patryk Wirkus}{}
Miejsca zerowe naszego wielomianu to: $13, 17, 19$.\\
Wielomian jest stopnia nieparzystego, ponadto znak współczynnika przy\linebreak najwyższej potędze x jest dodatni.\\ W związku z tym wykres wielomianu zaczyna się od lewej strony poniżej osi OX. A więc $$x \in [13,17] \cup [19,\infty).$$
\rozwStop
\odpStart
$x \in [13,17] \cup [19,\infty)$
\odpStop
\testStart
A.$x \in [13,17] \cup [19,\infty)$\\
B.$x \in (13,17) \cup [19,\infty)$\\
C.$x \in (13,17] \cup [19,\infty)$\\
D.$x \in [13,17) \cup [19,\infty)$\\
E.$x \in [13,17] \cup (19,\infty)$\\
F.$x \in (13,17) \cup (19,\infty)$\\
G.$x \in [13,17) \cup (19,\infty)$\\
H.$x \in (13,17] \cup (19,\infty)$
\testStop
\kluczStart
A
\kluczStop



\zadStart{Zadanie z Wikieł Z 1.62 a) moja wersja nr 1102}

Rozwiązać nierówności $(x-13)(x-17)(x-20)\ge0$.
\zadStop
\rozwStart{Patryk Wirkus}{}
Miejsca zerowe naszego wielomianu to: $13, 17, 20$.\\
Wielomian jest stopnia nieparzystego, ponadto znak współczynnika przy\linebreak najwyższej potędze x jest dodatni.\\ W związku z tym wykres wielomianu zaczyna się od lewej strony poniżej osi OX. A więc $$x \in [13,17] \cup [20,\infty).$$
\rozwStop
\odpStart
$x \in [13,17] \cup [20,\infty)$
\odpStop
\testStart
A.$x \in [13,17] \cup [20,\infty)$\\
B.$x \in (13,17) \cup [20,\infty)$\\
C.$x \in (13,17] \cup [20,\infty)$\\
D.$x \in [13,17) \cup [20,\infty)$\\
E.$x \in [13,17] \cup (20,\infty)$\\
F.$x \in (13,17) \cup (20,\infty)$\\
G.$x \in [13,17) \cup (20,\infty)$\\
H.$x \in (13,17] \cup (20,\infty)$
\testStop
\kluczStart
A
\kluczStop



\zadStart{Zadanie z Wikieł Z 1.62 a) moja wersja nr 1103}

Rozwiązać nierówności $(x-13)(x-18)(x-19)\ge0$.
\zadStop
\rozwStart{Patryk Wirkus}{}
Miejsca zerowe naszego wielomianu to: $13, 18, 19$.\\
Wielomian jest stopnia nieparzystego, ponadto znak współczynnika przy\linebreak najwyższej potędze x jest dodatni.\\ W związku z tym wykres wielomianu zaczyna się od lewej strony poniżej osi OX. A więc $$x \in [13,18] \cup [19,\infty).$$
\rozwStop
\odpStart
$x \in [13,18] \cup [19,\infty)$
\odpStop
\testStart
A.$x \in [13,18] \cup [19,\infty)$\\
B.$x \in (13,18) \cup [19,\infty)$\\
C.$x \in (13,18] \cup [19,\infty)$\\
D.$x \in [13,18) \cup [19,\infty)$\\
E.$x \in [13,18] \cup (19,\infty)$\\
F.$x \in (13,18) \cup (19,\infty)$\\
G.$x \in [13,18) \cup (19,\infty)$\\
H.$x \in (13,18] \cup (19,\infty)$
\testStop
\kluczStart
A
\kluczStop



\zadStart{Zadanie z Wikieł Z 1.62 a) moja wersja nr 1104}

Rozwiązać nierówności $(x-13)(x-18)(x-20)\ge0$.
\zadStop
\rozwStart{Patryk Wirkus}{}
Miejsca zerowe naszego wielomianu to: $13, 18, 20$.\\
Wielomian jest stopnia nieparzystego, ponadto znak współczynnika przy\linebreak najwyższej potędze x jest dodatni.\\ W związku z tym wykres wielomianu zaczyna się od lewej strony poniżej osi OX. A więc $$x \in [13,18] \cup [20,\infty).$$
\rozwStop
\odpStart
$x \in [13,18] \cup [20,\infty)$
\odpStop
\testStart
A.$x \in [13,18] \cup [20,\infty)$\\
B.$x \in (13,18) \cup [20,\infty)$\\
C.$x \in (13,18] \cup [20,\infty)$\\
D.$x \in [13,18) \cup [20,\infty)$\\
E.$x \in [13,18] \cup (20,\infty)$\\
F.$x \in (13,18) \cup (20,\infty)$\\
G.$x \in [13,18) \cup (20,\infty)$\\
H.$x \in (13,18] \cup (20,\infty)$
\testStop
\kluczStart
A
\kluczStop



\zadStart{Zadanie z Wikieł Z 1.62 a) moja wersja nr 1105}

Rozwiązać nierówności $(x-13)(x-19)(x-20)\ge0$.
\zadStop
\rozwStart{Patryk Wirkus}{}
Miejsca zerowe naszego wielomianu to: $13, 19, 20$.\\
Wielomian jest stopnia nieparzystego, ponadto znak współczynnika przy\linebreak najwyższej potędze x jest dodatni.\\ W związku z tym wykres wielomianu zaczyna się od lewej strony poniżej osi OX. A więc $$x \in [13,19] \cup [20,\infty).$$
\rozwStop
\odpStart
$x \in [13,19] \cup [20,\infty)$
\odpStop
\testStart
A.$x \in [13,19] \cup [20,\infty)$\\
B.$x \in (13,19) \cup [20,\infty)$\\
C.$x \in (13,19] \cup [20,\infty)$\\
D.$x \in [13,19) \cup [20,\infty)$\\
E.$x \in [13,19] \cup (20,\infty)$\\
F.$x \in (13,19) \cup (20,\infty)$\\
G.$x \in [13,19) \cup (20,\infty)$\\
H.$x \in (13,19] \cup (20,\infty)$
\testStop
\kluczStart
A
\kluczStop



\zadStart{Zadanie z Wikieł Z 1.62 a) moja wersja nr 1106}

Rozwiązać nierówności $(x-14)(x-15)(x-16)\ge0$.
\zadStop
\rozwStart{Patryk Wirkus}{}
Miejsca zerowe naszego wielomianu to: $14, 15, 16$.\\
Wielomian jest stopnia nieparzystego, ponadto znak współczynnika przy\linebreak najwyższej potędze x jest dodatni.\\ W związku z tym wykres wielomianu zaczyna się od lewej strony poniżej osi OX. A więc $$x \in [14,15] \cup [16,\infty).$$
\rozwStop
\odpStart
$x \in [14,15] \cup [16,\infty)$
\odpStop
\testStart
A.$x \in [14,15] \cup [16,\infty)$\\
B.$x \in (14,15) \cup [16,\infty)$\\
C.$x \in (14,15] \cup [16,\infty)$\\
D.$x \in [14,15) \cup [16,\infty)$\\
E.$x \in [14,15] \cup (16,\infty)$\\
F.$x \in (14,15) \cup (16,\infty)$\\
G.$x \in [14,15) \cup (16,\infty)$\\
H.$x \in (14,15] \cup (16,\infty)$
\testStop
\kluczStart
A
\kluczStop



\zadStart{Zadanie z Wikieł Z 1.62 a) moja wersja nr 1107}

Rozwiązać nierówności $(x-14)(x-15)(x-17)\ge0$.
\zadStop
\rozwStart{Patryk Wirkus}{}
Miejsca zerowe naszego wielomianu to: $14, 15, 17$.\\
Wielomian jest stopnia nieparzystego, ponadto znak współczynnika przy\linebreak najwyższej potędze x jest dodatni.\\ W związku z tym wykres wielomianu zaczyna się od lewej strony poniżej osi OX. A więc $$x \in [14,15] \cup [17,\infty).$$
\rozwStop
\odpStart
$x \in [14,15] \cup [17,\infty)$
\odpStop
\testStart
A.$x \in [14,15] \cup [17,\infty)$\\
B.$x \in (14,15) \cup [17,\infty)$\\
C.$x \in (14,15] \cup [17,\infty)$\\
D.$x \in [14,15) \cup [17,\infty)$\\
E.$x \in [14,15] \cup (17,\infty)$\\
F.$x \in (14,15) \cup (17,\infty)$\\
G.$x \in [14,15) \cup (17,\infty)$\\
H.$x \in (14,15] \cup (17,\infty)$
\testStop
\kluczStart
A
\kluczStop



\zadStart{Zadanie z Wikieł Z 1.62 a) moja wersja nr 1108}

Rozwiązać nierówności $(x-14)(x-15)(x-18)\ge0$.
\zadStop
\rozwStart{Patryk Wirkus}{}
Miejsca zerowe naszego wielomianu to: $14, 15, 18$.\\
Wielomian jest stopnia nieparzystego, ponadto znak współczynnika przy\linebreak najwyższej potędze x jest dodatni.\\ W związku z tym wykres wielomianu zaczyna się od lewej strony poniżej osi OX. A więc $$x \in [14,15] \cup [18,\infty).$$
\rozwStop
\odpStart
$x \in [14,15] \cup [18,\infty)$
\odpStop
\testStart
A.$x \in [14,15] \cup [18,\infty)$\\
B.$x \in (14,15) \cup [18,\infty)$\\
C.$x \in (14,15] \cup [18,\infty)$\\
D.$x \in [14,15) \cup [18,\infty)$\\
E.$x \in [14,15] \cup (18,\infty)$\\
F.$x \in (14,15) \cup (18,\infty)$\\
G.$x \in [14,15) \cup (18,\infty)$\\
H.$x \in (14,15] \cup (18,\infty)$
\testStop
\kluczStart
A
\kluczStop



\zadStart{Zadanie z Wikieł Z 1.62 a) moja wersja nr 1109}

Rozwiązać nierówności $(x-14)(x-15)(x-19)\ge0$.
\zadStop
\rozwStart{Patryk Wirkus}{}
Miejsca zerowe naszego wielomianu to: $14, 15, 19$.\\
Wielomian jest stopnia nieparzystego, ponadto znak współczynnika przy\linebreak najwyższej potędze x jest dodatni.\\ W związku z tym wykres wielomianu zaczyna się od lewej strony poniżej osi OX. A więc $$x \in [14,15] \cup [19,\infty).$$
\rozwStop
\odpStart
$x \in [14,15] \cup [19,\infty)$
\odpStop
\testStart
A.$x \in [14,15] \cup [19,\infty)$\\
B.$x \in (14,15) \cup [19,\infty)$\\
C.$x \in (14,15] \cup [19,\infty)$\\
D.$x \in [14,15) \cup [19,\infty)$\\
E.$x \in [14,15] \cup (19,\infty)$\\
F.$x \in (14,15) \cup (19,\infty)$\\
G.$x \in [14,15) \cup (19,\infty)$\\
H.$x \in (14,15] \cup (19,\infty)$
\testStop
\kluczStart
A
\kluczStop



\zadStart{Zadanie z Wikieł Z 1.62 a) moja wersja nr 1110}

Rozwiązać nierówności $(x-14)(x-15)(x-20)\ge0$.
\zadStop
\rozwStart{Patryk Wirkus}{}
Miejsca zerowe naszego wielomianu to: $14, 15, 20$.\\
Wielomian jest stopnia nieparzystego, ponadto znak współczynnika przy\linebreak najwyższej potędze x jest dodatni.\\ W związku z tym wykres wielomianu zaczyna się od lewej strony poniżej osi OX. A więc $$x \in [14,15] \cup [20,\infty).$$
\rozwStop
\odpStart
$x \in [14,15] \cup [20,\infty)$
\odpStop
\testStart
A.$x \in [14,15] \cup [20,\infty)$\\
B.$x \in (14,15) \cup [20,\infty)$\\
C.$x \in (14,15] \cup [20,\infty)$\\
D.$x \in [14,15) \cup [20,\infty)$\\
E.$x \in [14,15] \cup (20,\infty)$\\
F.$x \in (14,15) \cup (20,\infty)$\\
G.$x \in [14,15) \cup (20,\infty)$\\
H.$x \in (14,15] \cup (20,\infty)$
\testStop
\kluczStart
A
\kluczStop



\zadStart{Zadanie z Wikieł Z 1.62 a) moja wersja nr 1111}

Rozwiązać nierówności $(x-14)(x-16)(x-17)\ge0$.
\zadStop
\rozwStart{Patryk Wirkus}{}
Miejsca zerowe naszego wielomianu to: $14, 16, 17$.\\
Wielomian jest stopnia nieparzystego, ponadto znak współczynnika przy\linebreak najwyższej potędze x jest dodatni.\\ W związku z tym wykres wielomianu zaczyna się od lewej strony poniżej osi OX. A więc $$x \in [14,16] \cup [17,\infty).$$
\rozwStop
\odpStart
$x \in [14,16] \cup [17,\infty)$
\odpStop
\testStart
A.$x \in [14,16] \cup [17,\infty)$\\
B.$x \in (14,16) \cup [17,\infty)$\\
C.$x \in (14,16] \cup [17,\infty)$\\
D.$x \in [14,16) \cup [17,\infty)$\\
E.$x \in [14,16] \cup (17,\infty)$\\
F.$x \in (14,16) \cup (17,\infty)$\\
G.$x \in [14,16) \cup (17,\infty)$\\
H.$x \in (14,16] \cup (17,\infty)$
\testStop
\kluczStart
A
\kluczStop



\zadStart{Zadanie z Wikieł Z 1.62 a) moja wersja nr 1112}

Rozwiązać nierówności $(x-14)(x-16)(x-18)\ge0$.
\zadStop
\rozwStart{Patryk Wirkus}{}
Miejsca zerowe naszego wielomianu to: $14, 16, 18$.\\
Wielomian jest stopnia nieparzystego, ponadto znak współczynnika przy\linebreak najwyższej potędze x jest dodatni.\\ W związku z tym wykres wielomianu zaczyna się od lewej strony poniżej osi OX. A więc $$x \in [14,16] \cup [18,\infty).$$
\rozwStop
\odpStart
$x \in [14,16] \cup [18,\infty)$
\odpStop
\testStart
A.$x \in [14,16] \cup [18,\infty)$\\
B.$x \in (14,16) \cup [18,\infty)$\\
C.$x \in (14,16] \cup [18,\infty)$\\
D.$x \in [14,16) \cup [18,\infty)$\\
E.$x \in [14,16] \cup (18,\infty)$\\
F.$x \in (14,16) \cup (18,\infty)$\\
G.$x \in [14,16) \cup (18,\infty)$\\
H.$x \in (14,16] \cup (18,\infty)$
\testStop
\kluczStart
A
\kluczStop



\zadStart{Zadanie z Wikieł Z 1.62 a) moja wersja nr 1113}

Rozwiązać nierówności $(x-14)(x-16)(x-19)\ge0$.
\zadStop
\rozwStart{Patryk Wirkus}{}
Miejsca zerowe naszego wielomianu to: $14, 16, 19$.\\
Wielomian jest stopnia nieparzystego, ponadto znak współczynnika przy\linebreak najwyższej potędze x jest dodatni.\\ W związku z tym wykres wielomianu zaczyna się od lewej strony poniżej osi OX. A więc $$x \in [14,16] \cup [19,\infty).$$
\rozwStop
\odpStart
$x \in [14,16] \cup [19,\infty)$
\odpStop
\testStart
A.$x \in [14,16] \cup [19,\infty)$\\
B.$x \in (14,16) \cup [19,\infty)$\\
C.$x \in (14,16] \cup [19,\infty)$\\
D.$x \in [14,16) \cup [19,\infty)$\\
E.$x \in [14,16] \cup (19,\infty)$\\
F.$x \in (14,16) \cup (19,\infty)$\\
G.$x \in [14,16) \cup (19,\infty)$\\
H.$x \in (14,16] \cup (19,\infty)$
\testStop
\kluczStart
A
\kluczStop



\zadStart{Zadanie z Wikieł Z 1.62 a) moja wersja nr 1114}

Rozwiązać nierówności $(x-14)(x-16)(x-20)\ge0$.
\zadStop
\rozwStart{Patryk Wirkus}{}
Miejsca zerowe naszego wielomianu to: $14, 16, 20$.\\
Wielomian jest stopnia nieparzystego, ponadto znak współczynnika przy\linebreak najwyższej potędze x jest dodatni.\\ W związku z tym wykres wielomianu zaczyna się od lewej strony poniżej osi OX. A więc $$x \in [14,16] \cup [20,\infty).$$
\rozwStop
\odpStart
$x \in [14,16] \cup [20,\infty)$
\odpStop
\testStart
A.$x \in [14,16] \cup [20,\infty)$\\
B.$x \in (14,16) \cup [20,\infty)$\\
C.$x \in (14,16] \cup [20,\infty)$\\
D.$x \in [14,16) \cup [20,\infty)$\\
E.$x \in [14,16] \cup (20,\infty)$\\
F.$x \in (14,16) \cup (20,\infty)$\\
G.$x \in [14,16) \cup (20,\infty)$\\
H.$x \in (14,16] \cup (20,\infty)$
\testStop
\kluczStart
A
\kluczStop



\zadStart{Zadanie z Wikieł Z 1.62 a) moja wersja nr 1115}

Rozwiązać nierówności $(x-14)(x-17)(x-18)\ge0$.
\zadStop
\rozwStart{Patryk Wirkus}{}
Miejsca zerowe naszego wielomianu to: $14, 17, 18$.\\
Wielomian jest stopnia nieparzystego, ponadto znak współczynnika przy\linebreak najwyższej potędze x jest dodatni.\\ W związku z tym wykres wielomianu zaczyna się od lewej strony poniżej osi OX. A więc $$x \in [14,17] \cup [18,\infty).$$
\rozwStop
\odpStart
$x \in [14,17] \cup [18,\infty)$
\odpStop
\testStart
A.$x \in [14,17] \cup [18,\infty)$\\
B.$x \in (14,17) \cup [18,\infty)$\\
C.$x \in (14,17] \cup [18,\infty)$\\
D.$x \in [14,17) \cup [18,\infty)$\\
E.$x \in [14,17] \cup (18,\infty)$\\
F.$x \in (14,17) \cup (18,\infty)$\\
G.$x \in [14,17) \cup (18,\infty)$\\
H.$x \in (14,17] \cup (18,\infty)$
\testStop
\kluczStart
A
\kluczStop



\zadStart{Zadanie z Wikieł Z 1.62 a) moja wersja nr 1116}

Rozwiązać nierówności $(x-14)(x-17)(x-19)\ge0$.
\zadStop
\rozwStart{Patryk Wirkus}{}
Miejsca zerowe naszego wielomianu to: $14, 17, 19$.\\
Wielomian jest stopnia nieparzystego, ponadto znak współczynnika przy\linebreak najwyższej potędze x jest dodatni.\\ W związku z tym wykres wielomianu zaczyna się od lewej strony poniżej osi OX. A więc $$x \in [14,17] \cup [19,\infty).$$
\rozwStop
\odpStart
$x \in [14,17] \cup [19,\infty)$
\odpStop
\testStart
A.$x \in [14,17] \cup [19,\infty)$\\
B.$x \in (14,17) \cup [19,\infty)$\\
C.$x \in (14,17] \cup [19,\infty)$\\
D.$x \in [14,17) \cup [19,\infty)$\\
E.$x \in [14,17] \cup (19,\infty)$\\
F.$x \in (14,17) \cup (19,\infty)$\\
G.$x \in [14,17) \cup (19,\infty)$\\
H.$x \in (14,17] \cup (19,\infty)$
\testStop
\kluczStart
A
\kluczStop



\zadStart{Zadanie z Wikieł Z 1.62 a) moja wersja nr 1117}

Rozwiązać nierówności $(x-14)(x-17)(x-20)\ge0$.
\zadStop
\rozwStart{Patryk Wirkus}{}
Miejsca zerowe naszego wielomianu to: $14, 17, 20$.\\
Wielomian jest stopnia nieparzystego, ponadto znak współczynnika przy\linebreak najwyższej potędze x jest dodatni.\\ W związku z tym wykres wielomianu zaczyna się od lewej strony poniżej osi OX. A więc $$x \in [14,17] \cup [20,\infty).$$
\rozwStop
\odpStart
$x \in [14,17] \cup [20,\infty)$
\odpStop
\testStart
A.$x \in [14,17] \cup [20,\infty)$\\
B.$x \in (14,17) \cup [20,\infty)$\\
C.$x \in (14,17] \cup [20,\infty)$\\
D.$x \in [14,17) \cup [20,\infty)$\\
E.$x \in [14,17] \cup (20,\infty)$\\
F.$x \in (14,17) \cup (20,\infty)$\\
G.$x \in [14,17) \cup (20,\infty)$\\
H.$x \in (14,17] \cup (20,\infty)$
\testStop
\kluczStart
A
\kluczStop



\zadStart{Zadanie z Wikieł Z 1.62 a) moja wersja nr 1118}

Rozwiązać nierówności $(x-14)(x-18)(x-19)\ge0$.
\zadStop
\rozwStart{Patryk Wirkus}{}
Miejsca zerowe naszego wielomianu to: $14, 18, 19$.\\
Wielomian jest stopnia nieparzystego, ponadto znak współczynnika przy\linebreak najwyższej potędze x jest dodatni.\\ W związku z tym wykres wielomianu zaczyna się od lewej strony poniżej osi OX. A więc $$x \in [14,18] \cup [19,\infty).$$
\rozwStop
\odpStart
$x \in [14,18] \cup [19,\infty)$
\odpStop
\testStart
A.$x \in [14,18] \cup [19,\infty)$\\
B.$x \in (14,18) \cup [19,\infty)$\\
C.$x \in (14,18] \cup [19,\infty)$\\
D.$x \in [14,18) \cup [19,\infty)$\\
E.$x \in [14,18] \cup (19,\infty)$\\
F.$x \in (14,18) \cup (19,\infty)$\\
G.$x \in [14,18) \cup (19,\infty)$\\
H.$x \in (14,18] \cup (19,\infty)$
\testStop
\kluczStart
A
\kluczStop



\zadStart{Zadanie z Wikieł Z 1.62 a) moja wersja nr 1119}

Rozwiązać nierówności $(x-14)(x-18)(x-20)\ge0$.
\zadStop
\rozwStart{Patryk Wirkus}{}
Miejsca zerowe naszego wielomianu to: $14, 18, 20$.\\
Wielomian jest stopnia nieparzystego, ponadto znak współczynnika przy\linebreak najwyższej potędze x jest dodatni.\\ W związku z tym wykres wielomianu zaczyna się od lewej strony poniżej osi OX. A więc $$x \in [14,18] \cup [20,\infty).$$
\rozwStop
\odpStart
$x \in [14,18] \cup [20,\infty)$
\odpStop
\testStart
A.$x \in [14,18] \cup [20,\infty)$\\
B.$x \in (14,18) \cup [20,\infty)$\\
C.$x \in (14,18] \cup [20,\infty)$\\
D.$x \in [14,18) \cup [20,\infty)$\\
E.$x \in [14,18] \cup (20,\infty)$\\
F.$x \in (14,18) \cup (20,\infty)$\\
G.$x \in [14,18) \cup (20,\infty)$\\
H.$x \in (14,18] \cup (20,\infty)$
\testStop
\kluczStart
A
\kluczStop



\zadStart{Zadanie z Wikieł Z 1.62 a) moja wersja nr 1120}

Rozwiązać nierówności $(x-14)(x-19)(x-20)\ge0$.
\zadStop
\rozwStart{Patryk Wirkus}{}
Miejsca zerowe naszego wielomianu to: $14, 19, 20$.\\
Wielomian jest stopnia nieparzystego, ponadto znak współczynnika przy\linebreak najwyższej potędze x jest dodatni.\\ W związku z tym wykres wielomianu zaczyna się od lewej strony poniżej osi OX. A więc $$x \in [14,19] \cup [20,\infty).$$
\rozwStop
\odpStart
$x \in [14,19] \cup [20,\infty)$
\odpStop
\testStart
A.$x \in [14,19] \cup [20,\infty)$\\
B.$x \in (14,19) \cup [20,\infty)$\\
C.$x \in (14,19] \cup [20,\infty)$\\
D.$x \in [14,19) \cup [20,\infty)$\\
E.$x \in [14,19] \cup (20,\infty)$\\
F.$x \in (14,19) \cup (20,\infty)$\\
G.$x \in [14,19) \cup (20,\infty)$\\
H.$x \in (14,19] \cup (20,\infty)$
\testStop
\kluczStart
A
\kluczStop



\zadStart{Zadanie z Wikieł Z 1.62 a) moja wersja nr 1121}

Rozwiązać nierówności $(x-15)(x-16)(x-17)\ge0$.
\zadStop
\rozwStart{Patryk Wirkus}{}
Miejsca zerowe naszego wielomianu to: $15, 16, 17$.\\
Wielomian jest stopnia nieparzystego, ponadto znak współczynnika przy\linebreak najwyższej potędze x jest dodatni.\\ W związku z tym wykres wielomianu zaczyna się od lewej strony poniżej osi OX. A więc $$x \in [15,16] \cup [17,\infty).$$
\rozwStop
\odpStart
$x \in [15,16] \cup [17,\infty)$
\odpStop
\testStart
A.$x \in [15,16] \cup [17,\infty)$\\
B.$x \in (15,16) \cup [17,\infty)$\\
C.$x \in (15,16] \cup [17,\infty)$\\
D.$x \in [15,16) \cup [17,\infty)$\\
E.$x \in [15,16] \cup (17,\infty)$\\
F.$x \in (15,16) \cup (17,\infty)$\\
G.$x \in [15,16) \cup (17,\infty)$\\
H.$x \in (15,16] \cup (17,\infty)$
\testStop
\kluczStart
A
\kluczStop



\zadStart{Zadanie z Wikieł Z 1.62 a) moja wersja nr 1122}

Rozwiązać nierówności $(x-15)(x-16)(x-18)\ge0$.
\zadStop
\rozwStart{Patryk Wirkus}{}
Miejsca zerowe naszego wielomianu to: $15, 16, 18$.\\
Wielomian jest stopnia nieparzystego, ponadto znak współczynnika przy\linebreak najwyższej potędze x jest dodatni.\\ W związku z tym wykres wielomianu zaczyna się od lewej strony poniżej osi OX. A więc $$x \in [15,16] \cup [18,\infty).$$
\rozwStop
\odpStart
$x \in [15,16] \cup [18,\infty)$
\odpStop
\testStart
A.$x \in [15,16] \cup [18,\infty)$\\
B.$x \in (15,16) \cup [18,\infty)$\\
C.$x \in (15,16] \cup [18,\infty)$\\
D.$x \in [15,16) \cup [18,\infty)$\\
E.$x \in [15,16] \cup (18,\infty)$\\
F.$x \in (15,16) \cup (18,\infty)$\\
G.$x \in [15,16) \cup (18,\infty)$\\
H.$x \in (15,16] \cup (18,\infty)$
\testStop
\kluczStart
A
\kluczStop



\zadStart{Zadanie z Wikieł Z 1.62 a) moja wersja nr 1123}

Rozwiązać nierówności $(x-15)(x-16)(x-19)\ge0$.
\zadStop
\rozwStart{Patryk Wirkus}{}
Miejsca zerowe naszego wielomianu to: $15, 16, 19$.\\
Wielomian jest stopnia nieparzystego, ponadto znak współczynnika przy\linebreak najwyższej potędze x jest dodatni.\\ W związku z tym wykres wielomianu zaczyna się od lewej strony poniżej osi OX. A więc $$x \in [15,16] \cup [19,\infty).$$
\rozwStop
\odpStart
$x \in [15,16] \cup [19,\infty)$
\odpStop
\testStart
A.$x \in [15,16] \cup [19,\infty)$\\
B.$x \in (15,16) \cup [19,\infty)$\\
C.$x \in (15,16] \cup [19,\infty)$\\
D.$x \in [15,16) \cup [19,\infty)$\\
E.$x \in [15,16] \cup (19,\infty)$\\
F.$x \in (15,16) \cup (19,\infty)$\\
G.$x \in [15,16) \cup (19,\infty)$\\
H.$x \in (15,16] \cup (19,\infty)$
\testStop
\kluczStart
A
\kluczStop



\zadStart{Zadanie z Wikieł Z 1.62 a) moja wersja nr 1124}

Rozwiązać nierówności $(x-15)(x-16)(x-20)\ge0$.
\zadStop
\rozwStart{Patryk Wirkus}{}
Miejsca zerowe naszego wielomianu to: $15, 16, 20$.\\
Wielomian jest stopnia nieparzystego, ponadto znak współczynnika przy\linebreak najwyższej potędze x jest dodatni.\\ W związku z tym wykres wielomianu zaczyna się od lewej strony poniżej osi OX. A więc $$x \in [15,16] \cup [20,\infty).$$
\rozwStop
\odpStart
$x \in [15,16] \cup [20,\infty)$
\odpStop
\testStart
A.$x \in [15,16] \cup [20,\infty)$\\
B.$x \in (15,16) \cup [20,\infty)$\\
C.$x \in (15,16] \cup [20,\infty)$\\
D.$x \in [15,16) \cup [20,\infty)$\\
E.$x \in [15,16] \cup (20,\infty)$\\
F.$x \in (15,16) \cup (20,\infty)$\\
G.$x \in [15,16) \cup (20,\infty)$\\
H.$x \in (15,16] \cup (20,\infty)$
\testStop
\kluczStart
A
\kluczStop



\zadStart{Zadanie z Wikieł Z 1.62 a) moja wersja nr 1125}

Rozwiązać nierówności $(x-15)(x-17)(x-18)\ge0$.
\zadStop
\rozwStart{Patryk Wirkus}{}
Miejsca zerowe naszego wielomianu to: $15, 17, 18$.\\
Wielomian jest stopnia nieparzystego, ponadto znak współczynnika przy\linebreak najwyższej potędze x jest dodatni.\\ W związku z tym wykres wielomianu zaczyna się od lewej strony poniżej osi OX. A więc $$x \in [15,17] \cup [18,\infty).$$
\rozwStop
\odpStart
$x \in [15,17] \cup [18,\infty)$
\odpStop
\testStart
A.$x \in [15,17] \cup [18,\infty)$\\
B.$x \in (15,17) \cup [18,\infty)$\\
C.$x \in (15,17] \cup [18,\infty)$\\
D.$x \in [15,17) \cup [18,\infty)$\\
E.$x \in [15,17] \cup (18,\infty)$\\
F.$x \in (15,17) \cup (18,\infty)$\\
G.$x \in [15,17) \cup (18,\infty)$\\
H.$x \in (15,17] \cup (18,\infty)$
\testStop
\kluczStart
A
\kluczStop



\zadStart{Zadanie z Wikieł Z 1.62 a) moja wersja nr 1126}

Rozwiązać nierówności $(x-15)(x-17)(x-19)\ge0$.
\zadStop
\rozwStart{Patryk Wirkus}{}
Miejsca zerowe naszego wielomianu to: $15, 17, 19$.\\
Wielomian jest stopnia nieparzystego, ponadto znak współczynnika przy\linebreak najwyższej potędze x jest dodatni.\\ W związku z tym wykres wielomianu zaczyna się od lewej strony poniżej osi OX. A więc $$x \in [15,17] \cup [19,\infty).$$
\rozwStop
\odpStart
$x \in [15,17] \cup [19,\infty)$
\odpStop
\testStart
A.$x \in [15,17] \cup [19,\infty)$\\
B.$x \in (15,17) \cup [19,\infty)$\\
C.$x \in (15,17] \cup [19,\infty)$\\
D.$x \in [15,17) \cup [19,\infty)$\\
E.$x \in [15,17] \cup (19,\infty)$\\
F.$x \in (15,17) \cup (19,\infty)$\\
G.$x \in [15,17) \cup (19,\infty)$\\
H.$x \in (15,17] \cup (19,\infty)$
\testStop
\kluczStart
A
\kluczStop



\zadStart{Zadanie z Wikieł Z 1.62 a) moja wersja nr 1127}

Rozwiązać nierówności $(x-15)(x-17)(x-20)\ge0$.
\zadStop
\rozwStart{Patryk Wirkus}{}
Miejsca zerowe naszego wielomianu to: $15, 17, 20$.\\
Wielomian jest stopnia nieparzystego, ponadto znak współczynnika przy\linebreak najwyższej potędze x jest dodatni.\\ W związku z tym wykres wielomianu zaczyna się od lewej strony poniżej osi OX. A więc $$x \in [15,17] \cup [20,\infty).$$
\rozwStop
\odpStart
$x \in [15,17] \cup [20,\infty)$
\odpStop
\testStart
A.$x \in [15,17] \cup [20,\infty)$\\
B.$x \in (15,17) \cup [20,\infty)$\\
C.$x \in (15,17] \cup [20,\infty)$\\
D.$x \in [15,17) \cup [20,\infty)$\\
E.$x \in [15,17] \cup (20,\infty)$\\
F.$x \in (15,17) \cup (20,\infty)$\\
G.$x \in [15,17) \cup (20,\infty)$\\
H.$x \in (15,17] \cup (20,\infty)$
\testStop
\kluczStart
A
\kluczStop



\zadStart{Zadanie z Wikieł Z 1.62 a) moja wersja nr 1128}

Rozwiązać nierówności $(x-15)(x-18)(x-19)\ge0$.
\zadStop
\rozwStart{Patryk Wirkus}{}
Miejsca zerowe naszego wielomianu to: $15, 18, 19$.\\
Wielomian jest stopnia nieparzystego, ponadto znak współczynnika przy\linebreak najwyższej potędze x jest dodatni.\\ W związku z tym wykres wielomianu zaczyna się od lewej strony poniżej osi OX. A więc $$x \in [15,18] \cup [19,\infty).$$
\rozwStop
\odpStart
$x \in [15,18] \cup [19,\infty)$
\odpStop
\testStart
A.$x \in [15,18] \cup [19,\infty)$\\
B.$x \in (15,18) \cup [19,\infty)$\\
C.$x \in (15,18] \cup [19,\infty)$\\
D.$x \in [15,18) \cup [19,\infty)$\\
E.$x \in [15,18] \cup (19,\infty)$\\
F.$x \in (15,18) \cup (19,\infty)$\\
G.$x \in [15,18) \cup (19,\infty)$\\
H.$x \in (15,18] \cup (19,\infty)$
\testStop
\kluczStart
A
\kluczStop



\zadStart{Zadanie z Wikieł Z 1.62 a) moja wersja nr 1129}

Rozwiązać nierówności $(x-15)(x-18)(x-20)\ge0$.
\zadStop
\rozwStart{Patryk Wirkus}{}
Miejsca zerowe naszego wielomianu to: $15, 18, 20$.\\
Wielomian jest stopnia nieparzystego, ponadto znak współczynnika przy\linebreak najwyższej potędze x jest dodatni.\\ W związku z tym wykres wielomianu zaczyna się od lewej strony poniżej osi OX. A więc $$x \in [15,18] \cup [20,\infty).$$
\rozwStop
\odpStart
$x \in [15,18] \cup [20,\infty)$
\odpStop
\testStart
A.$x \in [15,18] \cup [20,\infty)$\\
B.$x \in (15,18) \cup [20,\infty)$\\
C.$x \in (15,18] \cup [20,\infty)$\\
D.$x \in [15,18) \cup [20,\infty)$\\
E.$x \in [15,18] \cup (20,\infty)$\\
F.$x \in (15,18) \cup (20,\infty)$\\
G.$x \in [15,18) \cup (20,\infty)$\\
H.$x \in (15,18] \cup (20,\infty)$
\testStop
\kluczStart
A
\kluczStop



\zadStart{Zadanie z Wikieł Z 1.62 a) moja wersja nr 1130}

Rozwiązać nierówności $(x-15)(x-19)(x-20)\ge0$.
\zadStop
\rozwStart{Patryk Wirkus}{}
Miejsca zerowe naszego wielomianu to: $15, 19, 20$.\\
Wielomian jest stopnia nieparzystego, ponadto znak współczynnika przy\linebreak najwyższej potędze x jest dodatni.\\ W związku z tym wykres wielomianu zaczyna się od lewej strony poniżej osi OX. A więc $$x \in [15,19] \cup [20,\infty).$$
\rozwStop
\odpStart
$x \in [15,19] \cup [20,\infty)$
\odpStop
\testStart
A.$x \in [15,19] \cup [20,\infty)$\\
B.$x \in (15,19) \cup [20,\infty)$\\
C.$x \in (15,19] \cup [20,\infty)$\\
D.$x \in [15,19) \cup [20,\infty)$\\
E.$x \in [15,19] \cup (20,\infty)$\\
F.$x \in (15,19) \cup (20,\infty)$\\
G.$x \in [15,19) \cup (20,\infty)$\\
H.$x \in (15,19] \cup (20,\infty)$
\testStop
\kluczStart
A
\kluczStop



\zadStart{Zadanie z Wikieł Z 1.62 a) moja wersja nr 1131}

Rozwiązać nierówności $(x-16)(x-17)(x-18)\ge0$.
\zadStop
\rozwStart{Patryk Wirkus}{}
Miejsca zerowe naszego wielomianu to: $16, 17, 18$.\\
Wielomian jest stopnia nieparzystego, ponadto znak współczynnika przy\linebreak najwyższej potędze x jest dodatni.\\ W związku z tym wykres wielomianu zaczyna się od lewej strony poniżej osi OX. A więc $$x \in [16,17] \cup [18,\infty).$$
\rozwStop
\odpStart
$x \in [16,17] \cup [18,\infty)$
\odpStop
\testStart
A.$x \in [16,17] \cup [18,\infty)$\\
B.$x \in (16,17) \cup [18,\infty)$\\
C.$x \in (16,17] \cup [18,\infty)$\\
D.$x \in [16,17) \cup [18,\infty)$\\
E.$x \in [16,17] \cup (18,\infty)$\\
F.$x \in (16,17) \cup (18,\infty)$\\
G.$x \in [16,17) \cup (18,\infty)$\\
H.$x \in (16,17] \cup (18,\infty)$
\testStop
\kluczStart
A
\kluczStop



\zadStart{Zadanie z Wikieł Z 1.62 a) moja wersja nr 1132}

Rozwiązać nierówności $(x-16)(x-17)(x-19)\ge0$.
\zadStop
\rozwStart{Patryk Wirkus}{}
Miejsca zerowe naszego wielomianu to: $16, 17, 19$.\\
Wielomian jest stopnia nieparzystego, ponadto znak współczynnika przy\linebreak najwyższej potędze x jest dodatni.\\ W związku z tym wykres wielomianu zaczyna się od lewej strony poniżej osi OX. A więc $$x \in [16,17] \cup [19,\infty).$$
\rozwStop
\odpStart
$x \in [16,17] \cup [19,\infty)$
\odpStop
\testStart
A.$x \in [16,17] \cup [19,\infty)$\\
B.$x \in (16,17) \cup [19,\infty)$\\
C.$x \in (16,17] \cup [19,\infty)$\\
D.$x \in [16,17) \cup [19,\infty)$\\
E.$x \in [16,17] \cup (19,\infty)$\\
F.$x \in (16,17) \cup (19,\infty)$\\
G.$x \in [16,17) \cup (19,\infty)$\\
H.$x \in (16,17] \cup (19,\infty)$
\testStop
\kluczStart
A
\kluczStop



\zadStart{Zadanie z Wikieł Z 1.62 a) moja wersja nr 1133}

Rozwiązać nierówności $(x-16)(x-17)(x-20)\ge0$.
\zadStop
\rozwStart{Patryk Wirkus}{}
Miejsca zerowe naszego wielomianu to: $16, 17, 20$.\\
Wielomian jest stopnia nieparzystego, ponadto znak współczynnika przy\linebreak najwyższej potędze x jest dodatni.\\ W związku z tym wykres wielomianu zaczyna się od lewej strony poniżej osi OX. A więc $$x \in [16,17] \cup [20,\infty).$$
\rozwStop
\odpStart
$x \in [16,17] \cup [20,\infty)$
\odpStop
\testStart
A.$x \in [16,17] \cup [20,\infty)$\\
B.$x \in (16,17) \cup [20,\infty)$\\
C.$x \in (16,17] \cup [20,\infty)$\\
D.$x \in [16,17) \cup [20,\infty)$\\
E.$x \in [16,17] \cup (20,\infty)$\\
F.$x \in (16,17) \cup (20,\infty)$\\
G.$x \in [16,17) \cup (20,\infty)$\\
H.$x \in (16,17] \cup (20,\infty)$
\testStop
\kluczStart
A
\kluczStop



\zadStart{Zadanie z Wikieł Z 1.62 a) moja wersja nr 1134}

Rozwiązać nierówności $(x-16)(x-18)(x-19)\ge0$.
\zadStop
\rozwStart{Patryk Wirkus}{}
Miejsca zerowe naszego wielomianu to: $16, 18, 19$.\\
Wielomian jest stopnia nieparzystego, ponadto znak współczynnika przy\linebreak najwyższej potędze x jest dodatni.\\ W związku z tym wykres wielomianu zaczyna się od lewej strony poniżej osi OX. A więc $$x \in [16,18] \cup [19,\infty).$$
\rozwStop
\odpStart
$x \in [16,18] \cup [19,\infty)$
\odpStop
\testStart
A.$x \in [16,18] \cup [19,\infty)$\\
B.$x \in (16,18) \cup [19,\infty)$\\
C.$x \in (16,18] \cup [19,\infty)$\\
D.$x \in [16,18) \cup [19,\infty)$\\
E.$x \in [16,18] \cup (19,\infty)$\\
F.$x \in (16,18) \cup (19,\infty)$\\
G.$x \in [16,18) \cup (19,\infty)$\\
H.$x \in (16,18] \cup (19,\infty)$
\testStop
\kluczStart
A
\kluczStop



\zadStart{Zadanie z Wikieł Z 1.62 a) moja wersja nr 1135}

Rozwiązać nierówności $(x-16)(x-18)(x-20)\ge0$.
\zadStop
\rozwStart{Patryk Wirkus}{}
Miejsca zerowe naszego wielomianu to: $16, 18, 20$.\\
Wielomian jest stopnia nieparzystego, ponadto znak współczynnika przy\linebreak najwyższej potędze x jest dodatni.\\ W związku z tym wykres wielomianu zaczyna się od lewej strony poniżej osi OX. A więc $$x \in [16,18] \cup [20,\infty).$$
\rozwStop
\odpStart
$x \in [16,18] \cup [20,\infty)$
\odpStop
\testStart
A.$x \in [16,18] \cup [20,\infty)$\\
B.$x \in (16,18) \cup [20,\infty)$\\
C.$x \in (16,18] \cup [20,\infty)$\\
D.$x \in [16,18) \cup [20,\infty)$\\
E.$x \in [16,18] \cup (20,\infty)$\\
F.$x \in (16,18) \cup (20,\infty)$\\
G.$x \in [16,18) \cup (20,\infty)$\\
H.$x \in (16,18] \cup (20,\infty)$
\testStop
\kluczStart
A
\kluczStop



\zadStart{Zadanie z Wikieł Z 1.62 a) moja wersja nr 1136}

Rozwiązać nierówności $(x-16)(x-19)(x-20)\ge0$.
\zadStop
\rozwStart{Patryk Wirkus}{}
Miejsca zerowe naszego wielomianu to: $16, 19, 20$.\\
Wielomian jest stopnia nieparzystego, ponadto znak współczynnika przy\linebreak najwyższej potędze x jest dodatni.\\ W związku z tym wykres wielomianu zaczyna się od lewej strony poniżej osi OX. A więc $$x \in [16,19] \cup [20,\infty).$$
\rozwStop
\odpStart
$x \in [16,19] \cup [20,\infty)$
\odpStop
\testStart
A.$x \in [16,19] \cup [20,\infty)$\\
B.$x \in (16,19) \cup [20,\infty)$\\
C.$x \in (16,19] \cup [20,\infty)$\\
D.$x \in [16,19) \cup [20,\infty)$\\
E.$x \in [16,19] \cup (20,\infty)$\\
F.$x \in (16,19) \cup (20,\infty)$\\
G.$x \in [16,19) \cup (20,\infty)$\\
H.$x \in (16,19] \cup (20,\infty)$
\testStop
\kluczStart
A
\kluczStop



\zadStart{Zadanie z Wikieł Z 1.62 a) moja wersja nr 1137}

Rozwiązać nierówności $(x-17)(x-18)(x-19)\ge0$.
\zadStop
\rozwStart{Patryk Wirkus}{}
Miejsca zerowe naszego wielomianu to: $17, 18, 19$.\\
Wielomian jest stopnia nieparzystego, ponadto znak współczynnika przy\linebreak najwyższej potędze x jest dodatni.\\ W związku z tym wykres wielomianu zaczyna się od lewej strony poniżej osi OX. A więc $$x \in [17,18] \cup [19,\infty).$$
\rozwStop
\odpStart
$x \in [17,18] \cup [19,\infty)$
\odpStop
\testStart
A.$x \in [17,18] \cup [19,\infty)$\\
B.$x \in (17,18) \cup [19,\infty)$\\
C.$x \in (17,18] \cup [19,\infty)$\\
D.$x \in [17,18) \cup [19,\infty)$\\
E.$x \in [17,18] \cup (19,\infty)$\\
F.$x \in (17,18) \cup (19,\infty)$\\
G.$x \in [17,18) \cup (19,\infty)$\\
H.$x \in (17,18] \cup (19,\infty)$
\testStop
\kluczStart
A
\kluczStop



\zadStart{Zadanie z Wikieł Z 1.62 a) moja wersja nr 1138}

Rozwiązać nierówności $(x-17)(x-18)(x-20)\ge0$.
\zadStop
\rozwStart{Patryk Wirkus}{}
Miejsca zerowe naszego wielomianu to: $17, 18, 20$.\\
Wielomian jest stopnia nieparzystego, ponadto znak współczynnika przy\linebreak najwyższej potędze x jest dodatni.\\ W związku z tym wykres wielomianu zaczyna się od lewej strony poniżej osi OX. A więc $$x \in [17,18] \cup [20,\infty).$$
\rozwStop
\odpStart
$x \in [17,18] \cup [20,\infty)$
\odpStop
\testStart
A.$x \in [17,18] \cup [20,\infty)$\\
B.$x \in (17,18) \cup [20,\infty)$\\
C.$x \in (17,18] \cup [20,\infty)$\\
D.$x \in [17,18) \cup [20,\infty)$\\
E.$x \in [17,18] \cup (20,\infty)$\\
F.$x \in (17,18) \cup (20,\infty)$\\
G.$x \in [17,18) \cup (20,\infty)$\\
H.$x \in (17,18] \cup (20,\infty)$
\testStop
\kluczStart
A
\kluczStop



\zadStart{Zadanie z Wikieł Z 1.62 a) moja wersja nr 1139}

Rozwiązać nierówności $(x-17)(x-19)(x-20)\ge0$.
\zadStop
\rozwStart{Patryk Wirkus}{}
Miejsca zerowe naszego wielomianu to: $17, 19, 20$.\\
Wielomian jest stopnia nieparzystego, ponadto znak współczynnika przy\linebreak najwyższej potędze x jest dodatni.\\ W związku z tym wykres wielomianu zaczyna się od lewej strony poniżej osi OX. A więc $$x \in [17,19] \cup [20,\infty).$$
\rozwStop
\odpStart
$x \in [17,19] \cup [20,\infty)$
\odpStop
\testStart
A.$x \in [17,19] \cup [20,\infty)$\\
B.$x \in (17,19) \cup [20,\infty)$\\
C.$x \in (17,19] \cup [20,\infty)$\\
D.$x \in [17,19) \cup [20,\infty)$\\
E.$x \in [17,19] \cup (20,\infty)$\\
F.$x \in (17,19) \cup (20,\infty)$\\
G.$x \in [17,19) \cup (20,\infty)$\\
H.$x \in (17,19] \cup (20,\infty)$
\testStop
\kluczStart
A
\kluczStop



\zadStart{Zadanie z Wikieł Z 1.62 a) moja wersja nr 1140}

Rozwiązać nierówności $(x-18)(x-19)(x-20)\ge0$.
\zadStop
\rozwStart{Patryk Wirkus}{}
Miejsca zerowe naszego wielomianu to: $18, 19, 20$.\\
Wielomian jest stopnia nieparzystego, ponadto znak współczynnika przy\linebreak najwyższej potędze x jest dodatni.\\ W związku z tym wykres wielomianu zaczyna się od lewej strony poniżej osi OX. A więc $$x \in [18,19] \cup [20,\infty).$$
\rozwStop
\odpStart
$x \in [18,19] \cup [20,\infty)$
\odpStop
\testStart
A.$x \in [18,19] \cup [20,\infty)$\\
B.$x \in (18,19) \cup [20,\infty)$\\
C.$x \in (18,19] \cup [20,\infty)$\\
D.$x \in [18,19) \cup [20,\infty)$\\
E.$x \in [18,19] \cup (20,\infty)$\\
F.$x \in (18,19) \cup (20,\infty)$\\
G.$x \in [18,19) \cup (20,\infty)$\\
H.$x \in (18,19] \cup (20,\infty)$
\testStop
\kluczStart
A
\kluczStop





\end{document}
