\documentclass[12pt, a4paper]{article}
\usepackage[utf8]{inputenc}
\usepackage{polski}
\usepackage{amsthm}  %pakiet do tworzenia twierdzeń itp.
\usepackage{amsmath} %pakiet do niektórych symboli matematycznych
\usepackage{amssymb} %pakiet do symboli mat., np. \nsubseteq
\usepackage{amsfonts}
\usepackage{graphicx} %obsługa plików graficznych z rozszerzeniem png, jpg
\theoremstyle{definition} %styl dla definicji
\newtheorem{zad}{} 
\title{Multizestaw zadań}
\author{Patryk Wirkus}
%\date{\today}
\date{}
\newcommand{\kategoria}[1]{\section{#1}}
\newcommand{\zadStart}[1]{\begin{zad}#1\newline}
\newcommand{\zadStop}{\end{zad}}
\newcommand{\rozwStart}[2]{\noindent \textbf{Rozwiązanie (autor #1 , recenzent #2): }\newline}
\newcommand{\rozwStop}{\newline}                                           
\newcommand{\odpStart}{\noindent \textbf{Odpowiedź:}\newline}
\newcommand{\odpStop}{\newline}
\newcommand{\testStart}{\noindent \textbf{Test:}\newline}
\newcommand{\testStop}{\newline}
\newcommand{\kluczStart}{\noindent \textbf{Test poprawna odpowiedź:}\newline}
\newcommand{\kluczStop}{\newline}
\newcommand{\wstawGrafike}[2]{\begin{figure}[h] \includegraphics[scale=#2] {#1} \end{figure}}

\begin{document}
\maketitle

\kategoria{Wikieł/Z3.18a}


\zadStart{Zadanie z Wikieł Z 3.18 a) moja wersja nr 1}

Zamień poniższe ułamki dziesiętne okresowe na ułamki zwykłe $3,(12)$.
\zadStop
\rozwStart{Patryk Wirkus}{Martyna Czarnobaj}
$$3,(12)=3,121212=3,12+(0,0012+0,000012+...)=3,12+\frac{0,0012}{1-0,01}$$
$$=3.12+\frac{12}{9900}=\frac{312\cdot99+12}{9900}$$
\rozwStop
\odpStart
$\frac{312\cdot99+12}{9900}$
\odpStop
\testStart
A.$\frac{312\cdot99+12}{9900}$\\ B.$-\frac{312\cdot99+12}{9900}$\\ C.$3,12$\\ D.$\frac{312\cdot100}{9900}$
\testStop
\kluczStart
A
\kluczStop



\zadStart{Zadanie z Wikieł Z 3.18 a) moja wersja nr 2}

Zamień poniższe ułamki dziesiętne okresowe na ułamki zwykłe $3,(13)$.
\zadStop
\rozwStart{Patryk Wirkus}{Martyna Czarnobaj}
$$3,(13)=3,131313=3,13+(0,0013+0,000013+...)=3,13+\frac{0,0013}{1-0,01}$$
$$=3.13+\frac{13}{9900}=\frac{313\cdot99+13}{9900}$$
\rozwStop
\odpStart
$\frac{313\cdot99+13}{9900}$
\odpStop
\testStart
A.$\frac{313\cdot99+13}{9900}$\\ B.$-\frac{313\cdot99+13}{9900}$\\ C.$3,13$\\ D.$\frac{313\cdot100}{9900}$
\testStop
\kluczStart
A
\kluczStop



\zadStart{Zadanie z Wikieł Z 3.18 a) moja wersja nr 3}

Zamień poniższe ułamki dziesiętne okresowe na ułamki zwykłe $3,(14)$.
\zadStop
\rozwStart{Patryk Wirkus}{Martyna Czarnobaj}
$$3,(14)=3,141414=3,14+(0,0014+0,000014+...)=3,14+\frac{0,0014}{1-0,01}$$
$$=3.14+\frac{14}{9900}=\frac{314\cdot99+14}{9900}$$
\rozwStop
\odpStart
$\frac{314\cdot99+14}{9900}$
\odpStop
\testStart
A.$\frac{314\cdot99+14}{9900}$\\ B.$-\frac{314\cdot99+14}{9900}$\\ C.$3,14$\\ D.$\frac{314\cdot100}{9900}$
\testStop
\kluczStart
A
\kluczStop



\zadStart{Zadanie z Wikieł Z 3.18 a) moja wersja nr 4}

Zamień poniższe ułamki dziesiętne okresowe na ułamki zwykłe $3,(15)$.
\zadStop
\rozwStart{Patryk Wirkus}{Martyna Czarnobaj}
$$3,(15)=3,151515=3,15+(0,0015+0,000015+...)=3,15+\frac{0,0015}{1-0,01}$$
$$=3.15+\frac{15}{9900}=\frac{315\cdot99+15}{9900}$$
\rozwStop
\odpStart
$\frac{315\cdot99+15}{9900}$
\odpStop
\testStart
A.$\frac{315\cdot99+15}{9900}$\\ B.$-\frac{315\cdot99+15}{9900}$\\ C.$3,15$\\ D.$\frac{315\cdot100}{9900}$
\testStop
\kluczStart
A
\kluczStop



\zadStart{Zadanie z Wikieł Z 3.18 a) moja wersja nr 5}

Zamień poniższe ułamki dziesiętne okresowe na ułamki zwykłe $3,(16)$.
\zadStop
\rozwStart{Patryk Wirkus}{Martyna Czarnobaj}
$$3,(16)=3,161616=3,16+(0,0016+0,000016+...)=3,16+\frac{0,0016}{1-0,01}$$
$$=3.16+\frac{16}{9900}=\frac{316\cdot99+16}{9900}$$
\rozwStop
\odpStart
$\frac{316\cdot99+16}{9900}$
\odpStop
\testStart
A.$\frac{316\cdot99+16}{9900}$\\ B.$-\frac{316\cdot99+16}{9900}$\\ C.$3,16$\\ D.$\frac{316\cdot100}{9900}$
\testStop
\kluczStart
A
\kluczStop



\zadStart{Zadanie z Wikieł Z 3.18 a) moja wersja nr 6}

Zamień poniższe ułamki dziesiętne okresowe na ułamki zwykłe $3,(17)$.
\zadStop
\rozwStart{Patryk Wirkus}{Martyna Czarnobaj}
$$3,(17)=3,171717=3,17+(0,0017+0,000017+...)=3,17+\frac{0,0017}{1-0,01}$$
$$=3.17+\frac{17}{9900}=\frac{317\cdot99+17}{9900}$$
\rozwStop
\odpStart
$\frac{317\cdot99+17}{9900}$
\odpStop
\testStart
A.$\frac{317\cdot99+17}{9900}$\\ B.$-\frac{317\cdot99+17}{9900}$\\ C.$3,17$\\ D.$\frac{317\cdot100}{9900}$
\testStop
\kluczStart
A
\kluczStop



\zadStart{Zadanie z Wikieł Z 3.18 a) moja wersja nr 7}

Zamień poniższe ułamki dziesiętne okresowe na ułamki zwykłe $3,(18)$.
\zadStop
\rozwStart{Patryk Wirkus}{Martyna Czarnobaj}
$$3,(18)=3,181818=3,18+(0,0018+0,000018+...)=3,18+\frac{0,0018}{1-0,01}$$
$$=3.18+\frac{18}{9900}=\frac{318\cdot99+18}{9900}$$
\rozwStop
\odpStart
$\frac{318\cdot99+18}{9900}$
\odpStop
\testStart
A.$\frac{318\cdot99+18}{9900}$\\ B.$-\frac{318\cdot99+18}{9900}$\\ C.$3,18$\\ D.$\frac{318\cdot100}{9900}$
\testStop
\kluczStart
A
\kluczStop



\zadStart{Zadanie z Wikieł Z 3.18 a) moja wersja nr 8}

Zamień poniższe ułamki dziesiętne okresowe na ułamki zwykłe $3,(19)$.
\zadStop
\rozwStart{Patryk Wirkus}{Martyna Czarnobaj}
$$3,(19)=3,191919=3,19+(0,0019+0,000019+...)=3,19+\frac{0,0019}{1-0,01}$$
$$=3.19+\frac{19}{9900}=\frac{319\cdot99+19}{9900}$$
\rozwStop
\odpStart
$\frac{319\cdot99+19}{9900}$
\odpStop
\testStart
A.$\frac{319\cdot99+19}{9900}$\\ B.$-\frac{319\cdot99+19}{9900}$\\ C.$3,19$\\ D.$\frac{319\cdot100}{9900}$
\testStop
\kluczStart
A
\kluczStop



\zadStart{Zadanie z Wikieł Z 3.18 a) moja wersja nr 9}

Zamień poniższe ułamki dziesiętne okresowe na ułamki zwykłe $3,(34)$.
\zadStop
\rozwStart{Patryk Wirkus}{Martyna Czarnobaj}
$$3,(34)=3,343434=3,34+(0,0034+0,000034+...)=3,34+\frac{0,0034}{1-0,01}$$
$$=3.34+\frac{34}{9900}=\frac{334\cdot99+34}{9900}$$
\rozwStop
\odpStart
$\frac{334\cdot99+34}{9900}$
\odpStop
\testStart
A.$\frac{334\cdot99+34}{9900}$\\ B.$-\frac{334\cdot99+34}{9900}$\\ C.$3,34$\\ D.$\frac{334\cdot100}{9900}$
\testStop
\kluczStart
A
\kluczStop



\zadStart{Zadanie z Wikieł Z 3.18 a) moja wersja nr 10}

Zamień poniższe ułamki dziesiętne okresowe na ułamki zwykłe $3,(35)$.
\zadStop
\rozwStart{Patryk Wirkus}{Martyna Czarnobaj}
$$3,(35)=3,353535=3,35+(0,0035+0,000035+...)=3,35+\frac{0,0035}{1-0,01}$$
$$=3.35+\frac{35}{9900}=\frac{335\cdot99+35}{9900}$$
\rozwStop
\odpStart
$\frac{335\cdot99+35}{9900}$
\odpStop
\testStart
A.$\frac{335\cdot99+35}{9900}$\\ B.$-\frac{335\cdot99+35}{9900}$\\ C.$3,35$\\ D.$\frac{335\cdot100}{9900}$
\testStop
\kluczStart
A
\kluczStop



\zadStart{Zadanie z Wikieł Z 3.18 a) moja wersja nr 11}

Zamień poniższe ułamki dziesiętne okresowe na ułamki zwykłe $3,(36)$.
\zadStop
\rozwStart{Patryk Wirkus}{Martyna Czarnobaj}
$$3,(36)=3,363636=3,36+(0,0036+0,000036+...)=3,36+\frac{0,0036}{1-0,01}$$
$$=3.36+\frac{36}{9900}=\frac{336\cdot99+36}{9900}$$
\rozwStop
\odpStart
$\frac{336\cdot99+36}{9900}$
\odpStop
\testStart
A.$\frac{336\cdot99+36}{9900}$\\ B.$-\frac{336\cdot99+36}{9900}$\\ C.$3,36$\\ D.$\frac{336\cdot100}{9900}$
\testStop
\kluczStart
A
\kluczStop



\zadStart{Zadanie z Wikieł Z 3.18 a) moja wersja nr 12}

Zamień poniższe ułamki dziesiętne okresowe na ułamki zwykłe $3,(37)$.
\zadStop
\rozwStart{Patryk Wirkus}{Martyna Czarnobaj}
$$3,(37)=3,373737=3,37+(0,0037+0,000037+...)=3,37+\frac{0,0037}{1-0,01}$$
$$=3.37+\frac{37}{9900}=\frac{337\cdot99+37}{9900}$$
\rozwStop
\odpStart
$\frac{337\cdot99+37}{9900}$
\odpStop
\testStart
A.$\frac{337\cdot99+37}{9900}$\\ B.$-\frac{337\cdot99+37}{9900}$\\ C.$3,37$\\ D.$\frac{337\cdot100}{9900}$
\testStop
\kluczStart
A
\kluczStop



\zadStart{Zadanie z Wikieł Z 3.18 a) moja wersja nr 13}

Zamień poniższe ułamki dziesiętne okresowe na ułamki zwykłe $3,(38)$.
\zadStop
\rozwStart{Patryk Wirkus}{Martyna Czarnobaj}
$$3,(38)=3,383838=3,38+(0,0038+0,000038+...)=3,38+\frac{0,0038}{1-0,01}$$
$$=3.38+\frac{38}{9900}=\frac{338\cdot99+38}{9900}$$
\rozwStop
\odpStart
$\frac{338\cdot99+38}{9900}$
\odpStop
\testStart
A.$\frac{338\cdot99+38}{9900}$\\ B.$-\frac{338\cdot99+38}{9900}$\\ C.$3,38$\\ D.$\frac{338\cdot100}{9900}$
\testStop
\kluczStart
A
\kluczStop



\zadStart{Zadanie z Wikieł Z 3.18 a) moja wersja nr 14}

Zamień poniższe ułamki dziesiętne okresowe na ułamki zwykłe $3,(39)$.
\zadStop
\rozwStart{Patryk Wirkus}{Martyna Czarnobaj}
$$3,(39)=3,393939=3,39+(0,0039+0,000039+...)=3,39+\frac{0,0039}{1-0,01}$$
$$=3.39+\frac{39}{9900}=\frac{339\cdot99+39}{9900}$$
\rozwStop
\odpStart
$\frac{339\cdot99+39}{9900}$
\odpStop
\testStart
A.$\frac{339\cdot99+39}{9900}$\\ B.$-\frac{339\cdot99+39}{9900}$\\ C.$3,39$\\ D.$\frac{339\cdot100}{9900}$
\testStop
\kluczStart
A
\kluczStop



\zadStart{Zadanie z Wikieł Z 3.18 a) moja wersja nr 15}

Zamień poniższe ułamki dziesiętne okresowe na ułamki zwykłe $3,(56)$.
\zadStop
\rozwStart{Patryk Wirkus}{Martyna Czarnobaj}
$$3,(56)=3,565656=3,56+(0,0056+0,000056+...)=3,56+\frac{0,0056}{1-0,01}$$
$$=3.56+\frac{56}{9900}=\frac{356\cdot99+56}{9900}$$
\rozwStop
\odpStart
$\frac{356\cdot99+56}{9900}$
\odpStop
\testStart
A.$\frac{356\cdot99+56}{9900}$\\ B.$-\frac{356\cdot99+56}{9900}$\\ C.$3,56$\\ D.$\frac{356\cdot100}{9900}$
\testStop
\kluczStart
A
\kluczStop



\zadStart{Zadanie z Wikieł Z 3.18 a) moja wersja nr 16}

Zamień poniższe ułamki dziesiętne okresowe na ułamki zwykłe $3,(57)$.
\zadStop
\rozwStart{Patryk Wirkus}{Martyna Czarnobaj}
$$3,(57)=3,575757=3,57+(0,0057+0,000057+...)=3,57+\frac{0,0057}{1-0,01}$$
$$=3.57+\frac{57}{9900}=\frac{357\cdot99+57}{9900}$$
\rozwStop
\odpStart
$\frac{357\cdot99+57}{9900}$
\odpStop
\testStart
A.$\frac{357\cdot99+57}{9900}$\\ B.$-\frac{357\cdot99+57}{9900}$\\ C.$3,57$\\ D.$\frac{357\cdot100}{9900}$
\testStop
\kluczStart
A
\kluczStop



\zadStart{Zadanie z Wikieł Z 3.18 a) moja wersja nr 17}

Zamień poniższe ułamki dziesiętne okresowe na ułamki zwykłe $3,(58)$.
\zadStop
\rozwStart{Patryk Wirkus}{Martyna Czarnobaj}
$$3,(58)=3,585858=3,58+(0,0058+0,000058+...)=3,58+\frac{0,0058}{1-0,01}$$
$$=3.58+\frac{58}{9900}=\frac{358\cdot99+58}{9900}$$
\rozwStop
\odpStart
$\frac{358\cdot99+58}{9900}$
\odpStop
\testStart
A.$\frac{358\cdot99+58}{9900}$\\ B.$-\frac{358\cdot99+58}{9900}$\\ C.$3,58$\\ D.$\frac{358\cdot100}{9900}$
\testStop
\kluczStart
A
\kluczStop



\zadStart{Zadanie z Wikieł Z 3.18 a) moja wersja nr 18}

Zamień poniższe ułamki dziesiętne okresowe na ułamki zwykłe $3,(59)$.
\zadStop
\rozwStart{Patryk Wirkus}{Martyna Czarnobaj}
$$3,(59)=3,595959=3,59+(0,0059+0,000059+...)=3,59+\frac{0,0059}{1-0,01}$$
$$=3.59+\frac{59}{9900}=\frac{359\cdot99+59}{9900}$$
\rozwStop
\odpStart
$\frac{359\cdot99+59}{9900}$
\odpStop
\testStart
A.$\frac{359\cdot99+59}{9900}$\\ B.$-\frac{359\cdot99+59}{9900}$\\ C.$3,59$\\ D.$\frac{359\cdot100}{9900}$
\testStop
\kluczStart
A
\kluczStop



\zadStart{Zadanie z Wikieł Z 3.18 a) moja wersja nr 19}

Zamień poniższe ułamki dziesiętne okresowe na ułamki zwykłe $3,(78)$.
\zadStop
\rozwStart{Patryk Wirkus}{Martyna Czarnobaj}
$$3,(78)=3,787878=3,78+(0,0078+0,000078+...)=3,78+\frac{0,0078}{1-0,01}$$
$$=3.78+\frac{78}{9900}=\frac{378\cdot99+78}{9900}$$
\rozwStop
\odpStart
$\frac{378\cdot99+78}{9900}$
\odpStop
\testStart
A.$\frac{378\cdot99+78}{9900}$\\ B.$-\frac{378\cdot99+78}{9900}$\\ C.$3,78$\\ D.$\frac{378\cdot100}{9900}$
\testStop
\kluczStart
A
\kluczStop



\zadStart{Zadanie z Wikieł Z 3.18 a) moja wersja nr 20}

Zamień poniższe ułamki dziesiętne okresowe na ułamki zwykłe $3,(79)$.
\zadStop
\rozwStart{Patryk Wirkus}{Martyna Czarnobaj}
$$3,(79)=3,797979=3,79+(0,0079+0,000079+...)=3,79+\frac{0,0079}{1-0,01}$$
$$=3.79+\frac{79}{9900}=\frac{379\cdot99+79}{9900}$$
\rozwStop
\odpStart
$\frac{379\cdot99+79}{9900}$
\odpStop
\testStart
A.$\frac{379\cdot99+79}{9900}$\\ B.$-\frac{379\cdot99+79}{9900}$\\ C.$3,79$\\ D.$\frac{379\cdot100}{9900}$
\testStop
\kluczStart
A
\kluczStop





\end{document}
