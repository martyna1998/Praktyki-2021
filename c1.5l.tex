\documentclass[12pt, a4paper]{article}
\usepackage[utf8]{inputenc}
\usepackage{polski}

\usepackage{amsthm}  %pakiet do tworzenia twierdzeń itp.
\usepackage{amsmath} %pakiet do niektórych symboli matematycznych
\usepackage{amssymb} %pakiet do symboli mat., np. \nsubseteq
\usepackage{amsfonts}
\usepackage{graphicx} %obsługa plików graficznych z rozszerzeniem png, jpg
\theoremstyle{definition} %styl dla definicji
\newtheorem{zad}{} 
\title{Multizestaw zadań}
\author{Robert Fidytek}
%\date{\today}
\date{}
\newcounter{liczniksekcji}
\newcommand{\kategoria}[1]{\section{#1}} %olreślamy nazwę kateforii zadań
\newcommand{\zadStart}[1]{\begin{zad}#1\newline} %oznaczenie początku zadania
\newcommand{\zadStop}{\end{zad}}   %oznaczenie końca zadania
%Makra opcjonarne (nie muszą występować):
\newcommand{\rozwStart}[2]{\noindent \textbf{Rozwiązanie (autor #1 , recenzent #2): }\newline} %oznaczenie początku rozwiązania, opcjonarnie można wprowadzić informację o autorze rozwiązania zadania i recenzencie poprawności wykonania rozwiązania zadania
\newcommand{\rozwStop}{\newline}                                            %oznaczenie końca rozwiązania
\newcommand{\odpStart}{\noindent \textbf{Odpowiedź:}\newline}    %oznaczenie początku odpowiedzi końcowej (wypisanie wyniku)
\newcommand{\odpStop}{\newline}                                             %oznaczenie końca odpowiedzi końcowej (wypisanie wyniku)
\newcommand{\testStart}{\noindent \textbf{Test:}\newline} %ewentualne możliwe opcje odpowiedzi testowej: A. ? B. ? C. ? D. ? itd.
\newcommand{\testStop}{\newline} %koniec wprowadzania odpowiedzi testowych
\newcommand{\kluczStart}{\noindent \textbf{Test poprawna odpowiedź:}\newline} %klucz, poprawna odpowiedź pytania testowego (jedna literka): A lub B lub C lub D itd.
\newcommand{\kluczStop}{\newline} %koniec poprawnej odpowiedzi pytania testowego 
\newcommand{\wstawGrafike}[2]{\begin{figure}[h] \includegraphics[scale=#2] {#1} \end{figure}} %gdyby była potrzeba wstawienia obrazka, parametry: nazwa pliku, skala (jak nie wiesz co wpisać, to wpisz 1)

\begin{document}
\maketitle
\kategoria{Wikieł/C1.5l}
\zadStart{Zadanie z Wikieł C 1.5l moja wersja nr [nrWersji]}
%[b]:[6,8,10,12,14,16,18,20,22,24,26,28,30,32,34,36,38,40,42,44,46,48,50]
%[a]=int([b]/2)-1
%[c]=int([b]/2)
Oblicz całkę $$\int \frac{x^{[a]}}{1+x^{[b]}} dx,$$ całkując przez podstawienie.
\zadStop
\rozwStart{Justyna Chojecka}{}
Skorzystamy z następującego podstawienia
$$t=x^{[c]},$$
$$dt=[c]x^{[a]}dx.$$
Wówczas mamy, że
$$\int \frac{x^{[a]}}{1+x^{[b]}} dx=\int \frac{1}{[c]}\cdot \frac{1}{1+t^{2}}dt=\frac{1}{[c]}\int \frac{1}{1+t^{2}}dt$$$$=\frac{1}{[c]}arctg(t)+C=\frac{1}{[c]}arctg(x^{[c]})+C.$$
\rozwStop
\odpStart
$\frac{1}{[c]}arctg(x^{[c]})+C$
\odpStop
\testStart
A.$\frac{1}{[c]}arctg(x^{[c]})+C$\\
B.$\frac{1}{[a]}arctg(x^{[a]})+C$\\
C.$\frac{1}{[b]}arctg(x^{[a]})+C$\\
D.$\frac{1}{[c]}arctg(x^{[b]})+C$\\
E.$\frac{1}{[a]}arctg(x^{[c]})+C$\\
F.$-\frac{1}{[c]}arctg(x^{[c]})+C$\\
G.$\frac{1}{[b]}arctg(x^{[c]})+C$\\
H.$\frac{1}{[a]}arctg(x^{[b]})+C$\\
I.$\frac{1}{[c]}arctg(x^{[a]})+C$\\
\testStop
\kluczStart
A
\kluczStop



\end{document}