\documentclass[12pt, a4paper]{article}
\usepackage[utf8]{inputenc}
\usepackage{polski}

\usepackage{amsthm}  %pakiet do tworzenia twierdzeń itp.
\usepackage{amsmath} %pakiet do niektórych symboli matematycznych
\usepackage{amssymb} %pakiet do symboli mat., np. \nsubseteq
\usepackage{amsfonts}
\usepackage{graphicx} %obsługa plików graficznych z rozszerzeniem png, jpg
\theoremstyle{definition} %styl dla definicji
\newtheorem{zad}{} 
\title{Multizestaw zadań}
\author{Radosław Grzyb}
%\date{\today}
\date{}
\newcounter{liczniksekcji}
\newcommand{\kategoria}[1]{\section{#1}} %olreślamy nazwę kateforii zadań
\newcommand{\zadStart}[1]{\begin{zad}#1\newline} %oznaczenie początku zadania
\newcommand{\zadStop}{\end{zad}}   %oznaczenie końca zadania
%Makra opcjonarne (nie muszą występować):
\newcommand{\rozwStart}[2]{\noindent \textbf{Rozwiązanie (autor #1 , recenzent #2): }\newline} %oznaczenie początku rozwiązania, opcjonarnie można wprowadzić informację o autorze rozwiązania zadania i recenzencie poprawności wykonania rozwiązania zadania
\newcommand{\rozwStop}{\newline}                                            %oznaczenie końca rozwiązania
\newcommand{\odpStart}{\noindent \textbf{Odpowiedź:}\newline}    %oznaczenie początku odpowiedzi końcowej (wypisanie wyniku)
\newcommand{\odpStop}{\newline}                                             %oznaczenie końca odpowiedzi końcowej (wypisanie wyniku)
\newcommand{\testStart}{\noindent \textbf{Test:}\newline} %ewentualne możliwe opcje odpowiedzi testowej: A. ? B. ? C. ? D. ? itd.
\newcommand{\testStop}{\newline} %koniec wprowadzania odpowiedzi testowych
\newcommand{\kluczStart}{\noindent \textbf{Test poprawna odpowiedź:}\newline} %klucz, poprawna odpowiedź pytania testowego (jedna literka): A lub B lub C lub D itd.
\newcommand{\kluczStop}{\newline} %koniec poprawnej odpowiedzi pytania testowego 
\newcommand{\wstawGrafike}[2]{\begin{figure}[h] \includegraphics[scale=#2] {#1} \end{figure}} %gdyby była potrzeba wstawienia obrazka, parametry: nazwa pliku, skala (jak nie wiesz co wpisać, to wpisz 1)

\begin{document}
\maketitle


\kategoria{Wikieł/Z5.27}
\zadStart{Zadanie z Wikieł Z 5.27 moja wersja nr [nrWersji]}
%[p1]:[2,3,4,5,6,7,8,9,10,11,12,13,14,15,16]
%[d]=2*([p1]**2)
%[e]=2*[p1]
%[f]=[p1]*[e]
%[g]=2*[e]
%[h]=[p1]*[g]
%[i]=2*[g]
%[j]=4*[g]
%[k]=4*[h]
%[l]=8*[g]
%[m]=[g]**2
%[n]=4*[i]-[j]
%[o]=[i]*[g]-4*[h]+[k]
%[p]=[g]*[h]
%[r]=[o]**2+4*[n]*[p]
%[s]=2*[n]
%[t]=round(math.sqrt([r]),3)
%[a1]=round(([o]-[t])/[s],3)
%[a2]=round(([o]+[t])/[s],3)
%[u]=round([a2]+[p1],3)
%[w]=round(([e]*[a2]+[d])/(2*[a2]-[e]),3)
%[v]=round([w]+[p1],3)
%[a3]=round([u]**2+[v]**2,3)
%[a4]=round(math.sqrt([a3]),3)

Na kole o promieniu R=[p1] opisano trójkąt prostokątny o najmniejszym polu. Znaleźć długość boków tego trójkąta.
\zadStop
\rozwStart{Klaudia Klejdysz}{}
\wstawGrafike{z5.27rys.png}{0.4}\\
W celu wyznaczenia wartości zmiennych $x$ i $y$ skorzystamy z twierdzenia Pitagorasa:
$$a^2+b^2=c^2$$
$$(x+[p1])^2+(y+[p1])^2=(x+y)^2$$
$$x^2+2x*[p1]+[p1]^2+y^2+2y*[p1]+[p1]^2=x^2+2xy+y^2$$
$$2x*[p1]+2y*[p1]+[d]=2xy$$
$$[e]x+[e]y+[d]=2xy$$
$$[e]x+[d]=y(2x-[e])\Rightarrow y=\frac{[e]x+[d]}{2x-[e]}$$
Wyznaczamy pole trójkąta przedstawionego na powyższym rysunku:
$$P_{\Delta}=\frac{(x+[p1])(y+[p1])}{2}=\frac{(x+[p1])(\frac{[e]x+[d]}{2x-[e]}+[p1])}{2}=$$
$$=\frac{(x+[p1])(\frac{[e]x+[d]}{2x-[e]}+\frac{[p1](2x-[e])}{2x-[e]})}{2}=\frac{(x+[p1])(\frac{[e]x+[d]+[e]x-[f]}{2x-[e]})}{2}=$$
$$=\frac{(x+[p1])([g]x)}{2(2x-[e])}=\frac{[g]x^2+[h]x}{4x-[g]}$$
Obliczamy pochodną funkcji $P(x)$:
$$P'(x)=\frac{(2*[g1]x+[h])(4x-[g])-([g]x^2+[h]x)(4)}{(4x-[g])^2}=$$
$$=\frac{([i]x+[h])(4x-[g])-([j]x^2+[k]x)}{16x^2-[l]x+[m]}=\frac{[n]x^2-[o]x-[p]}{16x^2-[l]x+[m]}$$
Obliczamy miejsce zerowe $P'(x)$:
$$P'(x)=\frac{[n]x^2-[o]x-[p]}{16x^2-[l]x+[m]}=0$$
$$[n]x^2-[o]x-[p]=0$$
$$\Delta=[o]^2-4*[n]*(-[p])=[r]$$
$$\sqrt{\Delta}=\sqrt{[r]}$$
$$x_1=\frac{[o]-\sqrt{[r]}}{2*[n]}=\frac{[o]-[t]}{[s]}=[a1]<0$$
$$x_2=\frac{[o]+\sqrt{[r]}}{2*[n]}=\frac{[o]+[t]}{[s]}=[a2]$$
Obliczamy długość boków trójkatą:
$$x+[p1]=[a2]+[p1]=[u]$$
$$y+[p1]=\frac{[e]x+[d]}{2x-[e]}+[p1]=\frac{[e]*[a2]+[d]}{2*[a2]-[e]}+[p1]=[w]+[p1]=[v]$$
Obliczamy przeciwprostokątną:
$$(x+[p1])^2+(y+[p1])^2=[u]^2+[v]^2=[a3]$$
$$\sqrt{[a3]}=[a4]$$
\rozwStop
\odpStart
Boki trójkąta muszą mieć długość $[u]$, $[v]$ oraz $[a4]$.
\odpStop
\testStart
A. Boki trójkąta muszą mieć długość $[u]$, $[v]$ oraz $[a4]$.\\
B. Boki trójkąta muszą mieć długość $[u]$, $[v]$ oraz $[a3]$.\\
C. Boki trójkąta muszą mieć długość $[p1]$, $[v]$ oraz $[a4]$.\\
D. Boki trójkąta muszą mieć długość $[p1]$, $[a1]$ oraz $[a2]$.\\
E. Boki trójkąta muszą mieć długość $[o]$, $[n]$ oraz $[p]$.\\
F. Boki trójkąta muszą mieć długość $[u]$, $[o]$ oraz $[a2]$
\testStop
\kluczStart
A
\kluczStop


\end{document}
