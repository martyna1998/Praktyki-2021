\documentclass[12pt, a4paper]{article}
\usepackage[utf8]{inputenc}
\usepackage{polski}

\usepackage{amsthm}  %pakiet do tworzenia twierdzeń itp.
\usepackage{amsmath} %pakiet do niektórych symboli matematycznych
\usepackage{amssymb} %pakiet do symboli mat., np. \nsubseteq
\usepackage{amsfonts}
\usepackage{graphicx} %obsługa plików graficznych z rozszerzeniem png, jpg
\theoremstyle{definition} %styl dla definicji
\newtheorem{zad}{} 
\title{Multizestaw zadań}
\author{Robert Fidytek}
%\date{\today}
\date{}
\newcounter{liczniksekcji}
\newcommand{\kategoria}[1]{\section{#1}} %olreślamy nazwę kateforii zadań
\newcommand{\zadStart}[1]{\begin{zad}#1\newline} %oznaczenie początku zadania
\newcommand{\zadStop}{\end{zad}}   %oznaczenie końca zadania
%Makra opcjonarne (nie muszą występować):
\newcommand{\rozwStart}[2]{\noindent \textbf{Rozwiązanie (autor #1 , recenzent #2): }\newline} %oznaczenie początku rozwiązania, opcjonarnie można wprowadzić informację o autorze rozwiązania zadania i recenzencie poprawności wykonania rozwiązania zadania
\newcommand{\rozwStop}{\newline}                                            %oznaczenie końca rozwiązania
\newcommand{\odpStart}{\noindent \textbf{Odpowiedź:}\newline}    %oznaczenie początku odpowiedzi końcowej (wypisanie wyniku)
\newcommand{\odpStop}{\newline}                                             %oznaczenie końca odpowiedzi końcowej (wypisanie wyniku)
\newcommand{\testStart}{\noindent \textbf{Test:}\newline} %ewentualne możliwe opcje odpowiedzi testowej: A. ? B. ? C. ? D. ? itd.
\newcommand{\testStop}{\newline} %koniec wprowadzania odpowiedzi testowych
\newcommand{\kluczStart}{\noindent \textbf{Test poprawna odpowiedź:}\newline} %klucz, poprawna odpowiedź pytania testowego (jedna literka): A lub B lub C lub D itd.
\newcommand{\kluczStop}{\newline} %koniec poprawnej odpowiedzi pytania testowego 
\newcommand{\wstawGrafike}[2]{\begin{figure}[h] \includegraphics[scale=#2] {#1} \end{figure}} %gdyby była potrzeba wstawienia obrazka, parametry: nazwa pliku, skala (jak nie wiesz co wpisać, to wpisz 1)

\begin{document}
\maketitle


\kategoria{Wikieł/P3.4}
\zadStart{Zadanie z Wikieł P 3.4 moja wersja nr [nrWersji]}
%[a]:[2,3,4,5,6,7,8,9,10,11,12]
%[b]:[21,22,23,24,25,26,27,28,29,30,31,32,33,34,35,36,37,38,39,40,41,42,43,44,45,46,47,48,49,50,51,52,53,54,55,56,57,58,59]
%[ab]=[b]-[a]
%[r]=[ab]/7
%[cr]=int([r])
%[a2]=[a]+[cr]
%[a3]=[a2]+[cr]
%[a4]=[a3]+[cr]
%[a5]=[a4]+[cr]
%[a6]=[a5]+[cr]
%[a7]=[a6]+[cr]
%[r].is_integer()==True
Pomiędzy liczby [a] i [b] wstawić sześć liczb takich, aby wraz z danymi liczbami tworzyły ciąg arytmetyczny.
\zadStop
\rozwStart{Aleksandra Pasińska}{}
Z treści zadania wynika, że w szukanym ciągu $a_{1}=[a]$ i $a_{8}=[b]$. Czyli 
$$a_{1}+7\cdot r = [b]$$ 
$$r=[cr]$$
Zatem
$$a_{2}=a_{1}+r=[a2], a_{3}=a_{2}+r=[a3], a_{4}=a_{3}+r=[a4],$$ $$ a_{5}=a_{4}+r=[a5], a_{6}=a_{5}+r=[a6], a_{7}=a_{6}+r=[a7]$$
czyli otrzymaliśmy $[a],[a2],[a3],[a4],[a5],[a6],[a7],[b]$
\rozwStop
\odpStart
$[a],[a2],[a3],[a4],[a5],[a6],[a7],[b]$
\odpStop
\testStart
A.$[a],[a2],[a3],[a4],[a5],[a6],[a7],[b]$
B.$1,2,5,9,7,0$
C.$3,4,5,6,7,8,9$
D.$9,9,9,9,9$
E.$3,4,3,4,3,4,3,4$
F.$0,2,4,5,6,7$
G.$1,3,5,7,9$
H.$8,7,6,5,4,3,2,1$
I.$0,1,9,2,8,3,7,4$
\testStop
\kluczStart
A
\kluczStop



\end{document}