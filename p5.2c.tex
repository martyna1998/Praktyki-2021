\documentclass[12pt, a4paper]{article}
\usepackage[utf8]{inputenc}
\usepackage{polski}

\usepackage{amsthm}  %pakiet do tworzenia twierdzeń itp.
\usepackage{amsmath} %pakiet do niektórych symboli matematycznych
\usepackage{amssymb} %pakiet do symboli mat., np. \nsubseteq
\usepackage{amsfonts}
\usepackage{graphicx} %obsługa plików graficznych z rozszerzeniem png, jpg
\theoremstyle{definition} %styl dla definicji
\newtheorem{zad}{} 
\title{Multizestaw zadań}
\author{Jacek Jabłoński}
%\date{\today}
\date{}
\newcounter{liczniksekcji}
\newcommand{\kategoria}[1]{\section{#1}} %olreślamy nazwę kateforii zadań
\newcommand{\zadStart}[1]{\begin{zad}#1\newline} %oznaczenie początku zadania
\newcommand{\zadStop}{\end{zad}}   %oznaczenie końca zadania
%Makra opcjonarne (nie muszą występować):
\newcommand{\rozwStart}[2]{\noindent \textbf{Rozwiązanie (autor #1 , recenzent #2): }\newline} %oznaczenie początku rozwiązania, opcjonarnie można wprowadzić informację o autorze rozwiązania zadania i recenzencie poprawności wykonania rozwiązania zadania
\newcommand{\rozwStop}{\newline}                                            %oznaczenie końca rozwiązania
\newcommand{\odpStart}{\noindent \textbf{Odpowiedź:}\newline}    %oznaczenie początku odpowiedzi końcowej (wypisanie wyniku)
\newcommand{\odpStop}{\newline}                                             %oznaczenie końca odpowiedzi końcowej (wypisanie wyniku)
\newcommand{\testStart}{\noindent \textbf{Test:}\newline} %ewentualne możliwe opcje odpowiedzi testowej: A. ? B. ? C. ? D. ? itd.
\newcommand{\testStop}{\newline} %koniec wprowadzania odpowiedzi testowych
\newcommand{\kluczStart}{\noindent \textbf{Test poprawna odpowiedź:}\newline} %klucz, poprawna odpowiedź pytania testowego (jedna literka): A lub B lub C lub D itd.
\newcommand{\kluczStop}{\newline} %koniec poprawnej odpowiedzi pytania testowego 
\newcommand{\wstawGrafike}[2]{\begin{figure}[h] \includegraphics[scale=#2] {#1} \end{figure}} %gdyby była potrzeba wstawienia obrazka, parametry: nazwa pliku, skala (jak nie wiesz co wpisać, to wpisz 1)

\begin{document}
\maketitle


\kategoria{Wikieł/P5.2c}
\zadStart{Zadanie z Wikieł P 5.2c) moja wersja nr [nrWersji]}
%[a]:[2,3,4,5,6]
%[b]:[2,3,4,5,6]
%[c]:[2,3,4,5,6]
Korzystając z podanych wzorów i twierdzeń, wyznacz pochodną funkcji:
c) $f(x)=[a]^x + [b]ln(x) + [c]log(x)$ , gdzie $x>0$
\zadStop
\rozwStart{Jacek Jabłoński}{}
$$f'(x) = ([a]^x + [b]ln(x) +[c]log(x))' = ([a]^x)' + ([b]ln(x))' +([c]log[x])' = [a]^xln([a]) + \frac{[b]}{x} + \frac{[c]}{xln(10)}$$
\rozwStop
\odpStart
$$[a]^xln([a]) + \frac{[b]}{x} + \frac{[c]}{xln(10)}$$
\odpStop
\testStart
A. $$[a]^xln([a]) + \frac{[b]}{x} + \frac{[c]}{xln(10)}$$
B. $$[a]^xln([a]) - \frac{[b]}{x} + \frac{[c]}{xlog(x)}$$
C. $$[a]^xlog([a]) + \frac{[b]}{x} + \frac{[c]}{xln(10)}$$
D. $$[a]^xlog(x) + \frac{1}{[b]x} + \frac{1}{[c]log(x)}$$
E. $$[a]^xln([a]) + \frac{[b]}{x} - \frac{[c]}{xln(10)}$$
F. $$x^{[a]}ln([a]) + \frac{1}{[b]x} + \frac{[c]}{[xln(10)}$$
G. $$[a]^xln([a]) - \frac{[b]}{x} - \frac{[c]}{xln(10)}$$
H.$$x^{[a]}log(x) + \frac{1}{[b]x} + \frac{[c]}{xln(10)}$$
I. $$[a]^xln([a]) + \frac{[c]}{x} + \frac{[b]}{xln(10)}$$
\testStop
\kluczStart
A
\kluczStop



\end{document}