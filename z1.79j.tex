\documentclass[12pt, a4paper]{article}
\usepackage[utf8]{inputenc}
\usepackage{polski}

\usepackage{amsthm}  %pakiet do tworzenia twierdzeń itp.
\usepackage{amsmath} %pakiet do niektórych symboli matematycznych
\usepackage{amssymb} %pakiet do symboli mat., np. \nsubseteq
\usepackage{amsfonts}
\usepackage{graphicx} %obsługa plików graficznych z rozszerzeniem png, jpg
\theoremstyle{definition} %styl dla definicji
\newtheorem{zad}{} 
\title{Multizestaw zadań}
\author{Robert Fidytek}
%\date{\today}
\date{}
\newcounter{liczniksekcji}
\newcommand{\kategoria}[1]{\section{#1}} %olreślamy nazwę kateforii zadań
\newcommand{\zadStart}[1]{\begin{zad}#1\newline} %oznaczenie początku zadania
\newcommand{\zadStop}{\end{zad}}   %oznaczenie końca zadania
%Makra opcjonarne (nie muszą występować):
\newcommand{\rozwStart}[2]{\noindent \textbf{Rozwiązanie (autor #1 , recenzent #2): }\newline} %oznaczenie początku rozwiązania, opcjonarnie można wprowadzić informację o autorze rozwiązania zadania i recenzencie poprawności wykonania rozwiązania zadania
\newcommand{\rozwStop}{\newline}                                            %oznaczenie końca rozwiązania
\newcommand{\odpStart}{\noindent \textbf{Odpowiedź:}\newline}    %oznaczenie początku odpowiedzi końcowej (wypisanie wyniku)
\newcommand{\odpStop}{\newline}                                             %oznaczenie końca odpowiedzi końcowej (wypisanie wyniku)
\newcommand{\testStart}{\noindent \textbf{Test:}\newline} %ewentualne możliwe opcje odpowiedzi testowej: A. ? B. ? C. ? D. ? itd.
\newcommand{\testStop}{\newline} %koniec wprowadzania odpowiedzi testowych
\newcommand{\kluczStart}{\noindent \textbf{Test poprawna odpowiedź:}\newline} %klucz, poprawna odpowiedź pytania testowego (jedna literka): A lub B lub C lub D itd.
\newcommand{\kluczStop}{\newline} %koniec poprawnej odpowiedzi pytania testowego 
\newcommand{\wstawGrafike}[2]{\begin{figure}[h] \includegraphics[scale=#2] {#1} \end{figure}} %gdyby była potrzeba wstawienia obrazka, parametry: nazwa pliku, skala (jak nie wiesz co wpisać, to wpisz 1)

\begin{document}
\maketitle


\kategoria{Wikieł/Z1.79j}
\zadStart{Zadanie z Wikieł Z 1.79 j) moja wersja nr [nrWersji]}
%[a]:[2,3,4,5,6,7,8,9,10,11,12]
%[b]:[2,3,4,5,6,7,8,9,10,11,12]
%[c]=random.randint(-100,-14)
%[a]!=[b] and [a]!=[c] and [b]!=[c] and math.gcd([a],[b])==1
Rozwiązać nierówność
$$\sqrt{[a]x^2-[b]x}>[c].$$
\zadStop
\rozwStart{Adrianna Stobiecka}{}
Zakładamy, że $[a]x^2-[b]x\geq0$. Zatem:
$$[a]x^2-[b]x\geq\Leftrightarrow[a]x\bigg(x-\frac{[b]}{[a]}\bigg)\geq0$$
Zauważamy, że mamy parabolę z ramionami skierowanymi w górę oraz miejscami zerowymi w $x=0$ oraz $x=\frac{[b]}{[a]}$. Stąd $x\in\big(-\infty,0\big]\cup\big[\frac{[b]}{[a]},\infty\big)$.
\\Lewa strona rozważanej nierówności jest nieujemna, a prawa ujemna, więc nierówność zachodzi dla każdego $x\in\big(-\infty,0\big]\cup\big[\frac{[b]}{[a]},\infty\big)$.
\rozwStop
\odpStart
$x\in\big(-\infty,0\big]\cup\big[\frac{[b]}{[a]},\infty\big)$
\odpStop
\testStart
A.$x\in\big(-\infty,[c]\big]\cup\big[0,\infty\big)$
B.$x\in\mathbb{R}\setminus\{0\}$
C.$x\in\mathbb{R}$
D.$x\in\big(-\infty,0\big)\cup\big(\frac{[b]}{[a]},\infty\big)$
E.$x\in\big(-\infty,0\big]\cup\big[\frac{[b]}{[a]},\infty\big)$
F.$x\in\big(-\infty,0\big]\cup\big(\frac{[b]}{[a]},\infty\big)$
G.$x\in\big[\frac{[b]}{[a]},\infty\big)$
H.$x\in\big(-\infty,0\big)\cup\big[\frac{[b]}{[a]},\infty\big)$
I.$x\in\big(-\infty,0\big]$
\testStop
\kluczStart
E
\kluczStop



\end{document}
