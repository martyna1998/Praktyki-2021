\documentclass[12pt, a4paper]{article}
\usepackage[utf8]{inputenc}
\usepackage{polski}

\usepackage{amsthm}  %pakiet do tworzenia twierdzeń itp.
\usepackage{amsmath} %pakiet do niektórych symboli matematycznych
\usepackage{amssymb} %pakiet do symboli mat., np. \nsubseteq
\usepackage{amsfonts}
\usepackage{graphicx} %obsługa plików graficznych z rozszerzeniem png, jpg
\theoremstyle{definition} %styl dla definicji
\newtheorem{zad}{} 
\title{Multizestaw zadań}
\author{Robert Fidytek}
%\date{\today}
\date{}
\newcounter{liczniksekcji}
\newcommand{\kategoria}[1]{\section{#1}} %olreślamy nazwę kateforii zadań
\newcommand{\zadStart}[1]{\begin{zad}#1\newline} %oznaczenie początku zadania
\newcommand{\zadStop}{\end{zad}}   %oznaczenie końca zadania
%Makra opcjonarne (nie muszą występować):
\newcommand{\rozwStart}[2]{\noindent \textbf{Rozwiązanie (autor #1 , recenzent #2): }\newline} %oznaczenie początku rozwiązania, opcjonarnie można wprowadzić informację o autorze rozwiązania zadania i recenzencie poprawności wykonania rozwiązania zadania
\newcommand{\rozwStop}{\newline}                                            %oznaczenie końca rozwiązania
\newcommand{\odpStart}{\noindent \textbf{Odpowiedź:}\newline}    %oznaczenie początku odpowiedzi końcowej (wypisanie wyniku)
\newcommand{\odpStop}{\newline}                                             %oznaczenie końca odpowiedzi końcowej (wypisanie wyniku)
\newcommand{\testStart}{\noindent \textbf{Test:}\newline} %ewentualne możliwe opcje odpowiedzi testowej: A. ? B. ? C. ? D. ? itd.
\newcommand{\testStop}{\newline} %koniec wprowadzania odpowiedzi testowych
\newcommand{\kluczStart}{\noindent \textbf{Test poprawna odpowiedź:}\newline} %klucz, poprawna odpowiedź pytania testowego (jedna literka): A lub B lub C lub D itd.
\newcommand{\kluczStop}{\newline} %koniec poprawnej odpowiedzi pytania testowego 
\newcommand{\wstawGrafike}[2]{\begin{figure}[h] \includegraphics[scale=#2] {#1} \end{figure}} %gdyby była potrzeba wstawienia obrazka, parametry: nazwa pliku, skala (jak nie wiesz co wpisać, to wpisz 1)

\begin{document}
\maketitle


\kategoria{Wikieł/Z1.68a}
\zadStart{Zadanie z Wikieł Z 1.68 a) moja wersja nr [nrWersji]}
%[a]:[2,3,4]
%[b]:[2,3,4]
%[c]:[2,3,4]
%[d]:[2,3,4]
%[e]:[2,3,4]
%[f]:[2,3,4]
%[a]=random.randint(2,7)
%[b]=random.randint(2,7)
%[c]=random.randint(2,7)
%[d]=random.randint(2,7)
%[e]=random.randint(2,7)
%[f]=random.randint(2,7)
%[ekw]=[e]*[e]
%[4df]=4*[d]*[f]
%[delta1]=[ekw]+[4df]
%[pierw1]=pow([delta1],1/2)
%[pierw12]=int([pierw1].real)
%[2d]=2*[d]
%[r1l]=((-1)*[e]-[pierw12])
%[r1m]=[2d]
%[r1]=round([r1l]/[r1m],2)
%[r2l]=((-1)*[e]+[pierw12])
%[r2m]=[2d]
%[r2]=round([r2l]/[r2m],2)
%[4ac]=4*[a]*[c]
%[bkw]=[b]*[b]
%[delta2]=[bkw]+[4ac]
%[pierw2]=pow([delta2],1/2)
%[pierw22]=int([pierw2].real)
%[2a]=2*[a]
%[x1l]=((-1)*[b]-[pierw22])
%[x1m]=[2a]
%[x1]=round([x1l]/[x1m],2)
%[x2l]=((-1)*[b]+[pierw22])
%[x2m]=[2a]
%[x2]=round([x2l]/[x2m],2)
%[delta1]>0 and [pierw1].is_integer()==True and [delta2]>0 and [pierw2].is_integer()==True and [x1]!=[r1] and [x1]!=[r2] and [x2]!=[r1] and [x2]!=[r2] and [x1]!=[a] and [x1]!=[b] and [x1]!=[e] and [x1]!=[c]
Rozwiązać równania $\frac{[a]x^{2}+[b]x-[c]}{[d]x^{2}+[e]x-[f]}=0$
\zadStop
\rozwStart{Jakub Ulrych}{}
$$\frac{[a]x^{2}+[b]x-[c]}{[d]x^{2}+[e]x-[f]}$$
założenie:$$[d]x^{2}+[e]x-[f]\neq0$$
$$\Delta=[e]^{2}+[4df]=[delta1]$$
$$\sqrt{\Delta}=\sqrt{[delta1]}=[pierw12]$$
$$r_{1}=\frac{-[e]-[pierw12]}{[2d]}=\frac{[r1l]}{[r1m]}=[r1]$$
$$r_{2}=\frac{-[e]+[pierw12]}{[2d]}=\frac{[r2l]}{[r2m]}=[r2]$$
dziedzina:$$x\in\mathbb{R}-\{[r1],[r2]\}$$
rozwiązanie:$$\frac{[a]x^{2}+[b]x-[c]}{[d]x^{2}+[e]x-[f]}=0\Leftrightarrow[a]x^{2}+[b]x-[c]=0$$
$$\Delta_{2}=[b]^{2}+[4ac]=[delta2]$$
$$\sqrt{\Delta_{2}}=\sqrt{[delta2]}=[pierw22]$$
$$x_{1}=\frac{-[b]-[pierw22]}{[2a]}=\frac{[x1l]}{[x1m]}=[x1]$$
$$x_{2}=\frac{-[b]+[pierw22]}{[2a]}=\frac{[x2l]}{[x2m]}=[x2]$$
$$x\in\{[x1],[x2]\}$$
\rozwStop
\odpStart
$$x\in\{[x1],[x2]\}$$
\odpStop
\testStart
A.$x\in\{[x1],[x2]\}$
B.$x\in\{[x1],[c]\}$
C.$x\in\{[a],[b]\}$
D.$x\in\{[e],[b]]\}$
\testStop
\kluczStart
A
\kluczStop
\end{document}