\documentclass[12pt, a4paper]{article}
\usepackage[utf8]{inputenc}
\usepackage{polski}

\usepackage{amsthm}  %pakiet do tworzenia twierdzeń itp.
\usepackage{amsmath} %pakiet do niektórych symboli matematycznych
\usepackage{amssymb} %pakiet do symboli mat., np. \nsubseteq
\usepackage{amsfonts}
\usepackage{graphicx} %obsługa plików graficznych z rozszerzeniem png, jpg
\theoremstyle{definition} %styl dla definicji
\newtheorem{zad}{} 
\title{Multizestaw zadań}
\author{Robert Fidytek}
%\date{\today}
\date{}
\newcounter{liczniksekcji}
\newcommand{\kategoria}[1]{\section{#1}} %olreślamy nazwę kateforii zadań
\newcommand{\zadStart}[1]{\begin{zad}#1\newline} %oznaczenie początku zadania
\newcommand{\zadStop}{\end{zad}}   %oznaczenie końca zadania
%Makra opcjonarne (nie muszą występować):
\newcommand{\rozwStart}[2]{\noindent \textbf{Rozwiązanie (autor #1 , recenzent #2): }\newline} %oznaczenie początku rozwiązania, opcjonarnie można wprowadzić informację o autorze rozwiązania zadania i recenzencie poprawności wykonania rozwiązania zadania
\newcommand{\rozwStop}{\newline}                                            %oznaczenie końca rozwiązania
\newcommand{\odpStart}{\noindent \textbf{Odpowiedź:}\newline}    %oznaczenie początku odpowiedzi końcowej (wypisanie wyniku)
\newcommand{\odpStop}{\newline}                                             %oznaczenie końca odpowiedzi końcowej (wypisanie wyniku)
\newcommand{\testStart}{\noindent \textbf{Test:}\newline} %ewentualne możliwe opcje odpowiedzi testowej: A. ? B. ? C. ? D. ? itd.
\newcommand{\testStop}{\newline} %koniec wprowadzania odpowiedzi testowych
\newcommand{\kluczStart}{\noindent \textbf{Test poprawna odpowiedź:}\newline} %klucz, poprawna odpowiedź pytania testowego (jedna literka): A lub B lub C lub D itd.
\newcommand{\kluczStop}{\newline} %koniec poprawnej odpowiedzi pytania testowego 
\newcommand{\wstawGrafike}[2]{\begin{figure}[h] \includegraphics[scale=#2] {#1} \end{figure}} %gdyby była potrzeba wstawienia obrazka, parametry: nazwa pliku, skala (jak nie wiesz co wpisać, to wpisz 1)

\begin{document}
\maketitle


\kategoria{Wikieł/Z5.17 d}
\zadStart{Zadanie z Wikieł Z 5.17 d) moja wersja nr [nrWersji]}
%[a]:[2,3,4,5,6,7,8,9]
%[b]:[2,3,4,5,6,7,8,9]
%[c]=2*[a]
%[a]!=0
Obliczyć pochodną rzędu trzeciego funkcji $f(x)=[a]\ln([b]x)$.
\zadStop
\rozwStart{Joanna Świerzbin}{}
$$f(x)=[a]\ln([b]x)$$
$$f'(x)= \left([a]\ln([b]x) \right)' = \frac{[a]\cdot[b]}{[b]x}=\frac{[a]}{x}$$
$$f''(x)= \left( \frac{[a]}{x} \right)'= \left( [a] x^{-1} \right)' = -[a] x^{-2} = -\frac{[a]}{x^2}$$
$$f'''(x)= \left( -\frac{[a]}{x^2} \right)'=\frac{2\cdot[a]}{x^3} =\frac{[c]}{x^3}  $$
\rozwStop
\odpStart
$f'''(x) = \frac{[c]}{x^3} $
\odpStop
\testStart
A. $f'''(x) = \frac{[c]}{x^3} $\\
B. $f'''(x) = \frac{1}{x^3} $ \\
C. $f'''(x) = \frac{[a]}{x^3} $\\
D. $f'''(x) = \frac{[b]}{x^3} $\\
E. $f'''(x) = \frac{[c]}{x^2} $\\
F. $f'''(x) = \frac{[c]}{x} $
\testStop
\kluczStart
A
\kluczStop



\end{document}