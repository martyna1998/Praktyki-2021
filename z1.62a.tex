\documentclass[12pt, a4paper]{article}
\usepackage[utf8]{inputenc}
\usepackage{polski}
\usepackage{amsthm}  %pakiet do tworzenia twierdzeń itp.
\usepackage{amsmath} %pakiet do niektórych symboli matematycznych
\usepackage{amssymb} %pakiet do symboli mat., np. \nsubseteq
\usepackage{amsfonts}
\usepackage{graphicx} %obsługa plików graficznych z rozszerzeniem png, jpg
\theoremstyle{definition} %styl dla definicji
\newtheorem{zad}{} 
\title{Multizestaw zadań}
\author{Patryk Wirkus}
%\date{\today}
\date{}
\newcommand{\kategoria}[1]{\section{#1}}
\newcommand{\zadStart}[1]{\begin{zad}#1\newline}
\newcommand{\zadStop}{\end{zad}}
\newcommand{\rozwStart}[2]{\noindent \textbf{Rozwiązanie (autor #1 , recenzent #2): }\newline}
\newcommand{\rozwStop}{\newline}                                           
\newcommand{\odpStart}{\noindent \textbf{Odpowiedź:}\newline}
\newcommand{\odpStop}{\newline}
\newcommand{\testStart}{\noindent \textbf{Test:}\newline}
\newcommand{\testStop}{\newline}
\newcommand{\kluczStart}{\noindent \textbf{Test poprawna odpowiedź:}\newline}
\newcommand{\kluczStop}{\newline}
\newcommand{\wstawGrafike}[2]{\begin{figure}[h] \includegraphics[scale=#2] {#1} \end{figure}}

\begin{document}
\maketitle

\kategoria{Wikieł/1.62a}


\zadStart{Zadanie z Wikieł Z 1.62 a) moja wersja nr 1}

Rozwiązać nierówności $(x-1)(x-2)(x-3)\ge0$.
\zadStop
\rozwStart{Patryk Wirkus}{Laura Mieczkowska}
Miejsca zerowe naszego wielomianu to: $1, 2, 3$.\\
Wielomian jest stopnia nieparzystego, ponadto znak współczynnika przy\linebreak najwyższej potędze x jest dodatni.\\ W związku z tym wykres wielomianu zaczyna się od lewej strony poniżej osi OX. A więc $$x \in [1,2] \cup [3,\infty).$$
\rozwStop
\odpStart
$x \in [1,2] \cup [3,\infty)$
\odpStop
\testStart
A.$x \in [1,2] \cup [3,\infty)$\\
B.$x \in (1,2) \cup [3,\infty)$\\
C.$x \in (1,2] \cup [3,\infty)$\\
D.$x \in [1,2) \cup [3,\infty)$\\
E.$x \in [1,2] \cup (3,\infty)$\\
F.$x \in (1,2) \cup (3,\infty)$\\
G.$x \in [1,2) \cup (3,\infty)$\\
H.$x \in (1,2] \cup (3,\infty)$
\testStop
\kluczStart
A
\kluczStop



\zadStart{Zadanie z Wikieł Z 1.62 a) moja wersja nr 2}

Rozwiązać nierówności $(x-1)(x-2)(x-4)\ge0$.
\zadStop
\rozwStart{Patryk Wirkus}{Laura Mieczkowska}
Miejsca zerowe naszego wielomianu to: $1, 2, 4$.\\
Wielomian jest stopnia nieparzystego, ponadto znak współczynnika przy\linebreak najwyższej potędze x jest dodatni.\\ W związku z tym wykres wielomianu zaczyna się od lewej strony poniżej osi OX. A więc $$x \in [1,2] \cup [4,\infty).$$
\rozwStop
\odpStart
$x \in [1,2] \cup [4,\infty)$
\odpStop
\testStart
A.$x \in [1,2] \cup [4,\infty)$\\
B.$x \in (1,2) \cup [4,\infty)$\\
C.$x \in (1,2] \cup [4,\infty)$\\
D.$x \in [1,2) \cup [4,\infty)$\\
E.$x \in [1,2] \cup (4,\infty)$\\
F.$x \in (1,2) \cup (4,\infty)$\\
G.$x \in [1,2) \cup (4,\infty)$\\
H.$x \in (1,2] \cup (4,\infty)$
\testStop
\kluczStart
A
\kluczStop



\zadStart{Zadanie z Wikieł Z 1.62 a) moja wersja nr 3}

Rozwiązać nierówności $(x-1)(x-2)(x-5)\ge0$.
\zadStop
\rozwStart{Patryk Wirkus}{Laura Mieczkowska}
Miejsca zerowe naszego wielomianu to: $1, 2, 5$.\\
Wielomian jest stopnia nieparzystego, ponadto znak współczynnika przy\linebreak najwyższej potędze x jest dodatni.\\ W związku z tym wykres wielomianu zaczyna się od lewej strony poniżej osi OX. A więc $$x \in [1,2] \cup [5,\infty).$$
\rozwStop
\odpStart
$x \in [1,2] \cup [5,\infty)$
\odpStop
\testStart
A.$x \in [1,2] \cup [5,\infty)$\\
B.$x \in (1,2) \cup [5,\infty)$\\
C.$x \in (1,2] \cup [5,\infty)$\\
D.$x \in [1,2) \cup [5,\infty)$\\
E.$x \in [1,2] \cup (5,\infty)$\\
F.$x \in (1,2) \cup (5,\infty)$\\
G.$x \in [1,2) \cup (5,\infty)$\\
H.$x \in (1,2] \cup (5,\infty)$
\testStop
\kluczStart
A
\kluczStop



\zadStart{Zadanie z Wikieł Z 1.62 a) moja wersja nr 4}

Rozwiązać nierówności $(x-1)(x-2)(x-6)\ge0$.
\zadStop
\rozwStart{Patryk Wirkus}{Laura Mieczkowska}
Miejsca zerowe naszego wielomianu to: $1, 2, 6$.\\
Wielomian jest stopnia nieparzystego, ponadto znak współczynnika przy\linebreak najwyższej potędze x jest dodatni.\\ W związku z tym wykres wielomianu zaczyna się od lewej strony poniżej osi OX. A więc $$x \in [1,2] \cup [6,\infty).$$
\rozwStop
\odpStart
$x \in [1,2] \cup [6,\infty)$
\odpStop
\testStart
A.$x \in [1,2] \cup [6,\infty)$\\
B.$x \in (1,2) \cup [6,\infty)$\\
C.$x \in (1,2] \cup [6,\infty)$\\
D.$x \in [1,2) \cup [6,\infty)$\\
E.$x \in [1,2] \cup (6,\infty)$\\
F.$x \in (1,2) \cup (6,\infty)$\\
G.$x \in [1,2) \cup (6,\infty)$\\
H.$x \in (1,2] \cup (6,\infty)$
\testStop
\kluczStart
A
\kluczStop



\zadStart{Zadanie z Wikieł Z 1.62 a) moja wersja nr 5}

Rozwiązać nierówności $(x-1)(x-2)(x-7)\ge0$.
\zadStop
\rozwStart{Patryk Wirkus}{Laura Mieczkowska}
Miejsca zerowe naszego wielomianu to: $1, 2, 7$.\\
Wielomian jest stopnia nieparzystego, ponadto znak współczynnika przy\linebreak najwyższej potędze x jest dodatni.\\ W związku z tym wykres wielomianu zaczyna się od lewej strony poniżej osi OX. A więc $$x \in [1,2] \cup [7,\infty).$$
\rozwStop
\odpStart
$x \in [1,2] \cup [7,\infty)$
\odpStop
\testStart
A.$x \in [1,2] \cup [7,\infty)$\\
B.$x \in (1,2) \cup [7,\infty)$\\
C.$x \in (1,2] \cup [7,\infty)$\\
D.$x \in [1,2) \cup [7,\infty)$\\
E.$x \in [1,2] \cup (7,\infty)$\\
F.$x \in (1,2) \cup (7,\infty)$\\
G.$x \in [1,2) \cup (7,\infty)$\\
H.$x \in (1,2] \cup (7,\infty)$
\testStop
\kluczStart
A
\kluczStop



\zadStart{Zadanie z Wikieł Z 1.62 a) moja wersja nr 6}

Rozwiązać nierówności $(x-1)(x-2)(x-8)\ge0$.
\zadStop
\rozwStart{Patryk Wirkus}{Laura Mieczkowska}
Miejsca zerowe naszego wielomianu to: $1, 2, 8$.\\
Wielomian jest stopnia nieparzystego, ponadto znak współczynnika przy\linebreak najwyższej potędze x jest dodatni.\\ W związku z tym wykres wielomianu zaczyna się od lewej strony poniżej osi OX. A więc $$x \in [1,2] \cup [8,\infty).$$
\rozwStop
\odpStart
$x \in [1,2] \cup [8,\infty)$
\odpStop
\testStart
A.$x \in [1,2] \cup [8,\infty)$\\
B.$x \in (1,2) \cup [8,\infty)$\\
C.$x \in (1,2] \cup [8,\infty)$\\
D.$x \in [1,2) \cup [8,\infty)$\\
E.$x \in [1,2] \cup (8,\infty)$\\
F.$x \in (1,2) \cup (8,\infty)$\\
G.$x \in [1,2) \cup (8,\infty)$\\
H.$x \in (1,2] \cup (8,\infty)$
\testStop
\kluczStart
A
\kluczStop



\zadStart{Zadanie z Wikieł Z 1.62 a) moja wersja nr 7}

Rozwiązać nierówności $(x-1)(x-2)(x-9)\ge0$.
\zadStop
\rozwStart{Patryk Wirkus}{Laura Mieczkowska}
Miejsca zerowe naszego wielomianu to: $1, 2, 9$.\\
Wielomian jest stopnia nieparzystego, ponadto znak współczynnika przy\linebreak najwyższej potędze x jest dodatni.\\ W związku z tym wykres wielomianu zaczyna się od lewej strony poniżej osi OX. A więc $$x \in [1,2] \cup [9,\infty).$$
\rozwStop
\odpStart
$x \in [1,2] \cup [9,\infty)$
\odpStop
\testStart
A.$x \in [1,2] \cup [9,\infty)$\\
B.$x \in (1,2) \cup [9,\infty)$\\
C.$x \in (1,2] \cup [9,\infty)$\\
D.$x \in [1,2) \cup [9,\infty)$\\
E.$x \in [1,2] \cup (9,\infty)$\\
F.$x \in (1,2) \cup (9,\infty)$\\
G.$x \in [1,2) \cup (9,\infty)$\\
H.$x \in (1,2] \cup (9,\infty)$
\testStop
\kluczStart
A
\kluczStop



\zadStart{Zadanie z Wikieł Z 1.62 a) moja wersja nr 8}

Rozwiązać nierówności $(x-1)(x-2)(x-10)\ge0$.
\zadStop
\rozwStart{Patryk Wirkus}{Laura Mieczkowska}
Miejsca zerowe naszego wielomianu to: $1, 2, 10$.\\
Wielomian jest stopnia nieparzystego, ponadto znak współczynnika przy\linebreak najwyższej potędze x jest dodatni.\\ W związku z tym wykres wielomianu zaczyna się od lewej strony poniżej osi OX. A więc $$x \in [1,2] \cup [10,\infty).$$
\rozwStop
\odpStart
$x \in [1,2] \cup [10,\infty)$
\odpStop
\testStart
A.$x \in [1,2] \cup [10,\infty)$\\
B.$x \in (1,2) \cup [10,\infty)$\\
C.$x \in (1,2] \cup [10,\infty)$\\
D.$x \in [1,2) \cup [10,\infty)$\\
E.$x \in [1,2] \cup (10,\infty)$\\
F.$x \in (1,2) \cup (10,\infty)$\\
G.$x \in [1,2) \cup (10,\infty)$\\
H.$x \in (1,2] \cup (10,\infty)$
\testStop
\kluczStart
A
\kluczStop



\zadStart{Zadanie z Wikieł Z 1.62 a) moja wersja nr 9}

Rozwiązać nierówności $(x-1)(x-2)(x-11)\ge0$.
\zadStop
\rozwStart{Patryk Wirkus}{Laura Mieczkowska}
Miejsca zerowe naszego wielomianu to: $1, 2, 11$.\\
Wielomian jest stopnia nieparzystego, ponadto znak współczynnika przy\linebreak najwyższej potędze x jest dodatni.\\ W związku z tym wykres wielomianu zaczyna się od lewej strony poniżej osi OX. A więc $$x \in [1,2] \cup [11,\infty).$$
\rozwStop
\odpStart
$x \in [1,2] \cup [11,\infty)$
\odpStop
\testStart
A.$x \in [1,2] \cup [11,\infty)$\\
B.$x \in (1,2) \cup [11,\infty)$\\
C.$x \in (1,2] \cup [11,\infty)$\\
D.$x \in [1,2) \cup [11,\infty)$\\
E.$x \in [1,2] \cup (11,\infty)$\\
F.$x \in (1,2) \cup (11,\infty)$\\
G.$x \in [1,2) \cup (11,\infty)$\\
H.$x \in (1,2] \cup (11,\infty)$
\testStop
\kluczStart
A
\kluczStop



\zadStart{Zadanie z Wikieł Z 1.62 a) moja wersja nr 10}

Rozwiązać nierówności $(x-1)(x-2)(x-12)\ge0$.
\zadStop
\rozwStart{Patryk Wirkus}{Laura Mieczkowska}
Miejsca zerowe naszego wielomianu to: $1, 2, 12$.\\
Wielomian jest stopnia nieparzystego, ponadto znak współczynnika przy\linebreak najwyższej potędze x jest dodatni.\\ W związku z tym wykres wielomianu zaczyna się od lewej strony poniżej osi OX. A więc $$x \in [1,2] \cup [12,\infty).$$
\rozwStop
\odpStart
$x \in [1,2] \cup [12,\infty)$
\odpStop
\testStart
A.$x \in [1,2] \cup [12,\infty)$\\
B.$x \in (1,2) \cup [12,\infty)$\\
C.$x \in (1,2] \cup [12,\infty)$\\
D.$x \in [1,2) \cup [12,\infty)$\\
E.$x \in [1,2] \cup (12,\infty)$\\
F.$x \in (1,2) \cup (12,\infty)$\\
G.$x \in [1,2) \cup (12,\infty)$\\
H.$x \in (1,2] \cup (12,\infty)$
\testStop
\kluczStart
A
\kluczStop



\zadStart{Zadanie z Wikieł Z 1.62 a) moja wersja nr 11}

Rozwiązać nierówności $(x-1)(x-2)(x-13)\ge0$.
\zadStop
\rozwStart{Patryk Wirkus}{Laura Mieczkowska}
Miejsca zerowe naszego wielomianu to: $1, 2, 13$.\\
Wielomian jest stopnia nieparzystego, ponadto znak współczynnika przy\linebreak najwyższej potędze x jest dodatni.\\ W związku z tym wykres wielomianu zaczyna się od lewej strony poniżej osi OX. A więc $$x \in [1,2] \cup [13,\infty).$$
\rozwStop
\odpStart
$x \in [1,2] \cup [13,\infty)$
\odpStop
\testStart
A.$x \in [1,2] \cup [13,\infty)$\\
B.$x \in (1,2) \cup [13,\infty)$\\
C.$x \in (1,2] \cup [13,\infty)$\\
D.$x \in [1,2) \cup [13,\infty)$\\
E.$x \in [1,2] \cup (13,\infty)$\\
F.$x \in (1,2) \cup (13,\infty)$\\
G.$x \in [1,2) \cup (13,\infty)$\\
H.$x \in (1,2] \cup (13,\infty)$
\testStop
\kluczStart
A
\kluczStop



\zadStart{Zadanie z Wikieł Z 1.62 a) moja wersja nr 12}

Rozwiązać nierówności $(x-1)(x-2)(x-14)\ge0$.
\zadStop
\rozwStart{Patryk Wirkus}{Laura Mieczkowska}
Miejsca zerowe naszego wielomianu to: $1, 2, 14$.\\
Wielomian jest stopnia nieparzystego, ponadto znak współczynnika przy\linebreak najwyższej potędze x jest dodatni.\\ W związku z tym wykres wielomianu zaczyna się od lewej strony poniżej osi OX. A więc $$x \in [1,2] \cup [14,\infty).$$
\rozwStop
\odpStart
$x \in [1,2] \cup [14,\infty)$
\odpStop
\testStart
A.$x \in [1,2] \cup [14,\infty)$\\
B.$x \in (1,2) \cup [14,\infty)$\\
C.$x \in (1,2] \cup [14,\infty)$\\
D.$x \in [1,2) \cup [14,\infty)$\\
E.$x \in [1,2] \cup (14,\infty)$\\
F.$x \in (1,2) \cup (14,\infty)$\\
G.$x \in [1,2) \cup (14,\infty)$\\
H.$x \in (1,2] \cup (14,\infty)$
\testStop
\kluczStart
A
\kluczStop



\zadStart{Zadanie z Wikieł Z 1.62 a) moja wersja nr 13}

Rozwiązać nierówności $(x-1)(x-2)(x-15)\ge0$.
\zadStop
\rozwStart{Patryk Wirkus}{Laura Mieczkowska}
Miejsca zerowe naszego wielomianu to: $1, 2, 15$.\\
Wielomian jest stopnia nieparzystego, ponadto znak współczynnika przy\linebreak najwyższej potędze x jest dodatni.\\ W związku z tym wykres wielomianu zaczyna się od lewej strony poniżej osi OX. A więc $$x \in [1,2] \cup [15,\infty).$$
\rozwStop
\odpStart
$x \in [1,2] \cup [15,\infty)$
\odpStop
\testStart
A.$x \in [1,2] \cup [15,\infty)$\\
B.$x \in (1,2) \cup [15,\infty)$\\
C.$x \in (1,2] \cup [15,\infty)$\\
D.$x \in [1,2) \cup [15,\infty)$\\
E.$x \in [1,2] \cup (15,\infty)$\\
F.$x \in (1,2) \cup (15,\infty)$\\
G.$x \in [1,2) \cup (15,\infty)$\\
H.$x \in (1,2] \cup (15,\infty)$
\testStop
\kluczStart
A
\kluczStop



\zadStart{Zadanie z Wikieł Z 1.62 a) moja wersja nr 14}

Rozwiązać nierówności $(x-1)(x-3)(x-4)\ge0$.
\zadStop
\rozwStart{Patryk Wirkus}{Laura Mieczkowska}
Miejsca zerowe naszego wielomianu to: $1, 3, 4$.\\
Wielomian jest stopnia nieparzystego, ponadto znak współczynnika przy\linebreak najwyższej potędze x jest dodatni.\\ W związku z tym wykres wielomianu zaczyna się od lewej strony poniżej osi OX. A więc $$x \in [1,3] \cup [4,\infty).$$
\rozwStop
\odpStart
$x \in [1,3] \cup [4,\infty)$
\odpStop
\testStart
A.$x \in [1,3] \cup [4,\infty)$\\
B.$x \in (1,3) \cup [4,\infty)$\\
C.$x \in (1,3] \cup [4,\infty)$\\
D.$x \in [1,3) \cup [4,\infty)$\\
E.$x \in [1,3] \cup (4,\infty)$\\
F.$x \in (1,3) \cup (4,\infty)$\\
G.$x \in [1,3) \cup (4,\infty)$\\
H.$x \in (1,3] \cup (4,\infty)$
\testStop
\kluczStart
A
\kluczStop



\zadStart{Zadanie z Wikieł Z 1.62 a) moja wersja nr 15}

Rozwiązać nierówności $(x-1)(x-3)(x-5)\ge0$.
\zadStop
\rozwStart{Patryk Wirkus}{Laura Mieczkowska}
Miejsca zerowe naszego wielomianu to: $1, 3, 5$.\\
Wielomian jest stopnia nieparzystego, ponadto znak współczynnika przy\linebreak najwyższej potędze x jest dodatni.\\ W związku z tym wykres wielomianu zaczyna się od lewej strony poniżej osi OX. A więc $$x \in [1,3] \cup [5,\infty).$$
\rozwStop
\odpStart
$x \in [1,3] \cup [5,\infty)$
\odpStop
\testStart
A.$x \in [1,3] \cup [5,\infty)$\\
B.$x \in (1,3) \cup [5,\infty)$\\
C.$x \in (1,3] \cup [5,\infty)$\\
D.$x \in [1,3) \cup [5,\infty)$\\
E.$x \in [1,3] \cup (5,\infty)$\\
F.$x \in (1,3) \cup (5,\infty)$\\
G.$x \in [1,3) \cup (5,\infty)$\\
H.$x \in (1,3] \cup (5,\infty)$
\testStop
\kluczStart
A
\kluczStop



\zadStart{Zadanie z Wikieł Z 1.62 a) moja wersja nr 16}

Rozwiązać nierówności $(x-1)(x-3)(x-6)\ge0$.
\zadStop
\rozwStart{Patryk Wirkus}{Laura Mieczkowska}
Miejsca zerowe naszego wielomianu to: $1, 3, 6$.\\
Wielomian jest stopnia nieparzystego, ponadto znak współczynnika przy\linebreak najwyższej potędze x jest dodatni.\\ W związku z tym wykres wielomianu zaczyna się od lewej strony poniżej osi OX. A więc $$x \in [1,3] \cup [6,\infty).$$
\rozwStop
\odpStart
$x \in [1,3] \cup [6,\infty)$
\odpStop
\testStart
A.$x \in [1,3] \cup [6,\infty)$\\
B.$x \in (1,3) \cup [6,\infty)$\\
C.$x \in (1,3] \cup [6,\infty)$\\
D.$x \in [1,3) \cup [6,\infty)$\\
E.$x \in [1,3] \cup (6,\infty)$\\
F.$x \in (1,3) \cup (6,\infty)$\\
G.$x \in [1,3) \cup (6,\infty)$\\
H.$x \in (1,3] \cup (6,\infty)$
\testStop
\kluczStart
A
\kluczStop



\zadStart{Zadanie z Wikieł Z 1.62 a) moja wersja nr 17}

Rozwiązać nierówności $(x-1)(x-3)(x-7)\ge0$.
\zadStop
\rozwStart{Patryk Wirkus}{Laura Mieczkowska}
Miejsca zerowe naszego wielomianu to: $1, 3, 7$.\\
Wielomian jest stopnia nieparzystego, ponadto znak współczynnika przy\linebreak najwyższej potędze x jest dodatni.\\ W związku z tym wykres wielomianu zaczyna się od lewej strony poniżej osi OX. A więc $$x \in [1,3] \cup [7,\infty).$$
\rozwStop
\odpStart
$x \in [1,3] \cup [7,\infty)$
\odpStop
\testStart
A.$x \in [1,3] \cup [7,\infty)$\\
B.$x \in (1,3) \cup [7,\infty)$\\
C.$x \in (1,3] \cup [7,\infty)$\\
D.$x \in [1,3) \cup [7,\infty)$\\
E.$x \in [1,3] \cup (7,\infty)$\\
F.$x \in (1,3) \cup (7,\infty)$\\
G.$x \in [1,3) \cup (7,\infty)$\\
H.$x \in (1,3] \cup (7,\infty)$
\testStop
\kluczStart
A
\kluczStop



\zadStart{Zadanie z Wikieł Z 1.62 a) moja wersja nr 18}

Rozwiązać nierówności $(x-1)(x-3)(x-8)\ge0$.
\zadStop
\rozwStart{Patryk Wirkus}{Laura Mieczkowska}
Miejsca zerowe naszego wielomianu to: $1, 3, 8$.\\
Wielomian jest stopnia nieparzystego, ponadto znak współczynnika przy\linebreak najwyższej potędze x jest dodatni.\\ W związku z tym wykres wielomianu zaczyna się od lewej strony poniżej osi OX. A więc $$x \in [1,3] \cup [8,\infty).$$
\rozwStop
\odpStart
$x \in [1,3] \cup [8,\infty)$
\odpStop
\testStart
A.$x \in [1,3] \cup [8,\infty)$\\
B.$x \in (1,3) \cup [8,\infty)$\\
C.$x \in (1,3] \cup [8,\infty)$\\
D.$x \in [1,3) \cup [8,\infty)$\\
E.$x \in [1,3] \cup (8,\infty)$\\
F.$x \in (1,3) \cup (8,\infty)$\\
G.$x \in [1,3) \cup (8,\infty)$\\
H.$x \in (1,3] \cup (8,\infty)$
\testStop
\kluczStart
A
\kluczStop



\zadStart{Zadanie z Wikieł Z 1.62 a) moja wersja nr 19}

Rozwiązać nierówności $(x-1)(x-3)(x-9)\ge0$.
\zadStop
\rozwStart{Patryk Wirkus}{Laura Mieczkowska}
Miejsca zerowe naszego wielomianu to: $1, 3, 9$.\\
Wielomian jest stopnia nieparzystego, ponadto znak współczynnika przy\linebreak najwyższej potędze x jest dodatni.\\ W związku z tym wykres wielomianu zaczyna się od lewej strony poniżej osi OX. A więc $$x \in [1,3] \cup [9,\infty).$$
\rozwStop
\odpStart
$x \in [1,3] \cup [9,\infty)$
\odpStop
\testStart
A.$x \in [1,3] \cup [9,\infty)$\\
B.$x \in (1,3) \cup [9,\infty)$\\
C.$x \in (1,3] \cup [9,\infty)$\\
D.$x \in [1,3) \cup [9,\infty)$\\
E.$x \in [1,3] \cup (9,\infty)$\\
F.$x \in (1,3) \cup (9,\infty)$\\
G.$x \in [1,3) \cup (9,\infty)$\\
H.$x \in (1,3] \cup (9,\infty)$
\testStop
\kluczStart
A
\kluczStop



\zadStart{Zadanie z Wikieł Z 1.62 a) moja wersja nr 20}

Rozwiązać nierówności $(x-1)(x-3)(x-10)\ge0$.
\zadStop
\rozwStart{Patryk Wirkus}{Laura Mieczkowska}
Miejsca zerowe naszego wielomianu to: $1, 3, 10$.\\
Wielomian jest stopnia nieparzystego, ponadto znak współczynnika przy\linebreak najwyższej potędze x jest dodatni.\\ W związku z tym wykres wielomianu zaczyna się od lewej strony poniżej osi OX. A więc $$x \in [1,3] \cup [10,\infty).$$
\rozwStop
\odpStart
$x \in [1,3] \cup [10,\infty)$
\odpStop
\testStart
A.$x \in [1,3] \cup [10,\infty)$\\
B.$x \in (1,3) \cup [10,\infty)$\\
C.$x \in (1,3] \cup [10,\infty)$\\
D.$x \in [1,3) \cup [10,\infty)$\\
E.$x \in [1,3] \cup (10,\infty)$\\
F.$x \in (1,3) \cup (10,\infty)$\\
G.$x \in [1,3) \cup (10,\infty)$\\
H.$x \in (1,3] \cup (10,\infty)$
\testStop
\kluczStart
A
\kluczStop



\zadStart{Zadanie z Wikieł Z 1.62 a) moja wersja nr 21}

Rozwiązać nierówności $(x-1)(x-3)(x-11)\ge0$.
\zadStop
\rozwStart{Patryk Wirkus}{Laura Mieczkowska}
Miejsca zerowe naszego wielomianu to: $1, 3, 11$.\\
Wielomian jest stopnia nieparzystego, ponadto znak współczynnika przy\linebreak najwyższej potędze x jest dodatni.\\ W związku z tym wykres wielomianu zaczyna się od lewej strony poniżej osi OX. A więc $$x \in [1,3] \cup [11,\infty).$$
\rozwStop
\odpStart
$x \in [1,3] \cup [11,\infty)$
\odpStop
\testStart
A.$x \in [1,3] \cup [11,\infty)$\\
B.$x \in (1,3) \cup [11,\infty)$\\
C.$x \in (1,3] \cup [11,\infty)$\\
D.$x \in [1,3) \cup [11,\infty)$\\
E.$x \in [1,3] \cup (11,\infty)$\\
F.$x \in (1,3) \cup (11,\infty)$\\
G.$x \in [1,3) \cup (11,\infty)$\\
H.$x \in (1,3] \cup (11,\infty)$
\testStop
\kluczStart
A
\kluczStop



\zadStart{Zadanie z Wikieł Z 1.62 a) moja wersja nr 22}

Rozwiązać nierówności $(x-1)(x-3)(x-12)\ge0$.
\zadStop
\rozwStart{Patryk Wirkus}{Laura Mieczkowska}
Miejsca zerowe naszego wielomianu to: $1, 3, 12$.\\
Wielomian jest stopnia nieparzystego, ponadto znak współczynnika przy\linebreak najwyższej potędze x jest dodatni.\\ W związku z tym wykres wielomianu zaczyna się od lewej strony poniżej osi OX. A więc $$x \in [1,3] \cup [12,\infty).$$
\rozwStop
\odpStart
$x \in [1,3] \cup [12,\infty)$
\odpStop
\testStart
A.$x \in [1,3] \cup [12,\infty)$\\
B.$x \in (1,3) \cup [12,\infty)$\\
C.$x \in (1,3] \cup [12,\infty)$\\
D.$x \in [1,3) \cup [12,\infty)$\\
E.$x \in [1,3] \cup (12,\infty)$\\
F.$x \in (1,3) \cup (12,\infty)$\\
G.$x \in [1,3) \cup (12,\infty)$\\
H.$x \in (1,3] \cup (12,\infty)$
\testStop
\kluczStart
A
\kluczStop



\zadStart{Zadanie z Wikieł Z 1.62 a) moja wersja nr 23}

Rozwiązać nierówności $(x-1)(x-3)(x-13)\ge0$.
\zadStop
\rozwStart{Patryk Wirkus}{Laura Mieczkowska}
Miejsca zerowe naszego wielomianu to: $1, 3, 13$.\\
Wielomian jest stopnia nieparzystego, ponadto znak współczynnika przy\linebreak najwyższej potędze x jest dodatni.\\ W związku z tym wykres wielomianu zaczyna się od lewej strony poniżej osi OX. A więc $$x \in [1,3] \cup [13,\infty).$$
\rozwStop
\odpStart
$x \in [1,3] \cup [13,\infty)$
\odpStop
\testStart
A.$x \in [1,3] \cup [13,\infty)$\\
B.$x \in (1,3) \cup [13,\infty)$\\
C.$x \in (1,3] \cup [13,\infty)$\\
D.$x \in [1,3) \cup [13,\infty)$\\
E.$x \in [1,3] \cup (13,\infty)$\\
F.$x \in (1,3) \cup (13,\infty)$\\
G.$x \in [1,3) \cup (13,\infty)$\\
H.$x \in (1,3] \cup (13,\infty)$
\testStop
\kluczStart
A
\kluczStop



\zadStart{Zadanie z Wikieł Z 1.62 a) moja wersja nr 24}

Rozwiązać nierówności $(x-1)(x-3)(x-14)\ge0$.
\zadStop
\rozwStart{Patryk Wirkus}{Laura Mieczkowska}
Miejsca zerowe naszego wielomianu to: $1, 3, 14$.\\
Wielomian jest stopnia nieparzystego, ponadto znak współczynnika przy\linebreak najwyższej potędze x jest dodatni.\\ W związku z tym wykres wielomianu zaczyna się od lewej strony poniżej osi OX. A więc $$x \in [1,3] \cup [14,\infty).$$
\rozwStop
\odpStart
$x \in [1,3] \cup [14,\infty)$
\odpStop
\testStart
A.$x \in [1,3] \cup [14,\infty)$\\
B.$x \in (1,3) \cup [14,\infty)$\\
C.$x \in (1,3] \cup [14,\infty)$\\
D.$x \in [1,3) \cup [14,\infty)$\\
E.$x \in [1,3] \cup (14,\infty)$\\
F.$x \in (1,3) \cup (14,\infty)$\\
G.$x \in [1,3) \cup (14,\infty)$\\
H.$x \in (1,3] \cup (14,\infty)$
\testStop
\kluczStart
A
\kluczStop



\zadStart{Zadanie z Wikieł Z 1.62 a) moja wersja nr 25}

Rozwiązać nierówności $(x-1)(x-3)(x-15)\ge0$.
\zadStop
\rozwStart{Patryk Wirkus}{Laura Mieczkowska}
Miejsca zerowe naszego wielomianu to: $1, 3, 15$.\\
Wielomian jest stopnia nieparzystego, ponadto znak współczynnika przy\linebreak najwyższej potędze x jest dodatni.\\ W związku z tym wykres wielomianu zaczyna się od lewej strony poniżej osi OX. A więc $$x \in [1,3] \cup [15,\infty).$$
\rozwStop
\odpStart
$x \in [1,3] \cup [15,\infty)$
\odpStop
\testStart
A.$x \in [1,3] \cup [15,\infty)$\\
B.$x \in (1,3) \cup [15,\infty)$\\
C.$x \in (1,3] \cup [15,\infty)$\\
D.$x \in [1,3) \cup [15,\infty)$\\
E.$x \in [1,3] \cup (15,\infty)$\\
F.$x \in (1,3) \cup (15,\infty)$\\
G.$x \in [1,3) \cup (15,\infty)$\\
H.$x \in (1,3] \cup (15,\infty)$
\testStop
\kluczStart
A
\kluczStop



\zadStart{Zadanie z Wikieł Z 1.62 a) moja wersja nr 26}

Rozwiązać nierówności $(x-1)(x-4)(x-5)\ge0$.
\zadStop
\rozwStart{Patryk Wirkus}{Laura Mieczkowska}
Miejsca zerowe naszego wielomianu to: $1, 4, 5$.\\
Wielomian jest stopnia nieparzystego, ponadto znak współczynnika przy\linebreak najwyższej potędze x jest dodatni.\\ W związku z tym wykres wielomianu zaczyna się od lewej strony poniżej osi OX. A więc $$x \in [1,4] \cup [5,\infty).$$
\rozwStop
\odpStart
$x \in [1,4] \cup [5,\infty)$
\odpStop
\testStart
A.$x \in [1,4] \cup [5,\infty)$\\
B.$x \in (1,4) \cup [5,\infty)$\\
C.$x \in (1,4] \cup [5,\infty)$\\
D.$x \in [1,4) \cup [5,\infty)$\\
E.$x \in [1,4] \cup (5,\infty)$\\
F.$x \in (1,4) \cup (5,\infty)$\\
G.$x \in [1,4) \cup (5,\infty)$\\
H.$x \in (1,4] \cup (5,\infty)$
\testStop
\kluczStart
A
\kluczStop



\zadStart{Zadanie z Wikieł Z 1.62 a) moja wersja nr 27}

Rozwiązać nierówności $(x-1)(x-4)(x-6)\ge0$.
\zadStop
\rozwStart{Patryk Wirkus}{Laura Mieczkowska}
Miejsca zerowe naszego wielomianu to: $1, 4, 6$.\\
Wielomian jest stopnia nieparzystego, ponadto znak współczynnika przy\linebreak najwyższej potędze x jest dodatni.\\ W związku z tym wykres wielomianu zaczyna się od lewej strony poniżej osi OX. A więc $$x \in [1,4] \cup [6,\infty).$$
\rozwStop
\odpStart
$x \in [1,4] \cup [6,\infty)$
\odpStop
\testStart
A.$x \in [1,4] \cup [6,\infty)$\\
B.$x \in (1,4) \cup [6,\infty)$\\
C.$x \in (1,4] \cup [6,\infty)$\\
D.$x \in [1,4) \cup [6,\infty)$\\
E.$x \in [1,4] \cup (6,\infty)$\\
F.$x \in (1,4) \cup (6,\infty)$\\
G.$x \in [1,4) \cup (6,\infty)$\\
H.$x \in (1,4] \cup (6,\infty)$
\testStop
\kluczStart
A
\kluczStop



\zadStart{Zadanie z Wikieł Z 1.62 a) moja wersja nr 28}

Rozwiązać nierówności $(x-1)(x-4)(x-7)\ge0$.
\zadStop
\rozwStart{Patryk Wirkus}{Laura Mieczkowska}
Miejsca zerowe naszego wielomianu to: $1, 4, 7$.\\
Wielomian jest stopnia nieparzystego, ponadto znak współczynnika przy\linebreak najwyższej potędze x jest dodatni.\\ W związku z tym wykres wielomianu zaczyna się od lewej strony poniżej osi OX. A więc $$x \in [1,4] \cup [7,\infty).$$
\rozwStop
\odpStart
$x \in [1,4] \cup [7,\infty)$
\odpStop
\testStart
A.$x \in [1,4] \cup [7,\infty)$\\
B.$x \in (1,4) \cup [7,\infty)$\\
C.$x \in (1,4] \cup [7,\infty)$\\
D.$x \in [1,4) \cup [7,\infty)$\\
E.$x \in [1,4] \cup (7,\infty)$\\
F.$x \in (1,4) \cup (7,\infty)$\\
G.$x \in [1,4) \cup (7,\infty)$\\
H.$x \in (1,4] \cup (7,\infty)$
\testStop
\kluczStart
A
\kluczStop



\zadStart{Zadanie z Wikieł Z 1.62 a) moja wersja nr 29}

Rozwiązać nierówności $(x-1)(x-4)(x-8)\ge0$.
\zadStop
\rozwStart{Patryk Wirkus}{Laura Mieczkowska}
Miejsca zerowe naszego wielomianu to: $1, 4, 8$.\\
Wielomian jest stopnia nieparzystego, ponadto znak współczynnika przy\linebreak najwyższej potędze x jest dodatni.\\ W związku z tym wykres wielomianu zaczyna się od lewej strony poniżej osi OX. A więc $$x \in [1,4] \cup [8,\infty).$$
\rozwStop
\odpStart
$x \in [1,4] \cup [8,\infty)$
\odpStop
\testStart
A.$x \in [1,4] \cup [8,\infty)$\\
B.$x \in (1,4) \cup [8,\infty)$\\
C.$x \in (1,4] \cup [8,\infty)$\\
D.$x \in [1,4) \cup [8,\infty)$\\
E.$x \in [1,4] \cup (8,\infty)$\\
F.$x \in (1,4) \cup (8,\infty)$\\
G.$x \in [1,4) \cup (8,\infty)$\\
H.$x \in (1,4] \cup (8,\infty)$
\testStop
\kluczStart
A
\kluczStop



\zadStart{Zadanie z Wikieł Z 1.62 a) moja wersja nr 30}

Rozwiązać nierówności $(x-1)(x-4)(x-9)\ge0$.
\zadStop
\rozwStart{Patryk Wirkus}{Laura Mieczkowska}
Miejsca zerowe naszego wielomianu to: $1, 4, 9$.\\
Wielomian jest stopnia nieparzystego, ponadto znak współczynnika przy\linebreak najwyższej potędze x jest dodatni.\\ W związku z tym wykres wielomianu zaczyna się od lewej strony poniżej osi OX. A więc $$x \in [1,4] \cup [9,\infty).$$
\rozwStop
\odpStart
$x \in [1,4] \cup [9,\infty)$
\odpStop
\testStart
A.$x \in [1,4] \cup [9,\infty)$\\
B.$x \in (1,4) \cup [9,\infty)$\\
C.$x \in (1,4] \cup [9,\infty)$\\
D.$x \in [1,4) \cup [9,\infty)$\\
E.$x \in [1,4] \cup (9,\infty)$\\
F.$x \in (1,4) \cup (9,\infty)$\\
G.$x \in [1,4) \cup (9,\infty)$\\
H.$x \in (1,4] \cup (9,\infty)$
\testStop
\kluczStart
A
\kluczStop



\zadStart{Zadanie z Wikieł Z 1.62 a) moja wersja nr 31}

Rozwiązać nierówności $(x-1)(x-4)(x-10)\ge0$.
\zadStop
\rozwStart{Patryk Wirkus}{Laura Mieczkowska}
Miejsca zerowe naszego wielomianu to: $1, 4, 10$.\\
Wielomian jest stopnia nieparzystego, ponadto znak współczynnika przy\linebreak najwyższej potędze x jest dodatni.\\ W związku z tym wykres wielomianu zaczyna się od lewej strony poniżej osi OX. A więc $$x \in [1,4] \cup [10,\infty).$$
\rozwStop
\odpStart
$x \in [1,4] \cup [10,\infty)$
\odpStop
\testStart
A.$x \in [1,4] \cup [10,\infty)$\\
B.$x \in (1,4) \cup [10,\infty)$\\
C.$x \in (1,4] \cup [10,\infty)$\\
D.$x \in [1,4) \cup [10,\infty)$\\
E.$x \in [1,4] \cup (10,\infty)$\\
F.$x \in (1,4) \cup (10,\infty)$\\
G.$x \in [1,4) \cup (10,\infty)$\\
H.$x \in (1,4] \cup (10,\infty)$
\testStop
\kluczStart
A
\kluczStop



\zadStart{Zadanie z Wikieł Z 1.62 a) moja wersja nr 32}

Rozwiązać nierówności $(x-1)(x-4)(x-11)\ge0$.
\zadStop
\rozwStart{Patryk Wirkus}{Laura Mieczkowska}
Miejsca zerowe naszego wielomianu to: $1, 4, 11$.\\
Wielomian jest stopnia nieparzystego, ponadto znak współczynnika przy\linebreak najwyższej potędze x jest dodatni.\\ W związku z tym wykres wielomianu zaczyna się od lewej strony poniżej osi OX. A więc $$x \in [1,4] \cup [11,\infty).$$
\rozwStop
\odpStart
$x \in [1,4] \cup [11,\infty)$
\odpStop
\testStart
A.$x \in [1,4] \cup [11,\infty)$\\
B.$x \in (1,4) \cup [11,\infty)$\\
C.$x \in (1,4] \cup [11,\infty)$\\
D.$x \in [1,4) \cup [11,\infty)$\\
E.$x \in [1,4] \cup (11,\infty)$\\
F.$x \in (1,4) \cup (11,\infty)$\\
G.$x \in [1,4) \cup (11,\infty)$\\
H.$x \in (1,4] \cup (11,\infty)$
\testStop
\kluczStart
A
\kluczStop



\zadStart{Zadanie z Wikieł Z 1.62 a) moja wersja nr 33}

Rozwiązać nierówności $(x-1)(x-4)(x-12)\ge0$.
\zadStop
\rozwStart{Patryk Wirkus}{Laura Mieczkowska}
Miejsca zerowe naszego wielomianu to: $1, 4, 12$.\\
Wielomian jest stopnia nieparzystego, ponadto znak współczynnika przy\linebreak najwyższej potędze x jest dodatni.\\ W związku z tym wykres wielomianu zaczyna się od lewej strony poniżej osi OX. A więc $$x \in [1,4] \cup [12,\infty).$$
\rozwStop
\odpStart
$x \in [1,4] \cup [12,\infty)$
\odpStop
\testStart
A.$x \in [1,4] \cup [12,\infty)$\\
B.$x \in (1,4) \cup [12,\infty)$\\
C.$x \in (1,4] \cup [12,\infty)$\\
D.$x \in [1,4) \cup [12,\infty)$\\
E.$x \in [1,4] \cup (12,\infty)$\\
F.$x \in (1,4) \cup (12,\infty)$\\
G.$x \in [1,4) \cup (12,\infty)$\\
H.$x \in (1,4] \cup (12,\infty)$
\testStop
\kluczStart
A
\kluczStop



\zadStart{Zadanie z Wikieł Z 1.62 a) moja wersja nr 34}

Rozwiązać nierówności $(x-1)(x-4)(x-13)\ge0$.
\zadStop
\rozwStart{Patryk Wirkus}{Laura Mieczkowska}
Miejsca zerowe naszego wielomianu to: $1, 4, 13$.\\
Wielomian jest stopnia nieparzystego, ponadto znak współczynnika przy\linebreak najwyższej potędze x jest dodatni.\\ W związku z tym wykres wielomianu zaczyna się od lewej strony poniżej osi OX. A więc $$x \in [1,4] \cup [13,\infty).$$
\rozwStop
\odpStart
$x \in [1,4] \cup [13,\infty)$
\odpStop
\testStart
A.$x \in [1,4] \cup [13,\infty)$\\
B.$x \in (1,4) \cup [13,\infty)$\\
C.$x \in (1,4] \cup [13,\infty)$\\
D.$x \in [1,4) \cup [13,\infty)$\\
E.$x \in [1,4] \cup (13,\infty)$\\
F.$x \in (1,4) \cup (13,\infty)$\\
G.$x \in [1,4) \cup (13,\infty)$\\
H.$x \in (1,4] \cup (13,\infty)$
\testStop
\kluczStart
A
\kluczStop



\zadStart{Zadanie z Wikieł Z 1.62 a) moja wersja nr 35}

Rozwiązać nierówności $(x-1)(x-4)(x-14)\ge0$.
\zadStop
\rozwStart{Patryk Wirkus}{Laura Mieczkowska}
Miejsca zerowe naszego wielomianu to: $1, 4, 14$.\\
Wielomian jest stopnia nieparzystego, ponadto znak współczynnika przy\linebreak najwyższej potędze x jest dodatni.\\ W związku z tym wykres wielomianu zaczyna się od lewej strony poniżej osi OX. A więc $$x \in [1,4] \cup [14,\infty).$$
\rozwStop
\odpStart
$x \in [1,4] \cup [14,\infty)$
\odpStop
\testStart
A.$x \in [1,4] \cup [14,\infty)$\\
B.$x \in (1,4) \cup [14,\infty)$\\
C.$x \in (1,4] \cup [14,\infty)$\\
D.$x \in [1,4) \cup [14,\infty)$\\
E.$x \in [1,4] \cup (14,\infty)$\\
F.$x \in (1,4) \cup (14,\infty)$\\
G.$x \in [1,4) \cup (14,\infty)$\\
H.$x \in (1,4] \cup (14,\infty)$
\testStop
\kluczStart
A
\kluczStop



\zadStart{Zadanie z Wikieł Z 1.62 a) moja wersja nr 36}

Rozwiązać nierówności $(x-1)(x-4)(x-15)\ge0$.
\zadStop
\rozwStart{Patryk Wirkus}{Laura Mieczkowska}
Miejsca zerowe naszego wielomianu to: $1, 4, 15$.\\
Wielomian jest stopnia nieparzystego, ponadto znak współczynnika przy\linebreak najwyższej potędze x jest dodatni.\\ W związku z tym wykres wielomianu zaczyna się od lewej strony poniżej osi OX. A więc $$x \in [1,4] \cup [15,\infty).$$
\rozwStop
\odpStart
$x \in [1,4] \cup [15,\infty)$
\odpStop
\testStart
A.$x \in [1,4] \cup [15,\infty)$\\
B.$x \in (1,4) \cup [15,\infty)$\\
C.$x \in (1,4] \cup [15,\infty)$\\
D.$x \in [1,4) \cup [15,\infty)$\\
E.$x \in [1,4] \cup (15,\infty)$\\
F.$x \in (1,4) \cup (15,\infty)$\\
G.$x \in [1,4) \cup (15,\infty)$\\
H.$x \in (1,4] \cup (15,\infty)$
\testStop
\kluczStart
A
\kluczStop



\zadStart{Zadanie z Wikieł Z 1.62 a) moja wersja nr 37}

Rozwiązać nierówności $(x-1)(x-5)(x-6)\ge0$.
\zadStop
\rozwStart{Patryk Wirkus}{Laura Mieczkowska}
Miejsca zerowe naszego wielomianu to: $1, 5, 6$.\\
Wielomian jest stopnia nieparzystego, ponadto znak współczynnika przy\linebreak najwyższej potędze x jest dodatni.\\ W związku z tym wykres wielomianu zaczyna się od lewej strony poniżej osi OX. A więc $$x \in [1,5] \cup [6,\infty).$$
\rozwStop
\odpStart
$x \in [1,5] \cup [6,\infty)$
\odpStop
\testStart
A.$x \in [1,5] \cup [6,\infty)$\\
B.$x \in (1,5) \cup [6,\infty)$\\
C.$x \in (1,5] \cup [6,\infty)$\\
D.$x \in [1,5) \cup [6,\infty)$\\
E.$x \in [1,5] \cup (6,\infty)$\\
F.$x \in (1,5) \cup (6,\infty)$\\
G.$x \in [1,5) \cup (6,\infty)$\\
H.$x \in (1,5] \cup (6,\infty)$
\testStop
\kluczStart
A
\kluczStop



\zadStart{Zadanie z Wikieł Z 1.62 a) moja wersja nr 38}

Rozwiązać nierówności $(x-1)(x-5)(x-7)\ge0$.
\zadStop
\rozwStart{Patryk Wirkus}{Laura Mieczkowska}
Miejsca zerowe naszego wielomianu to: $1, 5, 7$.\\
Wielomian jest stopnia nieparzystego, ponadto znak współczynnika przy\linebreak najwyższej potędze x jest dodatni.\\ W związku z tym wykres wielomianu zaczyna się od lewej strony poniżej osi OX. A więc $$x \in [1,5] \cup [7,\infty).$$
\rozwStop
\odpStart
$x \in [1,5] \cup [7,\infty)$
\odpStop
\testStart
A.$x \in [1,5] \cup [7,\infty)$\\
B.$x \in (1,5) \cup [7,\infty)$\\
C.$x \in (1,5] \cup [7,\infty)$\\
D.$x \in [1,5) \cup [7,\infty)$\\
E.$x \in [1,5] \cup (7,\infty)$\\
F.$x \in (1,5) \cup (7,\infty)$\\
G.$x \in [1,5) \cup (7,\infty)$\\
H.$x \in (1,5] \cup (7,\infty)$
\testStop
\kluczStart
A
\kluczStop



\zadStart{Zadanie z Wikieł Z 1.62 a) moja wersja nr 39}

Rozwiązać nierówności $(x-1)(x-5)(x-8)\ge0$.
\zadStop
\rozwStart{Patryk Wirkus}{Laura Mieczkowska}
Miejsca zerowe naszego wielomianu to: $1, 5, 8$.\\
Wielomian jest stopnia nieparzystego, ponadto znak współczynnika przy\linebreak najwyższej potędze x jest dodatni.\\ W związku z tym wykres wielomianu zaczyna się od lewej strony poniżej osi OX. A więc $$x \in [1,5] \cup [8,\infty).$$
\rozwStop
\odpStart
$x \in [1,5] \cup [8,\infty)$
\odpStop
\testStart
A.$x \in [1,5] \cup [8,\infty)$\\
B.$x \in (1,5) \cup [8,\infty)$\\
C.$x \in (1,5] \cup [8,\infty)$\\
D.$x \in [1,5) \cup [8,\infty)$\\
E.$x \in [1,5] \cup (8,\infty)$\\
F.$x \in (1,5) \cup (8,\infty)$\\
G.$x \in [1,5) \cup (8,\infty)$\\
H.$x \in (1,5] \cup (8,\infty)$
\testStop
\kluczStart
A
\kluczStop



\zadStart{Zadanie z Wikieł Z 1.62 a) moja wersja nr 40}

Rozwiązać nierówności $(x-1)(x-5)(x-9)\ge0$.
\zadStop
\rozwStart{Patryk Wirkus}{Laura Mieczkowska}
Miejsca zerowe naszego wielomianu to: $1, 5, 9$.\\
Wielomian jest stopnia nieparzystego, ponadto znak współczynnika przy\linebreak najwyższej potędze x jest dodatni.\\ W związku z tym wykres wielomianu zaczyna się od lewej strony poniżej osi OX. A więc $$x \in [1,5] \cup [9,\infty).$$
\rozwStop
\odpStart
$x \in [1,5] \cup [9,\infty)$
\odpStop
\testStart
A.$x \in [1,5] \cup [9,\infty)$\\
B.$x \in (1,5) \cup [9,\infty)$\\
C.$x \in (1,5] \cup [9,\infty)$\\
D.$x \in [1,5) \cup [9,\infty)$\\
E.$x \in [1,5] \cup (9,\infty)$\\
F.$x \in (1,5) \cup (9,\infty)$\\
G.$x \in [1,5) \cup (9,\infty)$\\
H.$x \in (1,5] \cup (9,\infty)$
\testStop
\kluczStart
A
\kluczStop



\zadStart{Zadanie z Wikieł Z 1.62 a) moja wersja nr 41}

Rozwiązać nierówności $(x-1)(x-5)(x-10)\ge0$.
\zadStop
\rozwStart{Patryk Wirkus}{Laura Mieczkowska}
Miejsca zerowe naszego wielomianu to: $1, 5, 10$.\\
Wielomian jest stopnia nieparzystego, ponadto znak współczynnika przy\linebreak najwyższej potędze x jest dodatni.\\ W związku z tym wykres wielomianu zaczyna się od lewej strony poniżej osi OX. A więc $$x \in [1,5] \cup [10,\infty).$$
\rozwStop
\odpStart
$x \in [1,5] \cup [10,\infty)$
\odpStop
\testStart
A.$x \in [1,5] \cup [10,\infty)$\\
B.$x \in (1,5) \cup [10,\infty)$\\
C.$x \in (1,5] \cup [10,\infty)$\\
D.$x \in [1,5) \cup [10,\infty)$\\
E.$x \in [1,5] \cup (10,\infty)$\\
F.$x \in (1,5) \cup (10,\infty)$\\
G.$x \in [1,5) \cup (10,\infty)$\\
H.$x \in (1,5] \cup (10,\infty)$
\testStop
\kluczStart
A
\kluczStop



\zadStart{Zadanie z Wikieł Z 1.62 a) moja wersja nr 42}

Rozwiązać nierówności $(x-1)(x-5)(x-11)\ge0$.
\zadStop
\rozwStart{Patryk Wirkus}{Laura Mieczkowska}
Miejsca zerowe naszego wielomianu to: $1, 5, 11$.\\
Wielomian jest stopnia nieparzystego, ponadto znak współczynnika przy\linebreak najwyższej potędze x jest dodatni.\\ W związku z tym wykres wielomianu zaczyna się od lewej strony poniżej osi OX. A więc $$x \in [1,5] \cup [11,\infty).$$
\rozwStop
\odpStart
$x \in [1,5] \cup [11,\infty)$
\odpStop
\testStart
A.$x \in [1,5] \cup [11,\infty)$\\
B.$x \in (1,5) \cup [11,\infty)$\\
C.$x \in (1,5] \cup [11,\infty)$\\
D.$x \in [1,5) \cup [11,\infty)$\\
E.$x \in [1,5] \cup (11,\infty)$\\
F.$x \in (1,5) \cup (11,\infty)$\\
G.$x \in [1,5) \cup (11,\infty)$\\
H.$x \in (1,5] \cup (11,\infty)$
\testStop
\kluczStart
A
\kluczStop



\zadStart{Zadanie z Wikieł Z 1.62 a) moja wersja nr 43}

Rozwiązać nierówności $(x-1)(x-5)(x-12)\ge0$.
\zadStop
\rozwStart{Patryk Wirkus}{Laura Mieczkowska}
Miejsca zerowe naszego wielomianu to: $1, 5, 12$.\\
Wielomian jest stopnia nieparzystego, ponadto znak współczynnika przy\linebreak najwyższej potędze x jest dodatni.\\ W związku z tym wykres wielomianu zaczyna się od lewej strony poniżej osi OX. A więc $$x \in [1,5] \cup [12,\infty).$$
\rozwStop
\odpStart
$x \in [1,5] \cup [12,\infty)$
\odpStop
\testStart
A.$x \in [1,5] \cup [12,\infty)$\\
B.$x \in (1,5) \cup [12,\infty)$\\
C.$x \in (1,5] \cup [12,\infty)$\\
D.$x \in [1,5) \cup [12,\infty)$\\
E.$x \in [1,5] \cup (12,\infty)$\\
F.$x \in (1,5) \cup (12,\infty)$\\
G.$x \in [1,5) \cup (12,\infty)$\\
H.$x \in (1,5] \cup (12,\infty)$
\testStop
\kluczStart
A
\kluczStop



\zadStart{Zadanie z Wikieł Z 1.62 a) moja wersja nr 44}

Rozwiązać nierówności $(x-1)(x-5)(x-13)\ge0$.
\zadStop
\rozwStart{Patryk Wirkus}{Laura Mieczkowska}
Miejsca zerowe naszego wielomianu to: $1, 5, 13$.\\
Wielomian jest stopnia nieparzystego, ponadto znak współczynnika przy\linebreak najwyższej potędze x jest dodatni.\\ W związku z tym wykres wielomianu zaczyna się od lewej strony poniżej osi OX. A więc $$x \in [1,5] \cup [13,\infty).$$
\rozwStop
\odpStart
$x \in [1,5] \cup [13,\infty)$
\odpStop
\testStart
A.$x \in [1,5] \cup [13,\infty)$\\
B.$x \in (1,5) \cup [13,\infty)$\\
C.$x \in (1,5] \cup [13,\infty)$\\
D.$x \in [1,5) \cup [13,\infty)$\\
E.$x \in [1,5] \cup (13,\infty)$\\
F.$x \in (1,5) \cup (13,\infty)$\\
G.$x \in [1,5) \cup (13,\infty)$\\
H.$x \in (1,5] \cup (13,\infty)$
\testStop
\kluczStart
A
\kluczStop



\zadStart{Zadanie z Wikieł Z 1.62 a) moja wersja nr 45}

Rozwiązać nierówności $(x-1)(x-5)(x-14)\ge0$.
\zadStop
\rozwStart{Patryk Wirkus}{Laura Mieczkowska}
Miejsca zerowe naszego wielomianu to: $1, 5, 14$.\\
Wielomian jest stopnia nieparzystego, ponadto znak współczynnika przy\linebreak najwyższej potędze x jest dodatni.\\ W związku z tym wykres wielomianu zaczyna się od lewej strony poniżej osi OX. A więc $$x \in [1,5] \cup [14,\infty).$$
\rozwStop
\odpStart
$x \in [1,5] \cup [14,\infty)$
\odpStop
\testStart
A.$x \in [1,5] \cup [14,\infty)$\\
B.$x \in (1,5) \cup [14,\infty)$\\
C.$x \in (1,5] \cup [14,\infty)$\\
D.$x \in [1,5) \cup [14,\infty)$\\
E.$x \in [1,5] \cup (14,\infty)$\\
F.$x \in (1,5) \cup (14,\infty)$\\
G.$x \in [1,5) \cup (14,\infty)$\\
H.$x \in (1,5] \cup (14,\infty)$
\testStop
\kluczStart
A
\kluczStop



\zadStart{Zadanie z Wikieł Z 1.62 a) moja wersja nr 46}

Rozwiązać nierówności $(x-1)(x-5)(x-15)\ge0$.
\zadStop
\rozwStart{Patryk Wirkus}{Laura Mieczkowska}
Miejsca zerowe naszego wielomianu to: $1, 5, 15$.\\
Wielomian jest stopnia nieparzystego, ponadto znak współczynnika przy\linebreak najwyższej potędze x jest dodatni.\\ W związku z tym wykres wielomianu zaczyna się od lewej strony poniżej osi OX. A więc $$x \in [1,5] \cup [15,\infty).$$
\rozwStop
\odpStart
$x \in [1,5] \cup [15,\infty)$
\odpStop
\testStart
A.$x \in [1,5] \cup [15,\infty)$\\
B.$x \in (1,5) \cup [15,\infty)$\\
C.$x \in (1,5] \cup [15,\infty)$\\
D.$x \in [1,5) \cup [15,\infty)$\\
E.$x \in [1,5] \cup (15,\infty)$\\
F.$x \in (1,5) \cup (15,\infty)$\\
G.$x \in [1,5) \cup (15,\infty)$\\
H.$x \in (1,5] \cup (15,\infty)$
\testStop
\kluczStart
A
\kluczStop



\zadStart{Zadanie z Wikieł Z 1.62 a) moja wersja nr 47}

Rozwiązać nierówności $(x-1)(x-6)(x-7)\ge0$.
\zadStop
\rozwStart{Patryk Wirkus}{Laura Mieczkowska}
Miejsca zerowe naszego wielomianu to: $1, 6, 7$.\\
Wielomian jest stopnia nieparzystego, ponadto znak współczynnika przy\linebreak najwyższej potędze x jest dodatni.\\ W związku z tym wykres wielomianu zaczyna się od lewej strony poniżej osi OX. A więc $$x \in [1,6] \cup [7,\infty).$$
\rozwStop
\odpStart
$x \in [1,6] \cup [7,\infty)$
\odpStop
\testStart
A.$x \in [1,6] \cup [7,\infty)$\\
B.$x \in (1,6) \cup [7,\infty)$\\
C.$x \in (1,6] \cup [7,\infty)$\\
D.$x \in [1,6) \cup [7,\infty)$\\
E.$x \in [1,6] \cup (7,\infty)$\\
F.$x \in (1,6) \cup (7,\infty)$\\
G.$x \in [1,6) \cup (7,\infty)$\\
H.$x \in (1,6] \cup (7,\infty)$
\testStop
\kluczStart
A
\kluczStop



\zadStart{Zadanie z Wikieł Z 1.62 a) moja wersja nr 48}

Rozwiązać nierówności $(x-1)(x-6)(x-8)\ge0$.
\zadStop
\rozwStart{Patryk Wirkus}{Laura Mieczkowska}
Miejsca zerowe naszego wielomianu to: $1, 6, 8$.\\
Wielomian jest stopnia nieparzystego, ponadto znak współczynnika przy\linebreak najwyższej potędze x jest dodatni.\\ W związku z tym wykres wielomianu zaczyna się od lewej strony poniżej osi OX. A więc $$x \in [1,6] \cup [8,\infty).$$
\rozwStop
\odpStart
$x \in [1,6] \cup [8,\infty)$
\odpStop
\testStart
A.$x \in [1,6] \cup [8,\infty)$\\
B.$x \in (1,6) \cup [8,\infty)$\\
C.$x \in (1,6] \cup [8,\infty)$\\
D.$x \in [1,6) \cup [8,\infty)$\\
E.$x \in [1,6] \cup (8,\infty)$\\
F.$x \in (1,6) \cup (8,\infty)$\\
G.$x \in [1,6) \cup (8,\infty)$\\
H.$x \in (1,6] \cup (8,\infty)$
\testStop
\kluczStart
A
\kluczStop



\zadStart{Zadanie z Wikieł Z 1.62 a) moja wersja nr 49}

Rozwiązać nierówności $(x-1)(x-6)(x-9)\ge0$.
\zadStop
\rozwStart{Patryk Wirkus}{Laura Mieczkowska}
Miejsca zerowe naszego wielomianu to: $1, 6, 9$.\\
Wielomian jest stopnia nieparzystego, ponadto znak współczynnika przy\linebreak najwyższej potędze x jest dodatni.\\ W związku z tym wykres wielomianu zaczyna się od lewej strony poniżej osi OX. A więc $$x \in [1,6] \cup [9,\infty).$$
\rozwStop
\odpStart
$x \in [1,6] \cup [9,\infty)$
\odpStop
\testStart
A.$x \in [1,6] \cup [9,\infty)$\\
B.$x \in (1,6) \cup [9,\infty)$\\
C.$x \in (1,6] \cup [9,\infty)$\\
D.$x \in [1,6) \cup [9,\infty)$\\
E.$x \in [1,6] \cup (9,\infty)$\\
F.$x \in (1,6) \cup (9,\infty)$\\
G.$x \in [1,6) \cup (9,\infty)$\\
H.$x \in (1,6] \cup (9,\infty)$
\testStop
\kluczStart
A
\kluczStop



\zadStart{Zadanie z Wikieł Z 1.62 a) moja wersja nr 50}

Rozwiązać nierówności $(x-1)(x-6)(x-10)\ge0$.
\zadStop
\rozwStart{Patryk Wirkus}{Laura Mieczkowska}
Miejsca zerowe naszego wielomianu to: $1, 6, 10$.\\
Wielomian jest stopnia nieparzystego, ponadto znak współczynnika przy\linebreak najwyższej potędze x jest dodatni.\\ W związku z tym wykres wielomianu zaczyna się od lewej strony poniżej osi OX. A więc $$x \in [1,6] \cup [10,\infty).$$
\rozwStop
\odpStart
$x \in [1,6] \cup [10,\infty)$
\odpStop
\testStart
A.$x \in [1,6] \cup [10,\infty)$\\
B.$x \in (1,6) \cup [10,\infty)$\\
C.$x \in (1,6] \cup [10,\infty)$\\
D.$x \in [1,6) \cup [10,\infty)$\\
E.$x \in [1,6] \cup (10,\infty)$\\
F.$x \in (1,6) \cup (10,\infty)$\\
G.$x \in [1,6) \cup (10,\infty)$\\
H.$x \in (1,6] \cup (10,\infty)$
\testStop
\kluczStart
A
\kluczStop



\zadStart{Zadanie z Wikieł Z 1.62 a) moja wersja nr 51}

Rozwiązać nierówności $(x-1)(x-6)(x-11)\ge0$.
\zadStop
\rozwStart{Patryk Wirkus}{Laura Mieczkowska}
Miejsca zerowe naszego wielomianu to: $1, 6, 11$.\\
Wielomian jest stopnia nieparzystego, ponadto znak współczynnika przy\linebreak najwyższej potędze x jest dodatni.\\ W związku z tym wykres wielomianu zaczyna się od lewej strony poniżej osi OX. A więc $$x \in [1,6] \cup [11,\infty).$$
\rozwStop
\odpStart
$x \in [1,6] \cup [11,\infty)$
\odpStop
\testStart
A.$x \in [1,6] \cup [11,\infty)$\\
B.$x \in (1,6) \cup [11,\infty)$\\
C.$x \in (1,6] \cup [11,\infty)$\\
D.$x \in [1,6) \cup [11,\infty)$\\
E.$x \in [1,6] \cup (11,\infty)$\\
F.$x \in (1,6) \cup (11,\infty)$\\
G.$x \in [1,6) \cup (11,\infty)$\\
H.$x \in (1,6] \cup (11,\infty)$
\testStop
\kluczStart
A
\kluczStop



\zadStart{Zadanie z Wikieł Z 1.62 a) moja wersja nr 52}

Rozwiązać nierówności $(x-1)(x-6)(x-12)\ge0$.
\zadStop
\rozwStart{Patryk Wirkus}{Laura Mieczkowska}
Miejsca zerowe naszego wielomianu to: $1, 6, 12$.\\
Wielomian jest stopnia nieparzystego, ponadto znak współczynnika przy\linebreak najwyższej potędze x jest dodatni.\\ W związku z tym wykres wielomianu zaczyna się od lewej strony poniżej osi OX. A więc $$x \in [1,6] \cup [12,\infty).$$
\rozwStop
\odpStart
$x \in [1,6] \cup [12,\infty)$
\odpStop
\testStart
A.$x \in [1,6] \cup [12,\infty)$\\
B.$x \in (1,6) \cup [12,\infty)$\\
C.$x \in (1,6] \cup [12,\infty)$\\
D.$x \in [1,6) \cup [12,\infty)$\\
E.$x \in [1,6] \cup (12,\infty)$\\
F.$x \in (1,6) \cup (12,\infty)$\\
G.$x \in [1,6) \cup (12,\infty)$\\
H.$x \in (1,6] \cup (12,\infty)$
\testStop
\kluczStart
A
\kluczStop



\zadStart{Zadanie z Wikieł Z 1.62 a) moja wersja nr 53}

Rozwiązać nierówności $(x-1)(x-6)(x-13)\ge0$.
\zadStop
\rozwStart{Patryk Wirkus}{Laura Mieczkowska}
Miejsca zerowe naszego wielomianu to: $1, 6, 13$.\\
Wielomian jest stopnia nieparzystego, ponadto znak współczynnika przy\linebreak najwyższej potędze x jest dodatni.\\ W związku z tym wykres wielomianu zaczyna się od lewej strony poniżej osi OX. A więc $$x \in [1,6] \cup [13,\infty).$$
\rozwStop
\odpStart
$x \in [1,6] \cup [13,\infty)$
\odpStop
\testStart
A.$x \in [1,6] \cup [13,\infty)$\\
B.$x \in (1,6) \cup [13,\infty)$\\
C.$x \in (1,6] \cup [13,\infty)$\\
D.$x \in [1,6) \cup [13,\infty)$\\
E.$x \in [1,6] \cup (13,\infty)$\\
F.$x \in (1,6) \cup (13,\infty)$\\
G.$x \in [1,6) \cup (13,\infty)$\\
H.$x \in (1,6] \cup (13,\infty)$
\testStop
\kluczStart
A
\kluczStop



\zadStart{Zadanie z Wikieł Z 1.62 a) moja wersja nr 54}

Rozwiązać nierówności $(x-1)(x-6)(x-14)\ge0$.
\zadStop
\rozwStart{Patryk Wirkus}{Laura Mieczkowska}
Miejsca zerowe naszego wielomianu to: $1, 6, 14$.\\
Wielomian jest stopnia nieparzystego, ponadto znak współczynnika przy\linebreak najwyższej potędze x jest dodatni.\\ W związku z tym wykres wielomianu zaczyna się od lewej strony poniżej osi OX. A więc $$x \in [1,6] \cup [14,\infty).$$
\rozwStop
\odpStart
$x \in [1,6] \cup [14,\infty)$
\odpStop
\testStart
A.$x \in [1,6] \cup [14,\infty)$\\
B.$x \in (1,6) \cup [14,\infty)$\\
C.$x \in (1,6] \cup [14,\infty)$\\
D.$x \in [1,6) \cup [14,\infty)$\\
E.$x \in [1,6] \cup (14,\infty)$\\
F.$x \in (1,6) \cup (14,\infty)$\\
G.$x \in [1,6) \cup (14,\infty)$\\
H.$x \in (1,6] \cup (14,\infty)$
\testStop
\kluczStart
A
\kluczStop



\zadStart{Zadanie z Wikieł Z 1.62 a) moja wersja nr 55}

Rozwiązać nierówności $(x-1)(x-6)(x-15)\ge0$.
\zadStop
\rozwStart{Patryk Wirkus}{Laura Mieczkowska}
Miejsca zerowe naszego wielomianu to: $1, 6, 15$.\\
Wielomian jest stopnia nieparzystego, ponadto znak współczynnika przy\linebreak najwyższej potędze x jest dodatni.\\ W związku z tym wykres wielomianu zaczyna się od lewej strony poniżej osi OX. A więc $$x \in [1,6] \cup [15,\infty).$$
\rozwStop
\odpStart
$x \in [1,6] \cup [15,\infty)$
\odpStop
\testStart
A.$x \in [1,6] \cup [15,\infty)$\\
B.$x \in (1,6) \cup [15,\infty)$\\
C.$x \in (1,6] \cup [15,\infty)$\\
D.$x \in [1,6) \cup [15,\infty)$\\
E.$x \in [1,6] \cup (15,\infty)$\\
F.$x \in (1,6) \cup (15,\infty)$\\
G.$x \in [1,6) \cup (15,\infty)$\\
H.$x \in (1,6] \cup (15,\infty)$
\testStop
\kluczStart
A
\kluczStop



\zadStart{Zadanie z Wikieł Z 1.62 a) moja wersja nr 56}

Rozwiązać nierówności $(x-1)(x-7)(x-8)\ge0$.
\zadStop
\rozwStart{Patryk Wirkus}{Laura Mieczkowska}
Miejsca zerowe naszego wielomianu to: $1, 7, 8$.\\
Wielomian jest stopnia nieparzystego, ponadto znak współczynnika przy\linebreak najwyższej potędze x jest dodatni.\\ W związku z tym wykres wielomianu zaczyna się od lewej strony poniżej osi OX. A więc $$x \in [1,7] \cup [8,\infty).$$
\rozwStop
\odpStart
$x \in [1,7] \cup [8,\infty)$
\odpStop
\testStart
A.$x \in [1,7] \cup [8,\infty)$\\
B.$x \in (1,7) \cup [8,\infty)$\\
C.$x \in (1,7] \cup [8,\infty)$\\
D.$x \in [1,7) \cup [8,\infty)$\\
E.$x \in [1,7] \cup (8,\infty)$\\
F.$x \in (1,7) \cup (8,\infty)$\\
G.$x \in [1,7) \cup (8,\infty)$\\
H.$x \in (1,7] \cup (8,\infty)$
\testStop
\kluczStart
A
\kluczStop



\zadStart{Zadanie z Wikieł Z 1.62 a) moja wersja nr 57}

Rozwiązać nierówności $(x-1)(x-7)(x-9)\ge0$.
\zadStop
\rozwStart{Patryk Wirkus}{Laura Mieczkowska}
Miejsca zerowe naszego wielomianu to: $1, 7, 9$.\\
Wielomian jest stopnia nieparzystego, ponadto znak współczynnika przy\linebreak najwyższej potędze x jest dodatni.\\ W związku z tym wykres wielomianu zaczyna się od lewej strony poniżej osi OX. A więc $$x \in [1,7] \cup [9,\infty).$$
\rozwStop
\odpStart
$x \in [1,7] \cup [9,\infty)$
\odpStop
\testStart
A.$x \in [1,7] \cup [9,\infty)$\\
B.$x \in (1,7) \cup [9,\infty)$\\
C.$x \in (1,7] \cup [9,\infty)$\\
D.$x \in [1,7) \cup [9,\infty)$\\
E.$x \in [1,7] \cup (9,\infty)$\\
F.$x \in (1,7) \cup (9,\infty)$\\
G.$x \in [1,7) \cup (9,\infty)$\\
H.$x \in (1,7] \cup (9,\infty)$
\testStop
\kluczStart
A
\kluczStop



\zadStart{Zadanie z Wikieł Z 1.62 a) moja wersja nr 58}

Rozwiązać nierówności $(x-1)(x-7)(x-10)\ge0$.
\zadStop
\rozwStart{Patryk Wirkus}{Laura Mieczkowska}
Miejsca zerowe naszego wielomianu to: $1, 7, 10$.\\
Wielomian jest stopnia nieparzystego, ponadto znak współczynnika przy\linebreak najwyższej potędze x jest dodatni.\\ W związku z tym wykres wielomianu zaczyna się od lewej strony poniżej osi OX. A więc $$x \in [1,7] \cup [10,\infty).$$
\rozwStop
\odpStart
$x \in [1,7] \cup [10,\infty)$
\odpStop
\testStart
A.$x \in [1,7] \cup [10,\infty)$\\
B.$x \in (1,7) \cup [10,\infty)$\\
C.$x \in (1,7] \cup [10,\infty)$\\
D.$x \in [1,7) \cup [10,\infty)$\\
E.$x \in [1,7] \cup (10,\infty)$\\
F.$x \in (1,7) \cup (10,\infty)$\\
G.$x \in [1,7) \cup (10,\infty)$\\
H.$x \in (1,7] \cup (10,\infty)$
\testStop
\kluczStart
A
\kluczStop



\zadStart{Zadanie z Wikieł Z 1.62 a) moja wersja nr 59}

Rozwiązać nierówności $(x-1)(x-7)(x-11)\ge0$.
\zadStop
\rozwStart{Patryk Wirkus}{Laura Mieczkowska}
Miejsca zerowe naszego wielomianu to: $1, 7, 11$.\\
Wielomian jest stopnia nieparzystego, ponadto znak współczynnika przy\linebreak najwyższej potędze x jest dodatni.\\ W związku z tym wykres wielomianu zaczyna się od lewej strony poniżej osi OX. A więc $$x \in [1,7] \cup [11,\infty).$$
\rozwStop
\odpStart
$x \in [1,7] \cup [11,\infty)$
\odpStop
\testStart
A.$x \in [1,7] \cup [11,\infty)$\\
B.$x \in (1,7) \cup [11,\infty)$\\
C.$x \in (1,7] \cup [11,\infty)$\\
D.$x \in [1,7) \cup [11,\infty)$\\
E.$x \in [1,7] \cup (11,\infty)$\\
F.$x \in (1,7) \cup (11,\infty)$\\
G.$x \in [1,7) \cup (11,\infty)$\\
H.$x \in (1,7] \cup (11,\infty)$
\testStop
\kluczStart
A
\kluczStop



\zadStart{Zadanie z Wikieł Z 1.62 a) moja wersja nr 60}

Rozwiązać nierówności $(x-1)(x-7)(x-12)\ge0$.
\zadStop
\rozwStart{Patryk Wirkus}{Laura Mieczkowska}
Miejsca zerowe naszego wielomianu to: $1, 7, 12$.\\
Wielomian jest stopnia nieparzystego, ponadto znak współczynnika przy\linebreak najwyższej potędze x jest dodatni.\\ W związku z tym wykres wielomianu zaczyna się od lewej strony poniżej osi OX. A więc $$x \in [1,7] \cup [12,\infty).$$
\rozwStop
\odpStart
$x \in [1,7] \cup [12,\infty)$
\odpStop
\testStart
A.$x \in [1,7] \cup [12,\infty)$\\
B.$x \in (1,7) \cup [12,\infty)$\\
C.$x \in (1,7] \cup [12,\infty)$\\
D.$x \in [1,7) \cup [12,\infty)$\\
E.$x \in [1,7] \cup (12,\infty)$\\
F.$x \in (1,7) \cup (12,\infty)$\\
G.$x \in [1,7) \cup (12,\infty)$\\
H.$x \in (1,7] \cup (12,\infty)$
\testStop
\kluczStart
A
\kluczStop



\zadStart{Zadanie z Wikieł Z 1.62 a) moja wersja nr 61}

Rozwiązać nierówności $(x-1)(x-7)(x-13)\ge0$.
\zadStop
\rozwStart{Patryk Wirkus}{Laura Mieczkowska}
Miejsca zerowe naszego wielomianu to: $1, 7, 13$.\\
Wielomian jest stopnia nieparzystego, ponadto znak współczynnika przy\linebreak najwyższej potędze x jest dodatni.\\ W związku z tym wykres wielomianu zaczyna się od lewej strony poniżej osi OX. A więc $$x \in [1,7] \cup [13,\infty).$$
\rozwStop
\odpStart
$x \in [1,7] \cup [13,\infty)$
\odpStop
\testStart
A.$x \in [1,7] \cup [13,\infty)$\\
B.$x \in (1,7) \cup [13,\infty)$\\
C.$x \in (1,7] \cup [13,\infty)$\\
D.$x \in [1,7) \cup [13,\infty)$\\
E.$x \in [1,7] \cup (13,\infty)$\\
F.$x \in (1,7) \cup (13,\infty)$\\
G.$x \in [1,7) \cup (13,\infty)$\\
H.$x \in (1,7] \cup (13,\infty)$
\testStop
\kluczStart
A
\kluczStop



\zadStart{Zadanie z Wikieł Z 1.62 a) moja wersja nr 62}

Rozwiązać nierówności $(x-1)(x-7)(x-14)\ge0$.
\zadStop
\rozwStart{Patryk Wirkus}{Laura Mieczkowska}
Miejsca zerowe naszego wielomianu to: $1, 7, 14$.\\
Wielomian jest stopnia nieparzystego, ponadto znak współczynnika przy\linebreak najwyższej potędze x jest dodatni.\\ W związku z tym wykres wielomianu zaczyna się od lewej strony poniżej osi OX. A więc $$x \in [1,7] \cup [14,\infty).$$
\rozwStop
\odpStart
$x \in [1,7] \cup [14,\infty)$
\odpStop
\testStart
A.$x \in [1,7] \cup [14,\infty)$\\
B.$x \in (1,7) \cup [14,\infty)$\\
C.$x \in (1,7] \cup [14,\infty)$\\
D.$x \in [1,7) \cup [14,\infty)$\\
E.$x \in [1,7] \cup (14,\infty)$\\
F.$x \in (1,7) \cup (14,\infty)$\\
G.$x \in [1,7) \cup (14,\infty)$\\
H.$x \in (1,7] \cup (14,\infty)$
\testStop
\kluczStart
A
\kluczStop



\zadStart{Zadanie z Wikieł Z 1.62 a) moja wersja nr 63}

Rozwiązać nierówności $(x-1)(x-7)(x-15)\ge0$.
\zadStop
\rozwStart{Patryk Wirkus}{Laura Mieczkowska}
Miejsca zerowe naszego wielomianu to: $1, 7, 15$.\\
Wielomian jest stopnia nieparzystego, ponadto znak współczynnika przy\linebreak najwyższej potędze x jest dodatni.\\ W związku z tym wykres wielomianu zaczyna się od lewej strony poniżej osi OX. A więc $$x \in [1,7] \cup [15,\infty).$$
\rozwStop
\odpStart
$x \in [1,7] \cup [15,\infty)$
\odpStop
\testStart
A.$x \in [1,7] \cup [15,\infty)$\\
B.$x \in (1,7) \cup [15,\infty)$\\
C.$x \in (1,7] \cup [15,\infty)$\\
D.$x \in [1,7) \cup [15,\infty)$\\
E.$x \in [1,7] \cup (15,\infty)$\\
F.$x \in (1,7) \cup (15,\infty)$\\
G.$x \in [1,7) \cup (15,\infty)$\\
H.$x \in (1,7] \cup (15,\infty)$
\testStop
\kluczStart
A
\kluczStop



\zadStart{Zadanie z Wikieł Z 1.62 a) moja wersja nr 64}

Rozwiązać nierówności $(x-1)(x-8)(x-9)\ge0$.
\zadStop
\rozwStart{Patryk Wirkus}{Laura Mieczkowska}
Miejsca zerowe naszego wielomianu to: $1, 8, 9$.\\
Wielomian jest stopnia nieparzystego, ponadto znak współczynnika przy\linebreak najwyższej potędze x jest dodatni.\\ W związku z tym wykres wielomianu zaczyna się od lewej strony poniżej osi OX. A więc $$x \in [1,8] \cup [9,\infty).$$
\rozwStop
\odpStart
$x \in [1,8] \cup [9,\infty)$
\odpStop
\testStart
A.$x \in [1,8] \cup [9,\infty)$\\
B.$x \in (1,8) \cup [9,\infty)$\\
C.$x \in (1,8] \cup [9,\infty)$\\
D.$x \in [1,8) \cup [9,\infty)$\\
E.$x \in [1,8] \cup (9,\infty)$\\
F.$x \in (1,8) \cup (9,\infty)$\\
G.$x \in [1,8) \cup (9,\infty)$\\
H.$x \in (1,8] \cup (9,\infty)$
\testStop
\kluczStart
A
\kluczStop



\zadStart{Zadanie z Wikieł Z 1.62 a) moja wersja nr 65}

Rozwiązać nierówności $(x-1)(x-8)(x-10)\ge0$.
\zadStop
\rozwStart{Patryk Wirkus}{Laura Mieczkowska}
Miejsca zerowe naszego wielomianu to: $1, 8, 10$.\\
Wielomian jest stopnia nieparzystego, ponadto znak współczynnika przy\linebreak najwyższej potędze x jest dodatni.\\ W związku z tym wykres wielomianu zaczyna się od lewej strony poniżej osi OX. A więc $$x \in [1,8] \cup [10,\infty).$$
\rozwStop
\odpStart
$x \in [1,8] \cup [10,\infty)$
\odpStop
\testStart
A.$x \in [1,8] \cup [10,\infty)$\\
B.$x \in (1,8) \cup [10,\infty)$\\
C.$x \in (1,8] \cup [10,\infty)$\\
D.$x \in [1,8) \cup [10,\infty)$\\
E.$x \in [1,8] \cup (10,\infty)$\\
F.$x \in (1,8) \cup (10,\infty)$\\
G.$x \in [1,8) \cup (10,\infty)$\\
H.$x \in (1,8] \cup (10,\infty)$
\testStop
\kluczStart
A
\kluczStop



\zadStart{Zadanie z Wikieł Z 1.62 a) moja wersja nr 66}

Rozwiązać nierówności $(x-1)(x-8)(x-11)\ge0$.
\zadStop
\rozwStart{Patryk Wirkus}{Laura Mieczkowska}
Miejsca zerowe naszego wielomianu to: $1, 8, 11$.\\
Wielomian jest stopnia nieparzystego, ponadto znak współczynnika przy\linebreak najwyższej potędze x jest dodatni.\\ W związku z tym wykres wielomianu zaczyna się od lewej strony poniżej osi OX. A więc $$x \in [1,8] \cup [11,\infty).$$
\rozwStop
\odpStart
$x \in [1,8] \cup [11,\infty)$
\odpStop
\testStart
A.$x \in [1,8] \cup [11,\infty)$\\
B.$x \in (1,8) \cup [11,\infty)$\\
C.$x \in (1,8] \cup [11,\infty)$\\
D.$x \in [1,8) \cup [11,\infty)$\\
E.$x \in [1,8] \cup (11,\infty)$\\
F.$x \in (1,8) \cup (11,\infty)$\\
G.$x \in [1,8) \cup (11,\infty)$\\
H.$x \in (1,8] \cup (11,\infty)$
\testStop
\kluczStart
A
\kluczStop



\zadStart{Zadanie z Wikieł Z 1.62 a) moja wersja nr 67}

Rozwiązać nierówności $(x-1)(x-8)(x-12)\ge0$.
\zadStop
\rozwStart{Patryk Wirkus}{Laura Mieczkowska}
Miejsca zerowe naszego wielomianu to: $1, 8, 12$.\\
Wielomian jest stopnia nieparzystego, ponadto znak współczynnika przy\linebreak najwyższej potędze x jest dodatni.\\ W związku z tym wykres wielomianu zaczyna się od lewej strony poniżej osi OX. A więc $$x \in [1,8] \cup [12,\infty).$$
\rozwStop
\odpStart
$x \in [1,8] \cup [12,\infty)$
\odpStop
\testStart
A.$x \in [1,8] \cup [12,\infty)$\\
B.$x \in (1,8) \cup [12,\infty)$\\
C.$x \in (1,8] \cup [12,\infty)$\\
D.$x \in [1,8) \cup [12,\infty)$\\
E.$x \in [1,8] \cup (12,\infty)$\\
F.$x \in (1,8) \cup (12,\infty)$\\
G.$x \in [1,8) \cup (12,\infty)$\\
H.$x \in (1,8] \cup (12,\infty)$
\testStop
\kluczStart
A
\kluczStop



\zadStart{Zadanie z Wikieł Z 1.62 a) moja wersja nr 68}

Rozwiązać nierówności $(x-1)(x-8)(x-13)\ge0$.
\zadStop
\rozwStart{Patryk Wirkus}{Laura Mieczkowska}
Miejsca zerowe naszego wielomianu to: $1, 8, 13$.\\
Wielomian jest stopnia nieparzystego, ponadto znak współczynnika przy\linebreak najwyższej potędze x jest dodatni.\\ W związku z tym wykres wielomianu zaczyna się od lewej strony poniżej osi OX. A więc $$x \in [1,8] \cup [13,\infty).$$
\rozwStop
\odpStart
$x \in [1,8] \cup [13,\infty)$
\odpStop
\testStart
A.$x \in [1,8] \cup [13,\infty)$\\
B.$x \in (1,8) \cup [13,\infty)$\\
C.$x \in (1,8] \cup [13,\infty)$\\
D.$x \in [1,8) \cup [13,\infty)$\\
E.$x \in [1,8] \cup (13,\infty)$\\
F.$x \in (1,8) \cup (13,\infty)$\\
G.$x \in [1,8) \cup (13,\infty)$\\
H.$x \in (1,8] \cup (13,\infty)$
\testStop
\kluczStart
A
\kluczStop



\zadStart{Zadanie z Wikieł Z 1.62 a) moja wersja nr 69}

Rozwiązać nierówności $(x-1)(x-8)(x-14)\ge0$.
\zadStop
\rozwStart{Patryk Wirkus}{Laura Mieczkowska}
Miejsca zerowe naszego wielomianu to: $1, 8, 14$.\\
Wielomian jest stopnia nieparzystego, ponadto znak współczynnika przy\linebreak najwyższej potędze x jest dodatni.\\ W związku z tym wykres wielomianu zaczyna się od lewej strony poniżej osi OX. A więc $$x \in [1,8] \cup [14,\infty).$$
\rozwStop
\odpStart
$x \in [1,8] \cup [14,\infty)$
\odpStop
\testStart
A.$x \in [1,8] \cup [14,\infty)$\\
B.$x \in (1,8) \cup [14,\infty)$\\
C.$x \in (1,8] \cup [14,\infty)$\\
D.$x \in [1,8) \cup [14,\infty)$\\
E.$x \in [1,8] \cup (14,\infty)$\\
F.$x \in (1,8) \cup (14,\infty)$\\
G.$x \in [1,8) \cup (14,\infty)$\\
H.$x \in (1,8] \cup (14,\infty)$
\testStop
\kluczStart
A
\kluczStop



\zadStart{Zadanie z Wikieł Z 1.62 a) moja wersja nr 70}

Rozwiązać nierówności $(x-1)(x-8)(x-15)\ge0$.
\zadStop
\rozwStart{Patryk Wirkus}{Laura Mieczkowska}
Miejsca zerowe naszego wielomianu to: $1, 8, 15$.\\
Wielomian jest stopnia nieparzystego, ponadto znak współczynnika przy\linebreak najwyższej potędze x jest dodatni.\\ W związku z tym wykres wielomianu zaczyna się od lewej strony poniżej osi OX. A więc $$x \in [1,8] \cup [15,\infty).$$
\rozwStop
\odpStart
$x \in [1,8] \cup [15,\infty)$
\odpStop
\testStart
A.$x \in [1,8] \cup [15,\infty)$\\
B.$x \in (1,8) \cup [15,\infty)$\\
C.$x \in (1,8] \cup [15,\infty)$\\
D.$x \in [1,8) \cup [15,\infty)$\\
E.$x \in [1,8] \cup (15,\infty)$\\
F.$x \in (1,8) \cup (15,\infty)$\\
G.$x \in [1,8) \cup (15,\infty)$\\
H.$x \in (1,8] \cup (15,\infty)$
\testStop
\kluczStart
A
\kluczStop



\zadStart{Zadanie z Wikieł Z 1.62 a) moja wersja nr 71}

Rozwiązać nierówności $(x-1)(x-9)(x-10)\ge0$.
\zadStop
\rozwStart{Patryk Wirkus}{Laura Mieczkowska}
Miejsca zerowe naszego wielomianu to: $1, 9, 10$.\\
Wielomian jest stopnia nieparzystego, ponadto znak współczynnika przy\linebreak najwyższej potędze x jest dodatni.\\ W związku z tym wykres wielomianu zaczyna się od lewej strony poniżej osi OX. A więc $$x \in [1,9] \cup [10,\infty).$$
\rozwStop
\odpStart
$x \in [1,9] \cup [10,\infty)$
\odpStop
\testStart
A.$x \in [1,9] \cup [10,\infty)$\\
B.$x \in (1,9) \cup [10,\infty)$\\
C.$x \in (1,9] \cup [10,\infty)$\\
D.$x \in [1,9) \cup [10,\infty)$\\
E.$x \in [1,9] \cup (10,\infty)$\\
F.$x \in (1,9) \cup (10,\infty)$\\
G.$x \in [1,9) \cup (10,\infty)$\\
H.$x \in (1,9] \cup (10,\infty)$
\testStop
\kluczStart
A
\kluczStop



\zadStart{Zadanie z Wikieł Z 1.62 a) moja wersja nr 72}

Rozwiązać nierówności $(x-1)(x-9)(x-11)\ge0$.
\zadStop
\rozwStart{Patryk Wirkus}{Laura Mieczkowska}
Miejsca zerowe naszego wielomianu to: $1, 9, 11$.\\
Wielomian jest stopnia nieparzystego, ponadto znak współczynnika przy\linebreak najwyższej potędze x jest dodatni.\\ W związku z tym wykres wielomianu zaczyna się od lewej strony poniżej osi OX. A więc $$x \in [1,9] \cup [11,\infty).$$
\rozwStop
\odpStart
$x \in [1,9] \cup [11,\infty)$
\odpStop
\testStart
A.$x \in [1,9] \cup [11,\infty)$\\
B.$x \in (1,9) \cup [11,\infty)$\\
C.$x \in (1,9] \cup [11,\infty)$\\
D.$x \in [1,9) \cup [11,\infty)$\\
E.$x \in [1,9] \cup (11,\infty)$\\
F.$x \in (1,9) \cup (11,\infty)$\\
G.$x \in [1,9) \cup (11,\infty)$\\
H.$x \in (1,9] \cup (11,\infty)$
\testStop
\kluczStart
A
\kluczStop



\zadStart{Zadanie z Wikieł Z 1.62 a) moja wersja nr 73}

Rozwiązać nierówności $(x-1)(x-9)(x-12)\ge0$.
\zadStop
\rozwStart{Patryk Wirkus}{Laura Mieczkowska}
Miejsca zerowe naszego wielomianu to: $1, 9, 12$.\\
Wielomian jest stopnia nieparzystego, ponadto znak współczynnika przy\linebreak najwyższej potędze x jest dodatni.\\ W związku z tym wykres wielomianu zaczyna się od lewej strony poniżej osi OX. A więc $$x \in [1,9] \cup [12,\infty).$$
\rozwStop
\odpStart
$x \in [1,9] \cup [12,\infty)$
\odpStop
\testStart
A.$x \in [1,9] \cup [12,\infty)$\\
B.$x \in (1,9) \cup [12,\infty)$\\
C.$x \in (1,9] \cup [12,\infty)$\\
D.$x \in [1,9) \cup [12,\infty)$\\
E.$x \in [1,9] \cup (12,\infty)$\\
F.$x \in (1,9) \cup (12,\infty)$\\
G.$x \in [1,9) \cup (12,\infty)$\\
H.$x \in (1,9] \cup (12,\infty)$
\testStop
\kluczStart
A
\kluczStop



\zadStart{Zadanie z Wikieł Z 1.62 a) moja wersja nr 74}

Rozwiązać nierówności $(x-1)(x-9)(x-13)\ge0$.
\zadStop
\rozwStart{Patryk Wirkus}{Laura Mieczkowska}
Miejsca zerowe naszego wielomianu to: $1, 9, 13$.\\
Wielomian jest stopnia nieparzystego, ponadto znak współczynnika przy\linebreak najwyższej potędze x jest dodatni.\\ W związku z tym wykres wielomianu zaczyna się od lewej strony poniżej osi OX. A więc $$x \in [1,9] \cup [13,\infty).$$
\rozwStop
\odpStart
$x \in [1,9] \cup [13,\infty)$
\odpStop
\testStart
A.$x \in [1,9] \cup [13,\infty)$\\
B.$x \in (1,9) \cup [13,\infty)$\\
C.$x \in (1,9] \cup [13,\infty)$\\
D.$x \in [1,9) \cup [13,\infty)$\\
E.$x \in [1,9] \cup (13,\infty)$\\
F.$x \in (1,9) \cup (13,\infty)$\\
G.$x \in [1,9) \cup (13,\infty)$\\
H.$x \in (1,9] \cup (13,\infty)$
\testStop
\kluczStart
A
\kluczStop



\zadStart{Zadanie z Wikieł Z 1.62 a) moja wersja nr 75}

Rozwiązać nierówności $(x-1)(x-9)(x-14)\ge0$.
\zadStop
\rozwStart{Patryk Wirkus}{Laura Mieczkowska}
Miejsca zerowe naszego wielomianu to: $1, 9, 14$.\\
Wielomian jest stopnia nieparzystego, ponadto znak współczynnika przy\linebreak najwyższej potędze x jest dodatni.\\ W związku z tym wykres wielomianu zaczyna się od lewej strony poniżej osi OX. A więc $$x \in [1,9] \cup [14,\infty).$$
\rozwStop
\odpStart
$x \in [1,9] \cup [14,\infty)$
\odpStop
\testStart
A.$x \in [1,9] \cup [14,\infty)$\\
B.$x \in (1,9) \cup [14,\infty)$\\
C.$x \in (1,9] \cup [14,\infty)$\\
D.$x \in [1,9) \cup [14,\infty)$\\
E.$x \in [1,9] \cup (14,\infty)$\\
F.$x \in (1,9) \cup (14,\infty)$\\
G.$x \in [1,9) \cup (14,\infty)$\\
H.$x \in (1,9] \cup (14,\infty)$
\testStop
\kluczStart
A
\kluczStop



\zadStart{Zadanie z Wikieł Z 1.62 a) moja wersja nr 76}

Rozwiązać nierówności $(x-1)(x-9)(x-15)\ge0$.
\zadStop
\rozwStart{Patryk Wirkus}{Laura Mieczkowska}
Miejsca zerowe naszego wielomianu to: $1, 9, 15$.\\
Wielomian jest stopnia nieparzystego, ponadto znak współczynnika przy\linebreak najwyższej potędze x jest dodatni.\\ W związku z tym wykres wielomianu zaczyna się od lewej strony poniżej osi OX. A więc $$x \in [1,9] \cup [15,\infty).$$
\rozwStop
\odpStart
$x \in [1,9] \cup [15,\infty)$
\odpStop
\testStart
A.$x \in [1,9] \cup [15,\infty)$\\
B.$x \in (1,9) \cup [15,\infty)$\\
C.$x \in (1,9] \cup [15,\infty)$\\
D.$x \in [1,9) \cup [15,\infty)$\\
E.$x \in [1,9] \cup (15,\infty)$\\
F.$x \in (1,9) \cup (15,\infty)$\\
G.$x \in [1,9) \cup (15,\infty)$\\
H.$x \in (1,9] \cup (15,\infty)$
\testStop
\kluczStart
A
\kluczStop



\zadStart{Zadanie z Wikieł Z 1.62 a) moja wersja nr 77}

Rozwiązać nierówności $(x-1)(x-10)(x-11)\ge0$.
\zadStop
\rozwStart{Patryk Wirkus}{Laura Mieczkowska}
Miejsca zerowe naszego wielomianu to: $1, 10, 11$.\\
Wielomian jest stopnia nieparzystego, ponadto znak współczynnika przy\linebreak najwyższej potędze x jest dodatni.\\ W związku z tym wykres wielomianu zaczyna się od lewej strony poniżej osi OX. A więc $$x \in [1,10] \cup [11,\infty).$$
\rozwStop
\odpStart
$x \in [1,10] \cup [11,\infty)$
\odpStop
\testStart
A.$x \in [1,10] \cup [11,\infty)$\\
B.$x \in (1,10) \cup [11,\infty)$\\
C.$x \in (1,10] \cup [11,\infty)$\\
D.$x \in [1,10) \cup [11,\infty)$\\
E.$x \in [1,10] \cup (11,\infty)$\\
F.$x \in (1,10) \cup (11,\infty)$\\
G.$x \in [1,10) \cup (11,\infty)$\\
H.$x \in (1,10] \cup (11,\infty)$
\testStop
\kluczStart
A
\kluczStop



\zadStart{Zadanie z Wikieł Z 1.62 a) moja wersja nr 78}

Rozwiązać nierówności $(x-1)(x-10)(x-12)\ge0$.
\zadStop
\rozwStart{Patryk Wirkus}{Laura Mieczkowska}
Miejsca zerowe naszego wielomianu to: $1, 10, 12$.\\
Wielomian jest stopnia nieparzystego, ponadto znak współczynnika przy\linebreak najwyższej potędze x jest dodatni.\\ W związku z tym wykres wielomianu zaczyna się od lewej strony poniżej osi OX. A więc $$x \in [1,10] \cup [12,\infty).$$
\rozwStop
\odpStart
$x \in [1,10] \cup [12,\infty)$
\odpStop
\testStart
A.$x \in [1,10] \cup [12,\infty)$\\
B.$x \in (1,10) \cup [12,\infty)$\\
C.$x \in (1,10] \cup [12,\infty)$\\
D.$x \in [1,10) \cup [12,\infty)$\\
E.$x \in [1,10] \cup (12,\infty)$\\
F.$x \in (1,10) \cup (12,\infty)$\\
G.$x \in [1,10) \cup (12,\infty)$\\
H.$x \in (1,10] \cup (12,\infty)$
\testStop
\kluczStart
A
\kluczStop



\zadStart{Zadanie z Wikieł Z 1.62 a) moja wersja nr 79}

Rozwiązać nierówności $(x-1)(x-10)(x-13)\ge0$.
\zadStop
\rozwStart{Patryk Wirkus}{Laura Mieczkowska}
Miejsca zerowe naszego wielomianu to: $1, 10, 13$.\\
Wielomian jest stopnia nieparzystego, ponadto znak współczynnika przy\linebreak najwyższej potędze x jest dodatni.\\ W związku z tym wykres wielomianu zaczyna się od lewej strony poniżej osi OX. A więc $$x \in [1,10] \cup [13,\infty).$$
\rozwStop
\odpStart
$x \in [1,10] \cup [13,\infty)$
\odpStop
\testStart
A.$x \in [1,10] \cup [13,\infty)$\\
B.$x \in (1,10) \cup [13,\infty)$\\
C.$x \in (1,10] \cup [13,\infty)$\\
D.$x \in [1,10) \cup [13,\infty)$\\
E.$x \in [1,10] \cup (13,\infty)$\\
F.$x \in (1,10) \cup (13,\infty)$\\
G.$x \in [1,10) \cup (13,\infty)$\\
H.$x \in (1,10] \cup (13,\infty)$
\testStop
\kluczStart
A
\kluczStop



\zadStart{Zadanie z Wikieł Z 1.62 a) moja wersja nr 80}

Rozwiązać nierówności $(x-1)(x-10)(x-14)\ge0$.
\zadStop
\rozwStart{Patryk Wirkus}{Laura Mieczkowska}
Miejsca zerowe naszego wielomianu to: $1, 10, 14$.\\
Wielomian jest stopnia nieparzystego, ponadto znak współczynnika przy\linebreak najwyższej potędze x jest dodatni.\\ W związku z tym wykres wielomianu zaczyna się od lewej strony poniżej osi OX. A więc $$x \in [1,10] \cup [14,\infty).$$
\rozwStop
\odpStart
$x \in [1,10] \cup [14,\infty)$
\odpStop
\testStart
A.$x \in [1,10] \cup [14,\infty)$\\
B.$x \in (1,10) \cup [14,\infty)$\\
C.$x \in (1,10] \cup [14,\infty)$\\
D.$x \in [1,10) \cup [14,\infty)$\\
E.$x \in [1,10] \cup (14,\infty)$\\
F.$x \in (1,10) \cup (14,\infty)$\\
G.$x \in [1,10) \cup (14,\infty)$\\
H.$x \in (1,10] \cup (14,\infty)$
\testStop
\kluczStart
A
\kluczStop



\zadStart{Zadanie z Wikieł Z 1.62 a) moja wersja nr 81}

Rozwiązać nierówności $(x-1)(x-10)(x-15)\ge0$.
\zadStop
\rozwStart{Patryk Wirkus}{Laura Mieczkowska}
Miejsca zerowe naszego wielomianu to: $1, 10, 15$.\\
Wielomian jest stopnia nieparzystego, ponadto znak współczynnika przy\linebreak najwyższej potędze x jest dodatni.\\ W związku z tym wykres wielomianu zaczyna się od lewej strony poniżej osi OX. A więc $$x \in [1,10] \cup [15,\infty).$$
\rozwStop
\odpStart
$x \in [1,10] \cup [15,\infty)$
\odpStop
\testStart
A.$x \in [1,10] \cup [15,\infty)$\\
B.$x \in (1,10) \cup [15,\infty)$\\
C.$x \in (1,10] \cup [15,\infty)$\\
D.$x \in [1,10) \cup [15,\infty)$\\
E.$x \in [1,10] \cup (15,\infty)$\\
F.$x \in (1,10) \cup (15,\infty)$\\
G.$x \in [1,10) \cup (15,\infty)$\\
H.$x \in (1,10] \cup (15,\infty)$
\testStop
\kluczStart
A
\kluczStop



\zadStart{Zadanie z Wikieł Z 1.62 a) moja wersja nr 82}

Rozwiązać nierówności $(x-2)(x-3)(x-4)\ge0$.
\zadStop
\rozwStart{Patryk Wirkus}{Laura Mieczkowska}
Miejsca zerowe naszego wielomianu to: $2, 3, 4$.\\
Wielomian jest stopnia nieparzystego, ponadto znak współczynnika przy\linebreak najwyższej potędze x jest dodatni.\\ W związku z tym wykres wielomianu zaczyna się od lewej strony poniżej osi OX. A więc $$x \in [2,3] \cup [4,\infty).$$
\rozwStop
\odpStart
$x \in [2,3] \cup [4,\infty)$
\odpStop
\testStart
A.$x \in [2,3] \cup [4,\infty)$\\
B.$x \in (2,3) \cup [4,\infty)$\\
C.$x \in (2,3] \cup [4,\infty)$\\
D.$x \in [2,3) \cup [4,\infty)$\\
E.$x \in [2,3] \cup (4,\infty)$\\
F.$x \in (2,3) \cup (4,\infty)$\\
G.$x \in [2,3) \cup (4,\infty)$\\
H.$x \in (2,3] \cup (4,\infty)$
\testStop
\kluczStart
A
\kluczStop



\zadStart{Zadanie z Wikieł Z 1.62 a) moja wersja nr 83}

Rozwiązać nierówności $(x-2)(x-3)(x-5)\ge0$.
\zadStop
\rozwStart{Patryk Wirkus}{Laura Mieczkowska}
Miejsca zerowe naszego wielomianu to: $2, 3, 5$.\\
Wielomian jest stopnia nieparzystego, ponadto znak współczynnika przy\linebreak najwyższej potędze x jest dodatni.\\ W związku z tym wykres wielomianu zaczyna się od lewej strony poniżej osi OX. A więc $$x \in [2,3] \cup [5,\infty).$$
\rozwStop
\odpStart
$x \in [2,3] \cup [5,\infty)$
\odpStop
\testStart
A.$x \in [2,3] \cup [5,\infty)$\\
B.$x \in (2,3) \cup [5,\infty)$\\
C.$x \in (2,3] \cup [5,\infty)$\\
D.$x \in [2,3) \cup [5,\infty)$\\
E.$x \in [2,3] \cup (5,\infty)$\\
F.$x \in (2,3) \cup (5,\infty)$\\
G.$x \in [2,3) \cup (5,\infty)$\\
H.$x \in (2,3] \cup (5,\infty)$
\testStop
\kluczStart
A
\kluczStop



\zadStart{Zadanie z Wikieł Z 1.62 a) moja wersja nr 84}

Rozwiązać nierówności $(x-2)(x-3)(x-6)\ge0$.
\zadStop
\rozwStart{Patryk Wirkus}{Laura Mieczkowska}
Miejsca zerowe naszego wielomianu to: $2, 3, 6$.\\
Wielomian jest stopnia nieparzystego, ponadto znak współczynnika przy\linebreak najwyższej potędze x jest dodatni.\\ W związku z tym wykres wielomianu zaczyna się od lewej strony poniżej osi OX. A więc $$x \in [2,3] \cup [6,\infty).$$
\rozwStop
\odpStart
$x \in [2,3] \cup [6,\infty)$
\odpStop
\testStart
A.$x \in [2,3] \cup [6,\infty)$\\
B.$x \in (2,3) \cup [6,\infty)$\\
C.$x \in (2,3] \cup [6,\infty)$\\
D.$x \in [2,3) \cup [6,\infty)$\\
E.$x \in [2,3] \cup (6,\infty)$\\
F.$x \in (2,3) \cup (6,\infty)$\\
G.$x \in [2,3) \cup (6,\infty)$\\
H.$x \in (2,3] \cup (6,\infty)$
\testStop
\kluczStart
A
\kluczStop



\zadStart{Zadanie z Wikieł Z 1.62 a) moja wersja nr 85}

Rozwiązać nierówności $(x-2)(x-3)(x-7)\ge0$.
\zadStop
\rozwStart{Patryk Wirkus}{Laura Mieczkowska}
Miejsca zerowe naszego wielomianu to: $2, 3, 7$.\\
Wielomian jest stopnia nieparzystego, ponadto znak współczynnika przy\linebreak najwyższej potędze x jest dodatni.\\ W związku z tym wykres wielomianu zaczyna się od lewej strony poniżej osi OX. A więc $$x \in [2,3] \cup [7,\infty).$$
\rozwStop
\odpStart
$x \in [2,3] \cup [7,\infty)$
\odpStop
\testStart
A.$x \in [2,3] \cup [7,\infty)$\\
B.$x \in (2,3) \cup [7,\infty)$\\
C.$x \in (2,3] \cup [7,\infty)$\\
D.$x \in [2,3) \cup [7,\infty)$\\
E.$x \in [2,3] \cup (7,\infty)$\\
F.$x \in (2,3) \cup (7,\infty)$\\
G.$x \in [2,3) \cup (7,\infty)$\\
H.$x \in (2,3] \cup (7,\infty)$
\testStop
\kluczStart
A
\kluczStop



\zadStart{Zadanie z Wikieł Z 1.62 a) moja wersja nr 86}

Rozwiązać nierówności $(x-2)(x-3)(x-8)\ge0$.
\zadStop
\rozwStart{Patryk Wirkus}{Laura Mieczkowska}
Miejsca zerowe naszego wielomianu to: $2, 3, 8$.\\
Wielomian jest stopnia nieparzystego, ponadto znak współczynnika przy\linebreak najwyższej potędze x jest dodatni.\\ W związku z tym wykres wielomianu zaczyna się od lewej strony poniżej osi OX. A więc $$x \in [2,3] \cup [8,\infty).$$
\rozwStop
\odpStart
$x \in [2,3] \cup [8,\infty)$
\odpStop
\testStart
A.$x \in [2,3] \cup [8,\infty)$\\
B.$x \in (2,3) \cup [8,\infty)$\\
C.$x \in (2,3] \cup [8,\infty)$\\
D.$x \in [2,3) \cup [8,\infty)$\\
E.$x \in [2,3] \cup (8,\infty)$\\
F.$x \in (2,3) \cup (8,\infty)$\\
G.$x \in [2,3) \cup (8,\infty)$\\
H.$x \in (2,3] \cup (8,\infty)$
\testStop
\kluczStart
A
\kluczStop



\zadStart{Zadanie z Wikieł Z 1.62 a) moja wersja nr 87}

Rozwiązać nierówności $(x-2)(x-3)(x-9)\ge0$.
\zadStop
\rozwStart{Patryk Wirkus}{Laura Mieczkowska}
Miejsca zerowe naszego wielomianu to: $2, 3, 9$.\\
Wielomian jest stopnia nieparzystego, ponadto znak współczynnika przy\linebreak najwyższej potędze x jest dodatni.\\ W związku z tym wykres wielomianu zaczyna się od lewej strony poniżej osi OX. A więc $$x \in [2,3] \cup [9,\infty).$$
\rozwStop
\odpStart
$x \in [2,3] \cup [9,\infty)$
\odpStop
\testStart
A.$x \in [2,3] \cup [9,\infty)$\\
B.$x \in (2,3) \cup [9,\infty)$\\
C.$x \in (2,3] \cup [9,\infty)$\\
D.$x \in [2,3) \cup [9,\infty)$\\
E.$x \in [2,3] \cup (9,\infty)$\\
F.$x \in (2,3) \cup (9,\infty)$\\
G.$x \in [2,3) \cup (9,\infty)$\\
H.$x \in (2,3] \cup (9,\infty)$
\testStop
\kluczStart
A
\kluczStop



\zadStart{Zadanie z Wikieł Z 1.62 a) moja wersja nr 88}

Rozwiązać nierówności $(x-2)(x-3)(x-10)\ge0$.
\zadStop
\rozwStart{Patryk Wirkus}{Laura Mieczkowska}
Miejsca zerowe naszego wielomianu to: $2, 3, 10$.\\
Wielomian jest stopnia nieparzystego, ponadto znak współczynnika przy\linebreak najwyższej potędze x jest dodatni.\\ W związku z tym wykres wielomianu zaczyna się od lewej strony poniżej osi OX. A więc $$x \in [2,3] \cup [10,\infty).$$
\rozwStop
\odpStart
$x \in [2,3] \cup [10,\infty)$
\odpStop
\testStart
A.$x \in [2,3] \cup [10,\infty)$\\
B.$x \in (2,3) \cup [10,\infty)$\\
C.$x \in (2,3] \cup [10,\infty)$\\
D.$x \in [2,3) \cup [10,\infty)$\\
E.$x \in [2,3] \cup (10,\infty)$\\
F.$x \in (2,3) \cup (10,\infty)$\\
G.$x \in [2,3) \cup (10,\infty)$\\
H.$x \in (2,3] \cup (10,\infty)$
\testStop
\kluczStart
A
\kluczStop



\zadStart{Zadanie z Wikieł Z 1.62 a) moja wersja nr 89}

Rozwiązać nierówności $(x-2)(x-3)(x-11)\ge0$.
\zadStop
\rozwStart{Patryk Wirkus}{Laura Mieczkowska}
Miejsca zerowe naszego wielomianu to: $2, 3, 11$.\\
Wielomian jest stopnia nieparzystego, ponadto znak współczynnika przy\linebreak najwyższej potędze x jest dodatni.\\ W związku z tym wykres wielomianu zaczyna się od lewej strony poniżej osi OX. A więc $$x \in [2,3] \cup [11,\infty).$$
\rozwStop
\odpStart
$x \in [2,3] \cup [11,\infty)$
\odpStop
\testStart
A.$x \in [2,3] \cup [11,\infty)$\\
B.$x \in (2,3) \cup [11,\infty)$\\
C.$x \in (2,3] \cup [11,\infty)$\\
D.$x \in [2,3) \cup [11,\infty)$\\
E.$x \in [2,3] \cup (11,\infty)$\\
F.$x \in (2,3) \cup (11,\infty)$\\
G.$x \in [2,3) \cup (11,\infty)$\\
H.$x \in (2,3] \cup (11,\infty)$
\testStop
\kluczStart
A
\kluczStop



\zadStart{Zadanie z Wikieł Z 1.62 a) moja wersja nr 90}

Rozwiązać nierówności $(x-2)(x-3)(x-12)\ge0$.
\zadStop
\rozwStart{Patryk Wirkus}{Laura Mieczkowska}
Miejsca zerowe naszego wielomianu to: $2, 3, 12$.\\
Wielomian jest stopnia nieparzystego, ponadto znak współczynnika przy\linebreak najwyższej potędze x jest dodatni.\\ W związku z tym wykres wielomianu zaczyna się od lewej strony poniżej osi OX. A więc $$x \in [2,3] \cup [12,\infty).$$
\rozwStop
\odpStart
$x \in [2,3] \cup [12,\infty)$
\odpStop
\testStart
A.$x \in [2,3] \cup [12,\infty)$\\
B.$x \in (2,3) \cup [12,\infty)$\\
C.$x \in (2,3] \cup [12,\infty)$\\
D.$x \in [2,3) \cup [12,\infty)$\\
E.$x \in [2,3] \cup (12,\infty)$\\
F.$x \in (2,3) \cup (12,\infty)$\\
G.$x \in [2,3) \cup (12,\infty)$\\
H.$x \in (2,3] \cup (12,\infty)$
\testStop
\kluczStart
A
\kluczStop



\zadStart{Zadanie z Wikieł Z 1.62 a) moja wersja nr 91}

Rozwiązać nierówności $(x-2)(x-3)(x-13)\ge0$.
\zadStop
\rozwStart{Patryk Wirkus}{Laura Mieczkowska}
Miejsca zerowe naszego wielomianu to: $2, 3, 13$.\\
Wielomian jest stopnia nieparzystego, ponadto znak współczynnika przy\linebreak najwyższej potędze x jest dodatni.\\ W związku z tym wykres wielomianu zaczyna się od lewej strony poniżej osi OX. A więc $$x \in [2,3] \cup [13,\infty).$$
\rozwStop
\odpStart
$x \in [2,3] \cup [13,\infty)$
\odpStop
\testStart
A.$x \in [2,3] \cup [13,\infty)$\\
B.$x \in (2,3) \cup [13,\infty)$\\
C.$x \in (2,3] \cup [13,\infty)$\\
D.$x \in [2,3) \cup [13,\infty)$\\
E.$x \in [2,3] \cup (13,\infty)$\\
F.$x \in (2,3) \cup (13,\infty)$\\
G.$x \in [2,3) \cup (13,\infty)$\\
H.$x \in (2,3] \cup (13,\infty)$
\testStop
\kluczStart
A
\kluczStop



\zadStart{Zadanie z Wikieł Z 1.62 a) moja wersja nr 92}

Rozwiązać nierówności $(x-2)(x-3)(x-14)\ge0$.
\zadStop
\rozwStart{Patryk Wirkus}{Laura Mieczkowska}
Miejsca zerowe naszego wielomianu to: $2, 3, 14$.\\
Wielomian jest stopnia nieparzystego, ponadto znak współczynnika przy\linebreak najwyższej potędze x jest dodatni.\\ W związku z tym wykres wielomianu zaczyna się od lewej strony poniżej osi OX. A więc $$x \in [2,3] \cup [14,\infty).$$
\rozwStop
\odpStart
$x \in [2,3] \cup [14,\infty)$
\odpStop
\testStart
A.$x \in [2,3] \cup [14,\infty)$\\
B.$x \in (2,3) \cup [14,\infty)$\\
C.$x \in (2,3] \cup [14,\infty)$\\
D.$x \in [2,3) \cup [14,\infty)$\\
E.$x \in [2,3] \cup (14,\infty)$\\
F.$x \in (2,3) \cup (14,\infty)$\\
G.$x \in [2,3) \cup (14,\infty)$\\
H.$x \in (2,3] \cup (14,\infty)$
\testStop
\kluczStart
A
\kluczStop



\zadStart{Zadanie z Wikieł Z 1.62 a) moja wersja nr 93}

Rozwiązać nierówności $(x-2)(x-3)(x-15)\ge0$.
\zadStop
\rozwStart{Patryk Wirkus}{Laura Mieczkowska}
Miejsca zerowe naszego wielomianu to: $2, 3, 15$.\\
Wielomian jest stopnia nieparzystego, ponadto znak współczynnika przy\linebreak najwyższej potędze x jest dodatni.\\ W związku z tym wykres wielomianu zaczyna się od lewej strony poniżej osi OX. A więc $$x \in [2,3] \cup [15,\infty).$$
\rozwStop
\odpStart
$x \in [2,3] \cup [15,\infty)$
\odpStop
\testStart
A.$x \in [2,3] \cup [15,\infty)$\\
B.$x \in (2,3) \cup [15,\infty)$\\
C.$x \in (2,3] \cup [15,\infty)$\\
D.$x \in [2,3) \cup [15,\infty)$\\
E.$x \in [2,3] \cup (15,\infty)$\\
F.$x \in (2,3) \cup (15,\infty)$\\
G.$x \in [2,3) \cup (15,\infty)$\\
H.$x \in (2,3] \cup (15,\infty)$
\testStop
\kluczStart
A
\kluczStop



\zadStart{Zadanie z Wikieł Z 1.62 a) moja wersja nr 94}

Rozwiązać nierówności $(x-2)(x-4)(x-5)\ge0$.
\zadStop
\rozwStart{Patryk Wirkus}{Laura Mieczkowska}
Miejsca zerowe naszego wielomianu to: $2, 4, 5$.\\
Wielomian jest stopnia nieparzystego, ponadto znak współczynnika przy\linebreak najwyższej potędze x jest dodatni.\\ W związku z tym wykres wielomianu zaczyna się od lewej strony poniżej osi OX. A więc $$x \in [2,4] \cup [5,\infty).$$
\rozwStop
\odpStart
$x \in [2,4] \cup [5,\infty)$
\odpStop
\testStart
A.$x \in [2,4] \cup [5,\infty)$\\
B.$x \in (2,4) \cup [5,\infty)$\\
C.$x \in (2,4] \cup [5,\infty)$\\
D.$x \in [2,4) \cup [5,\infty)$\\
E.$x \in [2,4] \cup (5,\infty)$\\
F.$x \in (2,4) \cup (5,\infty)$\\
G.$x \in [2,4) \cup (5,\infty)$\\
H.$x \in (2,4] \cup (5,\infty)$
\testStop
\kluczStart
A
\kluczStop



\zadStart{Zadanie z Wikieł Z 1.62 a) moja wersja nr 95}

Rozwiązać nierówności $(x-2)(x-4)(x-6)\ge0$.
\zadStop
\rozwStart{Patryk Wirkus}{Laura Mieczkowska}
Miejsca zerowe naszego wielomianu to: $2, 4, 6$.\\
Wielomian jest stopnia nieparzystego, ponadto znak współczynnika przy\linebreak najwyższej potędze x jest dodatni.\\ W związku z tym wykres wielomianu zaczyna się od lewej strony poniżej osi OX. A więc $$x \in [2,4] \cup [6,\infty).$$
\rozwStop
\odpStart
$x \in [2,4] \cup [6,\infty)$
\odpStop
\testStart
A.$x \in [2,4] \cup [6,\infty)$\\
B.$x \in (2,4) \cup [6,\infty)$\\
C.$x \in (2,4] \cup [6,\infty)$\\
D.$x \in [2,4) \cup [6,\infty)$\\
E.$x \in [2,4] \cup (6,\infty)$\\
F.$x \in (2,4) \cup (6,\infty)$\\
G.$x \in [2,4) \cup (6,\infty)$\\
H.$x \in (2,4] \cup (6,\infty)$
\testStop
\kluczStart
A
\kluczStop



\zadStart{Zadanie z Wikieł Z 1.62 a) moja wersja nr 96}

Rozwiązać nierówności $(x-2)(x-4)(x-7)\ge0$.
\zadStop
\rozwStart{Patryk Wirkus}{Laura Mieczkowska}
Miejsca zerowe naszego wielomianu to: $2, 4, 7$.\\
Wielomian jest stopnia nieparzystego, ponadto znak współczynnika przy\linebreak najwyższej potędze x jest dodatni.\\ W związku z tym wykres wielomianu zaczyna się od lewej strony poniżej osi OX. A więc $$x \in [2,4] \cup [7,\infty).$$
\rozwStop
\odpStart
$x \in [2,4] \cup [7,\infty)$
\odpStop
\testStart
A.$x \in [2,4] \cup [7,\infty)$\\
B.$x \in (2,4) \cup [7,\infty)$\\
C.$x \in (2,4] \cup [7,\infty)$\\
D.$x \in [2,4) \cup [7,\infty)$\\
E.$x \in [2,4] \cup (7,\infty)$\\
F.$x \in (2,4) \cup (7,\infty)$\\
G.$x \in [2,4) \cup (7,\infty)$\\
H.$x \in (2,4] \cup (7,\infty)$
\testStop
\kluczStart
A
\kluczStop



\zadStart{Zadanie z Wikieł Z 1.62 a) moja wersja nr 97}

Rozwiązać nierówności $(x-2)(x-4)(x-8)\ge0$.
\zadStop
\rozwStart{Patryk Wirkus}{Laura Mieczkowska}
Miejsca zerowe naszego wielomianu to: $2, 4, 8$.\\
Wielomian jest stopnia nieparzystego, ponadto znak współczynnika przy\linebreak najwyższej potędze x jest dodatni.\\ W związku z tym wykres wielomianu zaczyna się od lewej strony poniżej osi OX. A więc $$x \in [2,4] \cup [8,\infty).$$
\rozwStop
\odpStart
$x \in [2,4] \cup [8,\infty)$
\odpStop
\testStart
A.$x \in [2,4] \cup [8,\infty)$\\
B.$x \in (2,4) \cup [8,\infty)$\\
C.$x \in (2,4] \cup [8,\infty)$\\
D.$x \in [2,4) \cup [8,\infty)$\\
E.$x \in [2,4] \cup (8,\infty)$\\
F.$x \in (2,4) \cup (8,\infty)$\\
G.$x \in [2,4) \cup (8,\infty)$\\
H.$x \in (2,4] \cup (8,\infty)$
\testStop
\kluczStart
A
\kluczStop



\zadStart{Zadanie z Wikieł Z 1.62 a) moja wersja nr 98}

Rozwiązać nierówności $(x-2)(x-4)(x-9)\ge0$.
\zadStop
\rozwStart{Patryk Wirkus}{Laura Mieczkowska}
Miejsca zerowe naszego wielomianu to: $2, 4, 9$.\\
Wielomian jest stopnia nieparzystego, ponadto znak współczynnika przy\linebreak najwyższej potędze x jest dodatni.\\ W związku z tym wykres wielomianu zaczyna się od lewej strony poniżej osi OX. A więc $$x \in [2,4] \cup [9,\infty).$$
\rozwStop
\odpStart
$x \in [2,4] \cup [9,\infty)$
\odpStop
\testStart
A.$x \in [2,4] \cup [9,\infty)$\\
B.$x \in (2,4) \cup [9,\infty)$\\
C.$x \in (2,4] \cup [9,\infty)$\\
D.$x \in [2,4) \cup [9,\infty)$\\
E.$x \in [2,4] \cup (9,\infty)$\\
F.$x \in (2,4) \cup (9,\infty)$\\
G.$x \in [2,4) \cup (9,\infty)$\\
H.$x \in (2,4] \cup (9,\infty)$
\testStop
\kluczStart
A
\kluczStop



\zadStart{Zadanie z Wikieł Z 1.62 a) moja wersja nr 99}

Rozwiązać nierówności $(x-2)(x-4)(x-10)\ge0$.
\zadStop
\rozwStart{Patryk Wirkus}{Laura Mieczkowska}
Miejsca zerowe naszego wielomianu to: $2, 4, 10$.\\
Wielomian jest stopnia nieparzystego, ponadto znak współczynnika przy\linebreak najwyższej potędze x jest dodatni.\\ W związku z tym wykres wielomianu zaczyna się od lewej strony poniżej osi OX. A więc $$x \in [2,4] \cup [10,\infty).$$
\rozwStop
\odpStart
$x \in [2,4] \cup [10,\infty)$
\odpStop
\testStart
A.$x \in [2,4] \cup [10,\infty)$\\
B.$x \in (2,4) \cup [10,\infty)$\\
C.$x \in (2,4] \cup [10,\infty)$\\
D.$x \in [2,4) \cup [10,\infty)$\\
E.$x \in [2,4] \cup (10,\infty)$\\
F.$x \in (2,4) \cup (10,\infty)$\\
G.$x \in [2,4) \cup (10,\infty)$\\
H.$x \in (2,4] \cup (10,\infty)$
\testStop
\kluczStart
A
\kluczStop



\zadStart{Zadanie z Wikieł Z 1.62 a) moja wersja nr 100}

Rozwiązać nierówności $(x-2)(x-4)(x-11)\ge0$.
\zadStop
\rozwStart{Patryk Wirkus}{Laura Mieczkowska}
Miejsca zerowe naszego wielomianu to: $2, 4, 11$.\\
Wielomian jest stopnia nieparzystego, ponadto znak współczynnika przy\linebreak najwyższej potędze x jest dodatni.\\ W związku z tym wykres wielomianu zaczyna się od lewej strony poniżej osi OX. A więc $$x \in [2,4] \cup [11,\infty).$$
\rozwStop
\odpStart
$x \in [2,4] \cup [11,\infty)$
\odpStop
\testStart
A.$x \in [2,4] \cup [11,\infty)$\\
B.$x \in (2,4) \cup [11,\infty)$\\
C.$x \in (2,4] \cup [11,\infty)$\\
D.$x \in [2,4) \cup [11,\infty)$\\
E.$x \in [2,4] \cup (11,\infty)$\\
F.$x \in (2,4) \cup (11,\infty)$\\
G.$x \in [2,4) \cup (11,\infty)$\\
H.$x \in (2,4] \cup (11,\infty)$
\testStop
\kluczStart
A
\kluczStop



\zadStart{Zadanie z Wikieł Z 1.62 a) moja wersja nr 101}

Rozwiązać nierówności $(x-2)(x-4)(x-12)\ge0$.
\zadStop
\rozwStart{Patryk Wirkus}{Laura Mieczkowska}
Miejsca zerowe naszego wielomianu to: $2, 4, 12$.\\
Wielomian jest stopnia nieparzystego, ponadto znak współczynnika przy\linebreak najwyższej potędze x jest dodatni.\\ W związku z tym wykres wielomianu zaczyna się od lewej strony poniżej osi OX. A więc $$x \in [2,4] \cup [12,\infty).$$
\rozwStop
\odpStart
$x \in [2,4] \cup [12,\infty)$
\odpStop
\testStart
A.$x \in [2,4] \cup [12,\infty)$\\
B.$x \in (2,4) \cup [12,\infty)$\\
C.$x \in (2,4] \cup [12,\infty)$\\
D.$x \in [2,4) \cup [12,\infty)$\\
E.$x \in [2,4] \cup (12,\infty)$\\
F.$x \in (2,4) \cup (12,\infty)$\\
G.$x \in [2,4) \cup (12,\infty)$\\
H.$x \in (2,4] \cup (12,\infty)$
\testStop
\kluczStart
A
\kluczStop



\zadStart{Zadanie z Wikieł Z 1.62 a) moja wersja nr 102}

Rozwiązać nierówności $(x-2)(x-4)(x-13)\ge0$.
\zadStop
\rozwStart{Patryk Wirkus}{Laura Mieczkowska}
Miejsca zerowe naszego wielomianu to: $2, 4, 13$.\\
Wielomian jest stopnia nieparzystego, ponadto znak współczynnika przy\linebreak najwyższej potędze x jest dodatni.\\ W związku z tym wykres wielomianu zaczyna się od lewej strony poniżej osi OX. A więc $$x \in [2,4] \cup [13,\infty).$$
\rozwStop
\odpStart
$x \in [2,4] \cup [13,\infty)$
\odpStop
\testStart
A.$x \in [2,4] \cup [13,\infty)$\\
B.$x \in (2,4) \cup [13,\infty)$\\
C.$x \in (2,4] \cup [13,\infty)$\\
D.$x \in [2,4) \cup [13,\infty)$\\
E.$x \in [2,4] \cup (13,\infty)$\\
F.$x \in (2,4) \cup (13,\infty)$\\
G.$x \in [2,4) \cup (13,\infty)$\\
H.$x \in (2,4] \cup (13,\infty)$
\testStop
\kluczStart
A
\kluczStop



\zadStart{Zadanie z Wikieł Z 1.62 a) moja wersja nr 103}

Rozwiązać nierówności $(x-2)(x-4)(x-14)\ge0$.
\zadStop
\rozwStart{Patryk Wirkus}{Laura Mieczkowska}
Miejsca zerowe naszego wielomianu to: $2, 4, 14$.\\
Wielomian jest stopnia nieparzystego, ponadto znak współczynnika przy\linebreak najwyższej potędze x jest dodatni.\\ W związku z tym wykres wielomianu zaczyna się od lewej strony poniżej osi OX. A więc $$x \in [2,4] \cup [14,\infty).$$
\rozwStop
\odpStart
$x \in [2,4] \cup [14,\infty)$
\odpStop
\testStart
A.$x \in [2,4] \cup [14,\infty)$\\
B.$x \in (2,4) \cup [14,\infty)$\\
C.$x \in (2,4] \cup [14,\infty)$\\
D.$x \in [2,4) \cup [14,\infty)$\\
E.$x \in [2,4] \cup (14,\infty)$\\
F.$x \in (2,4) \cup (14,\infty)$\\
G.$x \in [2,4) \cup (14,\infty)$\\
H.$x \in (2,4] \cup (14,\infty)$
\testStop
\kluczStart
A
\kluczStop



\zadStart{Zadanie z Wikieł Z 1.62 a) moja wersja nr 104}

Rozwiązać nierówności $(x-2)(x-4)(x-15)\ge0$.
\zadStop
\rozwStart{Patryk Wirkus}{Laura Mieczkowska}
Miejsca zerowe naszego wielomianu to: $2, 4, 15$.\\
Wielomian jest stopnia nieparzystego, ponadto znak współczynnika przy\linebreak najwyższej potędze x jest dodatni.\\ W związku z tym wykres wielomianu zaczyna się od lewej strony poniżej osi OX. A więc $$x \in [2,4] \cup [15,\infty).$$
\rozwStop
\odpStart
$x \in [2,4] \cup [15,\infty)$
\odpStop
\testStart
A.$x \in [2,4] \cup [15,\infty)$\\
B.$x \in (2,4) \cup [15,\infty)$\\
C.$x \in (2,4] \cup [15,\infty)$\\
D.$x \in [2,4) \cup [15,\infty)$\\
E.$x \in [2,4] \cup (15,\infty)$\\
F.$x \in (2,4) \cup (15,\infty)$\\
G.$x \in [2,4) \cup (15,\infty)$\\
H.$x \in (2,4] \cup (15,\infty)$
\testStop
\kluczStart
A
\kluczStop



\zadStart{Zadanie z Wikieł Z 1.62 a) moja wersja nr 105}

Rozwiązać nierówności $(x-2)(x-5)(x-6)\ge0$.
\zadStop
\rozwStart{Patryk Wirkus}{Laura Mieczkowska}
Miejsca zerowe naszego wielomianu to: $2, 5, 6$.\\
Wielomian jest stopnia nieparzystego, ponadto znak współczynnika przy\linebreak najwyższej potędze x jest dodatni.\\ W związku z tym wykres wielomianu zaczyna się od lewej strony poniżej osi OX. A więc $$x \in [2,5] \cup [6,\infty).$$
\rozwStop
\odpStart
$x \in [2,5] \cup [6,\infty)$
\odpStop
\testStart
A.$x \in [2,5] \cup [6,\infty)$\\
B.$x \in (2,5) \cup [6,\infty)$\\
C.$x \in (2,5] \cup [6,\infty)$\\
D.$x \in [2,5) \cup [6,\infty)$\\
E.$x \in [2,5] \cup (6,\infty)$\\
F.$x \in (2,5) \cup (6,\infty)$\\
G.$x \in [2,5) \cup (6,\infty)$\\
H.$x \in (2,5] \cup (6,\infty)$
\testStop
\kluczStart
A
\kluczStop



\zadStart{Zadanie z Wikieł Z 1.62 a) moja wersja nr 106}

Rozwiązać nierówności $(x-2)(x-5)(x-7)\ge0$.
\zadStop
\rozwStart{Patryk Wirkus}{Laura Mieczkowska}
Miejsca zerowe naszego wielomianu to: $2, 5, 7$.\\
Wielomian jest stopnia nieparzystego, ponadto znak współczynnika przy\linebreak najwyższej potędze x jest dodatni.\\ W związku z tym wykres wielomianu zaczyna się od lewej strony poniżej osi OX. A więc $$x \in [2,5] \cup [7,\infty).$$
\rozwStop
\odpStart
$x \in [2,5] \cup [7,\infty)$
\odpStop
\testStart
A.$x \in [2,5] \cup [7,\infty)$\\
B.$x \in (2,5) \cup [7,\infty)$\\
C.$x \in (2,5] \cup [7,\infty)$\\
D.$x \in [2,5) \cup [7,\infty)$\\
E.$x \in [2,5] \cup (7,\infty)$\\
F.$x \in (2,5) \cup (7,\infty)$\\
G.$x \in [2,5) \cup (7,\infty)$\\
H.$x \in (2,5] \cup (7,\infty)$
\testStop
\kluczStart
A
\kluczStop



\zadStart{Zadanie z Wikieł Z 1.62 a) moja wersja nr 107}

Rozwiązać nierówności $(x-2)(x-5)(x-8)\ge0$.
\zadStop
\rozwStart{Patryk Wirkus}{Laura Mieczkowska}
Miejsca zerowe naszego wielomianu to: $2, 5, 8$.\\
Wielomian jest stopnia nieparzystego, ponadto znak współczynnika przy\linebreak najwyższej potędze x jest dodatni.\\ W związku z tym wykres wielomianu zaczyna się od lewej strony poniżej osi OX. A więc $$x \in [2,5] \cup [8,\infty).$$
\rozwStop
\odpStart
$x \in [2,5] \cup [8,\infty)$
\odpStop
\testStart
A.$x \in [2,5] \cup [8,\infty)$\\
B.$x \in (2,5) \cup [8,\infty)$\\
C.$x \in (2,5] \cup [8,\infty)$\\
D.$x \in [2,5) \cup [8,\infty)$\\
E.$x \in [2,5] \cup (8,\infty)$\\
F.$x \in (2,5) \cup (8,\infty)$\\
G.$x \in [2,5) \cup (8,\infty)$\\
H.$x \in (2,5] \cup (8,\infty)$
\testStop
\kluczStart
A
\kluczStop



\zadStart{Zadanie z Wikieł Z 1.62 a) moja wersja nr 108}

Rozwiązać nierówności $(x-2)(x-5)(x-9)\ge0$.
\zadStop
\rozwStart{Patryk Wirkus}{Laura Mieczkowska}
Miejsca zerowe naszego wielomianu to: $2, 5, 9$.\\
Wielomian jest stopnia nieparzystego, ponadto znak współczynnika przy\linebreak najwyższej potędze x jest dodatni.\\ W związku z tym wykres wielomianu zaczyna się od lewej strony poniżej osi OX. A więc $$x \in [2,5] \cup [9,\infty).$$
\rozwStop
\odpStart
$x \in [2,5] \cup [9,\infty)$
\odpStop
\testStart
A.$x \in [2,5] \cup [9,\infty)$\\
B.$x \in (2,5) \cup [9,\infty)$\\
C.$x \in (2,5] \cup [9,\infty)$\\
D.$x \in [2,5) \cup [9,\infty)$\\
E.$x \in [2,5] \cup (9,\infty)$\\
F.$x \in (2,5) \cup (9,\infty)$\\
G.$x \in [2,5) \cup (9,\infty)$\\
H.$x \in (2,5] \cup (9,\infty)$
\testStop
\kluczStart
A
\kluczStop



\zadStart{Zadanie z Wikieł Z 1.62 a) moja wersja nr 109}

Rozwiązać nierówności $(x-2)(x-5)(x-10)\ge0$.
\zadStop
\rozwStart{Patryk Wirkus}{Laura Mieczkowska}
Miejsca zerowe naszego wielomianu to: $2, 5, 10$.\\
Wielomian jest stopnia nieparzystego, ponadto znak współczynnika przy\linebreak najwyższej potędze x jest dodatni.\\ W związku z tym wykres wielomianu zaczyna się od lewej strony poniżej osi OX. A więc $$x \in [2,5] \cup [10,\infty).$$
\rozwStop
\odpStart
$x \in [2,5] \cup [10,\infty)$
\odpStop
\testStart
A.$x \in [2,5] \cup [10,\infty)$\\
B.$x \in (2,5) \cup [10,\infty)$\\
C.$x \in (2,5] \cup [10,\infty)$\\
D.$x \in [2,5) \cup [10,\infty)$\\
E.$x \in [2,5] \cup (10,\infty)$\\
F.$x \in (2,5) \cup (10,\infty)$\\
G.$x \in [2,5) \cup (10,\infty)$\\
H.$x \in (2,5] \cup (10,\infty)$
\testStop
\kluczStart
A
\kluczStop



\zadStart{Zadanie z Wikieł Z 1.62 a) moja wersja nr 110}

Rozwiązać nierówności $(x-2)(x-5)(x-11)\ge0$.
\zadStop
\rozwStart{Patryk Wirkus}{Laura Mieczkowska}
Miejsca zerowe naszego wielomianu to: $2, 5, 11$.\\
Wielomian jest stopnia nieparzystego, ponadto znak współczynnika przy\linebreak najwyższej potędze x jest dodatni.\\ W związku z tym wykres wielomianu zaczyna się od lewej strony poniżej osi OX. A więc $$x \in [2,5] \cup [11,\infty).$$
\rozwStop
\odpStart
$x \in [2,5] \cup [11,\infty)$
\odpStop
\testStart
A.$x \in [2,5] \cup [11,\infty)$\\
B.$x \in (2,5) \cup [11,\infty)$\\
C.$x \in (2,5] \cup [11,\infty)$\\
D.$x \in [2,5) \cup [11,\infty)$\\
E.$x \in [2,5] \cup (11,\infty)$\\
F.$x \in (2,5) \cup (11,\infty)$\\
G.$x \in [2,5) \cup (11,\infty)$\\
H.$x \in (2,5] \cup (11,\infty)$
\testStop
\kluczStart
A
\kluczStop



\zadStart{Zadanie z Wikieł Z 1.62 a) moja wersja nr 111}

Rozwiązać nierówności $(x-2)(x-5)(x-12)\ge0$.
\zadStop
\rozwStart{Patryk Wirkus}{Laura Mieczkowska}
Miejsca zerowe naszego wielomianu to: $2, 5, 12$.\\
Wielomian jest stopnia nieparzystego, ponadto znak współczynnika przy\linebreak najwyższej potędze x jest dodatni.\\ W związku z tym wykres wielomianu zaczyna się od lewej strony poniżej osi OX. A więc $$x \in [2,5] \cup [12,\infty).$$
\rozwStop
\odpStart
$x \in [2,5] \cup [12,\infty)$
\odpStop
\testStart
A.$x \in [2,5] \cup [12,\infty)$\\
B.$x \in (2,5) \cup [12,\infty)$\\
C.$x \in (2,5] \cup [12,\infty)$\\
D.$x \in [2,5) \cup [12,\infty)$\\
E.$x \in [2,5] \cup (12,\infty)$\\
F.$x \in (2,5) \cup (12,\infty)$\\
G.$x \in [2,5) \cup (12,\infty)$\\
H.$x \in (2,5] \cup (12,\infty)$
\testStop
\kluczStart
A
\kluczStop



\zadStart{Zadanie z Wikieł Z 1.62 a) moja wersja nr 112}

Rozwiązać nierówności $(x-2)(x-5)(x-13)\ge0$.
\zadStop
\rozwStart{Patryk Wirkus}{Laura Mieczkowska}
Miejsca zerowe naszego wielomianu to: $2, 5, 13$.\\
Wielomian jest stopnia nieparzystego, ponadto znak współczynnika przy\linebreak najwyższej potędze x jest dodatni.\\ W związku z tym wykres wielomianu zaczyna się od lewej strony poniżej osi OX. A więc $$x \in [2,5] \cup [13,\infty).$$
\rozwStop
\odpStart
$x \in [2,5] \cup [13,\infty)$
\odpStop
\testStart
A.$x \in [2,5] \cup [13,\infty)$\\
B.$x \in (2,5) \cup [13,\infty)$\\
C.$x \in (2,5] \cup [13,\infty)$\\
D.$x \in [2,5) \cup [13,\infty)$\\
E.$x \in [2,5] \cup (13,\infty)$\\
F.$x \in (2,5) \cup (13,\infty)$\\
G.$x \in [2,5) \cup (13,\infty)$\\
H.$x \in (2,5] \cup (13,\infty)$
\testStop
\kluczStart
A
\kluczStop



\zadStart{Zadanie z Wikieł Z 1.62 a) moja wersja nr 113}

Rozwiązać nierówności $(x-2)(x-5)(x-14)\ge0$.
\zadStop
\rozwStart{Patryk Wirkus}{Laura Mieczkowska}
Miejsca zerowe naszego wielomianu to: $2, 5, 14$.\\
Wielomian jest stopnia nieparzystego, ponadto znak współczynnika przy\linebreak najwyższej potędze x jest dodatni.\\ W związku z tym wykres wielomianu zaczyna się od lewej strony poniżej osi OX. A więc $$x \in [2,5] \cup [14,\infty).$$
\rozwStop
\odpStart
$x \in [2,5] \cup [14,\infty)$
\odpStop
\testStart
A.$x \in [2,5] \cup [14,\infty)$\\
B.$x \in (2,5) \cup [14,\infty)$\\
C.$x \in (2,5] \cup [14,\infty)$\\
D.$x \in [2,5) \cup [14,\infty)$\\
E.$x \in [2,5] \cup (14,\infty)$\\
F.$x \in (2,5) \cup (14,\infty)$\\
G.$x \in [2,5) \cup (14,\infty)$\\
H.$x \in (2,5] \cup (14,\infty)$
\testStop
\kluczStart
A
\kluczStop



\zadStart{Zadanie z Wikieł Z 1.62 a) moja wersja nr 114}

Rozwiązać nierówności $(x-2)(x-5)(x-15)\ge0$.
\zadStop
\rozwStart{Patryk Wirkus}{Laura Mieczkowska}
Miejsca zerowe naszego wielomianu to: $2, 5, 15$.\\
Wielomian jest stopnia nieparzystego, ponadto znak współczynnika przy\linebreak najwyższej potędze x jest dodatni.\\ W związku z tym wykres wielomianu zaczyna się od lewej strony poniżej osi OX. A więc $$x \in [2,5] \cup [15,\infty).$$
\rozwStop
\odpStart
$x \in [2,5] \cup [15,\infty)$
\odpStop
\testStart
A.$x \in [2,5] \cup [15,\infty)$\\
B.$x \in (2,5) \cup [15,\infty)$\\
C.$x \in (2,5] \cup [15,\infty)$\\
D.$x \in [2,5) \cup [15,\infty)$\\
E.$x \in [2,5] \cup (15,\infty)$\\
F.$x \in (2,5) \cup (15,\infty)$\\
G.$x \in [2,5) \cup (15,\infty)$\\
H.$x \in (2,5] \cup (15,\infty)$
\testStop
\kluczStart
A
\kluczStop



\zadStart{Zadanie z Wikieł Z 1.62 a) moja wersja nr 115}

Rozwiązać nierówności $(x-2)(x-6)(x-7)\ge0$.
\zadStop
\rozwStart{Patryk Wirkus}{Laura Mieczkowska}
Miejsca zerowe naszego wielomianu to: $2, 6, 7$.\\
Wielomian jest stopnia nieparzystego, ponadto znak współczynnika przy\linebreak najwyższej potędze x jest dodatni.\\ W związku z tym wykres wielomianu zaczyna się od lewej strony poniżej osi OX. A więc $$x \in [2,6] \cup [7,\infty).$$
\rozwStop
\odpStart
$x \in [2,6] \cup [7,\infty)$
\odpStop
\testStart
A.$x \in [2,6] \cup [7,\infty)$\\
B.$x \in (2,6) \cup [7,\infty)$\\
C.$x \in (2,6] \cup [7,\infty)$\\
D.$x \in [2,6) \cup [7,\infty)$\\
E.$x \in [2,6] \cup (7,\infty)$\\
F.$x \in (2,6) \cup (7,\infty)$\\
G.$x \in [2,6) \cup (7,\infty)$\\
H.$x \in (2,6] \cup (7,\infty)$
\testStop
\kluczStart
A
\kluczStop



\zadStart{Zadanie z Wikieł Z 1.62 a) moja wersja nr 116}

Rozwiązać nierówności $(x-2)(x-6)(x-8)\ge0$.
\zadStop
\rozwStart{Patryk Wirkus}{Laura Mieczkowska}
Miejsca zerowe naszego wielomianu to: $2, 6, 8$.\\
Wielomian jest stopnia nieparzystego, ponadto znak współczynnika przy\linebreak najwyższej potędze x jest dodatni.\\ W związku z tym wykres wielomianu zaczyna się od lewej strony poniżej osi OX. A więc $$x \in [2,6] \cup [8,\infty).$$
\rozwStop
\odpStart
$x \in [2,6] \cup [8,\infty)$
\odpStop
\testStart
A.$x \in [2,6] \cup [8,\infty)$\\
B.$x \in (2,6) \cup [8,\infty)$\\
C.$x \in (2,6] \cup [8,\infty)$\\
D.$x \in [2,6) \cup [8,\infty)$\\
E.$x \in [2,6] \cup (8,\infty)$\\
F.$x \in (2,6) \cup (8,\infty)$\\
G.$x \in [2,6) \cup (8,\infty)$\\
H.$x \in (2,6] \cup (8,\infty)$
\testStop
\kluczStart
A
\kluczStop



\zadStart{Zadanie z Wikieł Z 1.62 a) moja wersja nr 117}

Rozwiązać nierówności $(x-2)(x-6)(x-9)\ge0$.
\zadStop
\rozwStart{Patryk Wirkus}{Laura Mieczkowska}
Miejsca zerowe naszego wielomianu to: $2, 6, 9$.\\
Wielomian jest stopnia nieparzystego, ponadto znak współczynnika przy\linebreak najwyższej potędze x jest dodatni.\\ W związku z tym wykres wielomianu zaczyna się od lewej strony poniżej osi OX. A więc $$x \in [2,6] \cup [9,\infty).$$
\rozwStop
\odpStart
$x \in [2,6] \cup [9,\infty)$
\odpStop
\testStart
A.$x \in [2,6] \cup [9,\infty)$\\
B.$x \in (2,6) \cup [9,\infty)$\\
C.$x \in (2,6] \cup [9,\infty)$\\
D.$x \in [2,6) \cup [9,\infty)$\\
E.$x \in [2,6] \cup (9,\infty)$\\
F.$x \in (2,6) \cup (9,\infty)$\\
G.$x \in [2,6) \cup (9,\infty)$\\
H.$x \in (2,6] \cup (9,\infty)$
\testStop
\kluczStart
A
\kluczStop



\zadStart{Zadanie z Wikieł Z 1.62 a) moja wersja nr 118}

Rozwiązać nierówności $(x-2)(x-6)(x-10)\ge0$.
\zadStop
\rozwStart{Patryk Wirkus}{Laura Mieczkowska}
Miejsca zerowe naszego wielomianu to: $2, 6, 10$.\\
Wielomian jest stopnia nieparzystego, ponadto znak współczynnika przy\linebreak najwyższej potędze x jest dodatni.\\ W związku z tym wykres wielomianu zaczyna się od lewej strony poniżej osi OX. A więc $$x \in [2,6] \cup [10,\infty).$$
\rozwStop
\odpStart
$x \in [2,6] \cup [10,\infty)$
\odpStop
\testStart
A.$x \in [2,6] \cup [10,\infty)$\\
B.$x \in (2,6) \cup [10,\infty)$\\
C.$x \in (2,6] \cup [10,\infty)$\\
D.$x \in [2,6) \cup [10,\infty)$\\
E.$x \in [2,6] \cup (10,\infty)$\\
F.$x \in (2,6) \cup (10,\infty)$\\
G.$x \in [2,6) \cup (10,\infty)$\\
H.$x \in (2,6] \cup (10,\infty)$
\testStop
\kluczStart
A
\kluczStop



\zadStart{Zadanie z Wikieł Z 1.62 a) moja wersja nr 119}

Rozwiązać nierówności $(x-2)(x-6)(x-11)\ge0$.
\zadStop
\rozwStart{Patryk Wirkus}{Laura Mieczkowska}
Miejsca zerowe naszego wielomianu to: $2, 6, 11$.\\
Wielomian jest stopnia nieparzystego, ponadto znak współczynnika przy\linebreak najwyższej potędze x jest dodatni.\\ W związku z tym wykres wielomianu zaczyna się od lewej strony poniżej osi OX. A więc $$x \in [2,6] \cup [11,\infty).$$
\rozwStop
\odpStart
$x \in [2,6] \cup [11,\infty)$
\odpStop
\testStart
A.$x \in [2,6] \cup [11,\infty)$\\
B.$x \in (2,6) \cup [11,\infty)$\\
C.$x \in (2,6] \cup [11,\infty)$\\
D.$x \in [2,6) \cup [11,\infty)$\\
E.$x \in [2,6] \cup (11,\infty)$\\
F.$x \in (2,6) \cup (11,\infty)$\\
G.$x \in [2,6) \cup (11,\infty)$\\
H.$x \in (2,6] \cup (11,\infty)$
\testStop
\kluczStart
A
\kluczStop



\zadStart{Zadanie z Wikieł Z 1.62 a) moja wersja nr 120}

Rozwiązać nierówności $(x-2)(x-6)(x-12)\ge0$.
\zadStop
\rozwStart{Patryk Wirkus}{Laura Mieczkowska}
Miejsca zerowe naszego wielomianu to: $2, 6, 12$.\\
Wielomian jest stopnia nieparzystego, ponadto znak współczynnika przy\linebreak najwyższej potędze x jest dodatni.\\ W związku z tym wykres wielomianu zaczyna się od lewej strony poniżej osi OX. A więc $$x \in [2,6] \cup [12,\infty).$$
\rozwStop
\odpStart
$x \in [2,6] \cup [12,\infty)$
\odpStop
\testStart
A.$x \in [2,6] \cup [12,\infty)$\\
B.$x \in (2,6) \cup [12,\infty)$\\
C.$x \in (2,6] \cup [12,\infty)$\\
D.$x \in [2,6) \cup [12,\infty)$\\
E.$x \in [2,6] \cup (12,\infty)$\\
F.$x \in (2,6) \cup (12,\infty)$\\
G.$x \in [2,6) \cup (12,\infty)$\\
H.$x \in (2,6] \cup (12,\infty)$
\testStop
\kluczStart
A
\kluczStop



\zadStart{Zadanie z Wikieł Z 1.62 a) moja wersja nr 121}

Rozwiązać nierówności $(x-2)(x-6)(x-13)\ge0$.
\zadStop
\rozwStart{Patryk Wirkus}{Laura Mieczkowska}
Miejsca zerowe naszego wielomianu to: $2, 6, 13$.\\
Wielomian jest stopnia nieparzystego, ponadto znak współczynnika przy\linebreak najwyższej potędze x jest dodatni.\\ W związku z tym wykres wielomianu zaczyna się od lewej strony poniżej osi OX. A więc $$x \in [2,6] \cup [13,\infty).$$
\rozwStop
\odpStart
$x \in [2,6] \cup [13,\infty)$
\odpStop
\testStart
A.$x \in [2,6] \cup [13,\infty)$\\
B.$x \in (2,6) \cup [13,\infty)$\\
C.$x \in (2,6] \cup [13,\infty)$\\
D.$x \in [2,6) \cup [13,\infty)$\\
E.$x \in [2,6] \cup (13,\infty)$\\
F.$x \in (2,6) \cup (13,\infty)$\\
G.$x \in [2,6) \cup (13,\infty)$\\
H.$x \in (2,6] \cup (13,\infty)$
\testStop
\kluczStart
A
\kluczStop



\zadStart{Zadanie z Wikieł Z 1.62 a) moja wersja nr 122}

Rozwiązać nierówności $(x-2)(x-6)(x-14)\ge0$.
\zadStop
\rozwStart{Patryk Wirkus}{Laura Mieczkowska}
Miejsca zerowe naszego wielomianu to: $2, 6, 14$.\\
Wielomian jest stopnia nieparzystego, ponadto znak współczynnika przy\linebreak najwyższej potędze x jest dodatni.\\ W związku z tym wykres wielomianu zaczyna się od lewej strony poniżej osi OX. A więc $$x \in [2,6] \cup [14,\infty).$$
\rozwStop
\odpStart
$x \in [2,6] \cup [14,\infty)$
\odpStop
\testStart
A.$x \in [2,6] \cup [14,\infty)$\\
B.$x \in (2,6) \cup [14,\infty)$\\
C.$x \in (2,6] \cup [14,\infty)$\\
D.$x \in [2,6) \cup [14,\infty)$\\
E.$x \in [2,6] \cup (14,\infty)$\\
F.$x \in (2,6) \cup (14,\infty)$\\
G.$x \in [2,6) \cup (14,\infty)$\\
H.$x \in (2,6] \cup (14,\infty)$
\testStop
\kluczStart
A
\kluczStop



\zadStart{Zadanie z Wikieł Z 1.62 a) moja wersja nr 123}

Rozwiązać nierówności $(x-2)(x-6)(x-15)\ge0$.
\zadStop
\rozwStart{Patryk Wirkus}{Laura Mieczkowska}
Miejsca zerowe naszego wielomianu to: $2, 6, 15$.\\
Wielomian jest stopnia nieparzystego, ponadto znak współczynnika przy\linebreak najwyższej potędze x jest dodatni.\\ W związku z tym wykres wielomianu zaczyna się od lewej strony poniżej osi OX. A więc $$x \in [2,6] \cup [15,\infty).$$
\rozwStop
\odpStart
$x \in [2,6] \cup [15,\infty)$
\odpStop
\testStart
A.$x \in [2,6] \cup [15,\infty)$\\
B.$x \in (2,6) \cup [15,\infty)$\\
C.$x \in (2,6] \cup [15,\infty)$\\
D.$x \in [2,6) \cup [15,\infty)$\\
E.$x \in [2,6] \cup (15,\infty)$\\
F.$x \in (2,6) \cup (15,\infty)$\\
G.$x \in [2,6) \cup (15,\infty)$\\
H.$x \in (2,6] \cup (15,\infty)$
\testStop
\kluczStart
A
\kluczStop



\zadStart{Zadanie z Wikieł Z 1.62 a) moja wersja nr 124}

Rozwiązać nierówności $(x-2)(x-7)(x-8)\ge0$.
\zadStop
\rozwStart{Patryk Wirkus}{Laura Mieczkowska}
Miejsca zerowe naszego wielomianu to: $2, 7, 8$.\\
Wielomian jest stopnia nieparzystego, ponadto znak współczynnika przy\linebreak najwyższej potędze x jest dodatni.\\ W związku z tym wykres wielomianu zaczyna się od lewej strony poniżej osi OX. A więc $$x \in [2,7] \cup [8,\infty).$$
\rozwStop
\odpStart
$x \in [2,7] \cup [8,\infty)$
\odpStop
\testStart
A.$x \in [2,7] \cup [8,\infty)$\\
B.$x \in (2,7) \cup [8,\infty)$\\
C.$x \in (2,7] \cup [8,\infty)$\\
D.$x \in [2,7) \cup [8,\infty)$\\
E.$x \in [2,7] \cup (8,\infty)$\\
F.$x \in (2,7) \cup (8,\infty)$\\
G.$x \in [2,7) \cup (8,\infty)$\\
H.$x \in (2,7] \cup (8,\infty)$
\testStop
\kluczStart
A
\kluczStop



\zadStart{Zadanie z Wikieł Z 1.62 a) moja wersja nr 125}

Rozwiązać nierówności $(x-2)(x-7)(x-9)\ge0$.
\zadStop
\rozwStart{Patryk Wirkus}{Laura Mieczkowska}
Miejsca zerowe naszego wielomianu to: $2, 7, 9$.\\
Wielomian jest stopnia nieparzystego, ponadto znak współczynnika przy\linebreak najwyższej potędze x jest dodatni.\\ W związku z tym wykres wielomianu zaczyna się od lewej strony poniżej osi OX. A więc $$x \in [2,7] \cup [9,\infty).$$
\rozwStop
\odpStart
$x \in [2,7] \cup [9,\infty)$
\odpStop
\testStart
A.$x \in [2,7] \cup [9,\infty)$\\
B.$x \in (2,7) \cup [9,\infty)$\\
C.$x \in (2,7] \cup [9,\infty)$\\
D.$x \in [2,7) \cup [9,\infty)$\\
E.$x \in [2,7] \cup (9,\infty)$\\
F.$x \in (2,7) \cup (9,\infty)$\\
G.$x \in [2,7) \cup (9,\infty)$\\
H.$x \in (2,7] \cup (9,\infty)$
\testStop
\kluczStart
A
\kluczStop



\zadStart{Zadanie z Wikieł Z 1.62 a) moja wersja nr 126}

Rozwiązać nierówności $(x-2)(x-7)(x-10)\ge0$.
\zadStop
\rozwStart{Patryk Wirkus}{Laura Mieczkowska}
Miejsca zerowe naszego wielomianu to: $2, 7, 10$.\\
Wielomian jest stopnia nieparzystego, ponadto znak współczynnika przy\linebreak najwyższej potędze x jest dodatni.\\ W związku z tym wykres wielomianu zaczyna się od lewej strony poniżej osi OX. A więc $$x \in [2,7] \cup [10,\infty).$$
\rozwStop
\odpStart
$x \in [2,7] \cup [10,\infty)$
\odpStop
\testStart
A.$x \in [2,7] \cup [10,\infty)$\\
B.$x \in (2,7) \cup [10,\infty)$\\
C.$x \in (2,7] \cup [10,\infty)$\\
D.$x \in [2,7) \cup [10,\infty)$\\
E.$x \in [2,7] \cup (10,\infty)$\\
F.$x \in (2,7) \cup (10,\infty)$\\
G.$x \in [2,7) \cup (10,\infty)$\\
H.$x \in (2,7] \cup (10,\infty)$
\testStop
\kluczStart
A
\kluczStop



\zadStart{Zadanie z Wikieł Z 1.62 a) moja wersja nr 127}

Rozwiązać nierówności $(x-2)(x-7)(x-11)\ge0$.
\zadStop
\rozwStart{Patryk Wirkus}{Laura Mieczkowska}
Miejsca zerowe naszego wielomianu to: $2, 7, 11$.\\
Wielomian jest stopnia nieparzystego, ponadto znak współczynnika przy\linebreak najwyższej potędze x jest dodatni.\\ W związku z tym wykres wielomianu zaczyna się od lewej strony poniżej osi OX. A więc $$x \in [2,7] \cup [11,\infty).$$
\rozwStop
\odpStart
$x \in [2,7] \cup [11,\infty)$
\odpStop
\testStart
A.$x \in [2,7] \cup [11,\infty)$\\
B.$x \in (2,7) \cup [11,\infty)$\\
C.$x \in (2,7] \cup [11,\infty)$\\
D.$x \in [2,7) \cup [11,\infty)$\\
E.$x \in [2,7] \cup (11,\infty)$\\
F.$x \in (2,7) \cup (11,\infty)$\\
G.$x \in [2,7) \cup (11,\infty)$\\
H.$x \in (2,7] \cup (11,\infty)$
\testStop
\kluczStart
A
\kluczStop



\zadStart{Zadanie z Wikieł Z 1.62 a) moja wersja nr 128}

Rozwiązać nierówności $(x-2)(x-7)(x-12)\ge0$.
\zadStop
\rozwStart{Patryk Wirkus}{Laura Mieczkowska}
Miejsca zerowe naszego wielomianu to: $2, 7, 12$.\\
Wielomian jest stopnia nieparzystego, ponadto znak współczynnika przy\linebreak najwyższej potędze x jest dodatni.\\ W związku z tym wykres wielomianu zaczyna się od lewej strony poniżej osi OX. A więc $$x \in [2,7] \cup [12,\infty).$$
\rozwStop
\odpStart
$x \in [2,7] \cup [12,\infty)$
\odpStop
\testStart
A.$x \in [2,7] \cup [12,\infty)$\\
B.$x \in (2,7) \cup [12,\infty)$\\
C.$x \in (2,7] \cup [12,\infty)$\\
D.$x \in [2,7) \cup [12,\infty)$\\
E.$x \in [2,7] \cup (12,\infty)$\\
F.$x \in (2,7) \cup (12,\infty)$\\
G.$x \in [2,7) \cup (12,\infty)$\\
H.$x \in (2,7] \cup (12,\infty)$
\testStop
\kluczStart
A
\kluczStop



\zadStart{Zadanie z Wikieł Z 1.62 a) moja wersja nr 129}

Rozwiązać nierówności $(x-2)(x-7)(x-13)\ge0$.
\zadStop
\rozwStart{Patryk Wirkus}{Laura Mieczkowska}
Miejsca zerowe naszego wielomianu to: $2, 7, 13$.\\
Wielomian jest stopnia nieparzystego, ponadto znak współczynnika przy\linebreak najwyższej potędze x jest dodatni.\\ W związku z tym wykres wielomianu zaczyna się od lewej strony poniżej osi OX. A więc $$x \in [2,7] \cup [13,\infty).$$
\rozwStop
\odpStart
$x \in [2,7] \cup [13,\infty)$
\odpStop
\testStart
A.$x \in [2,7] \cup [13,\infty)$\\
B.$x \in (2,7) \cup [13,\infty)$\\
C.$x \in (2,7] \cup [13,\infty)$\\
D.$x \in [2,7) \cup [13,\infty)$\\
E.$x \in [2,7] \cup (13,\infty)$\\
F.$x \in (2,7) \cup (13,\infty)$\\
G.$x \in [2,7) \cup (13,\infty)$\\
H.$x \in (2,7] \cup (13,\infty)$
\testStop
\kluczStart
A
\kluczStop



\zadStart{Zadanie z Wikieł Z 1.62 a) moja wersja nr 130}

Rozwiązać nierówności $(x-2)(x-7)(x-14)\ge0$.
\zadStop
\rozwStart{Patryk Wirkus}{Laura Mieczkowska}
Miejsca zerowe naszego wielomianu to: $2, 7, 14$.\\
Wielomian jest stopnia nieparzystego, ponadto znak współczynnika przy\linebreak najwyższej potędze x jest dodatni.\\ W związku z tym wykres wielomianu zaczyna się od lewej strony poniżej osi OX. A więc $$x \in [2,7] \cup [14,\infty).$$
\rozwStop
\odpStart
$x \in [2,7] \cup [14,\infty)$
\odpStop
\testStart
A.$x \in [2,7] \cup [14,\infty)$\\
B.$x \in (2,7) \cup [14,\infty)$\\
C.$x \in (2,7] \cup [14,\infty)$\\
D.$x \in [2,7) \cup [14,\infty)$\\
E.$x \in [2,7] \cup (14,\infty)$\\
F.$x \in (2,7) \cup (14,\infty)$\\
G.$x \in [2,7) \cup (14,\infty)$\\
H.$x \in (2,7] \cup (14,\infty)$
\testStop
\kluczStart
A
\kluczStop



\zadStart{Zadanie z Wikieł Z 1.62 a) moja wersja nr 131}

Rozwiązać nierówności $(x-2)(x-7)(x-15)\ge0$.
\zadStop
\rozwStart{Patryk Wirkus}{Laura Mieczkowska}
Miejsca zerowe naszego wielomianu to: $2, 7, 15$.\\
Wielomian jest stopnia nieparzystego, ponadto znak współczynnika przy\linebreak najwyższej potędze x jest dodatni.\\ W związku z tym wykres wielomianu zaczyna się od lewej strony poniżej osi OX. A więc $$x \in [2,7] \cup [15,\infty).$$
\rozwStop
\odpStart
$x \in [2,7] \cup [15,\infty)$
\odpStop
\testStart
A.$x \in [2,7] \cup [15,\infty)$\\
B.$x \in (2,7) \cup [15,\infty)$\\
C.$x \in (2,7] \cup [15,\infty)$\\
D.$x \in [2,7) \cup [15,\infty)$\\
E.$x \in [2,7] \cup (15,\infty)$\\
F.$x \in (2,7) \cup (15,\infty)$\\
G.$x \in [2,7) \cup (15,\infty)$\\
H.$x \in (2,7] \cup (15,\infty)$
\testStop
\kluczStart
A
\kluczStop



\zadStart{Zadanie z Wikieł Z 1.62 a) moja wersja nr 132}

Rozwiązać nierówności $(x-2)(x-8)(x-9)\ge0$.
\zadStop
\rozwStart{Patryk Wirkus}{Laura Mieczkowska}
Miejsca zerowe naszego wielomianu to: $2, 8, 9$.\\
Wielomian jest stopnia nieparzystego, ponadto znak współczynnika przy\linebreak najwyższej potędze x jest dodatni.\\ W związku z tym wykres wielomianu zaczyna się od lewej strony poniżej osi OX. A więc $$x \in [2,8] \cup [9,\infty).$$
\rozwStop
\odpStart
$x \in [2,8] \cup [9,\infty)$
\odpStop
\testStart
A.$x \in [2,8] \cup [9,\infty)$\\
B.$x \in (2,8) \cup [9,\infty)$\\
C.$x \in (2,8] \cup [9,\infty)$\\
D.$x \in [2,8) \cup [9,\infty)$\\
E.$x \in [2,8] \cup (9,\infty)$\\
F.$x \in (2,8) \cup (9,\infty)$\\
G.$x \in [2,8) \cup (9,\infty)$\\
H.$x \in (2,8] \cup (9,\infty)$
\testStop
\kluczStart
A
\kluczStop



\zadStart{Zadanie z Wikieł Z 1.62 a) moja wersja nr 133}

Rozwiązać nierówności $(x-2)(x-8)(x-10)\ge0$.
\zadStop
\rozwStart{Patryk Wirkus}{Laura Mieczkowska}
Miejsca zerowe naszego wielomianu to: $2, 8, 10$.\\
Wielomian jest stopnia nieparzystego, ponadto znak współczynnika przy\linebreak najwyższej potędze x jest dodatni.\\ W związku z tym wykres wielomianu zaczyna się od lewej strony poniżej osi OX. A więc $$x \in [2,8] \cup [10,\infty).$$
\rozwStop
\odpStart
$x \in [2,8] \cup [10,\infty)$
\odpStop
\testStart
A.$x \in [2,8] \cup [10,\infty)$\\
B.$x \in (2,8) \cup [10,\infty)$\\
C.$x \in (2,8] \cup [10,\infty)$\\
D.$x \in [2,8) \cup [10,\infty)$\\
E.$x \in [2,8] \cup (10,\infty)$\\
F.$x \in (2,8) \cup (10,\infty)$\\
G.$x \in [2,8) \cup (10,\infty)$\\
H.$x \in (2,8] \cup (10,\infty)$
\testStop
\kluczStart
A
\kluczStop



\zadStart{Zadanie z Wikieł Z 1.62 a) moja wersja nr 134}

Rozwiązać nierówności $(x-2)(x-8)(x-11)\ge0$.
\zadStop
\rozwStart{Patryk Wirkus}{Laura Mieczkowska}
Miejsca zerowe naszego wielomianu to: $2, 8, 11$.\\
Wielomian jest stopnia nieparzystego, ponadto znak współczynnika przy\linebreak najwyższej potędze x jest dodatni.\\ W związku z tym wykres wielomianu zaczyna się od lewej strony poniżej osi OX. A więc $$x \in [2,8] \cup [11,\infty).$$
\rozwStop
\odpStart
$x \in [2,8] \cup [11,\infty)$
\odpStop
\testStart
A.$x \in [2,8] \cup [11,\infty)$\\
B.$x \in (2,8) \cup [11,\infty)$\\
C.$x \in (2,8] \cup [11,\infty)$\\
D.$x \in [2,8) \cup [11,\infty)$\\
E.$x \in [2,8] \cup (11,\infty)$\\
F.$x \in (2,8) \cup (11,\infty)$\\
G.$x \in [2,8) \cup (11,\infty)$\\
H.$x \in (2,8] \cup (11,\infty)$
\testStop
\kluczStart
A
\kluczStop



\zadStart{Zadanie z Wikieł Z 1.62 a) moja wersja nr 135}

Rozwiązać nierówności $(x-2)(x-8)(x-12)\ge0$.
\zadStop
\rozwStart{Patryk Wirkus}{Laura Mieczkowska}
Miejsca zerowe naszego wielomianu to: $2, 8, 12$.\\
Wielomian jest stopnia nieparzystego, ponadto znak współczynnika przy\linebreak najwyższej potędze x jest dodatni.\\ W związku z tym wykres wielomianu zaczyna się od lewej strony poniżej osi OX. A więc $$x \in [2,8] \cup [12,\infty).$$
\rozwStop
\odpStart
$x \in [2,8] \cup [12,\infty)$
\odpStop
\testStart
A.$x \in [2,8] \cup [12,\infty)$\\
B.$x \in (2,8) \cup [12,\infty)$\\
C.$x \in (2,8] \cup [12,\infty)$\\
D.$x \in [2,8) \cup [12,\infty)$\\
E.$x \in [2,8] \cup (12,\infty)$\\
F.$x \in (2,8) \cup (12,\infty)$\\
G.$x \in [2,8) \cup (12,\infty)$\\
H.$x \in (2,8] \cup (12,\infty)$
\testStop
\kluczStart
A
\kluczStop



\zadStart{Zadanie z Wikieł Z 1.62 a) moja wersja nr 136}

Rozwiązać nierówności $(x-2)(x-8)(x-13)\ge0$.
\zadStop
\rozwStart{Patryk Wirkus}{Laura Mieczkowska}
Miejsca zerowe naszego wielomianu to: $2, 8, 13$.\\
Wielomian jest stopnia nieparzystego, ponadto znak współczynnika przy\linebreak najwyższej potędze x jest dodatni.\\ W związku z tym wykres wielomianu zaczyna się od lewej strony poniżej osi OX. A więc $$x \in [2,8] \cup [13,\infty).$$
\rozwStop
\odpStart
$x \in [2,8] \cup [13,\infty)$
\odpStop
\testStart
A.$x \in [2,8] \cup [13,\infty)$\\
B.$x \in (2,8) \cup [13,\infty)$\\
C.$x \in (2,8] \cup [13,\infty)$\\
D.$x \in [2,8) \cup [13,\infty)$\\
E.$x \in [2,8] \cup (13,\infty)$\\
F.$x \in (2,8) \cup (13,\infty)$\\
G.$x \in [2,8) \cup (13,\infty)$\\
H.$x \in (2,8] \cup (13,\infty)$
\testStop
\kluczStart
A
\kluczStop



\zadStart{Zadanie z Wikieł Z 1.62 a) moja wersja nr 137}

Rozwiązać nierówności $(x-2)(x-8)(x-14)\ge0$.
\zadStop
\rozwStart{Patryk Wirkus}{Laura Mieczkowska}
Miejsca zerowe naszego wielomianu to: $2, 8, 14$.\\
Wielomian jest stopnia nieparzystego, ponadto znak współczynnika przy\linebreak najwyższej potędze x jest dodatni.\\ W związku z tym wykres wielomianu zaczyna się od lewej strony poniżej osi OX. A więc $$x \in [2,8] \cup [14,\infty).$$
\rozwStop
\odpStart
$x \in [2,8] \cup [14,\infty)$
\odpStop
\testStart
A.$x \in [2,8] \cup [14,\infty)$\\
B.$x \in (2,8) \cup [14,\infty)$\\
C.$x \in (2,8] \cup [14,\infty)$\\
D.$x \in [2,8) \cup [14,\infty)$\\
E.$x \in [2,8] \cup (14,\infty)$\\
F.$x \in (2,8) \cup (14,\infty)$\\
G.$x \in [2,8) \cup (14,\infty)$\\
H.$x \in (2,8] \cup (14,\infty)$
\testStop
\kluczStart
A
\kluczStop



\zadStart{Zadanie z Wikieł Z 1.62 a) moja wersja nr 138}

Rozwiązać nierówności $(x-2)(x-8)(x-15)\ge0$.
\zadStop
\rozwStart{Patryk Wirkus}{Laura Mieczkowska}
Miejsca zerowe naszego wielomianu to: $2, 8, 15$.\\
Wielomian jest stopnia nieparzystego, ponadto znak współczynnika przy\linebreak najwyższej potędze x jest dodatni.\\ W związku z tym wykres wielomianu zaczyna się od lewej strony poniżej osi OX. A więc $$x \in [2,8] \cup [15,\infty).$$
\rozwStop
\odpStart
$x \in [2,8] \cup [15,\infty)$
\odpStop
\testStart
A.$x \in [2,8] \cup [15,\infty)$\\
B.$x \in (2,8) \cup [15,\infty)$\\
C.$x \in (2,8] \cup [15,\infty)$\\
D.$x \in [2,8) \cup [15,\infty)$\\
E.$x \in [2,8] \cup (15,\infty)$\\
F.$x \in (2,8) \cup (15,\infty)$\\
G.$x \in [2,8) \cup (15,\infty)$\\
H.$x \in (2,8] \cup (15,\infty)$
\testStop
\kluczStart
A
\kluczStop



\zadStart{Zadanie z Wikieł Z 1.62 a) moja wersja nr 139}

Rozwiązać nierówności $(x-2)(x-9)(x-10)\ge0$.
\zadStop
\rozwStart{Patryk Wirkus}{Laura Mieczkowska}
Miejsca zerowe naszego wielomianu to: $2, 9, 10$.\\
Wielomian jest stopnia nieparzystego, ponadto znak współczynnika przy\linebreak najwyższej potędze x jest dodatni.\\ W związku z tym wykres wielomianu zaczyna się od lewej strony poniżej osi OX. A więc $$x \in [2,9] \cup [10,\infty).$$
\rozwStop
\odpStart
$x \in [2,9] \cup [10,\infty)$
\odpStop
\testStart
A.$x \in [2,9] \cup [10,\infty)$\\
B.$x \in (2,9) \cup [10,\infty)$\\
C.$x \in (2,9] \cup [10,\infty)$\\
D.$x \in [2,9) \cup [10,\infty)$\\
E.$x \in [2,9] \cup (10,\infty)$\\
F.$x \in (2,9) \cup (10,\infty)$\\
G.$x \in [2,9) \cup (10,\infty)$\\
H.$x \in (2,9] \cup (10,\infty)$
\testStop
\kluczStart
A
\kluczStop



\zadStart{Zadanie z Wikieł Z 1.62 a) moja wersja nr 140}

Rozwiązać nierówności $(x-2)(x-9)(x-11)\ge0$.
\zadStop
\rozwStart{Patryk Wirkus}{Laura Mieczkowska}
Miejsca zerowe naszego wielomianu to: $2, 9, 11$.\\
Wielomian jest stopnia nieparzystego, ponadto znak współczynnika przy\linebreak najwyższej potędze x jest dodatni.\\ W związku z tym wykres wielomianu zaczyna się od lewej strony poniżej osi OX. A więc $$x \in [2,9] \cup [11,\infty).$$
\rozwStop
\odpStart
$x \in [2,9] \cup [11,\infty)$
\odpStop
\testStart
A.$x \in [2,9] \cup [11,\infty)$\\
B.$x \in (2,9) \cup [11,\infty)$\\
C.$x \in (2,9] \cup [11,\infty)$\\
D.$x \in [2,9) \cup [11,\infty)$\\
E.$x \in [2,9] \cup (11,\infty)$\\
F.$x \in (2,9) \cup (11,\infty)$\\
G.$x \in [2,9) \cup (11,\infty)$\\
H.$x \in (2,9] \cup (11,\infty)$
\testStop
\kluczStart
A
\kluczStop



\zadStart{Zadanie z Wikieł Z 1.62 a) moja wersja nr 141}

Rozwiązać nierówności $(x-2)(x-9)(x-12)\ge0$.
\zadStop
\rozwStart{Patryk Wirkus}{Laura Mieczkowska}
Miejsca zerowe naszego wielomianu to: $2, 9, 12$.\\
Wielomian jest stopnia nieparzystego, ponadto znak współczynnika przy\linebreak najwyższej potędze x jest dodatni.\\ W związku z tym wykres wielomianu zaczyna się od lewej strony poniżej osi OX. A więc $$x \in [2,9] \cup [12,\infty).$$
\rozwStop
\odpStart
$x \in [2,9] \cup [12,\infty)$
\odpStop
\testStart
A.$x \in [2,9] \cup [12,\infty)$\\
B.$x \in (2,9) \cup [12,\infty)$\\
C.$x \in (2,9] \cup [12,\infty)$\\
D.$x \in [2,9) \cup [12,\infty)$\\
E.$x \in [2,9] \cup (12,\infty)$\\
F.$x \in (2,9) \cup (12,\infty)$\\
G.$x \in [2,9) \cup (12,\infty)$\\
H.$x \in (2,9] \cup (12,\infty)$
\testStop
\kluczStart
A
\kluczStop



\zadStart{Zadanie z Wikieł Z 1.62 a) moja wersja nr 142}

Rozwiązać nierówności $(x-2)(x-9)(x-13)\ge0$.
\zadStop
\rozwStart{Patryk Wirkus}{Laura Mieczkowska}
Miejsca zerowe naszego wielomianu to: $2, 9, 13$.\\
Wielomian jest stopnia nieparzystego, ponadto znak współczynnika przy\linebreak najwyższej potędze x jest dodatni.\\ W związku z tym wykres wielomianu zaczyna się od lewej strony poniżej osi OX. A więc $$x \in [2,9] \cup [13,\infty).$$
\rozwStop
\odpStart
$x \in [2,9] \cup [13,\infty)$
\odpStop
\testStart
A.$x \in [2,9] \cup [13,\infty)$\\
B.$x \in (2,9) \cup [13,\infty)$\\
C.$x \in (2,9] \cup [13,\infty)$\\
D.$x \in [2,9) \cup [13,\infty)$\\
E.$x \in [2,9] \cup (13,\infty)$\\
F.$x \in (2,9) \cup (13,\infty)$\\
G.$x \in [2,9) \cup (13,\infty)$\\
H.$x \in (2,9] \cup (13,\infty)$
\testStop
\kluczStart
A
\kluczStop



\zadStart{Zadanie z Wikieł Z 1.62 a) moja wersja nr 143}

Rozwiązać nierówności $(x-2)(x-9)(x-14)\ge0$.
\zadStop
\rozwStart{Patryk Wirkus}{Laura Mieczkowska}
Miejsca zerowe naszego wielomianu to: $2, 9, 14$.\\
Wielomian jest stopnia nieparzystego, ponadto znak współczynnika przy\linebreak najwyższej potędze x jest dodatni.\\ W związku z tym wykres wielomianu zaczyna się od lewej strony poniżej osi OX. A więc $$x \in [2,9] \cup [14,\infty).$$
\rozwStop
\odpStart
$x \in [2,9] \cup [14,\infty)$
\odpStop
\testStart
A.$x \in [2,9] \cup [14,\infty)$\\
B.$x \in (2,9) \cup [14,\infty)$\\
C.$x \in (2,9] \cup [14,\infty)$\\
D.$x \in [2,9) \cup [14,\infty)$\\
E.$x \in [2,9] \cup (14,\infty)$\\
F.$x \in (2,9) \cup (14,\infty)$\\
G.$x \in [2,9) \cup (14,\infty)$\\
H.$x \in (2,9] \cup (14,\infty)$
\testStop
\kluczStart
A
\kluczStop



\zadStart{Zadanie z Wikieł Z 1.62 a) moja wersja nr 144}

Rozwiązać nierówności $(x-2)(x-9)(x-15)\ge0$.
\zadStop
\rozwStart{Patryk Wirkus}{Laura Mieczkowska}
Miejsca zerowe naszego wielomianu to: $2, 9, 15$.\\
Wielomian jest stopnia nieparzystego, ponadto znak współczynnika przy\linebreak najwyższej potędze x jest dodatni.\\ W związku z tym wykres wielomianu zaczyna się od lewej strony poniżej osi OX. A więc $$x \in [2,9] \cup [15,\infty).$$
\rozwStop
\odpStart
$x \in [2,9] \cup [15,\infty)$
\odpStop
\testStart
A.$x \in [2,9] \cup [15,\infty)$\\
B.$x \in (2,9) \cup [15,\infty)$\\
C.$x \in (2,9] \cup [15,\infty)$\\
D.$x \in [2,9) \cup [15,\infty)$\\
E.$x \in [2,9] \cup (15,\infty)$\\
F.$x \in (2,9) \cup (15,\infty)$\\
G.$x \in [2,9) \cup (15,\infty)$\\
H.$x \in (2,9] \cup (15,\infty)$
\testStop
\kluczStart
A
\kluczStop



\zadStart{Zadanie z Wikieł Z 1.62 a) moja wersja nr 145}

Rozwiązać nierówności $(x-2)(x-10)(x-11)\ge0$.
\zadStop
\rozwStart{Patryk Wirkus}{Laura Mieczkowska}
Miejsca zerowe naszego wielomianu to: $2, 10, 11$.\\
Wielomian jest stopnia nieparzystego, ponadto znak współczynnika przy\linebreak najwyższej potędze x jest dodatni.\\ W związku z tym wykres wielomianu zaczyna się od lewej strony poniżej osi OX. A więc $$x \in [2,10] \cup [11,\infty).$$
\rozwStop
\odpStart
$x \in [2,10] \cup [11,\infty)$
\odpStop
\testStart
A.$x \in [2,10] \cup [11,\infty)$\\
B.$x \in (2,10) \cup [11,\infty)$\\
C.$x \in (2,10] \cup [11,\infty)$\\
D.$x \in [2,10) \cup [11,\infty)$\\
E.$x \in [2,10] \cup (11,\infty)$\\
F.$x \in (2,10) \cup (11,\infty)$\\
G.$x \in [2,10) \cup (11,\infty)$\\
H.$x \in (2,10] \cup (11,\infty)$
\testStop
\kluczStart
A
\kluczStop



\zadStart{Zadanie z Wikieł Z 1.62 a) moja wersja nr 146}

Rozwiązać nierówności $(x-2)(x-10)(x-12)\ge0$.
\zadStop
\rozwStart{Patryk Wirkus}{Laura Mieczkowska}
Miejsca zerowe naszego wielomianu to: $2, 10, 12$.\\
Wielomian jest stopnia nieparzystego, ponadto znak współczynnika przy\linebreak najwyższej potędze x jest dodatni.\\ W związku z tym wykres wielomianu zaczyna się od lewej strony poniżej osi OX. A więc $$x \in [2,10] \cup [12,\infty).$$
\rozwStop
\odpStart
$x \in [2,10] \cup [12,\infty)$
\odpStop
\testStart
A.$x \in [2,10] \cup [12,\infty)$\\
B.$x \in (2,10) \cup [12,\infty)$\\
C.$x \in (2,10] \cup [12,\infty)$\\
D.$x \in [2,10) \cup [12,\infty)$\\
E.$x \in [2,10] \cup (12,\infty)$\\
F.$x \in (2,10) \cup (12,\infty)$\\
G.$x \in [2,10) \cup (12,\infty)$\\
H.$x \in (2,10] \cup (12,\infty)$
\testStop
\kluczStart
A
\kluczStop



\zadStart{Zadanie z Wikieł Z 1.62 a) moja wersja nr 147}

Rozwiązać nierówności $(x-2)(x-10)(x-13)\ge0$.
\zadStop
\rozwStart{Patryk Wirkus}{Laura Mieczkowska}
Miejsca zerowe naszego wielomianu to: $2, 10, 13$.\\
Wielomian jest stopnia nieparzystego, ponadto znak współczynnika przy\linebreak najwyższej potędze x jest dodatni.\\ W związku z tym wykres wielomianu zaczyna się od lewej strony poniżej osi OX. A więc $$x \in [2,10] \cup [13,\infty).$$
\rozwStop
\odpStart
$x \in [2,10] \cup [13,\infty)$
\odpStop
\testStart
A.$x \in [2,10] \cup [13,\infty)$\\
B.$x \in (2,10) \cup [13,\infty)$\\
C.$x \in (2,10] \cup [13,\infty)$\\
D.$x \in [2,10) \cup [13,\infty)$\\
E.$x \in [2,10] \cup (13,\infty)$\\
F.$x \in (2,10) \cup (13,\infty)$\\
G.$x \in [2,10) \cup (13,\infty)$\\
H.$x \in (2,10] \cup (13,\infty)$
\testStop
\kluczStart
A
\kluczStop



\zadStart{Zadanie z Wikieł Z 1.62 a) moja wersja nr 148}

Rozwiązać nierówności $(x-2)(x-10)(x-14)\ge0$.
\zadStop
\rozwStart{Patryk Wirkus}{Laura Mieczkowska}
Miejsca zerowe naszego wielomianu to: $2, 10, 14$.\\
Wielomian jest stopnia nieparzystego, ponadto znak współczynnika przy\linebreak najwyższej potędze x jest dodatni.\\ W związku z tym wykres wielomianu zaczyna się od lewej strony poniżej osi OX. A więc $$x \in [2,10] \cup [14,\infty).$$
\rozwStop
\odpStart
$x \in [2,10] \cup [14,\infty)$
\odpStop
\testStart
A.$x \in [2,10] \cup [14,\infty)$\\
B.$x \in (2,10) \cup [14,\infty)$\\
C.$x \in (2,10] \cup [14,\infty)$\\
D.$x \in [2,10) \cup [14,\infty)$\\
E.$x \in [2,10] \cup (14,\infty)$\\
F.$x \in (2,10) \cup (14,\infty)$\\
G.$x \in [2,10) \cup (14,\infty)$\\
H.$x \in (2,10] \cup (14,\infty)$
\testStop
\kluczStart
A
\kluczStop



\zadStart{Zadanie z Wikieł Z 1.62 a) moja wersja nr 149}

Rozwiązać nierówności $(x-2)(x-10)(x-15)\ge0$.
\zadStop
\rozwStart{Patryk Wirkus}{Laura Mieczkowska}
Miejsca zerowe naszego wielomianu to: $2, 10, 15$.\\
Wielomian jest stopnia nieparzystego, ponadto znak współczynnika przy\linebreak najwyższej potędze x jest dodatni.\\ W związku z tym wykres wielomianu zaczyna się od lewej strony poniżej osi OX. A więc $$x \in [2,10] \cup [15,\infty).$$
\rozwStop
\odpStart
$x \in [2,10] \cup [15,\infty)$
\odpStop
\testStart
A.$x \in [2,10] \cup [15,\infty)$\\
B.$x \in (2,10) \cup [15,\infty)$\\
C.$x \in (2,10] \cup [15,\infty)$\\
D.$x \in [2,10) \cup [15,\infty)$\\
E.$x \in [2,10] \cup (15,\infty)$\\
F.$x \in (2,10) \cup (15,\infty)$\\
G.$x \in [2,10) \cup (15,\infty)$\\
H.$x \in (2,10] \cup (15,\infty)$
\testStop
\kluczStart
A
\kluczStop



\zadStart{Zadanie z Wikieł Z 1.62 a) moja wersja nr 150}

Rozwiązać nierówności $(x-3)(x-4)(x-5)\ge0$.
\zadStop
\rozwStart{Patryk Wirkus}{Laura Mieczkowska}
Miejsca zerowe naszego wielomianu to: $3, 4, 5$.\\
Wielomian jest stopnia nieparzystego, ponadto znak współczynnika przy\linebreak najwyższej potędze x jest dodatni.\\ W związku z tym wykres wielomianu zaczyna się od lewej strony poniżej osi OX. A więc $$x \in [3,4] \cup [5,\infty).$$
\rozwStop
\odpStart
$x \in [3,4] \cup [5,\infty)$
\odpStop
\testStart
A.$x \in [3,4] \cup [5,\infty)$\\
B.$x \in (3,4) \cup [5,\infty)$\\
C.$x \in (3,4] \cup [5,\infty)$\\
D.$x \in [3,4) \cup [5,\infty)$\\
E.$x \in [3,4] \cup (5,\infty)$\\
F.$x \in (3,4) \cup (5,\infty)$\\
G.$x \in [3,4) \cup (5,\infty)$\\
H.$x \in (3,4] \cup (5,\infty)$
\testStop
\kluczStart
A
\kluczStop



\zadStart{Zadanie z Wikieł Z 1.62 a) moja wersja nr 151}

Rozwiązać nierówności $(x-3)(x-4)(x-6)\ge0$.
\zadStop
\rozwStart{Patryk Wirkus}{Laura Mieczkowska}
Miejsca zerowe naszego wielomianu to: $3, 4, 6$.\\
Wielomian jest stopnia nieparzystego, ponadto znak współczynnika przy\linebreak najwyższej potędze x jest dodatni.\\ W związku z tym wykres wielomianu zaczyna się od lewej strony poniżej osi OX. A więc $$x \in [3,4] \cup [6,\infty).$$
\rozwStop
\odpStart
$x \in [3,4] \cup [6,\infty)$
\odpStop
\testStart
A.$x \in [3,4] \cup [6,\infty)$\\
B.$x \in (3,4) \cup [6,\infty)$\\
C.$x \in (3,4] \cup [6,\infty)$\\
D.$x \in [3,4) \cup [6,\infty)$\\
E.$x \in [3,4] \cup (6,\infty)$\\
F.$x \in (3,4) \cup (6,\infty)$\\
G.$x \in [3,4) \cup (6,\infty)$\\
H.$x \in (3,4] \cup (6,\infty)$
\testStop
\kluczStart
A
\kluczStop



\zadStart{Zadanie z Wikieł Z 1.62 a) moja wersja nr 152}

Rozwiązać nierówności $(x-3)(x-4)(x-7)\ge0$.
\zadStop
\rozwStart{Patryk Wirkus}{Laura Mieczkowska}
Miejsca zerowe naszego wielomianu to: $3, 4, 7$.\\
Wielomian jest stopnia nieparzystego, ponadto znak współczynnika przy\linebreak najwyższej potędze x jest dodatni.\\ W związku z tym wykres wielomianu zaczyna się od lewej strony poniżej osi OX. A więc $$x \in [3,4] \cup [7,\infty).$$
\rozwStop
\odpStart
$x \in [3,4] \cup [7,\infty)$
\odpStop
\testStart
A.$x \in [3,4] \cup [7,\infty)$\\
B.$x \in (3,4) \cup [7,\infty)$\\
C.$x \in (3,4] \cup [7,\infty)$\\
D.$x \in [3,4) \cup [7,\infty)$\\
E.$x \in [3,4] \cup (7,\infty)$\\
F.$x \in (3,4) \cup (7,\infty)$\\
G.$x \in [3,4) \cup (7,\infty)$\\
H.$x \in (3,4] \cup (7,\infty)$
\testStop
\kluczStart
A
\kluczStop



\zadStart{Zadanie z Wikieł Z 1.62 a) moja wersja nr 153}

Rozwiązać nierówności $(x-3)(x-4)(x-8)\ge0$.
\zadStop
\rozwStart{Patryk Wirkus}{Laura Mieczkowska}
Miejsca zerowe naszego wielomianu to: $3, 4, 8$.\\
Wielomian jest stopnia nieparzystego, ponadto znak współczynnika przy\linebreak najwyższej potędze x jest dodatni.\\ W związku z tym wykres wielomianu zaczyna się od lewej strony poniżej osi OX. A więc $$x \in [3,4] \cup [8,\infty).$$
\rozwStop
\odpStart
$x \in [3,4] \cup [8,\infty)$
\odpStop
\testStart
A.$x \in [3,4] \cup [8,\infty)$\\
B.$x \in (3,4) \cup [8,\infty)$\\
C.$x \in (3,4] \cup [8,\infty)$\\
D.$x \in [3,4) \cup [8,\infty)$\\
E.$x \in [3,4] \cup (8,\infty)$\\
F.$x \in (3,4) \cup (8,\infty)$\\
G.$x \in [3,4) \cup (8,\infty)$\\
H.$x \in (3,4] \cup (8,\infty)$
\testStop
\kluczStart
A
\kluczStop



\zadStart{Zadanie z Wikieł Z 1.62 a) moja wersja nr 154}

Rozwiązać nierówności $(x-3)(x-4)(x-9)\ge0$.
\zadStop
\rozwStart{Patryk Wirkus}{Laura Mieczkowska}
Miejsca zerowe naszego wielomianu to: $3, 4, 9$.\\
Wielomian jest stopnia nieparzystego, ponadto znak współczynnika przy\linebreak najwyższej potędze x jest dodatni.\\ W związku z tym wykres wielomianu zaczyna się od lewej strony poniżej osi OX. A więc $$x \in [3,4] \cup [9,\infty).$$
\rozwStop
\odpStart
$x \in [3,4] \cup [9,\infty)$
\odpStop
\testStart
A.$x \in [3,4] \cup [9,\infty)$\\
B.$x \in (3,4) \cup [9,\infty)$\\
C.$x \in (3,4] \cup [9,\infty)$\\
D.$x \in [3,4) \cup [9,\infty)$\\
E.$x \in [3,4] \cup (9,\infty)$\\
F.$x \in (3,4) \cup (9,\infty)$\\
G.$x \in [3,4) \cup (9,\infty)$\\
H.$x \in (3,4] \cup (9,\infty)$
\testStop
\kluczStart
A
\kluczStop



\zadStart{Zadanie z Wikieł Z 1.62 a) moja wersja nr 155}

Rozwiązać nierówności $(x-3)(x-4)(x-10)\ge0$.
\zadStop
\rozwStart{Patryk Wirkus}{Laura Mieczkowska}
Miejsca zerowe naszego wielomianu to: $3, 4, 10$.\\
Wielomian jest stopnia nieparzystego, ponadto znak współczynnika przy\linebreak najwyższej potędze x jest dodatni.\\ W związku z tym wykres wielomianu zaczyna się od lewej strony poniżej osi OX. A więc $$x \in [3,4] \cup [10,\infty).$$
\rozwStop
\odpStart
$x \in [3,4] \cup [10,\infty)$
\odpStop
\testStart
A.$x \in [3,4] \cup [10,\infty)$\\
B.$x \in (3,4) \cup [10,\infty)$\\
C.$x \in (3,4] \cup [10,\infty)$\\
D.$x \in [3,4) \cup [10,\infty)$\\
E.$x \in [3,4] \cup (10,\infty)$\\
F.$x \in (3,4) \cup (10,\infty)$\\
G.$x \in [3,4) \cup (10,\infty)$\\
H.$x \in (3,4] \cup (10,\infty)$
\testStop
\kluczStart
A
\kluczStop



\zadStart{Zadanie z Wikieł Z 1.62 a) moja wersja nr 156}

Rozwiązać nierówności $(x-3)(x-4)(x-11)\ge0$.
\zadStop
\rozwStart{Patryk Wirkus}{Laura Mieczkowska}
Miejsca zerowe naszego wielomianu to: $3, 4, 11$.\\
Wielomian jest stopnia nieparzystego, ponadto znak współczynnika przy\linebreak najwyższej potędze x jest dodatni.\\ W związku z tym wykres wielomianu zaczyna się od lewej strony poniżej osi OX. A więc $$x \in [3,4] \cup [11,\infty).$$
\rozwStop
\odpStart
$x \in [3,4] \cup [11,\infty)$
\odpStop
\testStart
A.$x \in [3,4] \cup [11,\infty)$\\
B.$x \in (3,4) \cup [11,\infty)$\\
C.$x \in (3,4] \cup [11,\infty)$\\
D.$x \in [3,4) \cup [11,\infty)$\\
E.$x \in [3,4] \cup (11,\infty)$\\
F.$x \in (3,4) \cup (11,\infty)$\\
G.$x \in [3,4) \cup (11,\infty)$\\
H.$x \in (3,4] \cup (11,\infty)$
\testStop
\kluczStart
A
\kluczStop



\zadStart{Zadanie z Wikieł Z 1.62 a) moja wersja nr 157}

Rozwiązać nierówności $(x-3)(x-4)(x-12)\ge0$.
\zadStop
\rozwStart{Patryk Wirkus}{Laura Mieczkowska}
Miejsca zerowe naszego wielomianu to: $3, 4, 12$.\\
Wielomian jest stopnia nieparzystego, ponadto znak współczynnika przy\linebreak najwyższej potędze x jest dodatni.\\ W związku z tym wykres wielomianu zaczyna się od lewej strony poniżej osi OX. A więc $$x \in [3,4] \cup [12,\infty).$$
\rozwStop
\odpStart
$x \in [3,4] \cup [12,\infty)$
\odpStop
\testStart
A.$x \in [3,4] \cup [12,\infty)$\\
B.$x \in (3,4) \cup [12,\infty)$\\
C.$x \in (3,4] \cup [12,\infty)$\\
D.$x \in [3,4) \cup [12,\infty)$\\
E.$x \in [3,4] \cup (12,\infty)$\\
F.$x \in (3,4) \cup (12,\infty)$\\
G.$x \in [3,4) \cup (12,\infty)$\\
H.$x \in (3,4] \cup (12,\infty)$
\testStop
\kluczStart
A
\kluczStop



\zadStart{Zadanie z Wikieł Z 1.62 a) moja wersja nr 158}

Rozwiązać nierówności $(x-3)(x-4)(x-13)\ge0$.
\zadStop
\rozwStart{Patryk Wirkus}{Laura Mieczkowska}
Miejsca zerowe naszego wielomianu to: $3, 4, 13$.\\
Wielomian jest stopnia nieparzystego, ponadto znak współczynnika przy\linebreak najwyższej potędze x jest dodatni.\\ W związku z tym wykres wielomianu zaczyna się od lewej strony poniżej osi OX. A więc $$x \in [3,4] \cup [13,\infty).$$
\rozwStop
\odpStart
$x \in [3,4] \cup [13,\infty)$
\odpStop
\testStart
A.$x \in [3,4] \cup [13,\infty)$\\
B.$x \in (3,4) \cup [13,\infty)$\\
C.$x \in (3,4] \cup [13,\infty)$\\
D.$x \in [3,4) \cup [13,\infty)$\\
E.$x \in [3,4] \cup (13,\infty)$\\
F.$x \in (3,4) \cup (13,\infty)$\\
G.$x \in [3,4) \cup (13,\infty)$\\
H.$x \in (3,4] \cup (13,\infty)$
\testStop
\kluczStart
A
\kluczStop



\zadStart{Zadanie z Wikieł Z 1.62 a) moja wersja nr 159}

Rozwiązać nierówności $(x-3)(x-4)(x-14)\ge0$.
\zadStop
\rozwStart{Patryk Wirkus}{Laura Mieczkowska}
Miejsca zerowe naszego wielomianu to: $3, 4, 14$.\\
Wielomian jest stopnia nieparzystego, ponadto znak współczynnika przy\linebreak najwyższej potędze x jest dodatni.\\ W związku z tym wykres wielomianu zaczyna się od lewej strony poniżej osi OX. A więc $$x \in [3,4] \cup [14,\infty).$$
\rozwStop
\odpStart
$x \in [3,4] \cup [14,\infty)$
\odpStop
\testStart
A.$x \in [3,4] \cup [14,\infty)$\\
B.$x \in (3,4) \cup [14,\infty)$\\
C.$x \in (3,4] \cup [14,\infty)$\\
D.$x \in [3,4) \cup [14,\infty)$\\
E.$x \in [3,4] \cup (14,\infty)$\\
F.$x \in (3,4) \cup (14,\infty)$\\
G.$x \in [3,4) \cup (14,\infty)$\\
H.$x \in (3,4] \cup (14,\infty)$
\testStop
\kluczStart
A
\kluczStop



\zadStart{Zadanie z Wikieł Z 1.62 a) moja wersja nr 160}

Rozwiązać nierówności $(x-3)(x-4)(x-15)\ge0$.
\zadStop
\rozwStart{Patryk Wirkus}{Laura Mieczkowska}
Miejsca zerowe naszego wielomianu to: $3, 4, 15$.\\
Wielomian jest stopnia nieparzystego, ponadto znak współczynnika przy\linebreak najwyższej potędze x jest dodatni.\\ W związku z tym wykres wielomianu zaczyna się od lewej strony poniżej osi OX. A więc $$x \in [3,4] \cup [15,\infty).$$
\rozwStop
\odpStart
$x \in [3,4] \cup [15,\infty)$
\odpStop
\testStart
A.$x \in [3,4] \cup [15,\infty)$\\
B.$x \in (3,4) \cup [15,\infty)$\\
C.$x \in (3,4] \cup [15,\infty)$\\
D.$x \in [3,4) \cup [15,\infty)$\\
E.$x \in [3,4] \cup (15,\infty)$\\
F.$x \in (3,4) \cup (15,\infty)$\\
G.$x \in [3,4) \cup (15,\infty)$\\
H.$x \in (3,4] \cup (15,\infty)$
\testStop
\kluczStart
A
\kluczStop



\zadStart{Zadanie z Wikieł Z 1.62 a) moja wersja nr 161}

Rozwiązać nierówności $(x-3)(x-5)(x-6)\ge0$.
\zadStop
\rozwStart{Patryk Wirkus}{Laura Mieczkowska}
Miejsca zerowe naszego wielomianu to: $3, 5, 6$.\\
Wielomian jest stopnia nieparzystego, ponadto znak współczynnika przy\linebreak najwyższej potędze x jest dodatni.\\ W związku z tym wykres wielomianu zaczyna się od lewej strony poniżej osi OX. A więc $$x \in [3,5] \cup [6,\infty).$$
\rozwStop
\odpStart
$x \in [3,5] \cup [6,\infty)$
\odpStop
\testStart
A.$x \in [3,5] \cup [6,\infty)$\\
B.$x \in (3,5) \cup [6,\infty)$\\
C.$x \in (3,5] \cup [6,\infty)$\\
D.$x \in [3,5) \cup [6,\infty)$\\
E.$x \in [3,5] \cup (6,\infty)$\\
F.$x \in (3,5) \cup (6,\infty)$\\
G.$x \in [3,5) \cup (6,\infty)$\\
H.$x \in (3,5] \cup (6,\infty)$
\testStop
\kluczStart
A
\kluczStop



\zadStart{Zadanie z Wikieł Z 1.62 a) moja wersja nr 162}

Rozwiązać nierówności $(x-3)(x-5)(x-7)\ge0$.
\zadStop
\rozwStart{Patryk Wirkus}{Laura Mieczkowska}
Miejsca zerowe naszego wielomianu to: $3, 5, 7$.\\
Wielomian jest stopnia nieparzystego, ponadto znak współczynnika przy\linebreak najwyższej potędze x jest dodatni.\\ W związku z tym wykres wielomianu zaczyna się od lewej strony poniżej osi OX. A więc $$x \in [3,5] \cup [7,\infty).$$
\rozwStop
\odpStart
$x \in [3,5] \cup [7,\infty)$
\odpStop
\testStart
A.$x \in [3,5] \cup [7,\infty)$\\
B.$x \in (3,5) \cup [7,\infty)$\\
C.$x \in (3,5] \cup [7,\infty)$\\
D.$x \in [3,5) \cup [7,\infty)$\\
E.$x \in [3,5] \cup (7,\infty)$\\
F.$x \in (3,5) \cup (7,\infty)$\\
G.$x \in [3,5) \cup (7,\infty)$\\
H.$x \in (3,5] \cup (7,\infty)$
\testStop
\kluczStart
A
\kluczStop



\zadStart{Zadanie z Wikieł Z 1.62 a) moja wersja nr 163}

Rozwiązać nierówności $(x-3)(x-5)(x-8)\ge0$.
\zadStop
\rozwStart{Patryk Wirkus}{Laura Mieczkowska}
Miejsca zerowe naszego wielomianu to: $3, 5, 8$.\\
Wielomian jest stopnia nieparzystego, ponadto znak współczynnika przy\linebreak najwyższej potędze x jest dodatni.\\ W związku z tym wykres wielomianu zaczyna się od lewej strony poniżej osi OX. A więc $$x \in [3,5] \cup [8,\infty).$$
\rozwStop
\odpStart
$x \in [3,5] \cup [8,\infty)$
\odpStop
\testStart
A.$x \in [3,5] \cup [8,\infty)$\\
B.$x \in (3,5) \cup [8,\infty)$\\
C.$x \in (3,5] \cup [8,\infty)$\\
D.$x \in [3,5) \cup [8,\infty)$\\
E.$x \in [3,5] \cup (8,\infty)$\\
F.$x \in (3,5) \cup (8,\infty)$\\
G.$x \in [3,5) \cup (8,\infty)$\\
H.$x \in (3,5] \cup (8,\infty)$
\testStop
\kluczStart
A
\kluczStop



\zadStart{Zadanie z Wikieł Z 1.62 a) moja wersja nr 164}

Rozwiązać nierówności $(x-3)(x-5)(x-9)\ge0$.
\zadStop
\rozwStart{Patryk Wirkus}{Laura Mieczkowska}
Miejsca zerowe naszego wielomianu to: $3, 5, 9$.\\
Wielomian jest stopnia nieparzystego, ponadto znak współczynnika przy\linebreak najwyższej potędze x jest dodatni.\\ W związku z tym wykres wielomianu zaczyna się od lewej strony poniżej osi OX. A więc $$x \in [3,5] \cup [9,\infty).$$
\rozwStop
\odpStart
$x \in [3,5] \cup [9,\infty)$
\odpStop
\testStart
A.$x \in [3,5] \cup [9,\infty)$\\
B.$x \in (3,5) \cup [9,\infty)$\\
C.$x \in (3,5] \cup [9,\infty)$\\
D.$x \in [3,5) \cup [9,\infty)$\\
E.$x \in [3,5] \cup (9,\infty)$\\
F.$x \in (3,5) \cup (9,\infty)$\\
G.$x \in [3,5) \cup (9,\infty)$\\
H.$x \in (3,5] \cup (9,\infty)$
\testStop
\kluczStart
A
\kluczStop



\zadStart{Zadanie z Wikieł Z 1.62 a) moja wersja nr 165}

Rozwiązać nierówności $(x-3)(x-5)(x-10)\ge0$.
\zadStop
\rozwStart{Patryk Wirkus}{Laura Mieczkowska}
Miejsca zerowe naszego wielomianu to: $3, 5, 10$.\\
Wielomian jest stopnia nieparzystego, ponadto znak współczynnika przy\linebreak najwyższej potędze x jest dodatni.\\ W związku z tym wykres wielomianu zaczyna się od lewej strony poniżej osi OX. A więc $$x \in [3,5] \cup [10,\infty).$$
\rozwStop
\odpStart
$x \in [3,5] \cup [10,\infty)$
\odpStop
\testStart
A.$x \in [3,5] \cup [10,\infty)$\\
B.$x \in (3,5) \cup [10,\infty)$\\
C.$x \in (3,5] \cup [10,\infty)$\\
D.$x \in [3,5) \cup [10,\infty)$\\
E.$x \in [3,5] \cup (10,\infty)$\\
F.$x \in (3,5) \cup (10,\infty)$\\
G.$x \in [3,5) \cup (10,\infty)$\\
H.$x \in (3,5] \cup (10,\infty)$
\testStop
\kluczStart
A
\kluczStop



\zadStart{Zadanie z Wikieł Z 1.62 a) moja wersja nr 166}

Rozwiązać nierówności $(x-3)(x-5)(x-11)\ge0$.
\zadStop
\rozwStart{Patryk Wirkus}{Laura Mieczkowska}
Miejsca zerowe naszego wielomianu to: $3, 5, 11$.\\
Wielomian jest stopnia nieparzystego, ponadto znak współczynnika przy\linebreak najwyższej potędze x jest dodatni.\\ W związku z tym wykres wielomianu zaczyna się od lewej strony poniżej osi OX. A więc $$x \in [3,5] \cup [11,\infty).$$
\rozwStop
\odpStart
$x \in [3,5] \cup [11,\infty)$
\odpStop
\testStart
A.$x \in [3,5] \cup [11,\infty)$\\
B.$x \in (3,5) \cup [11,\infty)$\\
C.$x \in (3,5] \cup [11,\infty)$\\
D.$x \in [3,5) \cup [11,\infty)$\\
E.$x \in [3,5] \cup (11,\infty)$\\
F.$x \in (3,5) \cup (11,\infty)$\\
G.$x \in [3,5) \cup (11,\infty)$\\
H.$x \in (3,5] \cup (11,\infty)$
\testStop
\kluczStart
A
\kluczStop



\zadStart{Zadanie z Wikieł Z 1.62 a) moja wersja nr 167}

Rozwiązać nierówności $(x-3)(x-5)(x-12)\ge0$.
\zadStop
\rozwStart{Patryk Wirkus}{Laura Mieczkowska}
Miejsca zerowe naszego wielomianu to: $3, 5, 12$.\\
Wielomian jest stopnia nieparzystego, ponadto znak współczynnika przy\linebreak najwyższej potędze x jest dodatni.\\ W związku z tym wykres wielomianu zaczyna się od lewej strony poniżej osi OX. A więc $$x \in [3,5] \cup [12,\infty).$$
\rozwStop
\odpStart
$x \in [3,5] \cup [12,\infty)$
\odpStop
\testStart
A.$x \in [3,5] \cup [12,\infty)$\\
B.$x \in (3,5) \cup [12,\infty)$\\
C.$x \in (3,5] \cup [12,\infty)$\\
D.$x \in [3,5) \cup [12,\infty)$\\
E.$x \in [3,5] \cup (12,\infty)$\\
F.$x \in (3,5) \cup (12,\infty)$\\
G.$x \in [3,5) \cup (12,\infty)$\\
H.$x \in (3,5] \cup (12,\infty)$
\testStop
\kluczStart
A
\kluczStop



\zadStart{Zadanie z Wikieł Z 1.62 a) moja wersja nr 168}

Rozwiązać nierówności $(x-3)(x-5)(x-13)\ge0$.
\zadStop
\rozwStart{Patryk Wirkus}{Laura Mieczkowska}
Miejsca zerowe naszego wielomianu to: $3, 5, 13$.\\
Wielomian jest stopnia nieparzystego, ponadto znak współczynnika przy\linebreak najwyższej potędze x jest dodatni.\\ W związku z tym wykres wielomianu zaczyna się od lewej strony poniżej osi OX. A więc $$x \in [3,5] \cup [13,\infty).$$
\rozwStop
\odpStart
$x \in [3,5] \cup [13,\infty)$
\odpStop
\testStart
A.$x \in [3,5] \cup [13,\infty)$\\
B.$x \in (3,5) \cup [13,\infty)$\\
C.$x \in (3,5] \cup [13,\infty)$\\
D.$x \in [3,5) \cup [13,\infty)$\\
E.$x \in [3,5] \cup (13,\infty)$\\
F.$x \in (3,5) \cup (13,\infty)$\\
G.$x \in [3,5) \cup (13,\infty)$\\
H.$x \in (3,5] \cup (13,\infty)$
\testStop
\kluczStart
A
\kluczStop



\zadStart{Zadanie z Wikieł Z 1.62 a) moja wersja nr 169}

Rozwiązać nierówności $(x-3)(x-5)(x-14)\ge0$.
\zadStop
\rozwStart{Patryk Wirkus}{Laura Mieczkowska}
Miejsca zerowe naszego wielomianu to: $3, 5, 14$.\\
Wielomian jest stopnia nieparzystego, ponadto znak współczynnika przy\linebreak najwyższej potędze x jest dodatni.\\ W związku z tym wykres wielomianu zaczyna się od lewej strony poniżej osi OX. A więc $$x \in [3,5] \cup [14,\infty).$$
\rozwStop
\odpStart
$x \in [3,5] \cup [14,\infty)$
\odpStop
\testStart
A.$x \in [3,5] \cup [14,\infty)$\\
B.$x \in (3,5) \cup [14,\infty)$\\
C.$x \in (3,5] \cup [14,\infty)$\\
D.$x \in [3,5) \cup [14,\infty)$\\
E.$x \in [3,5] \cup (14,\infty)$\\
F.$x \in (3,5) \cup (14,\infty)$\\
G.$x \in [3,5) \cup (14,\infty)$\\
H.$x \in (3,5] \cup (14,\infty)$
\testStop
\kluczStart
A
\kluczStop



\zadStart{Zadanie z Wikieł Z 1.62 a) moja wersja nr 170}

Rozwiązać nierówności $(x-3)(x-5)(x-15)\ge0$.
\zadStop
\rozwStart{Patryk Wirkus}{Laura Mieczkowska}
Miejsca zerowe naszego wielomianu to: $3, 5, 15$.\\
Wielomian jest stopnia nieparzystego, ponadto znak współczynnika przy\linebreak najwyższej potędze x jest dodatni.\\ W związku z tym wykres wielomianu zaczyna się od lewej strony poniżej osi OX. A więc $$x \in [3,5] \cup [15,\infty).$$
\rozwStop
\odpStart
$x \in [3,5] \cup [15,\infty)$
\odpStop
\testStart
A.$x \in [3,5] \cup [15,\infty)$\\
B.$x \in (3,5) \cup [15,\infty)$\\
C.$x \in (3,5] \cup [15,\infty)$\\
D.$x \in [3,5) \cup [15,\infty)$\\
E.$x \in [3,5] \cup (15,\infty)$\\
F.$x \in (3,5) \cup (15,\infty)$\\
G.$x \in [3,5) \cup (15,\infty)$\\
H.$x \in (3,5] \cup (15,\infty)$
\testStop
\kluczStart
A
\kluczStop



\zadStart{Zadanie z Wikieł Z 1.62 a) moja wersja nr 171}

Rozwiązać nierówności $(x-3)(x-6)(x-7)\ge0$.
\zadStop
\rozwStart{Patryk Wirkus}{Laura Mieczkowska}
Miejsca zerowe naszego wielomianu to: $3, 6, 7$.\\
Wielomian jest stopnia nieparzystego, ponadto znak współczynnika przy\linebreak najwyższej potędze x jest dodatni.\\ W związku z tym wykres wielomianu zaczyna się od lewej strony poniżej osi OX. A więc $$x \in [3,6] \cup [7,\infty).$$
\rozwStop
\odpStart
$x \in [3,6] \cup [7,\infty)$
\odpStop
\testStart
A.$x \in [3,6] \cup [7,\infty)$\\
B.$x \in (3,6) \cup [7,\infty)$\\
C.$x \in (3,6] \cup [7,\infty)$\\
D.$x \in [3,6) \cup [7,\infty)$\\
E.$x \in [3,6] \cup (7,\infty)$\\
F.$x \in (3,6) \cup (7,\infty)$\\
G.$x \in [3,6) \cup (7,\infty)$\\
H.$x \in (3,6] \cup (7,\infty)$
\testStop
\kluczStart
A
\kluczStop



\zadStart{Zadanie z Wikieł Z 1.62 a) moja wersja nr 172}

Rozwiązać nierówności $(x-3)(x-6)(x-8)\ge0$.
\zadStop
\rozwStart{Patryk Wirkus}{Laura Mieczkowska}
Miejsca zerowe naszego wielomianu to: $3, 6, 8$.\\
Wielomian jest stopnia nieparzystego, ponadto znak współczynnika przy\linebreak najwyższej potędze x jest dodatni.\\ W związku z tym wykres wielomianu zaczyna się od lewej strony poniżej osi OX. A więc $$x \in [3,6] \cup [8,\infty).$$
\rozwStop
\odpStart
$x \in [3,6] \cup [8,\infty)$
\odpStop
\testStart
A.$x \in [3,6] \cup [8,\infty)$\\
B.$x \in (3,6) \cup [8,\infty)$\\
C.$x \in (3,6] \cup [8,\infty)$\\
D.$x \in [3,6) \cup [8,\infty)$\\
E.$x \in [3,6] \cup (8,\infty)$\\
F.$x \in (3,6) \cup (8,\infty)$\\
G.$x \in [3,6) \cup (8,\infty)$\\
H.$x \in (3,6] \cup (8,\infty)$
\testStop
\kluczStart
A
\kluczStop



\zadStart{Zadanie z Wikieł Z 1.62 a) moja wersja nr 173}

Rozwiązać nierówności $(x-3)(x-6)(x-9)\ge0$.
\zadStop
\rozwStart{Patryk Wirkus}{Laura Mieczkowska}
Miejsca zerowe naszego wielomianu to: $3, 6, 9$.\\
Wielomian jest stopnia nieparzystego, ponadto znak współczynnika przy\linebreak najwyższej potędze x jest dodatni.\\ W związku z tym wykres wielomianu zaczyna się od lewej strony poniżej osi OX. A więc $$x \in [3,6] \cup [9,\infty).$$
\rozwStop
\odpStart
$x \in [3,6] \cup [9,\infty)$
\odpStop
\testStart
A.$x \in [3,6] \cup [9,\infty)$\\
B.$x \in (3,6) \cup [9,\infty)$\\
C.$x \in (3,6] \cup [9,\infty)$\\
D.$x \in [3,6) \cup [9,\infty)$\\
E.$x \in [3,6] \cup (9,\infty)$\\
F.$x \in (3,6) \cup (9,\infty)$\\
G.$x \in [3,6) \cup (9,\infty)$\\
H.$x \in (3,6] \cup (9,\infty)$
\testStop
\kluczStart
A
\kluczStop



\zadStart{Zadanie z Wikieł Z 1.62 a) moja wersja nr 174}

Rozwiązać nierówności $(x-3)(x-6)(x-10)\ge0$.
\zadStop
\rozwStart{Patryk Wirkus}{Laura Mieczkowska}
Miejsca zerowe naszego wielomianu to: $3, 6, 10$.\\
Wielomian jest stopnia nieparzystego, ponadto znak współczynnika przy\linebreak najwyższej potędze x jest dodatni.\\ W związku z tym wykres wielomianu zaczyna się od lewej strony poniżej osi OX. A więc $$x \in [3,6] \cup [10,\infty).$$
\rozwStop
\odpStart
$x \in [3,6] \cup [10,\infty)$
\odpStop
\testStart
A.$x \in [3,6] \cup [10,\infty)$\\
B.$x \in (3,6) \cup [10,\infty)$\\
C.$x \in (3,6] \cup [10,\infty)$\\
D.$x \in [3,6) \cup [10,\infty)$\\
E.$x \in [3,6] \cup (10,\infty)$\\
F.$x \in (3,6) \cup (10,\infty)$\\
G.$x \in [3,6) \cup (10,\infty)$\\
H.$x \in (3,6] \cup (10,\infty)$
\testStop
\kluczStart
A
\kluczStop



\zadStart{Zadanie z Wikieł Z 1.62 a) moja wersja nr 175}

Rozwiązać nierówności $(x-3)(x-6)(x-11)\ge0$.
\zadStop
\rozwStart{Patryk Wirkus}{Laura Mieczkowska}
Miejsca zerowe naszego wielomianu to: $3, 6, 11$.\\
Wielomian jest stopnia nieparzystego, ponadto znak współczynnika przy\linebreak najwyższej potędze x jest dodatni.\\ W związku z tym wykres wielomianu zaczyna się od lewej strony poniżej osi OX. A więc $$x \in [3,6] \cup [11,\infty).$$
\rozwStop
\odpStart
$x \in [3,6] \cup [11,\infty)$
\odpStop
\testStart
A.$x \in [3,6] \cup [11,\infty)$\\
B.$x \in (3,6) \cup [11,\infty)$\\
C.$x \in (3,6] \cup [11,\infty)$\\
D.$x \in [3,6) \cup [11,\infty)$\\
E.$x \in [3,6] \cup (11,\infty)$\\
F.$x \in (3,6) \cup (11,\infty)$\\
G.$x \in [3,6) \cup (11,\infty)$\\
H.$x \in (3,6] \cup (11,\infty)$
\testStop
\kluczStart
A
\kluczStop



\zadStart{Zadanie z Wikieł Z 1.62 a) moja wersja nr 176}

Rozwiązać nierówności $(x-3)(x-6)(x-12)\ge0$.
\zadStop
\rozwStart{Patryk Wirkus}{Laura Mieczkowska}
Miejsca zerowe naszego wielomianu to: $3, 6, 12$.\\
Wielomian jest stopnia nieparzystego, ponadto znak współczynnika przy\linebreak najwyższej potędze x jest dodatni.\\ W związku z tym wykres wielomianu zaczyna się od lewej strony poniżej osi OX. A więc $$x \in [3,6] \cup [12,\infty).$$
\rozwStop
\odpStart
$x \in [3,6] \cup [12,\infty)$
\odpStop
\testStart
A.$x \in [3,6] \cup [12,\infty)$\\
B.$x \in (3,6) \cup [12,\infty)$\\
C.$x \in (3,6] \cup [12,\infty)$\\
D.$x \in [3,6) \cup [12,\infty)$\\
E.$x \in [3,6] \cup (12,\infty)$\\
F.$x \in (3,6) \cup (12,\infty)$\\
G.$x \in [3,6) \cup (12,\infty)$\\
H.$x \in (3,6] \cup (12,\infty)$
\testStop
\kluczStart
A
\kluczStop



\zadStart{Zadanie z Wikieł Z 1.62 a) moja wersja nr 177}

Rozwiązać nierówności $(x-3)(x-6)(x-13)\ge0$.
\zadStop
\rozwStart{Patryk Wirkus}{Laura Mieczkowska}
Miejsca zerowe naszego wielomianu to: $3, 6, 13$.\\
Wielomian jest stopnia nieparzystego, ponadto znak współczynnika przy\linebreak najwyższej potędze x jest dodatni.\\ W związku z tym wykres wielomianu zaczyna się od lewej strony poniżej osi OX. A więc $$x \in [3,6] \cup [13,\infty).$$
\rozwStop
\odpStart
$x \in [3,6] \cup [13,\infty)$
\odpStop
\testStart
A.$x \in [3,6] \cup [13,\infty)$\\
B.$x \in (3,6) \cup [13,\infty)$\\
C.$x \in (3,6] \cup [13,\infty)$\\
D.$x \in [3,6) \cup [13,\infty)$\\
E.$x \in [3,6] \cup (13,\infty)$\\
F.$x \in (3,6) \cup (13,\infty)$\\
G.$x \in [3,6) \cup (13,\infty)$\\
H.$x \in (3,6] \cup (13,\infty)$
\testStop
\kluczStart
A
\kluczStop



\zadStart{Zadanie z Wikieł Z 1.62 a) moja wersja nr 178}

Rozwiązać nierówności $(x-3)(x-6)(x-14)\ge0$.
\zadStop
\rozwStart{Patryk Wirkus}{Laura Mieczkowska}
Miejsca zerowe naszego wielomianu to: $3, 6, 14$.\\
Wielomian jest stopnia nieparzystego, ponadto znak współczynnika przy\linebreak najwyższej potędze x jest dodatni.\\ W związku z tym wykres wielomianu zaczyna się od lewej strony poniżej osi OX. A więc $$x \in [3,6] \cup [14,\infty).$$
\rozwStop
\odpStart
$x \in [3,6] \cup [14,\infty)$
\odpStop
\testStart
A.$x \in [3,6] \cup [14,\infty)$\\
B.$x \in (3,6) \cup [14,\infty)$\\
C.$x \in (3,6] \cup [14,\infty)$\\
D.$x \in [3,6) \cup [14,\infty)$\\
E.$x \in [3,6] \cup (14,\infty)$\\
F.$x \in (3,6) \cup (14,\infty)$\\
G.$x \in [3,6) \cup (14,\infty)$\\
H.$x \in (3,6] \cup (14,\infty)$
\testStop
\kluczStart
A
\kluczStop



\zadStart{Zadanie z Wikieł Z 1.62 a) moja wersja nr 179}

Rozwiązać nierówności $(x-3)(x-6)(x-15)\ge0$.
\zadStop
\rozwStart{Patryk Wirkus}{Laura Mieczkowska}
Miejsca zerowe naszego wielomianu to: $3, 6, 15$.\\
Wielomian jest stopnia nieparzystego, ponadto znak współczynnika przy\linebreak najwyższej potędze x jest dodatni.\\ W związku z tym wykres wielomianu zaczyna się od lewej strony poniżej osi OX. A więc $$x \in [3,6] \cup [15,\infty).$$
\rozwStop
\odpStart
$x \in [3,6] \cup [15,\infty)$
\odpStop
\testStart
A.$x \in [3,6] \cup [15,\infty)$\\
B.$x \in (3,6) \cup [15,\infty)$\\
C.$x \in (3,6] \cup [15,\infty)$\\
D.$x \in [3,6) \cup [15,\infty)$\\
E.$x \in [3,6] \cup (15,\infty)$\\
F.$x \in (3,6) \cup (15,\infty)$\\
G.$x \in [3,6) \cup (15,\infty)$\\
H.$x \in (3,6] \cup (15,\infty)$
\testStop
\kluczStart
A
\kluczStop



\zadStart{Zadanie z Wikieł Z 1.62 a) moja wersja nr 180}

Rozwiązać nierówności $(x-3)(x-7)(x-8)\ge0$.
\zadStop
\rozwStart{Patryk Wirkus}{Laura Mieczkowska}
Miejsca zerowe naszego wielomianu to: $3, 7, 8$.\\
Wielomian jest stopnia nieparzystego, ponadto znak współczynnika przy\linebreak najwyższej potędze x jest dodatni.\\ W związku z tym wykres wielomianu zaczyna się od lewej strony poniżej osi OX. A więc $$x \in [3,7] \cup [8,\infty).$$
\rozwStop
\odpStart
$x \in [3,7] \cup [8,\infty)$
\odpStop
\testStart
A.$x \in [3,7] \cup [8,\infty)$\\
B.$x \in (3,7) \cup [8,\infty)$\\
C.$x \in (3,7] \cup [8,\infty)$\\
D.$x \in [3,7) \cup [8,\infty)$\\
E.$x \in [3,7] \cup (8,\infty)$\\
F.$x \in (3,7) \cup (8,\infty)$\\
G.$x \in [3,7) \cup (8,\infty)$\\
H.$x \in (3,7] \cup (8,\infty)$
\testStop
\kluczStart
A
\kluczStop



\zadStart{Zadanie z Wikieł Z 1.62 a) moja wersja nr 181}

Rozwiązać nierówności $(x-3)(x-7)(x-9)\ge0$.
\zadStop
\rozwStart{Patryk Wirkus}{Laura Mieczkowska}
Miejsca zerowe naszego wielomianu to: $3, 7, 9$.\\
Wielomian jest stopnia nieparzystego, ponadto znak współczynnika przy\linebreak najwyższej potędze x jest dodatni.\\ W związku z tym wykres wielomianu zaczyna się od lewej strony poniżej osi OX. A więc $$x \in [3,7] \cup [9,\infty).$$
\rozwStop
\odpStart
$x \in [3,7] \cup [9,\infty)$
\odpStop
\testStart
A.$x \in [3,7] \cup [9,\infty)$\\
B.$x \in (3,7) \cup [9,\infty)$\\
C.$x \in (3,7] \cup [9,\infty)$\\
D.$x \in [3,7) \cup [9,\infty)$\\
E.$x \in [3,7] \cup (9,\infty)$\\
F.$x \in (3,7) \cup (9,\infty)$\\
G.$x \in [3,7) \cup (9,\infty)$\\
H.$x \in (3,7] \cup (9,\infty)$
\testStop
\kluczStart
A
\kluczStop



\zadStart{Zadanie z Wikieł Z 1.62 a) moja wersja nr 182}

Rozwiązać nierówności $(x-3)(x-7)(x-10)\ge0$.
\zadStop
\rozwStart{Patryk Wirkus}{Laura Mieczkowska}
Miejsca zerowe naszego wielomianu to: $3, 7, 10$.\\
Wielomian jest stopnia nieparzystego, ponadto znak współczynnika przy\linebreak najwyższej potędze x jest dodatni.\\ W związku z tym wykres wielomianu zaczyna się od lewej strony poniżej osi OX. A więc $$x \in [3,7] \cup [10,\infty).$$
\rozwStop
\odpStart
$x \in [3,7] \cup [10,\infty)$
\odpStop
\testStart
A.$x \in [3,7] \cup [10,\infty)$\\
B.$x \in (3,7) \cup [10,\infty)$\\
C.$x \in (3,7] \cup [10,\infty)$\\
D.$x \in [3,7) \cup [10,\infty)$\\
E.$x \in [3,7] \cup (10,\infty)$\\
F.$x \in (3,7) \cup (10,\infty)$\\
G.$x \in [3,7) \cup (10,\infty)$\\
H.$x \in (3,7] \cup (10,\infty)$
\testStop
\kluczStart
A
\kluczStop



\zadStart{Zadanie z Wikieł Z 1.62 a) moja wersja nr 183}

Rozwiązać nierówności $(x-3)(x-7)(x-11)\ge0$.
\zadStop
\rozwStart{Patryk Wirkus}{Laura Mieczkowska}
Miejsca zerowe naszego wielomianu to: $3, 7, 11$.\\
Wielomian jest stopnia nieparzystego, ponadto znak współczynnika przy\linebreak najwyższej potędze x jest dodatni.\\ W związku z tym wykres wielomianu zaczyna się od lewej strony poniżej osi OX. A więc $$x \in [3,7] \cup [11,\infty).$$
\rozwStop
\odpStart
$x \in [3,7] \cup [11,\infty)$
\odpStop
\testStart
A.$x \in [3,7] \cup [11,\infty)$\\
B.$x \in (3,7) \cup [11,\infty)$\\
C.$x \in (3,7] \cup [11,\infty)$\\
D.$x \in [3,7) \cup [11,\infty)$\\
E.$x \in [3,7] \cup (11,\infty)$\\
F.$x \in (3,7) \cup (11,\infty)$\\
G.$x \in [3,7) \cup (11,\infty)$\\
H.$x \in (3,7] \cup (11,\infty)$
\testStop
\kluczStart
A
\kluczStop



\zadStart{Zadanie z Wikieł Z 1.62 a) moja wersja nr 184}

Rozwiązać nierówności $(x-3)(x-7)(x-12)\ge0$.
\zadStop
\rozwStart{Patryk Wirkus}{Laura Mieczkowska}
Miejsca zerowe naszego wielomianu to: $3, 7, 12$.\\
Wielomian jest stopnia nieparzystego, ponadto znak współczynnika przy\linebreak najwyższej potędze x jest dodatni.\\ W związku z tym wykres wielomianu zaczyna się od lewej strony poniżej osi OX. A więc $$x \in [3,7] \cup [12,\infty).$$
\rozwStop
\odpStart
$x \in [3,7] \cup [12,\infty)$
\odpStop
\testStart
A.$x \in [3,7] \cup [12,\infty)$\\
B.$x \in (3,7) \cup [12,\infty)$\\
C.$x \in (3,7] \cup [12,\infty)$\\
D.$x \in [3,7) \cup [12,\infty)$\\
E.$x \in [3,7] \cup (12,\infty)$\\
F.$x \in (3,7) \cup (12,\infty)$\\
G.$x \in [3,7) \cup (12,\infty)$\\
H.$x \in (3,7] \cup (12,\infty)$
\testStop
\kluczStart
A
\kluczStop



\zadStart{Zadanie z Wikieł Z 1.62 a) moja wersja nr 185}

Rozwiązać nierówności $(x-3)(x-7)(x-13)\ge0$.
\zadStop
\rozwStart{Patryk Wirkus}{Laura Mieczkowska}
Miejsca zerowe naszego wielomianu to: $3, 7, 13$.\\
Wielomian jest stopnia nieparzystego, ponadto znak współczynnika przy\linebreak najwyższej potędze x jest dodatni.\\ W związku z tym wykres wielomianu zaczyna się od lewej strony poniżej osi OX. A więc $$x \in [3,7] \cup [13,\infty).$$
\rozwStop
\odpStart
$x \in [3,7] \cup [13,\infty)$
\odpStop
\testStart
A.$x \in [3,7] \cup [13,\infty)$\\
B.$x \in (3,7) \cup [13,\infty)$\\
C.$x \in (3,7] \cup [13,\infty)$\\
D.$x \in [3,7) \cup [13,\infty)$\\
E.$x \in [3,7] \cup (13,\infty)$\\
F.$x \in (3,7) \cup (13,\infty)$\\
G.$x \in [3,7) \cup (13,\infty)$\\
H.$x \in (3,7] \cup (13,\infty)$
\testStop
\kluczStart
A
\kluczStop



\zadStart{Zadanie z Wikieł Z 1.62 a) moja wersja nr 186}

Rozwiązać nierówności $(x-3)(x-7)(x-14)\ge0$.
\zadStop
\rozwStart{Patryk Wirkus}{Laura Mieczkowska}
Miejsca zerowe naszego wielomianu to: $3, 7, 14$.\\
Wielomian jest stopnia nieparzystego, ponadto znak współczynnika przy\linebreak najwyższej potędze x jest dodatni.\\ W związku z tym wykres wielomianu zaczyna się od lewej strony poniżej osi OX. A więc $$x \in [3,7] \cup [14,\infty).$$
\rozwStop
\odpStart
$x \in [3,7] \cup [14,\infty)$
\odpStop
\testStart
A.$x \in [3,7] \cup [14,\infty)$\\
B.$x \in (3,7) \cup [14,\infty)$\\
C.$x \in (3,7] \cup [14,\infty)$\\
D.$x \in [3,7) \cup [14,\infty)$\\
E.$x \in [3,7] \cup (14,\infty)$\\
F.$x \in (3,7) \cup (14,\infty)$\\
G.$x \in [3,7) \cup (14,\infty)$\\
H.$x \in (3,7] \cup (14,\infty)$
\testStop
\kluczStart
A
\kluczStop



\zadStart{Zadanie z Wikieł Z 1.62 a) moja wersja nr 187}

Rozwiązać nierówności $(x-3)(x-7)(x-15)\ge0$.
\zadStop
\rozwStart{Patryk Wirkus}{Laura Mieczkowska}
Miejsca zerowe naszego wielomianu to: $3, 7, 15$.\\
Wielomian jest stopnia nieparzystego, ponadto znak współczynnika przy\linebreak najwyższej potędze x jest dodatni.\\ W związku z tym wykres wielomianu zaczyna się od lewej strony poniżej osi OX. A więc $$x \in [3,7] \cup [15,\infty).$$
\rozwStop
\odpStart
$x \in [3,7] \cup [15,\infty)$
\odpStop
\testStart
A.$x \in [3,7] \cup [15,\infty)$\\
B.$x \in (3,7) \cup [15,\infty)$\\
C.$x \in (3,7] \cup [15,\infty)$\\
D.$x \in [3,7) \cup [15,\infty)$\\
E.$x \in [3,7] \cup (15,\infty)$\\
F.$x \in (3,7) \cup (15,\infty)$\\
G.$x \in [3,7) \cup (15,\infty)$\\
H.$x \in (3,7] \cup (15,\infty)$
\testStop
\kluczStart
A
\kluczStop



\zadStart{Zadanie z Wikieł Z 1.62 a) moja wersja nr 188}

Rozwiązać nierówności $(x-3)(x-8)(x-9)\ge0$.
\zadStop
\rozwStart{Patryk Wirkus}{Laura Mieczkowska}
Miejsca zerowe naszego wielomianu to: $3, 8, 9$.\\
Wielomian jest stopnia nieparzystego, ponadto znak współczynnika przy\linebreak najwyższej potędze x jest dodatni.\\ W związku z tym wykres wielomianu zaczyna się od lewej strony poniżej osi OX. A więc $$x \in [3,8] \cup [9,\infty).$$
\rozwStop
\odpStart
$x \in [3,8] \cup [9,\infty)$
\odpStop
\testStart
A.$x \in [3,8] \cup [9,\infty)$\\
B.$x \in (3,8) \cup [9,\infty)$\\
C.$x \in (3,8] \cup [9,\infty)$\\
D.$x \in [3,8) \cup [9,\infty)$\\
E.$x \in [3,8] \cup (9,\infty)$\\
F.$x \in (3,8) \cup (9,\infty)$\\
G.$x \in [3,8) \cup (9,\infty)$\\
H.$x \in (3,8] \cup (9,\infty)$
\testStop
\kluczStart
A
\kluczStop



\zadStart{Zadanie z Wikieł Z 1.62 a) moja wersja nr 189}

Rozwiązać nierówności $(x-3)(x-8)(x-10)\ge0$.
\zadStop
\rozwStart{Patryk Wirkus}{Laura Mieczkowska}
Miejsca zerowe naszego wielomianu to: $3, 8, 10$.\\
Wielomian jest stopnia nieparzystego, ponadto znak współczynnika przy\linebreak najwyższej potędze x jest dodatni.\\ W związku z tym wykres wielomianu zaczyna się od lewej strony poniżej osi OX. A więc $$x \in [3,8] \cup [10,\infty).$$
\rozwStop
\odpStart
$x \in [3,8] \cup [10,\infty)$
\odpStop
\testStart
A.$x \in [3,8] \cup [10,\infty)$\\
B.$x \in (3,8) \cup [10,\infty)$\\
C.$x \in (3,8] \cup [10,\infty)$\\
D.$x \in [3,8) \cup [10,\infty)$\\
E.$x \in [3,8] \cup (10,\infty)$\\
F.$x \in (3,8) \cup (10,\infty)$\\
G.$x \in [3,8) \cup (10,\infty)$\\
H.$x \in (3,8] \cup (10,\infty)$
\testStop
\kluczStart
A
\kluczStop



\zadStart{Zadanie z Wikieł Z 1.62 a) moja wersja nr 190}

Rozwiązać nierówności $(x-3)(x-8)(x-11)\ge0$.
\zadStop
\rozwStart{Patryk Wirkus}{Laura Mieczkowska}
Miejsca zerowe naszego wielomianu to: $3, 8, 11$.\\
Wielomian jest stopnia nieparzystego, ponadto znak współczynnika przy\linebreak najwyższej potędze x jest dodatni.\\ W związku z tym wykres wielomianu zaczyna się od lewej strony poniżej osi OX. A więc $$x \in [3,8] \cup [11,\infty).$$
\rozwStop
\odpStart
$x \in [3,8] \cup [11,\infty)$
\odpStop
\testStart
A.$x \in [3,8] \cup [11,\infty)$\\
B.$x \in (3,8) \cup [11,\infty)$\\
C.$x \in (3,8] \cup [11,\infty)$\\
D.$x \in [3,8) \cup [11,\infty)$\\
E.$x \in [3,8] \cup (11,\infty)$\\
F.$x \in (3,8) \cup (11,\infty)$\\
G.$x \in [3,8) \cup (11,\infty)$\\
H.$x \in (3,8] \cup (11,\infty)$
\testStop
\kluczStart
A
\kluczStop



\zadStart{Zadanie z Wikieł Z 1.62 a) moja wersja nr 191}

Rozwiązać nierówności $(x-3)(x-8)(x-12)\ge0$.
\zadStop
\rozwStart{Patryk Wirkus}{Laura Mieczkowska}
Miejsca zerowe naszego wielomianu to: $3, 8, 12$.\\
Wielomian jest stopnia nieparzystego, ponadto znak współczynnika przy\linebreak najwyższej potędze x jest dodatni.\\ W związku z tym wykres wielomianu zaczyna się od lewej strony poniżej osi OX. A więc $$x \in [3,8] \cup [12,\infty).$$
\rozwStop
\odpStart
$x \in [3,8] \cup [12,\infty)$
\odpStop
\testStart
A.$x \in [3,8] \cup [12,\infty)$\\
B.$x \in (3,8) \cup [12,\infty)$\\
C.$x \in (3,8] \cup [12,\infty)$\\
D.$x \in [3,8) \cup [12,\infty)$\\
E.$x \in [3,8] \cup (12,\infty)$\\
F.$x \in (3,8) \cup (12,\infty)$\\
G.$x \in [3,8) \cup (12,\infty)$\\
H.$x \in (3,8] \cup (12,\infty)$
\testStop
\kluczStart
A
\kluczStop



\zadStart{Zadanie z Wikieł Z 1.62 a) moja wersja nr 192}

Rozwiązać nierówności $(x-3)(x-8)(x-13)\ge0$.
\zadStop
\rozwStart{Patryk Wirkus}{Laura Mieczkowska}
Miejsca zerowe naszego wielomianu to: $3, 8, 13$.\\
Wielomian jest stopnia nieparzystego, ponadto znak współczynnika przy\linebreak najwyższej potędze x jest dodatni.\\ W związku z tym wykres wielomianu zaczyna się od lewej strony poniżej osi OX. A więc $$x \in [3,8] \cup [13,\infty).$$
\rozwStop
\odpStart
$x \in [3,8] \cup [13,\infty)$
\odpStop
\testStart
A.$x \in [3,8] \cup [13,\infty)$\\
B.$x \in (3,8) \cup [13,\infty)$\\
C.$x \in (3,8] \cup [13,\infty)$\\
D.$x \in [3,8) \cup [13,\infty)$\\
E.$x \in [3,8] \cup (13,\infty)$\\
F.$x \in (3,8) \cup (13,\infty)$\\
G.$x \in [3,8) \cup (13,\infty)$\\
H.$x \in (3,8] \cup (13,\infty)$
\testStop
\kluczStart
A
\kluczStop



\zadStart{Zadanie z Wikieł Z 1.62 a) moja wersja nr 193}

Rozwiązać nierówności $(x-3)(x-8)(x-14)\ge0$.
\zadStop
\rozwStart{Patryk Wirkus}{Laura Mieczkowska}
Miejsca zerowe naszego wielomianu to: $3, 8, 14$.\\
Wielomian jest stopnia nieparzystego, ponadto znak współczynnika przy\linebreak najwyższej potędze x jest dodatni.\\ W związku z tym wykres wielomianu zaczyna się od lewej strony poniżej osi OX. A więc $$x \in [3,8] \cup [14,\infty).$$
\rozwStop
\odpStart
$x \in [3,8] \cup [14,\infty)$
\odpStop
\testStart
A.$x \in [3,8] \cup [14,\infty)$\\
B.$x \in (3,8) \cup [14,\infty)$\\
C.$x \in (3,8] \cup [14,\infty)$\\
D.$x \in [3,8) \cup [14,\infty)$\\
E.$x \in [3,8] \cup (14,\infty)$\\
F.$x \in (3,8) \cup (14,\infty)$\\
G.$x \in [3,8) \cup (14,\infty)$\\
H.$x \in (3,8] \cup (14,\infty)$
\testStop
\kluczStart
A
\kluczStop



\zadStart{Zadanie z Wikieł Z 1.62 a) moja wersja nr 194}

Rozwiązać nierówności $(x-3)(x-8)(x-15)\ge0$.
\zadStop
\rozwStart{Patryk Wirkus}{Laura Mieczkowska}
Miejsca zerowe naszego wielomianu to: $3, 8, 15$.\\
Wielomian jest stopnia nieparzystego, ponadto znak współczynnika przy\linebreak najwyższej potędze x jest dodatni.\\ W związku z tym wykres wielomianu zaczyna się od lewej strony poniżej osi OX. A więc $$x \in [3,8] \cup [15,\infty).$$
\rozwStop
\odpStart
$x \in [3,8] \cup [15,\infty)$
\odpStop
\testStart
A.$x \in [3,8] \cup [15,\infty)$\\
B.$x \in (3,8) \cup [15,\infty)$\\
C.$x \in (3,8] \cup [15,\infty)$\\
D.$x \in [3,8) \cup [15,\infty)$\\
E.$x \in [3,8] \cup (15,\infty)$\\
F.$x \in (3,8) \cup (15,\infty)$\\
G.$x \in [3,8) \cup (15,\infty)$\\
H.$x \in (3,8] \cup (15,\infty)$
\testStop
\kluczStart
A
\kluczStop



\zadStart{Zadanie z Wikieł Z 1.62 a) moja wersja nr 195}

Rozwiązać nierówności $(x-3)(x-9)(x-10)\ge0$.
\zadStop
\rozwStart{Patryk Wirkus}{Laura Mieczkowska}
Miejsca zerowe naszego wielomianu to: $3, 9, 10$.\\
Wielomian jest stopnia nieparzystego, ponadto znak współczynnika przy\linebreak najwyższej potędze x jest dodatni.\\ W związku z tym wykres wielomianu zaczyna się od lewej strony poniżej osi OX. A więc $$x \in [3,9] \cup [10,\infty).$$
\rozwStop
\odpStart
$x \in [3,9] \cup [10,\infty)$
\odpStop
\testStart
A.$x \in [3,9] \cup [10,\infty)$\\
B.$x \in (3,9) \cup [10,\infty)$\\
C.$x \in (3,9] \cup [10,\infty)$\\
D.$x \in [3,9) \cup [10,\infty)$\\
E.$x \in [3,9] \cup (10,\infty)$\\
F.$x \in (3,9) \cup (10,\infty)$\\
G.$x \in [3,9) \cup (10,\infty)$\\
H.$x \in (3,9] \cup (10,\infty)$
\testStop
\kluczStart
A
\kluczStop



\zadStart{Zadanie z Wikieł Z 1.62 a) moja wersja nr 196}

Rozwiązać nierówności $(x-3)(x-9)(x-11)\ge0$.
\zadStop
\rozwStart{Patryk Wirkus}{Laura Mieczkowska}
Miejsca zerowe naszego wielomianu to: $3, 9, 11$.\\
Wielomian jest stopnia nieparzystego, ponadto znak współczynnika przy\linebreak najwyższej potędze x jest dodatni.\\ W związku z tym wykres wielomianu zaczyna się od lewej strony poniżej osi OX. A więc $$x \in [3,9] \cup [11,\infty).$$
\rozwStop
\odpStart
$x \in [3,9] \cup [11,\infty)$
\odpStop
\testStart
A.$x \in [3,9] \cup [11,\infty)$\\
B.$x \in (3,9) \cup [11,\infty)$\\
C.$x \in (3,9] \cup [11,\infty)$\\
D.$x \in [3,9) \cup [11,\infty)$\\
E.$x \in [3,9] \cup (11,\infty)$\\
F.$x \in (3,9) \cup (11,\infty)$\\
G.$x \in [3,9) \cup (11,\infty)$\\
H.$x \in (3,9] \cup (11,\infty)$
\testStop
\kluczStart
A
\kluczStop



\zadStart{Zadanie z Wikieł Z 1.62 a) moja wersja nr 197}

Rozwiązać nierówności $(x-3)(x-9)(x-12)\ge0$.
\zadStop
\rozwStart{Patryk Wirkus}{Laura Mieczkowska}
Miejsca zerowe naszego wielomianu to: $3, 9, 12$.\\
Wielomian jest stopnia nieparzystego, ponadto znak współczynnika przy\linebreak najwyższej potędze x jest dodatni.\\ W związku z tym wykres wielomianu zaczyna się od lewej strony poniżej osi OX. A więc $$x \in [3,9] \cup [12,\infty).$$
\rozwStop
\odpStart
$x \in [3,9] \cup [12,\infty)$
\odpStop
\testStart
A.$x \in [3,9] \cup [12,\infty)$\\
B.$x \in (3,9) \cup [12,\infty)$\\
C.$x \in (3,9] \cup [12,\infty)$\\
D.$x \in [3,9) \cup [12,\infty)$\\
E.$x \in [3,9] \cup (12,\infty)$\\
F.$x \in (3,9) \cup (12,\infty)$\\
G.$x \in [3,9) \cup (12,\infty)$\\
H.$x \in (3,9] \cup (12,\infty)$
\testStop
\kluczStart
A
\kluczStop



\zadStart{Zadanie z Wikieł Z 1.62 a) moja wersja nr 198}

Rozwiązać nierówności $(x-3)(x-9)(x-13)\ge0$.
\zadStop
\rozwStart{Patryk Wirkus}{Laura Mieczkowska}
Miejsca zerowe naszego wielomianu to: $3, 9, 13$.\\
Wielomian jest stopnia nieparzystego, ponadto znak współczynnika przy\linebreak najwyższej potędze x jest dodatni.\\ W związku z tym wykres wielomianu zaczyna się od lewej strony poniżej osi OX. A więc $$x \in [3,9] \cup [13,\infty).$$
\rozwStop
\odpStart
$x \in [3,9] \cup [13,\infty)$
\odpStop
\testStart
A.$x \in [3,9] \cup [13,\infty)$\\
B.$x \in (3,9) \cup [13,\infty)$\\
C.$x \in (3,9] \cup [13,\infty)$\\
D.$x \in [3,9) \cup [13,\infty)$\\
E.$x \in [3,9] \cup (13,\infty)$\\
F.$x \in (3,9) \cup (13,\infty)$\\
G.$x \in [3,9) \cup (13,\infty)$\\
H.$x \in (3,9] \cup (13,\infty)$
\testStop
\kluczStart
A
\kluczStop



\zadStart{Zadanie z Wikieł Z 1.62 a) moja wersja nr 199}

Rozwiązać nierówności $(x-3)(x-9)(x-14)\ge0$.
\zadStop
\rozwStart{Patryk Wirkus}{Laura Mieczkowska}
Miejsca zerowe naszego wielomianu to: $3, 9, 14$.\\
Wielomian jest stopnia nieparzystego, ponadto znak współczynnika przy\linebreak najwyższej potędze x jest dodatni.\\ W związku z tym wykres wielomianu zaczyna się od lewej strony poniżej osi OX. A więc $$x \in [3,9] \cup [14,\infty).$$
\rozwStop
\odpStart
$x \in [3,9] \cup [14,\infty)$
\odpStop
\testStart
A.$x \in [3,9] \cup [14,\infty)$\\
B.$x \in (3,9) \cup [14,\infty)$\\
C.$x \in (3,9] \cup [14,\infty)$\\
D.$x \in [3,9) \cup [14,\infty)$\\
E.$x \in [3,9] \cup (14,\infty)$\\
F.$x \in (3,9) \cup (14,\infty)$\\
G.$x \in [3,9) \cup (14,\infty)$\\
H.$x \in (3,9] \cup (14,\infty)$
\testStop
\kluczStart
A
\kluczStop



\zadStart{Zadanie z Wikieł Z 1.62 a) moja wersja nr 200}

Rozwiązać nierówności $(x-3)(x-9)(x-15)\ge0$.
\zadStop
\rozwStart{Patryk Wirkus}{Laura Mieczkowska}
Miejsca zerowe naszego wielomianu to: $3, 9, 15$.\\
Wielomian jest stopnia nieparzystego, ponadto znak współczynnika przy\linebreak najwyższej potędze x jest dodatni.\\ W związku z tym wykres wielomianu zaczyna się od lewej strony poniżej osi OX. A więc $$x \in [3,9] \cup [15,\infty).$$
\rozwStop
\odpStart
$x \in [3,9] \cup [15,\infty)$
\odpStop
\testStart
A.$x \in [3,9] \cup [15,\infty)$\\
B.$x \in (3,9) \cup [15,\infty)$\\
C.$x \in (3,9] \cup [15,\infty)$\\
D.$x \in [3,9) \cup [15,\infty)$\\
E.$x \in [3,9] \cup (15,\infty)$\\
F.$x \in (3,9) \cup (15,\infty)$\\
G.$x \in [3,9) \cup (15,\infty)$\\
H.$x \in (3,9] \cup (15,\infty)$
\testStop
\kluczStart
A
\kluczStop



\zadStart{Zadanie z Wikieł Z 1.62 a) moja wersja nr 201}

Rozwiązać nierówności $(x-3)(x-10)(x-11)\ge0$.
\zadStop
\rozwStart{Patryk Wirkus}{Laura Mieczkowska}
Miejsca zerowe naszego wielomianu to: $3, 10, 11$.\\
Wielomian jest stopnia nieparzystego, ponadto znak współczynnika przy\linebreak najwyższej potędze x jest dodatni.\\ W związku z tym wykres wielomianu zaczyna się od lewej strony poniżej osi OX. A więc $$x \in [3,10] \cup [11,\infty).$$
\rozwStop
\odpStart
$x \in [3,10] \cup [11,\infty)$
\odpStop
\testStart
A.$x \in [3,10] \cup [11,\infty)$\\
B.$x \in (3,10) \cup [11,\infty)$\\
C.$x \in (3,10] \cup [11,\infty)$\\
D.$x \in [3,10) \cup [11,\infty)$\\
E.$x \in [3,10] \cup (11,\infty)$\\
F.$x \in (3,10) \cup (11,\infty)$\\
G.$x \in [3,10) \cup (11,\infty)$\\
H.$x \in (3,10] \cup (11,\infty)$
\testStop
\kluczStart
A
\kluczStop



\zadStart{Zadanie z Wikieł Z 1.62 a) moja wersja nr 202}

Rozwiązać nierówności $(x-3)(x-10)(x-12)\ge0$.
\zadStop
\rozwStart{Patryk Wirkus}{Laura Mieczkowska}
Miejsca zerowe naszego wielomianu to: $3, 10, 12$.\\
Wielomian jest stopnia nieparzystego, ponadto znak współczynnika przy\linebreak najwyższej potędze x jest dodatni.\\ W związku z tym wykres wielomianu zaczyna się od lewej strony poniżej osi OX. A więc $$x \in [3,10] \cup [12,\infty).$$
\rozwStop
\odpStart
$x \in [3,10] \cup [12,\infty)$
\odpStop
\testStart
A.$x \in [3,10] \cup [12,\infty)$\\
B.$x \in (3,10) \cup [12,\infty)$\\
C.$x \in (3,10] \cup [12,\infty)$\\
D.$x \in [3,10) \cup [12,\infty)$\\
E.$x \in [3,10] \cup (12,\infty)$\\
F.$x \in (3,10) \cup (12,\infty)$\\
G.$x \in [3,10) \cup (12,\infty)$\\
H.$x \in (3,10] \cup (12,\infty)$
\testStop
\kluczStart
A
\kluczStop



\zadStart{Zadanie z Wikieł Z 1.62 a) moja wersja nr 203}

Rozwiązać nierówności $(x-3)(x-10)(x-13)\ge0$.
\zadStop
\rozwStart{Patryk Wirkus}{Laura Mieczkowska}
Miejsca zerowe naszego wielomianu to: $3, 10, 13$.\\
Wielomian jest stopnia nieparzystego, ponadto znak współczynnika przy\linebreak najwyższej potędze x jest dodatni.\\ W związku z tym wykres wielomianu zaczyna się od lewej strony poniżej osi OX. A więc $$x \in [3,10] \cup [13,\infty).$$
\rozwStop
\odpStart
$x \in [3,10] \cup [13,\infty)$
\odpStop
\testStart
A.$x \in [3,10] \cup [13,\infty)$\\
B.$x \in (3,10) \cup [13,\infty)$\\
C.$x \in (3,10] \cup [13,\infty)$\\
D.$x \in [3,10) \cup [13,\infty)$\\
E.$x \in [3,10] \cup (13,\infty)$\\
F.$x \in (3,10) \cup (13,\infty)$\\
G.$x \in [3,10) \cup (13,\infty)$\\
H.$x \in (3,10] \cup (13,\infty)$
\testStop
\kluczStart
A
\kluczStop



\zadStart{Zadanie z Wikieł Z 1.62 a) moja wersja nr 204}

Rozwiązać nierówności $(x-3)(x-10)(x-14)\ge0$.
\zadStop
\rozwStart{Patryk Wirkus}{Laura Mieczkowska}
Miejsca zerowe naszego wielomianu to: $3, 10, 14$.\\
Wielomian jest stopnia nieparzystego, ponadto znak współczynnika przy\linebreak najwyższej potędze x jest dodatni.\\ W związku z tym wykres wielomianu zaczyna się od lewej strony poniżej osi OX. A więc $$x \in [3,10] \cup [14,\infty).$$
\rozwStop
\odpStart
$x \in [3,10] \cup [14,\infty)$
\odpStop
\testStart
A.$x \in [3,10] \cup [14,\infty)$\\
B.$x \in (3,10) \cup [14,\infty)$\\
C.$x \in (3,10] \cup [14,\infty)$\\
D.$x \in [3,10) \cup [14,\infty)$\\
E.$x \in [3,10] \cup (14,\infty)$\\
F.$x \in (3,10) \cup (14,\infty)$\\
G.$x \in [3,10) \cup (14,\infty)$\\
H.$x \in (3,10] \cup (14,\infty)$
\testStop
\kluczStart
A
\kluczStop



\zadStart{Zadanie z Wikieł Z 1.62 a) moja wersja nr 205}

Rozwiązać nierówności $(x-3)(x-10)(x-15)\ge0$.
\zadStop
\rozwStart{Patryk Wirkus}{Laura Mieczkowska}
Miejsca zerowe naszego wielomianu to: $3, 10, 15$.\\
Wielomian jest stopnia nieparzystego, ponadto znak współczynnika przy\linebreak najwyższej potędze x jest dodatni.\\ W związku z tym wykres wielomianu zaczyna się od lewej strony poniżej osi OX. A więc $$x \in [3,10] \cup [15,\infty).$$
\rozwStop
\odpStart
$x \in [3,10] \cup [15,\infty)$
\odpStop
\testStart
A.$x \in [3,10] \cup [15,\infty)$\\
B.$x \in (3,10) \cup [15,\infty)$\\
C.$x \in (3,10] \cup [15,\infty)$\\
D.$x \in [3,10) \cup [15,\infty)$\\
E.$x \in [3,10] \cup (15,\infty)$\\
F.$x \in (3,10) \cup (15,\infty)$\\
G.$x \in [3,10) \cup (15,\infty)$\\
H.$x \in (3,10] \cup (15,\infty)$
\testStop
\kluczStart
A
\kluczStop



\zadStart{Zadanie z Wikieł Z 1.62 a) moja wersja nr 206}

Rozwiązać nierówności $(x-4)(x-5)(x-6)\ge0$.
\zadStop
\rozwStart{Patryk Wirkus}{Laura Mieczkowska}
Miejsca zerowe naszego wielomianu to: $4, 5, 6$.\\
Wielomian jest stopnia nieparzystego, ponadto znak współczynnika przy\linebreak najwyższej potędze x jest dodatni.\\ W związku z tym wykres wielomianu zaczyna się od lewej strony poniżej osi OX. A więc $$x \in [4,5] \cup [6,\infty).$$
\rozwStop
\odpStart
$x \in [4,5] \cup [6,\infty)$
\odpStop
\testStart
A.$x \in [4,5] \cup [6,\infty)$\\
B.$x \in (4,5) \cup [6,\infty)$\\
C.$x \in (4,5] \cup [6,\infty)$\\
D.$x \in [4,5) \cup [6,\infty)$\\
E.$x \in [4,5] \cup (6,\infty)$\\
F.$x \in (4,5) \cup (6,\infty)$\\
G.$x \in [4,5) \cup (6,\infty)$\\
H.$x \in (4,5] \cup (6,\infty)$
\testStop
\kluczStart
A
\kluczStop



\zadStart{Zadanie z Wikieł Z 1.62 a) moja wersja nr 207}

Rozwiązać nierówności $(x-4)(x-5)(x-7)\ge0$.
\zadStop
\rozwStart{Patryk Wirkus}{Laura Mieczkowska}
Miejsca zerowe naszego wielomianu to: $4, 5, 7$.\\
Wielomian jest stopnia nieparzystego, ponadto znak współczynnika przy\linebreak najwyższej potędze x jest dodatni.\\ W związku z tym wykres wielomianu zaczyna się od lewej strony poniżej osi OX. A więc $$x \in [4,5] \cup [7,\infty).$$
\rozwStop
\odpStart
$x \in [4,5] \cup [7,\infty)$
\odpStop
\testStart
A.$x \in [4,5] \cup [7,\infty)$\\
B.$x \in (4,5) \cup [7,\infty)$\\
C.$x \in (4,5] \cup [7,\infty)$\\
D.$x \in [4,5) \cup [7,\infty)$\\
E.$x \in [4,5] \cup (7,\infty)$\\
F.$x \in (4,5) \cup (7,\infty)$\\
G.$x \in [4,5) \cup (7,\infty)$\\
H.$x \in (4,5] \cup (7,\infty)$
\testStop
\kluczStart
A
\kluczStop



\zadStart{Zadanie z Wikieł Z 1.62 a) moja wersja nr 208}

Rozwiązać nierówności $(x-4)(x-5)(x-8)\ge0$.
\zadStop
\rozwStart{Patryk Wirkus}{Laura Mieczkowska}
Miejsca zerowe naszego wielomianu to: $4, 5, 8$.\\
Wielomian jest stopnia nieparzystego, ponadto znak współczynnika przy\linebreak najwyższej potędze x jest dodatni.\\ W związku z tym wykres wielomianu zaczyna się od lewej strony poniżej osi OX. A więc $$x \in [4,5] \cup [8,\infty).$$
\rozwStop
\odpStart
$x \in [4,5] \cup [8,\infty)$
\odpStop
\testStart
A.$x \in [4,5] \cup [8,\infty)$\\
B.$x \in (4,5) \cup [8,\infty)$\\
C.$x \in (4,5] \cup [8,\infty)$\\
D.$x \in [4,5) \cup [8,\infty)$\\
E.$x \in [4,5] \cup (8,\infty)$\\
F.$x \in (4,5) \cup (8,\infty)$\\
G.$x \in [4,5) \cup (8,\infty)$\\
H.$x \in (4,5] \cup (8,\infty)$
\testStop
\kluczStart
A
\kluczStop



\zadStart{Zadanie z Wikieł Z 1.62 a) moja wersja nr 209}

Rozwiązać nierówności $(x-4)(x-5)(x-9)\ge0$.
\zadStop
\rozwStart{Patryk Wirkus}{Laura Mieczkowska}
Miejsca zerowe naszego wielomianu to: $4, 5, 9$.\\
Wielomian jest stopnia nieparzystego, ponadto znak współczynnika przy\linebreak najwyższej potędze x jest dodatni.\\ W związku z tym wykres wielomianu zaczyna się od lewej strony poniżej osi OX. A więc $$x \in [4,5] \cup [9,\infty).$$
\rozwStop
\odpStart
$x \in [4,5] \cup [9,\infty)$
\odpStop
\testStart
A.$x \in [4,5] \cup [9,\infty)$\\
B.$x \in (4,5) \cup [9,\infty)$\\
C.$x \in (4,5] \cup [9,\infty)$\\
D.$x \in [4,5) \cup [9,\infty)$\\
E.$x \in [4,5] \cup (9,\infty)$\\
F.$x \in (4,5) \cup (9,\infty)$\\
G.$x \in [4,5) \cup (9,\infty)$\\
H.$x \in (4,5] \cup (9,\infty)$
\testStop
\kluczStart
A
\kluczStop



\zadStart{Zadanie z Wikieł Z 1.62 a) moja wersja nr 210}

Rozwiązać nierówności $(x-4)(x-5)(x-10)\ge0$.
\zadStop
\rozwStart{Patryk Wirkus}{Laura Mieczkowska}
Miejsca zerowe naszego wielomianu to: $4, 5, 10$.\\
Wielomian jest stopnia nieparzystego, ponadto znak współczynnika przy\linebreak najwyższej potędze x jest dodatni.\\ W związku z tym wykres wielomianu zaczyna się od lewej strony poniżej osi OX. A więc $$x \in [4,5] \cup [10,\infty).$$
\rozwStop
\odpStart
$x \in [4,5] \cup [10,\infty)$
\odpStop
\testStart
A.$x \in [4,5] \cup [10,\infty)$\\
B.$x \in (4,5) \cup [10,\infty)$\\
C.$x \in (4,5] \cup [10,\infty)$\\
D.$x \in [4,5) \cup [10,\infty)$\\
E.$x \in [4,5] \cup (10,\infty)$\\
F.$x \in (4,5) \cup (10,\infty)$\\
G.$x \in [4,5) \cup (10,\infty)$\\
H.$x \in (4,5] \cup (10,\infty)$
\testStop
\kluczStart
A
\kluczStop



\zadStart{Zadanie z Wikieł Z 1.62 a) moja wersja nr 211}

Rozwiązać nierówności $(x-4)(x-5)(x-11)\ge0$.
\zadStop
\rozwStart{Patryk Wirkus}{Laura Mieczkowska}
Miejsca zerowe naszego wielomianu to: $4, 5, 11$.\\
Wielomian jest stopnia nieparzystego, ponadto znak współczynnika przy\linebreak najwyższej potędze x jest dodatni.\\ W związku z tym wykres wielomianu zaczyna się od lewej strony poniżej osi OX. A więc $$x \in [4,5] \cup [11,\infty).$$
\rozwStop
\odpStart
$x \in [4,5] \cup [11,\infty)$
\odpStop
\testStart
A.$x \in [4,5] \cup [11,\infty)$\\
B.$x \in (4,5) \cup [11,\infty)$\\
C.$x \in (4,5] \cup [11,\infty)$\\
D.$x \in [4,5) \cup [11,\infty)$\\
E.$x \in [4,5] \cup (11,\infty)$\\
F.$x \in (4,5) \cup (11,\infty)$\\
G.$x \in [4,5) \cup (11,\infty)$\\
H.$x \in (4,5] \cup (11,\infty)$
\testStop
\kluczStart
A
\kluczStop



\zadStart{Zadanie z Wikieł Z 1.62 a) moja wersja nr 212}

Rozwiązać nierówności $(x-4)(x-5)(x-12)\ge0$.
\zadStop
\rozwStart{Patryk Wirkus}{Laura Mieczkowska}
Miejsca zerowe naszego wielomianu to: $4, 5, 12$.\\
Wielomian jest stopnia nieparzystego, ponadto znak współczynnika przy\linebreak najwyższej potędze x jest dodatni.\\ W związku z tym wykres wielomianu zaczyna się od lewej strony poniżej osi OX. A więc $$x \in [4,5] \cup [12,\infty).$$
\rozwStop
\odpStart
$x \in [4,5] \cup [12,\infty)$
\odpStop
\testStart
A.$x \in [4,5] \cup [12,\infty)$\\
B.$x \in (4,5) \cup [12,\infty)$\\
C.$x \in (4,5] \cup [12,\infty)$\\
D.$x \in [4,5) \cup [12,\infty)$\\
E.$x \in [4,5] \cup (12,\infty)$\\
F.$x \in (4,5) \cup (12,\infty)$\\
G.$x \in [4,5) \cup (12,\infty)$\\
H.$x \in (4,5] \cup (12,\infty)$
\testStop
\kluczStart
A
\kluczStop



\zadStart{Zadanie z Wikieł Z 1.62 a) moja wersja nr 213}

Rozwiązać nierówności $(x-4)(x-5)(x-13)\ge0$.
\zadStop
\rozwStart{Patryk Wirkus}{Laura Mieczkowska}
Miejsca zerowe naszego wielomianu to: $4, 5, 13$.\\
Wielomian jest stopnia nieparzystego, ponadto znak współczynnika przy\linebreak najwyższej potędze x jest dodatni.\\ W związku z tym wykres wielomianu zaczyna się od lewej strony poniżej osi OX. A więc $$x \in [4,5] \cup [13,\infty).$$
\rozwStop
\odpStart
$x \in [4,5] \cup [13,\infty)$
\odpStop
\testStart
A.$x \in [4,5] \cup [13,\infty)$\\
B.$x \in (4,5) \cup [13,\infty)$\\
C.$x \in (4,5] \cup [13,\infty)$\\
D.$x \in [4,5) \cup [13,\infty)$\\
E.$x \in [4,5] \cup (13,\infty)$\\
F.$x \in (4,5) \cup (13,\infty)$\\
G.$x \in [4,5) \cup (13,\infty)$\\
H.$x \in (4,5] \cup (13,\infty)$
\testStop
\kluczStart
A
\kluczStop



\zadStart{Zadanie z Wikieł Z 1.62 a) moja wersja nr 214}

Rozwiązać nierówności $(x-4)(x-5)(x-14)\ge0$.
\zadStop
\rozwStart{Patryk Wirkus}{Laura Mieczkowska}
Miejsca zerowe naszego wielomianu to: $4, 5, 14$.\\
Wielomian jest stopnia nieparzystego, ponadto znak współczynnika przy\linebreak najwyższej potędze x jest dodatni.\\ W związku z tym wykres wielomianu zaczyna się od lewej strony poniżej osi OX. A więc $$x \in [4,5] \cup [14,\infty).$$
\rozwStop
\odpStart
$x \in [4,5] \cup [14,\infty)$
\odpStop
\testStart
A.$x \in [4,5] \cup [14,\infty)$\\
B.$x \in (4,5) \cup [14,\infty)$\\
C.$x \in (4,5] \cup [14,\infty)$\\
D.$x \in [4,5) \cup [14,\infty)$\\
E.$x \in [4,5] \cup (14,\infty)$\\
F.$x \in (4,5) \cup (14,\infty)$\\
G.$x \in [4,5) \cup (14,\infty)$\\
H.$x \in (4,5] \cup (14,\infty)$
\testStop
\kluczStart
A
\kluczStop



\zadStart{Zadanie z Wikieł Z 1.62 a) moja wersja nr 215}

Rozwiązać nierówności $(x-4)(x-5)(x-15)\ge0$.
\zadStop
\rozwStart{Patryk Wirkus}{Laura Mieczkowska}
Miejsca zerowe naszego wielomianu to: $4, 5, 15$.\\
Wielomian jest stopnia nieparzystego, ponadto znak współczynnika przy\linebreak najwyższej potędze x jest dodatni.\\ W związku z tym wykres wielomianu zaczyna się od lewej strony poniżej osi OX. A więc $$x \in [4,5] \cup [15,\infty).$$
\rozwStop
\odpStart
$x \in [4,5] \cup [15,\infty)$
\odpStop
\testStart
A.$x \in [4,5] \cup [15,\infty)$\\
B.$x \in (4,5) \cup [15,\infty)$\\
C.$x \in (4,5] \cup [15,\infty)$\\
D.$x \in [4,5) \cup [15,\infty)$\\
E.$x \in [4,5] \cup (15,\infty)$\\
F.$x \in (4,5) \cup (15,\infty)$\\
G.$x \in [4,5) \cup (15,\infty)$\\
H.$x \in (4,5] \cup (15,\infty)$
\testStop
\kluczStart
A
\kluczStop



\zadStart{Zadanie z Wikieł Z 1.62 a) moja wersja nr 216}

Rozwiązać nierówności $(x-4)(x-6)(x-7)\ge0$.
\zadStop
\rozwStart{Patryk Wirkus}{Laura Mieczkowska}
Miejsca zerowe naszego wielomianu to: $4, 6, 7$.\\
Wielomian jest stopnia nieparzystego, ponadto znak współczynnika przy\linebreak najwyższej potędze x jest dodatni.\\ W związku z tym wykres wielomianu zaczyna się od lewej strony poniżej osi OX. A więc $$x \in [4,6] \cup [7,\infty).$$
\rozwStop
\odpStart
$x \in [4,6] \cup [7,\infty)$
\odpStop
\testStart
A.$x \in [4,6] \cup [7,\infty)$\\
B.$x \in (4,6) \cup [7,\infty)$\\
C.$x \in (4,6] \cup [7,\infty)$\\
D.$x \in [4,6) \cup [7,\infty)$\\
E.$x \in [4,6] \cup (7,\infty)$\\
F.$x \in (4,6) \cup (7,\infty)$\\
G.$x \in [4,6) \cup (7,\infty)$\\
H.$x \in (4,6] \cup (7,\infty)$
\testStop
\kluczStart
A
\kluczStop



\zadStart{Zadanie z Wikieł Z 1.62 a) moja wersja nr 217}

Rozwiązać nierówności $(x-4)(x-6)(x-8)\ge0$.
\zadStop
\rozwStart{Patryk Wirkus}{Laura Mieczkowska}
Miejsca zerowe naszego wielomianu to: $4, 6, 8$.\\
Wielomian jest stopnia nieparzystego, ponadto znak współczynnika przy\linebreak najwyższej potędze x jest dodatni.\\ W związku z tym wykres wielomianu zaczyna się od lewej strony poniżej osi OX. A więc $$x \in [4,6] \cup [8,\infty).$$
\rozwStop
\odpStart
$x \in [4,6] \cup [8,\infty)$
\odpStop
\testStart
A.$x \in [4,6] \cup [8,\infty)$\\
B.$x \in (4,6) \cup [8,\infty)$\\
C.$x \in (4,6] \cup [8,\infty)$\\
D.$x \in [4,6) \cup [8,\infty)$\\
E.$x \in [4,6] \cup (8,\infty)$\\
F.$x \in (4,6) \cup (8,\infty)$\\
G.$x \in [4,6) \cup (8,\infty)$\\
H.$x \in (4,6] \cup (8,\infty)$
\testStop
\kluczStart
A
\kluczStop



\zadStart{Zadanie z Wikieł Z 1.62 a) moja wersja nr 218}

Rozwiązać nierówności $(x-4)(x-6)(x-9)\ge0$.
\zadStop
\rozwStart{Patryk Wirkus}{Laura Mieczkowska}
Miejsca zerowe naszego wielomianu to: $4, 6, 9$.\\
Wielomian jest stopnia nieparzystego, ponadto znak współczynnika przy\linebreak najwyższej potędze x jest dodatni.\\ W związku z tym wykres wielomianu zaczyna się od lewej strony poniżej osi OX. A więc $$x \in [4,6] \cup [9,\infty).$$
\rozwStop
\odpStart
$x \in [4,6] \cup [9,\infty)$
\odpStop
\testStart
A.$x \in [4,6] \cup [9,\infty)$\\
B.$x \in (4,6) \cup [9,\infty)$\\
C.$x \in (4,6] \cup [9,\infty)$\\
D.$x \in [4,6) \cup [9,\infty)$\\
E.$x \in [4,6] \cup (9,\infty)$\\
F.$x \in (4,6) \cup (9,\infty)$\\
G.$x \in [4,6) \cup (9,\infty)$\\
H.$x \in (4,6] \cup (9,\infty)$
\testStop
\kluczStart
A
\kluczStop



\zadStart{Zadanie z Wikieł Z 1.62 a) moja wersja nr 219}

Rozwiązać nierówności $(x-4)(x-6)(x-10)\ge0$.
\zadStop
\rozwStart{Patryk Wirkus}{Laura Mieczkowska}
Miejsca zerowe naszego wielomianu to: $4, 6, 10$.\\
Wielomian jest stopnia nieparzystego, ponadto znak współczynnika przy\linebreak najwyższej potędze x jest dodatni.\\ W związku z tym wykres wielomianu zaczyna się od lewej strony poniżej osi OX. A więc $$x \in [4,6] \cup [10,\infty).$$
\rozwStop
\odpStart
$x \in [4,6] \cup [10,\infty)$
\odpStop
\testStart
A.$x \in [4,6] \cup [10,\infty)$\\
B.$x \in (4,6) \cup [10,\infty)$\\
C.$x \in (4,6] \cup [10,\infty)$\\
D.$x \in [4,6) \cup [10,\infty)$\\
E.$x \in [4,6] \cup (10,\infty)$\\
F.$x \in (4,6) \cup (10,\infty)$\\
G.$x \in [4,6) \cup (10,\infty)$\\
H.$x \in (4,6] \cup (10,\infty)$
\testStop
\kluczStart
A
\kluczStop



\zadStart{Zadanie z Wikieł Z 1.62 a) moja wersja nr 220}

Rozwiązać nierówności $(x-4)(x-6)(x-11)\ge0$.
\zadStop
\rozwStart{Patryk Wirkus}{Laura Mieczkowska}
Miejsca zerowe naszego wielomianu to: $4, 6, 11$.\\
Wielomian jest stopnia nieparzystego, ponadto znak współczynnika przy\linebreak najwyższej potędze x jest dodatni.\\ W związku z tym wykres wielomianu zaczyna się od lewej strony poniżej osi OX. A więc $$x \in [4,6] \cup [11,\infty).$$
\rozwStop
\odpStart
$x \in [4,6] \cup [11,\infty)$
\odpStop
\testStart
A.$x \in [4,6] \cup [11,\infty)$\\
B.$x \in (4,6) \cup [11,\infty)$\\
C.$x \in (4,6] \cup [11,\infty)$\\
D.$x \in [4,6) \cup [11,\infty)$\\
E.$x \in [4,6] \cup (11,\infty)$\\
F.$x \in (4,6) \cup (11,\infty)$\\
G.$x \in [4,6) \cup (11,\infty)$\\
H.$x \in (4,6] \cup (11,\infty)$
\testStop
\kluczStart
A
\kluczStop



\zadStart{Zadanie z Wikieł Z 1.62 a) moja wersja nr 221}

Rozwiązać nierówności $(x-4)(x-6)(x-12)\ge0$.
\zadStop
\rozwStart{Patryk Wirkus}{Laura Mieczkowska}
Miejsca zerowe naszego wielomianu to: $4, 6, 12$.\\
Wielomian jest stopnia nieparzystego, ponadto znak współczynnika przy\linebreak najwyższej potędze x jest dodatni.\\ W związku z tym wykres wielomianu zaczyna się od lewej strony poniżej osi OX. A więc $$x \in [4,6] \cup [12,\infty).$$
\rozwStop
\odpStart
$x \in [4,6] \cup [12,\infty)$
\odpStop
\testStart
A.$x \in [4,6] \cup [12,\infty)$\\
B.$x \in (4,6) \cup [12,\infty)$\\
C.$x \in (4,6] \cup [12,\infty)$\\
D.$x \in [4,6) \cup [12,\infty)$\\
E.$x \in [4,6] \cup (12,\infty)$\\
F.$x \in (4,6) \cup (12,\infty)$\\
G.$x \in [4,6) \cup (12,\infty)$\\
H.$x \in (4,6] \cup (12,\infty)$
\testStop
\kluczStart
A
\kluczStop



\zadStart{Zadanie z Wikieł Z 1.62 a) moja wersja nr 222}

Rozwiązać nierówności $(x-4)(x-6)(x-13)\ge0$.
\zadStop
\rozwStart{Patryk Wirkus}{Laura Mieczkowska}
Miejsca zerowe naszego wielomianu to: $4, 6, 13$.\\
Wielomian jest stopnia nieparzystego, ponadto znak współczynnika przy\linebreak najwyższej potędze x jest dodatni.\\ W związku z tym wykres wielomianu zaczyna się od lewej strony poniżej osi OX. A więc $$x \in [4,6] \cup [13,\infty).$$
\rozwStop
\odpStart
$x \in [4,6] \cup [13,\infty)$
\odpStop
\testStart
A.$x \in [4,6] \cup [13,\infty)$\\
B.$x \in (4,6) \cup [13,\infty)$\\
C.$x \in (4,6] \cup [13,\infty)$\\
D.$x \in [4,6) \cup [13,\infty)$\\
E.$x \in [4,6] \cup (13,\infty)$\\
F.$x \in (4,6) \cup (13,\infty)$\\
G.$x \in [4,6) \cup (13,\infty)$\\
H.$x \in (4,6] \cup (13,\infty)$
\testStop
\kluczStart
A
\kluczStop



\zadStart{Zadanie z Wikieł Z 1.62 a) moja wersja nr 223}

Rozwiązać nierówności $(x-4)(x-6)(x-14)\ge0$.
\zadStop
\rozwStart{Patryk Wirkus}{Laura Mieczkowska}
Miejsca zerowe naszego wielomianu to: $4, 6, 14$.\\
Wielomian jest stopnia nieparzystego, ponadto znak współczynnika przy\linebreak najwyższej potędze x jest dodatni.\\ W związku z tym wykres wielomianu zaczyna się od lewej strony poniżej osi OX. A więc $$x \in [4,6] \cup [14,\infty).$$
\rozwStop
\odpStart
$x \in [4,6] \cup [14,\infty)$
\odpStop
\testStart
A.$x \in [4,6] \cup [14,\infty)$\\
B.$x \in (4,6) \cup [14,\infty)$\\
C.$x \in (4,6] \cup [14,\infty)$\\
D.$x \in [4,6) \cup [14,\infty)$\\
E.$x \in [4,6] \cup (14,\infty)$\\
F.$x \in (4,6) \cup (14,\infty)$\\
G.$x \in [4,6) \cup (14,\infty)$\\
H.$x \in (4,6] \cup (14,\infty)$
\testStop
\kluczStart
A
\kluczStop



\zadStart{Zadanie z Wikieł Z 1.62 a) moja wersja nr 224}

Rozwiązać nierówności $(x-4)(x-6)(x-15)\ge0$.
\zadStop
\rozwStart{Patryk Wirkus}{Laura Mieczkowska}
Miejsca zerowe naszego wielomianu to: $4, 6, 15$.\\
Wielomian jest stopnia nieparzystego, ponadto znak współczynnika przy\linebreak najwyższej potędze x jest dodatni.\\ W związku z tym wykres wielomianu zaczyna się od lewej strony poniżej osi OX. A więc $$x \in [4,6] \cup [15,\infty).$$
\rozwStop
\odpStart
$x \in [4,6] \cup [15,\infty)$
\odpStop
\testStart
A.$x \in [4,6] \cup [15,\infty)$\\
B.$x \in (4,6) \cup [15,\infty)$\\
C.$x \in (4,6] \cup [15,\infty)$\\
D.$x \in [4,6) \cup [15,\infty)$\\
E.$x \in [4,6] \cup (15,\infty)$\\
F.$x \in (4,6) \cup (15,\infty)$\\
G.$x \in [4,6) \cup (15,\infty)$\\
H.$x \in (4,6] \cup (15,\infty)$
\testStop
\kluczStart
A
\kluczStop



\zadStart{Zadanie z Wikieł Z 1.62 a) moja wersja nr 225}

Rozwiązać nierówności $(x-4)(x-7)(x-8)\ge0$.
\zadStop
\rozwStart{Patryk Wirkus}{Laura Mieczkowska}
Miejsca zerowe naszego wielomianu to: $4, 7, 8$.\\
Wielomian jest stopnia nieparzystego, ponadto znak współczynnika przy\linebreak najwyższej potędze x jest dodatni.\\ W związku z tym wykres wielomianu zaczyna się od lewej strony poniżej osi OX. A więc $$x \in [4,7] \cup [8,\infty).$$
\rozwStop
\odpStart
$x \in [4,7] \cup [8,\infty)$
\odpStop
\testStart
A.$x \in [4,7] \cup [8,\infty)$\\
B.$x \in (4,7) \cup [8,\infty)$\\
C.$x \in (4,7] \cup [8,\infty)$\\
D.$x \in [4,7) \cup [8,\infty)$\\
E.$x \in [4,7] \cup (8,\infty)$\\
F.$x \in (4,7) \cup (8,\infty)$\\
G.$x \in [4,7) \cup (8,\infty)$\\
H.$x \in (4,7] \cup (8,\infty)$
\testStop
\kluczStart
A
\kluczStop



\zadStart{Zadanie z Wikieł Z 1.62 a) moja wersja nr 226}

Rozwiązać nierówności $(x-4)(x-7)(x-9)\ge0$.
\zadStop
\rozwStart{Patryk Wirkus}{Laura Mieczkowska}
Miejsca zerowe naszego wielomianu to: $4, 7, 9$.\\
Wielomian jest stopnia nieparzystego, ponadto znak współczynnika przy\linebreak najwyższej potędze x jest dodatni.\\ W związku z tym wykres wielomianu zaczyna się od lewej strony poniżej osi OX. A więc $$x \in [4,7] \cup [9,\infty).$$
\rozwStop
\odpStart
$x \in [4,7] \cup [9,\infty)$
\odpStop
\testStart
A.$x \in [4,7] \cup [9,\infty)$\\
B.$x \in (4,7) \cup [9,\infty)$\\
C.$x \in (4,7] \cup [9,\infty)$\\
D.$x \in [4,7) \cup [9,\infty)$\\
E.$x \in [4,7] \cup (9,\infty)$\\
F.$x \in (4,7) \cup (9,\infty)$\\
G.$x \in [4,7) \cup (9,\infty)$\\
H.$x \in (4,7] \cup (9,\infty)$
\testStop
\kluczStart
A
\kluczStop



\zadStart{Zadanie z Wikieł Z 1.62 a) moja wersja nr 227}

Rozwiązać nierówności $(x-4)(x-7)(x-10)\ge0$.
\zadStop
\rozwStart{Patryk Wirkus}{Laura Mieczkowska}
Miejsca zerowe naszego wielomianu to: $4, 7, 10$.\\
Wielomian jest stopnia nieparzystego, ponadto znak współczynnika przy\linebreak najwyższej potędze x jest dodatni.\\ W związku z tym wykres wielomianu zaczyna się od lewej strony poniżej osi OX. A więc $$x \in [4,7] \cup [10,\infty).$$
\rozwStop
\odpStart
$x \in [4,7] \cup [10,\infty)$
\odpStop
\testStart
A.$x \in [4,7] \cup [10,\infty)$\\
B.$x \in (4,7) \cup [10,\infty)$\\
C.$x \in (4,7] \cup [10,\infty)$\\
D.$x \in [4,7) \cup [10,\infty)$\\
E.$x \in [4,7] \cup (10,\infty)$\\
F.$x \in (4,7) \cup (10,\infty)$\\
G.$x \in [4,7) \cup (10,\infty)$\\
H.$x \in (4,7] \cup (10,\infty)$
\testStop
\kluczStart
A
\kluczStop



\zadStart{Zadanie z Wikieł Z 1.62 a) moja wersja nr 228}

Rozwiązać nierówności $(x-4)(x-7)(x-11)\ge0$.
\zadStop
\rozwStart{Patryk Wirkus}{Laura Mieczkowska}
Miejsca zerowe naszego wielomianu to: $4, 7, 11$.\\
Wielomian jest stopnia nieparzystego, ponadto znak współczynnika przy\linebreak najwyższej potędze x jest dodatni.\\ W związku z tym wykres wielomianu zaczyna się od lewej strony poniżej osi OX. A więc $$x \in [4,7] \cup [11,\infty).$$
\rozwStop
\odpStart
$x \in [4,7] \cup [11,\infty)$
\odpStop
\testStart
A.$x \in [4,7] \cup [11,\infty)$\\
B.$x \in (4,7) \cup [11,\infty)$\\
C.$x \in (4,7] \cup [11,\infty)$\\
D.$x \in [4,7) \cup [11,\infty)$\\
E.$x \in [4,7] \cup (11,\infty)$\\
F.$x \in (4,7) \cup (11,\infty)$\\
G.$x \in [4,7) \cup (11,\infty)$\\
H.$x \in (4,7] \cup (11,\infty)$
\testStop
\kluczStart
A
\kluczStop



\zadStart{Zadanie z Wikieł Z 1.62 a) moja wersja nr 229}

Rozwiązać nierówności $(x-4)(x-7)(x-12)\ge0$.
\zadStop
\rozwStart{Patryk Wirkus}{Laura Mieczkowska}
Miejsca zerowe naszego wielomianu to: $4, 7, 12$.\\
Wielomian jest stopnia nieparzystego, ponadto znak współczynnika przy\linebreak najwyższej potędze x jest dodatni.\\ W związku z tym wykres wielomianu zaczyna się od lewej strony poniżej osi OX. A więc $$x \in [4,7] \cup [12,\infty).$$
\rozwStop
\odpStart
$x \in [4,7] \cup [12,\infty)$
\odpStop
\testStart
A.$x \in [4,7] \cup [12,\infty)$\\
B.$x \in (4,7) \cup [12,\infty)$\\
C.$x \in (4,7] \cup [12,\infty)$\\
D.$x \in [4,7) \cup [12,\infty)$\\
E.$x \in [4,7] \cup (12,\infty)$\\
F.$x \in (4,7) \cup (12,\infty)$\\
G.$x \in [4,7) \cup (12,\infty)$\\
H.$x \in (4,7] \cup (12,\infty)$
\testStop
\kluczStart
A
\kluczStop



\zadStart{Zadanie z Wikieł Z 1.62 a) moja wersja nr 230}

Rozwiązać nierówności $(x-4)(x-7)(x-13)\ge0$.
\zadStop
\rozwStart{Patryk Wirkus}{Laura Mieczkowska}
Miejsca zerowe naszego wielomianu to: $4, 7, 13$.\\
Wielomian jest stopnia nieparzystego, ponadto znak współczynnika przy\linebreak najwyższej potędze x jest dodatni.\\ W związku z tym wykres wielomianu zaczyna się od lewej strony poniżej osi OX. A więc $$x \in [4,7] \cup [13,\infty).$$
\rozwStop
\odpStart
$x \in [4,7] \cup [13,\infty)$
\odpStop
\testStart
A.$x \in [4,7] \cup [13,\infty)$\\
B.$x \in (4,7) \cup [13,\infty)$\\
C.$x \in (4,7] \cup [13,\infty)$\\
D.$x \in [4,7) \cup [13,\infty)$\\
E.$x \in [4,7] \cup (13,\infty)$\\
F.$x \in (4,7) \cup (13,\infty)$\\
G.$x \in [4,7) \cup (13,\infty)$\\
H.$x \in (4,7] \cup (13,\infty)$
\testStop
\kluczStart
A
\kluczStop



\zadStart{Zadanie z Wikieł Z 1.62 a) moja wersja nr 231}

Rozwiązać nierówności $(x-4)(x-7)(x-14)\ge0$.
\zadStop
\rozwStart{Patryk Wirkus}{Laura Mieczkowska}
Miejsca zerowe naszego wielomianu to: $4, 7, 14$.\\
Wielomian jest stopnia nieparzystego, ponadto znak współczynnika przy\linebreak najwyższej potędze x jest dodatni.\\ W związku z tym wykres wielomianu zaczyna się od lewej strony poniżej osi OX. A więc $$x \in [4,7] \cup [14,\infty).$$
\rozwStop
\odpStart
$x \in [4,7] \cup [14,\infty)$
\odpStop
\testStart
A.$x \in [4,7] \cup [14,\infty)$\\
B.$x \in (4,7) \cup [14,\infty)$\\
C.$x \in (4,7] \cup [14,\infty)$\\
D.$x \in [4,7) \cup [14,\infty)$\\
E.$x \in [4,7] \cup (14,\infty)$\\
F.$x \in (4,7) \cup (14,\infty)$\\
G.$x \in [4,7) \cup (14,\infty)$\\
H.$x \in (4,7] \cup (14,\infty)$
\testStop
\kluczStart
A
\kluczStop



\zadStart{Zadanie z Wikieł Z 1.62 a) moja wersja nr 232}

Rozwiązać nierówności $(x-4)(x-7)(x-15)\ge0$.
\zadStop
\rozwStart{Patryk Wirkus}{Laura Mieczkowska}
Miejsca zerowe naszego wielomianu to: $4, 7, 15$.\\
Wielomian jest stopnia nieparzystego, ponadto znak współczynnika przy\linebreak najwyższej potędze x jest dodatni.\\ W związku z tym wykres wielomianu zaczyna się od lewej strony poniżej osi OX. A więc $$x \in [4,7] \cup [15,\infty).$$
\rozwStop
\odpStart
$x \in [4,7] \cup [15,\infty)$
\odpStop
\testStart
A.$x \in [4,7] \cup [15,\infty)$\\
B.$x \in (4,7) \cup [15,\infty)$\\
C.$x \in (4,7] \cup [15,\infty)$\\
D.$x \in [4,7) \cup [15,\infty)$\\
E.$x \in [4,7] \cup (15,\infty)$\\
F.$x \in (4,7) \cup (15,\infty)$\\
G.$x \in [4,7) \cup (15,\infty)$\\
H.$x \in (4,7] \cup (15,\infty)$
\testStop
\kluczStart
A
\kluczStop



\zadStart{Zadanie z Wikieł Z 1.62 a) moja wersja nr 233}

Rozwiązać nierówności $(x-4)(x-8)(x-9)\ge0$.
\zadStop
\rozwStart{Patryk Wirkus}{Laura Mieczkowska}
Miejsca zerowe naszego wielomianu to: $4, 8, 9$.\\
Wielomian jest stopnia nieparzystego, ponadto znak współczynnika przy\linebreak najwyższej potędze x jest dodatni.\\ W związku z tym wykres wielomianu zaczyna się od lewej strony poniżej osi OX. A więc $$x \in [4,8] \cup [9,\infty).$$
\rozwStop
\odpStart
$x \in [4,8] \cup [9,\infty)$
\odpStop
\testStart
A.$x \in [4,8] \cup [9,\infty)$\\
B.$x \in (4,8) \cup [9,\infty)$\\
C.$x \in (4,8] \cup [9,\infty)$\\
D.$x \in [4,8) \cup [9,\infty)$\\
E.$x \in [4,8] \cup (9,\infty)$\\
F.$x \in (4,8) \cup (9,\infty)$\\
G.$x \in [4,8) \cup (9,\infty)$\\
H.$x \in (4,8] \cup (9,\infty)$
\testStop
\kluczStart
A
\kluczStop



\zadStart{Zadanie z Wikieł Z 1.62 a) moja wersja nr 234}

Rozwiązać nierówności $(x-4)(x-8)(x-10)\ge0$.
\zadStop
\rozwStart{Patryk Wirkus}{Laura Mieczkowska}
Miejsca zerowe naszego wielomianu to: $4, 8, 10$.\\
Wielomian jest stopnia nieparzystego, ponadto znak współczynnika przy\linebreak najwyższej potędze x jest dodatni.\\ W związku z tym wykres wielomianu zaczyna się od lewej strony poniżej osi OX. A więc $$x \in [4,8] \cup [10,\infty).$$
\rozwStop
\odpStart
$x \in [4,8] \cup [10,\infty)$
\odpStop
\testStart
A.$x \in [4,8] \cup [10,\infty)$\\
B.$x \in (4,8) \cup [10,\infty)$\\
C.$x \in (4,8] \cup [10,\infty)$\\
D.$x \in [4,8) \cup [10,\infty)$\\
E.$x \in [4,8] \cup (10,\infty)$\\
F.$x \in (4,8) \cup (10,\infty)$\\
G.$x \in [4,8) \cup (10,\infty)$\\
H.$x \in (4,8] \cup (10,\infty)$
\testStop
\kluczStart
A
\kluczStop



\zadStart{Zadanie z Wikieł Z 1.62 a) moja wersja nr 235}

Rozwiązać nierówności $(x-4)(x-8)(x-11)\ge0$.
\zadStop
\rozwStart{Patryk Wirkus}{Laura Mieczkowska}
Miejsca zerowe naszego wielomianu to: $4, 8, 11$.\\
Wielomian jest stopnia nieparzystego, ponadto znak współczynnika przy\linebreak najwyższej potędze x jest dodatni.\\ W związku z tym wykres wielomianu zaczyna się od lewej strony poniżej osi OX. A więc $$x \in [4,8] \cup [11,\infty).$$
\rozwStop
\odpStart
$x \in [4,8] \cup [11,\infty)$
\odpStop
\testStart
A.$x \in [4,8] \cup [11,\infty)$\\
B.$x \in (4,8) \cup [11,\infty)$\\
C.$x \in (4,8] \cup [11,\infty)$\\
D.$x \in [4,8) \cup [11,\infty)$\\
E.$x \in [4,8] \cup (11,\infty)$\\
F.$x \in (4,8) \cup (11,\infty)$\\
G.$x \in [4,8) \cup (11,\infty)$\\
H.$x \in (4,8] \cup (11,\infty)$
\testStop
\kluczStart
A
\kluczStop



\zadStart{Zadanie z Wikieł Z 1.62 a) moja wersja nr 236}

Rozwiązać nierówności $(x-4)(x-8)(x-12)\ge0$.
\zadStop
\rozwStart{Patryk Wirkus}{Laura Mieczkowska}
Miejsca zerowe naszego wielomianu to: $4, 8, 12$.\\
Wielomian jest stopnia nieparzystego, ponadto znak współczynnika przy\linebreak najwyższej potędze x jest dodatni.\\ W związku z tym wykres wielomianu zaczyna się od lewej strony poniżej osi OX. A więc $$x \in [4,8] \cup [12,\infty).$$
\rozwStop
\odpStart
$x \in [4,8] \cup [12,\infty)$
\odpStop
\testStart
A.$x \in [4,8] \cup [12,\infty)$\\
B.$x \in (4,8) \cup [12,\infty)$\\
C.$x \in (4,8] \cup [12,\infty)$\\
D.$x \in [4,8) \cup [12,\infty)$\\
E.$x \in [4,8] \cup (12,\infty)$\\
F.$x \in (4,8) \cup (12,\infty)$\\
G.$x \in [4,8) \cup (12,\infty)$\\
H.$x \in (4,8] \cup (12,\infty)$
\testStop
\kluczStart
A
\kluczStop



\zadStart{Zadanie z Wikieł Z 1.62 a) moja wersja nr 237}

Rozwiązać nierówności $(x-4)(x-8)(x-13)\ge0$.
\zadStop
\rozwStart{Patryk Wirkus}{Laura Mieczkowska}
Miejsca zerowe naszego wielomianu to: $4, 8, 13$.\\
Wielomian jest stopnia nieparzystego, ponadto znak współczynnika przy\linebreak najwyższej potędze x jest dodatni.\\ W związku z tym wykres wielomianu zaczyna się od lewej strony poniżej osi OX. A więc $$x \in [4,8] \cup [13,\infty).$$
\rozwStop
\odpStart
$x \in [4,8] \cup [13,\infty)$
\odpStop
\testStart
A.$x \in [4,8] \cup [13,\infty)$\\
B.$x \in (4,8) \cup [13,\infty)$\\
C.$x \in (4,8] \cup [13,\infty)$\\
D.$x \in [4,8) \cup [13,\infty)$\\
E.$x \in [4,8] \cup (13,\infty)$\\
F.$x \in (4,8) \cup (13,\infty)$\\
G.$x \in [4,8) \cup (13,\infty)$\\
H.$x \in (4,8] \cup (13,\infty)$
\testStop
\kluczStart
A
\kluczStop



\zadStart{Zadanie z Wikieł Z 1.62 a) moja wersja nr 238}

Rozwiązać nierówności $(x-4)(x-8)(x-14)\ge0$.
\zadStop
\rozwStart{Patryk Wirkus}{Laura Mieczkowska}
Miejsca zerowe naszego wielomianu to: $4, 8, 14$.\\
Wielomian jest stopnia nieparzystego, ponadto znak współczynnika przy\linebreak najwyższej potędze x jest dodatni.\\ W związku z tym wykres wielomianu zaczyna się od lewej strony poniżej osi OX. A więc $$x \in [4,8] \cup [14,\infty).$$
\rozwStop
\odpStart
$x \in [4,8] \cup [14,\infty)$
\odpStop
\testStart
A.$x \in [4,8] \cup [14,\infty)$\\
B.$x \in (4,8) \cup [14,\infty)$\\
C.$x \in (4,8] \cup [14,\infty)$\\
D.$x \in [4,8) \cup [14,\infty)$\\
E.$x \in [4,8] \cup (14,\infty)$\\
F.$x \in (4,8) \cup (14,\infty)$\\
G.$x \in [4,8) \cup (14,\infty)$\\
H.$x \in (4,8] \cup (14,\infty)$
\testStop
\kluczStart
A
\kluczStop



\zadStart{Zadanie z Wikieł Z 1.62 a) moja wersja nr 239}

Rozwiązać nierówności $(x-4)(x-8)(x-15)\ge0$.
\zadStop
\rozwStart{Patryk Wirkus}{Laura Mieczkowska}
Miejsca zerowe naszego wielomianu to: $4, 8, 15$.\\
Wielomian jest stopnia nieparzystego, ponadto znak współczynnika przy\linebreak najwyższej potędze x jest dodatni.\\ W związku z tym wykres wielomianu zaczyna się od lewej strony poniżej osi OX. A więc $$x \in [4,8] \cup [15,\infty).$$
\rozwStop
\odpStart
$x \in [4,8] \cup [15,\infty)$
\odpStop
\testStart
A.$x \in [4,8] \cup [15,\infty)$\\
B.$x \in (4,8) \cup [15,\infty)$\\
C.$x \in (4,8] \cup [15,\infty)$\\
D.$x \in [4,8) \cup [15,\infty)$\\
E.$x \in [4,8] \cup (15,\infty)$\\
F.$x \in (4,8) \cup (15,\infty)$\\
G.$x \in [4,8) \cup (15,\infty)$\\
H.$x \in (4,8] \cup (15,\infty)$
\testStop
\kluczStart
A
\kluczStop



\zadStart{Zadanie z Wikieł Z 1.62 a) moja wersja nr 240}

Rozwiązać nierówności $(x-4)(x-9)(x-10)\ge0$.
\zadStop
\rozwStart{Patryk Wirkus}{Laura Mieczkowska}
Miejsca zerowe naszego wielomianu to: $4, 9, 10$.\\
Wielomian jest stopnia nieparzystego, ponadto znak współczynnika przy\linebreak najwyższej potędze x jest dodatni.\\ W związku z tym wykres wielomianu zaczyna się od lewej strony poniżej osi OX. A więc $$x \in [4,9] \cup [10,\infty).$$
\rozwStop
\odpStart
$x \in [4,9] \cup [10,\infty)$
\odpStop
\testStart
A.$x \in [4,9] \cup [10,\infty)$\\
B.$x \in (4,9) \cup [10,\infty)$\\
C.$x \in (4,9] \cup [10,\infty)$\\
D.$x \in [4,9) \cup [10,\infty)$\\
E.$x \in [4,9] \cup (10,\infty)$\\
F.$x \in (4,9) \cup (10,\infty)$\\
G.$x \in [4,9) \cup (10,\infty)$\\
H.$x \in (4,9] \cup (10,\infty)$
\testStop
\kluczStart
A
\kluczStop



\zadStart{Zadanie z Wikieł Z 1.62 a) moja wersja nr 241}

Rozwiązać nierówności $(x-4)(x-9)(x-11)\ge0$.
\zadStop
\rozwStart{Patryk Wirkus}{Laura Mieczkowska}
Miejsca zerowe naszego wielomianu to: $4, 9, 11$.\\
Wielomian jest stopnia nieparzystego, ponadto znak współczynnika przy\linebreak najwyższej potędze x jest dodatni.\\ W związku z tym wykres wielomianu zaczyna się od lewej strony poniżej osi OX. A więc $$x \in [4,9] \cup [11,\infty).$$
\rozwStop
\odpStart
$x \in [4,9] \cup [11,\infty)$
\odpStop
\testStart
A.$x \in [4,9] \cup [11,\infty)$\\
B.$x \in (4,9) \cup [11,\infty)$\\
C.$x \in (4,9] \cup [11,\infty)$\\
D.$x \in [4,9) \cup [11,\infty)$\\
E.$x \in [4,9] \cup (11,\infty)$\\
F.$x \in (4,9) \cup (11,\infty)$\\
G.$x \in [4,9) \cup (11,\infty)$\\
H.$x \in (4,9] \cup (11,\infty)$
\testStop
\kluczStart
A
\kluczStop



\zadStart{Zadanie z Wikieł Z 1.62 a) moja wersja nr 242}

Rozwiązać nierówności $(x-4)(x-9)(x-12)\ge0$.
\zadStop
\rozwStart{Patryk Wirkus}{Laura Mieczkowska}
Miejsca zerowe naszego wielomianu to: $4, 9, 12$.\\
Wielomian jest stopnia nieparzystego, ponadto znak współczynnika przy\linebreak najwyższej potędze x jest dodatni.\\ W związku z tym wykres wielomianu zaczyna się od lewej strony poniżej osi OX. A więc $$x \in [4,9] \cup [12,\infty).$$
\rozwStop
\odpStart
$x \in [4,9] \cup [12,\infty)$
\odpStop
\testStart
A.$x \in [4,9] \cup [12,\infty)$\\
B.$x \in (4,9) \cup [12,\infty)$\\
C.$x \in (4,9] \cup [12,\infty)$\\
D.$x \in [4,9) \cup [12,\infty)$\\
E.$x \in [4,9] \cup (12,\infty)$\\
F.$x \in (4,9) \cup (12,\infty)$\\
G.$x \in [4,9) \cup (12,\infty)$\\
H.$x \in (4,9] \cup (12,\infty)$
\testStop
\kluczStart
A
\kluczStop



\zadStart{Zadanie z Wikieł Z 1.62 a) moja wersja nr 243}

Rozwiązać nierówności $(x-4)(x-9)(x-13)\ge0$.
\zadStop
\rozwStart{Patryk Wirkus}{Laura Mieczkowska}
Miejsca zerowe naszego wielomianu to: $4, 9, 13$.\\
Wielomian jest stopnia nieparzystego, ponadto znak współczynnika przy\linebreak najwyższej potędze x jest dodatni.\\ W związku z tym wykres wielomianu zaczyna się od lewej strony poniżej osi OX. A więc $$x \in [4,9] \cup [13,\infty).$$
\rozwStop
\odpStart
$x \in [4,9] \cup [13,\infty)$
\odpStop
\testStart
A.$x \in [4,9] \cup [13,\infty)$\\
B.$x \in (4,9) \cup [13,\infty)$\\
C.$x \in (4,9] \cup [13,\infty)$\\
D.$x \in [4,9) \cup [13,\infty)$\\
E.$x \in [4,9] \cup (13,\infty)$\\
F.$x \in (4,9) \cup (13,\infty)$\\
G.$x \in [4,9) \cup (13,\infty)$\\
H.$x \in (4,9] \cup (13,\infty)$
\testStop
\kluczStart
A
\kluczStop



\zadStart{Zadanie z Wikieł Z 1.62 a) moja wersja nr 244}

Rozwiązać nierówności $(x-4)(x-9)(x-14)\ge0$.
\zadStop
\rozwStart{Patryk Wirkus}{Laura Mieczkowska}
Miejsca zerowe naszego wielomianu to: $4, 9, 14$.\\
Wielomian jest stopnia nieparzystego, ponadto znak współczynnika przy\linebreak najwyższej potędze x jest dodatni.\\ W związku z tym wykres wielomianu zaczyna się od lewej strony poniżej osi OX. A więc $$x \in [4,9] \cup [14,\infty).$$
\rozwStop
\odpStart
$x \in [4,9] \cup [14,\infty)$
\odpStop
\testStart
A.$x \in [4,9] \cup [14,\infty)$\\
B.$x \in (4,9) \cup [14,\infty)$\\
C.$x \in (4,9] \cup [14,\infty)$\\
D.$x \in [4,9) \cup [14,\infty)$\\
E.$x \in [4,9] \cup (14,\infty)$\\
F.$x \in (4,9) \cup (14,\infty)$\\
G.$x \in [4,9) \cup (14,\infty)$\\
H.$x \in (4,9] \cup (14,\infty)$
\testStop
\kluczStart
A
\kluczStop



\zadStart{Zadanie z Wikieł Z 1.62 a) moja wersja nr 245}

Rozwiązać nierówności $(x-4)(x-9)(x-15)\ge0$.
\zadStop
\rozwStart{Patryk Wirkus}{Laura Mieczkowska}
Miejsca zerowe naszego wielomianu to: $4, 9, 15$.\\
Wielomian jest stopnia nieparzystego, ponadto znak współczynnika przy\linebreak najwyższej potędze x jest dodatni.\\ W związku z tym wykres wielomianu zaczyna się od lewej strony poniżej osi OX. A więc $$x \in [4,9] \cup [15,\infty).$$
\rozwStop
\odpStart
$x \in [4,9] \cup [15,\infty)$
\odpStop
\testStart
A.$x \in [4,9] \cup [15,\infty)$\\
B.$x \in (4,9) \cup [15,\infty)$\\
C.$x \in (4,9] \cup [15,\infty)$\\
D.$x \in [4,9) \cup [15,\infty)$\\
E.$x \in [4,9] \cup (15,\infty)$\\
F.$x \in (4,9) \cup (15,\infty)$\\
G.$x \in [4,9) \cup (15,\infty)$\\
H.$x \in (4,9] \cup (15,\infty)$
\testStop
\kluczStart
A
\kluczStop



\zadStart{Zadanie z Wikieł Z 1.62 a) moja wersja nr 246}

Rozwiązać nierówności $(x-4)(x-10)(x-11)\ge0$.
\zadStop
\rozwStart{Patryk Wirkus}{Laura Mieczkowska}
Miejsca zerowe naszego wielomianu to: $4, 10, 11$.\\
Wielomian jest stopnia nieparzystego, ponadto znak współczynnika przy\linebreak najwyższej potędze x jest dodatni.\\ W związku z tym wykres wielomianu zaczyna się od lewej strony poniżej osi OX. A więc $$x \in [4,10] \cup [11,\infty).$$
\rozwStop
\odpStart
$x \in [4,10] \cup [11,\infty)$
\odpStop
\testStart
A.$x \in [4,10] \cup [11,\infty)$\\
B.$x \in (4,10) \cup [11,\infty)$\\
C.$x \in (4,10] \cup [11,\infty)$\\
D.$x \in [4,10) \cup [11,\infty)$\\
E.$x \in [4,10] \cup (11,\infty)$\\
F.$x \in (4,10) \cup (11,\infty)$\\
G.$x \in [4,10) \cup (11,\infty)$\\
H.$x \in (4,10] \cup (11,\infty)$
\testStop
\kluczStart
A
\kluczStop



\zadStart{Zadanie z Wikieł Z 1.62 a) moja wersja nr 247}

Rozwiązać nierówności $(x-4)(x-10)(x-12)\ge0$.
\zadStop
\rozwStart{Patryk Wirkus}{Laura Mieczkowska}
Miejsca zerowe naszego wielomianu to: $4, 10, 12$.\\
Wielomian jest stopnia nieparzystego, ponadto znak współczynnika przy\linebreak najwyższej potędze x jest dodatni.\\ W związku z tym wykres wielomianu zaczyna się od lewej strony poniżej osi OX. A więc $$x \in [4,10] \cup [12,\infty).$$
\rozwStop
\odpStart
$x \in [4,10] \cup [12,\infty)$
\odpStop
\testStart
A.$x \in [4,10] \cup [12,\infty)$\\
B.$x \in (4,10) \cup [12,\infty)$\\
C.$x \in (4,10] \cup [12,\infty)$\\
D.$x \in [4,10) \cup [12,\infty)$\\
E.$x \in [4,10] \cup (12,\infty)$\\
F.$x \in (4,10) \cup (12,\infty)$\\
G.$x \in [4,10) \cup (12,\infty)$\\
H.$x \in (4,10] \cup (12,\infty)$
\testStop
\kluczStart
A
\kluczStop



\zadStart{Zadanie z Wikieł Z 1.62 a) moja wersja nr 248}

Rozwiązać nierówności $(x-4)(x-10)(x-13)\ge0$.
\zadStop
\rozwStart{Patryk Wirkus}{Laura Mieczkowska}
Miejsca zerowe naszego wielomianu to: $4, 10, 13$.\\
Wielomian jest stopnia nieparzystego, ponadto znak współczynnika przy\linebreak najwyższej potędze x jest dodatni.\\ W związku z tym wykres wielomianu zaczyna się od lewej strony poniżej osi OX. A więc $$x \in [4,10] \cup [13,\infty).$$
\rozwStop
\odpStart
$x \in [4,10] \cup [13,\infty)$
\odpStop
\testStart
A.$x \in [4,10] \cup [13,\infty)$\\
B.$x \in (4,10) \cup [13,\infty)$\\
C.$x \in (4,10] \cup [13,\infty)$\\
D.$x \in [4,10) \cup [13,\infty)$\\
E.$x \in [4,10] \cup (13,\infty)$\\
F.$x \in (4,10) \cup (13,\infty)$\\
G.$x \in [4,10) \cup (13,\infty)$\\
H.$x \in (4,10] \cup (13,\infty)$
\testStop
\kluczStart
A
\kluczStop



\zadStart{Zadanie z Wikieł Z 1.62 a) moja wersja nr 249}

Rozwiązać nierówności $(x-4)(x-10)(x-14)\ge0$.
\zadStop
\rozwStart{Patryk Wirkus}{Laura Mieczkowska}
Miejsca zerowe naszego wielomianu to: $4, 10, 14$.\\
Wielomian jest stopnia nieparzystego, ponadto znak współczynnika przy\linebreak najwyższej potędze x jest dodatni.\\ W związku z tym wykres wielomianu zaczyna się od lewej strony poniżej osi OX. A więc $$x \in [4,10] \cup [14,\infty).$$
\rozwStop
\odpStart
$x \in [4,10] \cup [14,\infty)$
\odpStop
\testStart
A.$x \in [4,10] \cup [14,\infty)$\\
B.$x \in (4,10) \cup [14,\infty)$\\
C.$x \in (4,10] \cup [14,\infty)$\\
D.$x \in [4,10) \cup [14,\infty)$\\
E.$x \in [4,10] \cup (14,\infty)$\\
F.$x \in (4,10) \cup (14,\infty)$\\
G.$x \in [4,10) \cup (14,\infty)$\\
H.$x \in (4,10] \cup (14,\infty)$
\testStop
\kluczStart
A
\kluczStop



\zadStart{Zadanie z Wikieł Z 1.62 a) moja wersja nr 250}

Rozwiązać nierówności $(x-4)(x-10)(x-15)\ge0$.
\zadStop
\rozwStart{Patryk Wirkus}{Laura Mieczkowska}
Miejsca zerowe naszego wielomianu to: $4, 10, 15$.\\
Wielomian jest stopnia nieparzystego, ponadto znak współczynnika przy\linebreak najwyższej potędze x jest dodatni.\\ W związku z tym wykres wielomianu zaczyna się od lewej strony poniżej osi OX. A więc $$x \in [4,10] \cup [15,\infty).$$
\rozwStop
\odpStart
$x \in [4,10] \cup [15,\infty)$
\odpStop
\testStart
A.$x \in [4,10] \cup [15,\infty)$\\
B.$x \in (4,10) \cup [15,\infty)$\\
C.$x \in (4,10] \cup [15,\infty)$\\
D.$x \in [4,10) \cup [15,\infty)$\\
E.$x \in [4,10] \cup (15,\infty)$\\
F.$x \in (4,10) \cup (15,\infty)$\\
G.$x \in [4,10) \cup (15,\infty)$\\
H.$x \in (4,10] \cup (15,\infty)$
\testStop
\kluczStart
A
\kluczStop



\zadStart{Zadanie z Wikieł Z 1.62 a) moja wersja nr 251}

Rozwiązać nierówności $(x-5)(x-6)(x-7)\ge0$.
\zadStop
\rozwStart{Patryk Wirkus}{Laura Mieczkowska}
Miejsca zerowe naszego wielomianu to: $5, 6, 7$.\\
Wielomian jest stopnia nieparzystego, ponadto znak współczynnika przy\linebreak najwyższej potędze x jest dodatni.\\ W związku z tym wykres wielomianu zaczyna się od lewej strony poniżej osi OX. A więc $$x \in [5,6] \cup [7,\infty).$$
\rozwStop
\odpStart
$x \in [5,6] \cup [7,\infty)$
\odpStop
\testStart
A.$x \in [5,6] \cup [7,\infty)$\\
B.$x \in (5,6) \cup [7,\infty)$\\
C.$x \in (5,6] \cup [7,\infty)$\\
D.$x \in [5,6) \cup [7,\infty)$\\
E.$x \in [5,6] \cup (7,\infty)$\\
F.$x \in (5,6) \cup (7,\infty)$\\
G.$x \in [5,6) \cup (7,\infty)$\\
H.$x \in (5,6] \cup (7,\infty)$
\testStop
\kluczStart
A
\kluczStop



\zadStart{Zadanie z Wikieł Z 1.62 a) moja wersja nr 252}

Rozwiązać nierówności $(x-5)(x-6)(x-8)\ge0$.
\zadStop
\rozwStart{Patryk Wirkus}{Laura Mieczkowska}
Miejsca zerowe naszego wielomianu to: $5, 6, 8$.\\
Wielomian jest stopnia nieparzystego, ponadto znak współczynnika przy\linebreak najwyższej potędze x jest dodatni.\\ W związku z tym wykres wielomianu zaczyna się od lewej strony poniżej osi OX. A więc $$x \in [5,6] \cup [8,\infty).$$
\rozwStop
\odpStart
$x \in [5,6] \cup [8,\infty)$
\odpStop
\testStart
A.$x \in [5,6] \cup [8,\infty)$\\
B.$x \in (5,6) \cup [8,\infty)$\\
C.$x \in (5,6] \cup [8,\infty)$\\
D.$x \in [5,6) \cup [8,\infty)$\\
E.$x \in [5,6] \cup (8,\infty)$\\
F.$x \in (5,6) \cup (8,\infty)$\\
G.$x \in [5,6) \cup (8,\infty)$\\
H.$x \in (5,6] \cup (8,\infty)$
\testStop
\kluczStart
A
\kluczStop



\zadStart{Zadanie z Wikieł Z 1.62 a) moja wersja nr 253}

Rozwiązać nierówności $(x-5)(x-6)(x-9)\ge0$.
\zadStop
\rozwStart{Patryk Wirkus}{Laura Mieczkowska}
Miejsca zerowe naszego wielomianu to: $5, 6, 9$.\\
Wielomian jest stopnia nieparzystego, ponadto znak współczynnika przy\linebreak najwyższej potędze x jest dodatni.\\ W związku z tym wykres wielomianu zaczyna się od lewej strony poniżej osi OX. A więc $$x \in [5,6] \cup [9,\infty).$$
\rozwStop
\odpStart
$x \in [5,6] \cup [9,\infty)$
\odpStop
\testStart
A.$x \in [5,6] \cup [9,\infty)$\\
B.$x \in (5,6) \cup [9,\infty)$\\
C.$x \in (5,6] \cup [9,\infty)$\\
D.$x \in [5,6) \cup [9,\infty)$\\
E.$x \in [5,6] \cup (9,\infty)$\\
F.$x \in (5,6) \cup (9,\infty)$\\
G.$x \in [5,6) \cup (9,\infty)$\\
H.$x \in (5,6] \cup (9,\infty)$
\testStop
\kluczStart
A
\kluczStop



\zadStart{Zadanie z Wikieł Z 1.62 a) moja wersja nr 254}

Rozwiązać nierówności $(x-5)(x-6)(x-10)\ge0$.
\zadStop
\rozwStart{Patryk Wirkus}{Laura Mieczkowska}
Miejsca zerowe naszego wielomianu to: $5, 6, 10$.\\
Wielomian jest stopnia nieparzystego, ponadto znak współczynnika przy\linebreak najwyższej potędze x jest dodatni.\\ W związku z tym wykres wielomianu zaczyna się od lewej strony poniżej osi OX. A więc $$x \in [5,6] \cup [10,\infty).$$
\rozwStop
\odpStart
$x \in [5,6] \cup [10,\infty)$
\odpStop
\testStart
A.$x \in [5,6] \cup [10,\infty)$\\
B.$x \in (5,6) \cup [10,\infty)$\\
C.$x \in (5,6] \cup [10,\infty)$\\
D.$x \in [5,6) \cup [10,\infty)$\\
E.$x \in [5,6] \cup (10,\infty)$\\
F.$x \in (5,6) \cup (10,\infty)$\\
G.$x \in [5,6) \cup (10,\infty)$\\
H.$x \in (5,6] \cup (10,\infty)$
\testStop
\kluczStart
A
\kluczStop



\zadStart{Zadanie z Wikieł Z 1.62 a) moja wersja nr 255}

Rozwiązać nierówności $(x-5)(x-6)(x-11)\ge0$.
\zadStop
\rozwStart{Patryk Wirkus}{Laura Mieczkowska}
Miejsca zerowe naszego wielomianu to: $5, 6, 11$.\\
Wielomian jest stopnia nieparzystego, ponadto znak współczynnika przy\linebreak najwyższej potędze x jest dodatni.\\ W związku z tym wykres wielomianu zaczyna się od lewej strony poniżej osi OX. A więc $$x \in [5,6] \cup [11,\infty).$$
\rozwStop
\odpStart
$x \in [5,6] \cup [11,\infty)$
\odpStop
\testStart
A.$x \in [5,6] \cup [11,\infty)$\\
B.$x \in (5,6) \cup [11,\infty)$\\
C.$x \in (5,6] \cup [11,\infty)$\\
D.$x \in [5,6) \cup [11,\infty)$\\
E.$x \in [5,6] \cup (11,\infty)$\\
F.$x \in (5,6) \cup (11,\infty)$\\
G.$x \in [5,6) \cup (11,\infty)$\\
H.$x \in (5,6] \cup (11,\infty)$
\testStop
\kluczStart
A
\kluczStop



\zadStart{Zadanie z Wikieł Z 1.62 a) moja wersja nr 256}

Rozwiązać nierówności $(x-5)(x-6)(x-12)\ge0$.
\zadStop
\rozwStart{Patryk Wirkus}{Laura Mieczkowska}
Miejsca zerowe naszego wielomianu to: $5, 6, 12$.\\
Wielomian jest stopnia nieparzystego, ponadto znak współczynnika przy\linebreak najwyższej potędze x jest dodatni.\\ W związku z tym wykres wielomianu zaczyna się od lewej strony poniżej osi OX. A więc $$x \in [5,6] \cup [12,\infty).$$
\rozwStop
\odpStart
$x \in [5,6] \cup [12,\infty)$
\odpStop
\testStart
A.$x \in [5,6] \cup [12,\infty)$\\
B.$x \in (5,6) \cup [12,\infty)$\\
C.$x \in (5,6] \cup [12,\infty)$\\
D.$x \in [5,6) \cup [12,\infty)$\\
E.$x \in [5,6] \cup (12,\infty)$\\
F.$x \in (5,6) \cup (12,\infty)$\\
G.$x \in [5,6) \cup (12,\infty)$\\
H.$x \in (5,6] \cup (12,\infty)$
\testStop
\kluczStart
A
\kluczStop



\zadStart{Zadanie z Wikieł Z 1.62 a) moja wersja nr 257}

Rozwiązać nierówności $(x-5)(x-6)(x-13)\ge0$.
\zadStop
\rozwStart{Patryk Wirkus}{Laura Mieczkowska}
Miejsca zerowe naszego wielomianu to: $5, 6, 13$.\\
Wielomian jest stopnia nieparzystego, ponadto znak współczynnika przy\linebreak najwyższej potędze x jest dodatni.\\ W związku z tym wykres wielomianu zaczyna się od lewej strony poniżej osi OX. A więc $$x \in [5,6] \cup [13,\infty).$$
\rozwStop
\odpStart
$x \in [5,6] \cup [13,\infty)$
\odpStop
\testStart
A.$x \in [5,6] \cup [13,\infty)$\\
B.$x \in (5,6) \cup [13,\infty)$\\
C.$x \in (5,6] \cup [13,\infty)$\\
D.$x \in [5,6) \cup [13,\infty)$\\
E.$x \in [5,6] \cup (13,\infty)$\\
F.$x \in (5,6) \cup (13,\infty)$\\
G.$x \in [5,6) \cup (13,\infty)$\\
H.$x \in (5,6] \cup (13,\infty)$
\testStop
\kluczStart
A
\kluczStop



\zadStart{Zadanie z Wikieł Z 1.62 a) moja wersja nr 258}

Rozwiązać nierówności $(x-5)(x-6)(x-14)\ge0$.
\zadStop
\rozwStart{Patryk Wirkus}{Laura Mieczkowska}
Miejsca zerowe naszego wielomianu to: $5, 6, 14$.\\
Wielomian jest stopnia nieparzystego, ponadto znak współczynnika przy\linebreak najwyższej potędze x jest dodatni.\\ W związku z tym wykres wielomianu zaczyna się od lewej strony poniżej osi OX. A więc $$x \in [5,6] \cup [14,\infty).$$
\rozwStop
\odpStart
$x \in [5,6] \cup [14,\infty)$
\odpStop
\testStart
A.$x \in [5,6] \cup [14,\infty)$\\
B.$x \in (5,6) \cup [14,\infty)$\\
C.$x \in (5,6] \cup [14,\infty)$\\
D.$x \in [5,6) \cup [14,\infty)$\\
E.$x \in [5,6] \cup (14,\infty)$\\
F.$x \in (5,6) \cup (14,\infty)$\\
G.$x \in [5,6) \cup (14,\infty)$\\
H.$x \in (5,6] \cup (14,\infty)$
\testStop
\kluczStart
A
\kluczStop



\zadStart{Zadanie z Wikieł Z 1.62 a) moja wersja nr 259}

Rozwiązać nierówności $(x-5)(x-6)(x-15)\ge0$.
\zadStop
\rozwStart{Patryk Wirkus}{Laura Mieczkowska}
Miejsca zerowe naszego wielomianu to: $5, 6, 15$.\\
Wielomian jest stopnia nieparzystego, ponadto znak współczynnika przy\linebreak najwyższej potędze x jest dodatni.\\ W związku z tym wykres wielomianu zaczyna się od lewej strony poniżej osi OX. A więc $$x \in [5,6] \cup [15,\infty).$$
\rozwStop
\odpStart
$x \in [5,6] \cup [15,\infty)$
\odpStop
\testStart
A.$x \in [5,6] \cup [15,\infty)$\\
B.$x \in (5,6) \cup [15,\infty)$\\
C.$x \in (5,6] \cup [15,\infty)$\\
D.$x \in [5,6) \cup [15,\infty)$\\
E.$x \in [5,6] \cup (15,\infty)$\\
F.$x \in (5,6) \cup (15,\infty)$\\
G.$x \in [5,6) \cup (15,\infty)$\\
H.$x \in (5,6] \cup (15,\infty)$
\testStop
\kluczStart
A
\kluczStop



\zadStart{Zadanie z Wikieł Z 1.62 a) moja wersja nr 260}

Rozwiązać nierówności $(x-5)(x-7)(x-8)\ge0$.
\zadStop
\rozwStart{Patryk Wirkus}{Laura Mieczkowska}
Miejsca zerowe naszego wielomianu to: $5, 7, 8$.\\
Wielomian jest stopnia nieparzystego, ponadto znak współczynnika przy\linebreak najwyższej potędze x jest dodatni.\\ W związku z tym wykres wielomianu zaczyna się od lewej strony poniżej osi OX. A więc $$x \in [5,7] \cup [8,\infty).$$
\rozwStop
\odpStart
$x \in [5,7] \cup [8,\infty)$
\odpStop
\testStart
A.$x \in [5,7] \cup [8,\infty)$\\
B.$x \in (5,7) \cup [8,\infty)$\\
C.$x \in (5,7] \cup [8,\infty)$\\
D.$x \in [5,7) \cup [8,\infty)$\\
E.$x \in [5,7] \cup (8,\infty)$\\
F.$x \in (5,7) \cup (8,\infty)$\\
G.$x \in [5,7) \cup (8,\infty)$\\
H.$x \in (5,7] \cup (8,\infty)$
\testStop
\kluczStart
A
\kluczStop



\zadStart{Zadanie z Wikieł Z 1.62 a) moja wersja nr 261}

Rozwiązać nierówności $(x-5)(x-7)(x-9)\ge0$.
\zadStop
\rozwStart{Patryk Wirkus}{Laura Mieczkowska}
Miejsca zerowe naszego wielomianu to: $5, 7, 9$.\\
Wielomian jest stopnia nieparzystego, ponadto znak współczynnika przy\linebreak najwyższej potędze x jest dodatni.\\ W związku z tym wykres wielomianu zaczyna się od lewej strony poniżej osi OX. A więc $$x \in [5,7] \cup [9,\infty).$$
\rozwStop
\odpStart
$x \in [5,7] \cup [9,\infty)$
\odpStop
\testStart
A.$x \in [5,7] \cup [9,\infty)$\\
B.$x \in (5,7) \cup [9,\infty)$\\
C.$x \in (5,7] \cup [9,\infty)$\\
D.$x \in [5,7) \cup [9,\infty)$\\
E.$x \in [5,7] \cup (9,\infty)$\\
F.$x \in (5,7) \cup (9,\infty)$\\
G.$x \in [5,7) \cup (9,\infty)$\\
H.$x \in (5,7] \cup (9,\infty)$
\testStop
\kluczStart
A
\kluczStop



\zadStart{Zadanie z Wikieł Z 1.62 a) moja wersja nr 262}

Rozwiązać nierówności $(x-5)(x-7)(x-10)\ge0$.
\zadStop
\rozwStart{Patryk Wirkus}{Laura Mieczkowska}
Miejsca zerowe naszego wielomianu to: $5, 7, 10$.\\
Wielomian jest stopnia nieparzystego, ponadto znak współczynnika przy\linebreak najwyższej potędze x jest dodatni.\\ W związku z tym wykres wielomianu zaczyna się od lewej strony poniżej osi OX. A więc $$x \in [5,7] \cup [10,\infty).$$
\rozwStop
\odpStart
$x \in [5,7] \cup [10,\infty)$
\odpStop
\testStart
A.$x \in [5,7] \cup [10,\infty)$\\
B.$x \in (5,7) \cup [10,\infty)$\\
C.$x \in (5,7] \cup [10,\infty)$\\
D.$x \in [5,7) \cup [10,\infty)$\\
E.$x \in [5,7] \cup (10,\infty)$\\
F.$x \in (5,7) \cup (10,\infty)$\\
G.$x \in [5,7) \cup (10,\infty)$\\
H.$x \in (5,7] \cup (10,\infty)$
\testStop
\kluczStart
A
\kluczStop



\zadStart{Zadanie z Wikieł Z 1.62 a) moja wersja nr 263}

Rozwiązać nierówności $(x-5)(x-7)(x-11)\ge0$.
\zadStop
\rozwStart{Patryk Wirkus}{Laura Mieczkowska}
Miejsca zerowe naszego wielomianu to: $5, 7, 11$.\\
Wielomian jest stopnia nieparzystego, ponadto znak współczynnika przy\linebreak najwyższej potędze x jest dodatni.\\ W związku z tym wykres wielomianu zaczyna się od lewej strony poniżej osi OX. A więc $$x \in [5,7] \cup [11,\infty).$$
\rozwStop
\odpStart
$x \in [5,7] \cup [11,\infty)$
\odpStop
\testStart
A.$x \in [5,7] \cup [11,\infty)$\\
B.$x \in (5,7) \cup [11,\infty)$\\
C.$x \in (5,7] \cup [11,\infty)$\\
D.$x \in [5,7) \cup [11,\infty)$\\
E.$x \in [5,7] \cup (11,\infty)$\\
F.$x \in (5,7) \cup (11,\infty)$\\
G.$x \in [5,7) \cup (11,\infty)$\\
H.$x \in (5,7] \cup (11,\infty)$
\testStop
\kluczStart
A
\kluczStop



\zadStart{Zadanie z Wikieł Z 1.62 a) moja wersja nr 264}

Rozwiązać nierówności $(x-5)(x-7)(x-12)\ge0$.
\zadStop
\rozwStart{Patryk Wirkus}{Laura Mieczkowska}
Miejsca zerowe naszego wielomianu to: $5, 7, 12$.\\
Wielomian jest stopnia nieparzystego, ponadto znak współczynnika przy\linebreak najwyższej potędze x jest dodatni.\\ W związku z tym wykres wielomianu zaczyna się od lewej strony poniżej osi OX. A więc $$x \in [5,7] \cup [12,\infty).$$
\rozwStop
\odpStart
$x \in [5,7] \cup [12,\infty)$
\odpStop
\testStart
A.$x \in [5,7] \cup [12,\infty)$\\
B.$x \in (5,7) \cup [12,\infty)$\\
C.$x \in (5,7] \cup [12,\infty)$\\
D.$x \in [5,7) \cup [12,\infty)$\\
E.$x \in [5,7] \cup (12,\infty)$\\
F.$x \in (5,7) \cup (12,\infty)$\\
G.$x \in [5,7) \cup (12,\infty)$\\
H.$x \in (5,7] \cup (12,\infty)$
\testStop
\kluczStart
A
\kluczStop



\zadStart{Zadanie z Wikieł Z 1.62 a) moja wersja nr 265}

Rozwiązać nierówności $(x-5)(x-7)(x-13)\ge0$.
\zadStop
\rozwStart{Patryk Wirkus}{Laura Mieczkowska}
Miejsca zerowe naszego wielomianu to: $5, 7, 13$.\\
Wielomian jest stopnia nieparzystego, ponadto znak współczynnika przy\linebreak najwyższej potędze x jest dodatni.\\ W związku z tym wykres wielomianu zaczyna się od lewej strony poniżej osi OX. A więc $$x \in [5,7] \cup [13,\infty).$$
\rozwStop
\odpStart
$x \in [5,7] \cup [13,\infty)$
\odpStop
\testStart
A.$x \in [5,7] \cup [13,\infty)$\\
B.$x \in (5,7) \cup [13,\infty)$\\
C.$x \in (5,7] \cup [13,\infty)$\\
D.$x \in [5,7) \cup [13,\infty)$\\
E.$x \in [5,7] \cup (13,\infty)$\\
F.$x \in (5,7) \cup (13,\infty)$\\
G.$x \in [5,7) \cup (13,\infty)$\\
H.$x \in (5,7] \cup (13,\infty)$
\testStop
\kluczStart
A
\kluczStop



\zadStart{Zadanie z Wikieł Z 1.62 a) moja wersja nr 266}

Rozwiązać nierówności $(x-5)(x-7)(x-14)\ge0$.
\zadStop
\rozwStart{Patryk Wirkus}{Laura Mieczkowska}
Miejsca zerowe naszego wielomianu to: $5, 7, 14$.\\
Wielomian jest stopnia nieparzystego, ponadto znak współczynnika przy\linebreak najwyższej potędze x jest dodatni.\\ W związku z tym wykres wielomianu zaczyna się od lewej strony poniżej osi OX. A więc $$x \in [5,7] \cup [14,\infty).$$
\rozwStop
\odpStart
$x \in [5,7] \cup [14,\infty)$
\odpStop
\testStart
A.$x \in [5,7] \cup [14,\infty)$\\
B.$x \in (5,7) \cup [14,\infty)$\\
C.$x \in (5,7] \cup [14,\infty)$\\
D.$x \in [5,7) \cup [14,\infty)$\\
E.$x \in [5,7] \cup (14,\infty)$\\
F.$x \in (5,7) \cup (14,\infty)$\\
G.$x \in [5,7) \cup (14,\infty)$\\
H.$x \in (5,7] \cup (14,\infty)$
\testStop
\kluczStart
A
\kluczStop



\zadStart{Zadanie z Wikieł Z 1.62 a) moja wersja nr 267}

Rozwiązać nierówności $(x-5)(x-7)(x-15)\ge0$.
\zadStop
\rozwStart{Patryk Wirkus}{Laura Mieczkowska}
Miejsca zerowe naszego wielomianu to: $5, 7, 15$.\\
Wielomian jest stopnia nieparzystego, ponadto znak współczynnika przy\linebreak najwyższej potędze x jest dodatni.\\ W związku z tym wykres wielomianu zaczyna się od lewej strony poniżej osi OX. A więc $$x \in [5,7] \cup [15,\infty).$$
\rozwStop
\odpStart
$x \in [5,7] \cup [15,\infty)$
\odpStop
\testStart
A.$x \in [5,7] \cup [15,\infty)$\\
B.$x \in (5,7) \cup [15,\infty)$\\
C.$x \in (5,7] \cup [15,\infty)$\\
D.$x \in [5,7) \cup [15,\infty)$\\
E.$x \in [5,7] \cup (15,\infty)$\\
F.$x \in (5,7) \cup (15,\infty)$\\
G.$x \in [5,7) \cup (15,\infty)$\\
H.$x \in (5,7] \cup (15,\infty)$
\testStop
\kluczStart
A
\kluczStop



\zadStart{Zadanie z Wikieł Z 1.62 a) moja wersja nr 268}

Rozwiązać nierówności $(x-5)(x-8)(x-9)\ge0$.
\zadStop
\rozwStart{Patryk Wirkus}{Laura Mieczkowska}
Miejsca zerowe naszego wielomianu to: $5, 8, 9$.\\
Wielomian jest stopnia nieparzystego, ponadto znak współczynnika przy\linebreak najwyższej potędze x jest dodatni.\\ W związku z tym wykres wielomianu zaczyna się od lewej strony poniżej osi OX. A więc $$x \in [5,8] \cup [9,\infty).$$
\rozwStop
\odpStart
$x \in [5,8] \cup [9,\infty)$
\odpStop
\testStart
A.$x \in [5,8] \cup [9,\infty)$\\
B.$x \in (5,8) \cup [9,\infty)$\\
C.$x \in (5,8] \cup [9,\infty)$\\
D.$x \in [5,8) \cup [9,\infty)$\\
E.$x \in [5,8] \cup (9,\infty)$\\
F.$x \in (5,8) \cup (9,\infty)$\\
G.$x \in [5,8) \cup (9,\infty)$\\
H.$x \in (5,8] \cup (9,\infty)$
\testStop
\kluczStart
A
\kluczStop



\zadStart{Zadanie z Wikieł Z 1.62 a) moja wersja nr 269}

Rozwiązać nierówności $(x-5)(x-8)(x-10)\ge0$.
\zadStop
\rozwStart{Patryk Wirkus}{Laura Mieczkowska}
Miejsca zerowe naszego wielomianu to: $5, 8, 10$.\\
Wielomian jest stopnia nieparzystego, ponadto znak współczynnika przy\linebreak najwyższej potędze x jest dodatni.\\ W związku z tym wykres wielomianu zaczyna się od lewej strony poniżej osi OX. A więc $$x \in [5,8] \cup [10,\infty).$$
\rozwStop
\odpStart
$x \in [5,8] \cup [10,\infty)$
\odpStop
\testStart
A.$x \in [5,8] \cup [10,\infty)$\\
B.$x \in (5,8) \cup [10,\infty)$\\
C.$x \in (5,8] \cup [10,\infty)$\\
D.$x \in [5,8) \cup [10,\infty)$\\
E.$x \in [5,8] \cup (10,\infty)$\\
F.$x \in (5,8) \cup (10,\infty)$\\
G.$x \in [5,8) \cup (10,\infty)$\\
H.$x \in (5,8] \cup (10,\infty)$
\testStop
\kluczStart
A
\kluczStop



\zadStart{Zadanie z Wikieł Z 1.62 a) moja wersja nr 270}

Rozwiązać nierówności $(x-5)(x-8)(x-11)\ge0$.
\zadStop
\rozwStart{Patryk Wirkus}{Laura Mieczkowska}
Miejsca zerowe naszego wielomianu to: $5, 8, 11$.\\
Wielomian jest stopnia nieparzystego, ponadto znak współczynnika przy\linebreak najwyższej potędze x jest dodatni.\\ W związku z tym wykres wielomianu zaczyna się od lewej strony poniżej osi OX. A więc $$x \in [5,8] \cup [11,\infty).$$
\rozwStop
\odpStart
$x \in [5,8] \cup [11,\infty)$
\odpStop
\testStart
A.$x \in [5,8] \cup [11,\infty)$\\
B.$x \in (5,8) \cup [11,\infty)$\\
C.$x \in (5,8] \cup [11,\infty)$\\
D.$x \in [5,8) \cup [11,\infty)$\\
E.$x \in [5,8] \cup (11,\infty)$\\
F.$x \in (5,8) \cup (11,\infty)$\\
G.$x \in [5,8) \cup (11,\infty)$\\
H.$x \in (5,8] \cup (11,\infty)$
\testStop
\kluczStart
A
\kluczStop



\zadStart{Zadanie z Wikieł Z 1.62 a) moja wersja nr 271}

Rozwiązać nierówności $(x-5)(x-8)(x-12)\ge0$.
\zadStop
\rozwStart{Patryk Wirkus}{Laura Mieczkowska}
Miejsca zerowe naszego wielomianu to: $5, 8, 12$.\\
Wielomian jest stopnia nieparzystego, ponadto znak współczynnika przy\linebreak najwyższej potędze x jest dodatni.\\ W związku z tym wykres wielomianu zaczyna się od lewej strony poniżej osi OX. A więc $$x \in [5,8] \cup [12,\infty).$$
\rozwStop
\odpStart
$x \in [5,8] \cup [12,\infty)$
\odpStop
\testStart
A.$x \in [5,8] \cup [12,\infty)$\\
B.$x \in (5,8) \cup [12,\infty)$\\
C.$x \in (5,8] \cup [12,\infty)$\\
D.$x \in [5,8) \cup [12,\infty)$\\
E.$x \in [5,8] \cup (12,\infty)$\\
F.$x \in (5,8) \cup (12,\infty)$\\
G.$x \in [5,8) \cup (12,\infty)$\\
H.$x \in (5,8] \cup (12,\infty)$
\testStop
\kluczStart
A
\kluczStop



\zadStart{Zadanie z Wikieł Z 1.62 a) moja wersja nr 272}

Rozwiązać nierówności $(x-5)(x-8)(x-13)\ge0$.
\zadStop
\rozwStart{Patryk Wirkus}{Laura Mieczkowska}
Miejsca zerowe naszego wielomianu to: $5, 8, 13$.\\
Wielomian jest stopnia nieparzystego, ponadto znak współczynnika przy\linebreak najwyższej potędze x jest dodatni.\\ W związku z tym wykres wielomianu zaczyna się od lewej strony poniżej osi OX. A więc $$x \in [5,8] \cup [13,\infty).$$
\rozwStop
\odpStart
$x \in [5,8] \cup [13,\infty)$
\odpStop
\testStart
A.$x \in [5,8] \cup [13,\infty)$\\
B.$x \in (5,8) \cup [13,\infty)$\\
C.$x \in (5,8] \cup [13,\infty)$\\
D.$x \in [5,8) \cup [13,\infty)$\\
E.$x \in [5,8] \cup (13,\infty)$\\
F.$x \in (5,8) \cup (13,\infty)$\\
G.$x \in [5,8) \cup (13,\infty)$\\
H.$x \in (5,8] \cup (13,\infty)$
\testStop
\kluczStart
A
\kluczStop



\zadStart{Zadanie z Wikieł Z 1.62 a) moja wersja nr 273}

Rozwiązać nierówności $(x-5)(x-8)(x-14)\ge0$.
\zadStop
\rozwStart{Patryk Wirkus}{Laura Mieczkowska}
Miejsca zerowe naszego wielomianu to: $5, 8, 14$.\\
Wielomian jest stopnia nieparzystego, ponadto znak współczynnika przy\linebreak najwyższej potędze x jest dodatni.\\ W związku z tym wykres wielomianu zaczyna się od lewej strony poniżej osi OX. A więc $$x \in [5,8] \cup [14,\infty).$$
\rozwStop
\odpStart
$x \in [5,8] \cup [14,\infty)$
\odpStop
\testStart
A.$x \in [5,8] \cup [14,\infty)$\\
B.$x \in (5,8) \cup [14,\infty)$\\
C.$x \in (5,8] \cup [14,\infty)$\\
D.$x \in [5,8) \cup [14,\infty)$\\
E.$x \in [5,8] \cup (14,\infty)$\\
F.$x \in (5,8) \cup (14,\infty)$\\
G.$x \in [5,8) \cup (14,\infty)$\\
H.$x \in (5,8] \cup (14,\infty)$
\testStop
\kluczStart
A
\kluczStop



\zadStart{Zadanie z Wikieł Z 1.62 a) moja wersja nr 274}

Rozwiązać nierówności $(x-5)(x-8)(x-15)\ge0$.
\zadStop
\rozwStart{Patryk Wirkus}{Laura Mieczkowska}
Miejsca zerowe naszego wielomianu to: $5, 8, 15$.\\
Wielomian jest stopnia nieparzystego, ponadto znak współczynnika przy\linebreak najwyższej potędze x jest dodatni.\\ W związku z tym wykres wielomianu zaczyna się od lewej strony poniżej osi OX. A więc $$x \in [5,8] \cup [15,\infty).$$
\rozwStop
\odpStart
$x \in [5,8] \cup [15,\infty)$
\odpStop
\testStart
A.$x \in [5,8] \cup [15,\infty)$\\
B.$x \in (5,8) \cup [15,\infty)$\\
C.$x \in (5,8] \cup [15,\infty)$\\
D.$x \in [5,8) \cup [15,\infty)$\\
E.$x \in [5,8] \cup (15,\infty)$\\
F.$x \in (5,8) \cup (15,\infty)$\\
G.$x \in [5,8) \cup (15,\infty)$\\
H.$x \in (5,8] \cup (15,\infty)$
\testStop
\kluczStart
A
\kluczStop



\zadStart{Zadanie z Wikieł Z 1.62 a) moja wersja nr 275}

Rozwiązać nierówności $(x-5)(x-9)(x-10)\ge0$.
\zadStop
\rozwStart{Patryk Wirkus}{Laura Mieczkowska}
Miejsca zerowe naszego wielomianu to: $5, 9, 10$.\\
Wielomian jest stopnia nieparzystego, ponadto znak współczynnika przy\linebreak najwyższej potędze x jest dodatni.\\ W związku z tym wykres wielomianu zaczyna się od lewej strony poniżej osi OX. A więc $$x \in [5,9] \cup [10,\infty).$$
\rozwStop
\odpStart
$x \in [5,9] \cup [10,\infty)$
\odpStop
\testStart
A.$x \in [5,9] \cup [10,\infty)$\\
B.$x \in (5,9) \cup [10,\infty)$\\
C.$x \in (5,9] \cup [10,\infty)$\\
D.$x \in [5,9) \cup [10,\infty)$\\
E.$x \in [5,9] \cup (10,\infty)$\\
F.$x \in (5,9) \cup (10,\infty)$\\
G.$x \in [5,9) \cup (10,\infty)$\\
H.$x \in (5,9] \cup (10,\infty)$
\testStop
\kluczStart
A
\kluczStop



\zadStart{Zadanie z Wikieł Z 1.62 a) moja wersja nr 276}

Rozwiązać nierówności $(x-5)(x-9)(x-11)\ge0$.
\zadStop
\rozwStart{Patryk Wirkus}{Laura Mieczkowska}
Miejsca zerowe naszego wielomianu to: $5, 9, 11$.\\
Wielomian jest stopnia nieparzystego, ponadto znak współczynnika przy\linebreak najwyższej potędze x jest dodatni.\\ W związku z tym wykres wielomianu zaczyna się od lewej strony poniżej osi OX. A więc $$x \in [5,9] \cup [11,\infty).$$
\rozwStop
\odpStart
$x \in [5,9] \cup [11,\infty)$
\odpStop
\testStart
A.$x \in [5,9] \cup [11,\infty)$\\
B.$x \in (5,9) \cup [11,\infty)$\\
C.$x \in (5,9] \cup [11,\infty)$\\
D.$x \in [5,9) \cup [11,\infty)$\\
E.$x \in [5,9] \cup (11,\infty)$\\
F.$x \in (5,9) \cup (11,\infty)$\\
G.$x \in [5,9) \cup (11,\infty)$\\
H.$x \in (5,9] \cup (11,\infty)$
\testStop
\kluczStart
A
\kluczStop



\zadStart{Zadanie z Wikieł Z 1.62 a) moja wersja nr 277}

Rozwiązać nierówności $(x-5)(x-9)(x-12)\ge0$.
\zadStop
\rozwStart{Patryk Wirkus}{Laura Mieczkowska}
Miejsca zerowe naszego wielomianu to: $5, 9, 12$.\\
Wielomian jest stopnia nieparzystego, ponadto znak współczynnika przy\linebreak najwyższej potędze x jest dodatni.\\ W związku z tym wykres wielomianu zaczyna się od lewej strony poniżej osi OX. A więc $$x \in [5,9] \cup [12,\infty).$$
\rozwStop
\odpStart
$x \in [5,9] \cup [12,\infty)$
\odpStop
\testStart
A.$x \in [5,9] \cup [12,\infty)$\\
B.$x \in (5,9) \cup [12,\infty)$\\
C.$x \in (5,9] \cup [12,\infty)$\\
D.$x \in [5,9) \cup [12,\infty)$\\
E.$x \in [5,9] \cup (12,\infty)$\\
F.$x \in (5,9) \cup (12,\infty)$\\
G.$x \in [5,9) \cup (12,\infty)$\\
H.$x \in (5,9] \cup (12,\infty)$
\testStop
\kluczStart
A
\kluczStop



\zadStart{Zadanie z Wikieł Z 1.62 a) moja wersja nr 278}

Rozwiązać nierówności $(x-5)(x-9)(x-13)\ge0$.
\zadStop
\rozwStart{Patryk Wirkus}{Laura Mieczkowska}
Miejsca zerowe naszego wielomianu to: $5, 9, 13$.\\
Wielomian jest stopnia nieparzystego, ponadto znak współczynnika przy\linebreak najwyższej potędze x jest dodatni.\\ W związku z tym wykres wielomianu zaczyna się od lewej strony poniżej osi OX. A więc $$x \in [5,9] \cup [13,\infty).$$
\rozwStop
\odpStart
$x \in [5,9] \cup [13,\infty)$
\odpStop
\testStart
A.$x \in [5,9] \cup [13,\infty)$\\
B.$x \in (5,9) \cup [13,\infty)$\\
C.$x \in (5,9] \cup [13,\infty)$\\
D.$x \in [5,9) \cup [13,\infty)$\\
E.$x \in [5,9] \cup (13,\infty)$\\
F.$x \in (5,9) \cup (13,\infty)$\\
G.$x \in [5,9) \cup (13,\infty)$\\
H.$x \in (5,9] \cup (13,\infty)$
\testStop
\kluczStart
A
\kluczStop



\zadStart{Zadanie z Wikieł Z 1.62 a) moja wersja nr 279}

Rozwiązać nierówności $(x-5)(x-9)(x-14)\ge0$.
\zadStop
\rozwStart{Patryk Wirkus}{Laura Mieczkowska}
Miejsca zerowe naszego wielomianu to: $5, 9, 14$.\\
Wielomian jest stopnia nieparzystego, ponadto znak współczynnika przy\linebreak najwyższej potędze x jest dodatni.\\ W związku z tym wykres wielomianu zaczyna się od lewej strony poniżej osi OX. A więc $$x \in [5,9] \cup [14,\infty).$$
\rozwStop
\odpStart
$x \in [5,9] \cup [14,\infty)$
\odpStop
\testStart
A.$x \in [5,9] \cup [14,\infty)$\\
B.$x \in (5,9) \cup [14,\infty)$\\
C.$x \in (5,9] \cup [14,\infty)$\\
D.$x \in [5,9) \cup [14,\infty)$\\
E.$x \in [5,9] \cup (14,\infty)$\\
F.$x \in (5,9) \cup (14,\infty)$\\
G.$x \in [5,9) \cup (14,\infty)$\\
H.$x \in (5,9] \cup (14,\infty)$
\testStop
\kluczStart
A
\kluczStop



\zadStart{Zadanie z Wikieł Z 1.62 a) moja wersja nr 280}

Rozwiązać nierówności $(x-5)(x-9)(x-15)\ge0$.
\zadStop
\rozwStart{Patryk Wirkus}{Laura Mieczkowska}
Miejsca zerowe naszego wielomianu to: $5, 9, 15$.\\
Wielomian jest stopnia nieparzystego, ponadto znak współczynnika przy\linebreak najwyższej potędze x jest dodatni.\\ W związku z tym wykres wielomianu zaczyna się od lewej strony poniżej osi OX. A więc $$x \in [5,9] \cup [15,\infty).$$
\rozwStop
\odpStart
$x \in [5,9] \cup [15,\infty)$
\odpStop
\testStart
A.$x \in [5,9] \cup [15,\infty)$\\
B.$x \in (5,9) \cup [15,\infty)$\\
C.$x \in (5,9] \cup [15,\infty)$\\
D.$x \in [5,9) \cup [15,\infty)$\\
E.$x \in [5,9] \cup (15,\infty)$\\
F.$x \in (5,9) \cup (15,\infty)$\\
G.$x \in [5,9) \cup (15,\infty)$\\
H.$x \in (5,9] \cup (15,\infty)$
\testStop
\kluczStart
A
\kluczStop



\zadStart{Zadanie z Wikieł Z 1.62 a) moja wersja nr 281}

Rozwiązać nierówności $(x-5)(x-10)(x-11)\ge0$.
\zadStop
\rozwStart{Patryk Wirkus}{Laura Mieczkowska}
Miejsca zerowe naszego wielomianu to: $5, 10, 11$.\\
Wielomian jest stopnia nieparzystego, ponadto znak współczynnika przy\linebreak najwyższej potędze x jest dodatni.\\ W związku z tym wykres wielomianu zaczyna się od lewej strony poniżej osi OX. A więc $$x \in [5,10] \cup [11,\infty).$$
\rozwStop
\odpStart
$x \in [5,10] \cup [11,\infty)$
\odpStop
\testStart
A.$x \in [5,10] \cup [11,\infty)$\\
B.$x \in (5,10) \cup [11,\infty)$\\
C.$x \in (5,10] \cup [11,\infty)$\\
D.$x \in [5,10) \cup [11,\infty)$\\
E.$x \in [5,10] \cup (11,\infty)$\\
F.$x \in (5,10) \cup (11,\infty)$\\
G.$x \in [5,10) \cup (11,\infty)$\\
H.$x \in (5,10] \cup (11,\infty)$
\testStop
\kluczStart
A
\kluczStop



\zadStart{Zadanie z Wikieł Z 1.62 a) moja wersja nr 282}

Rozwiązać nierówności $(x-5)(x-10)(x-12)\ge0$.
\zadStop
\rozwStart{Patryk Wirkus}{Laura Mieczkowska}
Miejsca zerowe naszego wielomianu to: $5, 10, 12$.\\
Wielomian jest stopnia nieparzystego, ponadto znak współczynnika przy\linebreak najwyższej potędze x jest dodatni.\\ W związku z tym wykres wielomianu zaczyna się od lewej strony poniżej osi OX. A więc $$x \in [5,10] \cup [12,\infty).$$
\rozwStop
\odpStart
$x \in [5,10] \cup [12,\infty)$
\odpStop
\testStart
A.$x \in [5,10] \cup [12,\infty)$\\
B.$x \in (5,10) \cup [12,\infty)$\\
C.$x \in (5,10] \cup [12,\infty)$\\
D.$x \in [5,10) \cup [12,\infty)$\\
E.$x \in [5,10] \cup (12,\infty)$\\
F.$x \in (5,10) \cup (12,\infty)$\\
G.$x \in [5,10) \cup (12,\infty)$\\
H.$x \in (5,10] \cup (12,\infty)$
\testStop
\kluczStart
A
\kluczStop



\zadStart{Zadanie z Wikieł Z 1.62 a) moja wersja nr 283}

Rozwiązać nierówności $(x-5)(x-10)(x-13)\ge0$.
\zadStop
\rozwStart{Patryk Wirkus}{Laura Mieczkowska}
Miejsca zerowe naszego wielomianu to: $5, 10, 13$.\\
Wielomian jest stopnia nieparzystego, ponadto znak współczynnika przy\linebreak najwyższej potędze x jest dodatni.\\ W związku z tym wykres wielomianu zaczyna się od lewej strony poniżej osi OX. A więc $$x \in [5,10] \cup [13,\infty).$$
\rozwStop
\odpStart
$x \in [5,10] \cup [13,\infty)$
\odpStop
\testStart
A.$x \in [5,10] \cup [13,\infty)$\\
B.$x \in (5,10) \cup [13,\infty)$\\
C.$x \in (5,10] \cup [13,\infty)$\\
D.$x \in [5,10) \cup [13,\infty)$\\
E.$x \in [5,10] \cup (13,\infty)$\\
F.$x \in (5,10) \cup (13,\infty)$\\
G.$x \in [5,10) \cup (13,\infty)$\\
H.$x \in (5,10] \cup (13,\infty)$
\testStop
\kluczStart
A
\kluczStop



\zadStart{Zadanie z Wikieł Z 1.62 a) moja wersja nr 284}

Rozwiązać nierówności $(x-5)(x-10)(x-14)\ge0$.
\zadStop
\rozwStart{Patryk Wirkus}{Laura Mieczkowska}
Miejsca zerowe naszego wielomianu to: $5, 10, 14$.\\
Wielomian jest stopnia nieparzystego, ponadto znak współczynnika przy\linebreak najwyższej potędze x jest dodatni.\\ W związku z tym wykres wielomianu zaczyna się od lewej strony poniżej osi OX. A więc $$x \in [5,10] \cup [14,\infty).$$
\rozwStop
\odpStart
$x \in [5,10] \cup [14,\infty)$
\odpStop
\testStart
A.$x \in [5,10] \cup [14,\infty)$\\
B.$x \in (5,10) \cup [14,\infty)$\\
C.$x \in (5,10] \cup [14,\infty)$\\
D.$x \in [5,10) \cup [14,\infty)$\\
E.$x \in [5,10] \cup (14,\infty)$\\
F.$x \in (5,10) \cup (14,\infty)$\\
G.$x \in [5,10) \cup (14,\infty)$\\
H.$x \in (5,10] \cup (14,\infty)$
\testStop
\kluczStart
A
\kluczStop



\zadStart{Zadanie z Wikieł Z 1.62 a) moja wersja nr 285}

Rozwiązać nierówności $(x-5)(x-10)(x-15)\ge0$.
\zadStop
\rozwStart{Patryk Wirkus}{Laura Mieczkowska}
Miejsca zerowe naszego wielomianu to: $5, 10, 15$.\\
Wielomian jest stopnia nieparzystego, ponadto znak współczynnika przy\linebreak najwyższej potędze x jest dodatni.\\ W związku z tym wykres wielomianu zaczyna się od lewej strony poniżej osi OX. A więc $$x \in [5,10] \cup [15,\infty).$$
\rozwStop
\odpStart
$x \in [5,10] \cup [15,\infty)$
\odpStop
\testStart
A.$x \in [5,10] \cup [15,\infty)$\\
B.$x \in (5,10) \cup [15,\infty)$\\
C.$x \in (5,10] \cup [15,\infty)$\\
D.$x \in [5,10) \cup [15,\infty)$\\
E.$x \in [5,10] \cup (15,\infty)$\\
F.$x \in (5,10) \cup (15,\infty)$\\
G.$x \in [5,10) \cup (15,\infty)$\\
H.$x \in (5,10] \cup (15,\infty)$
\testStop
\kluczStart
A
\kluczStop





\end{document}
