\documentclass[12pt, a4paper]{article}
\usepackage[utf8]{inputenc}
\usepackage{polski}

\usepackage{amsthm}  %pakiet do tworzenia twierdzeń itp.
\usepackage{amsmath} %pakiet do niektórych symboli matematycznych
\usepackage{amssymb} %pakiet do symboli mat., np. \nsubseteq
\usepackage{amsfonts}
\usepackage{graphicx} %obsługa plików graficznych z rozszerzeniem png, jpg
\theoremstyle{definition} %styl dla definicji
\newtheorem{zad}{} 
\title{Multizestaw zadań}
\author{Robert Fidytek}
%\date{\today}
\date{}
\newcounter{liczniksekcji}
\newcommand{\kategoria}[1]{\section{#1}} %olreślamy nazwę kateforii zadań
\newcommand{\zadStart}[1]{\begin{zad}#1\newline} %oznaczenie początku zadania
\newcommand{\zadStop}{\end{zad}}   %oznaczenie końca zadania
%Makra opcjonarne (nie muszą występować):
\newcommand{\rozwStart}[2]{\noindent \textbf{Rozwiązanie (autor #1 , recenzent #2): }\newline} %oznaczenie początku rozwiązania, opcjonarnie można wprowadzić informację o autorze rozwiązania zadania i recenzencie poprawności wykonania rozwiązania zadania
\newcommand{\rozwStop}{\newline}                                            %oznaczenie końca rozwiązania
\newcommand{\odpStart}{\noindent \textbf{Odpowiedź:}\newline}    %oznaczenie początku odpowiedzi końcowej (wypisanie wyniku)
\newcommand{\odpStop}{\newline}                                             %oznaczenie końca odpowiedzi końcowej (wypisanie wyniku)
\newcommand{\testStart}{\noindent \textbf{Test:}\newline} %ewentualne możliwe opcje odpowiedzi testowej: A. ? B. ? C. ? D. ? itd.
\newcommand{\testStop}{\newline} %koniec wprowadzania odpowiedzi testowych
\newcommand{\kluczStart}{\noindent \textbf{Test poprawna odpowiedź:}\newline} %klucz, poprawna odpowiedź pytania testowego (jedna literka): A lub B lub C lub D itd.
\newcommand{\kluczStop}{\newline} %koniec poprawnej odpowiedzi pytania testowego 
\newcommand{\wstawGrafike}[2]{\begin{figure}[h] \includegraphics[scale=#2] {#1} \end{figure}} %gdyby była potrzeba wstawienia obrazka, parametry: nazwa pliku, skala (jak nie wiesz co wpisać, to wpisz 1)

\begin{document}
\maketitle


\kategoria{Wikieł/Z5.20 f}
\zadStart{Zadanie z Wikieł Z 5.20 f) moja wersja nr [nrWersji]}
%[y]:[2]
%[b]:[4,9,16,25,36,49,64,81,100,121,144]
%[a]=random.randint(4,10)
%[c]=int(math.sqrt([b]))
Znaleźć równania asymptot wykresu funkcji $f$ danej wzorem.\\
 $f(x)=\frac{[a]}{\sqrt{[b]-x^2}}$
\zadStop
\rozwStart{Katarzyna Filipowicz}{}
1. Badamy istnienie asymptot pionowych.\\
Ustalmy dziedzinę funkcji.\\
$$
\sqrt{[b]-x^2}\neq0 \wedge [b]-x^2\geq 0\Rightarrow  [b]-x^2>0 \Rightarrow x<[c] \wedge x>-[c]
$$
Więc $D_f:[-[c],[c]]$
Obliczamy granice jednostronne funkcji w punktach $x=-[c]$ oraz $x=[c]$.
$$
lim_{x\rightarrow ([c])^{-}}\frac{[a]}{\sqrt{[b]-x^2}}
=\left[\frac{[a]}{0^{+}}\right]=\infty
$$
Więc prosta o równaniu $x=[c]$ jest asymptotą prawostronną  pionową.
$$
lim_{x\rightarrow (-[c])^{+}}\frac{[a]}{\sqrt{[b]-x^2}}
=\left[\frac{[a]}{0^{+}}\right]=\infty
$$
Więc prosta o równaniu $x=-[c]$ jest asymptotą lewostronną pionową.


2. Badamy istnienie asymptot ukośnych.\\
Ponieważ
$$ 
lim_{x\rightarrow-\infty} \frac{f(x)}{x}
=lim_{x\rightarrow-\infty}\frac{[a]}{x\sqrt{[b]-x^2}}
=lim_{x\rightarrow-\infty}\frac{[a]}{x^2(\sqrt{\frac{[b]}{x^2}-1})}=0
$$ $$
lim_{x\rightarrow-\infty}(f(x)-0\cdot x)
=lim_{x\rightarrow-\infty}\frac{[a]}{\sqrt{[b]-x^2}}
=lim_{x\rightarrow-\infty} \frac{[a]}{x\sqrt{\frac{[b]}{x^2}-1}}=0
$$
$$ 
lim_{x\rightarrow\infty} \frac{f(x)}{x}
=lim_{x\rightarrow\infty}\frac{[a]}{x\sqrt{[b]-x^2}}
=lim_{x\rightarrow\infty}\frac{[a]}{x^2(\sqrt{\frac{[b]}{x^2}-1})}=0
$$ $$
lim_{x\rightarrow\infty}(f(x)-0\cdot x)
=lim_{x\rightarrow\infty}\frac{[a]}{\sqrt{[b]-x^2}}
=lim_{x\rightarrow\infty} \frac{[a]}{x\sqrt{\frac{[b]}{x^2}-1}}=0
$$
Więc prosta o równaniu $y=0\cdot x+0 \Rightarrow y=0$ jest asymptotą ukośną prawostronną.
\rozwStop
\odpStart
$x=-[c],x=[c],y=0$
\odpStop
\testStart
A.$x=-[c],x=[c],y=0$\\
B.$y=-1,y=1$\\
C.$x=-[c],x=[c]$\\
D.$x=-[c],x=1,y=-1,y=1$\\
E.$x=[c],y=-1,y=1$\\
F.$x=-[c],x=[c],y=1$\\
G.$x=-[c],x=[c],y=-1,y=1$\\
H.$x=0,x=0,y=-1,y=1$\\
I.$x=-4,y=-1,y=1$
\testStop
\kluczStart
A
\kluczStop



\end{document}