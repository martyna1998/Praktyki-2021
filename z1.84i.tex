\documentclass[12pt, a4paper]{article}
\usepackage[utf8]{inputenc}
\usepackage{polski}
\usepackage{amsthm}  %pakiet do tworzenia twierdzeń itp.
\usepackage{amsmath} %pakiet do niektórych symboli matematycznych
\usepackage{amssymb} %pakiet do symboli mat., np. \nsubseteq
\usepackage{amsfonts}
\usepackage{graphicx} %obsługa plików graficznych z rozszerzeniem png, jpg
\theoremstyle{definition} %styl dla definicji
\newtheorem{zad}{} 
\title{Multizestaw zadań}
\author{Radosław Grzyb}
%\date{\today}
\date{}
\newcounter{liczniksekcji}
\newcommand{\kategoria}[1]{\section{#1}} %olreślamy nazwę kateforii zadań
\newcommand{\zadStart}[1]{\begin{zad}#1\newline} %oznaczenie początku zadania
\newcommand{\zadStop}{\end{zad}}   %oznaczenie końca zadania
%Makra opcjonarne (nie muszą występować):
\newcommand{\rozwStart}[2]{\noindent \textbf{Rozwiązanie (autor #1 , recenzent #2): }\newline} %oznaczenie początku rozwiązania, opcjonarnie można wprowadzić informację o autorze rozwiązania zadania i recenzencie poprawności wykonania rozwiązania zadania
\newcommand{\rozwStop}{\newline}                                            %oznaczenie końca rozwiązania
\newcommand{\odpStart}{\noindent \textbf{Odpowiedź:}\newline}    %oznaczenie początku odpowiedzi końcowej (wypisanie wyniku)
\newcommand{\odpStop}{\newline}                                             %oznaczenie końca odpowiedzi końcowej (wypisanie wyniku)
\newcommand{\testStart}{\noindent \textbf{Test:}\newline} %ewentualne możliwe opcje odpowiedzi testowej: A. ? B. ? C. ? D. ? itd.
\newcommand{\testStop}{\newline} %koniec wprowadzania odpowiedzi testowych
\newcommand{\kluczStart}{\noindent \textbf{Test poprawna odpowiedź:}\newline} %klucz, poprawna odpowiedź pytania testowego (jedna literka): A lub B lub C lub D itd.
\newcommand{\kluczStop}{\newline} %koniec poprawnej odpowiedzi pytania testowego 
\newcommand{\wstawGrafike}[2]{\begin{figure}[h] \includegraphics[scale=#2] {#1} \end{figure}} %gdyby była potrzeba wstawienia obrazka, parametry: nazwa pliku, skala (jak nie wiesz co wpisać, to wpisz 1)
\begin{document}
\maketitle
\kategoria{Wikieł/Z1.84i}
\zadStart{Zadanie z Wikieł Z 1.84i moja wersja nr [nrWersji]}
%[p1]:[3,5,7,9,11,13,15,17,19,21]
%[c3]=[p1]/2
%[zlywynik1]=[p1]+1
Rozwiązać równanie:
$$\left(\frac{2}{3}\right)^{2-[p1]x}+1.5^{[p1]x+1}-2.25^{[c3]x}=\left(\frac{3}{2}\right)^{[p1]x}-\frac{9}{32}$$
\zadStop
\rozwStart{Radosław Grzyb}{}
$$\left(\frac{3}{2}\right)^{[p1]x-2}+\left(\frac{3}{2}\right)^{[p1]x+1}-\left(\frac{9}{4}\right)^{[c3]x}=\left(\frac{3}{2}\right)^{[p1]x}-\frac{9}{32}$$
$$\left(\frac{3}{2}\right)^{[p1]x-2}+\left(\frac{3}{2}\right)^{[p1]x+1}-\left(\frac{3}{2}\right)^{[p1]x}=\left(\frac{3}{2}\right)^{[p1]x}-\frac{9}{32}$$
$$\left(\frac{2}{3}\right)^{2}\cdot\left(\frac{3}{2}\right)^{[p1]x}+\left(\frac{3}{2}\right)\cdot\left(\frac{3}{2}\right)^{[p1]x}-2\cdot\left(\frac{3}{2}\right)^{[p1]x}=-\frac{9}{32}$$
$$\left(\frac{2}{3}\right)^{2}\cdot\left(\frac{3}{2}\right)^{[p1]x}+\left(\frac{3}{2}\right)\cdot\left(\frac{3}{2}\right)^{[p1]x}-2\cdot\left(\frac{3}{2}\right)^{[p1]x}=-\frac{9}{32}$$
$$-\frac{1}{18}\cdot\left(\frac{3}{2}\right)^{[p1]x}=-\frac{9}{32}$$
Mnożymy obie strony równania przez $-18$ i otrzymujemy:
$$\left(\frac{3}{2}\right)^{[p1]x}=\frac{81}{16}$$
$$\left(\frac{3}{2}\right)^{[p1]x}=\left(\frac{3}{2}\right)^{4}$$
Logarytmując obustronnie otrzymujemy finalny wynik:
$$[p1]x=4\implies x=\frac{4}{[p1]}$$
\rozwStop
\odpStart
$\frac{4}{[p1]}$
\odpStop
\testStart
A.$\frac{4}{[zlywynik1]}$
B.$\frac{4}{[p1]}$
C.$\frac{2}{[p1]}$
D.$\frac{[p1]}{4}$
\testStop
\kluczStart
B
\kluczStop
\end{document}