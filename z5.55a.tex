\documentclass[12pt, a4paper]{article}
\usepackage[utf8]{inputenc}
\usepackage{polski}

\usepackage{amsthm}  %pakiet do tworzenia twierdzeń itp.
\usepackage{amsmath} %pakiet do niektórych symboli matematycznych
\usepackage{amssymb} %pakiet do symboli mat., np. \nsubseteq
\usepackage{amsfonts}
\usepackage{graphicx} %obsługa plików graficznych z rozszerzeniem png, jpg
\theoremstyle{definition} %styl dla definicji
\newtheorem{zad}{} 
\title{Multizestaw zadań}
\author{Robert Fidytek}
%\date{\today}
\date{}
\newcounter{liczniksekcji}
\newcommand{\kategoria}[1]{\section{#1}} %olreślamy nazwę kateforii zadań
\newcommand{\zadStart}[1]{\begin{zad}#1\newline} %oznaczenie początku zadania
\newcommand{\zadStop}{\end{zad}}   %oznaczenie końca zadania
%Makra opcjonarne (nie muszą występować):
\newcommand{\rozwStart}[2]{\noindent \textbf{Rozwiązanie (autor #1 , recenzent #2): }\newline} %oznaczenie początku rozwiązania, opcjonarnie można wprowadzić informację o autorze rozwiązania zadania i recenzencie poprawności wykonania rozwiązania zadania
\newcommand{\rozwStop}{\newline}                                            %oznaczenie końca rozwiązania
\newcommand{\odpStart}{\noindent \textbf{Odpowiedź:}\newline}    %oznaczenie początku odpowiedzi końcowej (wypisanie wyniku)
\newcommand{\odpStop}{\newline}                                             %oznaczenie końca odpowiedzi końcowej (wypisanie wyniku)
\newcommand{\testStart}{\noindent \textbf{Test:}\newline} %ewentualne możliwe opcje odpowiedzi testowej: A. ? B. ? C. ? D. ? itd.
\newcommand{\testStop}{\newline} %koniec wprowadzania odpowiedzi testowych
\newcommand{\kluczStart}{\noindent \textbf{Test poprawna odpowiedź:}\newline} %klucz, poprawna odpowiedź pytania testowego (jedna literka): A lub B lub C lub D itd.
\newcommand{\kluczStop}{\newline} %koniec poprawnej odpowiedzi pytania testowego 
\newcommand{\wstawGrafike}[2]{\begin{figure}[h] \includegraphics[scale=#2] {#1} \end{figure}} %gdyby była potrzeba wstawienia obrazka, parametry: nazwa pliku, skala (jak nie wiesz co wpisać, to wpisz 1)

\begin{document}
\maketitle


\kategoria{Wikieł/Z5.55a}
\zadStart{Zadanie z Wikieł Z 5.55a) moja wersja nr [nrWersji]}
%[a]:[2,3,4,5,6]
%[b]:[2,3,4,5,6]
%[c]:[0,1,2,3]
%[d]=random.randint(0,5)
%[e]:[0,1,2,3,4,5]
%[f]=random.randint(0,10)
%[cc]=[c]*[c]
%[adt]=3*[a]*[d]
%[ee]=[e]*[e]
%[bcp]=2*[b]*[c]
%[w]=[adt]*[ee]-[bcp]*[f]
%[w]!=[adt] and [w]!=[ee] and [w]!=[bcp] and [w]!=[b] and [adt]!=[ee] and [adt]!=[bcp] and [adt]!=[b] and [ee]!=[bcp] and [ee]!=[b] and [bcp]!=[b] and [adt]!=0 and [ee]!=0 and [bcp]!=0 and [b]!=0
Na podstawie podanych wartości $f'([c])=[d],$ $f([c])=[e],$ $g'([cc])=[f]$ obliczyć wartość następującej pochodnej $\frac{d}{dx}\left[[a]f^{3}(x)-[b]g(x^{2})\right]|_{x=[c]}$.
\zadStop
\rozwStart{Justyna Chojecka}{}
Zauważmy, że 
$$\left([a]f^{3}(x)-[b]g(x^{2})\right)'=[a]\left(f^{3}(x)\right)'-[b]\left(g(x^{2})\right)'.$$
Obliczymy najpierw $\left(f^{3}(x)\right)'$.
$$\left(f^{3}(x)\right)'=\left(f(x)\cdot f(x) \cdot f(x)\right)'=f'(x)\cdot f(x)\cdot f(x)+f(x)\cdot \left(f(x)\cdot f(x)\right)'$$$$=f'(x)\cdot f^{2}(x)+f(x)\cdot \left[f'(x)\cdot f(x)+f(x)\cdot f'(x)\right]$$$$=f'(x)\cdot f^{2}(x)+2f'(x)\cdot f^{2}(x)=3\cdot f'(x)\cdot f^{2}(x)$$
Następnie obliczamy $\left(g(x^{2})\right)'$.
$$\left(g(x^{2})\right)'=g'(x^{2})\cdot (x^{2})'=2x\cdot g'(x^{2})$$
Zatem otrzymujemy, że 
$$\left([a]f^{3}(x)-[b]g(x^{2})\right)'=[a]\left(f^{3}(x)\right)'-[b]\left(g(x^{2})\right)'$$$$=[a]\cdot 3\cdot f'(x)\cdot f^{2}(x)-[b]\cdot 2x\cdot g'(x^{2})$$
Obliczamy wartość pochodnej $\left([a]f^{3}(x)-[b]g(x^{2})\right)'$ dla $x=[c]$.
$$\left([a]f^{3}([c])-[b]g([c]^{2})\right)'=[a]\cdot 3 \cdot f'([c])\cdot f^{2}([c])-[b]\cdot 2x\cdot g'([c]^{2})$$$$=[a]\cdot 3 \cdot [d]\cdot [e]^{2}-[b]\cdot 2\cdot [c]\cdot g'([cc])=[adt]\cdot [ee]-[bcp]\cdot [f]=[w]$$
\rozwStop
\odpStart
$[w]$
\odpStop
\testStart
A.$[w]$
B.$[adt]$
C.$[ee]$
D.$[bcp]$
E.$-[adt]$
F.$-[b]$
G.$[b]$
H.$-[ee]$
I.$-[bcp]$
\testStop
\kluczStart
A
\kluczStop



\end{document}