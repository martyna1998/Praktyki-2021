\documentclass[12pt, a4paper]{article}
\usepackage[utf8]{inputenc}
\usepackage{polski}

\usepackage{amsthm}  %pakiet do tworzenia twierdzeń itp.
\usepackage{amsmath} %pakiet do niektórych symboli matematycznych
\usepackage{amssymb} %pakiet do symboli mat., np. \nsubseteq
\usepackage{amsfonts}
\usepackage{graphicx} %obsługa plików graficznych z rozszerzeniem png, jpg
\theoremstyle{definition} %styl dla definicji
\newtheorem{zad}{} 
\title{Multizestaw zadań}
\author{Robert Fidytek}
%\date{\today}
\date{}
\newcounter{liczniksekcji}
\newcommand{\kategoria}[1]{\section{#1}} %olreślamy nazwę kateforii zadań
\newcommand{\zadStart}[1]{\begin{zad}#1\newline} %oznaczenie początku zadania
\newcommand{\zadStop}{\end{zad}}   %oznaczenie końca zadania
%Makra opcjonarne (nie muszą występować):
\newcommand{\rozwStart}[2]{\noindent \textbf{Rozwiązanie (autor #1 , recenzent #2): }\newline} %oznaczenie początku rozwiązania, opcjonarnie można wprowadzić informację o autorze rozwiązania zadania i recenzencie poprawności wykonania rozwiązania zadania
\newcommand{\rozwStop}{\newline}                                            %oznaczenie końca rozwiązania
\newcommand{\odpStart}{\noindent \textbf{Odpowiedź:}\newline}    %oznaczenie początku odpowiedzi końcowej (wypisanie wyniku)
\newcommand{\odpStop}{\newline}                                             %oznaczenie końca odpowiedzi końcowej (wypisanie wyniku)
\newcommand{\testStart}{\noindent \textbf{Test:}\newline} %ewentualne możliwe opcje odpowiedzi testowej: A. ? B. ? C. ? D. ? itd.
\newcommand{\testStop}{\newline} %koniec wprowadzania odpowiedzi testowych
\newcommand{\kluczStart}{\noindent \textbf{Test poprawna odpowiedź:}\newline} %klucz, poprawna odpowiedź pytania testowego (jedna literka): A lub B lub C lub D itd.
\newcommand{\kluczStop}{\newline} %koniec poprawnej odpowiedzi pytania testowego 
\newcommand{\wstawGrafike}[2]{\begin{figure}[h] \includegraphics[scale=#2] {#1} \end{figure}} %gdyby była potrzeba wstawienia obrazka, parametry: nazwa pliku, skala (jak nie wiesz co wpisać, to wpisz 1)

\begin{document}
\maketitle


\kategoria{Wikieł/Z5.20i}
\zadStart{Zadanie z Wikieł Z 5.20i) moja wersja nr [nrWersji]}
%[a]:[1,2,3,4,5]
%[b]:[5,6,7]
%[c]=[b]-2
%[d]=[b]-1
%[e]=[c]-1
%[f]=[a]**[b]
%[g]=[a]**[c]
%[aa]=[a]*[a]
%[l]=[b]*[aa]
%math.gcd([l],[c])==1
Znaleźć równania asymptot wykresu funkcji $f(x)=\frac{x^{[b]}-[f]}{x^{[c]}-[g]}$.
\zadStop
\rozwStart{Justyna Chojecka}{}
1. Badamy istnienie asymptot pionowych.\\
Ustalamy dziedzinę funkcji $f(x)$. Ponieważ $x^{[c]}-[a]\neq 0$, to 
$$D_{f}=(-\infty,[a]) \cup ([a],\infty).$$
Następnie obliczamy granice jednostronne w punkcie $x=[a].$
$$\lim\limits_{x\to [a]^{-}}f(x)=\lim\limits_{x\to [a]^{-}}\frac{x^{[b]}-[f]}{x^{[c]}-[g]}=\left[\frac{0}{0}\right]\overset{l'H} {=}\lim\limits_{x\to [a]^{-}}\frac{[b]x^{[d]}}{[c]x^{[e]}}=\frac{[b]\cdot [a]^{2}}{[c]}=\frac{[l]}{[c]}$$
$$\lim\limits_{x\to [a]^{+}}f(x)=\lim\limits_{x\to [a]^{+}}\frac{x^{[b]}-[f]}{x^{[c]}-[g]}=\left[\frac{0}{0}\right]\overset{l'H} {=}\lim\limits_{x\to [a]^{+}}\frac{[b]x^{[d]}}{[c]x^{[e]}}=\frac{[b]\cdot [a]^{2}}{[c]}=\frac{[l]}{[c]}$$
Skoro $\lim\limits_{x\to [a]^{-}}f(x)=\lim\limits_{x\to [a]^{+}}f(x)=\frac{[l]}{[c]}$, to $\lim\limits_{x\to [a]}f(x)=\frac{[l]}{[c]}$. W punkcie $x=[a]$ otrzymaliśmy granicę właściwą, zatem  prosta o równaniu $x=[a]$ nie jest asymptotą pionową wykresu funkcji $f(x)$.\\
2. Badamy istnienie asymptot ukośnych.\\
$$\lim\limits_{x\to -\infty}\frac{f(x)}{x}=\lim\limits_{x\to -\infty}\frac{x^{[b]}-[f]}{x(x^{[c]}-[g])}=-\infty$$
oraz
$$\lim\limits_{x\to \infty}\frac{f(x)}{x}=\lim\limits_{x\to \infty}\frac{x^{[b]}-[f]}{x(x^{[c]}-[g])}=\infty$$
Zatem wykres nie ma asymptot ukośnych.
\rozwStop
\odpStart
brak
\odpStop
\testStart
A.brak
B.$x=[g]$
C.$y=-[g]$
D.$x=[b]$
E.$y=[b]x+[a]$
F.$x=-[g]$
G.$y=[b]x-[a]$
H.$y=[g]$
I.$y=[b]$
\testStop
\kluczStart
A
\kluczStop



\end{document}