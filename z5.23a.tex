\documentclass[12pt, a4paper]{article}
\usepackage[utf8]{inputenc}
\usepackage{polski}

\usepackage{amsthm}  %pakiet do tworzenia twierdzeń itp.
\usepackage{amsmath} %pakiet do niektórych symboli matematycznych
\usepackage{amssymb} %pakiet do symboli mat., np. \nsubseteq
\usepackage{amsfonts}
\usepackage{graphicx} %obsługa plików graficznych z rozszerzeniem png, jpg
\theoremstyle{definition} %styl dla definicji
\newtheorem{zad}{} 
\title{Multizestaw zadań}
\author{Laura Mieczkowska}
%\date{\today}
\date{}
\newcounter{liczniksekcji}
\newcommand{\kategoria}[1]{\section{#1}} %olreślamy nazwę kateforii zadań
\newcommand{\zadStart}[1]{\begin{zad}#1\newline} %oznaczenie początku zadania
\newcommand{\zadStop}{\end{zad}}   %oznaczenie końca zadania
%Makra opcjonarne (nie muszą występować):
\newcommand{\rozwStart}[2]{\noindent \textbf{Rozwiązanie (autor #1 , recenzent #2): }\newline} %oznaczenie początku rozwiązania, opcjonarnie można wprowadzić informację o autorze rozwiązania zadania i recenzencie poprawności wykonania rozwiązania zadania
\newcommand{\rozwStop}{\newline}                                            %oznaczenie końca rozwiązania
\newcommand{\odpStart}{\noindent \textbf{Odpowiedź:}\newline}    %oznaczenie początku odpowiedzi końcowej (wypisanie wyniku)
\newcommand{\odpStop}{\newline}                                             %oznaczenie końca odpowiedzi końcowej (wypisanie wyniku)
\newcommand{\testStart}{\noindent \textbf{Test:}\newline} %ewentualne możliwe opcje odpowiedzi testowej: A. ? B. ? C. ? D. ? itd.
\newcommand{\testStop}{\newline} %koniec wprowadzania odpowiedzi testowych
\newcommand{\kluczStart}{\noindent \textbf{Test poprawna odpowiedź:}\newline} %klucz, poprawna odpowiedź pytania testowego (jedna literka): A lub B lub C lub D itd.
\newcommand{\kluczStop}{\newline} %koniec poprawnej odpowiedzi pytania testowego 
\newcommand{\wstawGrafike}[2]{\begin{figure}[h] \includegraphics[scale=#2] {#1} \end{figure}} %gdyby była potrzeba wstawienia obrazka, parametry: nazwa pliku, skala (jak nie wiesz co wpisać, to wpisz 1)

\begin{document}
\maketitle


\kategoria{Wikieł/Z5.23a}
\zadStart{Zadanie z Wikieł Z 5.23 a) moja wersja nr [nrWersji]}
%[a]:[2,3,4,5,6,7,8,9,10,11,12,13,14,15,16]
%[b]:[1,2,3,4,5,6,7,8,9,10,11,12,13,14,15,16]
%[c]=[a]*[b]
%[2cc]=[c]/2
%[2c]=int([2cc])
%[d]=2*[a]
%[e]=abs([a]-[d])
%[pierw1]=math.sqrt([2c])
%[pierw]=int([pierw1])
%[licz]=[a]*[pierw]
%[mian]=[b]+[pierw]**2
%[pierw1].is_integer()==True and [2cc].is_integer()==True and math.gcd([licz],[mian])==1
Znaleźć ekstrema lokalne funkcji $y=\frac{[a]x}{x^2+[b]}$.
\zadStop
\rozwStart{Laura Mieczkowska}{}
$$y=\frac{[a]x}{x^2+[b]}$$
$$y'=\frac{([a]x)'(x^2+[b])-[a]x(x^2+[b])'}{(x^2+[b])^2}=
\frac{[a](x^2+[b])-[a]x\cdot2x}{(x^2+[b])^2}=$$
$$=\frac{[a]x^2+[c]-[d]x^2}{(x^2+[b])^2}=\frac{-[e]x^2+[c]}{(x^2+[b])^2}=\frac{-[e](x^2-[2c])}{(x^2+[b])^2}$$
\\
$$\frac{-[e](x^2-[2c])}{(x^2+[b])^2}=0 \Rightarrow -[e](x^2-[2c])=0 \Rightarrow
x_1=[pierw] \vee x_2=-[pierw]$$
$$y(x_1)=y([pierw])=\frac{[a]\cdot[pierw]}{[pierw]^2+[b]}=\frac{[licz]}{[mian]}$$
$$y(x_2)=y(-[pierw])=\frac{[a]\cdot(-[pierw])}{(-[pierw])^2+[b]}=-\frac{[licz]}{[mian]}$$

\odpStart
$y_{min}=y(-[pierw])=-\frac{[licz]}{[mian]}$, $y_{max}=y([pierw])=\frac{[licz]}{[mian]}$
\odpStop
\testStart
A. $y_{min}=y(-[pierw])=-\frac{[licz]}{[mian]}$, $y_{max}=y([pierw])=\frac{[licz]}{[mian]}$\\
B. $y_{max}=y(-[pierw])=-\frac{[licz]}{[mian]}$, $y_{min}=y([pierw])=\frac{[licz]}{[mian]}$ \\
C. $y_{min}=y(-[pierw])=\frac{[licz]}{[mian]}$, $y_{max}=y([pierw])=-\frac{[licz]}{[mian]}$ \\
D. $y_{min}=y(-[pierw])=-\frac{1}{[mian]}$, $y_{max}=y([pierw])=\frac{1}{[mian]}$ 
\testStop
\kluczStart
A
\kluczStop



\end{document}