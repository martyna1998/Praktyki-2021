\documentclass[12pt, a4paper]{article}
\usepackage[utf8]{inputenc}
\usepackage{polski}

\usepackage{amsthm}  %pakiet do tworzenia twierdzeń itp.
\usepackage{amsmath} %pakiet do niektórych symboli matematycznych
\usepackage{amssymb} %pakiet do symboli mat., np. \nsubseteq
\usepackage{amsfonts}
\usepackage{graphicx} %obsługa plików graficznych z rozszerzeniem png, jpg
\theoremstyle{definition} %styl dla definicji
\newtheorem{zad}{} 
\title{Multizestaw zadań}
\author{Robert Fidytek}
%\date{\today}
\date{}
\newcounter{liczniksekcji}
\newcommand{\kategoria}[1]{\section{#1}} %olreślamy nazwę kateforii zadań
\newcommand{\zadStart}[1]{\begin{zad}#1\newline} %oznaczenie początku zadania
\newcommand{\zadStop}{\end{zad}}   %oznaczenie końca zadania
%Makra opcjonarne (nie muszą występować):
\newcommand{\rozwStart}[2]{\noindent \textbf{Rozwiązanie (autor #1 , recenzent #2): }\newline} %oznaczenie początku rozwiązania, opcjonarnie można wprowadzić informację o autorze rozwiązania zadania i recenzencie poprawności wykonania rozwiązania zadania
\newcommand{\rozwStop}{\newline}                                            %oznaczenie końca rozwiązania
\newcommand{\odpStart}{\noindent \textbf{Odpowiedź:}\newline}    %oznaczenie początku odpowiedzi końcowej (wypisanie wyniku)
\newcommand{\odpStop}{\newline}                                             %oznaczenie końca odpowiedzi końcowej (wypisanie wyniku)
\newcommand{\testStart}{\noindent \textbf{Test:}\newline} %ewentualne możliwe opcje odpowiedzi testowej: A. ? B. ? C. ? D. ? itd.
\newcommand{\testStop}{\newline} %koniec wprowadzania odpowiedzi testowych
\newcommand{\kluczStart}{\noindent \textbf{Test poprawna odpowiedź:}\newline} %klucz, poprawna odpowiedź pytania testowego (jedna literka): A lub B lub C lub D itd.
\newcommand{\kluczStop}{\newline} %koniec poprawnej odpowiedzi pytania testowego 
\newcommand{\wstawGrafike}[2]{\begin{figure}[h] \includegraphics[scale=#2] {#1} \end{figure}} %gdyby była potrzeba wstawienia obrazka, parametry: nazwa pliku, skala (jak nie wiesz co wpisać, to wpisz 1)

\begin{document}
\maketitle


\kategoria{Wikieł/Z1.89t}
\zadStart{Zadanie z Wikieł Z 1.89 t) moja wersja nr [nrWersji]}
%[a]:[5,6,7,10,11,12]
%[b]:[2,3,4,5,6,7,8,9]
%[c]:[2,3,4,5,6,7,8,9]
%[k]:[2,3,4,5,6,7,8,9]
%[p]=(pow([a],2))
%[p2]=(pow([c],2))
%[p3]=(pow([b],2))
%[z]=([k]/[p2])
%[z1]=int([z])
%[zz]=[z1]*[p3]
%[p2]<81 and [a]!=[b] and [b]!=[c] and [a]!=[c] and [z].is_integer()==True
Oblicz warto\'sć wyrażenia $[k] \cdot [p]^{\frac{1}{2}\log_{[a]}{[p3]}-\log_{[a]}{[c]}}$.
\zadStop
\rozwStart{Małgorzata Ugowska}{}
$$[k] \cdot [p]^{\frac{1}{2}\log_{[a]}{[p3]}-\log_{[a]}{[c]}} = [k] \cdot [p]^{\log_{[a]}{[b]}-\log_{[a]}{[c]}} = [k] \cdot ([a]^2)^{\log_{[a]}{\frac{[b]}{[c]}}}$$
$$=[k] \cdot [a]^{\log_{[a]}{(\frac{[b]}{[c]})^2}} = [k] \cdot \Big(\frac{[b]}{[c]}\Big)^2 = [k] \cdot \frac{[p3]}{[p2]} = [z1] \cdot [p3] = [zz]$$
\rozwStop
\odpStart
$[zz]$
\odpStop
\testStart
A. $[b]$\\
B. $[p3]$\\
C. $31$\\
D. $[p2]$\\
E. $[zz]$
\testStop
\kluczStart
E
\kluczStop



\end{document}