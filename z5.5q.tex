\documentclass[12pt, a4paper]{article}
\usepackage[utf8]{inputenc}
\usepackage{polski}

\usepackage{amsthm}  %pakiet do tworzenia twierdzeń itp.
\usepackage{amsmath} %pakiet do niektórych symboli matematycznych
\usepackage{amssymb} %pakiet do symboli mat., np. \nsubseteq
\usepackage{amsfonts}
\usepackage{graphicx} %obsługa plików graficznych z rozszerzeniem png, jpg
\theoremstyle{definition} %styl dla definicji
\newtheorem{zad}{} 
\title{Multizestaw zadań}
\author{Robert Fidytek}
%\date{\today}
\date{}
\newcounter{liczniksekcji}
\newcommand{\kategoria}[1]{\section{#1}} %olreślamy nazwę kateforii zadań
\newcommand{\zadStart}[1]{\begin{zad}#1\newline} %oznaczenie początku zadania
\newcommand{\zadStop}{\end{zad}}   %oznaczenie końca zadania
%Makra opcjonarne (nie muszą występować):
\newcommand{\rozwStart}[2]{\noindent \textbf{Rozwiązanie (autor #1 , recenzent #2): }\newline} %oznaczenie początku rozwiązania, opcjonarnie można wprowadzić informację o autorze rozwiązania zadania i recenzencie poprawności wykonania rozwiązania zadania
\newcommand{\rozwStop}{\newline}                                            %oznaczenie końca rozwiązania
\newcommand{\odpStart}{\noindent \textbf{Odpowiedź:}\newline}    %oznaczenie początku odpowiedzi końcowej (wypisanie wyniku)
\newcommand{\odpStop}{\newline}                                             %oznaczenie końca odpowiedzi końcowej (wypisanie wyniku)
\newcommand{\testStart}{\noindent \textbf{Test:}\newline} %ewentualne możliwe opcje odpowiedzi testowej: A. ? B. ? C. ? D. ? itd.
\newcommand{\testStop}{\newline} %koniec wprowadzania odpowiedzi testowych
\newcommand{\kluczStart}{\noindent \textbf{Test poprawna odpowiedź:}\newline} %klucz, poprawna odpowiedź pytania testowego (jedna literka): A lub B lub C lub D itd.
\newcommand{\kluczStop}{\newline} %koniec poprawnej odpowiedzi pytania testowego 
\newcommand{\wstawGrafike}[2]{\begin{figure}[h] \includegraphics[scale=#2] {#1} \end{figure}} %gdyby była potrzeba wstawienia obrazka, parametry: nazwa pliku, skala (jak nie wiesz co wpisać, to wpisz 1)

\begin{document}
\maketitle


\kategoria{Wikieł/Z5.5q}
\zadStart{Zadanie z Wikieł Z 5.5 q) moja wersja nr [nrWersji]}
%[x]:[2,3,4,5,6,7,8,9,10,11,12,13]
%[y]:[2,3,4,5,6,7,8,9,10,11,12,13]
%[a]=random.randint(2,10)
%[b]=random.randint(2,10)
%[c]=random.randint(2,10)
%[d]=random.randint(2,10)
%[e]=random.randint(2,10)
%[f]=random.randint(2,10)
%[g]=4*[a]*[e]
%[h]=3*[b]*[e]
%[i]=2*[c]*[e]
%[j]=[d]*[e]
%[u]=4*[f]*[a]
%[l]=[f]*3*[b]
%[m]=2*[f]*[c]
%[n]=[f]*[d]
%[k]=2*[e]*[a]
%[p]=2*[e]*[b]
%[r]=2*[e]*[c]
%[s]=2*[e]*[d]
%[t1]=2*[a]*[e]
%[t2]=[e]*[b]
%[t3]=([d]*[e])-([f]*3*[b])
%[j]>[l]
Korzystając z podstawowych twierdzeń i wzorów, wyznaczyć pochodną funkcji (bez określania zakresu zmienności $x$).\\ 
$f(x)=\frac{[a]x^4+[b]x^3-[c]x^2-[d]x}{[e]x^2-[f]}$.
\zadStop
\rozwStart{Katarzyna Filipowicz}{}
$$f(x)=\frac{[a]x^4+[b]x^3-[c]x^2-[d]x}{[e]x^2-[f]}$$
$$f'(x)=\left(\frac{[a]x^4+[b]x^3-[c]x^2-[d]x}{[e]x^2-[f]}\right)' = $$
$$ = \frac{(4\cdot[a]x^3+3\cdot[b]x^2-2\cdot[c] x-[d])([e]x^2-[f])-([a]x^4+[b]x^3-[c]x^2-[d]x)(2\cdot[e]x)}{([e]x^2-[f])^2}=
$$ $$
=\frac{(4\cdot[a]\cdot[e] x^5+3\cdot[b]\cdot[e] x^4-2\cdot[c]\cdot[e] x^3-[d]\cdot[e] x^2-[f]\cdot4\cdot[a] x^3-[f]\cdot3\cdot[b] x^2+2\cdot[f]\cdot[c] x+[f]\cdot[d])}{([e]x^2-[f])^2}-
$$ $$
-\frac{(2\cdot [e]\cdot [a] x^5+2\cdot [e]\cdot [b]x^4-2\cdot [e]\cdot [c]x^3-[d]\cdot 2\cdot [e] x^2)}{([e]x^2-[f])^2}=
$$ $$
=\frac{[g]x^5+[h]x^4-[i]x^3-[j]x^2-[u]x^3-[l]x^2+[m]x+[n]-[k]x^5-[p]x^4+[r]x^3+[s]x^2}{([e]x^2-[f])^2}=
$$ $$
=\frac{[k]x^5+[t2]x^4-[u]x^3+[t3]x^2+[m]x+[n]}{([e]x^2-[f])^2}
$$
\rozwStop
\odpStart
$ f'(x)=\frac{[k]x^5+[t2]x^4-[u]x^3+[t3]x^2+[m]x+[n]}{([e]x^2-[f])^2}$
\odpStop
\testStart
A. $ f'(x)=\frac{[k]x^5+[t2]x^4-[u]x^3+[t3]x^2+[m]x+[n]}{([e]x^2-[f])^2}$\\
B. $ f'(x)=\frac{[g]x^5+[t2]x^4-[u]x^3+[t3]x^2+[m]x+[n]}{([e]x^2-[f])^2}$\\
C. $ f'(x)=\frac{[a]x^5+[b]x^4-[u]x^3+[t3]x^2+[m]x+[n]}{([e]x^2-[f])^2}$ \\
D. $ f'(x)=\frac{[k]x^5+[l]x^4-[u]x^3+[t3]x^2+[m]x+[n]}{([e]x^2-[f])^2}$\\
E. $ f'(x)=\frac{[k]x^5+[t2]x^4-[u]x^3+[t3]x^2+[m]x+[n]}{([e]x^2-[f])}$\\
F. $ f'(x)=\frac{[k]x^5+[t2]x^4-[u]x^3+[t3]x^2+[m]x+[j]}{([e]x^2-[f])^2}$
\testStop
\kluczStart
A
\kluczStop



\end{document}