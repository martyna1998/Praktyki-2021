\documentclass[12pt, a4paper]{article}
\usepackage[utf8]{inputenc}
\usepackage{polski}

\usepackage{amsthm}  %pakiet do tworzenia twierdzeń itp.
\usepackage{amsmath} %pakiet do niektórych symboli matematycznych
\usepackage{amssymb} %pakiet do symboli mat., np. \nsubseteq
\usepackage{amsfonts}
\usepackage{graphicx} %obsługa plików graficznych z rozszerzeniem png, jpg
\theoremstyle{definition} %styl dla definicji
\newtheorem{zad}{} 
\title{Multizestaw zadań}
\author{Robert Fidytek}
%\date{\today}
\date{}
\newcounter{liczniksekcji}
\newcommand{\kategoria}[1]{\section{#1}} %olreślamy nazwę kateforii zadań
\newcommand{\zadStart}[1]{\begin{zad}#1\newline} %oznaczenie początku zadania
\newcommand{\zadStop}{\end{zad}}   %oznaczenie końca zadania
%Makra opcjonarne (nie muszą występować):
\newcommand{\rozwStart}[2]{\noindent \textbf{Rozwiązanie (autor #1 , recenzent #2): }\newline} %oznaczenie początku rozwiązania, opcjonarnie można wprowadzić informację o autorze rozwiązania zadania i recenzencie poprawności wykonania rozwiązania zadania
\newcommand{\rozwStop}{\newline}                                            %oznaczenie końca rozwiązania
\newcommand{\odpStart}{\noindent \textbf{Odpowiedź:}\newline}    %oznaczenie początku odpowiedzi końcowej (wypisanie wyniku)
\newcommand{\odpStop}{\newline}                                             %oznaczenie końca odpowiedzi końcowej (wypisanie wyniku)
\newcommand{\testStart}{\noindent \textbf{Test:}\newline} %ewentualne możliwe opcje odpowiedzi testowej: A. ? B. ? C. ? D. ? itd.
\newcommand{\testStop}{\newline} %koniec wprowadzania odpowiedzi testowych
\newcommand{\kluczStart}{\noindent \textbf{Test poprawna odpowiedź:}\newline} %klucz, poprawna odpowiedź pytania testowego (jedna literka): A lub B lub C lub D itd.
\newcommand{\kluczStop}{\newline} %koniec poprawnej odpowiedzi pytania testowego 
\newcommand{\wstawGrafike}[2]{\begin{figure}[h] \includegraphics[scale=#2] {#1} \end{figure}} %gdyby była potrzeba wstawienia obrazka, parametry: nazwa pliku, skala (jak nie wiesz co wpisać, to wpisz 1)

\begin{document}
\maketitle


\kategoria{Wikieł/Z5.6f}
\zadStart{Zadanie z Wikieł Z 5.6f) moja wersja nr [nrWersji]}
%[a]:[2,3,4,5,6,7,8,9]
%[b]:[2,3,4,5,6,7,8,9]
%[c]=random.randint(2,10)
%[d]=random.randint(2,10)
%[a1]=[b]*[c]
%[a2]=[d]*[a]
%[a3]=[a1]-[a2]
%(-[a]/[b])<(-[c]/[d]) and math.gcd([a],[b])==1 and math.gcd([c],[d])==1
Obliczyć pochodną funkcji $f$ oraz określić dziedzinę funkcji $f$ i funkcji pochodnej $f'$.\\
$f(x)=\log{\frac{[a]+[b]x}{[c]+[d]x}}$
\zadStop
\rozwStart{Joanna Świerzbin}{}
Dziedzina $D_f: $
\begin{enumerate}
\item $$[c]+[d]x \neq 0$$
$$[d]x \neq -[c]$$
$$x \neq -\frac{[c]}{[d]}$$
$$x\in \mathbb{R} \backslash \left\{ -\frac{[c]}{[d]} \right\} $$
\item $$ \frac{[a]+[b]x}{[c]+[d]x} >0 $$
$$ ([a]+[b]x)([c]+[d]x) >0 $$
Jest to funkcja homograficzna, która ma miejsce zerowe w punkcje \\ $x=-\frac{[a]}{[b]}$ i asymptotę pionową $x=-\frac{[c]}{[d]}$. Funkcja ta znajduje się nad osią $y$ dla $x \in \left( -\infty , -\frac{[a]}{[b]} \right) \cup \left( -\frac{[c]}{[d]}, \infty \right)$.
\end{enumerate}
$$ f'(x)=\left( \log{\frac{[a]+[b]x}{[c]+[d]x}}  \right)' =$$ 
$$= \frac{1}{\left(\frac{[a]+[b]x}{[c]+[d]x}\right)\ln(10)}\left( \frac{[a]+[b]x}{[c]+[d]x} \right)' =$$ 
$$= \frac{1}{\left(\frac{[a]+[b]x}{[c]+[d]x}\right)\ln(10)}\left( \frac{([a]+[b]x)'([c]+[d]x) - ([a]+[b]x) ([c]+[d]x)'}{([c]+[d]x)^2} \right) =$$ 
$$= \frac{1}{\left(\frac{[a]+[b]x}{[c]+[d]x}\right)\ln(10)}\left( \frac{[b]([c]+[d]x) - ([a]+[b]x) [d]}{([c]+[d]x)^2} \right) =$$ 
$$= \frac{1}{\left(\frac{[a]+[b]x}{[c]+[d]x}\right)\ln(10)}\left( \frac{[b]\cdot[c]+[b]\cdot[d]x - [d]\cdot[a]-[d]\cdot[b]x}{([c]+[d]x)^2} \right) =$$ 
$$= \frac{1}{\left(\frac{[a]+[b]x}{[c]+[d]x}\right)\ln(10)}\left( \frac{[b]\cdot[c] - [d]\cdot[a]}{([c]+[d]x)^2} \right) =$$ 
$$= \frac{[a3]}{([c]+[d]x) \left([a]+[b]x \right)\ln(10)} $$ 
Dziedzina $D_{f'}:$
$$ {([c]+[d]x) \left([a]+[b]x \right)\ln(10)} \neq 0 $$
$$ ([c]+[d]x) ([a]+[b]x) \neq 0 $$
$$ [c]+[d]x \neq 0 \land  [a]+[b]x \neq 0 $$
$$ x \neq -\frac{[c]}{[d]} \land  x \neq -\frac{[a]}{[b]} $$
Jest to parabola, która ma miejsca zerowe w punktach $ x = -\frac{[c]}{[d]}$ i $  x = -\frac{[a]}{[b]} $ Dlatego $D_{f'}:$
$$ x \in \mathbb{R} \backslash \left\{-\frac{[a]}{[b]}, -\frac{[c]}{[d]} \right\} $$
\rozwStop
\odpStart
$f'(x)= \frac{[a3]}{([c]+[d]x) \left([a]+[b]x \right)\ln(10)}, D_f: x \in \left( -\infty , -\frac{[a]}{[b]} \right) \cup \left( -\frac{[c]}{[d]}, \infty \right) , \\ D_{f'}: x \in \mathbb{R} \backslash \left\{-\frac{[a]}{[b]}, -\frac{[c]}{[d]} \right\}$
\odpStop
\testStart
A. $f'(x)= \frac{[a3]}{([c]+[d]x) \left([a]+[b]x \right)\ln(10)}, D_f: x \in \left( -\infty , -\frac{[a]}{[b]} \right) \cup \left( -\frac{[c]}{[d]}, \infty \right) ,\\  D_{f'}: x \in \mathbb(R) \backslash \left\{-\frac{[a]}{[b]}, -\frac{[c]}{[d]} \right\}$\\
B. $f'(x)= \frac{[b]}{([c]+[d]x) \left([a]+[b]x \right)\ln(10)}, D_f: x \in \left( -\infty , -\frac{[a]}{[b]} \right) \cup \left( -\frac{[c]}{[d]}, \infty \right) , \\ D_{f'}: x \in \mathbb{R} \backslash \left\{-\frac{[a]}{[b]}, -\frac{[c]}{[d]} \right\}$\\
C. $f'(x)= \frac{1}{([c]+[d]x) \left([a]+[b]x \right)\ln([a])}, D_f: x \in \left( -\infty , -\frac{[a]}{[b]} \right) \cup \left( -\frac{[c]}{[d]}, \infty \right) , \\ D_{f'}: x \in \mathbb{R} \backslash \left\{-\frac{[a]}{[b]}, -\frac{[c]}{[d]} \right\}$\\
D. $f'(x)= \frac{[a3]}{([c]+[d]x) \left([a]+[b]x \right)\ln(10)}, D_f: x \in \left( -\infty , -\frac{[a]}{[b]} \right) \cup \left( -\frac{[c]}{[d]}, \infty \right) , \\ D_{f'}: x \in \mathbb{R} $\\
E. $f'(x)= \frac{[a3]}{([d]x) \left([a]+[b]x \right)\ln(10)}, D_f: x \in \left( -\infty , -\frac{[a]}{[b]} \right) \cup \left( -\frac{[c]}{[d]}, \infty \right) , \\ D_{f'}: x \in \mathbb{R} \backslash \left\{-\frac{[a]}{[b]}, -\frac{[c]}{[d]} \right\}$\\
F. $f'(x)= \frac{[a3]}{([c]+[d]x) \left([a]+[b]x \right)}, D_f: x \in \left( -\infty , -\frac{[a]}{[b]} \right) \cup \left( -\frac{[c]}{[d]}, \infty \right) , \\ D_{f'}: x \in \mathbb{R} \backslash \left\{-\frac{[a]}{[b]}, -\frac{[c]}{[d]} \right\}$
\testStop
\kluczStart
A
\kluczStop



\end{document}