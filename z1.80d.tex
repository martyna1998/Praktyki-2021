\documentclass[12pt, a4paper]{article}
\usepackage[utf8]{inputenc}
\usepackage{polski}

\usepackage{amsthm}  %pakiet do tworzenia twierdzeń itp.
\usepackage{amsmath} %pakiet do niektórych symboli matematycznych
\usepackage{amssymb} %pakiet do symboli mat., np. \nsubseteq
\usepackage{amsfonts}
\usepackage{graphicx} %obsługa plików graficznych z rozszerzeniem png, jpg
\theoremstyle{definition} %styl dla definicji
\newtheorem{zad}{} 
\title{Multizestaw zadań}
\author{Robert Fidytek}
%\date{\today}
\date{}
\newcounter{liczniksekcji}
\newcommand{\kategoria}[1]{\section{#1}} %olreślamy nazwę kateforii zadań
\newcommand{\zadStart}[1]{\begin{zad}#1\newline} %oznaczenie początku zadania
\newcommand{\zadStop}{\end{zad}}   %oznaczenie końca zadania
%Makra opcjonarne (nie muszą występować):
\newcommand{\rozwStart}[2]{\noindent \textbf{Rozwiązanie (autor #1 , recenzent #2): }\newline} %oznaczenie początku rozwiązania, opcjonarnie można wprowadzić informację o autorze rozwiązania zadania i recenzencie poprawności wykonania rozwiązania zadania
\newcommand{\rozwStop}{\newline}                                            %oznaczenie końca rozwiązania
\newcommand{\odpStart}{\noindent \textbf{Odpowiedź:}\newline}    %oznaczenie początku odpowiedzi końcowej (wypisanie wyniku)
\newcommand{\odpStop}{\newline}                                             %oznaczenie końca odpowiedzi końcowej (wypisanie wyniku)
\newcommand{\testStart}{\noindent \textbf{Test:}\newline} %ewentualne możliwe opcje odpowiedzi testowej: A. ? B. ? C. ? D. ? itd.
\newcommand{\testStop}{\newline} %koniec wprowadzania odpowiedzi testowych
\newcommand{\kluczStart}{\noindent \textbf{Test poprawna odpowiedź:}\newline} %klucz, poprawna odpowiedź pytania testowego (jedna literka): A lub B lub C lub D itd.
\newcommand{\kluczStop}{\newline} %koniec poprawnej odpowiedzi pytania testowego 
\newcommand{\wstawGrafike}[2]{\begin{figure}[h] \includegraphics[scale=#2] {#1} \end{figure}} %gdyby była potrzeba wstawienia obrazka, parametry: nazwa pliku, skala (jak nie wiesz co wpisać, to wpisz 1)

\begin{document}
\maketitle


\kategoria{Wikieł/Z1.80d}
\zadStart{Zadanie z Wikieł Z 1.80 d)  moja wersja nr [nrWersji]}

%[p1]:[2,3,4,5,6,7,8,9,10]
%[p2]:[2,3,4,5,6,7,8,9,10]
%[p1k]=[p1]*[p1]
%[p1k1]=[p1k]-1
%[p12]=[p1]+2
%[p21]=-[p2]-1
%[2p1k1]=2*[p1k1]
%[del]=round(([p12]*[p12])/([p1k]*[p1k])-(4*[p1k1]*[p21])/[p1k],2)
%[pdel]=round(math.sqrt(abs([del])),2)
%[x1]=round(((-[p12]/[p1])-[pdel])*[p1k]/[2p1k1],2)
%[x2]=round(((-[p12]/[p1])+[pdel])*[p1k]/[2p1k1],2)
%math.gcd([p1k1],[p1k])==1 and math.gcd([p12],[p1])==1



Rozwiązać równanie 
$$(x^{2}+x-[p2])^{\frac{1}{2}}=\frac{1}{[p1]}x-1$$
\zadStop

\rozwStart{Maja Szabłowska}{}
$$(x^{2}+x-[p2])^{\frac{1}{2}}=\frac{1}{[p1]}x-1$$
$$x^{2}+x-[p2]=\left(\frac{1}{[p1]}x-1\right)^2$$
$$x^{2}+x-[p2]=\frac{1}{[p1k]}x^{2}-\frac{2}{[p1]}x+1$$
$$\frac{[p1k1]}{[p1k]}x^{2}+\frac{[p12]}{[p1]}x[p21]=0$$
$$\Delta=\left(\frac{[p12]}{[p1]}\right)^2 - 4\cdot\frac{[p1k1]}{[p1k]}\cdot([p21])=[del]\Rightarrow \sqrt{\Delta}=[pdel]$$
$$x_{1}=\frac{-\frac{[p12]}{[p1]}-[pdel]}{\frac{[2p1k1]}{[p1k]}}=[x1], \quad x_{2}=\frac{-\frac{[p12]}{[p1]}+[pdel]}{\frac{[2p1k1]}{[p1k]}}=[x2]$$
\rozwStop
\odpStart
$x_{1}=[x1], \quad x_{2}=[x2]$
\odpStop
\testStart
A.$x_{1}=[x1], \quad x_{2}=[x2]$\\
B.$x_{1}=[p1], \quad x_{2}=[x2]$\\
D.$x_{1}=[del], \quad x_{2}=[pdel]$\\
E.$x_{1}=[p2], \quad x_{2}=[p21]$\\
F.$x_{1}=\frac{[p1]}{[p1k1]}, \quad x_{2}=\frac{[p12]}{[p1]}$\\
G.$x_{1}=[x1], \quad x_{2}=[p1k1]$\\
H.$x_{1}=0, \quad x_{2}=1$\\
\testStop
\kluczStart
A
\kluczStop



\end{document}
