\documentclass[12pt, a4paper]{article}
\usepackage[utf8]{inputenc}
\usepackage{polski}

\usepackage{amsthm}  %pakiet do tworzenia twierdzeń itp.
\usepackage{amsmath} %pakiet do niektórych symboli matematycznych
\usepackage{amssymb} %pakiet do symboli mat., np. \nsubseteq
\usepackage{amsfonts}
\usepackage{graphicx} %obsługa plików graficznych z rozszerzeniem png, jpg
\theoremstyle{definition} %styl dla definicji
\newtheorem{zad}{} 
\title{Multizestaw zadań}
\author{Robert Fidytek}
%\date{\today}
\date{}
\newcounter{liczniksekcji}
\newcommand{\kategoria}[1]{\section{#1}} %olreślamy nazwę kateforii zadań
\newcommand{\zadStart}[1]{\begin{zad}#1\newline} %oznaczenie początku zadania
\newcommand{\zadStop}{\end{zad}}   %oznaczenie końca zadania
%Makra opcjonarne (nie muszą występować):
\newcommand{\rozwStart}[2]{\noindent \textbf{Rozwiązanie (autor #1 , recenzent #2): }\newline} %oznaczenie początku rozwiązania, opcjonarnie można wprowadzić informację o autorze rozwiązania zadania i recenzencie poprawności wykonania rozwiązania zadania
\newcommand{\rozwStop}{\newline}                                            %oznaczenie końca rozwiązania
\newcommand{\odpStart}{\noindent \textbf{Odpowiedź:}\newline}    %oznaczenie początku odpowiedzi końcowej (wypisanie wyniku)
\newcommand{\odpStop}{\newline}                                             %oznaczenie końca odpowiedzi końcowej (wypisanie wyniku)
\newcommand{\testStart}{\noindent \textbf{Test:}\newline} %ewentualne możliwe opcje odpowiedzi testowej: A. ? B. ? C. ? D. ? itd.
\newcommand{\testStop}{\newline} %koniec wprowadzania odpowiedzi testowych
\newcommand{\kluczStart}{\noindent \textbf{Test poprawna odpowiedź:}\newline} %klucz, poprawna odpowiedź pytania testowego (jedna literka): A lub B lub C lub D itd.
\newcommand{\kluczStop}{\newline} %koniec poprawnej odpowiedzi pytania testowego 
\newcommand{\wstawGrafike}[2]{\begin{figure}[h] \includegraphics[scale=#2] {#1} \end{figure}} %gdyby była potrzeba wstawienia obrazka, parametry: nazwa pliku, skala (jak nie wiesz co wpisać, to wpisz 1)

\begin{document}
\maketitle


\kategoria{Wikieł/Z1.5d}
\zadStart{Zadanie z Wikieł Z 1.5 d) moja wersja nr [nrWersji]}
%[a]:[2,3,4,5,6,7,8,9]
%[c]:[2,3,4,5,6,7,8,9]
%[e]:[2,3,4,5,6,7,8,9]
%[f]=random.randint(2,100)
%[d]=random.randint(2,10)
%[3c]=3*[c]
%[a3]=[a]*3
%[3e]=3*[e]
%[ec]=[e]*[c]
%[3ec]=[3e]*[3c]
%[w]=[a]*3*[c]
%[p3c]=pow([3c],1/2)
%([c]+3-[e])==0 and [c]!=4 and [c]!=9 and [e]!=4 and [e]!=9 and [3c]!=0 and [3c]!=1 and [p3c].is_integer()==False
Usunąć niewymierność z mianownika $\frac{[a]\sqrt{3}}{\sqrt{[c]}+\sqrt{3}+\sqrt{[e]}}$.
\zadStop
\rozwStart{Jakub Ulrych}{Pascal Nawrocki}
$$\frac{[a]\sqrt{3}}{\sqrt{[c]}+\sqrt{3}+\sqrt{[e]}}$$
$$\frac{[a]\sqrt{3}}{\sqrt{[c]}+\sqrt{3}+\sqrt{[e]}}*\frac{\sqrt{[c]}+\sqrt{3}-\sqrt{[e]}}{\sqrt{[c]}+\sqrt{3}-\sqrt{[e]}}$$
$$\frac{[a]\sqrt{[3c]}+[a3]-[a]\sqrt{[3e]}}{2\sqrt{[3c]}}$$
$$\frac{[a]\sqrt{[3c]}+[a3]-[a]\sqrt{[3e]}}{2\sqrt{[3c]}}*\frac{\sqrt{[3c]}}{\sqrt{[3c]}}$$
$$\frac{[a]*[3c]+[a3]\sqrt{[3c]}-[a]\sqrt{[3ec]}}{[3c]}$$
$$\frac{[w]+[a3]\sqrt{[3c]}-[a]\sqrt{[3ec]}}{[3c]}$$
\rozwStop
\odpStart
$$\frac{[w]+[a3]\sqrt{[3c]}-[a]\sqrt{[3ec]}}{[3c]}$$
\odpStop
\testStart
A.$\frac{[w]+[a3]\sqrt{[3c]}-[a]\sqrt{[3ec]}}{[3c]}$
B.$\frac{[w]+[3e]\sqrt{[ec]}-[a]\sqrt{[3ec]}}{[3c]}$
C.$\frac{[f]+[a3]\sqrt{[3c]}-[a]\sqrt{[f]}}{[a3]}$
D.$\frac{[3ec]+[3e]\sqrt{[3c]}-[a]\sqrt{[f]}}{[a3]}$
\testStop
\kluczStart
A
\kluczStop



\end{document}