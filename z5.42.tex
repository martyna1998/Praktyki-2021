\documentclass[12pt, a4paper]{article}
\usepackage[utf8]{inputenc}
\usepackage{polski}

\usepackage{amsthm}  %pakiet do tworzenia twierdzeń itp.
\usepackage{amsmath} %pakiet do niektórych symboli matematycznych
\usepackage{amssymb} %pakiet do symboli mat., np. \nsubseteq
\usepackage{amsfonts}
\usepackage{graphicx} %obsługa plików graficznych z rozszerzeniem png, jpg
\theoremstyle{definition} %styl dla definicji
\newtheorem{zad}{} 
\title{Multizestaw zadań}
\author{Robert Fidytek}
%\date{\today}
\date{}
\newcounter{liczniksekcji}
\newcommand{\kategoria}[1]{\section{#1}} %olreślamy nazwę kateforii zadań
\newcommand{\zadStart}[1]{\begin{zad}#1\newline} %oznaczenie początku zadania
\newcommand{\zadStop}{\end{zad}}   %oznaczenie końca zadania
%Makra opcjonarne (nie muszą występować):
\newcommand{\rozwStart}[2]{\noindent \textbf{Rozwiązanie (autor #1 , recenzent #2): }\newline} %oznaczenie początku rozwiązania, opcjonarnie można wprowadzić informację o autorze rozwiązania zadania i recenzencie poprawności wykonania rozwiązania zadania
\newcommand{\rozwStop}{\newline}                                            %oznaczenie końca rozwiązania
\newcommand{\odpStart}{\noindent \textbf{Odpowiedź:}\newline}    %oznaczenie początku odpowiedzi końcowej (wypisanie wyniku)
\newcommand{\odpStop}{\newline}                                             %oznaczenie końca odpowiedzi końcowej (wypisanie wyniku)
\newcommand{\testStart}{\noindent \textbf{Test:}\newline} %ewentualne możliwe opcje odpowiedzi testowej: A. ? B. ? C. ? D. ? itd.
\newcommand{\testStop}{\newline} %koniec wprowadzania odpowiedzi testowych
\newcommand{\kluczStart}{\noindent \textbf{Test poprawna odpowiedź:}\newline} %klucz, poprawna odpowiedź pytania testowego (jedna literka): A lub B lub C lub D itd.
\newcommand{\kluczStop}{\newline} %koniec poprawnej odpowiedzi pytania testowego 
\newcommand{\wstawGrafike}[2]{\begin{figure}[h] \centering \includegraphics[scale=#2] {#1} \end{figure}} %gdyby była potrzeba wstawienia obrazka, parametry: nazwa pliku, skala (jak nie wiesz co wpisać, to wpisz 1)

\begin{document}
\maketitle

\kategoria{Wikieł/Z5.42}

\zadStart{Zadanie z Wikieł Z 5.42 moja wersja nr [nrWersji]}
%[a]:[1,2,3,4,5,6,7,8,9,10,11]
%[b]:[2,3,4,5,6,7,8,9,10,11]
%[c]=[b]+[a]
%[d]=[b]-[a]
%[b]>[a]
Obliczyć pochodną funkcji.
$$f(x) = 
\begin{cases} 
[a] - x & \text{dla} \ -\infty < x < [a] \\
([a] - x)([b] - x) & \text{dla} \ [a] \le x \le [b] \\
-([b] - x) & \text{dla} \ [b] < x < \infty
\end{cases} 
$$
\zadStop

\rozwStart{Natalia Danieluk}{}
$$f'(x) = 
\begin{cases} 
- 1 & \text{dla} \ -\infty < x < [a] \\
-([b] - x) - ([a] - x) & \text{dla} \ [a] \le x \le [b] \\
1 & \text{dla} \ [b] < x < \infty
\end{cases} 
=
\begin{cases} 
- 1 & \text{dla} \ -\infty < x < [a] \\
2x - [c] & \text{dla} \ [a] \le x \le [b] \\
1 & \text{dla} \ [b] < x < \infty
\end{cases} 
$$
\rozwStop

\odpStart
$
\begin{cases} 
- 1 & \text{dla} \ -\infty < x < [a] \\
2x - [c] & \text{dla} \ [a] \le x \le [b] \\
1 & \text{dla} \ [b] < x < \infty
\end{cases} 
$
\odpStop

\testStart
A. $
\begin{cases} 
- 1 & \text{dla} \ -\infty < x < [a] \\
2x - [c] & \text{dla} \ [a] \le x \le [b] \\
1 & \text{dla} \ [b] < x < \infty
\end{cases} 
$\\
B. $
\begin{cases} 
1 & \text{dla} \ -\infty < x < [a] \\
2x - [c] & \text{dla} \ [a] \le x \le [b] \\
-1 & \text{dla} \ [b] < x < \infty
\end{cases} 
$\\
C. $
\begin{cases} 
-1 & \text{dla} \ -\infty < x < [a] \\
2x - [d] & \text{dla} \ [a] \le x \le [b] \\
1 & \text{dla} \ [b] < x < \infty
\end{cases} 
$\\
D. $
\begin{cases} 
- 1 & \text{dla} \ -\infty < x < [a] \\
2x + [c] & \text{dla} \ [a] \le x \le [b] \\
1 & \text{dla} \ [b] < x < \infty
\end{cases} 
$
\testStop

\kluczStart
A
\kluczStop

\end{document}