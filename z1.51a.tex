\documentclass[12pt, a4paper]{article}
\usepackage[utf8]{inputenc}
\usepackage{polski}

\usepackage{amsthm}  %pakiet do tworzenia twierdzeń itp.
\usepackage{amsmath} %pakiet do niektórych symboli matematycznych
\usepackage{amssymb} %pakiet do symboli mat., np. \nsubseteq
\usepackage{amsfonts}
\usepackage{graphicx} %obsługa plików graficznych z rozszerzeniem png, jpg
\theoremstyle{definition} %styl dla definicji
\newtheorem{zad}{} 
\title{Multizestaw zadań}
\author{Robert Fidytek}
%\date{\today}
\date{}
\newcounter{liczniksekcji}
\newcommand{\kategoria}[1]{\section{#1}} %olreślamy nazwę kateforii zadań
\newcommand{\zadStart}[1]{\begin{zad}#1\newline} %oznaczenie początku zadania
\newcommand{\zadStop}{\end{zad}}   %oznaczenie końca zadania
%Makra opcjonarne (nie muszą występować):
\newcommand{\rozwStart}[2]{\noindent \textbf{Rozwiązanie (autor #1 , recenzent #2): }\newline} %oznaczenie początku rozwiązania, opcjonarnie można wprowadzić informację o autorze rozwiązania zadania i recenzencie poprawności wykonania rozwiązania zadania
\newcommand{\rozwStop}{\newline}                                            %oznaczenie końca rozwiązania
\newcommand{\odpStart}{\noindent \textbf{Odpowiedź:}\newline}    %oznaczenie początku odpowiedzi końcowej (wypisanie wyniku)
\newcommand{\odpStop}{\newline}                                             %oznaczenie końca odpowiedzi końcowej (wypisanie wyniku)
\newcommand{\testStart}{\noindent \textbf{Test:}\newline} %ewentualne możliwe opcje odpowiedzi testowej: A. ? B. ? C. ? D. ? itd.
\newcommand{\testStop}{\newline} %koniec wprowadzania odpowiedzi testowych
\newcommand{\kluczStart}{\noindent \textbf{Test poprawna odpowiedź:}\newline} %klucz, poprawna odpowiedź pytania testowego (jedna literka): A lub B lub C lub D itd.
\newcommand{\kluczStop}{\newline} %koniec poprawnej odpowiedzi pytania testowego 
\newcommand{\wstawGrafike}[2]{\begin{figure}[h] \includegraphics[scale=#2] {#1} \end{figure}} %gdyby była potrzeba wstawienia obrazka, parametry: nazwa pliku, skala (jak nie wiesz co wpisać, to wpisz 1)

\begin{document}
\maketitle


\kategoria{Wikieł/Z1.51a}
\zadStart{Zadanie z Wikieł Z 1.51 a) moja wersja nr [nrWersji]}
%[a]:[2,3,4,5,6]
%[b]:[2,3,4,5,6]
%[c]:[2,3,4,5,6]
%[d]=random.randint(2,10)
%[e]=random.randint(2,10)
%[f]=random.randint(2,10)
%[g]=random.randint(2,10)
%[h]=random.randint(2,10)
%[aer]=[a]*[e]
%[afr]=[a]*[f]
%[agr]=[a]*[g]
%[ahr]=[a]*[h]
%[ber]=[b]*[e]
%[bfr]=[b]*[f]
%[bgr]=[b]*[g]
%[bhr]=[b]*[h]
%[cer]=[c]*[e]
%[cfr]=[c]*[f]
%[cgr]=[c]*[g]
%[chr]=[c]*[h]
%[der]=[d]*[e]
%[dfr]=[d]*[f]
%[dgr]=[d]*[g]
%[dhr]=[d]*[h]
%[x5]=[ber]-[afr]
%[x4]=[agr]-[bfr]+[cer]
%[x3]=[ahr]+[bgr]-[cfr]-[der]
%[x2]=[bhr]+[cgr]+[dfr]
%[x1]=[chr]-[dgr]
%[absx1]=abs([x1])
%[x5]>0 and [x4]>0 and [x3]>0 and [x1]<0 and [x5]!=1 and [x4]!=1 and [x3]!=1 and [x1]!=-1
Obliczyć iloczyn wielomianów $([a]x^{3}+[b]x^{2}+[c]x-[d])([e]x^{3}-[f]x^{2}+[g]x+[h])$.
\zadStop
\rozwStart{Wojciech Przybylski}{Laura Mieczkowska}
$$([a]x^{3}+[b]x^{2}+[c]x-[d])([e]x^{3}-[f]x^{2}+[g]x+[h])=$$
$$=[aer]x^{6}-[afr]x^{5}+[agr]x^{4}+[ahr]x^{3}+[ber]x^{5}-[bfr]x^{4}+[bgr]x^{3}+[bhr]x^{2}+$$
$$+[cer]x^{4}-[cfr]x^{3}+[cgr]x^{2}+[chr]x-[der]x^{3}+[dfr]x^{2}-[dgr]x-[dhr]=$$
$$=[aer]x^{6}+[x5]x^{5}+[x4]x^{4}+[x3]x^{3}+[x2]x^{2}-[absx1]x-[dhr]$$
\rozwStop
\odpStart
$[aer]x^{6}+[x5]x^{5}+[x4]x^{4}+[x3]x^{3}+[x2]x^{2}-[absx1]x-[dhr]$
\odpStop
\testStart
A. $[aer]x^{6}+[x5]x^{5}+[x4]x^{4}+[x3]x^{3}+[x2]x^{2}-[absx1]x-[dhr]$\\
B. $[aer]x^{7}+[x5]x^{5}+[x4]x^{4}+[x3]x^{3}+[x2]x^{2}-[absx1]x-[dhr]$\\
C. $[aer]x^{6}+[x4]x^{4}+[x3]x^{3}+[x2]x^{2}-[absx1]x-[dhr]$\\
D. $[aer]x^{6}+[x5]x^{5}+[x4]x^{4}+[x3]x^{3}+[x2]x^{2}-[absx1]x$\\
E. $[dhr]x^{6}+[x5]x^{5}+[x4]x^{4}+[x3]x^{3}+[x2]x^{2}-[absx1]x-[dhr]$\\
F. $[aer]x^{7}+[x5]x^{6}+[x4]x^{5}+[x3]x^{4}+[x2]x^{3}-[absx1]x^{2}-[dhr]x$
\testStop
\kluczStart
A
\kluczStop



\end{document}