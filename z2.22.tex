\documentclass[12pt, a4paper]{article}
\usepackage[utf8]{inputenc}
\usepackage{polski}

\usepackage{amsthm}  %pakiet do tworzenia twierdzeń itp.
\usepackage{amsmath} %pakiet do niektórych symboli matematycznych
\usepackage{amssymb} %pakiet do symboli mat., np. \nsubseteq
\usepackage{amsfonts}
\usepackage{graphicx} %obsługa plików graficznych z rozszerzeniem png, jpg
\theoremstyle{definition} %styl dla definicji
\newtheorem{zad}{} 
\title{Multizestaw zadań}
\author{Robert Fidytek}
%\date{\today}
\date{}
\newcounter{liczniksekcji}
\newcommand{\kategoria}[1]{\section{#1}} %olreślamy nazwę kateforii zadań
\newcommand{\zadStart}[1]{\begin{zad}#1\newline} %oznaczenie początku zadania
\newcommand{\zadStop}{\end{zad}}   %oznaczenie końca zadania
%Makra opcjonarne (nie muszą występować):
\newcommand{\rozwStart}[2]{\noindent \textbf{Rozwiązanie (autor #1 , recenzent #2): }\newline} %oznaczenie początku rozwiązania, opcjonarnie można wprowadzić informację o autorze rozwiązania zadania i recenzencie poprawności wykonania rozwiązania zadania
\newcommand{\rozwStop}{\newline}                                            %oznaczenie końca rozwiązania
\newcommand{\odpStart}{\noindent \textbf{Odpowiedź:}\newline}    %oznaczenie początku odpowiedzi końcowej (wypisanie wyniku)
\newcommand{\odpStop}{\newline}                                             %oznaczenie końca odpowiedzi końcowej (wypisanie wyniku)
\newcommand{\testStart}{\noindent \textbf{Test:}\newline} %ewentualne możliwe opcje odpowiedzi testowej: A. ? B. ? C. ? D. ? itd.
\newcommand{\testStop}{\newline} %koniec wprowadzania odpowiedzi testowych
\newcommand{\kluczStart}{\noindent \textbf{Test poprawna odpowiedź:}\newline} %klucz, poprawna odpowiedź pytania testowego (jedna literka): A lub B lub C lub D itd.
\newcommand{\kluczStop}{\newline} %koniec poprawnej odpowiedzi pytania testowego 
\newcommand{\wstawGrafike}[2]{\begin{figure}[h] \includegraphics[scale=#2] {#1} \end{figure}} %gdyby była potrzeba wstawienia obrazka, parametry: nazwa pliku, skala (jak nie wiesz co wpisać, to wpisz 1)

\begin{document}
\maketitle


\kategoria{Wikieł/Z2.22}
\zadStart{Zadanie z Wikieł Z 2.22 moja wersja nr [nrWersji]}
%[b1]:[2,3,4,5,6,7,8,9]
%[a1]:[2,3,4,5,6,7,8,9]
%[a2]=[a1]-1
%[b2]=[b1]
%[c1]=[b1]+1
%[c2]=[b1]-1
%[ab1]=[b1]-[a1]
%[ab2]=[b2]-[a2]
%[cb1]=[c1]-[b1]
%[cb2]=[b2]-[c2]
%[ca1]=[c1]-[a1]
%[ca2]=[c2]-[a2]
%[kab1]=[ab1]*[ab1]
%[kab2]=[ab2]*[ab2]
%[kcb1]=[cb1]*[cb1]
%[kcb2]=[cb2]*[cb2]
%[kac1]=[ca1]*[ca1]
%[kac2]=[ca2]*[ca2]
%[AB]=[kab1]+[kab2]
%[BC]=[kcb1]+[kcb2]
%[AC]=[kac1]+[kac2]
%[abc1]=[cb1]*[a1]
%[abc2]=[cb2]*[a2]
%[xy]=[abc1]-[abc2]
%[BC]>0 and [AC]>0 and [AB]>0 
Podać równanie osi symetrii trójkąta o wierzchołkach A([a1],[a2]), B([b1],[b2]), C([c1],[c2]).
\zadStop
\rozwStart{Aleksandra Pasińska}{}
$$|AB|=\sqrt{([b1]-[a1])^2+([b2]-[a2])^2}=\sqrt{[kab1]+[kab2]}=\sqrt{[AB]}$$
$$|BC|=\sqrt{([c1]-[b1])^2+([c2]-[b2])^2}=\sqrt{[kcb1]+[kcb2]}=\sqrt{[BC]}$$
$$|AC|=\sqrt{([c1]-[a1])^2+([c2]-[a2])^2}=\sqrt{[kac1]+[kac2]}=\sqrt{[AC]}$$
$$\overrightarrow{BC}=[[c1]-[b1],[c2]-[b2]]=[[cb1],-[cb2]],P=A$$
$$[cb1](x-[a1])-[cb2](y-[a2])=0$$
$$x-[abc1]-y+[abc2]=0$$
$$x-y-[xy]=0$$
$$y=x-[xy]$$
\rozwStop
\odpStart
$y=x-[xy]$\\
\odpStop
\testStart
A.$y=x-[xy]$
B.$[ab1]x-[xy]=0$
C.$[ab1]x-[ab2]y=7$
D.$[ab1]x-[ab2]y+[xy]=9$
E.$[ab2]y+[xy]=0$
F.$[ab1]x+[xy]=0$
G.$[ab1]x-[ab2]y=0$
H.$[ab1]x-[ab2]y+[xy]=4$
I.$[ab1]x-[ab2]y+[xy]=2$
\testStop
\kluczStart
A
\kluczStop



\end{document}