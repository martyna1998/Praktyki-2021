\documentclass[12pt, a4paper]{article}
\usepackage[utf8]{inputenc}
\usepackage{polski}

\usepackage{amsthm}  %pakiet do tworzenia twierdzeń itp.
\usepackage{amsmath} %pakiet do niektórych symboli matematycznych
\usepackage{amssymb} %pakiet do symboli mat., np. \nsubseteq
\usepackage{amsfonts}
\usepackage{graphicx} %obsługa plików graficznych z rozszerzeniem png, jpg
\theoremstyle{definition} %styl dla definicji
\newtheorem{zad}{} 
\title{Multizestaw zadań}
\author{Robert Fidytek}
%\date{\today}
\date{}\documentclass[12pt, a4paper]{article}
\usepackage[utf8]{inputenc}
\usepackage{polski}

\usepackage{amsthm}  %pakiet do tworzenia twierdzeń itp.
\usepackage{amsmath} %pakiet do niektórych symboli matematycznych
\usepackage{amssymb} %pakiet do symboli mat., np. \nsubseteq
\usepackage{amsfonts}
\usepackage{graphicx} %obsługa plików graficznych z rozszerzeniem png, jpg
\theoremstyle{definition} %styl dla definicji
\newtheorem{zad}{} 
\title{Multizestaw zadań}
\author{Robert Fidytek}
%\date{\today}
\date{}
\newcounter{liczniksekcji}
\newcommand{\kategoria}[1]{\section{#1}} %olreślamy nazwę kateforii zadań
\newcommand{\zadStart}[1]{\begin{zad}#1\newline} %oznaczenie początku zadania
\newcommand{\zadStop}{\end{zad}}   %oznaczenie końca zadania
%Makra opcjonarne (nie muszą występować):
\newcommand{\rozwStart}[2]{\noindent \textbf{Rozwiązanie (autor #1 , recenzent #2): }\newline} %oznaczenie początku rozwiązania, opcjonarnie można wprowadzić informację o autorze rozwiązania zadania i recenzencie poprawności wykonania rozwiązania zadania
\newcommand{\rozwStop}{\newline}                                            %oznaczenie końca rozwiązania
\newcommand{\odpStart}{\noindent \textbf{Odpowiedź:}\newline}    %oznaczenie początku odpowiedzi końcowej (wypisanie wyniku)
\newcommand{\odpStop}{\newline}                                             %oznaczenie końca odpowiedzi końcowej (wypisanie wyniku)
\newcommand{\testStart}{\noindent \textbf{Test:}\newline} %ewentualne możliwe opcje odpowiedzi testowej: A. ? B. ? C. ? D. ? itd.
\newcommand{\testStop}{\newline} %koniec wprowadzania odpowiedzi testowych
\newcommand{\kluczStart}{\noindent \textbf{Test poprawna odpowiedź:}\newline} %klucz, poprawna odpowiedź pytania testowego (jedna literka): A lub B lub C lub D itd.
\newcommand{\kluczStop}{\newline} %koniec poprawnej odpowiedzi pytania testowego 
\newcommand{\wstawGrafike}[2]{\begin{figure}[h] \includegraphics[scale=#2] {#1} \end{figure}} %gdyby była potrzeba wstawienia obrazka, parametry: nazwa pliku, skala (jak nie wiesz co wpisać, to wpisz 1)

\begin{document}
\maketitle


\kategoria{Wikieł/Z2.14}
\zadStart{Zadanie z Wikieł Z 2.14  moja wersja nr [nrWersji]}
%[p1]=random.randint(2,10)
%[p2]:[2,3,4,5,6,7,8,9,10]
%[p3]:[2,3,4,5,6,7,8,9,10]
%[p4]:[2,3,4,5,6,7,8,9,10]
%[p5]=random.randint(-10,-1)
%[p6]=random.randint(2,10)
%[p3p6]=[p3]*[p6]
%[p5p4]=[p5]*[p4]
%[w]=[p3p6]-[p5p4]
%[p1p6]=[p1]*[p6]
%[p5p2]=[p5]*[p2]
%[wt]=[p1p6]-[p5p2]
%[p3p2]=[p3]*[p2]
%[p1p4]=[p1]*[p4]
%[ws]=[p3p2]-[p1p4]
%[w]!=0 and math.gcd([ws],[w])==1 and math.gcd([wt],[w])==1
Kombinacją liniową wektorów $\vec{u}$ i $\vec{v}$ nazywamy wektor $\vec{w}=t\cdot\vec{u}+s\cdot\vec{v},$ gdzie $t$ i $s$ są dowolnymi liczbami. Przedstawić wektor $\vec{a}=[[p1], [p2]]$ jako kombinację liniową dwóch wektorów $\vec{e_{1}}=[[p3],[p4]]$ oraz $\vec{e_{2}}=[[p5],[p6]].$
\zadStop

\rozwStart{Maja Szabłowska}{}
$$
\left\{ \begin{array}{ll}
[p1]= t\cdot [p3] + s \cdot ([p5])\\
[p2]= t \cdot [p4] + s \cdot [p6]
\end{array} \right.
$$

$$
W=\left| \begin{array}{ccc}
[p3] & [p5] \\
[p4] & [p6] 
\end{array} \right| = [p3]\cdot[p6]-([p5])\cdot[p4]=[p3p6]-([p5p4])=[w]
$$

$$
W_{t}=\left| \begin{array}{ccc}
[p1] & [p5] \\
[p2] & [p6] 
\end{array} \right| =[p1]\cdot[p6]-([p5])\cdot[p2]=[wt]
$$

$$
W_{s}=\left| \begin{array}{ccc}
[p3] & [p1] \\
[p4] & [p2] 
\end{array} \right| =[p3]\cdot[p2]-[p1]\cdot[p4]=[ws]
$$

$$t=\frac{W_{t}}{W}=\frac{[wt]}{[w]}$$
$$s=\frac{W_{s}}{W}=\frac{[ws]}{[w]}$$
$$\vec{a}=\frac{[wt]}{[w]}\cdot\vec{u}+\frac{[ws]}{[w]}\cdot\vec{v}$$

\rozwStop


\odpStart
$t=\frac{[wt]}{[w]}, s=\frac{[ws]}{[w]}$
\odpStop
\testStart
A.$t=\frac{[wt]}{[w]}, s=\frac{[ws]}{[w]}$
B.$t=\frac{[ws]}{[w]}, s=0$
D.$t=\frac{[wt]}{[p1]}, s=\frac{[ws]}{[p2]}$
E.$t=2, s=3$
F.$t=[p3p2], s=\frac{[ws]}{[w]}$
G.$t=[p1p6], s=[p1p4]$
H.$t=[p5], s=\frac{[ws]}{[w]}$

\testStop
\kluczStart
A
\kluczStop



\end{document}
