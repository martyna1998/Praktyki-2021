\documentclass[12pt, a4paper]{article}
\usepackage[utf8]{inputenc}
\usepackage{polski}
\usepackage{amsthm}  %pakiet do tworzenia twierdzeń itp.
\usepackage{amsmath} %pakiet do niektórych symboli matematycznych
\usepackage{amssymb} %pakiet do symboli mat., np. \nsubseteq
\usepackage{amsfonts}
\usepackage{graphicx} %obsługa plików graficznych z rozszerzeniem png, jpg
\theoremstyle{definition} %styl dla definicji
\newtheorem{zad}{} 
\title{Multizestaw zadań}
\author{Robert Fidytek}
%\date{\today}
\date{}
\newcounter{liczniksekcji}
\newcommand{\kategoria}[1]{\section{#1}} %olreślamy nazwę kateforii zadań
\newcommand{\zadStart}[1]{\begin{zad}#1\newline} %oznaczenie początku zadania
\newcommand{\zadStop}{\end{zad}}   %oznaczenie końca zadania
%Makra opcjonarne (nie muszą występować):
\newcommand{\rozwStart}[2]{\noindent \textbf{Rozwiązanie (autor #1 , recenzent #2): }\newline} %oznaczenie początku rozwiązania, opcjonarnie można wprowadzić informację o autorze rozwiązania zadania i recenzencie poprawności wykonania rozwiązania zadania
\newcommand{\rozwStop}{\newline}                                            %oznaczenie końca rozwiązania
\newcommand{\odpStart}{\noindent \textbf{Odpowiedź:}\newline}    %oznaczenie początku odpowiedzi końcowej (wypisanie wyniku)
\newcommand{\odpStop}{\newline}                                             %oznaczenie końca odpowiedzi końcowej (wypisanie wyniku)
\newcommand{\testStart}{\noindent \textbf{Test:}\newline} %ewentualne możliwe opcje odpowiedzi testowej: A. ? B. ? C. ? D. ? itd.
\newcommand{\testStop}{\newline} %koniec wprowadzania odpowiedzi testowych
\newcommand{\kluczStart}{\noindent \textbf{Test poprawna odpowiedź:}\newline} %klucz, poprawna odpowiedź pytania testowego (jedna literka): A lub B lub C lub D itd.
\newcommand{\kluczStop}{\newline} %koniec poprawnej odpowiedzi pytania testowego 
\newcommand{\wstawGrafike}[2]{\begin{figure}[h] \includegraphics[scale=#2] {#1} \end{figure}} %gdyby była potrzeba wstawienia obrazka, parametry: nazwa pliku, skala (jak nie wiesz co wpisać, to wpisz 1)

\begin{document}
\maketitle


\kategoria{Wikieł/Z1.98a}
\zadStart{Zadanie z Wikieł Z1.98 a) moja wersja nr [nrWersji]}
%[a]:[1,2,3,4,5,6,7,8,9]
%[p]=int((1/2)**(-[a]))
%[o]=12-[p]
%[m]=int(4*1*[o])
%[o2]=49-[m]
%[pi]=int(math.sqrt([o2]))
%[o3]=7-[pi]
%[d]=int([o3]/2)
%[do]=7+[pi]
%[d2]=int([do]/2)
Wyznaczyć dziedzinę funkcji $ f_{(x)} = \log_{7}[\log_{\frac{1}{2}} (x^{2} - 7x + 12) + [a]] $\\
\zadStop
\rozwStart{Martyna Czarnobaj}{}
1)
\begin{center}
	$ \log_{\frac{1}{2}} (x^{2} - 7x + 12) + [a] > 0 $\\
	$ \log_{\frac{1}{2}} (x^{2} - 7x + 12) > -[a] $\\
	$ \log^{a} b = c \Leftrightarrow a^{c} = b $\\
	$ a = \frac{1}{2}, c = -[a], b = (\frac{1}{2})^{-[a]} = [p] $\\
	$ \log_{\frac{1}{2}} (x^{2} - 7x + 12) > \log_{\frac{1}{2}} [p], a < 1 $\\
	$ x^{2} - 7x + 12 < [p] $\\
	$ x^{2} - 7x + [o] < 0 $\\
	$ \Delta = 49 - [m] = [o2], \sqrt{\Delta}=[pi] $\\
	$ x_{1} = \frac{7 - [pi]}{2} = \frac{[o3]}{2} = [d] $\\
	$ x_{2} = \frac{7 + [pi]}{2} = \frac{[do]}{2} = [d2]  $\\
	$ x \in ([d],[d2]) $\\
\end{center}
2)
\begin{center}
	$ x^{2} - 7x + 12 > 0 $\\
	$ \Delta = 49 - 48 = 1 $\\
	$ x_{1} = \frac{7 - 1}{2} = 3 $\\
	$ x_{2} = \frac{7 + 1}{2} = 4 $\\
	$ x \in (- \infty, 3) \vee (4, \infty +) $\\
	
\end{center}
Wynik: $ x \in ([d],3) \vee (4,[d2]) $\\
Koniec rozwiązania.\\
\rozwStop
\odpStart
$ x \in ([d],3) \vee (4,[d2]) $\\
\odpStop
\testStart
A.$ x \in ([d],3) \vee (4,[d2]) $\\
B$ x \in ([d],3) \vee (-4,[d2]) $\\
C.$ x \in ([d],-3) \vee (4,[d2]) $\\
\testStop
\kluczStart
A
\kluczStop
\end{document}