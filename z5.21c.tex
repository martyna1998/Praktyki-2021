\documentclass[12pt, a4paper]{article}
\usepackage[utf8]{inputenc}
\usepackage{polski}

\usepackage{amsthm}  %pakiet do tworzenia twierdzeń itp.
\usepackage{amsmath} %pakiet do niektórych symboli matematycznych
\usepackage{amssymb} %pakiet do symboli mat., np. \nsubseteq
\usepackage{amsfonts}
\usepackage{graphicx} %obsługa plików graficznych z rozszerzeniem png, jpg
\theoremstyle{definition} %styl dla definicji
\newtheorem{zad}{} 
\title{Multizestaw zadań}
\author{Laura Mieczkowska}
%\date{\today}
\date{}
\newcounter{liczniksekcji}
\newcommand{\kategoria}[1]{\section{#1}} %olreślamy nazwę kateforii zadań
\newcommand{\zadStart}[1]{\begin{zad}#1\newline} %oznaczenie początku zadania
\newcommand{\zadStop}{\end{zad}}   %oznaczenie końca zadania
%Makra opcjonarne (nie muszą występować):
\newcommand{\rozwStart}[2]{\noindent \textbf{Rozwiązanie (autor #1 , recenzent #2): }\newline} %oznaczenie początku rozwiązania, opcjonarnie można wprowadzić informację o autorze rozwiązania zadania i recenzencie poprawności wykonania rozwiązania zadania
\newcommand{\rozwStop}{\newline}                                            %oznaczenie końca rozwiązania
\newcommand{\odpStart}{\noindent \textbf{Odpowiedź:}\newline}    %oznaczenie początku odpowiedzi końcowej (wypisanie wyniku)
\newcommand{\odpStop}{\newline}                                             %oznaczenie końca odpowiedzi końcowej (wypisanie wyniku)
\newcommand{\testStart}{\noindent \textbf{Test:}\newline} %ewentualne możliwe opcje odpowiedzi testowej: A. ? B. ? C. ? D. ? itd.
\newcommand{\testStop}{\newline} %koniec wprowadzania odpowiedzi testowych
\newcommand{\kluczStart}{\noindent \textbf{Test poprawna odpowiedź:}\newline} %klucz, poprawna odpowiedź pytania testowego (jedna literka): A lub B lub C lub D itd.
\newcommand{\kluczStop}{\newline} %koniec poprawnej odpowiedzi pytania testowego 
\newcommand{\wstawGrafike}[2]{\begin{figure}[h] \includegraphics[scale=#2] {#1} \end{figure}} %gdyby była potrzeba wstawienia obrazka, parametry: nazwa pliku, skala (jak nie wiesz co wpisać, to wpisz 1)

\begin{document}
\maketitle


\kategoria{Wikieł/Z5.21c}
\zadStart{Zadanie z Wikieł Z 5.21 c) moja wersja nr [nrWersji]}
%[b]:[2,3,4,5,6,7,8,9,10,11,12,13,14,15]
%[a]:[2,3,4,5,6,7,8,9,10]
%[c]=[a]*3
%[m]=2*[c]
%[delta]=4*[c]*[b]
%[pierw1]=math.sqrt([delta])
%[pierw]=int([pierw1])
%[ulamek1]=[pierw]/[m]
%[ulamek]=int([ulamek1])
%[x]=[ulamek]+1
%[xkw]=[x]**2
%[p1]=[c]*[xkw]-[b]
%[pierw1].is_integer()==True and [ulamek1].is_integer()==True
Wyznaczyć przedziały monotoniczności funkcji $f(x)=[a]x^3-[b]x$.
\zadStop
\rozwStart{Laura Mieczkowska}{}
$$f(x)=[a]x^3-[b]x$$
Dziedziną funkcji $f$ jest zbiór liczb rzeczywistych tj. $\mathnormal{D_f}=\mathbb{R}.$
\\\\Obliczamy pochodną funkcji
$$f'(x)=[c]x^2-[b]$$
$$\Delta=0-4\cdot[c]\cdot(-[b])=[delta]\Rightarrow \sqrt{\Delta}=[pierw]$$
$$x_1=\frac{-[pierw]}{[m]} \vee x_2=\frac{[pierw]}{[m]}$$
$$x_1=-[ulamek] \vee x_2=[ulamek]$$
Otrzymujemy trzy przedziały monotoniczności $(-\infty;-[ulamek]), (-[ulamek],[ulamek])$ oraz $([ulamek],\infty)$, w których następnie należy sprawdzić znak funkcji.
\\\\$(-\infty;-[ulamek])$
$$f'(-[x])=[c]\cdot(-[x])^2-[b]=[p1]$$
$(-[ulamek],[ulamek])$
$$f'(0)=0-[b]=-[b]$$
$([ulamek],\infty)$
$$f'([x])=[c]\cdot[x]^2-[b]=[p1]$$
Więc funkcja $f(x)=[a]x^3-[b]x$ jest rosnąca na przedziale $(-\infty;-[ulamek])$ i $([ulamek],\infty)$ oraz malejąca na przedziale $(-[ulamek],[ulamek])$.

\odpStart
$\nearrow$ w $(-\infty;-[ulamek])$ i $([ulamek],\infty)$,$\searrow$ w $(-[ulamek],[ulamek])$
\odpStop
\testStart
A. $\nearrow$ w $(-\infty;-[ulamek])$,$\searrow$ w $(-[ulamek],[ulamek])$ i $([ulamek],\infty)$\\
B. $\searrow$ w $(-\infty;-[ulamek])$,$\nearrow$ w $(-[ulamek],[ulamek])$ i $([ulamek],\infty)$ \\
C. $\nearrow$ w $(-\infty;-[ulamek])$ i $([ulamek],\infty)$,$\searrow$ w $(-[ulamek],[ulamek])$ \\
D. $\searrow$ w $(-\infty;-[ulamek])$ i $([ulamek],\infty)$,$\nearrow$ w $(-[ulamek],[ulamek])$ 
\testStop
\kluczStart
C
\kluczStop



\end{document}