\documentclass[12pt, a4paper]{article}
\usepackage[utf8]{inputenc}
\usepackage{polski}
\usepackage{amsthm}  %pakiet do tworzenia twierdzeń itp.
\usepackage{amsmath} %pakiet do niektórych symboli matematycznych
\usepackage{amssymb} %pakiet do symboli mat., np. \nsubseteq
\usepackage{amsfonts}
\usepackage{graphicx} %obsługa plików graficznych z rozszerzeniem png, jpg
\theoremstyle{definition} %styl dla definicji
\newtheorem{zad}{} 
\title{Multizestaw zadań}
\author{Robert Fidytek}
%\date{\today}
\date{}
\newcounter{liczniksekcji}
\newcommand{\kategoria}[1]{\section{#1}} %olreślamy nazwę kateforii zadań
\newcommand{\zadStart}[1]{\begin{zad}#1\newline} %oznaczenie początku zadania
\newcommand{\zadStop}{\end{zad}}   %oznaczenie końca zadania
%Makra opcjonarne (nie muszą występować):
\newcommand{\rozwStart}[2]{\noindent \textbf{Rozwiązanie (autor #1 , recenzent #2): }\newline} %oznaczenie początku rozwiązania, opcjonarnie można wprowadzić informację o autorze rozwiązania zadania i recenzencie poprawności wykonania rozwiązania zadania
\newcommand{\rozwStop}{\newline}                                            %oznaczenie końca rozwiązania
\newcommand{\odpStart}{\noindent \textbf{Odpowiedź:}\newline}    %oznaczenie początku odpowiedzi końcowej (wypisanie wyniku)
\newcommand{\odpStop}{\newline}                                             %oznaczenie końca odpowiedzi końcowej (wypisanie wyniku)
\newcommand{\testStart}{\noindent \textbf{Test:}\newline} %ewentualne możliwe opcje odpowiedzi testowej: A. ? B. ? C. ? D. ? itd.
\newcommand{\testStop}{\newline} %koniec wprowadzania odpowiedzi testowych
\newcommand{\kluczStart}{\noindent \textbf{Test poprawna odpowiedź:}\newline} %klucz, poprawna odpowiedź pytania testowego (jedna literka): A lub B lub C lub D itd.
\newcommand{\kluczStop}{\newline} %koniec poprawnej odpowiedzi pytania testowego 
\newcommand{\wstawGrafike}[2]{\begin{figure}[h] \includegraphics[scale=#2] {#1} \end{figure}} %gdyby była potrzeba wstawienia obrazka, parametry: nazwa pliku, skala (jak nie wiesz co wpisać, to wpisz 1)

\begin{document}
\maketitle


\kategoria{Wikieł/Z3.7a}
\zadStart{Zadanie z Wikieł Z3.7 a) moja wersja nr [nrWersji]}
%[a]:[1,2,3,4,5,6,7,8,9,10]
%[d1]=[a]+[a]
%[d2]=[a]+[d1]
%[d3]=[d1]+[d2]
Wypisać pięć początkowych wyrazów ciągu okreslonego rekurencyjnie.\\
$ a_{1} = a_{2} = [a] $ i $ a_{n} = a_{n-2} + a_{n-1} $ dla $ n \geq 3 $.\\
\zadStop
\rozwStart{Martyna Czarnobaj}{}
\begin{center}
	$ a_{1} = [a] $\\
	$ a_{2} = [a] $\\
	$ a_{3} = a_{1} + a_{2} = [d1] $\\
	$ a_{4} = a_{2} + a_{3} = [d2] $\\
	$ a_{5} = a_{3} + a_{4} = [d3] $\\
\end{center}

Koniec rozwiązania.\\
\rozwStop
\odpStart
$ a_{1} = [a] $\\
$ a_{2} = [a] $\\
$ a_{3} = a_{1} + a_{2} = [d1] $\\
$ a_{4} = a_{2} + a_{3} = [d2] $\\
$ a_{5} = a_{3} + a_{4} = [d3] $\\
\odpStop
\testStart
A.$ a_{1} = [a] $\\
$ a_{2} = [a] $\\
$ a_{3} = a_{1} + a_{2} = [d1] $\\
$ a_{4} = a_{2} + a_{3} = [d2] $\\
$ a_{5} = a_{3} + a_{4} = [d3] $\\
B$ a_{1} = [a] $\\
$ a_{2} = [a] $\\
$ a_{3} = a_{1} + a_{2} = [a] $\\
$ a_{4} = a_{2} + a_{3} = [d2] $\\
$ a_{5} = a_{3} + a_{4} = [d3] $\\
C.$ a_{1} = [a] $\\
$ a_{2} = [a] $\\
$ a_{3} = a_{1} + a_{2} = [d2] $\\
$ a_{4} = a_{2} + a_{3} = [d1] $\\
$ a_{5} = a_{3} + a_{4} = [d3] $\\
\testStop
\kluczStart
A
\kluczStop
\end{document}