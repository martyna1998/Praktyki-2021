\documentclass[12pt, a4paper]{article}
\usepackage[utf8]{inputenc}
\usepackage{polski}

\usepackage{amsthm}  %pakiet do tworzenia twierdzeń itp.
\usepackage{amsmath} %pakiet do niektórych symboli matematycznych
\usepackage{amssymb} %pakiet do symboli mat., np. \nsubseteq
\usepackage{amsfonts}
\usepackage{graphicx} %obsługa plików graficznych z rozszerzeniem png, jpg
\theoremstyle{definition} %styl dla definicji
\newtheorem{zad}{} 
\title{Multizestaw zadań}
\author{Laura Mieczkowska}
%\date{\today}
\date{}
\newcounter{liczniksekcji}
\newcommand{\kategoria}[1]{\section{#1}} %olreślamy nazwę kateforii zadań
\newcommand{\zadStart}[1]{\begin{zad}#1\newline} %oznaczenie początku zadania
\newcommand{\zadStop}{\end{zad}}   %oznaczenie końca zadania
%Makra opcjonarne (nie muszą występować):
\newcommand{\rozwStart}[2]{\noindent \textbf{Rozwiązanie (autor #1 , recenzent #2): }\newline} %oznaczenie początku rozwiązania, opcjonarnie można wprowadzić informację o autorze rozwiązania zadania i recenzencie poprawności wykonania rozwiązania zadania
\newcommand{\rozwStop}{\newline}                                            %oznaczenie końca rozwiązania
\newcommand{\odpStart}{\noindent \textbf{Odpowiedź:}\newline}    %oznaczenie początku odpowiedzi końcowej (wypisanie wyniku)
\newcommand{\odpStop}{\newline}                                             %oznaczenie końca odpowiedzi końcowej (wypisanie wyniku)
\newcommand{\testStart}{\noindent \textbf{Test:}\newline} %ewentualne możliwe opcje odpowiedzi testowej: A. ? B. ? C. ? D. ? itd.
\newcommand{\testStop}{\newline} %koniec wprowadzania odpowiedzi testowych
\newcommand{\kluczStart}{\noindent \textbf{Test poprawna odpowiedź:}\newline} %klucz, poprawna odpowiedź pytania testowego (jedna literka): A lub B lub C lub D itd.
\newcommand{\kluczStop}{\newline} %koniec poprawnej odpowiedzi pytania testowego 
\newcommand{\wstawGrafike}[2]{\begin{figure}[h] \includegraphics[scale=#2] {#1} \end{figure}} %gdyby była potrzeba wstawienia obrazka, parametry: nazwa pliku, skala (jak nie wiesz co wpisać, to wpisz 1)

\begin{document}
\maketitle


\kategoria{Wikieł/Z1.14k}
\zadStart{Zadanie z Wikieł Z 1.14 k) moja wersja nr [nrWersji]}
%[a]:[2,3,4,5,6,7,8,9,10]
%[b]:[2,3,4,5,6,7,8,9,10]
%[c]:[3,4,5,6,7,8,9,10]
%[d]=[c]*[b]
%[e]=[c]+1
%[f]=1-[c]
%[g]=[d]-[a]
%[h]=-[d]-[a]
%[hb]=abs([h])
%[fb]=abs([f])
%[h]>0
%[c]>0
%math.gcd([e],[g])==1 and math.gcd([f],[h])==1
Rozwiązać równanie $\big|\frac{x+[a]}{[b]-x}\big|=[c]$.
\zadStop
\rozwStart{Laura Mieczkowska}{}
$$\bigg|\frac{x+[a]}{[b]-x}\bigg|=[c]$$ 
$$\frac{x+[a]}{[b]-x}=[c] \vee \frac{x+[a]}{[b]-x}=-[c]$$
$$x+[a]=[d]-[c]x \vee x+[a]=-[d]+[c]x$$
$$[e]x=[g] \vee [f]x=[h]$$
$$x=\frac{[g]}{[e]} \vee x=\frac{[hb]}{[fb]}$$

\odpStart
$x=\frac{[g]}{[e]} \vee x=\frac{[hb]}{[fb]}$
\odpStop
\testStart
A. $x=\frac{[e]}{[g]} \vee x=\frac{[hb]}{[fb]}$ \\
B. $x=\frac{[g]}{[e]} \vee x=-\frac{[hb]}{[fb]}$ \\
C. $x=\frac{[g]}{[e]} \vee x=\frac{[hb]}{[fb]}$ \\
D. $x=-\frac{[g]}{[e]} \vee x=\frac{[hb]}{[fb]}$ 
\testStop
\kluczStart
C
\kluczStop



\end{document}