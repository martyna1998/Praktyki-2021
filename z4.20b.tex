\documentclass[12pt, a4paper]{article}
\usepackage[utf8]{inputenc}
\usepackage{polski}

\usepackage{amsthm}  %pakiet do tworzenia twierdzeń itp.
\usepackage{amsmath} %pakiet do niektórych symboli matematycznych
\usepackage{amssymb} %pakiet do symboli mat., np. \nsubseteq
\usepackage{amsfonts}
\usepackage{graphicx} %obsługa plików graficznych z rozszerzeniem png, jpg
\theoremstyle{definition} %styl dla definicji
\newtheorem{zad}{} 
\title{Multizestaw zadań}
\author{Robert Fidytek}
%\date{\today}
\date{}
\newcounter{liczniksekcji}
\newcommand{\kategoria}[1]{\section{#1}} %olreślamy nazwę kateforii zadań
\newcommand{\zadStart}[1]{\begin{zad}#1\newline} %oznaczenie początku zadania
\newcommand{\zadStop}{\end{zad}}   %oznaczenie końca zadania
%Makra opcjonarne (nie muszą występować):
\newcommand{\rozwStart}[2]{\noindent \textbf{Rozwiązanie (autor #1 , recenzent #2): }\newline} %oznaczenie początku rozwiązania, opcjonarnie można wprowadzić informację o autorze rozwiązania zadania i recenzencie poprawności wykonania rozwiązania zadania
\newcommand{\rozwStop}{\newline}                                            %oznaczenie końca rozwiązania
\newcommand{\odpStart}{\noindent \textbf{Odpowiedź:}\newline}    %oznaczenie początku odpowiedzi końcowej (wypisanie wyniku)
\newcommand{\odpStop}{\newline}                                             %oznaczenie końca odpowiedzi końcowej (wypisanie wyniku)
\newcommand{\testStart}{\noindent \textbf{Test:}\newline} %ewentualne możliwe opcje odpowiedzi testowej: A. ? B. ? C. ? D. ? itd.
\newcommand{\testStop}{\newline} %koniec wprowadzania odpowiedzi testowych
\newcommand{\kluczStart}{\noindent \textbf{Test poprawna odpowiedź:}\newline} %klucz, poprawna odpowiedź pytania testowego (jedna literka): A lub B lub C lub D itd.
\newcommand{\kluczStop}{\newline} %koniec poprawnej odpowiedzi pytania testowego 
\newcommand{\wstawGrafike}[2]{\begin{figure}[h] \includegraphics[scale=#2] {#1} \end{figure}} %gdyby była potrzeba wstawienia obrazka, parametry: nazwa pliku, skala (jak nie wiesz co wpisać, to wpisz 1)

\begin{document}
\maketitle


\kategoria{Wikieł/Z4.20b}
\zadStart{Zadanie z Wikieł Z 4.20b) moja wersja nr [nrWersji]}
%[b]:[1,2,3,4,5,6]
%[c]:[1,2,3,4,5,6]
%[xo]:[1,2,3,4,5,6]
%[d]=[xo]+[c]
%[dd]=[d]*[d]
%[e]=[b]-[dd]
%[f]=[e]-[xo]
%[fp]=-[f]
%[dp]=-[d]
%[b]<[dd]
Wyznaczyć wartości parametru tak, aby funkcja $
f(x) = \left\{ \begin{array}{ll}
[b]-(x+[c])^{2} & \textrm{dla $x<[xo]$}\\
x+a & \textrm{dla $x\geq [xo]$}
\end{array} \right.
$ była ciągła.
\zadStop
\rozwStart{Justyna Chojecka}{}
Dla dowolnej wartości parametru $a$ funkcja jest ciągła w przedziałach $(-\infty,[xo])$ i $([xo],\infty)$. Jedynym punktem, w którym funkcja mogłaby być nieciągła, jest punkt $x_{0}=[xo]$.\\
Wartość funkcji w tym punkcie jest równa $f([xo])=[xo]+a$.\\
Granice jednostronne są równe odpowiednio:
$$\lim\limits_{x\to [xo]^{-}}f(x)=\lim\limits_{x\to [xo]^{-}}[b]-(x+[c])^{2}=[b]-([xo]+[c])^{2}=[e],$$
$$\lim\limits_{x\to [xo]^{+}}f(x)=\lim\limits_{x\to [xo]^{+}}x+a=[xo]+a.$$
A zatem
$$\lim\limits_{x\to [xo]^{-}}f(x)=\lim\limits_{x\to [xo]^{+}}f(x)\iff [e]=[xo]+a \iff a=[e]-[xo] \iff a=[f].$$
Stąd dla wartości parametru $a=[f]$ funkcja jest ciągła dla wszystkich $x\in\mathbb{R}.$
\rozwStop
\odpStart
$a=[f]$
\odpStop
\testStart
A.$a=[f]$
B.$a=[e]$
C.$a=[d]$
D.$a=[fp]$
E.$a\in\{[f],[fp]\}$
F.$a=-[c]$
G.$a\in\{[dp],[d]\}$
H.$a=-[d]$
I.$a=[c]$
\testStop
\kluczStart
A
\kluczStop



\end{document}