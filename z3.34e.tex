\documentclass[12pt, a4paper]{article}
\usepackage[utf8]{inputenc}
\usepackage{polski}

\usepackage{amsthm}  %pakiet do tworzenia twierdzeń itp.
\usepackage{amsmath} %pakiet do niektórych symboli matematycznych
\usepackage{amssymb} %pakiet do symboli mat., np. \nsubseteq
\usepackage{amsfonts}
\usepackage{graphicx} %obsługa plików graficznych z rozszerzeniem png, jpg
\theoremstyle{definition} %styl dla definicji
\newtheorem{zad}{} 
\title{Multizestaw zadań}
\author{Robert Fidytek}
%\date{\today}
\date{}
\newcounter{liczniksekcji}
\newcommand{\kategoria}[1]{\section{#1}} %olreślamy nazwę kateforii zadań
\newcommand{\zadStart}[1]{\begin{zad}#1\newline} %oznaczenie początku zadania
\newcommand{\zadStop}{\end{zad}}   %oznaczenie końca zadania
%Makra opcjonarne (nie muszą występować):
\newcommand{\rozwStart}[2]{\noindent \textbf{Rozwiązanie (autor #1 , recenzent #2): }\newline} %oznaczenie początku rozwiązania, opcjonarnie można wprowadzić informację o autorze rozwiązania zadania i recenzencie poprawności wykonania rozwiązania zadania
\newcommand{\rozwStop}{\newline}                                            %oznaczenie końca rozwiązania
\newcommand{\odpStart}{\noindent \textbf{Odpowiedź:}\newline}    %oznaczenie początku odpowiedzi końcowej (wypisanie wyniku)
\newcommand{\odpStop}{\newline}                                             %oznaczenie końca odpowiedzi końcowej (wypisanie wyniku)
\newcommand{\testStart}{\noindent \textbf{Test:}\newline} %ewentualne możliwe opcje odpowiedzi testowej: A. ? B. ? C. ? D. ? itd.
\newcommand{\testStop}{\newline} %koniec wprowadzania odpowiedzi testowych
\newcommand{\kluczStart}{\noindent \textbf{Test poprawna odpowiedź:}\newline} %klucz, poprawna odpowiedź pytania testowego (jedna literka): A lub B lub C lub D itd.
\newcommand{\kluczStop}{\newline} %koniec poprawnej odpowiedzi pytania testowego 
\newcommand{\wstawGrafike}[2]{\begin{figure}[h] \includegraphics[scale=#2] {#1} \end{figure}} %gdyby była potrzeba wstawienia obrazka, parametry: nazwa pliku, skala (jak nie wiesz co wpisać, to wpisz 1)

\begin{document}
\maketitle


\kategoria{Wikieł/Z3.34e}
\zadStart{Zadanie z Wikieł Z 3.34 e) moja wersja nr [nrWersji]}
%[a]:[1,2,3,4,5,6,7,8,9,10,11,12,13,14,15,16,17,18,19,20,21,22,23,24,25,26,27,28,29,30,31,32,33,34,35,36,37,38,39,40,41,42,43,44]
%math.gcd([a],3)==1
Obliczyć granicę ciągu. $$a_n=\bigg(\sqrt[3]{1-\frac{[a]}{n}}-1\bigg)n$$
\zadStop
\rozwStart{Aleksandra Pasińska}{}
$$\lim_{n\rightarrow \infty}\bigg(\sqrt[3]{1-\frac{[a]}{n}}-1\bigg)n=$$
$$=\lim_{n\rightarrow \infty}\bigg(\frac{(\sqrt[3]{1-\frac{[a]}{n}}-1)n(1+\sqrt[3]{1-\frac{[a]}{n}}+(1-\frac{[a]}{n})^{\frac{2}{3}})}{1+\sqrt[3]{1-\frac{[a]}{n}}+(1-\frac{[a]}{n})^{\frac{2}{3}}}\bigg)=$$
$$=\lim_{n\rightarrow \infty}\bigg(\frac{\sqrt[3]{1-\frac{[a]}{n}}n+(1-\frac{[a]}{n})^\frac{2}{3}n+(1-\frac{[a]}{n})n-n-\sqrt[3]{1-\frac{[a]}{n}}n-(1-\frac{[a]}{n})^\frac{2}{3}n}{1+\sqrt[3]{1-\frac{[a]}{n}}+(1-\frac{[a]}{n})^{\frac{2}{3}}}\bigg)=$$
$$=\lim_{n\rightarrow \infty}\bigg(\frac{(1-\frac{[a]}{n})n-n}{1+\sqrt[3]{1-\frac{[a]}{n}}+(1-\frac{[a]}{n})^{\frac{2}{3}}}\bigg)=$$
$$=\frac{\lim_{n\rightarrow \infty}((1-\frac{[a]}{n})n-n)}{\lim_{n\rightarrow \infty}(1+\sqrt[3]{1-\frac{[a]}{n}}+(1-\frac{[a]}{n})^{\frac{2}{3}})}=\frac{\lim_{n\rightarrow \infty}(n-[a]-n)}{\lim_{n\rightarrow \infty}(1+\sqrt[3]{1-\frac{[a]}{n}}+(1-\frac{[a]}{n})^{\frac{2}{3}})}=$$
$$=\frac{-[a]}{1+\sqrt[3]{1-0}+(1-0)^{\frac{2}{3}}}=-\frac{[a]}{3}$$
\rozwStop
\odpStart
$-\frac{[a]}{3}$\\
\odpStop
\testStart
A.$-\frac{[a]}{3}$
B.$\infty$
C.$-\infty$
D.$1$
E.$9$
F.$e$
G.$4$
H.$7$
I.$-7$
\testStop
\kluczStart
A
\kluczStop



\end{document}