\documentclass[12pt, a4paper]{article}
\usepackage[utf8]{inputenc}
\usepackage{polski}

\usepackage{amsthm}  %pakiet do tworzenia twierdzeń itp.
\usepackage{amsmath} %pakiet do niektórych symboli matematycznych
\usepackage{amssymb} %pakiet do symboli mat., np. \nsubseteq
\usepackage{amsfonts}
\usepackage{graphicx} %obsługa plików graficznych z rozszerzeniem png, jpg
\theoremstyle{definition} %styl dla definicji
\newtheorem{zad}{} 
\title{Multizestaw zadań}
\author{Robert Fidytek}
%\date{\today}
\date{}
\newcounter{liczniksekcji}
\newcommand{\kategoria}[1]{\section{#1}} %olreślamy nazwę kateforii zadań
\newcommand{\zadStart}[1]{\begin{zad}#1\newline} %oznaczenie początku zadania
\newcommand{\zadStop}{\end{zad}}   %oznaczenie końca zadania
%Makra opcjonarne (nie muszą występować):
\newcommand{\rozwStart}[2]{\noindent \textbf{Rozwiązanie (autor #1 , recenzent #2): }\newline} %oznaczenie początku rozwiązania, opcjonarnie można wprowadzić informację o autorze rozwiązania zadania i recenzencie poprawności wykonania rozwiązania zadania
\newcommand{\rozwStop}{\newline}                                            %oznaczenie końca rozwiązania
\newcommand{\odpStart}{\noindent \textbf{Odpowiedź:}\newline}    %oznaczenie początku odpowiedzi końcowej (wypisanie wyniku)
\newcommand{\odpStop}{\newline}                                             %oznaczenie końca odpowiedzi końcowej (wypisanie wyniku)
\newcommand{\testStart}{\noindent \textbf{Test:}\newline} %ewentualne możliwe opcje odpowiedzi testowej: A. ? B. ? C. ? D. ? itd.
\newcommand{\testStop}{\newline} %koniec wprowadzania odpowiedzi testowych
\newcommand{\kluczStart}{\noindent \textbf{Test poprawna odpowiedź:}\newline} %klucz, poprawna odpowiedź pytania testowego (jedna literka): A lub B lub C lub D itd.
\newcommand{\kluczStop}{\newline} %koniec poprawnej odpowiedzi pytania testowego 
\newcommand{\wstawGrafike}[2]{\begin{figure}[h] \centering \includegraphics[scale=#2] {#1} \end{figure}} %gdyby była potrzeba wstawienia obrazka, parametry: nazwa pliku, skala (jak nie wiesz co wpisać, to wpisz 1)

\begin{document}
\maketitle

\kategoria{Wikieł/Z5.43h}

\zadStart{Zadanie z Wikieł Z 5.43 h) moja wersja nr [nrWersji]}
%[a]:[1,2,3,4,5,6,7,8,9,10,11]
%[b]:[1,2,3,4,5,6,7,8,9,10,11]
%[c]:[2,3,4,5,6,7,8,9,10,11]
%[d]=[a]**2
%[e]=[b]**2
%[f]=2*[b]*[c]
%[g]=[c]**2
%[h]=[a]*[c]
%[i]=[e]+[d]
%[j]=[b]/[c]
%[k]=round([j],2)
%math.gcd([h],[g],[f],[i])==1
Obliczyć i przedstawić w najprostszej postaci pochodną funkcji f.
$$f(x) = \ln \Big(\text{arctg}\ \frac{[a]}{[b]+[c]x}\Big)$$
\zadStop

\rozwStart{Natalia Danieluk}{}
Dziedzina: $\quad \mathcal{D}_f\approx(-[k],\infty)$
$$f'(x) \mathrel{\stackrel{\makebox[0pt]{\mbox{\normalfont\scriptsize\textbf{(*)}}}}{=}}
\frac{1}{\text{arctg}\ \big(\frac{[a]}{[b]+[c]x}\big)} \cdot \frac{1}{1+\big(\frac{[a]}{[b]+[c]x}\big)^2} \cdot \Bigg(\frac{-[a]}{([b]+[c]x)^2}\Bigg) \cdot [c] =$$
$$= \frac{1}{\text{arctg}\ \big(\frac{[a]}{[b]+[c]x}\big)} \cdot \frac{1}{1+\frac{[d]}{[e]+[f]x+[g]x^2}} \cdot \Bigg(\frac{-[h]}{([b]+[c]x)^2}\Bigg) =$$
$$= \frac{1}{\text{arctg}\ \big(\frac{[a]}{[b]+[c]x}\big)} \cdot \frac{1}{\frac{[g]x^2+[f]x+[i]}{[e]+[f]x+[g]x^2}} \cdot \Bigg(\frac{-[h]}{([b]+[c]x)^2}\Bigg) =$$
$$= \frac{1}{\text{arctg}\ \big(\frac{[a]}{[b]+[c]x}\big)} \cdot \frac{([b]+[c]x)^2}{[g]x^2+[f]x+[i]} \cdot \Bigg(\frac{-[h]}{([b]+[c]x)^2}\Bigg) =$$
$$= \frac{-[h]}{([g]x^2+[f]x+[i])\ \text{arctg}\ \big(\frac{[a]}{[b]+[c]x}\big)}$$
{\normalfont\scriptsize\textbf{(*)}\\
$y = \text{arctg}\ x \Leftrightarrow x = \tg y$\\
$(\text{arctg}\ x)' = \frac{1}{(\tg y)'} = \cos^2y = \frac{\cos^2y}{1} = \frac{\cos^2y}{\sin^2y + \cos^2y} = \frac{1}{1 + \tg^2y}= \frac{1}{1 + x^2}$}
\rozwStop

\odpStart
$f'(x) = \frac{-[h]}{([g]x^2+[f]x+[i])\ \text{arctg}\ \big(\frac{[a]}{[b]+[c]x}\big)}$
\odpStop

\testStart
A. $f'(x) = \ln \Big(\text{arctg}\ \frac{[a]}{[b]+[c]x}\Big)$\\
B. $f'(x) = \frac{1}{\text{arctg}\ \big(\frac{[a]}{[b]+[c]x}\big)} \cdot \frac{1}{1+\big(\frac{[a]}{[b]+[c]x}\big)^2} \cdot \Bigg(\frac{-[a]}{([b]+[c]x)^2}\Bigg) \cdot [c]$\\
C. $f'(x) = \frac{-[a]}{([g]x^2+[f]x+[i])\ \text{arctg}\ \big(\frac{[a]}{[b]+[c]x}\big)}$\\
D. $f'(x) = \frac{-[h]}{([g]x^2+[f]x+[i])\ \text{arctg}\ \big(\frac{[a]}{[b]+[c]x}\big)}$
\testStop

\kluczStart
D
\kluczStop

\end{document}