\documentclass[12pt, a4paper]{article}
\usepackage[utf8]{inputenc}
\usepackage{polski}

\usepackage{amsthm}  %pakiet do tworzenia twierdzeń itp.
\usepackage{amsmath} %pakiet do niektórych symboli matematycznych
\usepackage{amssymb} %pakiet do symboli mat., np. \nsubseteq
\usepackage{amsfonts}
\usepackage{graphicx} %obsługa plików graficznych z rozszerzeniem png, jpg
\theoremstyle{definition} %styl dla definicji
\newtheorem{zad}{} 
\title{Multizestaw zadań}
\author{Robert Fidytek}
%\date{\today}
\date{}
\newcounter{liczniksekcji}
\newcommand{\kategoria}[1]{\section{#1}} %olreślamy nazwę kateforii zadań
\newcommand{\zadStart}[1]{\begin{zad}#1\newline} %oznaczenie początku zadania
\newcommand{\zadStop}{\end{zad}}   %oznaczenie końca zadania
%Makra opcjonarne (nie muszą występować):
\newcommand{\rozwStart}[2]{\noindent \textbf{Rozwiązanie (autor #1 , recenzent #2): }\newline} %oznaczenie początku rozwiązania, opcjonarnie można wprowadzić informację o autorze rozwiązania zadania i recenzencie poprawności wykonania rozwiązania zadania
\newcommand{\rozwStop}{\newline}                                            %oznaczenie końca rozwiązania
\newcommand{\odpStart}{\noindent \textbf{Odpowiedź:}\newline}    %oznaczenie początku odpowiedzi końcowej (wypisanie wyniku)
\newcommand{\odpStop}{\newline}                                             %oznaczenie końca odpowiedzi końcowej (wypisanie wyniku)
\newcommand{\testStart}{\noindent \textbf{Test:}\newline} %ewentualne możliwe opcje odpowiedzi testowej: A. ? B. ? C. ? D. ? itd.
\newcommand{\testStop}{\newline} %koniec wprowadzania odpowiedzi testowych
\newcommand{\kluczStart}{\noindent \textbf{Test poprawna odpowiedź:}\newline} %klucz, poprawna odpowiedź pytania testowego (jedna literka): A lub B lub C lub D itd.
\newcommand{\kluczStop}{\newline} %koniec poprawnej odpowiedzi pytania testowego 
\newcommand{\wstawGrafike}[2]{\begin{figure}[h] \includegraphics[scale=#2] {#1} \end{figure}} %gdyby była potrzeba wstawienia obrazka, parametry: nazwa pliku, skala (jak nie wiesz co wpisać, to wpisz 1)

\begin{document}
\maketitle


\kategoria{Wikieł/Z1.97e}
\zadStart{Zadanie z Wikieł Z 1.97 e) moja wersja nr [nrWersji]}
%[a]:[2,3,4,5]
%[b]:[2,3,4,5,6,7,8,9,10]
%[c]=pow([a],3)
Rozwiązać nierówno\'sć $[c]^{\log_{[a]}{x}}-[b]x^2 \ge x -[b]$
\zadStop
\rozwStart{Małgorzata Ugowska}{}
Dziedzina: $D = (0, \infty)$\\
Rozwiązujemy nierówno\'sć:
$$[c]^{\log_{[a]}{x}}-[b]x^2 \ge x -[b] \quad \Longleftrightarrow \quad [a]^{3\log_{[a]}{x}}-[b]x^2 -x+[b] \ge 0 $$
$$\Longleftrightarrow \quad x^3 -[b]x^2 -x+[b] \ge 0  \quad \Longleftrightarrow \quad x^2(x-[b]) -(x-[b]) \ge 0 $$
$$ \Longleftrightarrow \quad (x^2-1)(x-[b]) \ge 0 \quad \Longleftrightarrow \quad  (x-1)(x+1)(x-[b]) \ge 0 $$
$$ \Longleftrightarrow \quad x \in [-1, 1] \cup [[b], \infty) \quad \vee \quad x \in D \quad \Longrightarrow \quad x \in (0, 1] \cup [[b], \infty)$$
\rozwStop
\odpStart
$x \in (0, 1] \cup [[b], \infty)$
\odpStop
\testStart
A. $x \in [-1, 1] \cup [[b], \infty)$\\
B. $x \in ([b], \infty)$\\
C. $x \in (-\infty, -4) \cup (3,\infty)$\\
D. $x \in (0, 1] \cup [[b], \infty)$\\
E. $x \in \emptyset$
\testStop
\kluczStart
D
\kluczStop



\end{document}