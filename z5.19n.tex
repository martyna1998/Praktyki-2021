\documentclass[12pt, a4paper]{article}
\usepackage[utf8]{inputenc}
\usepackage{polski}

\usepackage{amsthm}  %pakiet do tworzenia twierdzeń itp.
\usepackage{amsmath} %pakiet do niektórych symboli matematycznych
\usepackage{amssymb} %pakiet do symboli mat., np. \nsubseteq
\usepackage{amsfonts}
\usepackage{graphicx} %obsługa plików graficznych z rozszerzeniem png, jpg
\theoremstyle{definition} %styl dla definicji
\newtheorem{zad}{} 
\title{Multizestaw zadań}
\author{Robert Fidytek}
%\date{\today}
\date{}
\newcounter{liczniksekcji}
\newcommand{\kategoria}[1]{\section{#1}} %olreślamy nazwę kateforii zadań
\newcommand{\zadStart}[1]{\begin{zad}#1\newline} %oznaczenie początku zadania
\newcommand{\zadStop}{\end{zad}}   %oznaczenie końca zadania
%Makra opcjonarne (nie muszą występować):
\newcommand{\rozwStart}[2]{\noindent \textbf{Rozwiązanie (autor #1 , recenzent #2): }\newline} %oznaczenie początku rozwiązania, opcjonarnie można wprowadzić informację o autorze rozwiązania zadania i recenzencie poprawności wykonania rozwiązania zadania
\newcommand{\rozwStop}{\newline}                                            %oznaczenie końca rozwiązania
\newcommand{\odpStart}{\noindent \textbf{Odpowiedź:}\newline}    %oznaczenie początku odpowiedzi końcowej (wypisanie wyniku)
\newcommand{\odpStop}{\newline}                                             %oznaczenie końca odpowiedzi końcowej (wypisanie wyniku)
\newcommand{\testStart}{\noindent \textbf{Test:}\newline} %ewentualne możliwe opcje odpowiedzi testowej: A. ? B. ? C. ? D. ? itd.
\newcommand{\testStop}{\newline} %koniec wprowadzania odpowiedzi testowych
\newcommand{\kluczStart}{\noindent \textbf{Test poprawna odpowiedź:}\newline} %klucz, poprawna odpowiedź pytania testowego (jedna literka): A lub B lub C lub D itd.
\newcommand{\kluczStop}{\newline} %koniec poprawnej odpowiedzi pytania testowego 
\newcommand{\wstawGrafike}[2]{\begin{figure}[h] \includegraphics[scale=#2] {#1} \end{figure}} %gdyby była potrzeba wstawienia obrazka, parametry: nazwa pliku, skala (jak nie wiesz co wpisać, to wpisz 1)

\begin{document}
\maketitle


\kategoria{Wikieł/Z5.19 n}
\zadStart{Zadanie z Wikieł Z 5.19 n) moja wersja nr [nrWersji]}
%[a]:[2,3,4,5,6,7,8,9]
%[b]:[2,3,4,5,6,7,8,9]
%[b]!=0
Oblicz granicę $\lim_{x \rightarrow 0^{+}} \left( [a]x\right)^{\sin([b]x)}$.
\zadStop
\rozwStart{Joanna Świerzbin}{}
$$\lim_{x \rightarrow 0^{+}} \left( [a]x\right)^{\sin([b]x)}= \lim_{x \rightarrow 0^{+}} e^{\ln \left(\left( [a]x\right)^{\sin([b]x)} \right)}=
e^{\lim_{x \rightarrow 0^{+}}  {\sin([b]x)} \ln \left( [a]x\right)}$$
Policzmy ${\lim_{x \rightarrow 0^{+}}  {\sin([b]x)} \ln \left( [a]x\right)}$.
$${\lim_{x \rightarrow 0^{+}}  {\sin([b]x)} \ln \left( [a]x\right)} = \lim_{x \rightarrow 0^{+}} \frac{\ln \left( [a]x\right)}{\frac{1}{\sin([b]x)}}$$
Otrzymujemy $ \left[ \frac{-\infty}{\infty} \right] $ więc możemy skorzystać z twierdzenia de l'Hospitala.
$$\lim_{x \rightarrow 0^{+}} \frac{\left( \ln \left( [a]x\right) \right)'}{\left(\frac{1}{\sin([b]x)}\right)'} =
\lim_{x \rightarrow 0^{+}} \frac{\frac{1}{[a]x}([a]x)'}{\frac{-1}{\sin^2([b]x)}\left( \sin([b]x) \right)'} =
\lim_{x \rightarrow 0^{+}} \frac{\frac{[a]}{[a]x}}{\frac{-\cos([b]x)}{\sin^2([b]x)}\left( [b]x \right)'} =$$
$$=-\lim_{x \rightarrow 0^{+}} \frac{\frac{1}{x}}{\frac{[b]\cos([b]x)}{\sin^2([b]x)}} =
-\lim_{x \rightarrow 0^{+}} \frac{1}{x} \cdot \frac{\sin^2([b]x)}{[b]\cos([b]x)} =
-\frac{1}{[b]} \lim_{x \rightarrow 0^{+}} \frac{\sin^2([b]x)}{x} \lim_{x \rightarrow 0^{+}} \frac{1}{\cos([b]x)} =$$
$$ = -\frac{1}{[b]} \lim_{x \rightarrow 0^{+}} \frac{\sin^2([b]x)}{x}  $$
Otrzymujemy $ \left[ \frac{0}{0} \right] $ więc możemy skorzystać z twierdzenia de l'Hospitala.
$$ -\frac{1}{[b]} \lim_{x \rightarrow 0^{+}} \frac{\left(\sin^2([b]x)\right)'}{\left(x\right)'}=
-\frac{1}{[b]} \lim_{x \rightarrow 0^{+}} \frac{2\sin([b]x)\left( \sin([b]x) \right)'}{1}=$$
$$=-\frac{1}{[b]} \lim_{x \rightarrow 0^{+}} {2\sin([b]x) \cos([b]x)([b]x)'}= -\frac{1}{[b]} \lim_{x \rightarrow 0^{+}} {2\cdot [b]\sin([b]x) \cos([b]x)}=$$
$$= -2 \lim_{x \rightarrow 0^{+}} {\sin([b]x) \lim_{x \rightarrow 0^{+}} \cos([b]x)}= 0$$
Podstawmy do początkowego przykładu.
$$e^{\lim_{x \rightarrow 0^{+}}  {\sin([b]x)} \ln \left( [a]x\right)} = e^{0} =1 $$
\rozwStop
\odpStart
$\lim_{x \rightarrow 0^{+}} \left( [a]x\right)^{\sin([b]x)} = 1$
\odpStop
\testStart
A. $\lim_{x \rightarrow 0^{+}} \left( [a]x\right)^{\sin([b]x)} = 1$\\
B. $\lim_{x \rightarrow 0^{+}} \left( [a]x\right)^{\sin([b]x)} = \infty$\\
C. $\lim_{x \rightarrow 0^{+}} \left( [a]x\right)^{\sin([b]x)} = e$\\
D. $\lim_{x \rightarrow 0^{+}} \left( [a]x\right)^{\sin([b]x)} = -\infty$\\
E. $\lim_{x \rightarrow 0^{+}} \left( [a]x\right)^{\sin([b]x)} = 0$\\
F. $\lim_{x \rightarrow 0^{+}} \left( [a]x\right)^{\sin([b]x)} = [b]$
\testStop
\kluczStart
A
\kluczStop



\end{document}