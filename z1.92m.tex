\documentclass[12pt, a4paper]{article}
\usepackage[utf8]{inputenc}
\usepackage{polski}

\usepackage{amsthm}  %pakiet do tworzenia twierdzeń itp.
\usepackage{amsmath} %pakiet do niektórych symboli matematycznych
\usepackage{amssymb} %pakiet do symboli mat., np. \nsubseteq
\usepackage{amsfonts}
\usepackage{graphicx} %obsługa plików graficznych z rozszerzeniem png, jpg
\theoremstyle{definition} %styl dla definicji
\newtheorem{zad}{} 
\title{Multizestaw zadań}
\author{Robert Fidytek}
%\date{\today}
\date{}
\newcounter{liczniksekcji}
\newcommand{\kategoria}[1]{\section{#1}} %olreślamy nazwę kateforii zadań
\newcommand{\zadStart}[1]{\begin{zad}#1\newline} %oznaczenie początku zadania
\newcommand{\zadStop}{\end{zad}}   %oznaczenie końca zadania
%Makra opcjonarne (nie muszą występować):
\newcommand{\rozwStart}[2]{\noindent \textbf{Rozwiązanie (autor #1 , recenzent #2): }\newline} %oznaczenie początku rozwiązania, opcjonarnie można wprowadzić informację o autorze rozwiązania zadania i recenzencie poprawności wykonania rozwiązania zadania
\newcommand{\rozwStop}{\newline}                                            %oznaczenie końca rozwiązania
\newcommand{\odpStart}{\noindent \textbf{Odpowiedź:}\newline}    %oznaczenie początku odpowiedzi końcowej (wypisanie wyniku)
\newcommand{\odpStop}{\newline}                                             %oznaczenie końca odpowiedzi końcowej (wypisanie wyniku)
\newcommand{\testStart}{\noindent \textbf{Test:}\newline} %ewentualne możliwe opcje odpowiedzi testowej: A. ? B. ? C. ? D. ? itd.
\newcommand{\testStop}{\newline} %koniec wprowadzania odpowiedzi testowych
\newcommand{\kluczStart}{\noindent \textbf{Test poprawna odpowiedź:}\newline} %klucz, poprawna odpowiedź pytania testowego (jedna literka): A lub B lub C lub D itd.
\newcommand{\kluczStop}{\newline} %koniec poprawnej odpowiedzi pytania testowego 
\newcommand{\wstawGrafike}[2]{\begin{figure}[h] \includegraphics[scale=#2] {#1} \end{figure}} %gdyby była potrzeba wstawienia obrazka, parametry: nazwa pliku, skala (jak nie wiesz co wpisać, to wpisz 1)

\begin{document}
\maketitle


\kategoria{Wikieł/Z1.92m}
\zadStart{Zadanie z Wikieł Z 1.92 m) moja wersja nr [nrWersji]}
%[a]:[2,3,4,6,8,10,12,14]
%[b]:[5,7,9,11,13]
%[c]:[2,3,4,5,6,7,8,9]
%[d]:[1,2,3,4,5,6,7,8]
%[dd]=pow([d],2)
%[p]=([b]-[a])
%[2d]=(2*[d])
%[k]=([2d]*[c])
%[l]=[b]-([dd]*[c])
%[r2]=([p]/[c])
%[r]=int([r2])
%[delta]=(([k]**2)+4*([c]*[l]))
%[2c]=2*[c]
%[b]>[a] and [r2].is_integer()==True and [r2].is_integer()==True and [l]>0 and (2*[2d])<([k]+(pow([delta],1/2))) and [l]>0 and ([d]**2-([p]/[c]))==0
Rozwiązać równanie $ \log_{\frac{1}{[a]}}{\big([b]- [c] (x-[d])^2\big)} = -1$
\zadStop
\rozwStart{Małgorzata Ugowska}{}
Dziedzina:
$$[b]- [c] (x-[d])^2>0$$
$$ -[b]+[c] (x^2-[2d]x+[dd])<0$$
$$ [c] x^2-[k]x-[l]<0$$
$$ \bigtriangleup = [k]^2 + 4 \cdot [c] \cdot [l] = [delta] \quad  \Longrightarrow \quad \sqrt{\bigtriangleup} = \sqrt{[delta]}$$
$$x =\frac{[k]-\sqrt{[delta]}}{[2c]} \quad \vee \quad x =\frac{[k]-\sqrt{[delta]}}{[2c]}$$
$$D=\Big(\frac{[k]-\sqrt{[delta]}}{[2c]}, \frac{[k]+\sqrt{[delta]}}{[2c]}\Big)$$
Rozwiązujemy równanie:
$$\log_{\frac{1}{[a]}}{\big([b]- [c] (x-[d])^2\big)} = -1$$
$$ [b]- [c] (x-[d])^2= [a]$$
$$ [c] (x^2-[2d]x+[dd])= [p]$$
$$ x^2-[2d]x+[dd]= [r]$$
$$x^2-[2d]x=0$$
$$x(x-[2d])=0$$
$$x=0 \in D \quad \vee \quad x=[2d] \in D$$
\rozwStop
\odpStart
$x \in \{0, [2d] \}$
\odpStop
\testStart
A. $x \in \{-[r], 0 \}$\\
B. $x = -1$\\
C. $x \in \{0, [2d] \}$\\
D. $x = 3$\\
E. $x \in \{-1, 3\}$\\
F. $x \in \{-[2d], 0 \}$
\testStop
\kluczStart
C
\kluczStop



\end{document}