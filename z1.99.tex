\documentclass[12pt, a4paper]{article}
\usepackage[utf8]{inputenc}
\usepackage{polski}

\usepackage{amsthm}  %pakiet do tworzenia twierdzeń itp.
\usepackage{amsmath} %pakiet do niektórych symboli matematycznych
\usepackage{amssymb} %pakiet do symboli mat., np. \nsubseteq
\usepackage{amsfonts}
\usepackage{graphicx} %obsługa plików graficznych z rozszerzeniem png, jpg
\theoremstyle{definition} %styl dla definicji
\newtheorem{zad}{} 
\title{Multizestaw zadań}
\author{Robert Fidytek}
%\date{\today}
\date{}
\newcounter{liczniksekcji}
\newcommand{\kategoria}[1]{\section{#1}} %olreślamy nazwę kateforii zadań
\newcommand{\zadStart}[1]{\begin{zad}#1\newline} %oznaczenie początku zadania
\newcommand{\zadStop}{\end{zad}}   %oznaczenie końca zadania
%Makra opcjonarne (nie muszą występować):
\newcommand{\rozwStart}[2]{\noindent \textbf{Rozwiązanie (autor #1 , recenzent #2): }\newline} %oznaczenie początku rozwiązania, opcjonarnie można wprowadzić informację o autorze rozwiązania zadania i recenzencie poprawności wykonania rozwiązania zadania
\newcommand{\rozwStop}{\newline}                                            %oznaczenie końca rozwiązania
\newcommand{\odpStart}{\noindent \textbf{Odpowiedź:}\newline}    %oznaczenie początku odpowiedzi końcowej (wypisanie wyniku)
\newcommand{\odpStop}{\newline}                                             %oznaczenie końca odpowiedzi końcowej (wypisanie wyniku)
\newcommand{\testStart}{\noindent \textbf{Test:}\newline} %ewentualne możliwe opcje odpowiedzi testowej: A. ? B. ? C. ? D. ? itd.
\newcommand{\testStop}{\newline} %koniec wprowadzania odpowiedzi testowych
\newcommand{\kluczStart}{\noindent \textbf{Test poprawna odpowiedź:}\newline} %klucz, poprawna odpowiedź pytania testowego (jedna literka): A lub B lub C lub D itd.
\newcommand{\kluczStop}{\newline} %koniec poprawnej odpowiedzi pytania testowego 
\newcommand{\wstawGrafike}[2]{\begin{figure}[h] \includegraphics[scale=#2] {#1} \end{figure}} %gdyby była potrzeba wstawienia obrazka, parametry: nazwa pliku, skala (jak nie wiesz co wpisać, to wpisz 1)

\begin{document}
\maketitle


\kategoria{Wikieł/Z1.99}
\zadStart{Zadanie z Wikieł Z 1.99 moja wersja nr [nrWersji]}
%[a]:[2,3,4,5]
%[p]=random.randint(1,8)
%[r]:[2,3,4,5,6,7,8]
%[m]=random.randint(3,9)
%[ap]=[p]*[a]
%[rap]=[ap]+[r]
%math.gcd([r],[a])==1
Dane są funkcje $f(x) = \log_{[m]}{\frac{x-[p]}{x}}$ i $g(x)= \log_{[m]}{(\frac{[r]}{[a] x}+1)}$. Wyznaczyć warto\'sci $x$, dla których zachodzi nierówno\'sć $f(x)<g(x).$
\zadStop
\rozwStart{Małgorzata Ugowska}{}
$$f(x)<g(x) \quad \Longleftrightarrow \quad \log_{[m]}{\frac{x-[p]}{x}} \quad < \quad \log_{[m]}{\Big(\frac{[r]}{[a] x}+1\Big)}$$
$$ \Longleftrightarrow \quad \frac{x-[p]}{x} \quad < \quad \frac{[r]}{[a] x}+1 \quad \Longleftrightarrow \quad  \frac{[a]x-[ap]}{[a]x} \quad < \quad \frac{[r]+[a]x}{[a] x} $$
$$\quad \Longleftrightarrow \quad  \frac{[a]x-[ap]}{[a]x} - \frac{[r]+[a]x}{[a] x} \quad < \quad 0 \quad \Longleftrightarrow \quad  \frac{[a]x-[ap]-[r]-[a]x}{[a] x} \quad < \quad 0 $$
$$ \Longleftrightarrow \quad \frac{-[rap]}{[a] x} \quad < \quad 0 \quad \Longleftrightarrow \quad x>0$$
Należy jeszcze wziąć pod uwagę dziedziny funkcji $f(x)$ i $g(x)$.
$$D_{f} = (-\infty,0) \cup ([p], \infty)$$
$$D_{g} =\Big(-\infty,-\frac{[r]}{[a]}\Big) \cup (0,\infty)$$
Czę\'scią wspólną dziedziny funkcji $f(x)$, $g(x)$ oraz naszego rozwiązania jest przedział $([p], \infty)$
\rozwStop
\odpStart
$x \in ([p], \infty)$
\odpStop
\testStart
A. $x \in (0, \infty)$\\
B. $x \in ([p], \infty)$\\
C. $x \in (-\infty,-\frac{[r]}{[a]})$\\
D. $(-\infty,-[ap])$\\
E. $x \in ([rap], \infty)$
\testStop
\kluczStart
B
\kluczStop



\end{document}