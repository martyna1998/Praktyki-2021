\documentclass[12pt, a4paper]{article}
\usepackage[utf8]{inputenc}
\usepackage{polski}

\usepackage{amsthm}  %pakiet do tworzenia twierdzeń itp.
\usepackage{amsmath} %pakiet do niektórych symboli matematycznych
\usepackage{amssymb} %pakiet do symboli mat., np. \nsubseteq
\usepackage{amsfonts}
\usepackage{graphicx} %obsługa plików graficznych z rozszerzeniem png, jpg
\theoremstyle{definition} %styl dla definicji
\newtheorem{zad}{} 
\title{Multizestaw zadań}
\author{Robert Fidytek}
%\date{\today}
\date{}
\newcounter{liczniksekcji}
\newcommand{\kategoria}[1]{\section{#1}} %olreślamy nazwę kateforii zadań
\newcommand{\zadStart}[1]{\begin{zad}#1\newline} %oznaczenie początku zadania
\newcommand{\zadStop}{\end{zad}}   %oznaczenie końca zadania
%Makra opcjonarne (nie muszą występować):
\newcommand{\rozwStart}[2]{\noindent \textbf{Rozwiązanie (autor #1 , recenzent #2): }\newline} %oznaczenie początku rozwiązania, opcjonarnie można wprowadzić informację o autorze rozwiązania zadania i recenzencie poprawności wykonania rozwiązania zadania
\newcommand{\rozwStop}{\newline}                                            %oznaczenie końca rozwiązania
\newcommand{\odpStart}{\noindent \textbf{Odpowiedź:}\newline}    %oznaczenie początku odpowiedzi końcowej (wypisanie wyniku)
\newcommand{\odpStop}{\newline}                                             %oznaczenie końca odpowiedzi końcowej (wypisanie wyniku)
\newcommand{\testStart}{\noindent \textbf{Test:}\newline} %ewentualne możliwe opcje odpowiedzi testowej: A. ? B. ? C. ? D. ? itd.
\newcommand{\testStop}{\newline} %koniec wprowadzania odpowiedzi testowych
\newcommand{\kluczStart}{\noindent \textbf{Test poprawna odpowiedź:}\newline} %klucz, poprawna odpowiedź pytania testowego (jedna literka): A lub B lub C lub D itd.
\newcommand{\kluczStop}{\newline} %koniec poprawnej odpowiedzi pytania testowego 
\newcommand{\wstawGrafike}[2]{\begin{figure}[h] \includegraphics[scale=#2] {#1} \end{figure}} %gdyby była potrzeba wstawienia obrazka, parametry: nazwa pliku, skala (jak nie wiesz co wpisać, to wpisz 1)

\begin{document}
\maketitle


\kategoria{Wikieł/Z1.80f}
\zadStart{Zadanie z Wikieł Z 1.80 f)  moja wersja nr [nrWersji]}

%[p1]=random.randint(2,10)
%[a]:[2,3,4,5,6,7,8,9,10,11,12,13,14,15]
%[ak]=pow([a],2)
%[b]:[2,3,4,5,6,7,8,9,10,11,12,13,14,15]
%[bk]=pow([b],2)
%[c]=random.randint(2,10)
%[akc]=[ak]*[c]
%[dz]=math.gcd([bk],[akc])
%[l]=int([bk]/[dz])
%[m]=int([akc]/[dz])
%[p3]:[1,8,27,64,125,216]
%[p4]:[1,8,27,64,125,216]
%[nl]=int([l]**(1./3.))
%[nm]=int([m]**(1./3.))
%[nmnl]=[nm]-[nl]
%[del]=round(1-4*[nmnl]*[p1]/[nm],2)
%([akc]/[dz])==[p3] and ([bk]/[dz])==[p4] and [nmnl]!=0 and [del]<0 and [a]!=[b]


Rozwiązać równanie 
$$(x^{2}-x+[p1])^{-\frac{3}{2}}=\frac{[a]}{[b]}\sqrt{[c]}x^{-3}$$
\zadStop

\rozwStart{Maja Szabłowska}{}
$$(x^{2}-x+[p1])^{-\frac{3}{2}}=\frac{[a]}{[b]}\sqrt{[c]}x^{-3}$$
$$x^{2}-x+[p1]=\left(\frac{[a]}{[b]}\sqrt{[c]}x^{-3}\right)^{-\frac{2}{3}}$$
$$x^{2}-x+[p1]=\left(\frac{[b]}{[a]}\cdot\frac{1}{\sqrt{[c]}}x^{3}\right)^{\frac{2}{3}}$$
$$x^{2}-x+[p1]=\left(\frac{[bk]}{[ak]}\cdot\frac{1}{[c]}\right)^{\frac{1}{3}}x^{2}$$
$$x^{2}-x+[p1]=\left(\frac{[bk]}{[akc]}\right)^{\frac{1}{3}}x^{2}$$
$$x^{2}-x+[p1]=\left(\frac{[l]}{[m]}\right)^{\frac{1}{3}}x^{2}$$
$$x^{2}-x+[p1]=\frac{[nl]}{[nm]}x^{2}$$
$$\frac{[nmnl]}{[nm]}x^{2}-x+[p1]=0$$
$$\Delta=(-1)^2 - 4\cdot\frac{[nmnl]}{[nm]}\cdot[p1]=[del]<0$$
Brak rozwiązań w zbiorze liczb rzeczywistych.
\rozwStop
\odpStart
Brak rozwiązań w zbiorze liczb rzeczywistych.
\odpStop
\testStart
A.Brak rozwiązań w zbiorze liczb rzeczywistych.\\
B.$x_{1}=[a], \quad x_{2}=[bk]$\\
D.$x_{1}=[l], \quad x_{2}=[m]$\\
E.$x_{1}=[del], \quad x_{2}=[b]$\\
F.$x_{1}=[ak], \quad x_{2}=[c]$\\
G.$x_{1}=[p3], \quad x_{2}=[bk]$\\
H.$x_{1}=[akc], \quad x_{2}=[a]$\\
\testStop
\kluczStart
A
\kluczStop



\end{document}
