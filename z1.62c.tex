\documentclass[12pt, a4paper]{article}
\usepackage[utf8]{inputenc}
\usepackage{polski}
\usepackage{amsthm}  %pakiet do tworzenia twierdzeń itp.
\usepackage{amsmath} %pakiet do niektórych symboli matematycznych
\usepackage{amssymb} %pakiet do symboli mat., np. \nsubseteq
\usepackage{amsfonts}
\usepackage{graphicx} %obsługa plików graficznych z rozszerzeniem png, jpg
\theoremstyle{definition} %styl dla definicji
\newtheorem{zad}{} 
\title{Multizestaw zadań}
\author{Patryk Wirkus}
%\date{\today}
\date{}
\newcommand{\kategoria}[1]{\section{#1}}
\newcommand{\zadStart}[1]{\begin{zad}#1\newline}
\newcommand{\zadStop}{\end{zad}}
\newcommand{\rozwStart}[2]{\noindent \textbf{Rozwiązanie (autor #1 , recenzent #2): }\newline}
\newcommand{\rozwStop}{\newline}                                           
\newcommand{\odpStart}{\noindent \textbf{Odpowiedź:}\newline}
\newcommand{\odpStop}{\newline}
\newcommand{\testStart}{\noindent \textbf{Test:}\newline}
\newcommand{\testStop}{\newline}
\newcommand{\kluczStart}{\noindent \textbf{Test poprawna odpowiedź:}\newline}
\newcommand{\kluczStop}{\newline}
\newcommand{\wstawGrafike}[2]{\begin{figure}[h] \includegraphics[scale=#2] {#1} \end{figure}}

\begin{document}
\maketitle

\kategoria{Wikieł/1.62c}


\zadStart{Zadanie z Wikieł Z 1.62 c) moja wersja nr 1}

Rozwiązać nierówności $(2-x)(x+1)^{2}(3-x)^{3}\le0$.
\zadStop
\rozwStart{Patryk Wirkus}{}
Miejsca zerowe naszego wielomianu to: $2, -1, 3$.\\
Wielomian jest stopnia parzystego, ponadto znak współczynnika przy\linebreak najwyższej potędze x jest ujemny.\\ W związku z tym wykres wielomianu zaczyna się od lewej strony powyżej osi OX.\\
Ponadto w punkcie $-1$ wykres odbija się od osi poziomej.\\
A więc $$x \in \{-1\} \cup [2,3].$$
\rozwStop
\odpStart
$x \in \{-1\} \cup [2,3]$
\odpStop
\testStart
A.$x \in \{-1\} \cup [2,3]$\\
B.$x \in \{1\} \cup (2,3)$\\
C.$x \in \{-1\} \cup (2,3]$\\
D.$x \in \{1\} \cup (2,3]$\\
E.$x \in \{-1\} \cup [2,3)$\\
F.$x \in \{1\} \cup [2,3)$\\
G.$x \in \{-1\} \cup (2,3)$\\
H.$x \in \{1\} \cup [2,3]$
\testStop
\kluczStart
A
\kluczStop



\zadStart{Zadanie z Wikieł Z 1.62 c) moja wersja nr 2}

Rozwiązać nierówności $(2-x)(x+1)^{2}(4-x)^{3}\le0$.
\zadStop
\rozwStart{Patryk Wirkus}{}
Miejsca zerowe naszego wielomianu to: $2, -1, 4$.\\
Wielomian jest stopnia parzystego, ponadto znak współczynnika przy\linebreak najwyższej potędze x jest ujemny.\\ W związku z tym wykres wielomianu zaczyna się od lewej strony powyżej osi OX.\\
Ponadto w punkcie $-1$ wykres odbija się od osi poziomej.\\
A więc $$x \in \{-1\} \cup [2,4].$$
\rozwStop
\odpStart
$x \in \{-1\} \cup [2,4]$
\odpStop
\testStart
A.$x \in \{-1\} \cup [2,4]$\\
B.$x \in \{1\} \cup (2,4)$\\
C.$x \in \{-1\} \cup (2,4]$\\
D.$x \in \{1\} \cup (2,4]$\\
E.$x \in \{-1\} \cup [2,4)$\\
F.$x \in \{1\} \cup [2,4)$\\
G.$x \in \{-1\} \cup (2,4)$\\
H.$x \in \{1\} \cup [2,4]$
\testStop
\kluczStart
A
\kluczStop



\zadStart{Zadanie z Wikieł Z 1.62 c) moja wersja nr 3}

Rozwiązać nierówności $(2-x)(x+1)^{2}(5-x)^{3}\le0$.
\zadStop
\rozwStart{Patryk Wirkus}{}
Miejsca zerowe naszego wielomianu to: $2, -1, 5$.\\
Wielomian jest stopnia parzystego, ponadto znak współczynnika przy\linebreak najwyższej potędze x jest ujemny.\\ W związku z tym wykres wielomianu zaczyna się od lewej strony powyżej osi OX.\\
Ponadto w punkcie $-1$ wykres odbija się od osi poziomej.\\
A więc $$x \in \{-1\} \cup [2,5].$$
\rozwStop
\odpStart
$x \in \{-1\} \cup [2,5]$
\odpStop
\testStart
A.$x \in \{-1\} \cup [2,5]$\\
B.$x \in \{1\} \cup (2,5)$\\
C.$x \in \{-1\} \cup (2,5]$\\
D.$x \in \{1\} \cup (2,5]$\\
E.$x \in \{-1\} \cup [2,5)$\\
F.$x \in \{1\} \cup [2,5)$\\
G.$x \in \{-1\} \cup (2,5)$\\
H.$x \in \{1\} \cup [2,5]$
\testStop
\kluczStart
A
\kluczStop



\zadStart{Zadanie z Wikieł Z 1.62 c) moja wersja nr 4}

Rozwiązać nierówności $(2-x)(x+1)^{2}(6-x)^{3}\le0$.
\zadStop
\rozwStart{Patryk Wirkus}{}
Miejsca zerowe naszego wielomianu to: $2, -1, 6$.\\
Wielomian jest stopnia parzystego, ponadto znak współczynnika przy\linebreak najwyższej potędze x jest ujemny.\\ W związku z tym wykres wielomianu zaczyna się od lewej strony powyżej osi OX.\\
Ponadto w punkcie $-1$ wykres odbija się od osi poziomej.\\
A więc $$x \in \{-1\} \cup [2,6].$$
\rozwStop
\odpStart
$x \in \{-1\} \cup [2,6]$
\odpStop
\testStart
A.$x \in \{-1\} \cup [2,6]$\\
B.$x \in \{1\} \cup (2,6)$\\
C.$x \in \{-1\} \cup (2,6]$\\
D.$x \in \{1\} \cup (2,6]$\\
E.$x \in \{-1\} \cup [2,6)$\\
F.$x \in \{1\} \cup [2,6)$\\
G.$x \in \{-1\} \cup (2,6)$\\
H.$x \in \{1\} \cup [2,6]$
\testStop
\kluczStart
A
\kluczStop



\zadStart{Zadanie z Wikieł Z 1.62 c) moja wersja nr 5}

Rozwiązać nierówności $(2-x)(x+1)^{2}(7-x)^{3}\le0$.
\zadStop
\rozwStart{Patryk Wirkus}{}
Miejsca zerowe naszego wielomianu to: $2, -1, 7$.\\
Wielomian jest stopnia parzystego, ponadto znak współczynnika przy\linebreak najwyższej potędze x jest ujemny.\\ W związku z tym wykres wielomianu zaczyna się od lewej strony powyżej osi OX.\\
Ponadto w punkcie $-1$ wykres odbija się od osi poziomej.\\
A więc $$x \in \{-1\} \cup [2,7].$$
\rozwStop
\odpStart
$x \in \{-1\} \cup [2,7]$
\odpStop
\testStart
A.$x \in \{-1\} \cup [2,7]$\\
B.$x \in \{1\} \cup (2,7)$\\
C.$x \in \{-1\} \cup (2,7]$\\
D.$x \in \{1\} \cup (2,7]$\\
E.$x \in \{-1\} \cup [2,7)$\\
F.$x \in \{1\} \cup [2,7)$\\
G.$x \in \{-1\} \cup (2,7)$\\
H.$x \in \{1\} \cup [2,7]$
\testStop
\kluczStart
A
\kluczStop



\zadStart{Zadanie z Wikieł Z 1.62 c) moja wersja nr 6}

Rozwiązać nierówności $(2-x)(x+1)^{2}(8-x)^{3}\le0$.
\zadStop
\rozwStart{Patryk Wirkus}{}
Miejsca zerowe naszego wielomianu to: $2, -1, 8$.\\
Wielomian jest stopnia parzystego, ponadto znak współczynnika przy\linebreak najwyższej potędze x jest ujemny.\\ W związku z tym wykres wielomianu zaczyna się od lewej strony powyżej osi OX.\\
Ponadto w punkcie $-1$ wykres odbija się od osi poziomej.\\
A więc $$x \in \{-1\} \cup [2,8].$$
\rozwStop
\odpStart
$x \in \{-1\} \cup [2,8]$
\odpStop
\testStart
A.$x \in \{-1\} \cup [2,8]$\\
B.$x \in \{1\} \cup (2,8)$\\
C.$x \in \{-1\} \cup (2,8]$\\
D.$x \in \{1\} \cup (2,8]$\\
E.$x \in \{-1\} \cup [2,8)$\\
F.$x \in \{1\} \cup [2,8)$\\
G.$x \in \{-1\} \cup (2,8)$\\
H.$x \in \{1\} \cup [2,8]$
\testStop
\kluczStart
A
\kluczStop



\zadStart{Zadanie z Wikieł Z 1.62 c) moja wersja nr 7}

Rozwiązać nierówności $(2-x)(x+1)^{2}(9-x)^{3}\le0$.
\zadStop
\rozwStart{Patryk Wirkus}{}
Miejsca zerowe naszego wielomianu to: $2, -1, 9$.\\
Wielomian jest stopnia parzystego, ponadto znak współczynnika przy\linebreak najwyższej potędze x jest ujemny.\\ W związku z tym wykres wielomianu zaczyna się od lewej strony powyżej osi OX.\\
Ponadto w punkcie $-1$ wykres odbija się od osi poziomej.\\
A więc $$x \in \{-1\} \cup [2,9].$$
\rozwStop
\odpStart
$x \in \{-1\} \cup [2,9]$
\odpStop
\testStart
A.$x \in \{-1\} \cup [2,9]$\\
B.$x \in \{1\} \cup (2,9)$\\
C.$x \in \{-1\} \cup (2,9]$\\
D.$x \in \{1\} \cup (2,9]$\\
E.$x \in \{-1\} \cup [2,9)$\\
F.$x \in \{1\} \cup [2,9)$\\
G.$x \in \{-1\} \cup (2,9)$\\
H.$x \in \{1\} \cup [2,9]$
\testStop
\kluczStart
A
\kluczStop



\zadStart{Zadanie z Wikieł Z 1.62 c) moja wersja nr 8}

Rozwiązać nierówności $(2-x)(x+1)^{2}(10-x)^{3}\le0$.
\zadStop
\rozwStart{Patryk Wirkus}{}
Miejsca zerowe naszego wielomianu to: $2, -1, 10$.\\
Wielomian jest stopnia parzystego, ponadto znak współczynnika przy\linebreak najwyższej potędze x jest ujemny.\\ W związku z tym wykres wielomianu zaczyna się od lewej strony powyżej osi OX.\\
Ponadto w punkcie $-1$ wykres odbija się od osi poziomej.\\
A więc $$x \in \{-1\} \cup [2,10].$$
\rozwStop
\odpStart
$x \in \{-1\} \cup [2,10]$
\odpStop
\testStart
A.$x \in \{-1\} \cup [2,10]$\\
B.$x \in \{1\} \cup (2,10)$\\
C.$x \in \{-1\} \cup (2,10]$\\
D.$x \in \{1\} \cup (2,10]$\\
E.$x \in \{-1\} \cup [2,10)$\\
F.$x \in \{1\} \cup [2,10)$\\
G.$x \in \{-1\} \cup (2,10)$\\
H.$x \in \{1\} \cup [2,10]$
\testStop
\kluczStart
A
\kluczStop



\zadStart{Zadanie z Wikieł Z 1.62 c) moja wersja nr 9}

Rozwiązać nierówności $(2-x)(x+1)^{2}(11-x)^{3}\le0$.
\zadStop
\rozwStart{Patryk Wirkus}{}
Miejsca zerowe naszego wielomianu to: $2, -1, 11$.\\
Wielomian jest stopnia parzystego, ponadto znak współczynnika przy\linebreak najwyższej potędze x jest ujemny.\\ W związku z tym wykres wielomianu zaczyna się od lewej strony powyżej osi OX.\\
Ponadto w punkcie $-1$ wykres odbija się od osi poziomej.\\
A więc $$x \in \{-1\} \cup [2,11].$$
\rozwStop
\odpStart
$x \in \{-1\} \cup [2,11]$
\odpStop
\testStart
A.$x \in \{-1\} \cup [2,11]$\\
B.$x \in \{1\} \cup (2,11)$\\
C.$x \in \{-1\} \cup (2,11]$\\
D.$x \in \{1\} \cup (2,11]$\\
E.$x \in \{-1\} \cup [2,11)$\\
F.$x \in \{1\} \cup [2,11)$\\
G.$x \in \{-1\} \cup (2,11)$\\
H.$x \in \{1\} \cup [2,11]$
\testStop
\kluczStart
A
\kluczStop



\zadStart{Zadanie z Wikieł Z 1.62 c) moja wersja nr 10}

Rozwiązać nierówności $(2-x)(x+1)^{2}(12-x)^{3}\le0$.
\zadStop
\rozwStart{Patryk Wirkus}{}
Miejsca zerowe naszego wielomianu to: $2, -1, 12$.\\
Wielomian jest stopnia parzystego, ponadto znak współczynnika przy\linebreak najwyższej potędze x jest ujemny.\\ W związku z tym wykres wielomianu zaczyna się od lewej strony powyżej osi OX.\\
Ponadto w punkcie $-1$ wykres odbija się od osi poziomej.\\
A więc $$x \in \{-1\} \cup [2,12].$$
\rozwStop
\odpStart
$x \in \{-1\} \cup [2,12]$
\odpStop
\testStart
A.$x \in \{-1\} \cup [2,12]$\\
B.$x \in \{1\} \cup (2,12)$\\
C.$x \in \{-1\} \cup (2,12]$\\
D.$x \in \{1\} \cup (2,12]$\\
E.$x \in \{-1\} \cup [2,12)$\\
F.$x \in \{1\} \cup [2,12)$\\
G.$x \in \{-1\} \cup (2,12)$\\
H.$x \in \{1\} \cup [2,12]$
\testStop
\kluczStart
A
\kluczStop



\zadStart{Zadanie z Wikieł Z 1.62 c) moja wersja nr 11}

Rozwiązać nierówności $(2-x)(x+1)^{2}(13-x)^{3}\le0$.
\zadStop
\rozwStart{Patryk Wirkus}{}
Miejsca zerowe naszego wielomianu to: $2, -1, 13$.\\
Wielomian jest stopnia parzystego, ponadto znak współczynnika przy\linebreak najwyższej potędze x jest ujemny.\\ W związku z tym wykres wielomianu zaczyna się od lewej strony powyżej osi OX.\\
Ponadto w punkcie $-1$ wykres odbija się od osi poziomej.\\
A więc $$x \in \{-1\} \cup [2,13].$$
\rozwStop
\odpStart
$x \in \{-1\} \cup [2,13]$
\odpStop
\testStart
A.$x \in \{-1\} \cup [2,13]$\\
B.$x \in \{1\} \cup (2,13)$\\
C.$x \in \{-1\} \cup (2,13]$\\
D.$x \in \{1\} \cup (2,13]$\\
E.$x \in \{-1\} \cup [2,13)$\\
F.$x \in \{1\} \cup [2,13)$\\
G.$x \in \{-1\} \cup (2,13)$\\
H.$x \in \{1\} \cup [2,13]$
\testStop
\kluczStart
A
\kluczStop



\zadStart{Zadanie z Wikieł Z 1.62 c) moja wersja nr 12}

Rozwiązać nierówności $(2-x)(x+1)^{2}(14-x)^{3}\le0$.
\zadStop
\rozwStart{Patryk Wirkus}{}
Miejsca zerowe naszego wielomianu to: $2, -1, 14$.\\
Wielomian jest stopnia parzystego, ponadto znak współczynnika przy\linebreak najwyższej potędze x jest ujemny.\\ W związku z tym wykres wielomianu zaczyna się od lewej strony powyżej osi OX.\\
Ponadto w punkcie $-1$ wykres odbija się od osi poziomej.\\
A więc $$x \in \{-1\} \cup [2,14].$$
\rozwStop
\odpStart
$x \in \{-1\} \cup [2,14]$
\odpStop
\testStart
A.$x \in \{-1\} \cup [2,14]$\\
B.$x \in \{1\} \cup (2,14)$\\
C.$x \in \{-1\} \cup (2,14]$\\
D.$x \in \{1\} \cup (2,14]$\\
E.$x \in \{-1\} \cup [2,14)$\\
F.$x \in \{1\} \cup [2,14)$\\
G.$x \in \{-1\} \cup (2,14)$\\
H.$x \in \{1\} \cup [2,14]$
\testStop
\kluczStart
A
\kluczStop



\zadStart{Zadanie z Wikieł Z 1.62 c) moja wersja nr 13}

Rozwiązać nierówności $(2-x)(x+1)^{2}(15-x)^{3}\le0$.
\zadStop
\rozwStart{Patryk Wirkus}{}
Miejsca zerowe naszego wielomianu to: $2, -1, 15$.\\
Wielomian jest stopnia parzystego, ponadto znak współczynnika przy\linebreak najwyższej potędze x jest ujemny.\\ W związku z tym wykres wielomianu zaczyna się od lewej strony powyżej osi OX.\\
Ponadto w punkcie $-1$ wykres odbija się od osi poziomej.\\
A więc $$x \in \{-1\} \cup [2,15].$$
\rozwStop
\odpStart
$x \in \{-1\} \cup [2,15]$
\odpStop
\testStart
A.$x \in \{-1\} \cup [2,15]$\\
B.$x \in \{1\} \cup (2,15)$\\
C.$x \in \{-1\} \cup (2,15]$\\
D.$x \in \{1\} \cup (2,15]$\\
E.$x \in \{-1\} \cup [2,15)$\\
F.$x \in \{1\} \cup [2,15)$\\
G.$x \in \{-1\} \cup (2,15)$\\
H.$x \in \{1\} \cup [2,15]$
\testStop
\kluczStart
A
\kluczStop



\zadStart{Zadanie z Wikieł Z 1.62 c) moja wersja nr 14}

Rozwiązać nierówności $(2-x)(x+1)^{2}(16-x)^{3}\le0$.
\zadStop
\rozwStart{Patryk Wirkus}{}
Miejsca zerowe naszego wielomianu to: $2, -1, 16$.\\
Wielomian jest stopnia parzystego, ponadto znak współczynnika przy\linebreak najwyższej potędze x jest ujemny.\\ W związku z tym wykres wielomianu zaczyna się od lewej strony powyżej osi OX.\\
Ponadto w punkcie $-1$ wykres odbija się od osi poziomej.\\
A więc $$x \in \{-1\} \cup [2,16].$$
\rozwStop
\odpStart
$x \in \{-1\} \cup [2,16]$
\odpStop
\testStart
A.$x \in \{-1\} \cup [2,16]$\\
B.$x \in \{1\} \cup (2,16)$\\
C.$x \in \{-1\} \cup (2,16]$\\
D.$x \in \{1\} \cup (2,16]$\\
E.$x \in \{-1\} \cup [2,16)$\\
F.$x \in \{1\} \cup [2,16)$\\
G.$x \in \{-1\} \cup (2,16)$\\
H.$x \in \{1\} \cup [2,16]$
\testStop
\kluczStart
A
\kluczStop



\zadStart{Zadanie z Wikieł Z 1.62 c) moja wersja nr 15}

Rozwiązać nierówności $(2-x)(x+1)^{2}(17-x)^{3}\le0$.
\zadStop
\rozwStart{Patryk Wirkus}{}
Miejsca zerowe naszego wielomianu to: $2, -1, 17$.\\
Wielomian jest stopnia parzystego, ponadto znak współczynnika przy\linebreak najwyższej potędze x jest ujemny.\\ W związku z tym wykres wielomianu zaczyna się od lewej strony powyżej osi OX.\\
Ponadto w punkcie $-1$ wykres odbija się od osi poziomej.\\
A więc $$x \in \{-1\} \cup [2,17].$$
\rozwStop
\odpStart
$x \in \{-1\} \cup [2,17]$
\odpStop
\testStart
A.$x \in \{-1\} \cup [2,17]$\\
B.$x \in \{1\} \cup (2,17)$\\
C.$x \in \{-1\} \cup (2,17]$\\
D.$x \in \{1\} \cup (2,17]$\\
E.$x \in \{-1\} \cup [2,17)$\\
F.$x \in \{1\} \cup [2,17)$\\
G.$x \in \{-1\} \cup (2,17)$\\
H.$x \in \{1\} \cup [2,17]$
\testStop
\kluczStart
A
\kluczStop



\zadStart{Zadanie z Wikieł Z 1.62 c) moja wersja nr 16}

Rozwiązać nierówności $(2-x)(x+1)^{2}(18-x)^{3}\le0$.
\zadStop
\rozwStart{Patryk Wirkus}{}
Miejsca zerowe naszego wielomianu to: $2, -1, 18$.\\
Wielomian jest stopnia parzystego, ponadto znak współczynnika przy\linebreak najwyższej potędze x jest ujemny.\\ W związku z tym wykres wielomianu zaczyna się od lewej strony powyżej osi OX.\\
Ponadto w punkcie $-1$ wykres odbija się od osi poziomej.\\
A więc $$x \in \{-1\} \cup [2,18].$$
\rozwStop
\odpStart
$x \in \{-1\} \cup [2,18]$
\odpStop
\testStart
A.$x \in \{-1\} \cup [2,18]$\\
B.$x \in \{1\} \cup (2,18)$\\
C.$x \in \{-1\} \cup (2,18]$\\
D.$x \in \{1\} \cup (2,18]$\\
E.$x \in \{-1\} \cup [2,18)$\\
F.$x \in \{1\} \cup [2,18)$\\
G.$x \in \{-1\} \cup (2,18)$\\
H.$x \in \{1\} \cup [2,18]$
\testStop
\kluczStart
A
\kluczStop



\zadStart{Zadanie z Wikieł Z 1.62 c) moja wersja nr 17}

Rozwiązać nierówności $(2-x)(x+1)^{2}(19-x)^{3}\le0$.
\zadStop
\rozwStart{Patryk Wirkus}{}
Miejsca zerowe naszego wielomianu to: $2, -1, 19$.\\
Wielomian jest stopnia parzystego, ponadto znak współczynnika przy\linebreak najwyższej potędze x jest ujemny.\\ W związku z tym wykres wielomianu zaczyna się od lewej strony powyżej osi OX.\\
Ponadto w punkcie $-1$ wykres odbija się od osi poziomej.\\
A więc $$x \in \{-1\} \cup [2,19].$$
\rozwStop
\odpStart
$x \in \{-1\} \cup [2,19]$
\odpStop
\testStart
A.$x \in \{-1\} \cup [2,19]$\\
B.$x \in \{1\} \cup (2,19)$\\
C.$x \in \{-1\} \cup (2,19]$\\
D.$x \in \{1\} \cup (2,19]$\\
E.$x \in \{-1\} \cup [2,19)$\\
F.$x \in \{1\} \cup [2,19)$\\
G.$x \in \{-1\} \cup (2,19)$\\
H.$x \in \{1\} \cup [2,19]$
\testStop
\kluczStart
A
\kluczStop



\zadStart{Zadanie z Wikieł Z 1.62 c) moja wersja nr 18}

Rozwiązać nierówności $(2-x)(x+1)^{2}(20-x)^{3}\le0$.
\zadStop
\rozwStart{Patryk Wirkus}{}
Miejsca zerowe naszego wielomianu to: $2, -1, 20$.\\
Wielomian jest stopnia parzystego, ponadto znak współczynnika przy\linebreak najwyższej potędze x jest ujemny.\\ W związku z tym wykres wielomianu zaczyna się od lewej strony powyżej osi OX.\\
Ponadto w punkcie $-1$ wykres odbija się od osi poziomej.\\
A więc $$x \in \{-1\} \cup [2,20].$$
\rozwStop
\odpStart
$x \in \{-1\} \cup [2,20]$
\odpStop
\testStart
A.$x \in \{-1\} \cup [2,20]$\\
B.$x \in \{1\} \cup (2,20)$\\
C.$x \in \{-1\} \cup (2,20]$\\
D.$x \in \{1\} \cup (2,20]$\\
E.$x \in \{-1\} \cup [2,20)$\\
F.$x \in \{1\} \cup [2,20)$\\
G.$x \in \{-1\} \cup (2,20)$\\
H.$x \in \{1\} \cup [2,20]$
\testStop
\kluczStart
A
\kluczStop



\zadStart{Zadanie z Wikieł Z 1.62 c) moja wersja nr 19}

Rozwiązać nierówności $(3-x)(x+1)^{2}(4-x)^{3}\le0$.
\zadStop
\rozwStart{Patryk Wirkus}{}
Miejsca zerowe naszego wielomianu to: $3, -1, 4$.\\
Wielomian jest stopnia parzystego, ponadto znak współczynnika przy\linebreak najwyższej potędze x jest ujemny.\\ W związku z tym wykres wielomianu zaczyna się od lewej strony powyżej osi OX.\\
Ponadto w punkcie $-1$ wykres odbija się od osi poziomej.\\
A więc $$x \in \{-1\} \cup [3,4].$$
\rozwStop
\odpStart
$x \in \{-1\} \cup [3,4]$
\odpStop
\testStart
A.$x \in \{-1\} \cup [3,4]$\\
B.$x \in \{1\} \cup (3,4)$\\
C.$x \in \{-1\} \cup (3,4]$\\
D.$x \in \{1\} \cup (3,4]$\\
E.$x \in \{-1\} \cup [3,4)$\\
F.$x \in \{1\} \cup [3,4)$\\
G.$x \in \{-1\} \cup (3,4)$\\
H.$x \in \{1\} \cup [3,4]$
\testStop
\kluczStart
A
\kluczStop



\zadStart{Zadanie z Wikieł Z 1.62 c) moja wersja nr 20}

Rozwiązać nierówności $(3-x)(x+1)^{2}(5-x)^{3}\le0$.
\zadStop
\rozwStart{Patryk Wirkus}{}
Miejsca zerowe naszego wielomianu to: $3, -1, 5$.\\
Wielomian jest stopnia parzystego, ponadto znak współczynnika przy\linebreak najwyższej potędze x jest ujemny.\\ W związku z tym wykres wielomianu zaczyna się od lewej strony powyżej osi OX.\\
Ponadto w punkcie $-1$ wykres odbija się od osi poziomej.\\
A więc $$x \in \{-1\} \cup [3,5].$$
\rozwStop
\odpStart
$x \in \{-1\} \cup [3,5]$
\odpStop
\testStart
A.$x \in \{-1\} \cup [3,5]$\\
B.$x \in \{1\} \cup (3,5)$\\
C.$x \in \{-1\} \cup (3,5]$\\
D.$x \in \{1\} \cup (3,5]$\\
E.$x \in \{-1\} \cup [3,5)$\\
F.$x \in \{1\} \cup [3,5)$\\
G.$x \in \{-1\} \cup (3,5)$\\
H.$x \in \{1\} \cup [3,5]$
\testStop
\kluczStart
A
\kluczStop



\zadStart{Zadanie z Wikieł Z 1.62 c) moja wersja nr 21}

Rozwiązać nierówności $(3-x)(x+1)^{2}(6-x)^{3}\le0$.
\zadStop
\rozwStart{Patryk Wirkus}{}
Miejsca zerowe naszego wielomianu to: $3, -1, 6$.\\
Wielomian jest stopnia parzystego, ponadto znak współczynnika przy\linebreak najwyższej potędze x jest ujemny.\\ W związku z tym wykres wielomianu zaczyna się od lewej strony powyżej osi OX.\\
Ponadto w punkcie $-1$ wykres odbija się od osi poziomej.\\
A więc $$x \in \{-1\} \cup [3,6].$$
\rozwStop
\odpStart
$x \in \{-1\} \cup [3,6]$
\odpStop
\testStart
A.$x \in \{-1\} \cup [3,6]$\\
B.$x \in \{1\} \cup (3,6)$\\
C.$x \in \{-1\} \cup (3,6]$\\
D.$x \in \{1\} \cup (3,6]$\\
E.$x \in \{-1\} \cup [3,6)$\\
F.$x \in \{1\} \cup [3,6)$\\
G.$x \in \{-1\} \cup (3,6)$\\
H.$x \in \{1\} \cup [3,6]$
\testStop
\kluczStart
A
\kluczStop



\zadStart{Zadanie z Wikieł Z 1.62 c) moja wersja nr 22}

Rozwiązać nierówności $(3-x)(x+1)^{2}(7-x)^{3}\le0$.
\zadStop
\rozwStart{Patryk Wirkus}{}
Miejsca zerowe naszego wielomianu to: $3, -1, 7$.\\
Wielomian jest stopnia parzystego, ponadto znak współczynnika przy\linebreak najwyższej potędze x jest ujemny.\\ W związku z tym wykres wielomianu zaczyna się od lewej strony powyżej osi OX.\\
Ponadto w punkcie $-1$ wykres odbija się od osi poziomej.\\
A więc $$x \in \{-1\} \cup [3,7].$$
\rozwStop
\odpStart
$x \in \{-1\} \cup [3,7]$
\odpStop
\testStart
A.$x \in \{-1\} \cup [3,7]$\\
B.$x \in \{1\} \cup (3,7)$\\
C.$x \in \{-1\} \cup (3,7]$\\
D.$x \in \{1\} \cup (3,7]$\\
E.$x \in \{-1\} \cup [3,7)$\\
F.$x \in \{1\} \cup [3,7)$\\
G.$x \in \{-1\} \cup (3,7)$\\
H.$x \in \{1\} \cup [3,7]$
\testStop
\kluczStart
A
\kluczStop



\zadStart{Zadanie z Wikieł Z 1.62 c) moja wersja nr 23}

Rozwiązać nierówności $(3-x)(x+1)^{2}(8-x)^{3}\le0$.
\zadStop
\rozwStart{Patryk Wirkus}{}
Miejsca zerowe naszego wielomianu to: $3, -1, 8$.\\
Wielomian jest stopnia parzystego, ponadto znak współczynnika przy\linebreak najwyższej potędze x jest ujemny.\\ W związku z tym wykres wielomianu zaczyna się od lewej strony powyżej osi OX.\\
Ponadto w punkcie $-1$ wykres odbija się od osi poziomej.\\
A więc $$x \in \{-1\} \cup [3,8].$$
\rozwStop
\odpStart
$x \in \{-1\} \cup [3,8]$
\odpStop
\testStart
A.$x \in \{-1\} \cup [3,8]$\\
B.$x \in \{1\} \cup (3,8)$\\
C.$x \in \{-1\} \cup (3,8]$\\
D.$x \in \{1\} \cup (3,8]$\\
E.$x \in \{-1\} \cup [3,8)$\\
F.$x \in \{1\} \cup [3,8)$\\
G.$x \in \{-1\} \cup (3,8)$\\
H.$x \in \{1\} \cup [3,8]$
\testStop
\kluczStart
A
\kluczStop



\zadStart{Zadanie z Wikieł Z 1.62 c) moja wersja nr 24}

Rozwiązać nierówności $(3-x)(x+1)^{2}(9-x)^{3}\le0$.
\zadStop
\rozwStart{Patryk Wirkus}{}
Miejsca zerowe naszego wielomianu to: $3, -1, 9$.\\
Wielomian jest stopnia parzystego, ponadto znak współczynnika przy\linebreak najwyższej potędze x jest ujemny.\\ W związku z tym wykres wielomianu zaczyna się od lewej strony powyżej osi OX.\\
Ponadto w punkcie $-1$ wykres odbija się od osi poziomej.\\
A więc $$x \in \{-1\} \cup [3,9].$$
\rozwStop
\odpStart
$x \in \{-1\} \cup [3,9]$
\odpStop
\testStart
A.$x \in \{-1\} \cup [3,9]$\\
B.$x \in \{1\} \cup (3,9)$\\
C.$x \in \{-1\} \cup (3,9]$\\
D.$x \in \{1\} \cup (3,9]$\\
E.$x \in \{-1\} \cup [3,9)$\\
F.$x \in \{1\} \cup [3,9)$\\
G.$x \in \{-1\} \cup (3,9)$\\
H.$x \in \{1\} \cup [3,9]$
\testStop
\kluczStart
A
\kluczStop



\zadStart{Zadanie z Wikieł Z 1.62 c) moja wersja nr 25}

Rozwiązać nierówności $(3-x)(x+1)^{2}(10-x)^{3}\le0$.
\zadStop
\rozwStart{Patryk Wirkus}{}
Miejsca zerowe naszego wielomianu to: $3, -1, 10$.\\
Wielomian jest stopnia parzystego, ponadto znak współczynnika przy\linebreak najwyższej potędze x jest ujemny.\\ W związku z tym wykres wielomianu zaczyna się od lewej strony powyżej osi OX.\\
Ponadto w punkcie $-1$ wykres odbija się od osi poziomej.\\
A więc $$x \in \{-1\} \cup [3,10].$$
\rozwStop
\odpStart
$x \in \{-1\} \cup [3,10]$
\odpStop
\testStart
A.$x \in \{-1\} \cup [3,10]$\\
B.$x \in \{1\} \cup (3,10)$\\
C.$x \in \{-1\} \cup (3,10]$\\
D.$x \in \{1\} \cup (3,10]$\\
E.$x \in \{-1\} \cup [3,10)$\\
F.$x \in \{1\} \cup [3,10)$\\
G.$x \in \{-1\} \cup (3,10)$\\
H.$x \in \{1\} \cup [3,10]$
\testStop
\kluczStart
A
\kluczStop



\zadStart{Zadanie z Wikieł Z 1.62 c) moja wersja nr 26}

Rozwiązać nierówności $(3-x)(x+1)^{2}(11-x)^{3}\le0$.
\zadStop
\rozwStart{Patryk Wirkus}{}
Miejsca zerowe naszego wielomianu to: $3, -1, 11$.\\
Wielomian jest stopnia parzystego, ponadto znak współczynnika przy\linebreak najwyższej potędze x jest ujemny.\\ W związku z tym wykres wielomianu zaczyna się od lewej strony powyżej osi OX.\\
Ponadto w punkcie $-1$ wykres odbija się od osi poziomej.\\
A więc $$x \in \{-1\} \cup [3,11].$$
\rozwStop
\odpStart
$x \in \{-1\} \cup [3,11]$
\odpStop
\testStart
A.$x \in \{-1\} \cup [3,11]$\\
B.$x \in \{1\} \cup (3,11)$\\
C.$x \in \{-1\} \cup (3,11]$\\
D.$x \in \{1\} \cup (3,11]$\\
E.$x \in \{-1\} \cup [3,11)$\\
F.$x \in \{1\} \cup [3,11)$\\
G.$x \in \{-1\} \cup (3,11)$\\
H.$x \in \{1\} \cup [3,11]$
\testStop
\kluczStart
A
\kluczStop



\zadStart{Zadanie z Wikieł Z 1.62 c) moja wersja nr 27}

Rozwiązać nierówności $(3-x)(x+1)^{2}(12-x)^{3}\le0$.
\zadStop
\rozwStart{Patryk Wirkus}{}
Miejsca zerowe naszego wielomianu to: $3, -1, 12$.\\
Wielomian jest stopnia parzystego, ponadto znak współczynnika przy\linebreak najwyższej potędze x jest ujemny.\\ W związku z tym wykres wielomianu zaczyna się od lewej strony powyżej osi OX.\\
Ponadto w punkcie $-1$ wykres odbija się od osi poziomej.\\
A więc $$x \in \{-1\} \cup [3,12].$$
\rozwStop
\odpStart
$x \in \{-1\} \cup [3,12]$
\odpStop
\testStart
A.$x \in \{-1\} \cup [3,12]$\\
B.$x \in \{1\} \cup (3,12)$\\
C.$x \in \{-1\} \cup (3,12]$\\
D.$x \in \{1\} \cup (3,12]$\\
E.$x \in \{-1\} \cup [3,12)$\\
F.$x \in \{1\} \cup [3,12)$\\
G.$x \in \{-1\} \cup (3,12)$\\
H.$x \in \{1\} \cup [3,12]$
\testStop
\kluczStart
A
\kluczStop



\zadStart{Zadanie z Wikieł Z 1.62 c) moja wersja nr 28}

Rozwiązać nierówności $(3-x)(x+1)^{2}(13-x)^{3}\le0$.
\zadStop
\rozwStart{Patryk Wirkus}{}
Miejsca zerowe naszego wielomianu to: $3, -1, 13$.\\
Wielomian jest stopnia parzystego, ponadto znak współczynnika przy\linebreak najwyższej potędze x jest ujemny.\\ W związku z tym wykres wielomianu zaczyna się od lewej strony powyżej osi OX.\\
Ponadto w punkcie $-1$ wykres odbija się od osi poziomej.\\
A więc $$x \in \{-1\} \cup [3,13].$$
\rozwStop
\odpStart
$x \in \{-1\} \cup [3,13]$
\odpStop
\testStart
A.$x \in \{-1\} \cup [3,13]$\\
B.$x \in \{1\} \cup (3,13)$\\
C.$x \in \{-1\} \cup (3,13]$\\
D.$x \in \{1\} \cup (3,13]$\\
E.$x \in \{-1\} \cup [3,13)$\\
F.$x \in \{1\} \cup [3,13)$\\
G.$x \in \{-1\} \cup (3,13)$\\
H.$x \in \{1\} \cup [3,13]$
\testStop
\kluczStart
A
\kluczStop



\zadStart{Zadanie z Wikieł Z 1.62 c) moja wersja nr 29}

Rozwiązać nierówności $(3-x)(x+1)^{2}(14-x)^{3}\le0$.
\zadStop
\rozwStart{Patryk Wirkus}{}
Miejsca zerowe naszego wielomianu to: $3, -1, 14$.\\
Wielomian jest stopnia parzystego, ponadto znak współczynnika przy\linebreak najwyższej potędze x jest ujemny.\\ W związku z tym wykres wielomianu zaczyna się od lewej strony powyżej osi OX.\\
Ponadto w punkcie $-1$ wykres odbija się od osi poziomej.\\
A więc $$x \in \{-1\} \cup [3,14].$$
\rozwStop
\odpStart
$x \in \{-1\} \cup [3,14]$
\odpStop
\testStart
A.$x \in \{-1\} \cup [3,14]$\\
B.$x \in \{1\} \cup (3,14)$\\
C.$x \in \{-1\} \cup (3,14]$\\
D.$x \in \{1\} \cup (3,14]$\\
E.$x \in \{-1\} \cup [3,14)$\\
F.$x \in \{1\} \cup [3,14)$\\
G.$x \in \{-1\} \cup (3,14)$\\
H.$x \in \{1\} \cup [3,14]$
\testStop
\kluczStart
A
\kluczStop



\zadStart{Zadanie z Wikieł Z 1.62 c) moja wersja nr 30}

Rozwiązać nierówności $(3-x)(x+1)^{2}(15-x)^{3}\le0$.
\zadStop
\rozwStart{Patryk Wirkus}{}
Miejsca zerowe naszego wielomianu to: $3, -1, 15$.\\
Wielomian jest stopnia parzystego, ponadto znak współczynnika przy\linebreak najwyższej potędze x jest ujemny.\\ W związku z tym wykres wielomianu zaczyna się od lewej strony powyżej osi OX.\\
Ponadto w punkcie $-1$ wykres odbija się od osi poziomej.\\
A więc $$x \in \{-1\} \cup [3,15].$$
\rozwStop
\odpStart
$x \in \{-1\} \cup [3,15]$
\odpStop
\testStart
A.$x \in \{-1\} \cup [3,15]$\\
B.$x \in \{1\} \cup (3,15)$\\
C.$x \in \{-1\} \cup (3,15]$\\
D.$x \in \{1\} \cup (3,15]$\\
E.$x \in \{-1\} \cup [3,15)$\\
F.$x \in \{1\} \cup [3,15)$\\
G.$x \in \{-1\} \cup (3,15)$\\
H.$x \in \{1\} \cup [3,15]$
\testStop
\kluczStart
A
\kluczStop



\zadStart{Zadanie z Wikieł Z 1.62 c) moja wersja nr 31}

Rozwiązać nierówności $(3-x)(x+1)^{2}(16-x)^{3}\le0$.
\zadStop
\rozwStart{Patryk Wirkus}{}
Miejsca zerowe naszego wielomianu to: $3, -1, 16$.\\
Wielomian jest stopnia parzystego, ponadto znak współczynnika przy\linebreak najwyższej potędze x jest ujemny.\\ W związku z tym wykres wielomianu zaczyna się od lewej strony powyżej osi OX.\\
Ponadto w punkcie $-1$ wykres odbija się od osi poziomej.\\
A więc $$x \in \{-1\} \cup [3,16].$$
\rozwStop
\odpStart
$x \in \{-1\} \cup [3,16]$
\odpStop
\testStart
A.$x \in \{-1\} \cup [3,16]$\\
B.$x \in \{1\} \cup (3,16)$\\
C.$x \in \{-1\} \cup (3,16]$\\
D.$x \in \{1\} \cup (3,16]$\\
E.$x \in \{-1\} \cup [3,16)$\\
F.$x \in \{1\} \cup [3,16)$\\
G.$x \in \{-1\} \cup (3,16)$\\
H.$x \in \{1\} \cup [3,16]$
\testStop
\kluczStart
A
\kluczStop



\zadStart{Zadanie z Wikieł Z 1.62 c) moja wersja nr 32}

Rozwiązać nierówności $(3-x)(x+1)^{2}(17-x)^{3}\le0$.
\zadStop
\rozwStart{Patryk Wirkus}{}
Miejsca zerowe naszego wielomianu to: $3, -1, 17$.\\
Wielomian jest stopnia parzystego, ponadto znak współczynnika przy\linebreak najwyższej potędze x jest ujemny.\\ W związku z tym wykres wielomianu zaczyna się od lewej strony powyżej osi OX.\\
Ponadto w punkcie $-1$ wykres odbija się od osi poziomej.\\
A więc $$x \in \{-1\} \cup [3,17].$$
\rozwStop
\odpStart
$x \in \{-1\} \cup [3,17]$
\odpStop
\testStart
A.$x \in \{-1\} \cup [3,17]$\\
B.$x \in \{1\} \cup (3,17)$\\
C.$x \in \{-1\} \cup (3,17]$\\
D.$x \in \{1\} \cup (3,17]$\\
E.$x \in \{-1\} \cup [3,17)$\\
F.$x \in \{1\} \cup [3,17)$\\
G.$x \in \{-1\} \cup (3,17)$\\
H.$x \in \{1\} \cup [3,17]$
\testStop
\kluczStart
A
\kluczStop



\zadStart{Zadanie z Wikieł Z 1.62 c) moja wersja nr 33}

Rozwiązać nierówności $(3-x)(x+1)^{2}(18-x)^{3}\le0$.
\zadStop
\rozwStart{Patryk Wirkus}{}
Miejsca zerowe naszego wielomianu to: $3, -1, 18$.\\
Wielomian jest stopnia parzystego, ponadto znak współczynnika przy\linebreak najwyższej potędze x jest ujemny.\\ W związku z tym wykres wielomianu zaczyna się od lewej strony powyżej osi OX.\\
Ponadto w punkcie $-1$ wykres odbija się od osi poziomej.\\
A więc $$x \in \{-1\} \cup [3,18].$$
\rozwStop
\odpStart
$x \in \{-1\} \cup [3,18]$
\odpStop
\testStart
A.$x \in \{-1\} \cup [3,18]$\\
B.$x \in \{1\} \cup (3,18)$\\
C.$x \in \{-1\} \cup (3,18]$\\
D.$x \in \{1\} \cup (3,18]$\\
E.$x \in \{-1\} \cup [3,18)$\\
F.$x \in \{1\} \cup [3,18)$\\
G.$x \in \{-1\} \cup (3,18)$\\
H.$x \in \{1\} \cup [3,18]$
\testStop
\kluczStart
A
\kluczStop



\zadStart{Zadanie z Wikieł Z 1.62 c) moja wersja nr 34}

Rozwiązać nierówności $(3-x)(x+1)^{2}(19-x)^{3}\le0$.
\zadStop
\rozwStart{Patryk Wirkus}{}
Miejsca zerowe naszego wielomianu to: $3, -1, 19$.\\
Wielomian jest stopnia parzystego, ponadto znak współczynnika przy\linebreak najwyższej potędze x jest ujemny.\\ W związku z tym wykres wielomianu zaczyna się od lewej strony powyżej osi OX.\\
Ponadto w punkcie $-1$ wykres odbija się od osi poziomej.\\
A więc $$x \in \{-1\} \cup [3,19].$$
\rozwStop
\odpStart
$x \in \{-1\} \cup [3,19]$
\odpStop
\testStart
A.$x \in \{-1\} \cup [3,19]$\\
B.$x \in \{1\} \cup (3,19)$\\
C.$x \in \{-1\} \cup (3,19]$\\
D.$x \in \{1\} \cup (3,19]$\\
E.$x \in \{-1\} \cup [3,19)$\\
F.$x \in \{1\} \cup [3,19)$\\
G.$x \in \{-1\} \cup (3,19)$\\
H.$x \in \{1\} \cup [3,19]$
\testStop
\kluczStart
A
\kluczStop



\zadStart{Zadanie z Wikieł Z 1.62 c) moja wersja nr 35}

Rozwiązać nierówności $(3-x)(x+1)^{2}(20-x)^{3}\le0$.
\zadStop
\rozwStart{Patryk Wirkus}{}
Miejsca zerowe naszego wielomianu to: $3, -1, 20$.\\
Wielomian jest stopnia parzystego, ponadto znak współczynnika przy\linebreak najwyższej potędze x jest ujemny.\\ W związku z tym wykres wielomianu zaczyna się od lewej strony powyżej osi OX.\\
Ponadto w punkcie $-1$ wykres odbija się od osi poziomej.\\
A więc $$x \in \{-1\} \cup [3,20].$$
\rozwStop
\odpStart
$x \in \{-1\} \cup [3,20]$
\odpStop
\testStart
A.$x \in \{-1\} \cup [3,20]$\\
B.$x \in \{1\} \cup (3,20)$\\
C.$x \in \{-1\} \cup (3,20]$\\
D.$x \in \{1\} \cup (3,20]$\\
E.$x \in \{-1\} \cup [3,20)$\\
F.$x \in \{1\} \cup [3,20)$\\
G.$x \in \{-1\} \cup (3,20)$\\
H.$x \in \{1\} \cup [3,20]$
\testStop
\kluczStart
A
\kluczStop



\zadStart{Zadanie z Wikieł Z 1.62 c) moja wersja nr 36}

Rozwiązać nierówności $(3-x)(x+2)^{2}(4-x)^{3}\le0$.
\zadStop
\rozwStart{Patryk Wirkus}{}
Miejsca zerowe naszego wielomianu to: $3, -2, 4$.\\
Wielomian jest stopnia parzystego, ponadto znak współczynnika przy\linebreak najwyższej potędze x jest ujemny.\\ W związku z tym wykres wielomianu zaczyna się od lewej strony powyżej osi OX.\\
Ponadto w punkcie $-2$ wykres odbija się od osi poziomej.\\
A więc $$x \in \{-2\} \cup [3,4].$$
\rozwStop
\odpStart
$x \in \{-2\} \cup [3,4]$
\odpStop
\testStart
A.$x \in \{-2\} \cup [3,4]$\\
B.$x \in \{2\} \cup (3,4)$\\
C.$x \in \{-2\} \cup (3,4]$\\
D.$x \in \{2\} \cup (3,4]$\\
E.$x \in \{-2\} \cup [3,4)$\\
F.$x \in \{2\} \cup [3,4)$\\
G.$x \in \{-2\} \cup (3,4)$\\
H.$x \in \{2\} \cup [3,4]$
\testStop
\kluczStart
A
\kluczStop



\zadStart{Zadanie z Wikieł Z 1.62 c) moja wersja nr 37}

Rozwiązać nierówności $(3-x)(x+2)^{2}(5-x)^{3}\le0$.
\zadStop
\rozwStart{Patryk Wirkus}{}
Miejsca zerowe naszego wielomianu to: $3, -2, 5$.\\
Wielomian jest stopnia parzystego, ponadto znak współczynnika przy\linebreak najwyższej potędze x jest ujemny.\\ W związku z tym wykres wielomianu zaczyna się od lewej strony powyżej osi OX.\\
Ponadto w punkcie $-2$ wykres odbija się od osi poziomej.\\
A więc $$x \in \{-2\} \cup [3,5].$$
\rozwStop
\odpStart
$x \in \{-2\} \cup [3,5]$
\odpStop
\testStart
A.$x \in \{-2\} \cup [3,5]$\\
B.$x \in \{2\} \cup (3,5)$\\
C.$x \in \{-2\} \cup (3,5]$\\
D.$x \in \{2\} \cup (3,5]$\\
E.$x \in \{-2\} \cup [3,5)$\\
F.$x \in \{2\} \cup [3,5)$\\
G.$x \in \{-2\} \cup (3,5)$\\
H.$x \in \{2\} \cup [3,5]$
\testStop
\kluczStart
A
\kluczStop



\zadStart{Zadanie z Wikieł Z 1.62 c) moja wersja nr 38}

Rozwiązać nierówności $(3-x)(x+2)^{2}(6-x)^{3}\le0$.
\zadStop
\rozwStart{Patryk Wirkus}{}
Miejsca zerowe naszego wielomianu to: $3, -2, 6$.\\
Wielomian jest stopnia parzystego, ponadto znak współczynnika przy\linebreak najwyższej potędze x jest ujemny.\\ W związku z tym wykres wielomianu zaczyna się od lewej strony powyżej osi OX.\\
Ponadto w punkcie $-2$ wykres odbija się od osi poziomej.\\
A więc $$x \in \{-2\} \cup [3,6].$$
\rozwStop
\odpStart
$x \in \{-2\} \cup [3,6]$
\odpStop
\testStart
A.$x \in \{-2\} \cup [3,6]$\\
B.$x \in \{2\} \cup (3,6)$\\
C.$x \in \{-2\} \cup (3,6]$\\
D.$x \in \{2\} \cup (3,6]$\\
E.$x \in \{-2\} \cup [3,6)$\\
F.$x \in \{2\} \cup [3,6)$\\
G.$x \in \{-2\} \cup (3,6)$\\
H.$x \in \{2\} \cup [3,6]$
\testStop
\kluczStart
A
\kluczStop



\zadStart{Zadanie z Wikieł Z 1.62 c) moja wersja nr 39}

Rozwiązać nierówności $(3-x)(x+2)^{2}(7-x)^{3}\le0$.
\zadStop
\rozwStart{Patryk Wirkus}{}
Miejsca zerowe naszego wielomianu to: $3, -2, 7$.\\
Wielomian jest stopnia parzystego, ponadto znak współczynnika przy\linebreak najwyższej potędze x jest ujemny.\\ W związku z tym wykres wielomianu zaczyna się od lewej strony powyżej osi OX.\\
Ponadto w punkcie $-2$ wykres odbija się od osi poziomej.\\
A więc $$x \in \{-2\} \cup [3,7].$$
\rozwStop
\odpStart
$x \in \{-2\} \cup [3,7]$
\odpStop
\testStart
A.$x \in \{-2\} \cup [3,7]$\\
B.$x \in \{2\} \cup (3,7)$\\
C.$x \in \{-2\} \cup (3,7]$\\
D.$x \in \{2\} \cup (3,7]$\\
E.$x \in \{-2\} \cup [3,7)$\\
F.$x \in \{2\} \cup [3,7)$\\
G.$x \in \{-2\} \cup (3,7)$\\
H.$x \in \{2\} \cup [3,7]$
\testStop
\kluczStart
A
\kluczStop



\zadStart{Zadanie z Wikieł Z 1.62 c) moja wersja nr 40}

Rozwiązać nierówności $(3-x)(x+2)^{2}(8-x)^{3}\le0$.
\zadStop
\rozwStart{Patryk Wirkus}{}
Miejsca zerowe naszego wielomianu to: $3, -2, 8$.\\
Wielomian jest stopnia parzystego, ponadto znak współczynnika przy\linebreak najwyższej potędze x jest ujemny.\\ W związku z tym wykres wielomianu zaczyna się od lewej strony powyżej osi OX.\\
Ponadto w punkcie $-2$ wykres odbija się od osi poziomej.\\
A więc $$x \in \{-2\} \cup [3,8].$$
\rozwStop
\odpStart
$x \in \{-2\} \cup [3,8]$
\odpStop
\testStart
A.$x \in \{-2\} \cup [3,8]$\\
B.$x \in \{2\} \cup (3,8)$\\
C.$x \in \{-2\} \cup (3,8]$\\
D.$x \in \{2\} \cup (3,8]$\\
E.$x \in \{-2\} \cup [3,8)$\\
F.$x \in \{2\} \cup [3,8)$\\
G.$x \in \{-2\} \cup (3,8)$\\
H.$x \in \{2\} \cup [3,8]$
\testStop
\kluczStart
A
\kluczStop



\zadStart{Zadanie z Wikieł Z 1.62 c) moja wersja nr 41}

Rozwiązać nierówności $(3-x)(x+2)^{2}(9-x)^{3}\le0$.
\zadStop
\rozwStart{Patryk Wirkus}{}
Miejsca zerowe naszego wielomianu to: $3, -2, 9$.\\
Wielomian jest stopnia parzystego, ponadto znak współczynnika przy\linebreak najwyższej potędze x jest ujemny.\\ W związku z tym wykres wielomianu zaczyna się od lewej strony powyżej osi OX.\\
Ponadto w punkcie $-2$ wykres odbija się od osi poziomej.\\
A więc $$x \in \{-2\} \cup [3,9].$$
\rozwStop
\odpStart
$x \in \{-2\} \cup [3,9]$
\odpStop
\testStart
A.$x \in \{-2\} \cup [3,9]$\\
B.$x \in \{2\} \cup (3,9)$\\
C.$x \in \{-2\} \cup (3,9]$\\
D.$x \in \{2\} \cup (3,9]$\\
E.$x \in \{-2\} \cup [3,9)$\\
F.$x \in \{2\} \cup [3,9)$\\
G.$x \in \{-2\} \cup (3,9)$\\
H.$x \in \{2\} \cup [3,9]$
\testStop
\kluczStart
A
\kluczStop



\zadStart{Zadanie z Wikieł Z 1.62 c) moja wersja nr 42}

Rozwiązać nierówności $(3-x)(x+2)^{2}(10-x)^{3}\le0$.
\zadStop
\rozwStart{Patryk Wirkus}{}
Miejsca zerowe naszego wielomianu to: $3, -2, 10$.\\
Wielomian jest stopnia parzystego, ponadto znak współczynnika przy\linebreak najwyższej potędze x jest ujemny.\\ W związku z tym wykres wielomianu zaczyna się od lewej strony powyżej osi OX.\\
Ponadto w punkcie $-2$ wykres odbija się od osi poziomej.\\
A więc $$x \in \{-2\} \cup [3,10].$$
\rozwStop
\odpStart
$x \in \{-2\} \cup [3,10]$
\odpStop
\testStart
A.$x \in \{-2\} \cup [3,10]$\\
B.$x \in \{2\} \cup (3,10)$\\
C.$x \in \{-2\} \cup (3,10]$\\
D.$x \in \{2\} \cup (3,10]$\\
E.$x \in \{-2\} \cup [3,10)$\\
F.$x \in \{2\} \cup [3,10)$\\
G.$x \in \{-2\} \cup (3,10)$\\
H.$x \in \{2\} \cup [3,10]$
\testStop
\kluczStart
A
\kluczStop



\zadStart{Zadanie z Wikieł Z 1.62 c) moja wersja nr 43}

Rozwiązać nierówności $(3-x)(x+2)^{2}(11-x)^{3}\le0$.
\zadStop
\rozwStart{Patryk Wirkus}{}
Miejsca zerowe naszego wielomianu to: $3, -2, 11$.\\
Wielomian jest stopnia parzystego, ponadto znak współczynnika przy\linebreak najwyższej potędze x jest ujemny.\\ W związku z tym wykres wielomianu zaczyna się od lewej strony powyżej osi OX.\\
Ponadto w punkcie $-2$ wykres odbija się od osi poziomej.\\
A więc $$x \in \{-2\} \cup [3,11].$$
\rozwStop
\odpStart
$x \in \{-2\} \cup [3,11]$
\odpStop
\testStart
A.$x \in \{-2\} \cup [3,11]$\\
B.$x \in \{2\} \cup (3,11)$\\
C.$x \in \{-2\} \cup (3,11]$\\
D.$x \in \{2\} \cup (3,11]$\\
E.$x \in \{-2\} \cup [3,11)$\\
F.$x \in \{2\} \cup [3,11)$\\
G.$x \in \{-2\} \cup (3,11)$\\
H.$x \in \{2\} \cup [3,11]$
\testStop
\kluczStart
A
\kluczStop



\zadStart{Zadanie z Wikieł Z 1.62 c) moja wersja nr 44}

Rozwiązać nierówności $(3-x)(x+2)^{2}(12-x)^{3}\le0$.
\zadStop
\rozwStart{Patryk Wirkus}{}
Miejsca zerowe naszego wielomianu to: $3, -2, 12$.\\
Wielomian jest stopnia parzystego, ponadto znak współczynnika przy\linebreak najwyższej potędze x jest ujemny.\\ W związku z tym wykres wielomianu zaczyna się od lewej strony powyżej osi OX.\\
Ponadto w punkcie $-2$ wykres odbija się od osi poziomej.\\
A więc $$x \in \{-2\} \cup [3,12].$$
\rozwStop
\odpStart
$x \in \{-2\} \cup [3,12]$
\odpStop
\testStart
A.$x \in \{-2\} \cup [3,12]$\\
B.$x \in \{2\} \cup (3,12)$\\
C.$x \in \{-2\} \cup (3,12]$\\
D.$x \in \{2\} \cup (3,12]$\\
E.$x \in \{-2\} \cup [3,12)$\\
F.$x \in \{2\} \cup [3,12)$\\
G.$x \in \{-2\} \cup (3,12)$\\
H.$x \in \{2\} \cup [3,12]$
\testStop
\kluczStart
A
\kluczStop



\zadStart{Zadanie z Wikieł Z 1.62 c) moja wersja nr 45}

Rozwiązać nierówności $(3-x)(x+2)^{2}(13-x)^{3}\le0$.
\zadStop
\rozwStart{Patryk Wirkus}{}
Miejsca zerowe naszego wielomianu to: $3, -2, 13$.\\
Wielomian jest stopnia parzystego, ponadto znak współczynnika przy\linebreak najwyższej potędze x jest ujemny.\\ W związku z tym wykres wielomianu zaczyna się od lewej strony powyżej osi OX.\\
Ponadto w punkcie $-2$ wykres odbija się od osi poziomej.\\
A więc $$x \in \{-2\} \cup [3,13].$$
\rozwStop
\odpStart
$x \in \{-2\} \cup [3,13]$
\odpStop
\testStart
A.$x \in \{-2\} \cup [3,13]$\\
B.$x \in \{2\} \cup (3,13)$\\
C.$x \in \{-2\} \cup (3,13]$\\
D.$x \in \{2\} \cup (3,13]$\\
E.$x \in \{-2\} \cup [3,13)$\\
F.$x \in \{2\} \cup [3,13)$\\
G.$x \in \{-2\} \cup (3,13)$\\
H.$x \in \{2\} \cup [3,13]$
\testStop
\kluczStart
A
\kluczStop



\zadStart{Zadanie z Wikieł Z 1.62 c) moja wersja nr 46}

Rozwiązać nierówności $(3-x)(x+2)^{2}(14-x)^{3}\le0$.
\zadStop
\rozwStart{Patryk Wirkus}{}
Miejsca zerowe naszego wielomianu to: $3, -2, 14$.\\
Wielomian jest stopnia parzystego, ponadto znak współczynnika przy\linebreak najwyższej potędze x jest ujemny.\\ W związku z tym wykres wielomianu zaczyna się od lewej strony powyżej osi OX.\\
Ponadto w punkcie $-2$ wykres odbija się od osi poziomej.\\
A więc $$x \in \{-2\} \cup [3,14].$$
\rozwStop
\odpStart
$x \in \{-2\} \cup [3,14]$
\odpStop
\testStart
A.$x \in \{-2\} \cup [3,14]$\\
B.$x \in \{2\} \cup (3,14)$\\
C.$x \in \{-2\} \cup (3,14]$\\
D.$x \in \{2\} \cup (3,14]$\\
E.$x \in \{-2\} \cup [3,14)$\\
F.$x \in \{2\} \cup [3,14)$\\
G.$x \in \{-2\} \cup (3,14)$\\
H.$x \in \{2\} \cup [3,14]$
\testStop
\kluczStart
A
\kluczStop



\zadStart{Zadanie z Wikieł Z 1.62 c) moja wersja nr 47}

Rozwiązać nierówności $(3-x)(x+2)^{2}(15-x)^{3}\le0$.
\zadStop
\rozwStart{Patryk Wirkus}{}
Miejsca zerowe naszego wielomianu to: $3, -2, 15$.\\
Wielomian jest stopnia parzystego, ponadto znak współczynnika przy\linebreak najwyższej potędze x jest ujemny.\\ W związku z tym wykres wielomianu zaczyna się od lewej strony powyżej osi OX.\\
Ponadto w punkcie $-2$ wykres odbija się od osi poziomej.\\
A więc $$x \in \{-2\} \cup [3,15].$$
\rozwStop
\odpStart
$x \in \{-2\} \cup [3,15]$
\odpStop
\testStart
A.$x \in \{-2\} \cup [3,15]$\\
B.$x \in \{2\} \cup (3,15)$\\
C.$x \in \{-2\} \cup (3,15]$\\
D.$x \in \{2\} \cup (3,15]$\\
E.$x \in \{-2\} \cup [3,15)$\\
F.$x \in \{2\} \cup [3,15)$\\
G.$x \in \{-2\} \cup (3,15)$\\
H.$x \in \{2\} \cup [3,15]$
\testStop
\kluczStart
A
\kluczStop



\zadStart{Zadanie z Wikieł Z 1.62 c) moja wersja nr 48}

Rozwiązać nierówności $(3-x)(x+2)^{2}(16-x)^{3}\le0$.
\zadStop
\rozwStart{Patryk Wirkus}{}
Miejsca zerowe naszego wielomianu to: $3, -2, 16$.\\
Wielomian jest stopnia parzystego, ponadto znak współczynnika przy\linebreak najwyższej potędze x jest ujemny.\\ W związku z tym wykres wielomianu zaczyna się od lewej strony powyżej osi OX.\\
Ponadto w punkcie $-2$ wykres odbija się od osi poziomej.\\
A więc $$x \in \{-2\} \cup [3,16].$$
\rozwStop
\odpStart
$x \in \{-2\} \cup [3,16]$
\odpStop
\testStart
A.$x \in \{-2\} \cup [3,16]$\\
B.$x \in \{2\} \cup (3,16)$\\
C.$x \in \{-2\} \cup (3,16]$\\
D.$x \in \{2\} \cup (3,16]$\\
E.$x \in \{-2\} \cup [3,16)$\\
F.$x \in \{2\} \cup [3,16)$\\
G.$x \in \{-2\} \cup (3,16)$\\
H.$x \in \{2\} \cup [3,16]$
\testStop
\kluczStart
A
\kluczStop



\zadStart{Zadanie z Wikieł Z 1.62 c) moja wersja nr 49}

Rozwiązać nierówności $(3-x)(x+2)^{2}(17-x)^{3}\le0$.
\zadStop
\rozwStart{Patryk Wirkus}{}
Miejsca zerowe naszego wielomianu to: $3, -2, 17$.\\
Wielomian jest stopnia parzystego, ponadto znak współczynnika przy\linebreak najwyższej potędze x jest ujemny.\\ W związku z tym wykres wielomianu zaczyna się od lewej strony powyżej osi OX.\\
Ponadto w punkcie $-2$ wykres odbija się od osi poziomej.\\
A więc $$x \in \{-2\} \cup [3,17].$$
\rozwStop
\odpStart
$x \in \{-2\} \cup [3,17]$
\odpStop
\testStart
A.$x \in \{-2\} \cup [3,17]$\\
B.$x \in \{2\} \cup (3,17)$\\
C.$x \in \{-2\} \cup (3,17]$\\
D.$x \in \{2\} \cup (3,17]$\\
E.$x \in \{-2\} \cup [3,17)$\\
F.$x \in \{2\} \cup [3,17)$\\
G.$x \in \{-2\} \cup (3,17)$\\
H.$x \in \{2\} \cup [3,17]$
\testStop
\kluczStart
A
\kluczStop



\zadStart{Zadanie z Wikieł Z 1.62 c) moja wersja nr 50}

Rozwiązać nierówności $(3-x)(x+2)^{2}(18-x)^{3}\le0$.
\zadStop
\rozwStart{Patryk Wirkus}{}
Miejsca zerowe naszego wielomianu to: $3, -2, 18$.\\
Wielomian jest stopnia parzystego, ponadto znak współczynnika przy\linebreak najwyższej potędze x jest ujemny.\\ W związku z tym wykres wielomianu zaczyna się od lewej strony powyżej osi OX.\\
Ponadto w punkcie $-2$ wykres odbija się od osi poziomej.\\
A więc $$x \in \{-2\} \cup [3,18].$$
\rozwStop
\odpStart
$x \in \{-2\} \cup [3,18]$
\odpStop
\testStart
A.$x \in \{-2\} \cup [3,18]$\\
B.$x \in \{2\} \cup (3,18)$\\
C.$x \in \{-2\} \cup (3,18]$\\
D.$x \in \{2\} \cup (3,18]$\\
E.$x \in \{-2\} \cup [3,18)$\\
F.$x \in \{2\} \cup [3,18)$\\
G.$x \in \{-2\} \cup (3,18)$\\
H.$x \in \{2\} \cup [3,18]$
\testStop
\kluczStart
A
\kluczStop



\zadStart{Zadanie z Wikieł Z 1.62 c) moja wersja nr 51}

Rozwiązać nierówności $(3-x)(x+2)^{2}(19-x)^{3}\le0$.
\zadStop
\rozwStart{Patryk Wirkus}{}
Miejsca zerowe naszego wielomianu to: $3, -2, 19$.\\
Wielomian jest stopnia parzystego, ponadto znak współczynnika przy\linebreak najwyższej potędze x jest ujemny.\\ W związku z tym wykres wielomianu zaczyna się od lewej strony powyżej osi OX.\\
Ponadto w punkcie $-2$ wykres odbija się od osi poziomej.\\
A więc $$x \in \{-2\} \cup [3,19].$$
\rozwStop
\odpStart
$x \in \{-2\} \cup [3,19]$
\odpStop
\testStart
A.$x \in \{-2\} \cup [3,19]$\\
B.$x \in \{2\} \cup (3,19)$\\
C.$x \in \{-2\} \cup (3,19]$\\
D.$x \in \{2\} \cup (3,19]$\\
E.$x \in \{-2\} \cup [3,19)$\\
F.$x \in \{2\} \cup [3,19)$\\
G.$x \in \{-2\} \cup (3,19)$\\
H.$x \in \{2\} \cup [3,19]$
\testStop
\kluczStart
A
\kluczStop



\zadStart{Zadanie z Wikieł Z 1.62 c) moja wersja nr 52}

Rozwiązać nierówności $(3-x)(x+2)^{2}(20-x)^{3}\le0$.
\zadStop
\rozwStart{Patryk Wirkus}{}
Miejsca zerowe naszego wielomianu to: $3, -2, 20$.\\
Wielomian jest stopnia parzystego, ponadto znak współczynnika przy\linebreak najwyższej potędze x jest ujemny.\\ W związku z tym wykres wielomianu zaczyna się od lewej strony powyżej osi OX.\\
Ponadto w punkcie $-2$ wykres odbija się od osi poziomej.\\
A więc $$x \in \{-2\} \cup [3,20].$$
\rozwStop
\odpStart
$x \in \{-2\} \cup [3,20]$
\odpStop
\testStart
A.$x \in \{-2\} \cup [3,20]$\\
B.$x \in \{2\} \cup (3,20)$\\
C.$x \in \{-2\} \cup (3,20]$\\
D.$x \in \{2\} \cup (3,20]$\\
E.$x \in \{-2\} \cup [3,20)$\\
F.$x \in \{2\} \cup [3,20)$\\
G.$x \in \{-2\} \cup (3,20)$\\
H.$x \in \{2\} \cup [3,20]$
\testStop
\kluczStart
A
\kluczStop



\zadStart{Zadanie z Wikieł Z 1.62 c) moja wersja nr 53}

Rozwiązać nierówności $(4-x)(x+1)^{2}(5-x)^{3}\le0$.
\zadStop
\rozwStart{Patryk Wirkus}{}
Miejsca zerowe naszego wielomianu to: $4, -1, 5$.\\
Wielomian jest stopnia parzystego, ponadto znak współczynnika przy\linebreak najwyższej potędze x jest ujemny.\\ W związku z tym wykres wielomianu zaczyna się od lewej strony powyżej osi OX.\\
Ponadto w punkcie $-1$ wykres odbija się od osi poziomej.\\
A więc $$x \in \{-1\} \cup [4,5].$$
\rozwStop
\odpStart
$x \in \{-1\} \cup [4,5]$
\odpStop
\testStart
A.$x \in \{-1\} \cup [4,5]$\\
B.$x \in \{1\} \cup (4,5)$\\
C.$x \in \{-1\} \cup (4,5]$\\
D.$x \in \{1\} \cup (4,5]$\\
E.$x \in \{-1\} \cup [4,5)$\\
F.$x \in \{1\} \cup [4,5)$\\
G.$x \in \{-1\} \cup (4,5)$\\
H.$x \in \{1\} \cup [4,5]$
\testStop
\kluczStart
A
\kluczStop



\zadStart{Zadanie z Wikieł Z 1.62 c) moja wersja nr 54}

Rozwiązać nierówności $(4-x)(x+1)^{2}(6-x)^{3}\le0$.
\zadStop
\rozwStart{Patryk Wirkus}{}
Miejsca zerowe naszego wielomianu to: $4, -1, 6$.\\
Wielomian jest stopnia parzystego, ponadto znak współczynnika przy\linebreak najwyższej potędze x jest ujemny.\\ W związku z tym wykres wielomianu zaczyna się od lewej strony powyżej osi OX.\\
Ponadto w punkcie $-1$ wykres odbija się od osi poziomej.\\
A więc $$x \in \{-1\} \cup [4,6].$$
\rozwStop
\odpStart
$x \in \{-1\} \cup [4,6]$
\odpStop
\testStart
A.$x \in \{-1\} \cup [4,6]$\\
B.$x \in \{1\} \cup (4,6)$\\
C.$x \in \{-1\} \cup (4,6]$\\
D.$x \in \{1\} \cup (4,6]$\\
E.$x \in \{-1\} \cup [4,6)$\\
F.$x \in \{1\} \cup [4,6)$\\
G.$x \in \{-1\} \cup (4,6)$\\
H.$x \in \{1\} \cup [4,6]$
\testStop
\kluczStart
A
\kluczStop



\zadStart{Zadanie z Wikieł Z 1.62 c) moja wersja nr 55}

Rozwiązać nierówności $(4-x)(x+1)^{2}(7-x)^{3}\le0$.
\zadStop
\rozwStart{Patryk Wirkus}{}
Miejsca zerowe naszego wielomianu to: $4, -1, 7$.\\
Wielomian jest stopnia parzystego, ponadto znak współczynnika przy\linebreak najwyższej potędze x jest ujemny.\\ W związku z tym wykres wielomianu zaczyna się od lewej strony powyżej osi OX.\\
Ponadto w punkcie $-1$ wykres odbija się od osi poziomej.\\
A więc $$x \in \{-1\} \cup [4,7].$$
\rozwStop
\odpStart
$x \in \{-1\} \cup [4,7]$
\odpStop
\testStart
A.$x \in \{-1\} \cup [4,7]$\\
B.$x \in \{1\} \cup (4,7)$\\
C.$x \in \{-1\} \cup (4,7]$\\
D.$x \in \{1\} \cup (4,7]$\\
E.$x \in \{-1\} \cup [4,7)$\\
F.$x \in \{1\} \cup [4,7)$\\
G.$x \in \{-1\} \cup (4,7)$\\
H.$x \in \{1\} \cup [4,7]$
\testStop
\kluczStart
A
\kluczStop



\zadStart{Zadanie z Wikieł Z 1.62 c) moja wersja nr 56}

Rozwiązać nierówności $(4-x)(x+1)^{2}(8-x)^{3}\le0$.
\zadStop
\rozwStart{Patryk Wirkus}{}
Miejsca zerowe naszego wielomianu to: $4, -1, 8$.\\
Wielomian jest stopnia parzystego, ponadto znak współczynnika przy\linebreak najwyższej potędze x jest ujemny.\\ W związku z tym wykres wielomianu zaczyna się od lewej strony powyżej osi OX.\\
Ponadto w punkcie $-1$ wykres odbija się od osi poziomej.\\
A więc $$x \in \{-1\} \cup [4,8].$$
\rozwStop
\odpStart
$x \in \{-1\} \cup [4,8]$
\odpStop
\testStart
A.$x \in \{-1\} \cup [4,8]$\\
B.$x \in \{1\} \cup (4,8)$\\
C.$x \in \{-1\} \cup (4,8]$\\
D.$x \in \{1\} \cup (4,8]$\\
E.$x \in \{-1\} \cup [4,8)$\\
F.$x \in \{1\} \cup [4,8)$\\
G.$x \in \{-1\} \cup (4,8)$\\
H.$x \in \{1\} \cup [4,8]$
\testStop
\kluczStart
A
\kluczStop



\zadStart{Zadanie z Wikieł Z 1.62 c) moja wersja nr 57}

Rozwiązać nierówności $(4-x)(x+1)^{2}(9-x)^{3}\le0$.
\zadStop
\rozwStart{Patryk Wirkus}{}
Miejsca zerowe naszego wielomianu to: $4, -1, 9$.\\
Wielomian jest stopnia parzystego, ponadto znak współczynnika przy\linebreak najwyższej potędze x jest ujemny.\\ W związku z tym wykres wielomianu zaczyna się od lewej strony powyżej osi OX.\\
Ponadto w punkcie $-1$ wykres odbija się od osi poziomej.\\
A więc $$x \in \{-1\} \cup [4,9].$$
\rozwStop
\odpStart
$x \in \{-1\} \cup [4,9]$
\odpStop
\testStart
A.$x \in \{-1\} \cup [4,9]$\\
B.$x \in \{1\} \cup (4,9)$\\
C.$x \in \{-1\} \cup (4,9]$\\
D.$x \in \{1\} \cup (4,9]$\\
E.$x \in \{-1\} \cup [4,9)$\\
F.$x \in \{1\} \cup [4,9)$\\
G.$x \in \{-1\} \cup (4,9)$\\
H.$x \in \{1\} \cup [4,9]$
\testStop
\kluczStart
A
\kluczStop



\zadStart{Zadanie z Wikieł Z 1.62 c) moja wersja nr 58}

Rozwiązać nierówności $(4-x)(x+1)^{2}(10-x)^{3}\le0$.
\zadStop
\rozwStart{Patryk Wirkus}{}
Miejsca zerowe naszego wielomianu to: $4, -1, 10$.\\
Wielomian jest stopnia parzystego, ponadto znak współczynnika przy\linebreak najwyższej potędze x jest ujemny.\\ W związku z tym wykres wielomianu zaczyna się od lewej strony powyżej osi OX.\\
Ponadto w punkcie $-1$ wykres odbija się od osi poziomej.\\
A więc $$x \in \{-1\} \cup [4,10].$$
\rozwStop
\odpStart
$x \in \{-1\} \cup [4,10]$
\odpStop
\testStart
A.$x \in \{-1\} \cup [4,10]$\\
B.$x \in \{1\} \cup (4,10)$\\
C.$x \in \{-1\} \cup (4,10]$\\
D.$x \in \{1\} \cup (4,10]$\\
E.$x \in \{-1\} \cup [4,10)$\\
F.$x \in \{1\} \cup [4,10)$\\
G.$x \in \{-1\} \cup (4,10)$\\
H.$x \in \{1\} \cup [4,10]$
\testStop
\kluczStart
A
\kluczStop



\zadStart{Zadanie z Wikieł Z 1.62 c) moja wersja nr 59}

Rozwiązać nierówności $(4-x)(x+1)^{2}(11-x)^{3}\le0$.
\zadStop
\rozwStart{Patryk Wirkus}{}
Miejsca zerowe naszego wielomianu to: $4, -1, 11$.\\
Wielomian jest stopnia parzystego, ponadto znak współczynnika przy\linebreak najwyższej potędze x jest ujemny.\\ W związku z tym wykres wielomianu zaczyna się od lewej strony powyżej osi OX.\\
Ponadto w punkcie $-1$ wykres odbija się od osi poziomej.\\
A więc $$x \in \{-1\} \cup [4,11].$$
\rozwStop
\odpStart
$x \in \{-1\} \cup [4,11]$
\odpStop
\testStart
A.$x \in \{-1\} \cup [4,11]$\\
B.$x \in \{1\} \cup (4,11)$\\
C.$x \in \{-1\} \cup (4,11]$\\
D.$x \in \{1\} \cup (4,11]$\\
E.$x \in \{-1\} \cup [4,11)$\\
F.$x \in \{1\} \cup [4,11)$\\
G.$x \in \{-1\} \cup (4,11)$\\
H.$x \in \{1\} \cup [4,11]$
\testStop
\kluczStart
A
\kluczStop



\zadStart{Zadanie z Wikieł Z 1.62 c) moja wersja nr 60}

Rozwiązać nierówności $(4-x)(x+1)^{2}(12-x)^{3}\le0$.
\zadStop
\rozwStart{Patryk Wirkus}{}
Miejsca zerowe naszego wielomianu to: $4, -1, 12$.\\
Wielomian jest stopnia parzystego, ponadto znak współczynnika przy\linebreak najwyższej potędze x jest ujemny.\\ W związku z tym wykres wielomianu zaczyna się od lewej strony powyżej osi OX.\\
Ponadto w punkcie $-1$ wykres odbija się od osi poziomej.\\
A więc $$x \in \{-1\} \cup [4,12].$$
\rozwStop
\odpStart
$x \in \{-1\} \cup [4,12]$
\odpStop
\testStart
A.$x \in \{-1\} \cup [4,12]$\\
B.$x \in \{1\} \cup (4,12)$\\
C.$x \in \{-1\} \cup (4,12]$\\
D.$x \in \{1\} \cup (4,12]$\\
E.$x \in \{-1\} \cup [4,12)$\\
F.$x \in \{1\} \cup [4,12)$\\
G.$x \in \{-1\} \cup (4,12)$\\
H.$x \in \{1\} \cup [4,12]$
\testStop
\kluczStart
A
\kluczStop



\zadStart{Zadanie z Wikieł Z 1.62 c) moja wersja nr 61}

Rozwiązać nierówności $(4-x)(x+1)^{2}(13-x)^{3}\le0$.
\zadStop
\rozwStart{Patryk Wirkus}{}
Miejsca zerowe naszego wielomianu to: $4, -1, 13$.\\
Wielomian jest stopnia parzystego, ponadto znak współczynnika przy\linebreak najwyższej potędze x jest ujemny.\\ W związku z tym wykres wielomianu zaczyna się od lewej strony powyżej osi OX.\\
Ponadto w punkcie $-1$ wykres odbija się od osi poziomej.\\
A więc $$x \in \{-1\} \cup [4,13].$$
\rozwStop
\odpStart
$x \in \{-1\} \cup [4,13]$
\odpStop
\testStart
A.$x \in \{-1\} \cup [4,13]$\\
B.$x \in \{1\} \cup (4,13)$\\
C.$x \in \{-1\} \cup (4,13]$\\
D.$x \in \{1\} \cup (4,13]$\\
E.$x \in \{-1\} \cup [4,13)$\\
F.$x \in \{1\} \cup [4,13)$\\
G.$x \in \{-1\} \cup (4,13)$\\
H.$x \in \{1\} \cup [4,13]$
\testStop
\kluczStart
A
\kluczStop



\zadStart{Zadanie z Wikieł Z 1.62 c) moja wersja nr 62}

Rozwiązać nierówności $(4-x)(x+1)^{2}(14-x)^{3}\le0$.
\zadStop
\rozwStart{Patryk Wirkus}{}
Miejsca zerowe naszego wielomianu to: $4, -1, 14$.\\
Wielomian jest stopnia parzystego, ponadto znak współczynnika przy\linebreak najwyższej potędze x jest ujemny.\\ W związku z tym wykres wielomianu zaczyna się od lewej strony powyżej osi OX.\\
Ponadto w punkcie $-1$ wykres odbija się od osi poziomej.\\
A więc $$x \in \{-1\} \cup [4,14].$$
\rozwStop
\odpStart
$x \in \{-1\} \cup [4,14]$
\odpStop
\testStart
A.$x \in \{-1\} \cup [4,14]$\\
B.$x \in \{1\} \cup (4,14)$\\
C.$x \in \{-1\} \cup (4,14]$\\
D.$x \in \{1\} \cup (4,14]$\\
E.$x \in \{-1\} \cup [4,14)$\\
F.$x \in \{1\} \cup [4,14)$\\
G.$x \in \{-1\} \cup (4,14)$\\
H.$x \in \{1\} \cup [4,14]$
\testStop
\kluczStart
A
\kluczStop



\zadStart{Zadanie z Wikieł Z 1.62 c) moja wersja nr 63}

Rozwiązać nierówności $(4-x)(x+1)^{2}(15-x)^{3}\le0$.
\zadStop
\rozwStart{Patryk Wirkus}{}
Miejsca zerowe naszego wielomianu to: $4, -1, 15$.\\
Wielomian jest stopnia parzystego, ponadto znak współczynnika przy\linebreak najwyższej potędze x jest ujemny.\\ W związku z tym wykres wielomianu zaczyna się od lewej strony powyżej osi OX.\\
Ponadto w punkcie $-1$ wykres odbija się od osi poziomej.\\
A więc $$x \in \{-1\} \cup [4,15].$$
\rozwStop
\odpStart
$x \in \{-1\} \cup [4,15]$
\odpStop
\testStart
A.$x \in \{-1\} \cup [4,15]$\\
B.$x \in \{1\} \cup (4,15)$\\
C.$x \in \{-1\} \cup (4,15]$\\
D.$x \in \{1\} \cup (4,15]$\\
E.$x \in \{-1\} \cup [4,15)$\\
F.$x \in \{1\} \cup [4,15)$\\
G.$x \in \{-1\} \cup (4,15)$\\
H.$x \in \{1\} \cup [4,15]$
\testStop
\kluczStart
A
\kluczStop



\zadStart{Zadanie z Wikieł Z 1.62 c) moja wersja nr 64}

Rozwiązać nierówności $(4-x)(x+1)^{2}(16-x)^{3}\le0$.
\zadStop
\rozwStart{Patryk Wirkus}{}
Miejsca zerowe naszego wielomianu to: $4, -1, 16$.\\
Wielomian jest stopnia parzystego, ponadto znak współczynnika przy\linebreak najwyższej potędze x jest ujemny.\\ W związku z tym wykres wielomianu zaczyna się od lewej strony powyżej osi OX.\\
Ponadto w punkcie $-1$ wykres odbija się od osi poziomej.\\
A więc $$x \in \{-1\} \cup [4,16].$$
\rozwStop
\odpStart
$x \in \{-1\} \cup [4,16]$
\odpStop
\testStart
A.$x \in \{-1\} \cup [4,16]$\\
B.$x \in \{1\} \cup (4,16)$\\
C.$x \in \{-1\} \cup (4,16]$\\
D.$x \in \{1\} \cup (4,16]$\\
E.$x \in \{-1\} \cup [4,16)$\\
F.$x \in \{1\} \cup [4,16)$\\
G.$x \in \{-1\} \cup (4,16)$\\
H.$x \in \{1\} \cup [4,16]$
\testStop
\kluczStart
A
\kluczStop



\zadStart{Zadanie z Wikieł Z 1.62 c) moja wersja nr 65}

Rozwiązać nierówności $(4-x)(x+1)^{2}(17-x)^{3}\le0$.
\zadStop
\rozwStart{Patryk Wirkus}{}
Miejsca zerowe naszego wielomianu to: $4, -1, 17$.\\
Wielomian jest stopnia parzystego, ponadto znak współczynnika przy\linebreak najwyższej potędze x jest ujemny.\\ W związku z tym wykres wielomianu zaczyna się od lewej strony powyżej osi OX.\\
Ponadto w punkcie $-1$ wykres odbija się od osi poziomej.\\
A więc $$x \in \{-1\} \cup [4,17].$$
\rozwStop
\odpStart
$x \in \{-1\} \cup [4,17]$
\odpStop
\testStart
A.$x \in \{-1\} \cup [4,17]$\\
B.$x \in \{1\} \cup (4,17)$\\
C.$x \in \{-1\} \cup (4,17]$\\
D.$x \in \{1\} \cup (4,17]$\\
E.$x \in \{-1\} \cup [4,17)$\\
F.$x \in \{1\} \cup [4,17)$\\
G.$x \in \{-1\} \cup (4,17)$\\
H.$x \in \{1\} \cup [4,17]$
\testStop
\kluczStart
A
\kluczStop



\zadStart{Zadanie z Wikieł Z 1.62 c) moja wersja nr 66}

Rozwiązać nierówności $(4-x)(x+1)^{2}(18-x)^{3}\le0$.
\zadStop
\rozwStart{Patryk Wirkus}{}
Miejsca zerowe naszego wielomianu to: $4, -1, 18$.\\
Wielomian jest stopnia parzystego, ponadto znak współczynnika przy\linebreak najwyższej potędze x jest ujemny.\\ W związku z tym wykres wielomianu zaczyna się od lewej strony powyżej osi OX.\\
Ponadto w punkcie $-1$ wykres odbija się od osi poziomej.\\
A więc $$x \in \{-1\} \cup [4,18].$$
\rozwStop
\odpStart
$x \in \{-1\} \cup [4,18]$
\odpStop
\testStart
A.$x \in \{-1\} \cup [4,18]$\\
B.$x \in \{1\} \cup (4,18)$\\
C.$x \in \{-1\} \cup (4,18]$\\
D.$x \in \{1\} \cup (4,18]$\\
E.$x \in \{-1\} \cup [4,18)$\\
F.$x \in \{1\} \cup [4,18)$\\
G.$x \in \{-1\} \cup (4,18)$\\
H.$x \in \{1\} \cup [4,18]$
\testStop
\kluczStart
A
\kluczStop



\zadStart{Zadanie z Wikieł Z 1.62 c) moja wersja nr 67}

Rozwiązać nierówności $(4-x)(x+1)^{2}(19-x)^{3}\le0$.
\zadStop
\rozwStart{Patryk Wirkus}{}
Miejsca zerowe naszego wielomianu to: $4, -1, 19$.\\
Wielomian jest stopnia parzystego, ponadto znak współczynnika przy\linebreak najwyższej potędze x jest ujemny.\\ W związku z tym wykres wielomianu zaczyna się od lewej strony powyżej osi OX.\\
Ponadto w punkcie $-1$ wykres odbija się od osi poziomej.\\
A więc $$x \in \{-1\} \cup [4,19].$$
\rozwStop
\odpStart
$x \in \{-1\} \cup [4,19]$
\odpStop
\testStart
A.$x \in \{-1\} \cup [4,19]$\\
B.$x \in \{1\} \cup (4,19)$\\
C.$x \in \{-1\} \cup (4,19]$\\
D.$x \in \{1\} \cup (4,19]$\\
E.$x \in \{-1\} \cup [4,19)$\\
F.$x \in \{1\} \cup [4,19)$\\
G.$x \in \{-1\} \cup (4,19)$\\
H.$x \in \{1\} \cup [4,19]$
\testStop
\kluczStart
A
\kluczStop



\zadStart{Zadanie z Wikieł Z 1.62 c) moja wersja nr 68}

Rozwiązać nierówności $(4-x)(x+1)^{2}(20-x)^{3}\le0$.
\zadStop
\rozwStart{Patryk Wirkus}{}
Miejsca zerowe naszego wielomianu to: $4, -1, 20$.\\
Wielomian jest stopnia parzystego, ponadto znak współczynnika przy\linebreak najwyższej potędze x jest ujemny.\\ W związku z tym wykres wielomianu zaczyna się od lewej strony powyżej osi OX.\\
Ponadto w punkcie $-1$ wykres odbija się od osi poziomej.\\
A więc $$x \in \{-1\} \cup [4,20].$$
\rozwStop
\odpStart
$x \in \{-1\} \cup [4,20]$
\odpStop
\testStart
A.$x \in \{-1\} \cup [4,20]$\\
B.$x \in \{1\} \cup (4,20)$\\
C.$x \in \{-1\} \cup (4,20]$\\
D.$x \in \{1\} \cup (4,20]$\\
E.$x \in \{-1\} \cup [4,20)$\\
F.$x \in \{1\} \cup [4,20)$\\
G.$x \in \{-1\} \cup (4,20)$\\
H.$x \in \{1\} \cup [4,20]$
\testStop
\kluczStart
A
\kluczStop



\zadStart{Zadanie z Wikieł Z 1.62 c) moja wersja nr 69}

Rozwiązać nierówności $(4-x)(x+2)^{2}(5-x)^{3}\le0$.
\zadStop
\rozwStart{Patryk Wirkus}{}
Miejsca zerowe naszego wielomianu to: $4, -2, 5$.\\
Wielomian jest stopnia parzystego, ponadto znak współczynnika przy\linebreak najwyższej potędze x jest ujemny.\\ W związku z tym wykres wielomianu zaczyna się od lewej strony powyżej osi OX.\\
Ponadto w punkcie $-2$ wykres odbija się od osi poziomej.\\
A więc $$x \in \{-2\} \cup [4,5].$$
\rozwStop
\odpStart
$x \in \{-2\} \cup [4,5]$
\odpStop
\testStart
A.$x \in \{-2\} \cup [4,5]$\\
B.$x \in \{2\} \cup (4,5)$\\
C.$x \in \{-2\} \cup (4,5]$\\
D.$x \in \{2\} \cup (4,5]$\\
E.$x \in \{-2\} \cup [4,5)$\\
F.$x \in \{2\} \cup [4,5)$\\
G.$x \in \{-2\} \cup (4,5)$\\
H.$x \in \{2\} \cup [4,5]$
\testStop
\kluczStart
A
\kluczStop



\zadStart{Zadanie z Wikieł Z 1.62 c) moja wersja nr 70}

Rozwiązać nierówności $(4-x)(x+2)^{2}(6-x)^{3}\le0$.
\zadStop
\rozwStart{Patryk Wirkus}{}
Miejsca zerowe naszego wielomianu to: $4, -2, 6$.\\
Wielomian jest stopnia parzystego, ponadto znak współczynnika przy\linebreak najwyższej potędze x jest ujemny.\\ W związku z tym wykres wielomianu zaczyna się od lewej strony powyżej osi OX.\\
Ponadto w punkcie $-2$ wykres odbija się od osi poziomej.\\
A więc $$x \in \{-2\} \cup [4,6].$$
\rozwStop
\odpStart
$x \in \{-2\} \cup [4,6]$
\odpStop
\testStart
A.$x \in \{-2\} \cup [4,6]$\\
B.$x \in \{2\} \cup (4,6)$\\
C.$x \in \{-2\} \cup (4,6]$\\
D.$x \in \{2\} \cup (4,6]$\\
E.$x \in \{-2\} \cup [4,6)$\\
F.$x \in \{2\} \cup [4,6)$\\
G.$x \in \{-2\} \cup (4,6)$\\
H.$x \in \{2\} \cup [4,6]$
\testStop
\kluczStart
A
\kluczStop



\zadStart{Zadanie z Wikieł Z 1.62 c) moja wersja nr 71}

Rozwiązać nierówności $(4-x)(x+2)^{2}(7-x)^{3}\le0$.
\zadStop
\rozwStart{Patryk Wirkus}{}
Miejsca zerowe naszego wielomianu to: $4, -2, 7$.\\
Wielomian jest stopnia parzystego, ponadto znak współczynnika przy\linebreak najwyższej potędze x jest ujemny.\\ W związku z tym wykres wielomianu zaczyna się od lewej strony powyżej osi OX.\\
Ponadto w punkcie $-2$ wykres odbija się od osi poziomej.\\
A więc $$x \in \{-2\} \cup [4,7].$$
\rozwStop
\odpStart
$x \in \{-2\} \cup [4,7]$
\odpStop
\testStart
A.$x \in \{-2\} \cup [4,7]$\\
B.$x \in \{2\} \cup (4,7)$\\
C.$x \in \{-2\} \cup (4,7]$\\
D.$x \in \{2\} \cup (4,7]$\\
E.$x \in \{-2\} \cup [4,7)$\\
F.$x \in \{2\} \cup [4,7)$\\
G.$x \in \{-2\} \cup (4,7)$\\
H.$x \in \{2\} \cup [4,7]$
\testStop
\kluczStart
A
\kluczStop



\zadStart{Zadanie z Wikieł Z 1.62 c) moja wersja nr 72}

Rozwiązać nierówności $(4-x)(x+2)^{2}(8-x)^{3}\le0$.
\zadStop
\rozwStart{Patryk Wirkus}{}
Miejsca zerowe naszego wielomianu to: $4, -2, 8$.\\
Wielomian jest stopnia parzystego, ponadto znak współczynnika przy\linebreak najwyższej potędze x jest ujemny.\\ W związku z tym wykres wielomianu zaczyna się od lewej strony powyżej osi OX.\\
Ponadto w punkcie $-2$ wykres odbija się od osi poziomej.\\
A więc $$x \in \{-2\} \cup [4,8].$$
\rozwStop
\odpStart
$x \in \{-2\} \cup [4,8]$
\odpStop
\testStart
A.$x \in \{-2\} \cup [4,8]$\\
B.$x \in \{2\} \cup (4,8)$\\
C.$x \in \{-2\} \cup (4,8]$\\
D.$x \in \{2\} \cup (4,8]$\\
E.$x \in \{-2\} \cup [4,8)$\\
F.$x \in \{2\} \cup [4,8)$\\
G.$x \in \{-2\} \cup (4,8)$\\
H.$x \in \{2\} \cup [4,8]$
\testStop
\kluczStart
A
\kluczStop



\zadStart{Zadanie z Wikieł Z 1.62 c) moja wersja nr 73}

Rozwiązać nierówności $(4-x)(x+2)^{2}(9-x)^{3}\le0$.
\zadStop
\rozwStart{Patryk Wirkus}{}
Miejsca zerowe naszego wielomianu to: $4, -2, 9$.\\
Wielomian jest stopnia parzystego, ponadto znak współczynnika przy\linebreak najwyższej potędze x jest ujemny.\\ W związku z tym wykres wielomianu zaczyna się od lewej strony powyżej osi OX.\\
Ponadto w punkcie $-2$ wykres odbija się od osi poziomej.\\
A więc $$x \in \{-2\} \cup [4,9].$$
\rozwStop
\odpStart
$x \in \{-2\} \cup [4,9]$
\odpStop
\testStart
A.$x \in \{-2\} \cup [4,9]$\\
B.$x \in \{2\} \cup (4,9)$\\
C.$x \in \{-2\} \cup (4,9]$\\
D.$x \in \{2\} \cup (4,9]$\\
E.$x \in \{-2\} \cup [4,9)$\\
F.$x \in \{2\} \cup [4,9)$\\
G.$x \in \{-2\} \cup (4,9)$\\
H.$x \in \{2\} \cup [4,9]$
\testStop
\kluczStart
A
\kluczStop



\zadStart{Zadanie z Wikieł Z 1.62 c) moja wersja nr 74}

Rozwiązać nierówności $(4-x)(x+2)^{2}(10-x)^{3}\le0$.
\zadStop
\rozwStart{Patryk Wirkus}{}
Miejsca zerowe naszego wielomianu to: $4, -2, 10$.\\
Wielomian jest stopnia parzystego, ponadto znak współczynnika przy\linebreak najwyższej potędze x jest ujemny.\\ W związku z tym wykres wielomianu zaczyna się od lewej strony powyżej osi OX.\\
Ponadto w punkcie $-2$ wykres odbija się od osi poziomej.\\
A więc $$x \in \{-2\} \cup [4,10].$$
\rozwStop
\odpStart
$x \in \{-2\} \cup [4,10]$
\odpStop
\testStart
A.$x \in \{-2\} \cup [4,10]$\\
B.$x \in \{2\} \cup (4,10)$\\
C.$x \in \{-2\} \cup (4,10]$\\
D.$x \in \{2\} \cup (4,10]$\\
E.$x \in \{-2\} \cup [4,10)$\\
F.$x \in \{2\} \cup [4,10)$\\
G.$x \in \{-2\} \cup (4,10)$\\
H.$x \in \{2\} \cup [4,10]$
\testStop
\kluczStart
A
\kluczStop



\zadStart{Zadanie z Wikieł Z 1.62 c) moja wersja nr 75}

Rozwiązać nierówności $(4-x)(x+2)^{2}(11-x)^{3}\le0$.
\zadStop
\rozwStart{Patryk Wirkus}{}
Miejsca zerowe naszego wielomianu to: $4, -2, 11$.\\
Wielomian jest stopnia parzystego, ponadto znak współczynnika przy\linebreak najwyższej potędze x jest ujemny.\\ W związku z tym wykres wielomianu zaczyna się od lewej strony powyżej osi OX.\\
Ponadto w punkcie $-2$ wykres odbija się od osi poziomej.\\
A więc $$x \in \{-2\} \cup [4,11].$$
\rozwStop
\odpStart
$x \in \{-2\} \cup [4,11]$
\odpStop
\testStart
A.$x \in \{-2\} \cup [4,11]$\\
B.$x \in \{2\} \cup (4,11)$\\
C.$x \in \{-2\} \cup (4,11]$\\
D.$x \in \{2\} \cup (4,11]$\\
E.$x \in \{-2\} \cup [4,11)$\\
F.$x \in \{2\} \cup [4,11)$\\
G.$x \in \{-2\} \cup (4,11)$\\
H.$x \in \{2\} \cup [4,11]$
\testStop
\kluczStart
A
\kluczStop



\zadStart{Zadanie z Wikieł Z 1.62 c) moja wersja nr 76}

Rozwiązać nierówności $(4-x)(x+2)^{2}(12-x)^{3}\le0$.
\zadStop
\rozwStart{Patryk Wirkus}{}
Miejsca zerowe naszego wielomianu to: $4, -2, 12$.\\
Wielomian jest stopnia parzystego, ponadto znak współczynnika przy\linebreak najwyższej potędze x jest ujemny.\\ W związku z tym wykres wielomianu zaczyna się od lewej strony powyżej osi OX.\\
Ponadto w punkcie $-2$ wykres odbija się od osi poziomej.\\
A więc $$x \in \{-2\} \cup [4,12].$$
\rozwStop
\odpStart
$x \in \{-2\} \cup [4,12]$
\odpStop
\testStart
A.$x \in \{-2\} \cup [4,12]$\\
B.$x \in \{2\} \cup (4,12)$\\
C.$x \in \{-2\} \cup (4,12]$\\
D.$x \in \{2\} \cup (4,12]$\\
E.$x \in \{-2\} \cup [4,12)$\\
F.$x \in \{2\} \cup [4,12)$\\
G.$x \in \{-2\} \cup (4,12)$\\
H.$x \in \{2\} \cup [4,12]$
\testStop
\kluczStart
A
\kluczStop



\zadStart{Zadanie z Wikieł Z 1.62 c) moja wersja nr 77}

Rozwiązać nierówności $(4-x)(x+2)^{2}(13-x)^{3}\le0$.
\zadStop
\rozwStart{Patryk Wirkus}{}
Miejsca zerowe naszego wielomianu to: $4, -2, 13$.\\
Wielomian jest stopnia parzystego, ponadto znak współczynnika przy\linebreak najwyższej potędze x jest ujemny.\\ W związku z tym wykres wielomianu zaczyna się od lewej strony powyżej osi OX.\\
Ponadto w punkcie $-2$ wykres odbija się od osi poziomej.\\
A więc $$x \in \{-2\} \cup [4,13].$$
\rozwStop
\odpStart
$x \in \{-2\} \cup [4,13]$
\odpStop
\testStart
A.$x \in \{-2\} \cup [4,13]$\\
B.$x \in \{2\} \cup (4,13)$\\
C.$x \in \{-2\} \cup (4,13]$\\
D.$x \in \{2\} \cup (4,13]$\\
E.$x \in \{-2\} \cup [4,13)$\\
F.$x \in \{2\} \cup [4,13)$\\
G.$x \in \{-2\} \cup (4,13)$\\
H.$x \in \{2\} \cup [4,13]$
\testStop
\kluczStart
A
\kluczStop



\zadStart{Zadanie z Wikieł Z 1.62 c) moja wersja nr 78}

Rozwiązać nierówności $(4-x)(x+2)^{2}(14-x)^{3}\le0$.
\zadStop
\rozwStart{Patryk Wirkus}{}
Miejsca zerowe naszego wielomianu to: $4, -2, 14$.\\
Wielomian jest stopnia parzystego, ponadto znak współczynnika przy\linebreak najwyższej potędze x jest ujemny.\\ W związku z tym wykres wielomianu zaczyna się od lewej strony powyżej osi OX.\\
Ponadto w punkcie $-2$ wykres odbija się od osi poziomej.\\
A więc $$x \in \{-2\} \cup [4,14].$$
\rozwStop
\odpStart
$x \in \{-2\} \cup [4,14]$
\odpStop
\testStart
A.$x \in \{-2\} \cup [4,14]$\\
B.$x \in \{2\} \cup (4,14)$\\
C.$x \in \{-2\} \cup (4,14]$\\
D.$x \in \{2\} \cup (4,14]$\\
E.$x \in \{-2\} \cup [4,14)$\\
F.$x \in \{2\} \cup [4,14)$\\
G.$x \in \{-2\} \cup (4,14)$\\
H.$x \in \{2\} \cup [4,14]$
\testStop
\kluczStart
A
\kluczStop



\zadStart{Zadanie z Wikieł Z 1.62 c) moja wersja nr 79}

Rozwiązać nierówności $(4-x)(x+2)^{2}(15-x)^{3}\le0$.
\zadStop
\rozwStart{Patryk Wirkus}{}
Miejsca zerowe naszego wielomianu to: $4, -2, 15$.\\
Wielomian jest stopnia parzystego, ponadto znak współczynnika przy\linebreak najwyższej potędze x jest ujemny.\\ W związku z tym wykres wielomianu zaczyna się od lewej strony powyżej osi OX.\\
Ponadto w punkcie $-2$ wykres odbija się od osi poziomej.\\
A więc $$x \in \{-2\} \cup [4,15].$$
\rozwStop
\odpStart
$x \in \{-2\} \cup [4,15]$
\odpStop
\testStart
A.$x \in \{-2\} \cup [4,15]$\\
B.$x \in \{2\} \cup (4,15)$\\
C.$x \in \{-2\} \cup (4,15]$\\
D.$x \in \{2\} \cup (4,15]$\\
E.$x \in \{-2\} \cup [4,15)$\\
F.$x \in \{2\} \cup [4,15)$\\
G.$x \in \{-2\} \cup (4,15)$\\
H.$x \in \{2\} \cup [4,15]$
\testStop
\kluczStart
A
\kluczStop



\zadStart{Zadanie z Wikieł Z 1.62 c) moja wersja nr 80}

Rozwiązać nierówności $(4-x)(x+2)^{2}(16-x)^{3}\le0$.
\zadStop
\rozwStart{Patryk Wirkus}{}
Miejsca zerowe naszego wielomianu to: $4, -2, 16$.\\
Wielomian jest stopnia parzystego, ponadto znak współczynnika przy\linebreak najwyższej potędze x jest ujemny.\\ W związku z tym wykres wielomianu zaczyna się od lewej strony powyżej osi OX.\\
Ponadto w punkcie $-2$ wykres odbija się od osi poziomej.\\
A więc $$x \in \{-2\} \cup [4,16].$$
\rozwStop
\odpStart
$x \in \{-2\} \cup [4,16]$
\odpStop
\testStart
A.$x \in \{-2\} \cup [4,16]$\\
B.$x \in \{2\} \cup (4,16)$\\
C.$x \in \{-2\} \cup (4,16]$\\
D.$x \in \{2\} \cup (4,16]$\\
E.$x \in \{-2\} \cup [4,16)$\\
F.$x \in \{2\} \cup [4,16)$\\
G.$x \in \{-2\} \cup (4,16)$\\
H.$x \in \{2\} \cup [4,16]$
\testStop
\kluczStart
A
\kluczStop



\zadStart{Zadanie z Wikieł Z 1.62 c) moja wersja nr 81}

Rozwiązać nierówności $(4-x)(x+2)^{2}(17-x)^{3}\le0$.
\zadStop
\rozwStart{Patryk Wirkus}{}
Miejsca zerowe naszego wielomianu to: $4, -2, 17$.\\
Wielomian jest stopnia parzystego, ponadto znak współczynnika przy\linebreak najwyższej potędze x jest ujemny.\\ W związku z tym wykres wielomianu zaczyna się od lewej strony powyżej osi OX.\\
Ponadto w punkcie $-2$ wykres odbija się od osi poziomej.\\
A więc $$x \in \{-2\} \cup [4,17].$$
\rozwStop
\odpStart
$x \in \{-2\} \cup [4,17]$
\odpStop
\testStart
A.$x \in \{-2\} \cup [4,17]$\\
B.$x \in \{2\} \cup (4,17)$\\
C.$x \in \{-2\} \cup (4,17]$\\
D.$x \in \{2\} \cup (4,17]$\\
E.$x \in \{-2\} \cup [4,17)$\\
F.$x \in \{2\} \cup [4,17)$\\
G.$x \in \{-2\} \cup (4,17)$\\
H.$x \in \{2\} \cup [4,17]$
\testStop
\kluczStart
A
\kluczStop



\zadStart{Zadanie z Wikieł Z 1.62 c) moja wersja nr 82}

Rozwiązać nierówności $(4-x)(x+2)^{2}(18-x)^{3}\le0$.
\zadStop
\rozwStart{Patryk Wirkus}{}
Miejsca zerowe naszego wielomianu to: $4, -2, 18$.\\
Wielomian jest stopnia parzystego, ponadto znak współczynnika przy\linebreak najwyższej potędze x jest ujemny.\\ W związku z tym wykres wielomianu zaczyna się od lewej strony powyżej osi OX.\\
Ponadto w punkcie $-2$ wykres odbija się od osi poziomej.\\
A więc $$x \in \{-2\} \cup [4,18].$$
\rozwStop
\odpStart
$x \in \{-2\} \cup [4,18]$
\odpStop
\testStart
A.$x \in \{-2\} \cup [4,18]$\\
B.$x \in \{2\} \cup (4,18)$\\
C.$x \in \{-2\} \cup (4,18]$\\
D.$x \in \{2\} \cup (4,18]$\\
E.$x \in \{-2\} \cup [4,18)$\\
F.$x \in \{2\} \cup [4,18)$\\
G.$x \in \{-2\} \cup (4,18)$\\
H.$x \in \{2\} \cup [4,18]$
\testStop
\kluczStart
A
\kluczStop



\zadStart{Zadanie z Wikieł Z 1.62 c) moja wersja nr 83}

Rozwiązać nierówności $(4-x)(x+2)^{2}(19-x)^{3}\le0$.
\zadStop
\rozwStart{Patryk Wirkus}{}
Miejsca zerowe naszego wielomianu to: $4, -2, 19$.\\
Wielomian jest stopnia parzystego, ponadto znak współczynnika przy\linebreak najwyższej potędze x jest ujemny.\\ W związku z tym wykres wielomianu zaczyna się od lewej strony powyżej osi OX.\\
Ponadto w punkcie $-2$ wykres odbija się od osi poziomej.\\
A więc $$x \in \{-2\} \cup [4,19].$$
\rozwStop
\odpStart
$x \in \{-2\} \cup [4,19]$
\odpStop
\testStart
A.$x \in \{-2\} \cup [4,19]$\\
B.$x \in \{2\} \cup (4,19)$\\
C.$x \in \{-2\} \cup (4,19]$\\
D.$x \in \{2\} \cup (4,19]$\\
E.$x \in \{-2\} \cup [4,19)$\\
F.$x \in \{2\} \cup [4,19)$\\
G.$x \in \{-2\} \cup (4,19)$\\
H.$x \in \{2\} \cup [4,19]$
\testStop
\kluczStart
A
\kluczStop



\zadStart{Zadanie z Wikieł Z 1.62 c) moja wersja nr 84}

Rozwiązać nierówności $(4-x)(x+2)^{2}(20-x)^{3}\le0$.
\zadStop
\rozwStart{Patryk Wirkus}{}
Miejsca zerowe naszego wielomianu to: $4, -2, 20$.\\
Wielomian jest stopnia parzystego, ponadto znak współczynnika przy\linebreak najwyższej potędze x jest ujemny.\\ W związku z tym wykres wielomianu zaczyna się od lewej strony powyżej osi OX.\\
Ponadto w punkcie $-2$ wykres odbija się od osi poziomej.\\
A więc $$x \in \{-2\} \cup [4,20].$$
\rozwStop
\odpStart
$x \in \{-2\} \cup [4,20]$
\odpStop
\testStart
A.$x \in \{-2\} \cup [4,20]$\\
B.$x \in \{2\} \cup (4,20)$\\
C.$x \in \{-2\} \cup (4,20]$\\
D.$x \in \{2\} \cup (4,20]$\\
E.$x \in \{-2\} \cup [4,20)$\\
F.$x \in \{2\} \cup [4,20)$\\
G.$x \in \{-2\} \cup (4,20)$\\
H.$x \in \{2\} \cup [4,20]$
\testStop
\kluczStart
A
\kluczStop



\zadStart{Zadanie z Wikieł Z 1.62 c) moja wersja nr 85}

Rozwiązać nierówności $(4-x)(x+3)^{2}(5-x)^{3}\le0$.
\zadStop
\rozwStart{Patryk Wirkus}{}
Miejsca zerowe naszego wielomianu to: $4, -3, 5$.\\
Wielomian jest stopnia parzystego, ponadto znak współczynnika przy\linebreak najwyższej potędze x jest ujemny.\\ W związku z tym wykres wielomianu zaczyna się od lewej strony powyżej osi OX.\\
Ponadto w punkcie $-3$ wykres odbija się od osi poziomej.\\
A więc $$x \in \{-3\} \cup [4,5].$$
\rozwStop
\odpStart
$x \in \{-3\} \cup [4,5]$
\odpStop
\testStart
A.$x \in \{-3\} \cup [4,5]$\\
B.$x \in \{3\} \cup (4,5)$\\
C.$x \in \{-3\} \cup (4,5]$\\
D.$x \in \{3\} \cup (4,5]$\\
E.$x \in \{-3\} \cup [4,5)$\\
F.$x \in \{3\} \cup [4,5)$\\
G.$x \in \{-3\} \cup (4,5)$\\
H.$x \in \{3\} \cup [4,5]$
\testStop
\kluczStart
A
\kluczStop



\zadStart{Zadanie z Wikieł Z 1.62 c) moja wersja nr 86}

Rozwiązać nierówności $(4-x)(x+3)^{2}(6-x)^{3}\le0$.
\zadStop
\rozwStart{Patryk Wirkus}{}
Miejsca zerowe naszego wielomianu to: $4, -3, 6$.\\
Wielomian jest stopnia parzystego, ponadto znak współczynnika przy\linebreak najwyższej potędze x jest ujemny.\\ W związku z tym wykres wielomianu zaczyna się od lewej strony powyżej osi OX.\\
Ponadto w punkcie $-3$ wykres odbija się od osi poziomej.\\
A więc $$x \in \{-3\} \cup [4,6].$$
\rozwStop
\odpStart
$x \in \{-3\} \cup [4,6]$
\odpStop
\testStart
A.$x \in \{-3\} \cup [4,6]$\\
B.$x \in \{3\} \cup (4,6)$\\
C.$x \in \{-3\} \cup (4,6]$\\
D.$x \in \{3\} \cup (4,6]$\\
E.$x \in \{-3\} \cup [4,6)$\\
F.$x \in \{3\} \cup [4,6)$\\
G.$x \in \{-3\} \cup (4,6)$\\
H.$x \in \{3\} \cup [4,6]$
\testStop
\kluczStart
A
\kluczStop



\zadStart{Zadanie z Wikieł Z 1.62 c) moja wersja nr 87}

Rozwiązać nierówności $(4-x)(x+3)^{2}(7-x)^{3}\le0$.
\zadStop
\rozwStart{Patryk Wirkus}{}
Miejsca zerowe naszego wielomianu to: $4, -3, 7$.\\
Wielomian jest stopnia parzystego, ponadto znak współczynnika przy\linebreak najwyższej potędze x jest ujemny.\\ W związku z tym wykres wielomianu zaczyna się od lewej strony powyżej osi OX.\\
Ponadto w punkcie $-3$ wykres odbija się od osi poziomej.\\
A więc $$x \in \{-3\} \cup [4,7].$$
\rozwStop
\odpStart
$x \in \{-3\} \cup [4,7]$
\odpStop
\testStart
A.$x \in \{-3\} \cup [4,7]$\\
B.$x \in \{3\} \cup (4,7)$\\
C.$x \in \{-3\} \cup (4,7]$\\
D.$x \in \{3\} \cup (4,7]$\\
E.$x \in \{-3\} \cup [4,7)$\\
F.$x \in \{3\} \cup [4,7)$\\
G.$x \in \{-3\} \cup (4,7)$\\
H.$x \in \{3\} \cup [4,7]$
\testStop
\kluczStart
A
\kluczStop



\zadStart{Zadanie z Wikieł Z 1.62 c) moja wersja nr 88}

Rozwiązać nierówności $(4-x)(x+3)^{2}(8-x)^{3}\le0$.
\zadStop
\rozwStart{Patryk Wirkus}{}
Miejsca zerowe naszego wielomianu to: $4, -3, 8$.\\
Wielomian jest stopnia parzystego, ponadto znak współczynnika przy\linebreak najwyższej potędze x jest ujemny.\\ W związku z tym wykres wielomianu zaczyna się od lewej strony powyżej osi OX.\\
Ponadto w punkcie $-3$ wykres odbija się od osi poziomej.\\
A więc $$x \in \{-3\} \cup [4,8].$$
\rozwStop
\odpStart
$x \in \{-3\} \cup [4,8]$
\odpStop
\testStart
A.$x \in \{-3\} \cup [4,8]$\\
B.$x \in \{3\} \cup (4,8)$\\
C.$x \in \{-3\} \cup (4,8]$\\
D.$x \in \{3\} \cup (4,8]$\\
E.$x \in \{-3\} \cup [4,8)$\\
F.$x \in \{3\} \cup [4,8)$\\
G.$x \in \{-3\} \cup (4,8)$\\
H.$x \in \{3\} \cup [4,8]$
\testStop
\kluczStart
A
\kluczStop



\zadStart{Zadanie z Wikieł Z 1.62 c) moja wersja nr 89}

Rozwiązać nierówności $(4-x)(x+3)^{2}(9-x)^{3}\le0$.
\zadStop
\rozwStart{Patryk Wirkus}{}
Miejsca zerowe naszego wielomianu to: $4, -3, 9$.\\
Wielomian jest stopnia parzystego, ponadto znak współczynnika przy\linebreak najwyższej potędze x jest ujemny.\\ W związku z tym wykres wielomianu zaczyna się od lewej strony powyżej osi OX.\\
Ponadto w punkcie $-3$ wykres odbija się od osi poziomej.\\
A więc $$x \in \{-3\} \cup [4,9].$$
\rozwStop
\odpStart
$x \in \{-3\} \cup [4,9]$
\odpStop
\testStart
A.$x \in \{-3\} \cup [4,9]$\\
B.$x \in \{3\} \cup (4,9)$\\
C.$x \in \{-3\} \cup (4,9]$\\
D.$x \in \{3\} \cup (4,9]$\\
E.$x \in \{-3\} \cup [4,9)$\\
F.$x \in \{3\} \cup [4,9)$\\
G.$x \in \{-3\} \cup (4,9)$\\
H.$x \in \{3\} \cup [4,9]$
\testStop
\kluczStart
A
\kluczStop



\zadStart{Zadanie z Wikieł Z 1.62 c) moja wersja nr 90}

Rozwiązać nierówności $(4-x)(x+3)^{2}(10-x)^{3}\le0$.
\zadStop
\rozwStart{Patryk Wirkus}{}
Miejsca zerowe naszego wielomianu to: $4, -3, 10$.\\
Wielomian jest stopnia parzystego, ponadto znak współczynnika przy\linebreak najwyższej potędze x jest ujemny.\\ W związku z tym wykres wielomianu zaczyna się od lewej strony powyżej osi OX.\\
Ponadto w punkcie $-3$ wykres odbija się od osi poziomej.\\
A więc $$x \in \{-3\} \cup [4,10].$$
\rozwStop
\odpStart
$x \in \{-3\} \cup [4,10]$
\odpStop
\testStart
A.$x \in \{-3\} \cup [4,10]$\\
B.$x \in \{3\} \cup (4,10)$\\
C.$x \in \{-3\} \cup (4,10]$\\
D.$x \in \{3\} \cup (4,10]$\\
E.$x \in \{-3\} \cup [4,10)$\\
F.$x \in \{3\} \cup [4,10)$\\
G.$x \in \{-3\} \cup (4,10)$\\
H.$x \in \{3\} \cup [4,10]$
\testStop
\kluczStart
A
\kluczStop



\zadStart{Zadanie z Wikieł Z 1.62 c) moja wersja nr 91}

Rozwiązać nierówności $(4-x)(x+3)^{2}(11-x)^{3}\le0$.
\zadStop
\rozwStart{Patryk Wirkus}{}
Miejsca zerowe naszego wielomianu to: $4, -3, 11$.\\
Wielomian jest stopnia parzystego, ponadto znak współczynnika przy\linebreak najwyższej potędze x jest ujemny.\\ W związku z tym wykres wielomianu zaczyna się od lewej strony powyżej osi OX.\\
Ponadto w punkcie $-3$ wykres odbija się od osi poziomej.\\
A więc $$x \in \{-3\} \cup [4,11].$$
\rozwStop
\odpStart
$x \in \{-3\} \cup [4,11]$
\odpStop
\testStart
A.$x \in \{-3\} \cup [4,11]$\\
B.$x \in \{3\} \cup (4,11)$\\
C.$x \in \{-3\} \cup (4,11]$\\
D.$x \in \{3\} \cup (4,11]$\\
E.$x \in \{-3\} \cup [4,11)$\\
F.$x \in \{3\} \cup [4,11)$\\
G.$x \in \{-3\} \cup (4,11)$\\
H.$x \in \{3\} \cup [4,11]$
\testStop
\kluczStart
A
\kluczStop



\zadStart{Zadanie z Wikieł Z 1.62 c) moja wersja nr 92}

Rozwiązać nierówności $(4-x)(x+3)^{2}(12-x)^{3}\le0$.
\zadStop
\rozwStart{Patryk Wirkus}{}
Miejsca zerowe naszego wielomianu to: $4, -3, 12$.\\
Wielomian jest stopnia parzystego, ponadto znak współczynnika przy\linebreak najwyższej potędze x jest ujemny.\\ W związku z tym wykres wielomianu zaczyna się od lewej strony powyżej osi OX.\\
Ponadto w punkcie $-3$ wykres odbija się od osi poziomej.\\
A więc $$x \in \{-3\} \cup [4,12].$$
\rozwStop
\odpStart
$x \in \{-3\} \cup [4,12]$
\odpStop
\testStart
A.$x \in \{-3\} \cup [4,12]$\\
B.$x \in \{3\} \cup (4,12)$\\
C.$x \in \{-3\} \cup (4,12]$\\
D.$x \in \{3\} \cup (4,12]$\\
E.$x \in \{-3\} \cup [4,12)$\\
F.$x \in \{3\} \cup [4,12)$\\
G.$x \in \{-3\} \cup (4,12)$\\
H.$x \in \{3\} \cup [4,12]$
\testStop
\kluczStart
A
\kluczStop



\zadStart{Zadanie z Wikieł Z 1.62 c) moja wersja nr 93}

Rozwiązać nierówności $(4-x)(x+3)^{2}(13-x)^{3}\le0$.
\zadStop
\rozwStart{Patryk Wirkus}{}
Miejsca zerowe naszego wielomianu to: $4, -3, 13$.\\
Wielomian jest stopnia parzystego, ponadto znak współczynnika przy\linebreak najwyższej potędze x jest ujemny.\\ W związku z tym wykres wielomianu zaczyna się od lewej strony powyżej osi OX.\\
Ponadto w punkcie $-3$ wykres odbija się od osi poziomej.\\
A więc $$x \in \{-3\} \cup [4,13].$$
\rozwStop
\odpStart
$x \in \{-3\} \cup [4,13]$
\odpStop
\testStart
A.$x \in \{-3\} \cup [4,13]$\\
B.$x \in \{3\} \cup (4,13)$\\
C.$x \in \{-3\} \cup (4,13]$\\
D.$x \in \{3\} \cup (4,13]$\\
E.$x \in \{-3\} \cup [4,13)$\\
F.$x \in \{3\} \cup [4,13)$\\
G.$x \in \{-3\} \cup (4,13)$\\
H.$x \in \{3\} \cup [4,13]$
\testStop
\kluczStart
A
\kluczStop



\zadStart{Zadanie z Wikieł Z 1.62 c) moja wersja nr 94}

Rozwiązać nierówności $(4-x)(x+3)^{2}(14-x)^{3}\le0$.
\zadStop
\rozwStart{Patryk Wirkus}{}
Miejsca zerowe naszego wielomianu to: $4, -3, 14$.\\
Wielomian jest stopnia parzystego, ponadto znak współczynnika przy\linebreak najwyższej potędze x jest ujemny.\\ W związku z tym wykres wielomianu zaczyna się od lewej strony powyżej osi OX.\\
Ponadto w punkcie $-3$ wykres odbija się od osi poziomej.\\
A więc $$x \in \{-3\} \cup [4,14].$$
\rozwStop
\odpStart
$x \in \{-3\} \cup [4,14]$
\odpStop
\testStart
A.$x \in \{-3\} \cup [4,14]$\\
B.$x \in \{3\} \cup (4,14)$\\
C.$x \in \{-3\} \cup (4,14]$\\
D.$x \in \{3\} \cup (4,14]$\\
E.$x \in \{-3\} \cup [4,14)$\\
F.$x \in \{3\} \cup [4,14)$\\
G.$x \in \{-3\} \cup (4,14)$\\
H.$x \in \{3\} \cup [4,14]$
\testStop
\kluczStart
A
\kluczStop



\zadStart{Zadanie z Wikieł Z 1.62 c) moja wersja nr 95}

Rozwiązać nierówności $(4-x)(x+3)^{2}(15-x)^{3}\le0$.
\zadStop
\rozwStart{Patryk Wirkus}{}
Miejsca zerowe naszego wielomianu to: $4, -3, 15$.\\
Wielomian jest stopnia parzystego, ponadto znak współczynnika przy\linebreak najwyższej potędze x jest ujemny.\\ W związku z tym wykres wielomianu zaczyna się od lewej strony powyżej osi OX.\\
Ponadto w punkcie $-3$ wykres odbija się od osi poziomej.\\
A więc $$x \in \{-3\} \cup [4,15].$$
\rozwStop
\odpStart
$x \in \{-3\} \cup [4,15]$
\odpStop
\testStart
A.$x \in \{-3\} \cup [4,15]$\\
B.$x \in \{3\} \cup (4,15)$\\
C.$x \in \{-3\} \cup (4,15]$\\
D.$x \in \{3\} \cup (4,15]$\\
E.$x \in \{-3\} \cup [4,15)$\\
F.$x \in \{3\} \cup [4,15)$\\
G.$x \in \{-3\} \cup (4,15)$\\
H.$x \in \{3\} \cup [4,15]$
\testStop
\kluczStart
A
\kluczStop



\zadStart{Zadanie z Wikieł Z 1.62 c) moja wersja nr 96}

Rozwiązać nierówności $(4-x)(x+3)^{2}(16-x)^{3}\le0$.
\zadStop
\rozwStart{Patryk Wirkus}{}
Miejsca zerowe naszego wielomianu to: $4, -3, 16$.\\
Wielomian jest stopnia parzystego, ponadto znak współczynnika przy\linebreak najwyższej potędze x jest ujemny.\\ W związku z tym wykres wielomianu zaczyna się od lewej strony powyżej osi OX.\\
Ponadto w punkcie $-3$ wykres odbija się od osi poziomej.\\
A więc $$x \in \{-3\} \cup [4,16].$$
\rozwStop
\odpStart
$x \in \{-3\} \cup [4,16]$
\odpStop
\testStart
A.$x \in \{-3\} \cup [4,16]$\\
B.$x \in \{3\} \cup (4,16)$\\
C.$x \in \{-3\} \cup (4,16]$\\
D.$x \in \{3\} \cup (4,16]$\\
E.$x \in \{-3\} \cup [4,16)$\\
F.$x \in \{3\} \cup [4,16)$\\
G.$x \in \{-3\} \cup (4,16)$\\
H.$x \in \{3\} \cup [4,16]$
\testStop
\kluczStart
A
\kluczStop



\zadStart{Zadanie z Wikieł Z 1.62 c) moja wersja nr 97}

Rozwiązać nierówności $(4-x)(x+3)^{2}(17-x)^{3}\le0$.
\zadStop
\rozwStart{Patryk Wirkus}{}
Miejsca zerowe naszego wielomianu to: $4, -3, 17$.\\
Wielomian jest stopnia parzystego, ponadto znak współczynnika przy\linebreak najwyższej potędze x jest ujemny.\\ W związku z tym wykres wielomianu zaczyna się od lewej strony powyżej osi OX.\\
Ponadto w punkcie $-3$ wykres odbija się od osi poziomej.\\
A więc $$x \in \{-3\} \cup [4,17].$$
\rozwStop
\odpStart
$x \in \{-3\} \cup [4,17]$
\odpStop
\testStart
A.$x \in \{-3\} \cup [4,17]$\\
B.$x \in \{3\} \cup (4,17)$\\
C.$x \in \{-3\} \cup (4,17]$\\
D.$x \in \{3\} \cup (4,17]$\\
E.$x \in \{-3\} \cup [4,17)$\\
F.$x \in \{3\} \cup [4,17)$\\
G.$x \in \{-3\} \cup (4,17)$\\
H.$x \in \{3\} \cup [4,17]$
\testStop
\kluczStart
A
\kluczStop



\zadStart{Zadanie z Wikieł Z 1.62 c) moja wersja nr 98}

Rozwiązać nierówności $(4-x)(x+3)^{2}(18-x)^{3}\le0$.
\zadStop
\rozwStart{Patryk Wirkus}{}
Miejsca zerowe naszego wielomianu to: $4, -3, 18$.\\
Wielomian jest stopnia parzystego, ponadto znak współczynnika przy\linebreak najwyższej potędze x jest ujemny.\\ W związku z tym wykres wielomianu zaczyna się od lewej strony powyżej osi OX.\\
Ponadto w punkcie $-3$ wykres odbija się od osi poziomej.\\
A więc $$x \in \{-3\} \cup [4,18].$$
\rozwStop
\odpStart
$x \in \{-3\} \cup [4,18]$
\odpStop
\testStart
A.$x \in \{-3\} \cup [4,18]$\\
B.$x \in \{3\} \cup (4,18)$\\
C.$x \in \{-3\} \cup (4,18]$\\
D.$x \in \{3\} \cup (4,18]$\\
E.$x \in \{-3\} \cup [4,18)$\\
F.$x \in \{3\} \cup [4,18)$\\
G.$x \in \{-3\} \cup (4,18)$\\
H.$x \in \{3\} \cup [4,18]$
\testStop
\kluczStart
A
\kluczStop



\zadStart{Zadanie z Wikieł Z 1.62 c) moja wersja nr 99}

Rozwiązać nierówności $(4-x)(x+3)^{2}(19-x)^{3}\le0$.
\zadStop
\rozwStart{Patryk Wirkus}{}
Miejsca zerowe naszego wielomianu to: $4, -3, 19$.\\
Wielomian jest stopnia parzystego, ponadto znak współczynnika przy\linebreak najwyższej potędze x jest ujemny.\\ W związku z tym wykres wielomianu zaczyna się od lewej strony powyżej osi OX.\\
Ponadto w punkcie $-3$ wykres odbija się od osi poziomej.\\
A więc $$x \in \{-3\} \cup [4,19].$$
\rozwStop
\odpStart
$x \in \{-3\} \cup [4,19]$
\odpStop
\testStart
A.$x \in \{-3\} \cup [4,19]$\\
B.$x \in \{3\} \cup (4,19)$\\
C.$x \in \{-3\} \cup (4,19]$\\
D.$x \in \{3\} \cup (4,19]$\\
E.$x \in \{-3\} \cup [4,19)$\\
F.$x \in \{3\} \cup [4,19)$\\
G.$x \in \{-3\} \cup (4,19)$\\
H.$x \in \{3\} \cup [4,19]$
\testStop
\kluczStart
A
\kluczStop



\zadStart{Zadanie z Wikieł Z 1.62 c) moja wersja nr 100}

Rozwiązać nierówności $(4-x)(x+3)^{2}(20-x)^{3}\le0$.
\zadStop
\rozwStart{Patryk Wirkus}{}
Miejsca zerowe naszego wielomianu to: $4, -3, 20$.\\
Wielomian jest stopnia parzystego, ponadto znak współczynnika przy\linebreak najwyższej potędze x jest ujemny.\\ W związku z tym wykres wielomianu zaczyna się od lewej strony powyżej osi OX.\\
Ponadto w punkcie $-3$ wykres odbija się od osi poziomej.\\
A więc $$x \in \{-3\} \cup [4,20].$$
\rozwStop
\odpStart
$x \in \{-3\} \cup [4,20]$
\odpStop
\testStart
A.$x \in \{-3\} \cup [4,20]$\\
B.$x \in \{3\} \cup (4,20)$\\
C.$x \in \{-3\} \cup (4,20]$\\
D.$x \in \{3\} \cup (4,20]$\\
E.$x \in \{-3\} \cup [4,20)$\\
F.$x \in \{3\} \cup [4,20)$\\
G.$x \in \{-3\} \cup (4,20)$\\
H.$x \in \{3\} \cup [4,20]$
\testStop
\kluczStart
A
\kluczStop



\zadStart{Zadanie z Wikieł Z 1.62 c) moja wersja nr 101}

Rozwiązać nierówności $(5-x)(x+1)^{2}(6-x)^{3}\le0$.
\zadStop
\rozwStart{Patryk Wirkus}{}
Miejsca zerowe naszego wielomianu to: $5, -1, 6$.\\
Wielomian jest stopnia parzystego, ponadto znak współczynnika przy\linebreak najwyższej potędze x jest ujemny.\\ W związku z tym wykres wielomianu zaczyna się od lewej strony powyżej osi OX.\\
Ponadto w punkcie $-1$ wykres odbija się od osi poziomej.\\
A więc $$x \in \{-1\} \cup [5,6].$$
\rozwStop
\odpStart
$x \in \{-1\} \cup [5,6]$
\odpStop
\testStart
A.$x \in \{-1\} \cup [5,6]$\\
B.$x \in \{1\} \cup (5,6)$\\
C.$x \in \{-1\} \cup (5,6]$\\
D.$x \in \{1\} \cup (5,6]$\\
E.$x \in \{-1\} \cup [5,6)$\\
F.$x \in \{1\} \cup [5,6)$\\
G.$x \in \{-1\} \cup (5,6)$\\
H.$x \in \{1\} \cup [5,6]$
\testStop
\kluczStart
A
\kluczStop



\zadStart{Zadanie z Wikieł Z 1.62 c) moja wersja nr 102}

Rozwiązać nierówności $(5-x)(x+1)^{2}(7-x)^{3}\le0$.
\zadStop
\rozwStart{Patryk Wirkus}{}
Miejsca zerowe naszego wielomianu to: $5, -1, 7$.\\
Wielomian jest stopnia parzystego, ponadto znak współczynnika przy\linebreak najwyższej potędze x jest ujemny.\\ W związku z tym wykres wielomianu zaczyna się od lewej strony powyżej osi OX.\\
Ponadto w punkcie $-1$ wykres odbija się od osi poziomej.\\
A więc $$x \in \{-1\} \cup [5,7].$$
\rozwStop
\odpStart
$x \in \{-1\} \cup [5,7]$
\odpStop
\testStart
A.$x \in \{-1\} \cup [5,7]$\\
B.$x \in \{1\} \cup (5,7)$\\
C.$x \in \{-1\} \cup (5,7]$\\
D.$x \in \{1\} \cup (5,7]$\\
E.$x \in \{-1\} \cup [5,7)$\\
F.$x \in \{1\} \cup [5,7)$\\
G.$x \in \{-1\} \cup (5,7)$\\
H.$x \in \{1\} \cup [5,7]$
\testStop
\kluczStart
A
\kluczStop



\zadStart{Zadanie z Wikieł Z 1.62 c) moja wersja nr 103}

Rozwiązać nierówności $(5-x)(x+1)^{2}(8-x)^{3}\le0$.
\zadStop
\rozwStart{Patryk Wirkus}{}
Miejsca zerowe naszego wielomianu to: $5, -1, 8$.\\
Wielomian jest stopnia parzystego, ponadto znak współczynnika przy\linebreak najwyższej potędze x jest ujemny.\\ W związku z tym wykres wielomianu zaczyna się od lewej strony powyżej osi OX.\\
Ponadto w punkcie $-1$ wykres odbija się od osi poziomej.\\
A więc $$x \in \{-1\} \cup [5,8].$$
\rozwStop
\odpStart
$x \in \{-1\} \cup [5,8]$
\odpStop
\testStart
A.$x \in \{-1\} \cup [5,8]$\\
B.$x \in \{1\} \cup (5,8)$\\
C.$x \in \{-1\} \cup (5,8]$\\
D.$x \in \{1\} \cup (5,8]$\\
E.$x \in \{-1\} \cup [5,8)$\\
F.$x \in \{1\} \cup [5,8)$\\
G.$x \in \{-1\} \cup (5,8)$\\
H.$x \in \{1\} \cup [5,8]$
\testStop
\kluczStart
A
\kluczStop



\zadStart{Zadanie z Wikieł Z 1.62 c) moja wersja nr 104}

Rozwiązać nierówności $(5-x)(x+1)^{2}(9-x)^{3}\le0$.
\zadStop
\rozwStart{Patryk Wirkus}{}
Miejsca zerowe naszego wielomianu to: $5, -1, 9$.\\
Wielomian jest stopnia parzystego, ponadto znak współczynnika przy\linebreak najwyższej potędze x jest ujemny.\\ W związku z tym wykres wielomianu zaczyna się od lewej strony powyżej osi OX.\\
Ponadto w punkcie $-1$ wykres odbija się od osi poziomej.\\
A więc $$x \in \{-1\} \cup [5,9].$$
\rozwStop
\odpStart
$x \in \{-1\} \cup [5,9]$
\odpStop
\testStart
A.$x \in \{-1\} \cup [5,9]$\\
B.$x \in \{1\} \cup (5,9)$\\
C.$x \in \{-1\} \cup (5,9]$\\
D.$x \in \{1\} \cup (5,9]$\\
E.$x \in \{-1\} \cup [5,9)$\\
F.$x \in \{1\} \cup [5,9)$\\
G.$x \in \{-1\} \cup (5,9)$\\
H.$x \in \{1\} \cup [5,9]$
\testStop
\kluczStart
A
\kluczStop



\zadStart{Zadanie z Wikieł Z 1.62 c) moja wersja nr 105}

Rozwiązać nierówności $(5-x)(x+1)^{2}(10-x)^{3}\le0$.
\zadStop
\rozwStart{Patryk Wirkus}{}
Miejsca zerowe naszego wielomianu to: $5, -1, 10$.\\
Wielomian jest stopnia parzystego, ponadto znak współczynnika przy\linebreak najwyższej potędze x jest ujemny.\\ W związku z tym wykres wielomianu zaczyna się od lewej strony powyżej osi OX.\\
Ponadto w punkcie $-1$ wykres odbija się od osi poziomej.\\
A więc $$x \in \{-1\} \cup [5,10].$$
\rozwStop
\odpStart
$x \in \{-1\} \cup [5,10]$
\odpStop
\testStart
A.$x \in \{-1\} \cup [5,10]$\\
B.$x \in \{1\} \cup (5,10)$\\
C.$x \in \{-1\} \cup (5,10]$\\
D.$x \in \{1\} \cup (5,10]$\\
E.$x \in \{-1\} \cup [5,10)$\\
F.$x \in \{1\} \cup [5,10)$\\
G.$x \in \{-1\} \cup (5,10)$\\
H.$x \in \{1\} \cup [5,10]$
\testStop
\kluczStart
A
\kluczStop



\zadStart{Zadanie z Wikieł Z 1.62 c) moja wersja nr 106}

Rozwiązać nierówności $(5-x)(x+1)^{2}(11-x)^{3}\le0$.
\zadStop
\rozwStart{Patryk Wirkus}{}
Miejsca zerowe naszego wielomianu to: $5, -1, 11$.\\
Wielomian jest stopnia parzystego, ponadto znak współczynnika przy\linebreak najwyższej potędze x jest ujemny.\\ W związku z tym wykres wielomianu zaczyna się od lewej strony powyżej osi OX.\\
Ponadto w punkcie $-1$ wykres odbija się od osi poziomej.\\
A więc $$x \in \{-1\} \cup [5,11].$$
\rozwStop
\odpStart
$x \in \{-1\} \cup [5,11]$
\odpStop
\testStart
A.$x \in \{-1\} \cup [5,11]$\\
B.$x \in \{1\} \cup (5,11)$\\
C.$x \in \{-1\} \cup (5,11]$\\
D.$x \in \{1\} \cup (5,11]$\\
E.$x \in \{-1\} \cup [5,11)$\\
F.$x \in \{1\} \cup [5,11)$\\
G.$x \in \{-1\} \cup (5,11)$\\
H.$x \in \{1\} \cup [5,11]$
\testStop
\kluczStart
A
\kluczStop



\zadStart{Zadanie z Wikieł Z 1.62 c) moja wersja nr 107}

Rozwiązać nierówności $(5-x)(x+1)^{2}(12-x)^{3}\le0$.
\zadStop
\rozwStart{Patryk Wirkus}{}
Miejsca zerowe naszego wielomianu to: $5, -1, 12$.\\
Wielomian jest stopnia parzystego, ponadto znak współczynnika przy\linebreak najwyższej potędze x jest ujemny.\\ W związku z tym wykres wielomianu zaczyna się od lewej strony powyżej osi OX.\\
Ponadto w punkcie $-1$ wykres odbija się od osi poziomej.\\
A więc $$x \in \{-1\} \cup [5,12].$$
\rozwStop
\odpStart
$x \in \{-1\} \cup [5,12]$
\odpStop
\testStart
A.$x \in \{-1\} \cup [5,12]$\\
B.$x \in \{1\} \cup (5,12)$\\
C.$x \in \{-1\} \cup (5,12]$\\
D.$x \in \{1\} \cup (5,12]$\\
E.$x \in \{-1\} \cup [5,12)$\\
F.$x \in \{1\} \cup [5,12)$\\
G.$x \in \{-1\} \cup (5,12)$\\
H.$x \in \{1\} \cup [5,12]$
\testStop
\kluczStart
A
\kluczStop



\zadStart{Zadanie z Wikieł Z 1.62 c) moja wersja nr 108}

Rozwiązać nierówności $(5-x)(x+1)^{2}(13-x)^{3}\le0$.
\zadStop
\rozwStart{Patryk Wirkus}{}
Miejsca zerowe naszego wielomianu to: $5, -1, 13$.\\
Wielomian jest stopnia parzystego, ponadto znak współczynnika przy\linebreak najwyższej potędze x jest ujemny.\\ W związku z tym wykres wielomianu zaczyna się od lewej strony powyżej osi OX.\\
Ponadto w punkcie $-1$ wykres odbija się od osi poziomej.\\
A więc $$x \in \{-1\} \cup [5,13].$$
\rozwStop
\odpStart
$x \in \{-1\} \cup [5,13]$
\odpStop
\testStart
A.$x \in \{-1\} \cup [5,13]$\\
B.$x \in \{1\} \cup (5,13)$\\
C.$x \in \{-1\} \cup (5,13]$\\
D.$x \in \{1\} \cup (5,13]$\\
E.$x \in \{-1\} \cup [5,13)$\\
F.$x \in \{1\} \cup [5,13)$\\
G.$x \in \{-1\} \cup (5,13)$\\
H.$x \in \{1\} \cup [5,13]$
\testStop
\kluczStart
A
\kluczStop



\zadStart{Zadanie z Wikieł Z 1.62 c) moja wersja nr 109}

Rozwiązać nierówności $(5-x)(x+1)^{2}(14-x)^{3}\le0$.
\zadStop
\rozwStart{Patryk Wirkus}{}
Miejsca zerowe naszego wielomianu to: $5, -1, 14$.\\
Wielomian jest stopnia parzystego, ponadto znak współczynnika przy\linebreak najwyższej potędze x jest ujemny.\\ W związku z tym wykres wielomianu zaczyna się od lewej strony powyżej osi OX.\\
Ponadto w punkcie $-1$ wykres odbija się od osi poziomej.\\
A więc $$x \in \{-1\} \cup [5,14].$$
\rozwStop
\odpStart
$x \in \{-1\} \cup [5,14]$
\odpStop
\testStart
A.$x \in \{-1\} \cup [5,14]$\\
B.$x \in \{1\} \cup (5,14)$\\
C.$x \in \{-1\} \cup (5,14]$\\
D.$x \in \{1\} \cup (5,14]$\\
E.$x \in \{-1\} \cup [5,14)$\\
F.$x \in \{1\} \cup [5,14)$\\
G.$x \in \{-1\} \cup (5,14)$\\
H.$x \in \{1\} \cup [5,14]$
\testStop
\kluczStart
A
\kluczStop



\zadStart{Zadanie z Wikieł Z 1.62 c) moja wersja nr 110}

Rozwiązać nierówności $(5-x)(x+1)^{2}(15-x)^{3}\le0$.
\zadStop
\rozwStart{Patryk Wirkus}{}
Miejsca zerowe naszego wielomianu to: $5, -1, 15$.\\
Wielomian jest stopnia parzystego, ponadto znak współczynnika przy\linebreak najwyższej potędze x jest ujemny.\\ W związku z tym wykres wielomianu zaczyna się od lewej strony powyżej osi OX.\\
Ponadto w punkcie $-1$ wykres odbija się od osi poziomej.\\
A więc $$x \in \{-1\} \cup [5,15].$$
\rozwStop
\odpStart
$x \in \{-1\} \cup [5,15]$
\odpStop
\testStart
A.$x \in \{-1\} \cup [5,15]$\\
B.$x \in \{1\} \cup (5,15)$\\
C.$x \in \{-1\} \cup (5,15]$\\
D.$x \in \{1\} \cup (5,15]$\\
E.$x \in \{-1\} \cup [5,15)$\\
F.$x \in \{1\} \cup [5,15)$\\
G.$x \in \{-1\} \cup (5,15)$\\
H.$x \in \{1\} \cup [5,15]$
\testStop
\kluczStart
A
\kluczStop



\zadStart{Zadanie z Wikieł Z 1.62 c) moja wersja nr 111}

Rozwiązać nierówności $(5-x)(x+1)^{2}(16-x)^{3}\le0$.
\zadStop
\rozwStart{Patryk Wirkus}{}
Miejsca zerowe naszego wielomianu to: $5, -1, 16$.\\
Wielomian jest stopnia parzystego, ponadto znak współczynnika przy\linebreak najwyższej potędze x jest ujemny.\\ W związku z tym wykres wielomianu zaczyna się od lewej strony powyżej osi OX.\\
Ponadto w punkcie $-1$ wykres odbija się od osi poziomej.\\
A więc $$x \in \{-1\} \cup [5,16].$$
\rozwStop
\odpStart
$x \in \{-1\} \cup [5,16]$
\odpStop
\testStart
A.$x \in \{-1\} \cup [5,16]$\\
B.$x \in \{1\} \cup (5,16)$\\
C.$x \in \{-1\} \cup (5,16]$\\
D.$x \in \{1\} \cup (5,16]$\\
E.$x \in \{-1\} \cup [5,16)$\\
F.$x \in \{1\} \cup [5,16)$\\
G.$x \in \{-1\} \cup (5,16)$\\
H.$x \in \{1\} \cup [5,16]$
\testStop
\kluczStart
A
\kluczStop



\zadStart{Zadanie z Wikieł Z 1.62 c) moja wersja nr 112}

Rozwiązać nierówności $(5-x)(x+1)^{2}(17-x)^{3}\le0$.
\zadStop
\rozwStart{Patryk Wirkus}{}
Miejsca zerowe naszego wielomianu to: $5, -1, 17$.\\
Wielomian jest stopnia parzystego, ponadto znak współczynnika przy\linebreak najwyższej potędze x jest ujemny.\\ W związku z tym wykres wielomianu zaczyna się od lewej strony powyżej osi OX.\\
Ponadto w punkcie $-1$ wykres odbija się od osi poziomej.\\
A więc $$x \in \{-1\} \cup [5,17].$$
\rozwStop
\odpStart
$x \in \{-1\} \cup [5,17]$
\odpStop
\testStart
A.$x \in \{-1\} \cup [5,17]$\\
B.$x \in \{1\} \cup (5,17)$\\
C.$x \in \{-1\} \cup (5,17]$\\
D.$x \in \{1\} \cup (5,17]$\\
E.$x \in \{-1\} \cup [5,17)$\\
F.$x \in \{1\} \cup [5,17)$\\
G.$x \in \{-1\} \cup (5,17)$\\
H.$x \in \{1\} \cup [5,17]$
\testStop
\kluczStart
A
\kluczStop



\zadStart{Zadanie z Wikieł Z 1.62 c) moja wersja nr 113}

Rozwiązać nierówności $(5-x)(x+1)^{2}(18-x)^{3}\le0$.
\zadStop
\rozwStart{Patryk Wirkus}{}
Miejsca zerowe naszego wielomianu to: $5, -1, 18$.\\
Wielomian jest stopnia parzystego, ponadto znak współczynnika przy\linebreak najwyższej potędze x jest ujemny.\\ W związku z tym wykres wielomianu zaczyna się od lewej strony powyżej osi OX.\\
Ponadto w punkcie $-1$ wykres odbija się od osi poziomej.\\
A więc $$x \in \{-1\} \cup [5,18].$$
\rozwStop
\odpStart
$x \in \{-1\} \cup [5,18]$
\odpStop
\testStart
A.$x \in \{-1\} \cup [5,18]$\\
B.$x \in \{1\} \cup (5,18)$\\
C.$x \in \{-1\} \cup (5,18]$\\
D.$x \in \{1\} \cup (5,18]$\\
E.$x \in \{-1\} \cup [5,18)$\\
F.$x \in \{1\} \cup [5,18)$\\
G.$x \in \{-1\} \cup (5,18)$\\
H.$x \in \{1\} \cup [5,18]$
\testStop
\kluczStart
A
\kluczStop



\zadStart{Zadanie z Wikieł Z 1.62 c) moja wersja nr 114}

Rozwiązać nierówności $(5-x)(x+1)^{2}(19-x)^{3}\le0$.
\zadStop
\rozwStart{Patryk Wirkus}{}
Miejsca zerowe naszego wielomianu to: $5, -1, 19$.\\
Wielomian jest stopnia parzystego, ponadto znak współczynnika przy\linebreak najwyższej potędze x jest ujemny.\\ W związku z tym wykres wielomianu zaczyna się od lewej strony powyżej osi OX.\\
Ponadto w punkcie $-1$ wykres odbija się od osi poziomej.\\
A więc $$x \in \{-1\} \cup [5,19].$$
\rozwStop
\odpStart
$x \in \{-1\} \cup [5,19]$
\odpStop
\testStart
A.$x \in \{-1\} \cup [5,19]$\\
B.$x \in \{1\} \cup (5,19)$\\
C.$x \in \{-1\} \cup (5,19]$\\
D.$x \in \{1\} \cup (5,19]$\\
E.$x \in \{-1\} \cup [5,19)$\\
F.$x \in \{1\} \cup [5,19)$\\
G.$x \in \{-1\} \cup (5,19)$\\
H.$x \in \{1\} \cup [5,19]$
\testStop
\kluczStart
A
\kluczStop



\zadStart{Zadanie z Wikieł Z 1.62 c) moja wersja nr 115}

Rozwiązać nierówności $(5-x)(x+1)^{2}(20-x)^{3}\le0$.
\zadStop
\rozwStart{Patryk Wirkus}{}
Miejsca zerowe naszego wielomianu to: $5, -1, 20$.\\
Wielomian jest stopnia parzystego, ponadto znak współczynnika przy\linebreak najwyższej potędze x jest ujemny.\\ W związku z tym wykres wielomianu zaczyna się od lewej strony powyżej osi OX.\\
Ponadto w punkcie $-1$ wykres odbija się od osi poziomej.\\
A więc $$x \in \{-1\} \cup [5,20].$$
\rozwStop
\odpStart
$x \in \{-1\} \cup [5,20]$
\odpStop
\testStart
A.$x \in \{-1\} \cup [5,20]$\\
B.$x \in \{1\} \cup (5,20)$\\
C.$x \in \{-1\} \cup (5,20]$\\
D.$x \in \{1\} \cup (5,20]$\\
E.$x \in \{-1\} \cup [5,20)$\\
F.$x \in \{1\} \cup [5,20)$\\
G.$x \in \{-1\} \cup (5,20)$\\
H.$x \in \{1\} \cup [5,20]$
\testStop
\kluczStart
A
\kluczStop



\zadStart{Zadanie z Wikieł Z 1.62 c) moja wersja nr 116}

Rozwiązać nierówności $(5-x)(x+2)^{2}(6-x)^{3}\le0$.
\zadStop
\rozwStart{Patryk Wirkus}{}
Miejsca zerowe naszego wielomianu to: $5, -2, 6$.\\
Wielomian jest stopnia parzystego, ponadto znak współczynnika przy\linebreak najwyższej potędze x jest ujemny.\\ W związku z tym wykres wielomianu zaczyna się od lewej strony powyżej osi OX.\\
Ponadto w punkcie $-2$ wykres odbija się od osi poziomej.\\
A więc $$x \in \{-2\} \cup [5,6].$$
\rozwStop
\odpStart
$x \in \{-2\} \cup [5,6]$
\odpStop
\testStart
A.$x \in \{-2\} \cup [5,6]$\\
B.$x \in \{2\} \cup (5,6)$\\
C.$x \in \{-2\} \cup (5,6]$\\
D.$x \in \{2\} \cup (5,6]$\\
E.$x \in \{-2\} \cup [5,6)$\\
F.$x \in \{2\} \cup [5,6)$\\
G.$x \in \{-2\} \cup (5,6)$\\
H.$x \in \{2\} \cup [5,6]$
\testStop
\kluczStart
A
\kluczStop



\zadStart{Zadanie z Wikieł Z 1.62 c) moja wersja nr 117}

Rozwiązać nierówności $(5-x)(x+2)^{2}(7-x)^{3}\le0$.
\zadStop
\rozwStart{Patryk Wirkus}{}
Miejsca zerowe naszego wielomianu to: $5, -2, 7$.\\
Wielomian jest stopnia parzystego, ponadto znak współczynnika przy\linebreak najwyższej potędze x jest ujemny.\\ W związku z tym wykres wielomianu zaczyna się od lewej strony powyżej osi OX.\\
Ponadto w punkcie $-2$ wykres odbija się od osi poziomej.\\
A więc $$x \in \{-2\} \cup [5,7].$$
\rozwStop
\odpStart
$x \in \{-2\} \cup [5,7]$
\odpStop
\testStart
A.$x \in \{-2\} \cup [5,7]$\\
B.$x \in \{2\} \cup (5,7)$\\
C.$x \in \{-2\} \cup (5,7]$\\
D.$x \in \{2\} \cup (5,7]$\\
E.$x \in \{-2\} \cup [5,7)$\\
F.$x \in \{2\} \cup [5,7)$\\
G.$x \in \{-2\} \cup (5,7)$\\
H.$x \in \{2\} \cup [5,7]$
\testStop
\kluczStart
A
\kluczStop



\zadStart{Zadanie z Wikieł Z 1.62 c) moja wersja nr 118}

Rozwiązać nierówności $(5-x)(x+2)^{2}(8-x)^{3}\le0$.
\zadStop
\rozwStart{Patryk Wirkus}{}
Miejsca zerowe naszego wielomianu to: $5, -2, 8$.\\
Wielomian jest stopnia parzystego, ponadto znak współczynnika przy\linebreak najwyższej potędze x jest ujemny.\\ W związku z tym wykres wielomianu zaczyna się od lewej strony powyżej osi OX.\\
Ponadto w punkcie $-2$ wykres odbija się od osi poziomej.\\
A więc $$x \in \{-2\} \cup [5,8].$$
\rozwStop
\odpStart
$x \in \{-2\} \cup [5,8]$
\odpStop
\testStart
A.$x \in \{-2\} \cup [5,8]$\\
B.$x \in \{2\} \cup (5,8)$\\
C.$x \in \{-2\} \cup (5,8]$\\
D.$x \in \{2\} \cup (5,8]$\\
E.$x \in \{-2\} \cup [5,8)$\\
F.$x \in \{2\} \cup [5,8)$\\
G.$x \in \{-2\} \cup (5,8)$\\
H.$x \in \{2\} \cup [5,8]$
\testStop
\kluczStart
A
\kluczStop



\zadStart{Zadanie z Wikieł Z 1.62 c) moja wersja nr 119}

Rozwiązać nierówności $(5-x)(x+2)^{2}(9-x)^{3}\le0$.
\zadStop
\rozwStart{Patryk Wirkus}{}
Miejsca zerowe naszego wielomianu to: $5, -2, 9$.\\
Wielomian jest stopnia parzystego, ponadto znak współczynnika przy\linebreak najwyższej potędze x jest ujemny.\\ W związku z tym wykres wielomianu zaczyna się od lewej strony powyżej osi OX.\\
Ponadto w punkcie $-2$ wykres odbija się od osi poziomej.\\
A więc $$x \in \{-2\} \cup [5,9].$$
\rozwStop
\odpStart
$x \in \{-2\} \cup [5,9]$
\odpStop
\testStart
A.$x \in \{-2\} \cup [5,9]$\\
B.$x \in \{2\} \cup (5,9)$\\
C.$x \in \{-2\} \cup (5,9]$\\
D.$x \in \{2\} \cup (5,9]$\\
E.$x \in \{-2\} \cup [5,9)$\\
F.$x \in \{2\} \cup [5,9)$\\
G.$x \in \{-2\} \cup (5,9)$\\
H.$x \in \{2\} \cup [5,9]$
\testStop
\kluczStart
A
\kluczStop



\zadStart{Zadanie z Wikieł Z 1.62 c) moja wersja nr 120}

Rozwiązać nierówności $(5-x)(x+2)^{2}(10-x)^{3}\le0$.
\zadStop
\rozwStart{Patryk Wirkus}{}
Miejsca zerowe naszego wielomianu to: $5, -2, 10$.\\
Wielomian jest stopnia parzystego, ponadto znak współczynnika przy\linebreak najwyższej potędze x jest ujemny.\\ W związku z tym wykres wielomianu zaczyna się od lewej strony powyżej osi OX.\\
Ponadto w punkcie $-2$ wykres odbija się od osi poziomej.\\
A więc $$x \in \{-2\} \cup [5,10].$$
\rozwStop
\odpStart
$x \in \{-2\} \cup [5,10]$
\odpStop
\testStart
A.$x \in \{-2\} \cup [5,10]$\\
B.$x \in \{2\} \cup (5,10)$\\
C.$x \in \{-2\} \cup (5,10]$\\
D.$x \in \{2\} \cup (5,10]$\\
E.$x \in \{-2\} \cup [5,10)$\\
F.$x \in \{2\} \cup [5,10)$\\
G.$x \in \{-2\} \cup (5,10)$\\
H.$x \in \{2\} \cup [5,10]$
\testStop
\kluczStart
A
\kluczStop



\zadStart{Zadanie z Wikieł Z 1.62 c) moja wersja nr 121}

Rozwiązać nierówności $(5-x)(x+2)^{2}(11-x)^{3}\le0$.
\zadStop
\rozwStart{Patryk Wirkus}{}
Miejsca zerowe naszego wielomianu to: $5, -2, 11$.\\
Wielomian jest stopnia parzystego, ponadto znak współczynnika przy\linebreak najwyższej potędze x jest ujemny.\\ W związku z tym wykres wielomianu zaczyna się od lewej strony powyżej osi OX.\\
Ponadto w punkcie $-2$ wykres odbija się od osi poziomej.\\
A więc $$x \in \{-2\} \cup [5,11].$$
\rozwStop
\odpStart
$x \in \{-2\} \cup [5,11]$
\odpStop
\testStart
A.$x \in \{-2\} \cup [5,11]$\\
B.$x \in \{2\} \cup (5,11)$\\
C.$x \in \{-2\} \cup (5,11]$\\
D.$x \in \{2\} \cup (5,11]$\\
E.$x \in \{-2\} \cup [5,11)$\\
F.$x \in \{2\} \cup [5,11)$\\
G.$x \in \{-2\} \cup (5,11)$\\
H.$x \in \{2\} \cup [5,11]$
\testStop
\kluczStart
A
\kluczStop



\zadStart{Zadanie z Wikieł Z 1.62 c) moja wersja nr 122}

Rozwiązać nierówności $(5-x)(x+2)^{2}(12-x)^{3}\le0$.
\zadStop
\rozwStart{Patryk Wirkus}{}
Miejsca zerowe naszego wielomianu to: $5, -2, 12$.\\
Wielomian jest stopnia parzystego, ponadto znak współczynnika przy\linebreak najwyższej potędze x jest ujemny.\\ W związku z tym wykres wielomianu zaczyna się od lewej strony powyżej osi OX.\\
Ponadto w punkcie $-2$ wykres odbija się od osi poziomej.\\
A więc $$x \in \{-2\} \cup [5,12].$$
\rozwStop
\odpStart
$x \in \{-2\} \cup [5,12]$
\odpStop
\testStart
A.$x \in \{-2\} \cup [5,12]$\\
B.$x \in \{2\} \cup (5,12)$\\
C.$x \in \{-2\} \cup (5,12]$\\
D.$x \in \{2\} \cup (5,12]$\\
E.$x \in \{-2\} \cup [5,12)$\\
F.$x \in \{2\} \cup [5,12)$\\
G.$x \in \{-2\} \cup (5,12)$\\
H.$x \in \{2\} \cup [5,12]$
\testStop
\kluczStart
A
\kluczStop



\zadStart{Zadanie z Wikieł Z 1.62 c) moja wersja nr 123}

Rozwiązać nierówności $(5-x)(x+2)^{2}(13-x)^{3}\le0$.
\zadStop
\rozwStart{Patryk Wirkus}{}
Miejsca zerowe naszego wielomianu to: $5, -2, 13$.\\
Wielomian jest stopnia parzystego, ponadto znak współczynnika przy\linebreak najwyższej potędze x jest ujemny.\\ W związku z tym wykres wielomianu zaczyna się od lewej strony powyżej osi OX.\\
Ponadto w punkcie $-2$ wykres odbija się od osi poziomej.\\
A więc $$x \in \{-2\} \cup [5,13].$$
\rozwStop
\odpStart
$x \in \{-2\} \cup [5,13]$
\odpStop
\testStart
A.$x \in \{-2\} \cup [5,13]$\\
B.$x \in \{2\} \cup (5,13)$\\
C.$x \in \{-2\} \cup (5,13]$\\
D.$x \in \{2\} \cup (5,13]$\\
E.$x \in \{-2\} \cup [5,13)$\\
F.$x \in \{2\} \cup [5,13)$\\
G.$x \in \{-2\} \cup (5,13)$\\
H.$x \in \{2\} \cup [5,13]$
\testStop
\kluczStart
A
\kluczStop



\zadStart{Zadanie z Wikieł Z 1.62 c) moja wersja nr 124}

Rozwiązać nierówności $(5-x)(x+2)^{2}(14-x)^{3}\le0$.
\zadStop
\rozwStart{Patryk Wirkus}{}
Miejsca zerowe naszego wielomianu to: $5, -2, 14$.\\
Wielomian jest stopnia parzystego, ponadto znak współczynnika przy\linebreak najwyższej potędze x jest ujemny.\\ W związku z tym wykres wielomianu zaczyna się od lewej strony powyżej osi OX.\\
Ponadto w punkcie $-2$ wykres odbija się od osi poziomej.\\
A więc $$x \in \{-2\} \cup [5,14].$$
\rozwStop
\odpStart
$x \in \{-2\} \cup [5,14]$
\odpStop
\testStart
A.$x \in \{-2\} \cup [5,14]$\\
B.$x \in \{2\} \cup (5,14)$\\
C.$x \in \{-2\} \cup (5,14]$\\
D.$x \in \{2\} \cup (5,14]$\\
E.$x \in \{-2\} \cup [5,14)$\\
F.$x \in \{2\} \cup [5,14)$\\
G.$x \in \{-2\} \cup (5,14)$\\
H.$x \in \{2\} \cup [5,14]$
\testStop
\kluczStart
A
\kluczStop



\zadStart{Zadanie z Wikieł Z 1.62 c) moja wersja nr 125}

Rozwiązać nierówności $(5-x)(x+2)^{2}(15-x)^{3}\le0$.
\zadStop
\rozwStart{Patryk Wirkus}{}
Miejsca zerowe naszego wielomianu to: $5, -2, 15$.\\
Wielomian jest stopnia parzystego, ponadto znak współczynnika przy\linebreak najwyższej potędze x jest ujemny.\\ W związku z tym wykres wielomianu zaczyna się od lewej strony powyżej osi OX.\\
Ponadto w punkcie $-2$ wykres odbija się od osi poziomej.\\
A więc $$x \in \{-2\} \cup [5,15].$$
\rozwStop
\odpStart
$x \in \{-2\} \cup [5,15]$
\odpStop
\testStart
A.$x \in \{-2\} \cup [5,15]$\\
B.$x \in \{2\} \cup (5,15)$\\
C.$x \in \{-2\} \cup (5,15]$\\
D.$x \in \{2\} \cup (5,15]$\\
E.$x \in \{-2\} \cup [5,15)$\\
F.$x \in \{2\} \cup [5,15)$\\
G.$x \in \{-2\} \cup (5,15)$\\
H.$x \in \{2\} \cup [5,15]$
\testStop
\kluczStart
A
\kluczStop



\zadStart{Zadanie z Wikieł Z 1.62 c) moja wersja nr 126}

Rozwiązać nierówności $(5-x)(x+2)^{2}(16-x)^{3}\le0$.
\zadStop
\rozwStart{Patryk Wirkus}{}
Miejsca zerowe naszego wielomianu to: $5, -2, 16$.\\
Wielomian jest stopnia parzystego, ponadto znak współczynnika przy\linebreak najwyższej potędze x jest ujemny.\\ W związku z tym wykres wielomianu zaczyna się od lewej strony powyżej osi OX.\\
Ponadto w punkcie $-2$ wykres odbija się od osi poziomej.\\
A więc $$x \in \{-2\} \cup [5,16].$$
\rozwStop
\odpStart
$x \in \{-2\} \cup [5,16]$
\odpStop
\testStart
A.$x \in \{-2\} \cup [5,16]$\\
B.$x \in \{2\} \cup (5,16)$\\
C.$x \in \{-2\} \cup (5,16]$\\
D.$x \in \{2\} \cup (5,16]$\\
E.$x \in \{-2\} \cup [5,16)$\\
F.$x \in \{2\} \cup [5,16)$\\
G.$x \in \{-2\} \cup (5,16)$\\
H.$x \in \{2\} \cup [5,16]$
\testStop
\kluczStart
A
\kluczStop



\zadStart{Zadanie z Wikieł Z 1.62 c) moja wersja nr 127}

Rozwiązać nierówności $(5-x)(x+2)^{2}(17-x)^{3}\le0$.
\zadStop
\rozwStart{Patryk Wirkus}{}
Miejsca zerowe naszego wielomianu to: $5, -2, 17$.\\
Wielomian jest stopnia parzystego, ponadto znak współczynnika przy\linebreak najwyższej potędze x jest ujemny.\\ W związku z tym wykres wielomianu zaczyna się od lewej strony powyżej osi OX.\\
Ponadto w punkcie $-2$ wykres odbija się od osi poziomej.\\
A więc $$x \in \{-2\} \cup [5,17].$$
\rozwStop
\odpStart
$x \in \{-2\} \cup [5,17]$
\odpStop
\testStart
A.$x \in \{-2\} \cup [5,17]$\\
B.$x \in \{2\} \cup (5,17)$\\
C.$x \in \{-2\} \cup (5,17]$\\
D.$x \in \{2\} \cup (5,17]$\\
E.$x \in \{-2\} \cup [5,17)$\\
F.$x \in \{2\} \cup [5,17)$\\
G.$x \in \{-2\} \cup (5,17)$\\
H.$x \in \{2\} \cup [5,17]$
\testStop
\kluczStart
A
\kluczStop



\zadStart{Zadanie z Wikieł Z 1.62 c) moja wersja nr 128}

Rozwiązać nierówności $(5-x)(x+2)^{2}(18-x)^{3}\le0$.
\zadStop
\rozwStart{Patryk Wirkus}{}
Miejsca zerowe naszego wielomianu to: $5, -2, 18$.\\
Wielomian jest stopnia parzystego, ponadto znak współczynnika przy\linebreak najwyższej potędze x jest ujemny.\\ W związku z tym wykres wielomianu zaczyna się od lewej strony powyżej osi OX.\\
Ponadto w punkcie $-2$ wykres odbija się od osi poziomej.\\
A więc $$x \in \{-2\} \cup [5,18].$$
\rozwStop
\odpStart
$x \in \{-2\} \cup [5,18]$
\odpStop
\testStart
A.$x \in \{-2\} \cup [5,18]$\\
B.$x \in \{2\} \cup (5,18)$\\
C.$x \in \{-2\} \cup (5,18]$\\
D.$x \in \{2\} \cup (5,18]$\\
E.$x \in \{-2\} \cup [5,18)$\\
F.$x \in \{2\} \cup [5,18)$\\
G.$x \in \{-2\} \cup (5,18)$\\
H.$x \in \{2\} \cup [5,18]$
\testStop
\kluczStart
A
\kluczStop



\zadStart{Zadanie z Wikieł Z 1.62 c) moja wersja nr 129}

Rozwiązać nierówności $(5-x)(x+2)^{2}(19-x)^{3}\le0$.
\zadStop
\rozwStart{Patryk Wirkus}{}
Miejsca zerowe naszego wielomianu to: $5, -2, 19$.\\
Wielomian jest stopnia parzystego, ponadto znak współczynnika przy\linebreak najwyższej potędze x jest ujemny.\\ W związku z tym wykres wielomianu zaczyna się od lewej strony powyżej osi OX.\\
Ponadto w punkcie $-2$ wykres odbija się od osi poziomej.\\
A więc $$x \in \{-2\} \cup [5,19].$$
\rozwStop
\odpStart
$x \in \{-2\} \cup [5,19]$
\odpStop
\testStart
A.$x \in \{-2\} \cup [5,19]$\\
B.$x \in \{2\} \cup (5,19)$\\
C.$x \in \{-2\} \cup (5,19]$\\
D.$x \in \{2\} \cup (5,19]$\\
E.$x \in \{-2\} \cup [5,19)$\\
F.$x \in \{2\} \cup [5,19)$\\
G.$x \in \{-2\} \cup (5,19)$\\
H.$x \in \{2\} \cup [5,19]$
\testStop
\kluczStart
A
\kluczStop



\zadStart{Zadanie z Wikieł Z 1.62 c) moja wersja nr 130}

Rozwiązać nierówności $(5-x)(x+2)^{2}(20-x)^{3}\le0$.
\zadStop
\rozwStart{Patryk Wirkus}{}
Miejsca zerowe naszego wielomianu to: $5, -2, 20$.\\
Wielomian jest stopnia parzystego, ponadto znak współczynnika przy\linebreak najwyższej potędze x jest ujemny.\\ W związku z tym wykres wielomianu zaczyna się od lewej strony powyżej osi OX.\\
Ponadto w punkcie $-2$ wykres odbija się od osi poziomej.\\
A więc $$x \in \{-2\} \cup [5,20].$$
\rozwStop
\odpStart
$x \in \{-2\} \cup [5,20]$
\odpStop
\testStart
A.$x \in \{-2\} \cup [5,20]$\\
B.$x \in \{2\} \cup (5,20)$\\
C.$x \in \{-2\} \cup (5,20]$\\
D.$x \in \{2\} \cup (5,20]$\\
E.$x \in \{-2\} \cup [5,20)$\\
F.$x \in \{2\} \cup [5,20)$\\
G.$x \in \{-2\} \cup (5,20)$\\
H.$x \in \{2\} \cup [5,20]$
\testStop
\kluczStart
A
\kluczStop



\zadStart{Zadanie z Wikieł Z 1.62 c) moja wersja nr 131}

Rozwiązać nierówności $(5-x)(x+3)^{2}(6-x)^{3}\le0$.
\zadStop
\rozwStart{Patryk Wirkus}{}
Miejsca zerowe naszego wielomianu to: $5, -3, 6$.\\
Wielomian jest stopnia parzystego, ponadto znak współczynnika przy\linebreak najwyższej potędze x jest ujemny.\\ W związku z tym wykres wielomianu zaczyna się od lewej strony powyżej osi OX.\\
Ponadto w punkcie $-3$ wykres odbija się od osi poziomej.\\
A więc $$x \in \{-3\} \cup [5,6].$$
\rozwStop
\odpStart
$x \in \{-3\} \cup [5,6]$
\odpStop
\testStart
A.$x \in \{-3\} \cup [5,6]$\\
B.$x \in \{3\} \cup (5,6)$\\
C.$x \in \{-3\} \cup (5,6]$\\
D.$x \in \{3\} \cup (5,6]$\\
E.$x \in \{-3\} \cup [5,6)$\\
F.$x \in \{3\} \cup [5,6)$\\
G.$x \in \{-3\} \cup (5,6)$\\
H.$x \in \{3\} \cup [5,6]$
\testStop
\kluczStart
A
\kluczStop



\zadStart{Zadanie z Wikieł Z 1.62 c) moja wersja nr 132}

Rozwiązać nierówności $(5-x)(x+3)^{2}(7-x)^{3}\le0$.
\zadStop
\rozwStart{Patryk Wirkus}{}
Miejsca zerowe naszego wielomianu to: $5, -3, 7$.\\
Wielomian jest stopnia parzystego, ponadto znak współczynnika przy\linebreak najwyższej potędze x jest ujemny.\\ W związku z tym wykres wielomianu zaczyna się od lewej strony powyżej osi OX.\\
Ponadto w punkcie $-3$ wykres odbija się od osi poziomej.\\
A więc $$x \in \{-3\} \cup [5,7].$$
\rozwStop
\odpStart
$x \in \{-3\} \cup [5,7]$
\odpStop
\testStart
A.$x \in \{-3\} \cup [5,7]$\\
B.$x \in \{3\} \cup (5,7)$\\
C.$x \in \{-3\} \cup (5,7]$\\
D.$x \in \{3\} \cup (5,7]$\\
E.$x \in \{-3\} \cup [5,7)$\\
F.$x \in \{3\} \cup [5,7)$\\
G.$x \in \{-3\} \cup (5,7)$\\
H.$x \in \{3\} \cup [5,7]$
\testStop
\kluczStart
A
\kluczStop



\zadStart{Zadanie z Wikieł Z 1.62 c) moja wersja nr 133}

Rozwiązać nierówności $(5-x)(x+3)^{2}(8-x)^{3}\le0$.
\zadStop
\rozwStart{Patryk Wirkus}{}
Miejsca zerowe naszego wielomianu to: $5, -3, 8$.\\
Wielomian jest stopnia parzystego, ponadto znak współczynnika przy\linebreak najwyższej potędze x jest ujemny.\\ W związku z tym wykres wielomianu zaczyna się od lewej strony powyżej osi OX.\\
Ponadto w punkcie $-3$ wykres odbija się od osi poziomej.\\
A więc $$x \in \{-3\} \cup [5,8].$$
\rozwStop
\odpStart
$x \in \{-3\} \cup [5,8]$
\odpStop
\testStart
A.$x \in \{-3\} \cup [5,8]$\\
B.$x \in \{3\} \cup (5,8)$\\
C.$x \in \{-3\} \cup (5,8]$\\
D.$x \in \{3\} \cup (5,8]$\\
E.$x \in \{-3\} \cup [5,8)$\\
F.$x \in \{3\} \cup [5,8)$\\
G.$x \in \{-3\} \cup (5,8)$\\
H.$x \in \{3\} \cup [5,8]$
\testStop
\kluczStart
A
\kluczStop



\zadStart{Zadanie z Wikieł Z 1.62 c) moja wersja nr 134}

Rozwiązać nierówności $(5-x)(x+3)^{2}(9-x)^{3}\le0$.
\zadStop
\rozwStart{Patryk Wirkus}{}
Miejsca zerowe naszego wielomianu to: $5, -3, 9$.\\
Wielomian jest stopnia parzystego, ponadto znak współczynnika przy\linebreak najwyższej potędze x jest ujemny.\\ W związku z tym wykres wielomianu zaczyna się od lewej strony powyżej osi OX.\\
Ponadto w punkcie $-3$ wykres odbija się od osi poziomej.\\
A więc $$x \in \{-3\} \cup [5,9].$$
\rozwStop
\odpStart
$x \in \{-3\} \cup [5,9]$
\odpStop
\testStart
A.$x \in \{-3\} \cup [5,9]$\\
B.$x \in \{3\} \cup (5,9)$\\
C.$x \in \{-3\} \cup (5,9]$\\
D.$x \in \{3\} \cup (5,9]$\\
E.$x \in \{-3\} \cup [5,9)$\\
F.$x \in \{3\} \cup [5,9)$\\
G.$x \in \{-3\} \cup (5,9)$\\
H.$x \in \{3\} \cup [5,9]$
\testStop
\kluczStart
A
\kluczStop



\zadStart{Zadanie z Wikieł Z 1.62 c) moja wersja nr 135}

Rozwiązać nierówności $(5-x)(x+3)^{2}(10-x)^{3}\le0$.
\zadStop
\rozwStart{Patryk Wirkus}{}
Miejsca zerowe naszego wielomianu to: $5, -3, 10$.\\
Wielomian jest stopnia parzystego, ponadto znak współczynnika przy\linebreak najwyższej potędze x jest ujemny.\\ W związku z tym wykres wielomianu zaczyna się od lewej strony powyżej osi OX.\\
Ponadto w punkcie $-3$ wykres odbija się od osi poziomej.\\
A więc $$x \in \{-3\} \cup [5,10].$$
\rozwStop
\odpStart
$x \in \{-3\} \cup [5,10]$
\odpStop
\testStart
A.$x \in \{-3\} \cup [5,10]$\\
B.$x \in \{3\} \cup (5,10)$\\
C.$x \in \{-3\} \cup (5,10]$\\
D.$x \in \{3\} \cup (5,10]$\\
E.$x \in \{-3\} \cup [5,10)$\\
F.$x \in \{3\} \cup [5,10)$\\
G.$x \in \{-3\} \cup (5,10)$\\
H.$x \in \{3\} \cup [5,10]$
\testStop
\kluczStart
A
\kluczStop



\zadStart{Zadanie z Wikieł Z 1.62 c) moja wersja nr 136}

Rozwiązać nierówności $(5-x)(x+3)^{2}(11-x)^{3}\le0$.
\zadStop
\rozwStart{Patryk Wirkus}{}
Miejsca zerowe naszego wielomianu to: $5, -3, 11$.\\
Wielomian jest stopnia parzystego, ponadto znak współczynnika przy\linebreak najwyższej potędze x jest ujemny.\\ W związku z tym wykres wielomianu zaczyna się od lewej strony powyżej osi OX.\\
Ponadto w punkcie $-3$ wykres odbija się od osi poziomej.\\
A więc $$x \in \{-3\} \cup [5,11].$$
\rozwStop
\odpStart
$x \in \{-3\} \cup [5,11]$
\odpStop
\testStart
A.$x \in \{-3\} \cup [5,11]$\\
B.$x \in \{3\} \cup (5,11)$\\
C.$x \in \{-3\} \cup (5,11]$\\
D.$x \in \{3\} \cup (5,11]$\\
E.$x \in \{-3\} \cup [5,11)$\\
F.$x \in \{3\} \cup [5,11)$\\
G.$x \in \{-3\} \cup (5,11)$\\
H.$x \in \{3\} \cup [5,11]$
\testStop
\kluczStart
A
\kluczStop



\zadStart{Zadanie z Wikieł Z 1.62 c) moja wersja nr 137}

Rozwiązać nierówności $(5-x)(x+3)^{2}(12-x)^{3}\le0$.
\zadStop
\rozwStart{Patryk Wirkus}{}
Miejsca zerowe naszego wielomianu to: $5, -3, 12$.\\
Wielomian jest stopnia parzystego, ponadto znak współczynnika przy\linebreak najwyższej potędze x jest ujemny.\\ W związku z tym wykres wielomianu zaczyna się od lewej strony powyżej osi OX.\\
Ponadto w punkcie $-3$ wykres odbija się od osi poziomej.\\
A więc $$x \in \{-3\} \cup [5,12].$$
\rozwStop
\odpStart
$x \in \{-3\} \cup [5,12]$
\odpStop
\testStart
A.$x \in \{-3\} \cup [5,12]$\\
B.$x \in \{3\} \cup (5,12)$\\
C.$x \in \{-3\} \cup (5,12]$\\
D.$x \in \{3\} \cup (5,12]$\\
E.$x \in \{-3\} \cup [5,12)$\\
F.$x \in \{3\} \cup [5,12)$\\
G.$x \in \{-3\} \cup (5,12)$\\
H.$x \in \{3\} \cup [5,12]$
\testStop
\kluczStart
A
\kluczStop



\zadStart{Zadanie z Wikieł Z 1.62 c) moja wersja nr 138}

Rozwiązać nierówności $(5-x)(x+3)^{2}(13-x)^{3}\le0$.
\zadStop
\rozwStart{Patryk Wirkus}{}
Miejsca zerowe naszego wielomianu to: $5, -3, 13$.\\
Wielomian jest stopnia parzystego, ponadto znak współczynnika przy\linebreak najwyższej potędze x jest ujemny.\\ W związku z tym wykres wielomianu zaczyna się od lewej strony powyżej osi OX.\\
Ponadto w punkcie $-3$ wykres odbija się od osi poziomej.\\
A więc $$x \in \{-3\} \cup [5,13].$$
\rozwStop
\odpStart
$x \in \{-3\} \cup [5,13]$
\odpStop
\testStart
A.$x \in \{-3\} \cup [5,13]$\\
B.$x \in \{3\} \cup (5,13)$\\
C.$x \in \{-3\} \cup (5,13]$\\
D.$x \in \{3\} \cup (5,13]$\\
E.$x \in \{-3\} \cup [5,13)$\\
F.$x \in \{3\} \cup [5,13)$\\
G.$x \in \{-3\} \cup (5,13)$\\
H.$x \in \{3\} \cup [5,13]$
\testStop
\kluczStart
A
\kluczStop



\zadStart{Zadanie z Wikieł Z 1.62 c) moja wersja nr 139}

Rozwiązać nierówności $(5-x)(x+3)^{2}(14-x)^{3}\le0$.
\zadStop
\rozwStart{Patryk Wirkus}{}
Miejsca zerowe naszego wielomianu to: $5, -3, 14$.\\
Wielomian jest stopnia parzystego, ponadto znak współczynnika przy\linebreak najwyższej potędze x jest ujemny.\\ W związku z tym wykres wielomianu zaczyna się od lewej strony powyżej osi OX.\\
Ponadto w punkcie $-3$ wykres odbija się od osi poziomej.\\
A więc $$x \in \{-3\} \cup [5,14].$$
\rozwStop
\odpStart
$x \in \{-3\} \cup [5,14]$
\odpStop
\testStart
A.$x \in \{-3\} \cup [5,14]$\\
B.$x \in \{3\} \cup (5,14)$\\
C.$x \in \{-3\} \cup (5,14]$\\
D.$x \in \{3\} \cup (5,14]$\\
E.$x \in \{-3\} \cup [5,14)$\\
F.$x \in \{3\} \cup [5,14)$\\
G.$x \in \{-3\} \cup (5,14)$\\
H.$x \in \{3\} \cup [5,14]$
\testStop
\kluczStart
A
\kluczStop



\zadStart{Zadanie z Wikieł Z 1.62 c) moja wersja nr 140}

Rozwiązać nierówności $(5-x)(x+3)^{2}(15-x)^{3}\le0$.
\zadStop
\rozwStart{Patryk Wirkus}{}
Miejsca zerowe naszego wielomianu to: $5, -3, 15$.\\
Wielomian jest stopnia parzystego, ponadto znak współczynnika przy\linebreak najwyższej potędze x jest ujemny.\\ W związku z tym wykres wielomianu zaczyna się od lewej strony powyżej osi OX.\\
Ponadto w punkcie $-3$ wykres odbija się od osi poziomej.\\
A więc $$x \in \{-3\} \cup [5,15].$$
\rozwStop
\odpStart
$x \in \{-3\} \cup [5,15]$
\odpStop
\testStart
A.$x \in \{-3\} \cup [5,15]$\\
B.$x \in \{3\} \cup (5,15)$\\
C.$x \in \{-3\} \cup (5,15]$\\
D.$x \in \{3\} \cup (5,15]$\\
E.$x \in \{-3\} \cup [5,15)$\\
F.$x \in \{3\} \cup [5,15)$\\
G.$x \in \{-3\} \cup (5,15)$\\
H.$x \in \{3\} \cup [5,15]$
\testStop
\kluczStart
A
\kluczStop



\zadStart{Zadanie z Wikieł Z 1.62 c) moja wersja nr 141}

Rozwiązać nierówności $(5-x)(x+3)^{2}(16-x)^{3}\le0$.
\zadStop
\rozwStart{Patryk Wirkus}{}
Miejsca zerowe naszego wielomianu to: $5, -3, 16$.\\
Wielomian jest stopnia parzystego, ponadto znak współczynnika przy\linebreak najwyższej potędze x jest ujemny.\\ W związku z tym wykres wielomianu zaczyna się od lewej strony powyżej osi OX.\\
Ponadto w punkcie $-3$ wykres odbija się od osi poziomej.\\
A więc $$x \in \{-3\} \cup [5,16].$$
\rozwStop
\odpStart
$x \in \{-3\} \cup [5,16]$
\odpStop
\testStart
A.$x \in \{-3\} \cup [5,16]$\\
B.$x \in \{3\} \cup (5,16)$\\
C.$x \in \{-3\} \cup (5,16]$\\
D.$x \in \{3\} \cup (5,16]$\\
E.$x \in \{-3\} \cup [5,16)$\\
F.$x \in \{3\} \cup [5,16)$\\
G.$x \in \{-3\} \cup (5,16)$\\
H.$x \in \{3\} \cup [5,16]$
\testStop
\kluczStart
A
\kluczStop



\zadStart{Zadanie z Wikieł Z 1.62 c) moja wersja nr 142}

Rozwiązać nierówności $(5-x)(x+3)^{2}(17-x)^{3}\le0$.
\zadStop
\rozwStart{Patryk Wirkus}{}
Miejsca zerowe naszego wielomianu to: $5, -3, 17$.\\
Wielomian jest stopnia parzystego, ponadto znak współczynnika przy\linebreak najwyższej potędze x jest ujemny.\\ W związku z tym wykres wielomianu zaczyna się od lewej strony powyżej osi OX.\\
Ponadto w punkcie $-3$ wykres odbija się od osi poziomej.\\
A więc $$x \in \{-3\} \cup [5,17].$$
\rozwStop
\odpStart
$x \in \{-3\} \cup [5,17]$
\odpStop
\testStart
A.$x \in \{-3\} \cup [5,17]$\\
B.$x \in \{3\} \cup (5,17)$\\
C.$x \in \{-3\} \cup (5,17]$\\
D.$x \in \{3\} \cup (5,17]$\\
E.$x \in \{-3\} \cup [5,17)$\\
F.$x \in \{3\} \cup [5,17)$\\
G.$x \in \{-3\} \cup (5,17)$\\
H.$x \in \{3\} \cup [5,17]$
\testStop
\kluczStart
A
\kluczStop



\zadStart{Zadanie z Wikieł Z 1.62 c) moja wersja nr 143}

Rozwiązać nierówności $(5-x)(x+3)^{2}(18-x)^{3}\le0$.
\zadStop
\rozwStart{Patryk Wirkus}{}
Miejsca zerowe naszego wielomianu to: $5, -3, 18$.\\
Wielomian jest stopnia parzystego, ponadto znak współczynnika przy\linebreak najwyższej potędze x jest ujemny.\\ W związku z tym wykres wielomianu zaczyna się od lewej strony powyżej osi OX.\\
Ponadto w punkcie $-3$ wykres odbija się od osi poziomej.\\
A więc $$x \in \{-3\} \cup [5,18].$$
\rozwStop
\odpStart
$x \in \{-3\} \cup [5,18]$
\odpStop
\testStart
A.$x \in \{-3\} \cup [5,18]$\\
B.$x \in \{3\} \cup (5,18)$\\
C.$x \in \{-3\} \cup (5,18]$\\
D.$x \in \{3\} \cup (5,18]$\\
E.$x \in \{-3\} \cup [5,18)$\\
F.$x \in \{3\} \cup [5,18)$\\
G.$x \in \{-3\} \cup (5,18)$\\
H.$x \in \{3\} \cup [5,18]$
\testStop
\kluczStart
A
\kluczStop



\zadStart{Zadanie z Wikieł Z 1.62 c) moja wersja nr 144}

Rozwiązać nierówności $(5-x)(x+3)^{2}(19-x)^{3}\le0$.
\zadStop
\rozwStart{Patryk Wirkus}{}
Miejsca zerowe naszego wielomianu to: $5, -3, 19$.\\
Wielomian jest stopnia parzystego, ponadto znak współczynnika przy\linebreak najwyższej potędze x jest ujemny.\\ W związku z tym wykres wielomianu zaczyna się od lewej strony powyżej osi OX.\\
Ponadto w punkcie $-3$ wykres odbija się od osi poziomej.\\
A więc $$x \in \{-3\} \cup [5,19].$$
\rozwStop
\odpStart
$x \in \{-3\} \cup [5,19]$
\odpStop
\testStart
A.$x \in \{-3\} \cup [5,19]$\\
B.$x \in \{3\} \cup (5,19)$\\
C.$x \in \{-3\} \cup (5,19]$\\
D.$x \in \{3\} \cup (5,19]$\\
E.$x \in \{-3\} \cup [5,19)$\\
F.$x \in \{3\} \cup [5,19)$\\
G.$x \in \{-3\} \cup (5,19)$\\
H.$x \in \{3\} \cup [5,19]$
\testStop
\kluczStart
A
\kluczStop



\zadStart{Zadanie z Wikieł Z 1.62 c) moja wersja nr 145}

Rozwiązać nierówności $(5-x)(x+3)^{2}(20-x)^{3}\le0$.
\zadStop
\rozwStart{Patryk Wirkus}{}
Miejsca zerowe naszego wielomianu to: $5, -3, 20$.\\
Wielomian jest stopnia parzystego, ponadto znak współczynnika przy\linebreak najwyższej potędze x jest ujemny.\\ W związku z tym wykres wielomianu zaczyna się od lewej strony powyżej osi OX.\\
Ponadto w punkcie $-3$ wykres odbija się od osi poziomej.\\
A więc $$x \in \{-3\} \cup [5,20].$$
\rozwStop
\odpStart
$x \in \{-3\} \cup [5,20]$
\odpStop
\testStart
A.$x \in \{-3\} \cup [5,20]$\\
B.$x \in \{3\} \cup (5,20)$\\
C.$x \in \{-3\} \cup (5,20]$\\
D.$x \in \{3\} \cup (5,20]$\\
E.$x \in \{-3\} \cup [5,20)$\\
F.$x \in \{3\} \cup [5,20)$\\
G.$x \in \{-3\} \cup (5,20)$\\
H.$x \in \{3\} \cup [5,20]$
\testStop
\kluczStart
A
\kluczStop



\zadStart{Zadanie z Wikieł Z 1.62 c) moja wersja nr 146}

Rozwiązać nierówności $(5-x)(x+4)^{2}(6-x)^{3}\le0$.
\zadStop
\rozwStart{Patryk Wirkus}{}
Miejsca zerowe naszego wielomianu to: $5, -4, 6$.\\
Wielomian jest stopnia parzystego, ponadto znak współczynnika przy\linebreak najwyższej potędze x jest ujemny.\\ W związku z tym wykres wielomianu zaczyna się od lewej strony powyżej osi OX.\\
Ponadto w punkcie $-4$ wykres odbija się od osi poziomej.\\
A więc $$x \in \{-4\} \cup [5,6].$$
\rozwStop
\odpStart
$x \in \{-4\} \cup [5,6]$
\odpStop
\testStart
A.$x \in \{-4\} \cup [5,6]$\\
B.$x \in \{4\} \cup (5,6)$\\
C.$x \in \{-4\} \cup (5,6]$\\
D.$x \in \{4\} \cup (5,6]$\\
E.$x \in \{-4\} \cup [5,6)$\\
F.$x \in \{4\} \cup [5,6)$\\
G.$x \in \{-4\} \cup (5,6)$\\
H.$x \in \{4\} \cup [5,6]$
\testStop
\kluczStart
A
\kluczStop



\zadStart{Zadanie z Wikieł Z 1.62 c) moja wersja nr 147}

Rozwiązać nierówności $(5-x)(x+4)^{2}(7-x)^{3}\le0$.
\zadStop
\rozwStart{Patryk Wirkus}{}
Miejsca zerowe naszego wielomianu to: $5, -4, 7$.\\
Wielomian jest stopnia parzystego, ponadto znak współczynnika przy\linebreak najwyższej potędze x jest ujemny.\\ W związku z tym wykres wielomianu zaczyna się od lewej strony powyżej osi OX.\\
Ponadto w punkcie $-4$ wykres odbija się od osi poziomej.\\
A więc $$x \in \{-4\} \cup [5,7].$$
\rozwStop
\odpStart
$x \in \{-4\} \cup [5,7]$
\odpStop
\testStart
A.$x \in \{-4\} \cup [5,7]$\\
B.$x \in \{4\} \cup (5,7)$\\
C.$x \in \{-4\} \cup (5,7]$\\
D.$x \in \{4\} \cup (5,7]$\\
E.$x \in \{-4\} \cup [5,7)$\\
F.$x \in \{4\} \cup [5,7)$\\
G.$x \in \{-4\} \cup (5,7)$\\
H.$x \in \{4\} \cup [5,7]$
\testStop
\kluczStart
A
\kluczStop



\zadStart{Zadanie z Wikieł Z 1.62 c) moja wersja nr 148}

Rozwiązać nierówności $(5-x)(x+4)^{2}(8-x)^{3}\le0$.
\zadStop
\rozwStart{Patryk Wirkus}{}
Miejsca zerowe naszego wielomianu to: $5, -4, 8$.\\
Wielomian jest stopnia parzystego, ponadto znak współczynnika przy\linebreak najwyższej potędze x jest ujemny.\\ W związku z tym wykres wielomianu zaczyna się od lewej strony powyżej osi OX.\\
Ponadto w punkcie $-4$ wykres odbija się od osi poziomej.\\
A więc $$x \in \{-4\} \cup [5,8].$$
\rozwStop
\odpStart
$x \in \{-4\} \cup [5,8]$
\odpStop
\testStart
A.$x \in \{-4\} \cup [5,8]$\\
B.$x \in \{4\} \cup (5,8)$\\
C.$x \in \{-4\} \cup (5,8]$\\
D.$x \in \{4\} \cup (5,8]$\\
E.$x \in \{-4\} \cup [5,8)$\\
F.$x \in \{4\} \cup [5,8)$\\
G.$x \in \{-4\} \cup (5,8)$\\
H.$x \in \{4\} \cup [5,8]$
\testStop
\kluczStart
A
\kluczStop



\zadStart{Zadanie z Wikieł Z 1.62 c) moja wersja nr 149}

Rozwiązać nierówności $(5-x)(x+4)^{2}(9-x)^{3}\le0$.
\zadStop
\rozwStart{Patryk Wirkus}{}
Miejsca zerowe naszego wielomianu to: $5, -4, 9$.\\
Wielomian jest stopnia parzystego, ponadto znak współczynnika przy\linebreak najwyższej potędze x jest ujemny.\\ W związku z tym wykres wielomianu zaczyna się od lewej strony powyżej osi OX.\\
Ponadto w punkcie $-4$ wykres odbija się od osi poziomej.\\
A więc $$x \in \{-4\} \cup [5,9].$$
\rozwStop
\odpStart
$x \in \{-4\} \cup [5,9]$
\odpStop
\testStart
A.$x \in \{-4\} \cup [5,9]$\\
B.$x \in \{4\} \cup (5,9)$\\
C.$x \in \{-4\} \cup (5,9]$\\
D.$x \in \{4\} \cup (5,9]$\\
E.$x \in \{-4\} \cup [5,9)$\\
F.$x \in \{4\} \cup [5,9)$\\
G.$x \in \{-4\} \cup (5,9)$\\
H.$x \in \{4\} \cup [5,9]$
\testStop
\kluczStart
A
\kluczStop



\zadStart{Zadanie z Wikieł Z 1.62 c) moja wersja nr 150}

Rozwiązać nierówności $(5-x)(x+4)^{2}(10-x)^{3}\le0$.
\zadStop
\rozwStart{Patryk Wirkus}{}
Miejsca zerowe naszego wielomianu to: $5, -4, 10$.\\
Wielomian jest stopnia parzystego, ponadto znak współczynnika przy\linebreak najwyższej potędze x jest ujemny.\\ W związku z tym wykres wielomianu zaczyna się od lewej strony powyżej osi OX.\\
Ponadto w punkcie $-4$ wykres odbija się od osi poziomej.\\
A więc $$x \in \{-4\} \cup [5,10].$$
\rozwStop
\odpStart
$x \in \{-4\} \cup [5,10]$
\odpStop
\testStart
A.$x \in \{-4\} \cup [5,10]$\\
B.$x \in \{4\} \cup (5,10)$\\
C.$x \in \{-4\} \cup (5,10]$\\
D.$x \in \{4\} \cup (5,10]$\\
E.$x \in \{-4\} \cup [5,10)$\\
F.$x \in \{4\} \cup [5,10)$\\
G.$x \in \{-4\} \cup (5,10)$\\
H.$x \in \{4\} \cup [5,10]$
\testStop
\kluczStart
A
\kluczStop



\zadStart{Zadanie z Wikieł Z 1.62 c) moja wersja nr 151}

Rozwiązać nierówności $(5-x)(x+4)^{2}(11-x)^{3}\le0$.
\zadStop
\rozwStart{Patryk Wirkus}{}
Miejsca zerowe naszego wielomianu to: $5, -4, 11$.\\
Wielomian jest stopnia parzystego, ponadto znak współczynnika przy\linebreak najwyższej potędze x jest ujemny.\\ W związku z tym wykres wielomianu zaczyna się od lewej strony powyżej osi OX.\\
Ponadto w punkcie $-4$ wykres odbija się od osi poziomej.\\
A więc $$x \in \{-4\} \cup [5,11].$$
\rozwStop
\odpStart
$x \in \{-4\} \cup [5,11]$
\odpStop
\testStart
A.$x \in \{-4\} \cup [5,11]$\\
B.$x \in \{4\} \cup (5,11)$\\
C.$x \in \{-4\} \cup (5,11]$\\
D.$x \in \{4\} \cup (5,11]$\\
E.$x \in \{-4\} \cup [5,11)$\\
F.$x \in \{4\} \cup [5,11)$\\
G.$x \in \{-4\} \cup (5,11)$\\
H.$x \in \{4\} \cup [5,11]$
\testStop
\kluczStart
A
\kluczStop



\zadStart{Zadanie z Wikieł Z 1.62 c) moja wersja nr 152}

Rozwiązać nierówności $(5-x)(x+4)^{2}(12-x)^{3}\le0$.
\zadStop
\rozwStart{Patryk Wirkus}{}
Miejsca zerowe naszego wielomianu to: $5, -4, 12$.\\
Wielomian jest stopnia parzystego, ponadto znak współczynnika przy\linebreak najwyższej potędze x jest ujemny.\\ W związku z tym wykres wielomianu zaczyna się od lewej strony powyżej osi OX.\\
Ponadto w punkcie $-4$ wykres odbija się od osi poziomej.\\
A więc $$x \in \{-4\} \cup [5,12].$$
\rozwStop
\odpStart
$x \in \{-4\} \cup [5,12]$
\odpStop
\testStart
A.$x \in \{-4\} \cup [5,12]$\\
B.$x \in \{4\} \cup (5,12)$\\
C.$x \in \{-4\} \cup (5,12]$\\
D.$x \in \{4\} \cup (5,12]$\\
E.$x \in \{-4\} \cup [5,12)$\\
F.$x \in \{4\} \cup [5,12)$\\
G.$x \in \{-4\} \cup (5,12)$\\
H.$x \in \{4\} \cup [5,12]$
\testStop
\kluczStart
A
\kluczStop



\zadStart{Zadanie z Wikieł Z 1.62 c) moja wersja nr 153}

Rozwiązać nierówności $(5-x)(x+4)^{2}(13-x)^{3}\le0$.
\zadStop
\rozwStart{Patryk Wirkus}{}
Miejsca zerowe naszego wielomianu to: $5, -4, 13$.\\
Wielomian jest stopnia parzystego, ponadto znak współczynnika przy\linebreak najwyższej potędze x jest ujemny.\\ W związku z tym wykres wielomianu zaczyna się od lewej strony powyżej osi OX.\\
Ponadto w punkcie $-4$ wykres odbija się od osi poziomej.\\
A więc $$x \in \{-4\} \cup [5,13].$$
\rozwStop
\odpStart
$x \in \{-4\} \cup [5,13]$
\odpStop
\testStart
A.$x \in \{-4\} \cup [5,13]$\\
B.$x \in \{4\} \cup (5,13)$\\
C.$x \in \{-4\} \cup (5,13]$\\
D.$x \in \{4\} \cup (5,13]$\\
E.$x \in \{-4\} \cup [5,13)$\\
F.$x \in \{4\} \cup [5,13)$\\
G.$x \in \{-4\} \cup (5,13)$\\
H.$x \in \{4\} \cup [5,13]$
\testStop
\kluczStart
A
\kluczStop



\zadStart{Zadanie z Wikieł Z 1.62 c) moja wersja nr 154}

Rozwiązać nierówności $(5-x)(x+4)^{2}(14-x)^{3}\le0$.
\zadStop
\rozwStart{Patryk Wirkus}{}
Miejsca zerowe naszego wielomianu to: $5, -4, 14$.\\
Wielomian jest stopnia parzystego, ponadto znak współczynnika przy\linebreak najwyższej potędze x jest ujemny.\\ W związku z tym wykres wielomianu zaczyna się od lewej strony powyżej osi OX.\\
Ponadto w punkcie $-4$ wykres odbija się od osi poziomej.\\
A więc $$x \in \{-4\} \cup [5,14].$$
\rozwStop
\odpStart
$x \in \{-4\} \cup [5,14]$
\odpStop
\testStart
A.$x \in \{-4\} \cup [5,14]$\\
B.$x \in \{4\} \cup (5,14)$\\
C.$x \in \{-4\} \cup (5,14]$\\
D.$x \in \{4\} \cup (5,14]$\\
E.$x \in \{-4\} \cup [5,14)$\\
F.$x \in \{4\} \cup [5,14)$\\
G.$x \in \{-4\} \cup (5,14)$\\
H.$x \in \{4\} \cup [5,14]$
\testStop
\kluczStart
A
\kluczStop



\zadStart{Zadanie z Wikieł Z 1.62 c) moja wersja nr 155}

Rozwiązać nierówności $(5-x)(x+4)^{2}(15-x)^{3}\le0$.
\zadStop
\rozwStart{Patryk Wirkus}{}
Miejsca zerowe naszego wielomianu to: $5, -4, 15$.\\
Wielomian jest stopnia parzystego, ponadto znak współczynnika przy\linebreak najwyższej potędze x jest ujemny.\\ W związku z tym wykres wielomianu zaczyna się od lewej strony powyżej osi OX.\\
Ponadto w punkcie $-4$ wykres odbija się od osi poziomej.\\
A więc $$x \in \{-4\} \cup [5,15].$$
\rozwStop
\odpStart
$x \in \{-4\} \cup [5,15]$
\odpStop
\testStart
A.$x \in \{-4\} \cup [5,15]$\\
B.$x \in \{4\} \cup (5,15)$\\
C.$x \in \{-4\} \cup (5,15]$\\
D.$x \in \{4\} \cup (5,15]$\\
E.$x \in \{-4\} \cup [5,15)$\\
F.$x \in \{4\} \cup [5,15)$\\
G.$x \in \{-4\} \cup (5,15)$\\
H.$x \in \{4\} \cup [5,15]$
\testStop
\kluczStart
A
\kluczStop



\zadStart{Zadanie z Wikieł Z 1.62 c) moja wersja nr 156}

Rozwiązać nierówności $(5-x)(x+4)^{2}(16-x)^{3}\le0$.
\zadStop
\rozwStart{Patryk Wirkus}{}
Miejsca zerowe naszego wielomianu to: $5, -4, 16$.\\
Wielomian jest stopnia parzystego, ponadto znak współczynnika przy\linebreak najwyższej potędze x jest ujemny.\\ W związku z tym wykres wielomianu zaczyna się od lewej strony powyżej osi OX.\\
Ponadto w punkcie $-4$ wykres odbija się od osi poziomej.\\
A więc $$x \in \{-4\} \cup [5,16].$$
\rozwStop
\odpStart
$x \in \{-4\} \cup [5,16]$
\odpStop
\testStart
A.$x \in \{-4\} \cup [5,16]$\\
B.$x \in \{4\} \cup (5,16)$\\
C.$x \in \{-4\} \cup (5,16]$\\
D.$x \in \{4\} \cup (5,16]$\\
E.$x \in \{-4\} \cup [5,16)$\\
F.$x \in \{4\} \cup [5,16)$\\
G.$x \in \{-4\} \cup (5,16)$\\
H.$x \in \{4\} \cup [5,16]$
\testStop
\kluczStart
A
\kluczStop



\zadStart{Zadanie z Wikieł Z 1.62 c) moja wersja nr 157}

Rozwiązać nierówności $(5-x)(x+4)^{2}(17-x)^{3}\le0$.
\zadStop
\rozwStart{Patryk Wirkus}{}
Miejsca zerowe naszego wielomianu to: $5, -4, 17$.\\
Wielomian jest stopnia parzystego, ponadto znak współczynnika przy\linebreak najwyższej potędze x jest ujemny.\\ W związku z tym wykres wielomianu zaczyna się od lewej strony powyżej osi OX.\\
Ponadto w punkcie $-4$ wykres odbija się od osi poziomej.\\
A więc $$x \in \{-4\} \cup [5,17].$$
\rozwStop
\odpStart
$x \in \{-4\} \cup [5,17]$
\odpStop
\testStart
A.$x \in \{-4\} \cup [5,17]$\\
B.$x \in \{4\} \cup (5,17)$\\
C.$x \in \{-4\} \cup (5,17]$\\
D.$x \in \{4\} \cup (5,17]$\\
E.$x \in \{-4\} \cup [5,17)$\\
F.$x \in \{4\} \cup [5,17)$\\
G.$x \in \{-4\} \cup (5,17)$\\
H.$x \in \{4\} \cup [5,17]$
\testStop
\kluczStart
A
\kluczStop



\zadStart{Zadanie z Wikieł Z 1.62 c) moja wersja nr 158}

Rozwiązać nierówności $(5-x)(x+4)^{2}(18-x)^{3}\le0$.
\zadStop
\rozwStart{Patryk Wirkus}{}
Miejsca zerowe naszego wielomianu to: $5, -4, 18$.\\
Wielomian jest stopnia parzystego, ponadto znak współczynnika przy\linebreak najwyższej potędze x jest ujemny.\\ W związku z tym wykres wielomianu zaczyna się od lewej strony powyżej osi OX.\\
Ponadto w punkcie $-4$ wykres odbija się od osi poziomej.\\
A więc $$x \in \{-4\} \cup [5,18].$$
\rozwStop
\odpStart
$x \in \{-4\} \cup [5,18]$
\odpStop
\testStart
A.$x \in \{-4\} \cup [5,18]$\\
B.$x \in \{4\} \cup (5,18)$\\
C.$x \in \{-4\} \cup (5,18]$\\
D.$x \in \{4\} \cup (5,18]$\\
E.$x \in \{-4\} \cup [5,18)$\\
F.$x \in \{4\} \cup [5,18)$\\
G.$x \in \{-4\} \cup (5,18)$\\
H.$x \in \{4\} \cup [5,18]$
\testStop
\kluczStart
A
\kluczStop



\zadStart{Zadanie z Wikieł Z 1.62 c) moja wersja nr 159}

Rozwiązać nierówności $(5-x)(x+4)^{2}(19-x)^{3}\le0$.
\zadStop
\rozwStart{Patryk Wirkus}{}
Miejsca zerowe naszego wielomianu to: $5, -4, 19$.\\
Wielomian jest stopnia parzystego, ponadto znak współczynnika przy\linebreak najwyższej potędze x jest ujemny.\\ W związku z tym wykres wielomianu zaczyna się od lewej strony powyżej osi OX.\\
Ponadto w punkcie $-4$ wykres odbija się od osi poziomej.\\
A więc $$x \in \{-4\} \cup [5,19].$$
\rozwStop
\odpStart
$x \in \{-4\} \cup [5,19]$
\odpStop
\testStart
A.$x \in \{-4\} \cup [5,19]$\\
B.$x \in \{4\} \cup (5,19)$\\
C.$x \in \{-4\} \cup (5,19]$\\
D.$x \in \{4\} \cup (5,19]$\\
E.$x \in \{-4\} \cup [5,19)$\\
F.$x \in \{4\} \cup [5,19)$\\
G.$x \in \{-4\} \cup (5,19)$\\
H.$x \in \{4\} \cup [5,19]$
\testStop
\kluczStart
A
\kluczStop



\zadStart{Zadanie z Wikieł Z 1.62 c) moja wersja nr 160}

Rozwiązać nierówności $(5-x)(x+4)^{2}(20-x)^{3}\le0$.
\zadStop
\rozwStart{Patryk Wirkus}{}
Miejsca zerowe naszego wielomianu to: $5, -4, 20$.\\
Wielomian jest stopnia parzystego, ponadto znak współczynnika przy\linebreak najwyższej potędze x jest ujemny.\\ W związku z tym wykres wielomianu zaczyna się od lewej strony powyżej osi OX.\\
Ponadto w punkcie $-4$ wykres odbija się od osi poziomej.\\
A więc $$x \in \{-4\} \cup [5,20].$$
\rozwStop
\odpStart
$x \in \{-4\} \cup [5,20]$
\odpStop
\testStart
A.$x \in \{-4\} \cup [5,20]$\\
B.$x \in \{4\} \cup (5,20)$\\
C.$x \in \{-4\} \cup (5,20]$\\
D.$x \in \{4\} \cup (5,20]$\\
E.$x \in \{-4\} \cup [5,20)$\\
F.$x \in \{4\} \cup [5,20)$\\
G.$x \in \{-4\} \cup (5,20)$\\
H.$x \in \{4\} \cup [5,20]$
\testStop
\kluczStart
A
\kluczStop



\zadStart{Zadanie z Wikieł Z 1.62 c) moja wersja nr 161}

Rozwiązać nierówności $(6-x)(x+1)^{2}(7-x)^{3}\le0$.
\zadStop
\rozwStart{Patryk Wirkus}{}
Miejsca zerowe naszego wielomianu to: $6, -1, 7$.\\
Wielomian jest stopnia parzystego, ponadto znak współczynnika przy\linebreak najwyższej potędze x jest ujemny.\\ W związku z tym wykres wielomianu zaczyna się od lewej strony powyżej osi OX.\\
Ponadto w punkcie $-1$ wykres odbija się od osi poziomej.\\
A więc $$x \in \{-1\} \cup [6,7].$$
\rozwStop
\odpStart
$x \in \{-1\} \cup [6,7]$
\odpStop
\testStart
A.$x \in \{-1\} \cup [6,7]$\\
B.$x \in \{1\} \cup (6,7)$\\
C.$x \in \{-1\} \cup (6,7]$\\
D.$x \in \{1\} \cup (6,7]$\\
E.$x \in \{-1\} \cup [6,7)$\\
F.$x \in \{1\} \cup [6,7)$\\
G.$x \in \{-1\} \cup (6,7)$\\
H.$x \in \{1\} \cup [6,7]$
\testStop
\kluczStart
A
\kluczStop



\zadStart{Zadanie z Wikieł Z 1.62 c) moja wersja nr 162}

Rozwiązać nierówności $(6-x)(x+1)^{2}(8-x)^{3}\le0$.
\zadStop
\rozwStart{Patryk Wirkus}{}
Miejsca zerowe naszego wielomianu to: $6, -1, 8$.\\
Wielomian jest stopnia parzystego, ponadto znak współczynnika przy\linebreak najwyższej potędze x jest ujemny.\\ W związku z tym wykres wielomianu zaczyna się od lewej strony powyżej osi OX.\\
Ponadto w punkcie $-1$ wykres odbija się od osi poziomej.\\
A więc $$x \in \{-1\} \cup [6,8].$$
\rozwStop
\odpStart
$x \in \{-1\} \cup [6,8]$
\odpStop
\testStart
A.$x \in \{-1\} \cup [6,8]$\\
B.$x \in \{1\} \cup (6,8)$\\
C.$x \in \{-1\} \cup (6,8]$\\
D.$x \in \{1\} \cup (6,8]$\\
E.$x \in \{-1\} \cup [6,8)$\\
F.$x \in \{1\} \cup [6,8)$\\
G.$x \in \{-1\} \cup (6,8)$\\
H.$x \in \{1\} \cup [6,8]$
\testStop
\kluczStart
A
\kluczStop



\zadStart{Zadanie z Wikieł Z 1.62 c) moja wersja nr 163}

Rozwiązać nierówności $(6-x)(x+1)^{2}(9-x)^{3}\le0$.
\zadStop
\rozwStart{Patryk Wirkus}{}
Miejsca zerowe naszego wielomianu to: $6, -1, 9$.\\
Wielomian jest stopnia parzystego, ponadto znak współczynnika przy\linebreak najwyższej potędze x jest ujemny.\\ W związku z tym wykres wielomianu zaczyna się od lewej strony powyżej osi OX.\\
Ponadto w punkcie $-1$ wykres odbija się od osi poziomej.\\
A więc $$x \in \{-1\} \cup [6,9].$$
\rozwStop
\odpStart
$x \in \{-1\} \cup [6,9]$
\odpStop
\testStart
A.$x \in \{-1\} \cup [6,9]$\\
B.$x \in \{1\} \cup (6,9)$\\
C.$x \in \{-1\} \cup (6,9]$\\
D.$x \in \{1\} \cup (6,9]$\\
E.$x \in \{-1\} \cup [6,9)$\\
F.$x \in \{1\} \cup [6,9)$\\
G.$x \in \{-1\} \cup (6,9)$\\
H.$x \in \{1\} \cup [6,9]$
\testStop
\kluczStart
A
\kluczStop



\zadStart{Zadanie z Wikieł Z 1.62 c) moja wersja nr 164}

Rozwiązać nierówności $(6-x)(x+1)^{2}(10-x)^{3}\le0$.
\zadStop
\rozwStart{Patryk Wirkus}{}
Miejsca zerowe naszego wielomianu to: $6, -1, 10$.\\
Wielomian jest stopnia parzystego, ponadto znak współczynnika przy\linebreak najwyższej potędze x jest ujemny.\\ W związku z tym wykres wielomianu zaczyna się od lewej strony powyżej osi OX.\\
Ponadto w punkcie $-1$ wykres odbija się od osi poziomej.\\
A więc $$x \in \{-1\} \cup [6,10].$$
\rozwStop
\odpStart
$x \in \{-1\} \cup [6,10]$
\odpStop
\testStart
A.$x \in \{-1\} \cup [6,10]$\\
B.$x \in \{1\} \cup (6,10)$\\
C.$x \in \{-1\} \cup (6,10]$\\
D.$x \in \{1\} \cup (6,10]$\\
E.$x \in \{-1\} \cup [6,10)$\\
F.$x \in \{1\} \cup [6,10)$\\
G.$x \in \{-1\} \cup (6,10)$\\
H.$x \in \{1\} \cup [6,10]$
\testStop
\kluczStart
A
\kluczStop



\zadStart{Zadanie z Wikieł Z 1.62 c) moja wersja nr 165}

Rozwiązać nierówności $(6-x)(x+1)^{2}(11-x)^{3}\le0$.
\zadStop
\rozwStart{Patryk Wirkus}{}
Miejsca zerowe naszego wielomianu to: $6, -1, 11$.\\
Wielomian jest stopnia parzystego, ponadto znak współczynnika przy\linebreak najwyższej potędze x jest ujemny.\\ W związku z tym wykres wielomianu zaczyna się od lewej strony powyżej osi OX.\\
Ponadto w punkcie $-1$ wykres odbija się od osi poziomej.\\
A więc $$x \in \{-1\} \cup [6,11].$$
\rozwStop
\odpStart
$x \in \{-1\} \cup [6,11]$
\odpStop
\testStart
A.$x \in \{-1\} \cup [6,11]$\\
B.$x \in \{1\} \cup (6,11)$\\
C.$x \in \{-1\} \cup (6,11]$\\
D.$x \in \{1\} \cup (6,11]$\\
E.$x \in \{-1\} \cup [6,11)$\\
F.$x \in \{1\} \cup [6,11)$\\
G.$x \in \{-1\} \cup (6,11)$\\
H.$x \in \{1\} \cup [6,11]$
\testStop
\kluczStart
A
\kluczStop



\zadStart{Zadanie z Wikieł Z 1.62 c) moja wersja nr 166}

Rozwiązać nierówności $(6-x)(x+1)^{2}(12-x)^{3}\le0$.
\zadStop
\rozwStart{Patryk Wirkus}{}
Miejsca zerowe naszego wielomianu to: $6, -1, 12$.\\
Wielomian jest stopnia parzystego, ponadto znak współczynnika przy\linebreak najwyższej potędze x jest ujemny.\\ W związku z tym wykres wielomianu zaczyna się od lewej strony powyżej osi OX.\\
Ponadto w punkcie $-1$ wykres odbija się od osi poziomej.\\
A więc $$x \in \{-1\} \cup [6,12].$$
\rozwStop
\odpStart
$x \in \{-1\} \cup [6,12]$
\odpStop
\testStart
A.$x \in \{-1\} \cup [6,12]$\\
B.$x \in \{1\} \cup (6,12)$\\
C.$x \in \{-1\} \cup (6,12]$\\
D.$x \in \{1\} \cup (6,12]$\\
E.$x \in \{-1\} \cup [6,12)$\\
F.$x \in \{1\} \cup [6,12)$\\
G.$x \in \{-1\} \cup (6,12)$\\
H.$x \in \{1\} \cup [6,12]$
\testStop
\kluczStart
A
\kluczStop



\zadStart{Zadanie z Wikieł Z 1.62 c) moja wersja nr 167}

Rozwiązać nierówności $(6-x)(x+1)^{2}(13-x)^{3}\le0$.
\zadStop
\rozwStart{Patryk Wirkus}{}
Miejsca zerowe naszego wielomianu to: $6, -1, 13$.\\
Wielomian jest stopnia parzystego, ponadto znak współczynnika przy\linebreak najwyższej potędze x jest ujemny.\\ W związku z tym wykres wielomianu zaczyna się od lewej strony powyżej osi OX.\\
Ponadto w punkcie $-1$ wykres odbija się od osi poziomej.\\
A więc $$x \in \{-1\} \cup [6,13].$$
\rozwStop
\odpStart
$x \in \{-1\} \cup [6,13]$
\odpStop
\testStart
A.$x \in \{-1\} \cup [6,13]$\\
B.$x \in \{1\} \cup (6,13)$\\
C.$x \in \{-1\} \cup (6,13]$\\
D.$x \in \{1\} \cup (6,13]$\\
E.$x \in \{-1\} \cup [6,13)$\\
F.$x \in \{1\} \cup [6,13)$\\
G.$x \in \{-1\} \cup (6,13)$\\
H.$x \in \{1\} \cup [6,13]$
\testStop
\kluczStart
A
\kluczStop



\zadStart{Zadanie z Wikieł Z 1.62 c) moja wersja nr 168}

Rozwiązać nierówności $(6-x)(x+1)^{2}(14-x)^{3}\le0$.
\zadStop
\rozwStart{Patryk Wirkus}{}
Miejsca zerowe naszego wielomianu to: $6, -1, 14$.\\
Wielomian jest stopnia parzystego, ponadto znak współczynnika przy\linebreak najwyższej potędze x jest ujemny.\\ W związku z tym wykres wielomianu zaczyna się od lewej strony powyżej osi OX.\\
Ponadto w punkcie $-1$ wykres odbija się od osi poziomej.\\
A więc $$x \in \{-1\} \cup [6,14].$$
\rozwStop
\odpStart
$x \in \{-1\} \cup [6,14]$
\odpStop
\testStart
A.$x \in \{-1\} \cup [6,14]$\\
B.$x \in \{1\} \cup (6,14)$\\
C.$x \in \{-1\} \cup (6,14]$\\
D.$x \in \{1\} \cup (6,14]$\\
E.$x \in \{-1\} \cup [6,14)$\\
F.$x \in \{1\} \cup [6,14)$\\
G.$x \in \{-1\} \cup (6,14)$\\
H.$x \in \{1\} \cup [6,14]$
\testStop
\kluczStart
A
\kluczStop



\zadStart{Zadanie z Wikieł Z 1.62 c) moja wersja nr 169}

Rozwiązać nierówności $(6-x)(x+1)^{2}(15-x)^{3}\le0$.
\zadStop
\rozwStart{Patryk Wirkus}{}
Miejsca zerowe naszego wielomianu to: $6, -1, 15$.\\
Wielomian jest stopnia parzystego, ponadto znak współczynnika przy\linebreak najwyższej potędze x jest ujemny.\\ W związku z tym wykres wielomianu zaczyna się od lewej strony powyżej osi OX.\\
Ponadto w punkcie $-1$ wykres odbija się od osi poziomej.\\
A więc $$x \in \{-1\} \cup [6,15].$$
\rozwStop
\odpStart
$x \in \{-1\} \cup [6,15]$
\odpStop
\testStart
A.$x \in \{-1\} \cup [6,15]$\\
B.$x \in \{1\} \cup (6,15)$\\
C.$x \in \{-1\} \cup (6,15]$\\
D.$x \in \{1\} \cup (6,15]$\\
E.$x \in \{-1\} \cup [6,15)$\\
F.$x \in \{1\} \cup [6,15)$\\
G.$x \in \{-1\} \cup (6,15)$\\
H.$x \in \{1\} \cup [6,15]$
\testStop
\kluczStart
A
\kluczStop



\zadStart{Zadanie z Wikieł Z 1.62 c) moja wersja nr 170}

Rozwiązać nierówności $(6-x)(x+1)^{2}(16-x)^{3}\le0$.
\zadStop
\rozwStart{Patryk Wirkus}{}
Miejsca zerowe naszego wielomianu to: $6, -1, 16$.\\
Wielomian jest stopnia parzystego, ponadto znak współczynnika przy\linebreak najwyższej potędze x jest ujemny.\\ W związku z tym wykres wielomianu zaczyna się od lewej strony powyżej osi OX.\\
Ponadto w punkcie $-1$ wykres odbija się od osi poziomej.\\
A więc $$x \in \{-1\} \cup [6,16].$$
\rozwStop
\odpStart
$x \in \{-1\} \cup [6,16]$
\odpStop
\testStart
A.$x \in \{-1\} \cup [6,16]$\\
B.$x \in \{1\} \cup (6,16)$\\
C.$x \in \{-1\} \cup (6,16]$\\
D.$x \in \{1\} \cup (6,16]$\\
E.$x \in \{-1\} \cup [6,16)$\\
F.$x \in \{1\} \cup [6,16)$\\
G.$x \in \{-1\} \cup (6,16)$\\
H.$x \in \{1\} \cup [6,16]$
\testStop
\kluczStart
A
\kluczStop



\zadStart{Zadanie z Wikieł Z 1.62 c) moja wersja nr 171}

Rozwiązać nierówności $(6-x)(x+1)^{2}(17-x)^{3}\le0$.
\zadStop
\rozwStart{Patryk Wirkus}{}
Miejsca zerowe naszego wielomianu to: $6, -1, 17$.\\
Wielomian jest stopnia parzystego, ponadto znak współczynnika przy\linebreak najwyższej potędze x jest ujemny.\\ W związku z tym wykres wielomianu zaczyna się od lewej strony powyżej osi OX.\\
Ponadto w punkcie $-1$ wykres odbija się od osi poziomej.\\
A więc $$x \in \{-1\} \cup [6,17].$$
\rozwStop
\odpStart
$x \in \{-1\} \cup [6,17]$
\odpStop
\testStart
A.$x \in \{-1\} \cup [6,17]$\\
B.$x \in \{1\} \cup (6,17)$\\
C.$x \in \{-1\} \cup (6,17]$\\
D.$x \in \{1\} \cup (6,17]$\\
E.$x \in \{-1\} \cup [6,17)$\\
F.$x \in \{1\} \cup [6,17)$\\
G.$x \in \{-1\} \cup (6,17)$\\
H.$x \in \{1\} \cup [6,17]$
\testStop
\kluczStart
A
\kluczStop



\zadStart{Zadanie z Wikieł Z 1.62 c) moja wersja nr 172}

Rozwiązać nierówności $(6-x)(x+1)^{2}(18-x)^{3}\le0$.
\zadStop
\rozwStart{Patryk Wirkus}{}
Miejsca zerowe naszego wielomianu to: $6, -1, 18$.\\
Wielomian jest stopnia parzystego, ponadto znak współczynnika przy\linebreak najwyższej potędze x jest ujemny.\\ W związku z tym wykres wielomianu zaczyna się od lewej strony powyżej osi OX.\\
Ponadto w punkcie $-1$ wykres odbija się od osi poziomej.\\
A więc $$x \in \{-1\} \cup [6,18].$$
\rozwStop
\odpStart
$x \in \{-1\} \cup [6,18]$
\odpStop
\testStart
A.$x \in \{-1\} \cup [6,18]$\\
B.$x \in \{1\} \cup (6,18)$\\
C.$x \in \{-1\} \cup (6,18]$\\
D.$x \in \{1\} \cup (6,18]$\\
E.$x \in \{-1\} \cup [6,18)$\\
F.$x \in \{1\} \cup [6,18)$\\
G.$x \in \{-1\} \cup (6,18)$\\
H.$x \in \{1\} \cup [6,18]$
\testStop
\kluczStart
A
\kluczStop



\zadStart{Zadanie z Wikieł Z 1.62 c) moja wersja nr 173}

Rozwiązać nierówności $(6-x)(x+1)^{2}(19-x)^{3}\le0$.
\zadStop
\rozwStart{Patryk Wirkus}{}
Miejsca zerowe naszego wielomianu to: $6, -1, 19$.\\
Wielomian jest stopnia parzystego, ponadto znak współczynnika przy\linebreak najwyższej potędze x jest ujemny.\\ W związku z tym wykres wielomianu zaczyna się od lewej strony powyżej osi OX.\\
Ponadto w punkcie $-1$ wykres odbija się od osi poziomej.\\
A więc $$x \in \{-1\} \cup [6,19].$$
\rozwStop
\odpStart
$x \in \{-1\} \cup [6,19]$
\odpStop
\testStart
A.$x \in \{-1\} \cup [6,19]$\\
B.$x \in \{1\} \cup (6,19)$\\
C.$x \in \{-1\} \cup (6,19]$\\
D.$x \in \{1\} \cup (6,19]$\\
E.$x \in \{-1\} \cup [6,19)$\\
F.$x \in \{1\} \cup [6,19)$\\
G.$x \in \{-1\} \cup (6,19)$\\
H.$x \in \{1\} \cup [6,19]$
\testStop
\kluczStart
A
\kluczStop



\zadStart{Zadanie z Wikieł Z 1.62 c) moja wersja nr 174}

Rozwiązać nierówności $(6-x)(x+1)^{2}(20-x)^{3}\le0$.
\zadStop
\rozwStart{Patryk Wirkus}{}
Miejsca zerowe naszego wielomianu to: $6, -1, 20$.\\
Wielomian jest stopnia parzystego, ponadto znak współczynnika przy\linebreak najwyższej potędze x jest ujemny.\\ W związku z tym wykres wielomianu zaczyna się od lewej strony powyżej osi OX.\\
Ponadto w punkcie $-1$ wykres odbija się od osi poziomej.\\
A więc $$x \in \{-1\} \cup [6,20].$$
\rozwStop
\odpStart
$x \in \{-1\} \cup [6,20]$
\odpStop
\testStart
A.$x \in \{-1\} \cup [6,20]$\\
B.$x \in \{1\} \cup (6,20)$\\
C.$x \in \{-1\} \cup (6,20]$\\
D.$x \in \{1\} \cup (6,20]$\\
E.$x \in \{-1\} \cup [6,20)$\\
F.$x \in \{1\} \cup [6,20)$\\
G.$x \in \{-1\} \cup (6,20)$\\
H.$x \in \{1\} \cup [6,20]$
\testStop
\kluczStart
A
\kluczStop



\zadStart{Zadanie z Wikieł Z 1.62 c) moja wersja nr 175}

Rozwiązać nierówności $(6-x)(x+2)^{2}(7-x)^{3}\le0$.
\zadStop
\rozwStart{Patryk Wirkus}{}
Miejsca zerowe naszego wielomianu to: $6, -2, 7$.\\
Wielomian jest stopnia parzystego, ponadto znak współczynnika przy\linebreak najwyższej potędze x jest ujemny.\\ W związku z tym wykres wielomianu zaczyna się od lewej strony powyżej osi OX.\\
Ponadto w punkcie $-2$ wykres odbija się od osi poziomej.\\
A więc $$x \in \{-2\} \cup [6,7].$$
\rozwStop
\odpStart
$x \in \{-2\} \cup [6,7]$
\odpStop
\testStart
A.$x \in \{-2\} \cup [6,7]$\\
B.$x \in \{2\} \cup (6,7)$\\
C.$x \in \{-2\} \cup (6,7]$\\
D.$x \in \{2\} \cup (6,7]$\\
E.$x \in \{-2\} \cup [6,7)$\\
F.$x \in \{2\} \cup [6,7)$\\
G.$x \in \{-2\} \cup (6,7)$\\
H.$x \in \{2\} \cup [6,7]$
\testStop
\kluczStart
A
\kluczStop



\zadStart{Zadanie z Wikieł Z 1.62 c) moja wersja nr 176}

Rozwiązać nierówności $(6-x)(x+2)^{2}(8-x)^{3}\le0$.
\zadStop
\rozwStart{Patryk Wirkus}{}
Miejsca zerowe naszego wielomianu to: $6, -2, 8$.\\
Wielomian jest stopnia parzystego, ponadto znak współczynnika przy\linebreak najwyższej potędze x jest ujemny.\\ W związku z tym wykres wielomianu zaczyna się od lewej strony powyżej osi OX.\\
Ponadto w punkcie $-2$ wykres odbija się od osi poziomej.\\
A więc $$x \in \{-2\} \cup [6,8].$$
\rozwStop
\odpStart
$x \in \{-2\} \cup [6,8]$
\odpStop
\testStart
A.$x \in \{-2\} \cup [6,8]$\\
B.$x \in \{2\} \cup (6,8)$\\
C.$x \in \{-2\} \cup (6,8]$\\
D.$x \in \{2\} \cup (6,8]$\\
E.$x \in \{-2\} \cup [6,8)$\\
F.$x \in \{2\} \cup [6,8)$\\
G.$x \in \{-2\} \cup (6,8)$\\
H.$x \in \{2\} \cup [6,8]$
\testStop
\kluczStart
A
\kluczStop



\zadStart{Zadanie z Wikieł Z 1.62 c) moja wersja nr 177}

Rozwiązać nierówności $(6-x)(x+2)^{2}(9-x)^{3}\le0$.
\zadStop
\rozwStart{Patryk Wirkus}{}
Miejsca zerowe naszego wielomianu to: $6, -2, 9$.\\
Wielomian jest stopnia parzystego, ponadto znak współczynnika przy\linebreak najwyższej potędze x jest ujemny.\\ W związku z tym wykres wielomianu zaczyna się od lewej strony powyżej osi OX.\\
Ponadto w punkcie $-2$ wykres odbija się od osi poziomej.\\
A więc $$x \in \{-2\} \cup [6,9].$$
\rozwStop
\odpStart
$x \in \{-2\} \cup [6,9]$
\odpStop
\testStart
A.$x \in \{-2\} \cup [6,9]$\\
B.$x \in \{2\} \cup (6,9)$\\
C.$x \in \{-2\} \cup (6,9]$\\
D.$x \in \{2\} \cup (6,9]$\\
E.$x \in \{-2\} \cup [6,9)$\\
F.$x \in \{2\} \cup [6,9)$\\
G.$x \in \{-2\} \cup (6,9)$\\
H.$x \in \{2\} \cup [6,9]$
\testStop
\kluczStart
A
\kluczStop



\zadStart{Zadanie z Wikieł Z 1.62 c) moja wersja nr 178}

Rozwiązać nierówności $(6-x)(x+2)^{2}(10-x)^{3}\le0$.
\zadStop
\rozwStart{Patryk Wirkus}{}
Miejsca zerowe naszego wielomianu to: $6, -2, 10$.\\
Wielomian jest stopnia parzystego, ponadto znak współczynnika przy\linebreak najwyższej potędze x jest ujemny.\\ W związku z tym wykres wielomianu zaczyna się od lewej strony powyżej osi OX.\\
Ponadto w punkcie $-2$ wykres odbija się od osi poziomej.\\
A więc $$x \in \{-2\} \cup [6,10].$$
\rozwStop
\odpStart
$x \in \{-2\} \cup [6,10]$
\odpStop
\testStart
A.$x \in \{-2\} \cup [6,10]$\\
B.$x \in \{2\} \cup (6,10)$\\
C.$x \in \{-2\} \cup (6,10]$\\
D.$x \in \{2\} \cup (6,10]$\\
E.$x \in \{-2\} \cup [6,10)$\\
F.$x \in \{2\} \cup [6,10)$\\
G.$x \in \{-2\} \cup (6,10)$\\
H.$x \in \{2\} \cup [6,10]$
\testStop
\kluczStart
A
\kluczStop



\zadStart{Zadanie z Wikieł Z 1.62 c) moja wersja nr 179}

Rozwiązać nierówności $(6-x)(x+2)^{2}(11-x)^{3}\le0$.
\zadStop
\rozwStart{Patryk Wirkus}{}
Miejsca zerowe naszego wielomianu to: $6, -2, 11$.\\
Wielomian jest stopnia parzystego, ponadto znak współczynnika przy\linebreak najwyższej potędze x jest ujemny.\\ W związku z tym wykres wielomianu zaczyna się od lewej strony powyżej osi OX.\\
Ponadto w punkcie $-2$ wykres odbija się od osi poziomej.\\
A więc $$x \in \{-2\} \cup [6,11].$$
\rozwStop
\odpStart
$x \in \{-2\} \cup [6,11]$
\odpStop
\testStart
A.$x \in \{-2\} \cup [6,11]$\\
B.$x \in \{2\} \cup (6,11)$\\
C.$x \in \{-2\} \cup (6,11]$\\
D.$x \in \{2\} \cup (6,11]$\\
E.$x \in \{-2\} \cup [6,11)$\\
F.$x \in \{2\} \cup [6,11)$\\
G.$x \in \{-2\} \cup (6,11)$\\
H.$x \in \{2\} \cup [6,11]$
\testStop
\kluczStart
A
\kluczStop



\zadStart{Zadanie z Wikieł Z 1.62 c) moja wersja nr 180}

Rozwiązać nierówności $(6-x)(x+2)^{2}(12-x)^{3}\le0$.
\zadStop
\rozwStart{Patryk Wirkus}{}
Miejsca zerowe naszego wielomianu to: $6, -2, 12$.\\
Wielomian jest stopnia parzystego, ponadto znak współczynnika przy\linebreak najwyższej potędze x jest ujemny.\\ W związku z tym wykres wielomianu zaczyna się od lewej strony powyżej osi OX.\\
Ponadto w punkcie $-2$ wykres odbija się od osi poziomej.\\
A więc $$x \in \{-2\} \cup [6,12].$$
\rozwStop
\odpStart
$x \in \{-2\} \cup [6,12]$
\odpStop
\testStart
A.$x \in \{-2\} \cup [6,12]$\\
B.$x \in \{2\} \cup (6,12)$\\
C.$x \in \{-2\} \cup (6,12]$\\
D.$x \in \{2\} \cup (6,12]$\\
E.$x \in \{-2\} \cup [6,12)$\\
F.$x \in \{2\} \cup [6,12)$\\
G.$x \in \{-2\} \cup (6,12)$\\
H.$x \in \{2\} \cup [6,12]$
\testStop
\kluczStart
A
\kluczStop



\zadStart{Zadanie z Wikieł Z 1.62 c) moja wersja nr 181}

Rozwiązać nierówności $(6-x)(x+2)^{2}(13-x)^{3}\le0$.
\zadStop
\rozwStart{Patryk Wirkus}{}
Miejsca zerowe naszego wielomianu to: $6, -2, 13$.\\
Wielomian jest stopnia parzystego, ponadto znak współczynnika przy\linebreak najwyższej potędze x jest ujemny.\\ W związku z tym wykres wielomianu zaczyna się od lewej strony powyżej osi OX.\\
Ponadto w punkcie $-2$ wykres odbija się od osi poziomej.\\
A więc $$x \in \{-2\} \cup [6,13].$$
\rozwStop
\odpStart
$x \in \{-2\} \cup [6,13]$
\odpStop
\testStart
A.$x \in \{-2\} \cup [6,13]$\\
B.$x \in \{2\} \cup (6,13)$\\
C.$x \in \{-2\} \cup (6,13]$\\
D.$x \in \{2\} \cup (6,13]$\\
E.$x \in \{-2\} \cup [6,13)$\\
F.$x \in \{2\} \cup [6,13)$\\
G.$x \in \{-2\} \cup (6,13)$\\
H.$x \in \{2\} \cup [6,13]$
\testStop
\kluczStart
A
\kluczStop



\zadStart{Zadanie z Wikieł Z 1.62 c) moja wersja nr 182}

Rozwiązać nierówności $(6-x)(x+2)^{2}(14-x)^{3}\le0$.
\zadStop
\rozwStart{Patryk Wirkus}{}
Miejsca zerowe naszego wielomianu to: $6, -2, 14$.\\
Wielomian jest stopnia parzystego, ponadto znak współczynnika przy\linebreak najwyższej potędze x jest ujemny.\\ W związku z tym wykres wielomianu zaczyna się od lewej strony powyżej osi OX.\\
Ponadto w punkcie $-2$ wykres odbija się od osi poziomej.\\
A więc $$x \in \{-2\} \cup [6,14].$$
\rozwStop
\odpStart
$x \in \{-2\} \cup [6,14]$
\odpStop
\testStart
A.$x \in \{-2\} \cup [6,14]$\\
B.$x \in \{2\} \cup (6,14)$\\
C.$x \in \{-2\} \cup (6,14]$\\
D.$x \in \{2\} \cup (6,14]$\\
E.$x \in \{-2\} \cup [6,14)$\\
F.$x \in \{2\} \cup [6,14)$\\
G.$x \in \{-2\} \cup (6,14)$\\
H.$x \in \{2\} \cup [6,14]$
\testStop
\kluczStart
A
\kluczStop



\zadStart{Zadanie z Wikieł Z 1.62 c) moja wersja nr 183}

Rozwiązać nierówności $(6-x)(x+2)^{2}(15-x)^{3}\le0$.
\zadStop
\rozwStart{Patryk Wirkus}{}
Miejsca zerowe naszego wielomianu to: $6, -2, 15$.\\
Wielomian jest stopnia parzystego, ponadto znak współczynnika przy\linebreak najwyższej potędze x jest ujemny.\\ W związku z tym wykres wielomianu zaczyna się od lewej strony powyżej osi OX.\\
Ponadto w punkcie $-2$ wykres odbija się od osi poziomej.\\
A więc $$x \in \{-2\} \cup [6,15].$$
\rozwStop
\odpStart
$x \in \{-2\} \cup [6,15]$
\odpStop
\testStart
A.$x \in \{-2\} \cup [6,15]$\\
B.$x \in \{2\} \cup (6,15)$\\
C.$x \in \{-2\} \cup (6,15]$\\
D.$x \in \{2\} \cup (6,15]$\\
E.$x \in \{-2\} \cup [6,15)$\\
F.$x \in \{2\} \cup [6,15)$\\
G.$x \in \{-2\} \cup (6,15)$\\
H.$x \in \{2\} \cup [6,15]$
\testStop
\kluczStart
A
\kluczStop



\zadStart{Zadanie z Wikieł Z 1.62 c) moja wersja nr 184}

Rozwiązać nierówności $(6-x)(x+2)^{2}(16-x)^{3}\le0$.
\zadStop
\rozwStart{Patryk Wirkus}{}
Miejsca zerowe naszego wielomianu to: $6, -2, 16$.\\
Wielomian jest stopnia parzystego, ponadto znak współczynnika przy\linebreak najwyższej potędze x jest ujemny.\\ W związku z tym wykres wielomianu zaczyna się od lewej strony powyżej osi OX.\\
Ponadto w punkcie $-2$ wykres odbija się od osi poziomej.\\
A więc $$x \in \{-2\} \cup [6,16].$$
\rozwStop
\odpStart
$x \in \{-2\} \cup [6,16]$
\odpStop
\testStart
A.$x \in \{-2\} \cup [6,16]$\\
B.$x \in \{2\} \cup (6,16)$\\
C.$x \in \{-2\} \cup (6,16]$\\
D.$x \in \{2\} \cup (6,16]$\\
E.$x \in \{-2\} \cup [6,16)$\\
F.$x \in \{2\} \cup [6,16)$\\
G.$x \in \{-2\} \cup (6,16)$\\
H.$x \in \{2\} \cup [6,16]$
\testStop
\kluczStart
A
\kluczStop



\zadStart{Zadanie z Wikieł Z 1.62 c) moja wersja nr 185}

Rozwiązać nierówności $(6-x)(x+2)^{2}(17-x)^{3}\le0$.
\zadStop
\rozwStart{Patryk Wirkus}{}
Miejsca zerowe naszego wielomianu to: $6, -2, 17$.\\
Wielomian jest stopnia parzystego, ponadto znak współczynnika przy\linebreak najwyższej potędze x jest ujemny.\\ W związku z tym wykres wielomianu zaczyna się od lewej strony powyżej osi OX.\\
Ponadto w punkcie $-2$ wykres odbija się od osi poziomej.\\
A więc $$x \in \{-2\} \cup [6,17].$$
\rozwStop
\odpStart
$x \in \{-2\} \cup [6,17]$
\odpStop
\testStart
A.$x \in \{-2\} \cup [6,17]$\\
B.$x \in \{2\} \cup (6,17)$\\
C.$x \in \{-2\} \cup (6,17]$\\
D.$x \in \{2\} \cup (6,17]$\\
E.$x \in \{-2\} \cup [6,17)$\\
F.$x \in \{2\} \cup [6,17)$\\
G.$x \in \{-2\} \cup (6,17)$\\
H.$x \in \{2\} \cup [6,17]$
\testStop
\kluczStart
A
\kluczStop



\zadStart{Zadanie z Wikieł Z 1.62 c) moja wersja nr 186}

Rozwiązać nierówności $(6-x)(x+2)^{2}(18-x)^{3}\le0$.
\zadStop
\rozwStart{Patryk Wirkus}{}
Miejsca zerowe naszego wielomianu to: $6, -2, 18$.\\
Wielomian jest stopnia parzystego, ponadto znak współczynnika przy\linebreak najwyższej potędze x jest ujemny.\\ W związku z tym wykres wielomianu zaczyna się od lewej strony powyżej osi OX.\\
Ponadto w punkcie $-2$ wykres odbija się od osi poziomej.\\
A więc $$x \in \{-2\} \cup [6,18].$$
\rozwStop
\odpStart
$x \in \{-2\} \cup [6,18]$
\odpStop
\testStart
A.$x \in \{-2\} \cup [6,18]$\\
B.$x \in \{2\} \cup (6,18)$\\
C.$x \in \{-2\} \cup (6,18]$\\
D.$x \in \{2\} \cup (6,18]$\\
E.$x \in \{-2\} \cup [6,18)$\\
F.$x \in \{2\} \cup [6,18)$\\
G.$x \in \{-2\} \cup (6,18)$\\
H.$x \in \{2\} \cup [6,18]$
\testStop
\kluczStart
A
\kluczStop



\zadStart{Zadanie z Wikieł Z 1.62 c) moja wersja nr 187}

Rozwiązać nierówności $(6-x)(x+2)^{2}(19-x)^{3}\le0$.
\zadStop
\rozwStart{Patryk Wirkus}{}
Miejsca zerowe naszego wielomianu to: $6, -2, 19$.\\
Wielomian jest stopnia parzystego, ponadto znak współczynnika przy\linebreak najwyższej potędze x jest ujemny.\\ W związku z tym wykres wielomianu zaczyna się od lewej strony powyżej osi OX.\\
Ponadto w punkcie $-2$ wykres odbija się od osi poziomej.\\
A więc $$x \in \{-2\} \cup [6,19].$$
\rozwStop
\odpStart
$x \in \{-2\} \cup [6,19]$
\odpStop
\testStart
A.$x \in \{-2\} \cup [6,19]$\\
B.$x \in \{2\} \cup (6,19)$\\
C.$x \in \{-2\} \cup (6,19]$\\
D.$x \in \{2\} \cup (6,19]$\\
E.$x \in \{-2\} \cup [6,19)$\\
F.$x \in \{2\} \cup [6,19)$\\
G.$x \in \{-2\} \cup (6,19)$\\
H.$x \in \{2\} \cup [6,19]$
\testStop
\kluczStart
A
\kluczStop



\zadStart{Zadanie z Wikieł Z 1.62 c) moja wersja nr 188}

Rozwiązać nierówności $(6-x)(x+2)^{2}(20-x)^{3}\le0$.
\zadStop
\rozwStart{Patryk Wirkus}{}
Miejsca zerowe naszego wielomianu to: $6, -2, 20$.\\
Wielomian jest stopnia parzystego, ponadto znak współczynnika przy\linebreak najwyższej potędze x jest ujemny.\\ W związku z tym wykres wielomianu zaczyna się od lewej strony powyżej osi OX.\\
Ponadto w punkcie $-2$ wykres odbija się od osi poziomej.\\
A więc $$x \in \{-2\} \cup [6,20].$$
\rozwStop
\odpStart
$x \in \{-2\} \cup [6,20]$
\odpStop
\testStart
A.$x \in \{-2\} \cup [6,20]$\\
B.$x \in \{2\} \cup (6,20)$\\
C.$x \in \{-2\} \cup (6,20]$\\
D.$x \in \{2\} \cup (6,20]$\\
E.$x \in \{-2\} \cup [6,20)$\\
F.$x \in \{2\} \cup [6,20)$\\
G.$x \in \{-2\} \cup (6,20)$\\
H.$x \in \{2\} \cup [6,20]$
\testStop
\kluczStart
A
\kluczStop



\zadStart{Zadanie z Wikieł Z 1.62 c) moja wersja nr 189}

Rozwiązać nierówności $(6-x)(x+3)^{2}(7-x)^{3}\le0$.
\zadStop
\rozwStart{Patryk Wirkus}{}
Miejsca zerowe naszego wielomianu to: $6, -3, 7$.\\
Wielomian jest stopnia parzystego, ponadto znak współczynnika przy\linebreak najwyższej potędze x jest ujemny.\\ W związku z tym wykres wielomianu zaczyna się od lewej strony powyżej osi OX.\\
Ponadto w punkcie $-3$ wykres odbija się od osi poziomej.\\
A więc $$x \in \{-3\} \cup [6,7].$$
\rozwStop
\odpStart
$x \in \{-3\} \cup [6,7]$
\odpStop
\testStart
A.$x \in \{-3\} \cup [6,7]$\\
B.$x \in \{3\} \cup (6,7)$\\
C.$x \in \{-3\} \cup (6,7]$\\
D.$x \in \{3\} \cup (6,7]$\\
E.$x \in \{-3\} \cup [6,7)$\\
F.$x \in \{3\} \cup [6,7)$\\
G.$x \in \{-3\} \cup (6,7)$\\
H.$x \in \{3\} \cup [6,7]$
\testStop
\kluczStart
A
\kluczStop



\zadStart{Zadanie z Wikieł Z 1.62 c) moja wersja nr 190}

Rozwiązać nierówności $(6-x)(x+3)^{2}(8-x)^{3}\le0$.
\zadStop
\rozwStart{Patryk Wirkus}{}
Miejsca zerowe naszego wielomianu to: $6, -3, 8$.\\
Wielomian jest stopnia parzystego, ponadto znak współczynnika przy\linebreak najwyższej potędze x jest ujemny.\\ W związku z tym wykres wielomianu zaczyna się od lewej strony powyżej osi OX.\\
Ponadto w punkcie $-3$ wykres odbija się od osi poziomej.\\
A więc $$x \in \{-3\} \cup [6,8].$$
\rozwStop
\odpStart
$x \in \{-3\} \cup [6,8]$
\odpStop
\testStart
A.$x \in \{-3\} \cup [6,8]$\\
B.$x \in \{3\} \cup (6,8)$\\
C.$x \in \{-3\} \cup (6,8]$\\
D.$x \in \{3\} \cup (6,8]$\\
E.$x \in \{-3\} \cup [6,8)$\\
F.$x \in \{3\} \cup [6,8)$\\
G.$x \in \{-3\} \cup (6,8)$\\
H.$x \in \{3\} \cup [6,8]$
\testStop
\kluczStart
A
\kluczStop



\zadStart{Zadanie z Wikieł Z 1.62 c) moja wersja nr 191}

Rozwiązać nierówności $(6-x)(x+3)^{2}(9-x)^{3}\le0$.
\zadStop
\rozwStart{Patryk Wirkus}{}
Miejsca zerowe naszego wielomianu to: $6, -3, 9$.\\
Wielomian jest stopnia parzystego, ponadto znak współczynnika przy\linebreak najwyższej potędze x jest ujemny.\\ W związku z tym wykres wielomianu zaczyna się od lewej strony powyżej osi OX.\\
Ponadto w punkcie $-3$ wykres odbija się od osi poziomej.\\
A więc $$x \in \{-3\} \cup [6,9].$$
\rozwStop
\odpStart
$x \in \{-3\} \cup [6,9]$
\odpStop
\testStart
A.$x \in \{-3\} \cup [6,9]$\\
B.$x \in \{3\} \cup (6,9)$\\
C.$x \in \{-3\} \cup (6,9]$\\
D.$x \in \{3\} \cup (6,9]$\\
E.$x \in \{-3\} \cup [6,9)$\\
F.$x \in \{3\} \cup [6,9)$\\
G.$x \in \{-3\} \cup (6,9)$\\
H.$x \in \{3\} \cup [6,9]$
\testStop
\kluczStart
A
\kluczStop



\zadStart{Zadanie z Wikieł Z 1.62 c) moja wersja nr 192}

Rozwiązać nierówności $(6-x)(x+3)^{2}(10-x)^{3}\le0$.
\zadStop
\rozwStart{Patryk Wirkus}{}
Miejsca zerowe naszego wielomianu to: $6, -3, 10$.\\
Wielomian jest stopnia parzystego, ponadto znak współczynnika przy\linebreak najwyższej potędze x jest ujemny.\\ W związku z tym wykres wielomianu zaczyna się od lewej strony powyżej osi OX.\\
Ponadto w punkcie $-3$ wykres odbija się od osi poziomej.\\
A więc $$x \in \{-3\} \cup [6,10].$$
\rozwStop
\odpStart
$x \in \{-3\} \cup [6,10]$
\odpStop
\testStart
A.$x \in \{-3\} \cup [6,10]$\\
B.$x \in \{3\} \cup (6,10)$\\
C.$x \in \{-3\} \cup (6,10]$\\
D.$x \in \{3\} \cup (6,10]$\\
E.$x \in \{-3\} \cup [6,10)$\\
F.$x \in \{3\} \cup [6,10)$\\
G.$x \in \{-3\} \cup (6,10)$\\
H.$x \in \{3\} \cup [6,10]$
\testStop
\kluczStart
A
\kluczStop



\zadStart{Zadanie z Wikieł Z 1.62 c) moja wersja nr 193}

Rozwiązać nierówności $(6-x)(x+3)^{2}(11-x)^{3}\le0$.
\zadStop
\rozwStart{Patryk Wirkus}{}
Miejsca zerowe naszego wielomianu to: $6, -3, 11$.\\
Wielomian jest stopnia parzystego, ponadto znak współczynnika przy\linebreak najwyższej potędze x jest ujemny.\\ W związku z tym wykres wielomianu zaczyna się od lewej strony powyżej osi OX.\\
Ponadto w punkcie $-3$ wykres odbija się od osi poziomej.\\
A więc $$x \in \{-3\} \cup [6,11].$$
\rozwStop
\odpStart
$x \in \{-3\} \cup [6,11]$
\odpStop
\testStart
A.$x \in \{-3\} \cup [6,11]$\\
B.$x \in \{3\} \cup (6,11)$\\
C.$x \in \{-3\} \cup (6,11]$\\
D.$x \in \{3\} \cup (6,11]$\\
E.$x \in \{-3\} \cup [6,11)$\\
F.$x \in \{3\} \cup [6,11)$\\
G.$x \in \{-3\} \cup (6,11)$\\
H.$x \in \{3\} \cup [6,11]$
\testStop
\kluczStart
A
\kluczStop



\zadStart{Zadanie z Wikieł Z 1.62 c) moja wersja nr 194}

Rozwiązać nierówności $(6-x)(x+3)^{2}(12-x)^{3}\le0$.
\zadStop
\rozwStart{Patryk Wirkus}{}
Miejsca zerowe naszego wielomianu to: $6, -3, 12$.\\
Wielomian jest stopnia parzystego, ponadto znak współczynnika przy\linebreak najwyższej potędze x jest ujemny.\\ W związku z tym wykres wielomianu zaczyna się od lewej strony powyżej osi OX.\\
Ponadto w punkcie $-3$ wykres odbija się od osi poziomej.\\
A więc $$x \in \{-3\} \cup [6,12].$$
\rozwStop
\odpStart
$x \in \{-3\} \cup [6,12]$
\odpStop
\testStart
A.$x \in \{-3\} \cup [6,12]$\\
B.$x \in \{3\} \cup (6,12)$\\
C.$x \in \{-3\} \cup (6,12]$\\
D.$x \in \{3\} \cup (6,12]$\\
E.$x \in \{-3\} \cup [6,12)$\\
F.$x \in \{3\} \cup [6,12)$\\
G.$x \in \{-3\} \cup (6,12)$\\
H.$x \in \{3\} \cup [6,12]$
\testStop
\kluczStart
A
\kluczStop



\zadStart{Zadanie z Wikieł Z 1.62 c) moja wersja nr 195}

Rozwiązać nierówności $(6-x)(x+3)^{2}(13-x)^{3}\le0$.
\zadStop
\rozwStart{Patryk Wirkus}{}
Miejsca zerowe naszego wielomianu to: $6, -3, 13$.\\
Wielomian jest stopnia parzystego, ponadto znak współczynnika przy\linebreak najwyższej potędze x jest ujemny.\\ W związku z tym wykres wielomianu zaczyna się od lewej strony powyżej osi OX.\\
Ponadto w punkcie $-3$ wykres odbija się od osi poziomej.\\
A więc $$x \in \{-3\} \cup [6,13].$$
\rozwStop
\odpStart
$x \in \{-3\} \cup [6,13]$
\odpStop
\testStart
A.$x \in \{-3\} \cup [6,13]$\\
B.$x \in \{3\} \cup (6,13)$\\
C.$x \in \{-3\} \cup (6,13]$\\
D.$x \in \{3\} \cup (6,13]$\\
E.$x \in \{-3\} \cup [6,13)$\\
F.$x \in \{3\} \cup [6,13)$\\
G.$x \in \{-3\} \cup (6,13)$\\
H.$x \in \{3\} \cup [6,13]$
\testStop
\kluczStart
A
\kluczStop



\zadStart{Zadanie z Wikieł Z 1.62 c) moja wersja nr 196}

Rozwiązać nierówności $(6-x)(x+3)^{2}(14-x)^{3}\le0$.
\zadStop
\rozwStart{Patryk Wirkus}{}
Miejsca zerowe naszego wielomianu to: $6, -3, 14$.\\
Wielomian jest stopnia parzystego, ponadto znak współczynnika przy\linebreak najwyższej potędze x jest ujemny.\\ W związku z tym wykres wielomianu zaczyna się od lewej strony powyżej osi OX.\\
Ponadto w punkcie $-3$ wykres odbija się od osi poziomej.\\
A więc $$x \in \{-3\} \cup [6,14].$$
\rozwStop
\odpStart
$x \in \{-3\} \cup [6,14]$
\odpStop
\testStart
A.$x \in \{-3\} \cup [6,14]$\\
B.$x \in \{3\} \cup (6,14)$\\
C.$x \in \{-3\} \cup (6,14]$\\
D.$x \in \{3\} \cup (6,14]$\\
E.$x \in \{-3\} \cup [6,14)$\\
F.$x \in \{3\} \cup [6,14)$\\
G.$x \in \{-3\} \cup (6,14)$\\
H.$x \in \{3\} \cup [6,14]$
\testStop
\kluczStart
A
\kluczStop



\zadStart{Zadanie z Wikieł Z 1.62 c) moja wersja nr 197}

Rozwiązać nierówności $(6-x)(x+3)^{2}(15-x)^{3}\le0$.
\zadStop
\rozwStart{Patryk Wirkus}{}
Miejsca zerowe naszego wielomianu to: $6, -3, 15$.\\
Wielomian jest stopnia parzystego, ponadto znak współczynnika przy\linebreak najwyższej potędze x jest ujemny.\\ W związku z tym wykres wielomianu zaczyna się od lewej strony powyżej osi OX.\\
Ponadto w punkcie $-3$ wykres odbija się od osi poziomej.\\
A więc $$x \in \{-3\} \cup [6,15].$$
\rozwStop
\odpStart
$x \in \{-3\} \cup [6,15]$
\odpStop
\testStart
A.$x \in \{-3\} \cup [6,15]$\\
B.$x \in \{3\} \cup (6,15)$\\
C.$x \in \{-3\} \cup (6,15]$\\
D.$x \in \{3\} \cup (6,15]$\\
E.$x \in \{-3\} \cup [6,15)$\\
F.$x \in \{3\} \cup [6,15)$\\
G.$x \in \{-3\} \cup (6,15)$\\
H.$x \in \{3\} \cup [6,15]$
\testStop
\kluczStart
A
\kluczStop



\zadStart{Zadanie z Wikieł Z 1.62 c) moja wersja nr 198}

Rozwiązać nierówności $(6-x)(x+3)^{2}(16-x)^{3}\le0$.
\zadStop
\rozwStart{Patryk Wirkus}{}
Miejsca zerowe naszego wielomianu to: $6, -3, 16$.\\
Wielomian jest stopnia parzystego, ponadto znak współczynnika przy\linebreak najwyższej potędze x jest ujemny.\\ W związku z tym wykres wielomianu zaczyna się od lewej strony powyżej osi OX.\\
Ponadto w punkcie $-3$ wykres odbija się od osi poziomej.\\
A więc $$x \in \{-3\} \cup [6,16].$$
\rozwStop
\odpStart
$x \in \{-3\} \cup [6,16]$
\odpStop
\testStart
A.$x \in \{-3\} \cup [6,16]$\\
B.$x \in \{3\} \cup (6,16)$\\
C.$x \in \{-3\} \cup (6,16]$\\
D.$x \in \{3\} \cup (6,16]$\\
E.$x \in \{-3\} \cup [6,16)$\\
F.$x \in \{3\} \cup [6,16)$\\
G.$x \in \{-3\} \cup (6,16)$\\
H.$x \in \{3\} \cup [6,16]$
\testStop
\kluczStart
A
\kluczStop



\zadStart{Zadanie z Wikieł Z 1.62 c) moja wersja nr 199}

Rozwiązać nierówności $(6-x)(x+3)^{2}(17-x)^{3}\le0$.
\zadStop
\rozwStart{Patryk Wirkus}{}
Miejsca zerowe naszego wielomianu to: $6, -3, 17$.\\
Wielomian jest stopnia parzystego, ponadto znak współczynnika przy\linebreak najwyższej potędze x jest ujemny.\\ W związku z tym wykres wielomianu zaczyna się od lewej strony powyżej osi OX.\\
Ponadto w punkcie $-3$ wykres odbija się od osi poziomej.\\
A więc $$x \in \{-3\} \cup [6,17].$$
\rozwStop
\odpStart
$x \in \{-3\} \cup [6,17]$
\odpStop
\testStart
A.$x \in \{-3\} \cup [6,17]$\\
B.$x \in \{3\} \cup (6,17)$\\
C.$x \in \{-3\} \cup (6,17]$\\
D.$x \in \{3\} \cup (6,17]$\\
E.$x \in \{-3\} \cup [6,17)$\\
F.$x \in \{3\} \cup [6,17)$\\
G.$x \in \{-3\} \cup (6,17)$\\
H.$x \in \{3\} \cup [6,17]$
\testStop
\kluczStart
A
\kluczStop



\zadStart{Zadanie z Wikieł Z 1.62 c) moja wersja nr 200}

Rozwiązać nierówności $(6-x)(x+3)^{2}(18-x)^{3}\le0$.
\zadStop
\rozwStart{Patryk Wirkus}{}
Miejsca zerowe naszego wielomianu to: $6, -3, 18$.\\
Wielomian jest stopnia parzystego, ponadto znak współczynnika przy\linebreak najwyższej potędze x jest ujemny.\\ W związku z tym wykres wielomianu zaczyna się od lewej strony powyżej osi OX.\\
Ponadto w punkcie $-3$ wykres odbija się od osi poziomej.\\
A więc $$x \in \{-3\} \cup [6,18].$$
\rozwStop
\odpStart
$x \in \{-3\} \cup [6,18]$
\odpStop
\testStart
A.$x \in \{-3\} \cup [6,18]$\\
B.$x \in \{3\} \cup (6,18)$\\
C.$x \in \{-3\} \cup (6,18]$\\
D.$x \in \{3\} \cup (6,18]$\\
E.$x \in \{-3\} \cup [6,18)$\\
F.$x \in \{3\} \cup [6,18)$\\
G.$x \in \{-3\} \cup (6,18)$\\
H.$x \in \{3\} \cup [6,18]$
\testStop
\kluczStart
A
\kluczStop



\zadStart{Zadanie z Wikieł Z 1.62 c) moja wersja nr 201}

Rozwiązać nierówności $(6-x)(x+3)^{2}(19-x)^{3}\le0$.
\zadStop
\rozwStart{Patryk Wirkus}{}
Miejsca zerowe naszego wielomianu to: $6, -3, 19$.\\
Wielomian jest stopnia parzystego, ponadto znak współczynnika przy\linebreak najwyższej potędze x jest ujemny.\\ W związku z tym wykres wielomianu zaczyna się od lewej strony powyżej osi OX.\\
Ponadto w punkcie $-3$ wykres odbija się od osi poziomej.\\
A więc $$x \in \{-3\} \cup [6,19].$$
\rozwStop
\odpStart
$x \in \{-3\} \cup [6,19]$
\odpStop
\testStart
A.$x \in \{-3\} \cup [6,19]$\\
B.$x \in \{3\} \cup (6,19)$\\
C.$x \in \{-3\} \cup (6,19]$\\
D.$x \in \{3\} \cup (6,19]$\\
E.$x \in \{-3\} \cup [6,19)$\\
F.$x \in \{3\} \cup [6,19)$\\
G.$x \in \{-3\} \cup (6,19)$\\
H.$x \in \{3\} \cup [6,19]$
\testStop
\kluczStart
A
\kluczStop



\zadStart{Zadanie z Wikieł Z 1.62 c) moja wersja nr 202}

Rozwiązać nierówności $(6-x)(x+3)^{2}(20-x)^{3}\le0$.
\zadStop
\rozwStart{Patryk Wirkus}{}
Miejsca zerowe naszego wielomianu to: $6, -3, 20$.\\
Wielomian jest stopnia parzystego, ponadto znak współczynnika przy\linebreak najwyższej potędze x jest ujemny.\\ W związku z tym wykres wielomianu zaczyna się od lewej strony powyżej osi OX.\\
Ponadto w punkcie $-3$ wykres odbija się od osi poziomej.\\
A więc $$x \in \{-3\} \cup [6,20].$$
\rozwStop
\odpStart
$x \in \{-3\} \cup [6,20]$
\odpStop
\testStart
A.$x \in \{-3\} \cup [6,20]$\\
B.$x \in \{3\} \cup (6,20)$\\
C.$x \in \{-3\} \cup (6,20]$\\
D.$x \in \{3\} \cup (6,20]$\\
E.$x \in \{-3\} \cup [6,20)$\\
F.$x \in \{3\} \cup [6,20)$\\
G.$x \in \{-3\} \cup (6,20)$\\
H.$x \in \{3\} \cup [6,20]$
\testStop
\kluczStart
A
\kluczStop



\zadStart{Zadanie z Wikieł Z 1.62 c) moja wersja nr 203}

Rozwiązać nierówności $(6-x)(x+4)^{2}(7-x)^{3}\le0$.
\zadStop
\rozwStart{Patryk Wirkus}{}
Miejsca zerowe naszego wielomianu to: $6, -4, 7$.\\
Wielomian jest stopnia parzystego, ponadto znak współczynnika przy\linebreak najwyższej potędze x jest ujemny.\\ W związku z tym wykres wielomianu zaczyna się od lewej strony powyżej osi OX.\\
Ponadto w punkcie $-4$ wykres odbija się od osi poziomej.\\
A więc $$x \in \{-4\} \cup [6,7].$$
\rozwStop
\odpStart
$x \in \{-4\} \cup [6,7]$
\odpStop
\testStart
A.$x \in \{-4\} \cup [6,7]$\\
B.$x \in \{4\} \cup (6,7)$\\
C.$x \in \{-4\} \cup (6,7]$\\
D.$x \in \{4\} \cup (6,7]$\\
E.$x \in \{-4\} \cup [6,7)$\\
F.$x \in \{4\} \cup [6,7)$\\
G.$x \in \{-4\} \cup (6,7)$\\
H.$x \in \{4\} \cup [6,7]$
\testStop
\kluczStart
A
\kluczStop



\zadStart{Zadanie z Wikieł Z 1.62 c) moja wersja nr 204}

Rozwiązać nierówności $(6-x)(x+4)^{2}(8-x)^{3}\le0$.
\zadStop
\rozwStart{Patryk Wirkus}{}
Miejsca zerowe naszego wielomianu to: $6, -4, 8$.\\
Wielomian jest stopnia parzystego, ponadto znak współczynnika przy\linebreak najwyższej potędze x jest ujemny.\\ W związku z tym wykres wielomianu zaczyna się od lewej strony powyżej osi OX.\\
Ponadto w punkcie $-4$ wykres odbija się od osi poziomej.\\
A więc $$x \in \{-4\} \cup [6,8].$$
\rozwStop
\odpStart
$x \in \{-4\} \cup [6,8]$
\odpStop
\testStart
A.$x \in \{-4\} \cup [6,8]$\\
B.$x \in \{4\} \cup (6,8)$\\
C.$x \in \{-4\} \cup (6,8]$\\
D.$x \in \{4\} \cup (6,8]$\\
E.$x \in \{-4\} \cup [6,8)$\\
F.$x \in \{4\} \cup [6,8)$\\
G.$x \in \{-4\} \cup (6,8)$\\
H.$x \in \{4\} \cup [6,8]$
\testStop
\kluczStart
A
\kluczStop



\zadStart{Zadanie z Wikieł Z 1.62 c) moja wersja nr 205}

Rozwiązać nierówności $(6-x)(x+4)^{2}(9-x)^{3}\le0$.
\zadStop
\rozwStart{Patryk Wirkus}{}
Miejsca zerowe naszego wielomianu to: $6, -4, 9$.\\
Wielomian jest stopnia parzystego, ponadto znak współczynnika przy\linebreak najwyższej potędze x jest ujemny.\\ W związku z tym wykres wielomianu zaczyna się od lewej strony powyżej osi OX.\\
Ponadto w punkcie $-4$ wykres odbija się od osi poziomej.\\
A więc $$x \in \{-4\} \cup [6,9].$$
\rozwStop
\odpStart
$x \in \{-4\} \cup [6,9]$
\odpStop
\testStart
A.$x \in \{-4\} \cup [6,9]$\\
B.$x \in \{4\} \cup (6,9)$\\
C.$x \in \{-4\} \cup (6,9]$\\
D.$x \in \{4\} \cup (6,9]$\\
E.$x \in \{-4\} \cup [6,9)$\\
F.$x \in \{4\} \cup [6,9)$\\
G.$x \in \{-4\} \cup (6,9)$\\
H.$x \in \{4\} \cup [6,9]$
\testStop
\kluczStart
A
\kluczStop



\zadStart{Zadanie z Wikieł Z 1.62 c) moja wersja nr 206}

Rozwiązać nierówności $(6-x)(x+4)^{2}(10-x)^{3}\le0$.
\zadStop
\rozwStart{Patryk Wirkus}{}
Miejsca zerowe naszego wielomianu to: $6, -4, 10$.\\
Wielomian jest stopnia parzystego, ponadto znak współczynnika przy\linebreak najwyższej potędze x jest ujemny.\\ W związku z tym wykres wielomianu zaczyna się od lewej strony powyżej osi OX.\\
Ponadto w punkcie $-4$ wykres odbija się od osi poziomej.\\
A więc $$x \in \{-4\} \cup [6,10].$$
\rozwStop
\odpStart
$x \in \{-4\} \cup [6,10]$
\odpStop
\testStart
A.$x \in \{-4\} \cup [6,10]$\\
B.$x \in \{4\} \cup (6,10)$\\
C.$x \in \{-4\} \cup (6,10]$\\
D.$x \in \{4\} \cup (6,10]$\\
E.$x \in \{-4\} \cup [6,10)$\\
F.$x \in \{4\} \cup [6,10)$\\
G.$x \in \{-4\} \cup (6,10)$\\
H.$x \in \{4\} \cup [6,10]$
\testStop
\kluczStart
A
\kluczStop



\zadStart{Zadanie z Wikieł Z 1.62 c) moja wersja nr 207}

Rozwiązać nierówności $(6-x)(x+4)^{2}(11-x)^{3}\le0$.
\zadStop
\rozwStart{Patryk Wirkus}{}
Miejsca zerowe naszego wielomianu to: $6, -4, 11$.\\
Wielomian jest stopnia parzystego, ponadto znak współczynnika przy\linebreak najwyższej potędze x jest ujemny.\\ W związku z tym wykres wielomianu zaczyna się od lewej strony powyżej osi OX.\\
Ponadto w punkcie $-4$ wykres odbija się od osi poziomej.\\
A więc $$x \in \{-4\} \cup [6,11].$$
\rozwStop
\odpStart
$x \in \{-4\} \cup [6,11]$
\odpStop
\testStart
A.$x \in \{-4\} \cup [6,11]$\\
B.$x \in \{4\} \cup (6,11)$\\
C.$x \in \{-4\} \cup (6,11]$\\
D.$x \in \{4\} \cup (6,11]$\\
E.$x \in \{-4\} \cup [6,11)$\\
F.$x \in \{4\} \cup [6,11)$\\
G.$x \in \{-4\} \cup (6,11)$\\
H.$x \in \{4\} \cup [6,11]$
\testStop
\kluczStart
A
\kluczStop



\zadStart{Zadanie z Wikieł Z 1.62 c) moja wersja nr 208}

Rozwiązać nierówności $(6-x)(x+4)^{2}(12-x)^{3}\le0$.
\zadStop
\rozwStart{Patryk Wirkus}{}
Miejsca zerowe naszego wielomianu to: $6, -4, 12$.\\
Wielomian jest stopnia parzystego, ponadto znak współczynnika przy\linebreak najwyższej potędze x jest ujemny.\\ W związku z tym wykres wielomianu zaczyna się od lewej strony powyżej osi OX.\\
Ponadto w punkcie $-4$ wykres odbija się od osi poziomej.\\
A więc $$x \in \{-4\} \cup [6,12].$$
\rozwStop
\odpStart
$x \in \{-4\} \cup [6,12]$
\odpStop
\testStart
A.$x \in \{-4\} \cup [6,12]$\\
B.$x \in \{4\} \cup (6,12)$\\
C.$x \in \{-4\} \cup (6,12]$\\
D.$x \in \{4\} \cup (6,12]$\\
E.$x \in \{-4\} \cup [6,12)$\\
F.$x \in \{4\} \cup [6,12)$\\
G.$x \in \{-4\} \cup (6,12)$\\
H.$x \in \{4\} \cup [6,12]$
\testStop
\kluczStart
A
\kluczStop



\zadStart{Zadanie z Wikieł Z 1.62 c) moja wersja nr 209}

Rozwiązać nierówności $(6-x)(x+4)^{2}(13-x)^{3}\le0$.
\zadStop
\rozwStart{Patryk Wirkus}{}
Miejsca zerowe naszego wielomianu to: $6, -4, 13$.\\
Wielomian jest stopnia parzystego, ponadto znak współczynnika przy\linebreak najwyższej potędze x jest ujemny.\\ W związku z tym wykres wielomianu zaczyna się od lewej strony powyżej osi OX.\\
Ponadto w punkcie $-4$ wykres odbija się od osi poziomej.\\
A więc $$x \in \{-4\} \cup [6,13].$$
\rozwStop
\odpStart
$x \in \{-4\} \cup [6,13]$
\odpStop
\testStart
A.$x \in \{-4\} \cup [6,13]$\\
B.$x \in \{4\} \cup (6,13)$\\
C.$x \in \{-4\} \cup (6,13]$\\
D.$x \in \{4\} \cup (6,13]$\\
E.$x \in \{-4\} \cup [6,13)$\\
F.$x \in \{4\} \cup [6,13)$\\
G.$x \in \{-4\} \cup (6,13)$\\
H.$x \in \{4\} \cup [6,13]$
\testStop
\kluczStart
A
\kluczStop



\zadStart{Zadanie z Wikieł Z 1.62 c) moja wersja nr 210}

Rozwiązać nierówności $(6-x)(x+4)^{2}(14-x)^{3}\le0$.
\zadStop
\rozwStart{Patryk Wirkus}{}
Miejsca zerowe naszego wielomianu to: $6, -4, 14$.\\
Wielomian jest stopnia parzystego, ponadto znak współczynnika przy\linebreak najwyższej potędze x jest ujemny.\\ W związku z tym wykres wielomianu zaczyna się od lewej strony powyżej osi OX.\\
Ponadto w punkcie $-4$ wykres odbija się od osi poziomej.\\
A więc $$x \in \{-4\} \cup [6,14].$$
\rozwStop
\odpStart
$x \in \{-4\} \cup [6,14]$
\odpStop
\testStart
A.$x \in \{-4\} \cup [6,14]$\\
B.$x \in \{4\} \cup (6,14)$\\
C.$x \in \{-4\} \cup (6,14]$\\
D.$x \in \{4\} \cup (6,14]$\\
E.$x \in \{-4\} \cup [6,14)$\\
F.$x \in \{4\} \cup [6,14)$\\
G.$x \in \{-4\} \cup (6,14)$\\
H.$x \in \{4\} \cup [6,14]$
\testStop
\kluczStart
A
\kluczStop



\zadStart{Zadanie z Wikieł Z 1.62 c) moja wersja nr 211}

Rozwiązać nierówności $(6-x)(x+4)^{2}(15-x)^{3}\le0$.
\zadStop
\rozwStart{Patryk Wirkus}{}
Miejsca zerowe naszego wielomianu to: $6, -4, 15$.\\
Wielomian jest stopnia parzystego, ponadto znak współczynnika przy\linebreak najwyższej potędze x jest ujemny.\\ W związku z tym wykres wielomianu zaczyna się od lewej strony powyżej osi OX.\\
Ponadto w punkcie $-4$ wykres odbija się od osi poziomej.\\
A więc $$x \in \{-4\} \cup [6,15].$$
\rozwStop
\odpStart
$x \in \{-4\} \cup [6,15]$
\odpStop
\testStart
A.$x \in \{-4\} \cup [6,15]$\\
B.$x \in \{4\} \cup (6,15)$\\
C.$x \in \{-4\} \cup (6,15]$\\
D.$x \in \{4\} \cup (6,15]$\\
E.$x \in \{-4\} \cup [6,15)$\\
F.$x \in \{4\} \cup [6,15)$\\
G.$x \in \{-4\} \cup (6,15)$\\
H.$x \in \{4\} \cup [6,15]$
\testStop
\kluczStart
A
\kluczStop



\zadStart{Zadanie z Wikieł Z 1.62 c) moja wersja nr 212}

Rozwiązać nierówności $(6-x)(x+4)^{2}(16-x)^{3}\le0$.
\zadStop
\rozwStart{Patryk Wirkus}{}
Miejsca zerowe naszego wielomianu to: $6, -4, 16$.\\
Wielomian jest stopnia parzystego, ponadto znak współczynnika przy\linebreak najwyższej potędze x jest ujemny.\\ W związku z tym wykres wielomianu zaczyna się od lewej strony powyżej osi OX.\\
Ponadto w punkcie $-4$ wykres odbija się od osi poziomej.\\
A więc $$x \in \{-4\} \cup [6,16].$$
\rozwStop
\odpStart
$x \in \{-4\} \cup [6,16]$
\odpStop
\testStart
A.$x \in \{-4\} \cup [6,16]$\\
B.$x \in \{4\} \cup (6,16)$\\
C.$x \in \{-4\} \cup (6,16]$\\
D.$x \in \{4\} \cup (6,16]$\\
E.$x \in \{-4\} \cup [6,16)$\\
F.$x \in \{4\} \cup [6,16)$\\
G.$x \in \{-4\} \cup (6,16)$\\
H.$x \in \{4\} \cup [6,16]$
\testStop
\kluczStart
A
\kluczStop



\zadStart{Zadanie z Wikieł Z 1.62 c) moja wersja nr 213}

Rozwiązać nierówności $(6-x)(x+4)^{2}(17-x)^{3}\le0$.
\zadStop
\rozwStart{Patryk Wirkus}{}
Miejsca zerowe naszego wielomianu to: $6, -4, 17$.\\
Wielomian jest stopnia parzystego, ponadto znak współczynnika przy\linebreak najwyższej potędze x jest ujemny.\\ W związku z tym wykres wielomianu zaczyna się od lewej strony powyżej osi OX.\\
Ponadto w punkcie $-4$ wykres odbija się od osi poziomej.\\
A więc $$x \in \{-4\} \cup [6,17].$$
\rozwStop
\odpStart
$x \in \{-4\} \cup [6,17]$
\odpStop
\testStart
A.$x \in \{-4\} \cup [6,17]$\\
B.$x \in \{4\} \cup (6,17)$\\
C.$x \in \{-4\} \cup (6,17]$\\
D.$x \in \{4\} \cup (6,17]$\\
E.$x \in \{-4\} \cup [6,17)$\\
F.$x \in \{4\} \cup [6,17)$\\
G.$x \in \{-4\} \cup (6,17)$\\
H.$x \in \{4\} \cup [6,17]$
\testStop
\kluczStart
A
\kluczStop



\zadStart{Zadanie z Wikieł Z 1.62 c) moja wersja nr 214}

Rozwiązać nierówności $(6-x)(x+4)^{2}(18-x)^{3}\le0$.
\zadStop
\rozwStart{Patryk Wirkus}{}
Miejsca zerowe naszego wielomianu to: $6, -4, 18$.\\
Wielomian jest stopnia parzystego, ponadto znak współczynnika przy\linebreak najwyższej potędze x jest ujemny.\\ W związku z tym wykres wielomianu zaczyna się od lewej strony powyżej osi OX.\\
Ponadto w punkcie $-4$ wykres odbija się od osi poziomej.\\
A więc $$x \in \{-4\} \cup [6,18].$$
\rozwStop
\odpStart
$x \in \{-4\} \cup [6,18]$
\odpStop
\testStart
A.$x \in \{-4\} \cup [6,18]$\\
B.$x \in \{4\} \cup (6,18)$\\
C.$x \in \{-4\} \cup (6,18]$\\
D.$x \in \{4\} \cup (6,18]$\\
E.$x \in \{-4\} \cup [6,18)$\\
F.$x \in \{4\} \cup [6,18)$\\
G.$x \in \{-4\} \cup (6,18)$\\
H.$x \in \{4\} \cup [6,18]$
\testStop
\kluczStart
A
\kluczStop



\zadStart{Zadanie z Wikieł Z 1.62 c) moja wersja nr 215}

Rozwiązać nierówności $(6-x)(x+4)^{2}(19-x)^{3}\le0$.
\zadStop
\rozwStart{Patryk Wirkus}{}
Miejsca zerowe naszego wielomianu to: $6, -4, 19$.\\
Wielomian jest stopnia parzystego, ponadto znak współczynnika przy\linebreak najwyższej potędze x jest ujemny.\\ W związku z tym wykres wielomianu zaczyna się od lewej strony powyżej osi OX.\\
Ponadto w punkcie $-4$ wykres odbija się od osi poziomej.\\
A więc $$x \in \{-4\} \cup [6,19].$$
\rozwStop
\odpStart
$x \in \{-4\} \cup [6,19]$
\odpStop
\testStart
A.$x \in \{-4\} \cup [6,19]$\\
B.$x \in \{4\} \cup (6,19)$\\
C.$x \in \{-4\} \cup (6,19]$\\
D.$x \in \{4\} \cup (6,19]$\\
E.$x \in \{-4\} \cup [6,19)$\\
F.$x \in \{4\} \cup [6,19)$\\
G.$x \in \{-4\} \cup (6,19)$\\
H.$x \in \{4\} \cup [6,19]$
\testStop
\kluczStart
A
\kluczStop



\zadStart{Zadanie z Wikieł Z 1.62 c) moja wersja nr 216}

Rozwiązać nierówności $(6-x)(x+4)^{2}(20-x)^{3}\le0$.
\zadStop
\rozwStart{Patryk Wirkus}{}
Miejsca zerowe naszego wielomianu to: $6, -4, 20$.\\
Wielomian jest stopnia parzystego, ponadto znak współczynnika przy\linebreak najwyższej potędze x jest ujemny.\\ W związku z tym wykres wielomianu zaczyna się od lewej strony powyżej osi OX.\\
Ponadto w punkcie $-4$ wykres odbija się od osi poziomej.\\
A więc $$x \in \{-4\} \cup [6,20].$$
\rozwStop
\odpStart
$x \in \{-4\} \cup [6,20]$
\odpStop
\testStart
A.$x \in \{-4\} \cup [6,20]$\\
B.$x \in \{4\} \cup (6,20)$\\
C.$x \in \{-4\} \cup (6,20]$\\
D.$x \in \{4\} \cup (6,20]$\\
E.$x \in \{-4\} \cup [6,20)$\\
F.$x \in \{4\} \cup [6,20)$\\
G.$x \in \{-4\} \cup (6,20)$\\
H.$x \in \{4\} \cup [6,20]$
\testStop
\kluczStart
A
\kluczStop



\zadStart{Zadanie z Wikieł Z 1.62 c) moja wersja nr 217}

Rozwiązać nierówności $(6-x)(x+5)^{2}(7-x)^{3}\le0$.
\zadStop
\rozwStart{Patryk Wirkus}{}
Miejsca zerowe naszego wielomianu to: $6, -5, 7$.\\
Wielomian jest stopnia parzystego, ponadto znak współczynnika przy\linebreak najwyższej potędze x jest ujemny.\\ W związku z tym wykres wielomianu zaczyna się od lewej strony powyżej osi OX.\\
Ponadto w punkcie $-5$ wykres odbija się od osi poziomej.\\
A więc $$x \in \{-5\} \cup [6,7].$$
\rozwStop
\odpStart
$x \in \{-5\} \cup [6,7]$
\odpStop
\testStart
A.$x \in \{-5\} \cup [6,7]$\\
B.$x \in \{5\} \cup (6,7)$\\
C.$x \in \{-5\} \cup (6,7]$\\
D.$x \in \{5\} \cup (6,7]$\\
E.$x \in \{-5\} \cup [6,7)$\\
F.$x \in \{5\} \cup [6,7)$\\
G.$x \in \{-5\} \cup (6,7)$\\
H.$x \in \{5\} \cup [6,7]$
\testStop
\kluczStart
A
\kluczStop



\zadStart{Zadanie z Wikieł Z 1.62 c) moja wersja nr 218}

Rozwiązać nierówności $(6-x)(x+5)^{2}(8-x)^{3}\le0$.
\zadStop
\rozwStart{Patryk Wirkus}{}
Miejsca zerowe naszego wielomianu to: $6, -5, 8$.\\
Wielomian jest stopnia parzystego, ponadto znak współczynnika przy\linebreak najwyższej potędze x jest ujemny.\\ W związku z tym wykres wielomianu zaczyna się od lewej strony powyżej osi OX.\\
Ponadto w punkcie $-5$ wykres odbija się od osi poziomej.\\
A więc $$x \in \{-5\} \cup [6,8].$$
\rozwStop
\odpStart
$x \in \{-5\} \cup [6,8]$
\odpStop
\testStart
A.$x \in \{-5\} \cup [6,8]$\\
B.$x \in \{5\} \cup (6,8)$\\
C.$x \in \{-5\} \cup (6,8]$\\
D.$x \in \{5\} \cup (6,8]$\\
E.$x \in \{-5\} \cup [6,8)$\\
F.$x \in \{5\} \cup [6,8)$\\
G.$x \in \{-5\} \cup (6,8)$\\
H.$x \in \{5\} \cup [6,8]$
\testStop
\kluczStart
A
\kluczStop



\zadStart{Zadanie z Wikieł Z 1.62 c) moja wersja nr 219}

Rozwiązać nierówności $(6-x)(x+5)^{2}(9-x)^{3}\le0$.
\zadStop
\rozwStart{Patryk Wirkus}{}
Miejsca zerowe naszego wielomianu to: $6, -5, 9$.\\
Wielomian jest stopnia parzystego, ponadto znak współczynnika przy\linebreak najwyższej potędze x jest ujemny.\\ W związku z tym wykres wielomianu zaczyna się od lewej strony powyżej osi OX.\\
Ponadto w punkcie $-5$ wykres odbija się od osi poziomej.\\
A więc $$x \in \{-5\} \cup [6,9].$$
\rozwStop
\odpStart
$x \in \{-5\} \cup [6,9]$
\odpStop
\testStart
A.$x \in \{-5\} \cup [6,9]$\\
B.$x \in \{5\} \cup (6,9)$\\
C.$x \in \{-5\} \cup (6,9]$\\
D.$x \in \{5\} \cup (6,9]$\\
E.$x \in \{-5\} \cup [6,9)$\\
F.$x \in \{5\} \cup [6,9)$\\
G.$x \in \{-5\} \cup (6,9)$\\
H.$x \in \{5\} \cup [6,9]$
\testStop
\kluczStart
A
\kluczStop



\zadStart{Zadanie z Wikieł Z 1.62 c) moja wersja nr 220}

Rozwiązać nierówności $(6-x)(x+5)^{2}(10-x)^{3}\le0$.
\zadStop
\rozwStart{Patryk Wirkus}{}
Miejsca zerowe naszego wielomianu to: $6, -5, 10$.\\
Wielomian jest stopnia parzystego, ponadto znak współczynnika przy\linebreak najwyższej potędze x jest ujemny.\\ W związku z tym wykres wielomianu zaczyna się od lewej strony powyżej osi OX.\\
Ponadto w punkcie $-5$ wykres odbija się od osi poziomej.\\
A więc $$x \in \{-5\} \cup [6,10].$$
\rozwStop
\odpStart
$x \in \{-5\} \cup [6,10]$
\odpStop
\testStart
A.$x \in \{-5\} \cup [6,10]$\\
B.$x \in \{5\} \cup (6,10)$\\
C.$x \in \{-5\} \cup (6,10]$\\
D.$x \in \{5\} \cup (6,10]$\\
E.$x \in \{-5\} \cup [6,10)$\\
F.$x \in \{5\} \cup [6,10)$\\
G.$x \in \{-5\} \cup (6,10)$\\
H.$x \in \{5\} \cup [6,10]$
\testStop
\kluczStart
A
\kluczStop



\zadStart{Zadanie z Wikieł Z 1.62 c) moja wersja nr 221}

Rozwiązać nierówności $(6-x)(x+5)^{2}(11-x)^{3}\le0$.
\zadStop
\rozwStart{Patryk Wirkus}{}
Miejsca zerowe naszego wielomianu to: $6, -5, 11$.\\
Wielomian jest stopnia parzystego, ponadto znak współczynnika przy\linebreak najwyższej potędze x jest ujemny.\\ W związku z tym wykres wielomianu zaczyna się od lewej strony powyżej osi OX.\\
Ponadto w punkcie $-5$ wykres odbija się od osi poziomej.\\
A więc $$x \in \{-5\} \cup [6,11].$$
\rozwStop
\odpStart
$x \in \{-5\} \cup [6,11]$
\odpStop
\testStart
A.$x \in \{-5\} \cup [6,11]$\\
B.$x \in \{5\} \cup (6,11)$\\
C.$x \in \{-5\} \cup (6,11]$\\
D.$x \in \{5\} \cup (6,11]$\\
E.$x \in \{-5\} \cup [6,11)$\\
F.$x \in \{5\} \cup [6,11)$\\
G.$x \in \{-5\} \cup (6,11)$\\
H.$x \in \{5\} \cup [6,11]$
\testStop
\kluczStart
A
\kluczStop



\zadStart{Zadanie z Wikieł Z 1.62 c) moja wersja nr 222}

Rozwiązać nierówności $(6-x)(x+5)^{2}(12-x)^{3}\le0$.
\zadStop
\rozwStart{Patryk Wirkus}{}
Miejsca zerowe naszego wielomianu to: $6, -5, 12$.\\
Wielomian jest stopnia parzystego, ponadto znak współczynnika przy\linebreak najwyższej potędze x jest ujemny.\\ W związku z tym wykres wielomianu zaczyna się od lewej strony powyżej osi OX.\\
Ponadto w punkcie $-5$ wykres odbija się od osi poziomej.\\
A więc $$x \in \{-5\} \cup [6,12].$$
\rozwStop
\odpStart
$x \in \{-5\} \cup [6,12]$
\odpStop
\testStart
A.$x \in \{-5\} \cup [6,12]$\\
B.$x \in \{5\} \cup (6,12)$\\
C.$x \in \{-5\} \cup (6,12]$\\
D.$x \in \{5\} \cup (6,12]$\\
E.$x \in \{-5\} \cup [6,12)$\\
F.$x \in \{5\} \cup [6,12)$\\
G.$x \in \{-5\} \cup (6,12)$\\
H.$x \in \{5\} \cup [6,12]$
\testStop
\kluczStart
A
\kluczStop



\zadStart{Zadanie z Wikieł Z 1.62 c) moja wersja nr 223}

Rozwiązać nierówności $(6-x)(x+5)^{2}(13-x)^{3}\le0$.
\zadStop
\rozwStart{Patryk Wirkus}{}
Miejsca zerowe naszego wielomianu to: $6, -5, 13$.\\
Wielomian jest stopnia parzystego, ponadto znak współczynnika przy\linebreak najwyższej potędze x jest ujemny.\\ W związku z tym wykres wielomianu zaczyna się od lewej strony powyżej osi OX.\\
Ponadto w punkcie $-5$ wykres odbija się od osi poziomej.\\
A więc $$x \in \{-5\} \cup [6,13].$$
\rozwStop
\odpStart
$x \in \{-5\} \cup [6,13]$
\odpStop
\testStart
A.$x \in \{-5\} \cup [6,13]$\\
B.$x \in \{5\} \cup (6,13)$\\
C.$x \in \{-5\} \cup (6,13]$\\
D.$x \in \{5\} \cup (6,13]$\\
E.$x \in \{-5\} \cup [6,13)$\\
F.$x \in \{5\} \cup [6,13)$\\
G.$x \in \{-5\} \cup (6,13)$\\
H.$x \in \{5\} \cup [6,13]$
\testStop
\kluczStart
A
\kluczStop



\zadStart{Zadanie z Wikieł Z 1.62 c) moja wersja nr 224}

Rozwiązać nierówności $(6-x)(x+5)^{2}(14-x)^{3}\le0$.
\zadStop
\rozwStart{Patryk Wirkus}{}
Miejsca zerowe naszego wielomianu to: $6, -5, 14$.\\
Wielomian jest stopnia parzystego, ponadto znak współczynnika przy\linebreak najwyższej potędze x jest ujemny.\\ W związku z tym wykres wielomianu zaczyna się od lewej strony powyżej osi OX.\\
Ponadto w punkcie $-5$ wykres odbija się od osi poziomej.\\
A więc $$x \in \{-5\} \cup [6,14].$$
\rozwStop
\odpStart
$x \in \{-5\} \cup [6,14]$
\odpStop
\testStart
A.$x \in \{-5\} \cup [6,14]$\\
B.$x \in \{5\} \cup (6,14)$\\
C.$x \in \{-5\} \cup (6,14]$\\
D.$x \in \{5\} \cup (6,14]$\\
E.$x \in \{-5\} \cup [6,14)$\\
F.$x \in \{5\} \cup [6,14)$\\
G.$x \in \{-5\} \cup (6,14)$\\
H.$x \in \{5\} \cup [6,14]$
\testStop
\kluczStart
A
\kluczStop



\zadStart{Zadanie z Wikieł Z 1.62 c) moja wersja nr 225}

Rozwiązać nierówności $(6-x)(x+5)^{2}(15-x)^{3}\le0$.
\zadStop
\rozwStart{Patryk Wirkus}{}
Miejsca zerowe naszego wielomianu to: $6, -5, 15$.\\
Wielomian jest stopnia parzystego, ponadto znak współczynnika przy\linebreak najwyższej potędze x jest ujemny.\\ W związku z tym wykres wielomianu zaczyna się od lewej strony powyżej osi OX.\\
Ponadto w punkcie $-5$ wykres odbija się od osi poziomej.\\
A więc $$x \in \{-5\} \cup [6,15].$$
\rozwStop
\odpStart
$x \in \{-5\} \cup [6,15]$
\odpStop
\testStart
A.$x \in \{-5\} \cup [6,15]$\\
B.$x \in \{5\} \cup (6,15)$\\
C.$x \in \{-5\} \cup (6,15]$\\
D.$x \in \{5\} \cup (6,15]$\\
E.$x \in \{-5\} \cup [6,15)$\\
F.$x \in \{5\} \cup [6,15)$\\
G.$x \in \{-5\} \cup (6,15)$\\
H.$x \in \{5\} \cup [6,15]$
\testStop
\kluczStart
A
\kluczStop



\zadStart{Zadanie z Wikieł Z 1.62 c) moja wersja nr 226}

Rozwiązać nierówności $(6-x)(x+5)^{2}(16-x)^{3}\le0$.
\zadStop
\rozwStart{Patryk Wirkus}{}
Miejsca zerowe naszego wielomianu to: $6, -5, 16$.\\
Wielomian jest stopnia parzystego, ponadto znak współczynnika przy\linebreak najwyższej potędze x jest ujemny.\\ W związku z tym wykres wielomianu zaczyna się od lewej strony powyżej osi OX.\\
Ponadto w punkcie $-5$ wykres odbija się od osi poziomej.\\
A więc $$x \in \{-5\} \cup [6,16].$$
\rozwStop
\odpStart
$x \in \{-5\} \cup [6,16]$
\odpStop
\testStart
A.$x \in \{-5\} \cup [6,16]$\\
B.$x \in \{5\} \cup (6,16)$\\
C.$x \in \{-5\} \cup (6,16]$\\
D.$x \in \{5\} \cup (6,16]$\\
E.$x \in \{-5\} \cup [6,16)$\\
F.$x \in \{5\} \cup [6,16)$\\
G.$x \in \{-5\} \cup (6,16)$\\
H.$x \in \{5\} \cup [6,16]$
\testStop
\kluczStart
A
\kluczStop



\zadStart{Zadanie z Wikieł Z 1.62 c) moja wersja nr 227}

Rozwiązać nierówności $(6-x)(x+5)^{2}(17-x)^{3}\le0$.
\zadStop
\rozwStart{Patryk Wirkus}{}
Miejsca zerowe naszego wielomianu to: $6, -5, 17$.\\
Wielomian jest stopnia parzystego, ponadto znak współczynnika przy\linebreak najwyższej potędze x jest ujemny.\\ W związku z tym wykres wielomianu zaczyna się od lewej strony powyżej osi OX.\\
Ponadto w punkcie $-5$ wykres odbija się od osi poziomej.\\
A więc $$x \in \{-5\} \cup [6,17].$$
\rozwStop
\odpStart
$x \in \{-5\} \cup [6,17]$
\odpStop
\testStart
A.$x \in \{-5\} \cup [6,17]$\\
B.$x \in \{5\} \cup (6,17)$\\
C.$x \in \{-5\} \cup (6,17]$\\
D.$x \in \{5\} \cup (6,17]$\\
E.$x \in \{-5\} \cup [6,17)$\\
F.$x \in \{5\} \cup [6,17)$\\
G.$x \in \{-5\} \cup (6,17)$\\
H.$x \in \{5\} \cup [6,17]$
\testStop
\kluczStart
A
\kluczStop



\zadStart{Zadanie z Wikieł Z 1.62 c) moja wersja nr 228}

Rozwiązać nierówności $(6-x)(x+5)^{2}(18-x)^{3}\le0$.
\zadStop
\rozwStart{Patryk Wirkus}{}
Miejsca zerowe naszego wielomianu to: $6, -5, 18$.\\
Wielomian jest stopnia parzystego, ponadto znak współczynnika przy\linebreak najwyższej potędze x jest ujemny.\\ W związku z tym wykres wielomianu zaczyna się od lewej strony powyżej osi OX.\\
Ponadto w punkcie $-5$ wykres odbija się od osi poziomej.\\
A więc $$x \in \{-5\} \cup [6,18].$$
\rozwStop
\odpStart
$x \in \{-5\} \cup [6,18]$
\odpStop
\testStart
A.$x \in \{-5\} \cup [6,18]$\\
B.$x \in \{5\} \cup (6,18)$\\
C.$x \in \{-5\} \cup (6,18]$\\
D.$x \in \{5\} \cup (6,18]$\\
E.$x \in \{-5\} \cup [6,18)$\\
F.$x \in \{5\} \cup [6,18)$\\
G.$x \in \{-5\} \cup (6,18)$\\
H.$x \in \{5\} \cup [6,18]$
\testStop
\kluczStart
A
\kluczStop



\zadStart{Zadanie z Wikieł Z 1.62 c) moja wersja nr 229}

Rozwiązać nierówności $(6-x)(x+5)^{2}(19-x)^{3}\le0$.
\zadStop
\rozwStart{Patryk Wirkus}{}
Miejsca zerowe naszego wielomianu to: $6, -5, 19$.\\
Wielomian jest stopnia parzystego, ponadto znak współczynnika przy\linebreak najwyższej potędze x jest ujemny.\\ W związku z tym wykres wielomianu zaczyna się od lewej strony powyżej osi OX.\\
Ponadto w punkcie $-5$ wykres odbija się od osi poziomej.\\
A więc $$x \in \{-5\} \cup [6,19].$$
\rozwStop
\odpStart
$x \in \{-5\} \cup [6,19]$
\odpStop
\testStart
A.$x \in \{-5\} \cup [6,19]$\\
B.$x \in \{5\} \cup (6,19)$\\
C.$x \in \{-5\} \cup (6,19]$\\
D.$x \in \{5\} \cup (6,19]$\\
E.$x \in \{-5\} \cup [6,19)$\\
F.$x \in \{5\} \cup [6,19)$\\
G.$x \in \{-5\} \cup (6,19)$\\
H.$x \in \{5\} \cup [6,19]$
\testStop
\kluczStart
A
\kluczStop



\zadStart{Zadanie z Wikieł Z 1.62 c) moja wersja nr 230}

Rozwiązać nierówności $(6-x)(x+5)^{2}(20-x)^{3}\le0$.
\zadStop
\rozwStart{Patryk Wirkus}{}
Miejsca zerowe naszego wielomianu to: $6, -5, 20$.\\
Wielomian jest stopnia parzystego, ponadto znak współczynnika przy\linebreak najwyższej potędze x jest ujemny.\\ W związku z tym wykres wielomianu zaczyna się od lewej strony powyżej osi OX.\\
Ponadto w punkcie $-5$ wykres odbija się od osi poziomej.\\
A więc $$x \in \{-5\} \cup [6,20].$$
\rozwStop
\odpStart
$x \in \{-5\} \cup [6,20]$
\odpStop
\testStart
A.$x \in \{-5\} \cup [6,20]$\\
B.$x \in \{5\} \cup (6,20)$\\
C.$x \in \{-5\} \cup (6,20]$\\
D.$x \in \{5\} \cup (6,20]$\\
E.$x \in \{-5\} \cup [6,20)$\\
F.$x \in \{5\} \cup [6,20)$\\
G.$x \in \{-5\} \cup (6,20)$\\
H.$x \in \{5\} \cup [6,20]$
\testStop
\kluczStart
A
\kluczStop



\zadStart{Zadanie z Wikieł Z 1.62 c) moja wersja nr 231}

Rozwiązać nierówności $(7-x)(x+1)^{2}(8-x)^{3}\le0$.
\zadStop
\rozwStart{Patryk Wirkus}{}
Miejsca zerowe naszego wielomianu to: $7, -1, 8$.\\
Wielomian jest stopnia parzystego, ponadto znak współczynnika przy\linebreak najwyższej potędze x jest ujemny.\\ W związku z tym wykres wielomianu zaczyna się od lewej strony powyżej osi OX.\\
Ponadto w punkcie $-1$ wykres odbija się od osi poziomej.\\
A więc $$x \in \{-1\} \cup [7,8].$$
\rozwStop
\odpStart
$x \in \{-1\} \cup [7,8]$
\odpStop
\testStart
A.$x \in \{-1\} \cup [7,8]$\\
B.$x \in \{1\} \cup (7,8)$\\
C.$x \in \{-1\} \cup (7,8]$\\
D.$x \in \{1\} \cup (7,8]$\\
E.$x \in \{-1\} \cup [7,8)$\\
F.$x \in \{1\} \cup [7,8)$\\
G.$x \in \{-1\} \cup (7,8)$\\
H.$x \in \{1\} \cup [7,8]$
\testStop
\kluczStart
A
\kluczStop



\zadStart{Zadanie z Wikieł Z 1.62 c) moja wersja nr 232}

Rozwiązać nierówności $(7-x)(x+1)^{2}(9-x)^{3}\le0$.
\zadStop
\rozwStart{Patryk Wirkus}{}
Miejsca zerowe naszego wielomianu to: $7, -1, 9$.\\
Wielomian jest stopnia parzystego, ponadto znak współczynnika przy\linebreak najwyższej potędze x jest ujemny.\\ W związku z tym wykres wielomianu zaczyna się od lewej strony powyżej osi OX.\\
Ponadto w punkcie $-1$ wykres odbija się od osi poziomej.\\
A więc $$x \in \{-1\} \cup [7,9].$$
\rozwStop
\odpStart
$x \in \{-1\} \cup [7,9]$
\odpStop
\testStart
A.$x \in \{-1\} \cup [7,9]$\\
B.$x \in \{1\} \cup (7,9)$\\
C.$x \in \{-1\} \cup (7,9]$\\
D.$x \in \{1\} \cup (7,9]$\\
E.$x \in \{-1\} \cup [7,9)$\\
F.$x \in \{1\} \cup [7,9)$\\
G.$x \in \{-1\} \cup (7,9)$\\
H.$x \in \{1\} \cup [7,9]$
\testStop
\kluczStart
A
\kluczStop



\zadStart{Zadanie z Wikieł Z 1.62 c) moja wersja nr 233}

Rozwiązać nierówności $(7-x)(x+1)^{2}(10-x)^{3}\le0$.
\zadStop
\rozwStart{Patryk Wirkus}{}
Miejsca zerowe naszego wielomianu to: $7, -1, 10$.\\
Wielomian jest stopnia parzystego, ponadto znak współczynnika przy\linebreak najwyższej potędze x jest ujemny.\\ W związku z tym wykres wielomianu zaczyna się od lewej strony powyżej osi OX.\\
Ponadto w punkcie $-1$ wykres odbija się od osi poziomej.\\
A więc $$x \in \{-1\} \cup [7,10].$$
\rozwStop
\odpStart
$x \in \{-1\} \cup [7,10]$
\odpStop
\testStart
A.$x \in \{-1\} \cup [7,10]$\\
B.$x \in \{1\} \cup (7,10)$\\
C.$x \in \{-1\} \cup (7,10]$\\
D.$x \in \{1\} \cup (7,10]$\\
E.$x \in \{-1\} \cup [7,10)$\\
F.$x \in \{1\} \cup [7,10)$\\
G.$x \in \{-1\} \cup (7,10)$\\
H.$x \in \{1\} \cup [7,10]$
\testStop
\kluczStart
A
\kluczStop



\zadStart{Zadanie z Wikieł Z 1.62 c) moja wersja nr 234}

Rozwiązać nierówności $(7-x)(x+1)^{2}(11-x)^{3}\le0$.
\zadStop
\rozwStart{Patryk Wirkus}{}
Miejsca zerowe naszego wielomianu to: $7, -1, 11$.\\
Wielomian jest stopnia parzystego, ponadto znak współczynnika przy\linebreak najwyższej potędze x jest ujemny.\\ W związku z tym wykres wielomianu zaczyna się od lewej strony powyżej osi OX.\\
Ponadto w punkcie $-1$ wykres odbija się od osi poziomej.\\
A więc $$x \in \{-1\} \cup [7,11].$$
\rozwStop
\odpStart
$x \in \{-1\} \cup [7,11]$
\odpStop
\testStart
A.$x \in \{-1\} \cup [7,11]$\\
B.$x \in \{1\} \cup (7,11)$\\
C.$x \in \{-1\} \cup (7,11]$\\
D.$x \in \{1\} \cup (7,11]$\\
E.$x \in \{-1\} \cup [7,11)$\\
F.$x \in \{1\} \cup [7,11)$\\
G.$x \in \{-1\} \cup (7,11)$\\
H.$x \in \{1\} \cup [7,11]$
\testStop
\kluczStart
A
\kluczStop



\zadStart{Zadanie z Wikieł Z 1.62 c) moja wersja nr 235}

Rozwiązać nierówności $(7-x)(x+1)^{2}(12-x)^{3}\le0$.
\zadStop
\rozwStart{Patryk Wirkus}{}
Miejsca zerowe naszego wielomianu to: $7, -1, 12$.\\
Wielomian jest stopnia parzystego, ponadto znak współczynnika przy\linebreak najwyższej potędze x jest ujemny.\\ W związku z tym wykres wielomianu zaczyna się od lewej strony powyżej osi OX.\\
Ponadto w punkcie $-1$ wykres odbija się od osi poziomej.\\
A więc $$x \in \{-1\} \cup [7,12].$$
\rozwStop
\odpStart
$x \in \{-1\} \cup [7,12]$
\odpStop
\testStart
A.$x \in \{-1\} \cup [7,12]$\\
B.$x \in \{1\} \cup (7,12)$\\
C.$x \in \{-1\} \cup (7,12]$\\
D.$x \in \{1\} \cup (7,12]$\\
E.$x \in \{-1\} \cup [7,12)$\\
F.$x \in \{1\} \cup [7,12)$\\
G.$x \in \{-1\} \cup (7,12)$\\
H.$x \in \{1\} \cup [7,12]$
\testStop
\kluczStart
A
\kluczStop



\zadStart{Zadanie z Wikieł Z 1.62 c) moja wersja nr 236}

Rozwiązać nierówności $(7-x)(x+1)^{2}(13-x)^{3}\le0$.
\zadStop
\rozwStart{Patryk Wirkus}{}
Miejsca zerowe naszego wielomianu to: $7, -1, 13$.\\
Wielomian jest stopnia parzystego, ponadto znak współczynnika przy\linebreak najwyższej potędze x jest ujemny.\\ W związku z tym wykres wielomianu zaczyna się od lewej strony powyżej osi OX.\\
Ponadto w punkcie $-1$ wykres odbija się od osi poziomej.\\
A więc $$x \in \{-1\} \cup [7,13].$$
\rozwStop
\odpStart
$x \in \{-1\} \cup [7,13]$
\odpStop
\testStart
A.$x \in \{-1\} \cup [7,13]$\\
B.$x \in \{1\} \cup (7,13)$\\
C.$x \in \{-1\} \cup (7,13]$\\
D.$x \in \{1\} \cup (7,13]$\\
E.$x \in \{-1\} \cup [7,13)$\\
F.$x \in \{1\} \cup [7,13)$\\
G.$x \in \{-1\} \cup (7,13)$\\
H.$x \in \{1\} \cup [7,13]$
\testStop
\kluczStart
A
\kluczStop



\zadStart{Zadanie z Wikieł Z 1.62 c) moja wersja nr 237}

Rozwiązać nierówności $(7-x)(x+1)^{2}(14-x)^{3}\le0$.
\zadStop
\rozwStart{Patryk Wirkus}{}
Miejsca zerowe naszego wielomianu to: $7, -1, 14$.\\
Wielomian jest stopnia parzystego, ponadto znak współczynnika przy\linebreak najwyższej potędze x jest ujemny.\\ W związku z tym wykres wielomianu zaczyna się od lewej strony powyżej osi OX.\\
Ponadto w punkcie $-1$ wykres odbija się od osi poziomej.\\
A więc $$x \in \{-1\} \cup [7,14].$$
\rozwStop
\odpStart
$x \in \{-1\} \cup [7,14]$
\odpStop
\testStart
A.$x \in \{-1\} \cup [7,14]$\\
B.$x \in \{1\} \cup (7,14)$\\
C.$x \in \{-1\} \cup (7,14]$\\
D.$x \in \{1\} \cup (7,14]$\\
E.$x \in \{-1\} \cup [7,14)$\\
F.$x \in \{1\} \cup [7,14)$\\
G.$x \in \{-1\} \cup (7,14)$\\
H.$x \in \{1\} \cup [7,14]$
\testStop
\kluczStart
A
\kluczStop



\zadStart{Zadanie z Wikieł Z 1.62 c) moja wersja nr 238}

Rozwiązać nierówności $(7-x)(x+1)^{2}(15-x)^{3}\le0$.
\zadStop
\rozwStart{Patryk Wirkus}{}
Miejsca zerowe naszego wielomianu to: $7, -1, 15$.\\
Wielomian jest stopnia parzystego, ponadto znak współczynnika przy\linebreak najwyższej potędze x jest ujemny.\\ W związku z tym wykres wielomianu zaczyna się od lewej strony powyżej osi OX.\\
Ponadto w punkcie $-1$ wykres odbija się od osi poziomej.\\
A więc $$x \in \{-1\} \cup [7,15].$$
\rozwStop
\odpStart
$x \in \{-1\} \cup [7,15]$
\odpStop
\testStart
A.$x \in \{-1\} \cup [7,15]$\\
B.$x \in \{1\} \cup (7,15)$\\
C.$x \in \{-1\} \cup (7,15]$\\
D.$x \in \{1\} \cup (7,15]$\\
E.$x \in \{-1\} \cup [7,15)$\\
F.$x \in \{1\} \cup [7,15)$\\
G.$x \in \{-1\} \cup (7,15)$\\
H.$x \in \{1\} \cup [7,15]$
\testStop
\kluczStart
A
\kluczStop



\zadStart{Zadanie z Wikieł Z 1.62 c) moja wersja nr 239}

Rozwiązać nierówności $(7-x)(x+1)^{2}(16-x)^{3}\le0$.
\zadStop
\rozwStart{Patryk Wirkus}{}
Miejsca zerowe naszego wielomianu to: $7, -1, 16$.\\
Wielomian jest stopnia parzystego, ponadto znak współczynnika przy\linebreak najwyższej potędze x jest ujemny.\\ W związku z tym wykres wielomianu zaczyna się od lewej strony powyżej osi OX.\\
Ponadto w punkcie $-1$ wykres odbija się od osi poziomej.\\
A więc $$x \in \{-1\} \cup [7,16].$$
\rozwStop
\odpStart
$x \in \{-1\} \cup [7,16]$
\odpStop
\testStart
A.$x \in \{-1\} \cup [7,16]$\\
B.$x \in \{1\} \cup (7,16)$\\
C.$x \in \{-1\} \cup (7,16]$\\
D.$x \in \{1\} \cup (7,16]$\\
E.$x \in \{-1\} \cup [7,16)$\\
F.$x \in \{1\} \cup [7,16)$\\
G.$x \in \{-1\} \cup (7,16)$\\
H.$x \in \{1\} \cup [7,16]$
\testStop
\kluczStart
A
\kluczStop



\zadStart{Zadanie z Wikieł Z 1.62 c) moja wersja nr 240}

Rozwiązać nierówności $(7-x)(x+1)^{2}(17-x)^{3}\le0$.
\zadStop
\rozwStart{Patryk Wirkus}{}
Miejsca zerowe naszego wielomianu to: $7, -1, 17$.\\
Wielomian jest stopnia parzystego, ponadto znak współczynnika przy\linebreak najwyższej potędze x jest ujemny.\\ W związku z tym wykres wielomianu zaczyna się od lewej strony powyżej osi OX.\\
Ponadto w punkcie $-1$ wykres odbija się od osi poziomej.\\
A więc $$x \in \{-1\} \cup [7,17].$$
\rozwStop
\odpStart
$x \in \{-1\} \cup [7,17]$
\odpStop
\testStart
A.$x \in \{-1\} \cup [7,17]$\\
B.$x \in \{1\} \cup (7,17)$\\
C.$x \in \{-1\} \cup (7,17]$\\
D.$x \in \{1\} \cup (7,17]$\\
E.$x \in \{-1\} \cup [7,17)$\\
F.$x \in \{1\} \cup [7,17)$\\
G.$x \in \{-1\} \cup (7,17)$\\
H.$x \in \{1\} \cup [7,17]$
\testStop
\kluczStart
A
\kluczStop



\zadStart{Zadanie z Wikieł Z 1.62 c) moja wersja nr 241}

Rozwiązać nierówności $(7-x)(x+1)^{2}(18-x)^{3}\le0$.
\zadStop
\rozwStart{Patryk Wirkus}{}
Miejsca zerowe naszego wielomianu to: $7, -1, 18$.\\
Wielomian jest stopnia parzystego, ponadto znak współczynnika przy\linebreak najwyższej potędze x jest ujemny.\\ W związku z tym wykres wielomianu zaczyna się od lewej strony powyżej osi OX.\\
Ponadto w punkcie $-1$ wykres odbija się od osi poziomej.\\
A więc $$x \in \{-1\} \cup [7,18].$$
\rozwStop
\odpStart
$x \in \{-1\} \cup [7,18]$
\odpStop
\testStart
A.$x \in \{-1\} \cup [7,18]$\\
B.$x \in \{1\} \cup (7,18)$\\
C.$x \in \{-1\} \cup (7,18]$\\
D.$x \in \{1\} \cup (7,18]$\\
E.$x \in \{-1\} \cup [7,18)$\\
F.$x \in \{1\} \cup [7,18)$\\
G.$x \in \{-1\} \cup (7,18)$\\
H.$x \in \{1\} \cup [7,18]$
\testStop
\kluczStart
A
\kluczStop



\zadStart{Zadanie z Wikieł Z 1.62 c) moja wersja nr 242}

Rozwiązać nierówności $(7-x)(x+1)^{2}(19-x)^{3}\le0$.
\zadStop
\rozwStart{Patryk Wirkus}{}
Miejsca zerowe naszego wielomianu to: $7, -1, 19$.\\
Wielomian jest stopnia parzystego, ponadto znak współczynnika przy\linebreak najwyższej potędze x jest ujemny.\\ W związku z tym wykres wielomianu zaczyna się od lewej strony powyżej osi OX.\\
Ponadto w punkcie $-1$ wykres odbija się od osi poziomej.\\
A więc $$x \in \{-1\} \cup [7,19].$$
\rozwStop
\odpStart
$x \in \{-1\} \cup [7,19]$
\odpStop
\testStart
A.$x \in \{-1\} \cup [7,19]$\\
B.$x \in \{1\} \cup (7,19)$\\
C.$x \in \{-1\} \cup (7,19]$\\
D.$x \in \{1\} \cup (7,19]$\\
E.$x \in \{-1\} \cup [7,19)$\\
F.$x \in \{1\} \cup [7,19)$\\
G.$x \in \{-1\} \cup (7,19)$\\
H.$x \in \{1\} \cup [7,19]$
\testStop
\kluczStart
A
\kluczStop



\zadStart{Zadanie z Wikieł Z 1.62 c) moja wersja nr 243}

Rozwiązać nierówności $(7-x)(x+1)^{2}(20-x)^{3}\le0$.
\zadStop
\rozwStart{Patryk Wirkus}{}
Miejsca zerowe naszego wielomianu to: $7, -1, 20$.\\
Wielomian jest stopnia parzystego, ponadto znak współczynnika przy\linebreak najwyższej potędze x jest ujemny.\\ W związku z tym wykres wielomianu zaczyna się od lewej strony powyżej osi OX.\\
Ponadto w punkcie $-1$ wykres odbija się od osi poziomej.\\
A więc $$x \in \{-1\} \cup [7,20].$$
\rozwStop
\odpStart
$x \in \{-1\} \cup [7,20]$
\odpStop
\testStart
A.$x \in \{-1\} \cup [7,20]$\\
B.$x \in \{1\} \cup (7,20)$\\
C.$x \in \{-1\} \cup (7,20]$\\
D.$x \in \{1\} \cup (7,20]$\\
E.$x \in \{-1\} \cup [7,20)$\\
F.$x \in \{1\} \cup [7,20)$\\
G.$x \in \{-1\} \cup (7,20)$\\
H.$x \in \{1\} \cup [7,20]$
\testStop
\kluczStart
A
\kluczStop



\zadStart{Zadanie z Wikieł Z 1.62 c) moja wersja nr 244}

Rozwiązać nierówności $(7-x)(x+2)^{2}(8-x)^{3}\le0$.
\zadStop
\rozwStart{Patryk Wirkus}{}
Miejsca zerowe naszego wielomianu to: $7, -2, 8$.\\
Wielomian jest stopnia parzystego, ponadto znak współczynnika przy\linebreak najwyższej potędze x jest ujemny.\\ W związku z tym wykres wielomianu zaczyna się od lewej strony powyżej osi OX.\\
Ponadto w punkcie $-2$ wykres odbija się od osi poziomej.\\
A więc $$x \in \{-2\} \cup [7,8].$$
\rozwStop
\odpStart
$x \in \{-2\} \cup [7,8]$
\odpStop
\testStart
A.$x \in \{-2\} \cup [7,8]$\\
B.$x \in \{2\} \cup (7,8)$\\
C.$x \in \{-2\} \cup (7,8]$\\
D.$x \in \{2\} \cup (7,8]$\\
E.$x \in \{-2\} \cup [7,8)$\\
F.$x \in \{2\} \cup [7,8)$\\
G.$x \in \{-2\} \cup (7,8)$\\
H.$x \in \{2\} \cup [7,8]$
\testStop
\kluczStart
A
\kluczStop



\zadStart{Zadanie z Wikieł Z 1.62 c) moja wersja nr 245}

Rozwiązać nierówności $(7-x)(x+2)^{2}(9-x)^{3}\le0$.
\zadStop
\rozwStart{Patryk Wirkus}{}
Miejsca zerowe naszego wielomianu to: $7, -2, 9$.\\
Wielomian jest stopnia parzystego, ponadto znak współczynnika przy\linebreak najwyższej potędze x jest ujemny.\\ W związku z tym wykres wielomianu zaczyna się od lewej strony powyżej osi OX.\\
Ponadto w punkcie $-2$ wykres odbija się od osi poziomej.\\
A więc $$x \in \{-2\} \cup [7,9].$$
\rozwStop
\odpStart
$x \in \{-2\} \cup [7,9]$
\odpStop
\testStart
A.$x \in \{-2\} \cup [7,9]$\\
B.$x \in \{2\} \cup (7,9)$\\
C.$x \in \{-2\} \cup (7,9]$\\
D.$x \in \{2\} \cup (7,9]$\\
E.$x \in \{-2\} \cup [7,9)$\\
F.$x \in \{2\} \cup [7,9)$\\
G.$x \in \{-2\} \cup (7,9)$\\
H.$x \in \{2\} \cup [7,9]$
\testStop
\kluczStart
A
\kluczStop



\zadStart{Zadanie z Wikieł Z 1.62 c) moja wersja nr 246}

Rozwiązać nierówności $(7-x)(x+2)^{2}(10-x)^{3}\le0$.
\zadStop
\rozwStart{Patryk Wirkus}{}
Miejsca zerowe naszego wielomianu to: $7, -2, 10$.\\
Wielomian jest stopnia parzystego, ponadto znak współczynnika przy\linebreak najwyższej potędze x jest ujemny.\\ W związku z tym wykres wielomianu zaczyna się od lewej strony powyżej osi OX.\\
Ponadto w punkcie $-2$ wykres odbija się od osi poziomej.\\
A więc $$x \in \{-2\} \cup [7,10].$$
\rozwStop
\odpStart
$x \in \{-2\} \cup [7,10]$
\odpStop
\testStart
A.$x \in \{-2\} \cup [7,10]$\\
B.$x \in \{2\} \cup (7,10)$\\
C.$x \in \{-2\} \cup (7,10]$\\
D.$x \in \{2\} \cup (7,10]$\\
E.$x \in \{-2\} \cup [7,10)$\\
F.$x \in \{2\} \cup [7,10)$\\
G.$x \in \{-2\} \cup (7,10)$\\
H.$x \in \{2\} \cup [7,10]$
\testStop
\kluczStart
A
\kluczStop



\zadStart{Zadanie z Wikieł Z 1.62 c) moja wersja nr 247}

Rozwiązać nierówności $(7-x)(x+2)^{2}(11-x)^{3}\le0$.
\zadStop
\rozwStart{Patryk Wirkus}{}
Miejsca zerowe naszego wielomianu to: $7, -2, 11$.\\
Wielomian jest stopnia parzystego, ponadto znak współczynnika przy\linebreak najwyższej potędze x jest ujemny.\\ W związku z tym wykres wielomianu zaczyna się od lewej strony powyżej osi OX.\\
Ponadto w punkcie $-2$ wykres odbija się od osi poziomej.\\
A więc $$x \in \{-2\} \cup [7,11].$$
\rozwStop
\odpStart
$x \in \{-2\} \cup [7,11]$
\odpStop
\testStart
A.$x \in \{-2\} \cup [7,11]$\\
B.$x \in \{2\} \cup (7,11)$\\
C.$x \in \{-2\} \cup (7,11]$\\
D.$x \in \{2\} \cup (7,11]$\\
E.$x \in \{-2\} \cup [7,11)$\\
F.$x \in \{2\} \cup [7,11)$\\
G.$x \in \{-2\} \cup (7,11)$\\
H.$x \in \{2\} \cup [7,11]$
\testStop
\kluczStart
A
\kluczStop



\zadStart{Zadanie z Wikieł Z 1.62 c) moja wersja nr 248}

Rozwiązać nierówności $(7-x)(x+2)^{2}(12-x)^{3}\le0$.
\zadStop
\rozwStart{Patryk Wirkus}{}
Miejsca zerowe naszego wielomianu to: $7, -2, 12$.\\
Wielomian jest stopnia parzystego, ponadto znak współczynnika przy\linebreak najwyższej potędze x jest ujemny.\\ W związku z tym wykres wielomianu zaczyna się od lewej strony powyżej osi OX.\\
Ponadto w punkcie $-2$ wykres odbija się od osi poziomej.\\
A więc $$x \in \{-2\} \cup [7,12].$$
\rozwStop
\odpStart
$x \in \{-2\} \cup [7,12]$
\odpStop
\testStart
A.$x \in \{-2\} \cup [7,12]$\\
B.$x \in \{2\} \cup (7,12)$\\
C.$x \in \{-2\} \cup (7,12]$\\
D.$x \in \{2\} \cup (7,12]$\\
E.$x \in \{-2\} \cup [7,12)$\\
F.$x \in \{2\} \cup [7,12)$\\
G.$x \in \{-2\} \cup (7,12)$\\
H.$x \in \{2\} \cup [7,12]$
\testStop
\kluczStart
A
\kluczStop



\zadStart{Zadanie z Wikieł Z 1.62 c) moja wersja nr 249}

Rozwiązać nierówności $(7-x)(x+2)^{2}(13-x)^{3}\le0$.
\zadStop
\rozwStart{Patryk Wirkus}{}
Miejsca zerowe naszego wielomianu to: $7, -2, 13$.\\
Wielomian jest stopnia parzystego, ponadto znak współczynnika przy\linebreak najwyższej potędze x jest ujemny.\\ W związku z tym wykres wielomianu zaczyna się od lewej strony powyżej osi OX.\\
Ponadto w punkcie $-2$ wykres odbija się od osi poziomej.\\
A więc $$x \in \{-2\} \cup [7,13].$$
\rozwStop
\odpStart
$x \in \{-2\} \cup [7,13]$
\odpStop
\testStart
A.$x \in \{-2\} \cup [7,13]$\\
B.$x \in \{2\} \cup (7,13)$\\
C.$x \in \{-2\} \cup (7,13]$\\
D.$x \in \{2\} \cup (7,13]$\\
E.$x \in \{-2\} \cup [7,13)$\\
F.$x \in \{2\} \cup [7,13)$\\
G.$x \in \{-2\} \cup (7,13)$\\
H.$x \in \{2\} \cup [7,13]$
\testStop
\kluczStart
A
\kluczStop



\zadStart{Zadanie z Wikieł Z 1.62 c) moja wersja nr 250}

Rozwiązać nierówności $(7-x)(x+2)^{2}(14-x)^{3}\le0$.
\zadStop
\rozwStart{Patryk Wirkus}{}
Miejsca zerowe naszego wielomianu to: $7, -2, 14$.\\
Wielomian jest stopnia parzystego, ponadto znak współczynnika przy\linebreak najwyższej potędze x jest ujemny.\\ W związku z tym wykres wielomianu zaczyna się od lewej strony powyżej osi OX.\\
Ponadto w punkcie $-2$ wykres odbija się od osi poziomej.\\
A więc $$x \in \{-2\} \cup [7,14].$$
\rozwStop
\odpStart
$x \in \{-2\} \cup [7,14]$
\odpStop
\testStart
A.$x \in \{-2\} \cup [7,14]$\\
B.$x \in \{2\} \cup (7,14)$\\
C.$x \in \{-2\} \cup (7,14]$\\
D.$x \in \{2\} \cup (7,14]$\\
E.$x \in \{-2\} \cup [7,14)$\\
F.$x \in \{2\} \cup [7,14)$\\
G.$x \in \{-2\} \cup (7,14)$\\
H.$x \in \{2\} \cup [7,14]$
\testStop
\kluczStart
A
\kluczStop



\zadStart{Zadanie z Wikieł Z 1.62 c) moja wersja nr 251}

Rozwiązać nierówności $(7-x)(x+2)^{2}(15-x)^{3}\le0$.
\zadStop
\rozwStart{Patryk Wirkus}{}
Miejsca zerowe naszego wielomianu to: $7, -2, 15$.\\
Wielomian jest stopnia parzystego, ponadto znak współczynnika przy\linebreak najwyższej potędze x jest ujemny.\\ W związku z tym wykres wielomianu zaczyna się od lewej strony powyżej osi OX.\\
Ponadto w punkcie $-2$ wykres odbija się od osi poziomej.\\
A więc $$x \in \{-2\} \cup [7,15].$$
\rozwStop
\odpStart
$x \in \{-2\} \cup [7,15]$
\odpStop
\testStart
A.$x \in \{-2\} \cup [7,15]$\\
B.$x \in \{2\} \cup (7,15)$\\
C.$x \in \{-2\} \cup (7,15]$\\
D.$x \in \{2\} \cup (7,15]$\\
E.$x \in \{-2\} \cup [7,15)$\\
F.$x \in \{2\} \cup [7,15)$\\
G.$x \in \{-2\} \cup (7,15)$\\
H.$x \in \{2\} \cup [7,15]$
\testStop
\kluczStart
A
\kluczStop



\zadStart{Zadanie z Wikieł Z 1.62 c) moja wersja nr 252}

Rozwiązać nierówności $(7-x)(x+2)^{2}(16-x)^{3}\le0$.
\zadStop
\rozwStart{Patryk Wirkus}{}
Miejsca zerowe naszego wielomianu to: $7, -2, 16$.\\
Wielomian jest stopnia parzystego, ponadto znak współczynnika przy\linebreak najwyższej potędze x jest ujemny.\\ W związku z tym wykres wielomianu zaczyna się od lewej strony powyżej osi OX.\\
Ponadto w punkcie $-2$ wykres odbija się od osi poziomej.\\
A więc $$x \in \{-2\} \cup [7,16].$$
\rozwStop
\odpStart
$x \in \{-2\} \cup [7,16]$
\odpStop
\testStart
A.$x \in \{-2\} \cup [7,16]$\\
B.$x \in \{2\} \cup (7,16)$\\
C.$x \in \{-2\} \cup (7,16]$\\
D.$x \in \{2\} \cup (7,16]$\\
E.$x \in \{-2\} \cup [7,16)$\\
F.$x \in \{2\} \cup [7,16)$\\
G.$x \in \{-2\} \cup (7,16)$\\
H.$x \in \{2\} \cup [7,16]$
\testStop
\kluczStart
A
\kluczStop



\zadStart{Zadanie z Wikieł Z 1.62 c) moja wersja nr 253}

Rozwiązać nierówności $(7-x)(x+2)^{2}(17-x)^{3}\le0$.
\zadStop
\rozwStart{Patryk Wirkus}{}
Miejsca zerowe naszego wielomianu to: $7, -2, 17$.\\
Wielomian jest stopnia parzystego, ponadto znak współczynnika przy\linebreak najwyższej potędze x jest ujemny.\\ W związku z tym wykres wielomianu zaczyna się od lewej strony powyżej osi OX.\\
Ponadto w punkcie $-2$ wykres odbija się od osi poziomej.\\
A więc $$x \in \{-2\} \cup [7,17].$$
\rozwStop
\odpStart
$x \in \{-2\} \cup [7,17]$
\odpStop
\testStart
A.$x \in \{-2\} \cup [7,17]$\\
B.$x \in \{2\} \cup (7,17)$\\
C.$x \in \{-2\} \cup (7,17]$\\
D.$x \in \{2\} \cup (7,17]$\\
E.$x \in \{-2\} \cup [7,17)$\\
F.$x \in \{2\} \cup [7,17)$\\
G.$x \in \{-2\} \cup (7,17)$\\
H.$x \in \{2\} \cup [7,17]$
\testStop
\kluczStart
A
\kluczStop



\zadStart{Zadanie z Wikieł Z 1.62 c) moja wersja nr 254}

Rozwiązać nierówności $(7-x)(x+2)^{2}(18-x)^{3}\le0$.
\zadStop
\rozwStart{Patryk Wirkus}{}
Miejsca zerowe naszego wielomianu to: $7, -2, 18$.\\
Wielomian jest stopnia parzystego, ponadto znak współczynnika przy\linebreak najwyższej potędze x jest ujemny.\\ W związku z tym wykres wielomianu zaczyna się od lewej strony powyżej osi OX.\\
Ponadto w punkcie $-2$ wykres odbija się od osi poziomej.\\
A więc $$x \in \{-2\} \cup [7,18].$$
\rozwStop
\odpStart
$x \in \{-2\} \cup [7,18]$
\odpStop
\testStart
A.$x \in \{-2\} \cup [7,18]$\\
B.$x \in \{2\} \cup (7,18)$\\
C.$x \in \{-2\} \cup (7,18]$\\
D.$x \in \{2\} \cup (7,18]$\\
E.$x \in \{-2\} \cup [7,18)$\\
F.$x \in \{2\} \cup [7,18)$\\
G.$x \in \{-2\} \cup (7,18)$\\
H.$x \in \{2\} \cup [7,18]$
\testStop
\kluczStart
A
\kluczStop



\zadStart{Zadanie z Wikieł Z 1.62 c) moja wersja nr 255}

Rozwiązać nierówności $(7-x)(x+2)^{2}(19-x)^{3}\le0$.
\zadStop
\rozwStart{Patryk Wirkus}{}
Miejsca zerowe naszego wielomianu to: $7, -2, 19$.\\
Wielomian jest stopnia parzystego, ponadto znak współczynnika przy\linebreak najwyższej potędze x jest ujemny.\\ W związku z tym wykres wielomianu zaczyna się od lewej strony powyżej osi OX.\\
Ponadto w punkcie $-2$ wykres odbija się od osi poziomej.\\
A więc $$x \in \{-2\} \cup [7,19].$$
\rozwStop
\odpStart
$x \in \{-2\} \cup [7,19]$
\odpStop
\testStart
A.$x \in \{-2\} \cup [7,19]$\\
B.$x \in \{2\} \cup (7,19)$\\
C.$x \in \{-2\} \cup (7,19]$\\
D.$x \in \{2\} \cup (7,19]$\\
E.$x \in \{-2\} \cup [7,19)$\\
F.$x \in \{2\} \cup [7,19)$\\
G.$x \in \{-2\} \cup (7,19)$\\
H.$x \in \{2\} \cup [7,19]$
\testStop
\kluczStart
A
\kluczStop



\zadStart{Zadanie z Wikieł Z 1.62 c) moja wersja nr 256}

Rozwiązać nierówności $(7-x)(x+2)^{2}(20-x)^{3}\le0$.
\zadStop
\rozwStart{Patryk Wirkus}{}
Miejsca zerowe naszego wielomianu to: $7, -2, 20$.\\
Wielomian jest stopnia parzystego, ponadto znak współczynnika przy\linebreak najwyższej potędze x jest ujemny.\\ W związku z tym wykres wielomianu zaczyna się od lewej strony powyżej osi OX.\\
Ponadto w punkcie $-2$ wykres odbija się od osi poziomej.\\
A więc $$x \in \{-2\} \cup [7,20].$$
\rozwStop
\odpStart
$x \in \{-2\} \cup [7,20]$
\odpStop
\testStart
A.$x \in \{-2\} \cup [7,20]$\\
B.$x \in \{2\} \cup (7,20)$\\
C.$x \in \{-2\} \cup (7,20]$\\
D.$x \in \{2\} \cup (7,20]$\\
E.$x \in \{-2\} \cup [7,20)$\\
F.$x \in \{2\} \cup [7,20)$\\
G.$x \in \{-2\} \cup (7,20)$\\
H.$x \in \{2\} \cup [7,20]$
\testStop
\kluczStart
A
\kluczStop



\zadStart{Zadanie z Wikieł Z 1.62 c) moja wersja nr 257}

Rozwiązać nierówności $(7-x)(x+3)^{2}(8-x)^{3}\le0$.
\zadStop
\rozwStart{Patryk Wirkus}{}
Miejsca zerowe naszego wielomianu to: $7, -3, 8$.\\
Wielomian jest stopnia parzystego, ponadto znak współczynnika przy\linebreak najwyższej potędze x jest ujemny.\\ W związku z tym wykres wielomianu zaczyna się od lewej strony powyżej osi OX.\\
Ponadto w punkcie $-3$ wykres odbija się od osi poziomej.\\
A więc $$x \in \{-3\} \cup [7,8].$$
\rozwStop
\odpStart
$x \in \{-3\} \cup [7,8]$
\odpStop
\testStart
A.$x \in \{-3\} \cup [7,8]$\\
B.$x \in \{3\} \cup (7,8)$\\
C.$x \in \{-3\} \cup (7,8]$\\
D.$x \in \{3\} \cup (7,8]$\\
E.$x \in \{-3\} \cup [7,8)$\\
F.$x \in \{3\} \cup [7,8)$\\
G.$x \in \{-3\} \cup (7,8)$\\
H.$x \in \{3\} \cup [7,8]$
\testStop
\kluczStart
A
\kluczStop



\zadStart{Zadanie z Wikieł Z 1.62 c) moja wersja nr 258}

Rozwiązać nierówności $(7-x)(x+3)^{2}(9-x)^{3}\le0$.
\zadStop
\rozwStart{Patryk Wirkus}{}
Miejsca zerowe naszego wielomianu to: $7, -3, 9$.\\
Wielomian jest stopnia parzystego, ponadto znak współczynnika przy\linebreak najwyższej potędze x jest ujemny.\\ W związku z tym wykres wielomianu zaczyna się od lewej strony powyżej osi OX.\\
Ponadto w punkcie $-3$ wykres odbija się od osi poziomej.\\
A więc $$x \in \{-3\} \cup [7,9].$$
\rozwStop
\odpStart
$x \in \{-3\} \cup [7,9]$
\odpStop
\testStart
A.$x \in \{-3\} \cup [7,9]$\\
B.$x \in \{3\} \cup (7,9)$\\
C.$x \in \{-3\} \cup (7,9]$\\
D.$x \in \{3\} \cup (7,9]$\\
E.$x \in \{-3\} \cup [7,9)$\\
F.$x \in \{3\} \cup [7,9)$\\
G.$x \in \{-3\} \cup (7,9)$\\
H.$x \in \{3\} \cup [7,9]$
\testStop
\kluczStart
A
\kluczStop



\zadStart{Zadanie z Wikieł Z 1.62 c) moja wersja nr 259}

Rozwiązać nierówności $(7-x)(x+3)^{2}(10-x)^{3}\le0$.
\zadStop
\rozwStart{Patryk Wirkus}{}
Miejsca zerowe naszego wielomianu to: $7, -3, 10$.\\
Wielomian jest stopnia parzystego, ponadto znak współczynnika przy\linebreak najwyższej potędze x jest ujemny.\\ W związku z tym wykres wielomianu zaczyna się od lewej strony powyżej osi OX.\\
Ponadto w punkcie $-3$ wykres odbija się od osi poziomej.\\
A więc $$x \in \{-3\} \cup [7,10].$$
\rozwStop
\odpStart
$x \in \{-3\} \cup [7,10]$
\odpStop
\testStart
A.$x \in \{-3\} \cup [7,10]$\\
B.$x \in \{3\} \cup (7,10)$\\
C.$x \in \{-3\} \cup (7,10]$\\
D.$x \in \{3\} \cup (7,10]$\\
E.$x \in \{-3\} \cup [7,10)$\\
F.$x \in \{3\} \cup [7,10)$\\
G.$x \in \{-3\} \cup (7,10)$\\
H.$x \in \{3\} \cup [7,10]$
\testStop
\kluczStart
A
\kluczStop



\zadStart{Zadanie z Wikieł Z 1.62 c) moja wersja nr 260}

Rozwiązać nierówności $(7-x)(x+3)^{2}(11-x)^{3}\le0$.
\zadStop
\rozwStart{Patryk Wirkus}{}
Miejsca zerowe naszego wielomianu to: $7, -3, 11$.\\
Wielomian jest stopnia parzystego, ponadto znak współczynnika przy\linebreak najwyższej potędze x jest ujemny.\\ W związku z tym wykres wielomianu zaczyna się od lewej strony powyżej osi OX.\\
Ponadto w punkcie $-3$ wykres odbija się od osi poziomej.\\
A więc $$x \in \{-3\} \cup [7,11].$$
\rozwStop
\odpStart
$x \in \{-3\} \cup [7,11]$
\odpStop
\testStart
A.$x \in \{-3\} \cup [7,11]$\\
B.$x \in \{3\} \cup (7,11)$\\
C.$x \in \{-3\} \cup (7,11]$\\
D.$x \in \{3\} \cup (7,11]$\\
E.$x \in \{-3\} \cup [7,11)$\\
F.$x \in \{3\} \cup [7,11)$\\
G.$x \in \{-3\} \cup (7,11)$\\
H.$x \in \{3\} \cup [7,11]$
\testStop
\kluczStart
A
\kluczStop



\zadStart{Zadanie z Wikieł Z 1.62 c) moja wersja nr 261}

Rozwiązać nierówności $(7-x)(x+3)^{2}(12-x)^{3}\le0$.
\zadStop
\rozwStart{Patryk Wirkus}{}
Miejsca zerowe naszego wielomianu to: $7, -3, 12$.\\
Wielomian jest stopnia parzystego, ponadto znak współczynnika przy\linebreak najwyższej potędze x jest ujemny.\\ W związku z tym wykres wielomianu zaczyna się od lewej strony powyżej osi OX.\\
Ponadto w punkcie $-3$ wykres odbija się od osi poziomej.\\
A więc $$x \in \{-3\} \cup [7,12].$$
\rozwStop
\odpStart
$x \in \{-3\} \cup [7,12]$
\odpStop
\testStart
A.$x \in \{-3\} \cup [7,12]$\\
B.$x \in \{3\} \cup (7,12)$\\
C.$x \in \{-3\} \cup (7,12]$\\
D.$x \in \{3\} \cup (7,12]$\\
E.$x \in \{-3\} \cup [7,12)$\\
F.$x \in \{3\} \cup [7,12)$\\
G.$x \in \{-3\} \cup (7,12)$\\
H.$x \in \{3\} \cup [7,12]$
\testStop
\kluczStart
A
\kluczStop



\zadStart{Zadanie z Wikieł Z 1.62 c) moja wersja nr 262}

Rozwiązać nierówności $(7-x)(x+3)^{2}(13-x)^{3}\le0$.
\zadStop
\rozwStart{Patryk Wirkus}{}
Miejsca zerowe naszego wielomianu to: $7, -3, 13$.\\
Wielomian jest stopnia parzystego, ponadto znak współczynnika przy\linebreak najwyższej potędze x jest ujemny.\\ W związku z tym wykres wielomianu zaczyna się od lewej strony powyżej osi OX.\\
Ponadto w punkcie $-3$ wykres odbija się od osi poziomej.\\
A więc $$x \in \{-3\} \cup [7,13].$$
\rozwStop
\odpStart
$x \in \{-3\} \cup [7,13]$
\odpStop
\testStart
A.$x \in \{-3\} \cup [7,13]$\\
B.$x \in \{3\} \cup (7,13)$\\
C.$x \in \{-3\} \cup (7,13]$\\
D.$x \in \{3\} \cup (7,13]$\\
E.$x \in \{-3\} \cup [7,13)$\\
F.$x \in \{3\} \cup [7,13)$\\
G.$x \in \{-3\} \cup (7,13)$\\
H.$x \in \{3\} \cup [7,13]$
\testStop
\kluczStart
A
\kluczStop



\zadStart{Zadanie z Wikieł Z 1.62 c) moja wersja nr 263}

Rozwiązać nierówności $(7-x)(x+3)^{2}(14-x)^{3}\le0$.
\zadStop
\rozwStart{Patryk Wirkus}{}
Miejsca zerowe naszego wielomianu to: $7, -3, 14$.\\
Wielomian jest stopnia parzystego, ponadto znak współczynnika przy\linebreak najwyższej potędze x jest ujemny.\\ W związku z tym wykres wielomianu zaczyna się od lewej strony powyżej osi OX.\\
Ponadto w punkcie $-3$ wykres odbija się od osi poziomej.\\
A więc $$x \in \{-3\} \cup [7,14].$$
\rozwStop
\odpStart
$x \in \{-3\} \cup [7,14]$
\odpStop
\testStart
A.$x \in \{-3\} \cup [7,14]$\\
B.$x \in \{3\} \cup (7,14)$\\
C.$x \in \{-3\} \cup (7,14]$\\
D.$x \in \{3\} \cup (7,14]$\\
E.$x \in \{-3\} \cup [7,14)$\\
F.$x \in \{3\} \cup [7,14)$\\
G.$x \in \{-3\} \cup (7,14)$\\
H.$x \in \{3\} \cup [7,14]$
\testStop
\kluczStart
A
\kluczStop



\zadStart{Zadanie z Wikieł Z 1.62 c) moja wersja nr 264}

Rozwiązać nierówności $(7-x)(x+3)^{2}(15-x)^{3}\le0$.
\zadStop
\rozwStart{Patryk Wirkus}{}
Miejsca zerowe naszego wielomianu to: $7, -3, 15$.\\
Wielomian jest stopnia parzystego, ponadto znak współczynnika przy\linebreak najwyższej potędze x jest ujemny.\\ W związku z tym wykres wielomianu zaczyna się od lewej strony powyżej osi OX.\\
Ponadto w punkcie $-3$ wykres odbija się od osi poziomej.\\
A więc $$x \in \{-3\} \cup [7,15].$$
\rozwStop
\odpStart
$x \in \{-3\} \cup [7,15]$
\odpStop
\testStart
A.$x \in \{-3\} \cup [7,15]$\\
B.$x \in \{3\} \cup (7,15)$\\
C.$x \in \{-3\} \cup (7,15]$\\
D.$x \in \{3\} \cup (7,15]$\\
E.$x \in \{-3\} \cup [7,15)$\\
F.$x \in \{3\} \cup [7,15)$\\
G.$x \in \{-3\} \cup (7,15)$\\
H.$x \in \{3\} \cup [7,15]$
\testStop
\kluczStart
A
\kluczStop



\zadStart{Zadanie z Wikieł Z 1.62 c) moja wersja nr 265}

Rozwiązać nierówności $(7-x)(x+3)^{2}(16-x)^{3}\le0$.
\zadStop
\rozwStart{Patryk Wirkus}{}
Miejsca zerowe naszego wielomianu to: $7, -3, 16$.\\
Wielomian jest stopnia parzystego, ponadto znak współczynnika przy\linebreak najwyższej potędze x jest ujemny.\\ W związku z tym wykres wielomianu zaczyna się od lewej strony powyżej osi OX.\\
Ponadto w punkcie $-3$ wykres odbija się od osi poziomej.\\
A więc $$x \in \{-3\} \cup [7,16].$$
\rozwStop
\odpStart
$x \in \{-3\} \cup [7,16]$
\odpStop
\testStart
A.$x \in \{-3\} \cup [7,16]$\\
B.$x \in \{3\} \cup (7,16)$\\
C.$x \in \{-3\} \cup (7,16]$\\
D.$x \in \{3\} \cup (7,16]$\\
E.$x \in \{-3\} \cup [7,16)$\\
F.$x \in \{3\} \cup [7,16)$\\
G.$x \in \{-3\} \cup (7,16)$\\
H.$x \in \{3\} \cup [7,16]$
\testStop
\kluczStart
A
\kluczStop



\zadStart{Zadanie z Wikieł Z 1.62 c) moja wersja nr 266}

Rozwiązać nierówności $(7-x)(x+3)^{2}(17-x)^{3}\le0$.
\zadStop
\rozwStart{Patryk Wirkus}{}
Miejsca zerowe naszego wielomianu to: $7, -3, 17$.\\
Wielomian jest stopnia parzystego, ponadto znak współczynnika przy\linebreak najwyższej potędze x jest ujemny.\\ W związku z tym wykres wielomianu zaczyna się od lewej strony powyżej osi OX.\\
Ponadto w punkcie $-3$ wykres odbija się od osi poziomej.\\
A więc $$x \in \{-3\} \cup [7,17].$$
\rozwStop
\odpStart
$x \in \{-3\} \cup [7,17]$
\odpStop
\testStart
A.$x \in \{-3\} \cup [7,17]$\\
B.$x \in \{3\} \cup (7,17)$\\
C.$x \in \{-3\} \cup (7,17]$\\
D.$x \in \{3\} \cup (7,17]$\\
E.$x \in \{-3\} \cup [7,17)$\\
F.$x \in \{3\} \cup [7,17)$\\
G.$x \in \{-3\} \cup (7,17)$\\
H.$x \in \{3\} \cup [7,17]$
\testStop
\kluczStart
A
\kluczStop



\zadStart{Zadanie z Wikieł Z 1.62 c) moja wersja nr 267}

Rozwiązać nierówności $(7-x)(x+3)^{2}(18-x)^{3}\le0$.
\zadStop
\rozwStart{Patryk Wirkus}{}
Miejsca zerowe naszego wielomianu to: $7, -3, 18$.\\
Wielomian jest stopnia parzystego, ponadto znak współczynnika przy\linebreak najwyższej potędze x jest ujemny.\\ W związku z tym wykres wielomianu zaczyna się od lewej strony powyżej osi OX.\\
Ponadto w punkcie $-3$ wykres odbija się od osi poziomej.\\
A więc $$x \in \{-3\} \cup [7,18].$$
\rozwStop
\odpStart
$x \in \{-3\} \cup [7,18]$
\odpStop
\testStart
A.$x \in \{-3\} \cup [7,18]$\\
B.$x \in \{3\} \cup (7,18)$\\
C.$x \in \{-3\} \cup (7,18]$\\
D.$x \in \{3\} \cup (7,18]$\\
E.$x \in \{-3\} \cup [7,18)$\\
F.$x \in \{3\} \cup [7,18)$\\
G.$x \in \{-3\} \cup (7,18)$\\
H.$x \in \{3\} \cup [7,18]$
\testStop
\kluczStart
A
\kluczStop



\zadStart{Zadanie z Wikieł Z 1.62 c) moja wersja nr 268}

Rozwiązać nierówności $(7-x)(x+3)^{2}(19-x)^{3}\le0$.
\zadStop
\rozwStart{Patryk Wirkus}{}
Miejsca zerowe naszego wielomianu to: $7, -3, 19$.\\
Wielomian jest stopnia parzystego, ponadto znak współczynnika przy\linebreak najwyższej potędze x jest ujemny.\\ W związku z tym wykres wielomianu zaczyna się od lewej strony powyżej osi OX.\\
Ponadto w punkcie $-3$ wykres odbija się od osi poziomej.\\
A więc $$x \in \{-3\} \cup [7,19].$$
\rozwStop
\odpStart
$x \in \{-3\} \cup [7,19]$
\odpStop
\testStart
A.$x \in \{-3\} \cup [7,19]$\\
B.$x \in \{3\} \cup (7,19)$\\
C.$x \in \{-3\} \cup (7,19]$\\
D.$x \in \{3\} \cup (7,19]$\\
E.$x \in \{-3\} \cup [7,19)$\\
F.$x \in \{3\} \cup [7,19)$\\
G.$x \in \{-3\} \cup (7,19)$\\
H.$x \in \{3\} \cup [7,19]$
\testStop
\kluczStart
A
\kluczStop



\zadStart{Zadanie z Wikieł Z 1.62 c) moja wersja nr 269}

Rozwiązać nierówności $(7-x)(x+3)^{2}(20-x)^{3}\le0$.
\zadStop
\rozwStart{Patryk Wirkus}{}
Miejsca zerowe naszego wielomianu to: $7, -3, 20$.\\
Wielomian jest stopnia parzystego, ponadto znak współczynnika przy\linebreak najwyższej potędze x jest ujemny.\\ W związku z tym wykres wielomianu zaczyna się od lewej strony powyżej osi OX.\\
Ponadto w punkcie $-3$ wykres odbija się od osi poziomej.\\
A więc $$x \in \{-3\} \cup [7,20].$$
\rozwStop
\odpStart
$x \in \{-3\} \cup [7,20]$
\odpStop
\testStart
A.$x \in \{-3\} \cup [7,20]$\\
B.$x \in \{3\} \cup (7,20)$\\
C.$x \in \{-3\} \cup (7,20]$\\
D.$x \in \{3\} \cup (7,20]$\\
E.$x \in \{-3\} \cup [7,20)$\\
F.$x \in \{3\} \cup [7,20)$\\
G.$x \in \{-3\} \cup (7,20)$\\
H.$x \in \{3\} \cup [7,20]$
\testStop
\kluczStart
A
\kluczStop



\zadStart{Zadanie z Wikieł Z 1.62 c) moja wersja nr 270}

Rozwiązać nierówności $(7-x)(x+4)^{2}(8-x)^{3}\le0$.
\zadStop
\rozwStart{Patryk Wirkus}{}
Miejsca zerowe naszego wielomianu to: $7, -4, 8$.\\
Wielomian jest stopnia parzystego, ponadto znak współczynnika przy\linebreak najwyższej potędze x jest ujemny.\\ W związku z tym wykres wielomianu zaczyna się od lewej strony powyżej osi OX.\\
Ponadto w punkcie $-4$ wykres odbija się od osi poziomej.\\
A więc $$x \in \{-4\} \cup [7,8].$$
\rozwStop
\odpStart
$x \in \{-4\} \cup [7,8]$
\odpStop
\testStart
A.$x \in \{-4\} \cup [7,8]$\\
B.$x \in \{4\} \cup (7,8)$\\
C.$x \in \{-4\} \cup (7,8]$\\
D.$x \in \{4\} \cup (7,8]$\\
E.$x \in \{-4\} \cup [7,8)$\\
F.$x \in \{4\} \cup [7,8)$\\
G.$x \in \{-4\} \cup (7,8)$\\
H.$x \in \{4\} \cup [7,8]$
\testStop
\kluczStart
A
\kluczStop



\zadStart{Zadanie z Wikieł Z 1.62 c) moja wersja nr 271}

Rozwiązać nierówności $(7-x)(x+4)^{2}(9-x)^{3}\le0$.
\zadStop
\rozwStart{Patryk Wirkus}{}
Miejsca zerowe naszego wielomianu to: $7, -4, 9$.\\
Wielomian jest stopnia parzystego, ponadto znak współczynnika przy\linebreak najwyższej potędze x jest ujemny.\\ W związku z tym wykres wielomianu zaczyna się od lewej strony powyżej osi OX.\\
Ponadto w punkcie $-4$ wykres odbija się od osi poziomej.\\
A więc $$x \in \{-4\} \cup [7,9].$$
\rozwStop
\odpStart
$x \in \{-4\} \cup [7,9]$
\odpStop
\testStart
A.$x \in \{-4\} \cup [7,9]$\\
B.$x \in \{4\} \cup (7,9)$\\
C.$x \in \{-4\} \cup (7,9]$\\
D.$x \in \{4\} \cup (7,9]$\\
E.$x \in \{-4\} \cup [7,9)$\\
F.$x \in \{4\} \cup [7,9)$\\
G.$x \in \{-4\} \cup (7,9)$\\
H.$x \in \{4\} \cup [7,9]$
\testStop
\kluczStart
A
\kluczStop



\zadStart{Zadanie z Wikieł Z 1.62 c) moja wersja nr 272}

Rozwiązać nierówności $(7-x)(x+4)^{2}(10-x)^{3}\le0$.
\zadStop
\rozwStart{Patryk Wirkus}{}
Miejsca zerowe naszego wielomianu to: $7, -4, 10$.\\
Wielomian jest stopnia parzystego, ponadto znak współczynnika przy\linebreak najwyższej potędze x jest ujemny.\\ W związku z tym wykres wielomianu zaczyna się od lewej strony powyżej osi OX.\\
Ponadto w punkcie $-4$ wykres odbija się od osi poziomej.\\
A więc $$x \in \{-4\} \cup [7,10].$$
\rozwStop
\odpStart
$x \in \{-4\} \cup [7,10]$
\odpStop
\testStart
A.$x \in \{-4\} \cup [7,10]$\\
B.$x \in \{4\} \cup (7,10)$\\
C.$x \in \{-4\} \cup (7,10]$\\
D.$x \in \{4\} \cup (7,10]$\\
E.$x \in \{-4\} \cup [7,10)$\\
F.$x \in \{4\} \cup [7,10)$\\
G.$x \in \{-4\} \cup (7,10)$\\
H.$x \in \{4\} \cup [7,10]$
\testStop
\kluczStart
A
\kluczStop



\zadStart{Zadanie z Wikieł Z 1.62 c) moja wersja nr 273}

Rozwiązać nierówności $(7-x)(x+4)^{2}(11-x)^{3}\le0$.
\zadStop
\rozwStart{Patryk Wirkus}{}
Miejsca zerowe naszego wielomianu to: $7, -4, 11$.\\
Wielomian jest stopnia parzystego, ponadto znak współczynnika przy\linebreak najwyższej potędze x jest ujemny.\\ W związku z tym wykres wielomianu zaczyna się od lewej strony powyżej osi OX.\\
Ponadto w punkcie $-4$ wykres odbija się od osi poziomej.\\
A więc $$x \in \{-4\} \cup [7,11].$$
\rozwStop
\odpStart
$x \in \{-4\} \cup [7,11]$
\odpStop
\testStart
A.$x \in \{-4\} \cup [7,11]$\\
B.$x \in \{4\} \cup (7,11)$\\
C.$x \in \{-4\} \cup (7,11]$\\
D.$x \in \{4\} \cup (7,11]$\\
E.$x \in \{-4\} \cup [7,11)$\\
F.$x \in \{4\} \cup [7,11)$\\
G.$x \in \{-4\} \cup (7,11)$\\
H.$x \in \{4\} \cup [7,11]$
\testStop
\kluczStart
A
\kluczStop



\zadStart{Zadanie z Wikieł Z 1.62 c) moja wersja nr 274}

Rozwiązać nierówności $(7-x)(x+4)^{2}(12-x)^{3}\le0$.
\zadStop
\rozwStart{Patryk Wirkus}{}
Miejsca zerowe naszego wielomianu to: $7, -4, 12$.\\
Wielomian jest stopnia parzystego, ponadto znak współczynnika przy\linebreak najwyższej potędze x jest ujemny.\\ W związku z tym wykres wielomianu zaczyna się od lewej strony powyżej osi OX.\\
Ponadto w punkcie $-4$ wykres odbija się od osi poziomej.\\
A więc $$x \in \{-4\} \cup [7,12].$$
\rozwStop
\odpStart
$x \in \{-4\} \cup [7,12]$
\odpStop
\testStart
A.$x \in \{-4\} \cup [7,12]$\\
B.$x \in \{4\} \cup (7,12)$\\
C.$x \in \{-4\} \cup (7,12]$\\
D.$x \in \{4\} \cup (7,12]$\\
E.$x \in \{-4\} \cup [7,12)$\\
F.$x \in \{4\} \cup [7,12)$\\
G.$x \in \{-4\} \cup (7,12)$\\
H.$x \in \{4\} \cup [7,12]$
\testStop
\kluczStart
A
\kluczStop



\zadStart{Zadanie z Wikieł Z 1.62 c) moja wersja nr 275}

Rozwiązać nierówności $(7-x)(x+4)^{2}(13-x)^{3}\le0$.
\zadStop
\rozwStart{Patryk Wirkus}{}
Miejsca zerowe naszego wielomianu to: $7, -4, 13$.\\
Wielomian jest stopnia parzystego, ponadto znak współczynnika przy\linebreak najwyższej potędze x jest ujemny.\\ W związku z tym wykres wielomianu zaczyna się od lewej strony powyżej osi OX.\\
Ponadto w punkcie $-4$ wykres odbija się od osi poziomej.\\
A więc $$x \in \{-4\} \cup [7,13].$$
\rozwStop
\odpStart
$x \in \{-4\} \cup [7,13]$
\odpStop
\testStart
A.$x \in \{-4\} \cup [7,13]$\\
B.$x \in \{4\} \cup (7,13)$\\
C.$x \in \{-4\} \cup (7,13]$\\
D.$x \in \{4\} \cup (7,13]$\\
E.$x \in \{-4\} \cup [7,13)$\\
F.$x \in \{4\} \cup [7,13)$\\
G.$x \in \{-4\} \cup (7,13)$\\
H.$x \in \{4\} \cup [7,13]$
\testStop
\kluczStart
A
\kluczStop



\zadStart{Zadanie z Wikieł Z 1.62 c) moja wersja nr 276}

Rozwiązać nierówności $(7-x)(x+4)^{2}(14-x)^{3}\le0$.
\zadStop
\rozwStart{Patryk Wirkus}{}
Miejsca zerowe naszego wielomianu to: $7, -4, 14$.\\
Wielomian jest stopnia parzystego, ponadto znak współczynnika przy\linebreak najwyższej potędze x jest ujemny.\\ W związku z tym wykres wielomianu zaczyna się od lewej strony powyżej osi OX.\\
Ponadto w punkcie $-4$ wykres odbija się od osi poziomej.\\
A więc $$x \in \{-4\} \cup [7,14].$$
\rozwStop
\odpStart
$x \in \{-4\} \cup [7,14]$
\odpStop
\testStart
A.$x \in \{-4\} \cup [7,14]$\\
B.$x \in \{4\} \cup (7,14)$\\
C.$x \in \{-4\} \cup (7,14]$\\
D.$x \in \{4\} \cup (7,14]$\\
E.$x \in \{-4\} \cup [7,14)$\\
F.$x \in \{4\} \cup [7,14)$\\
G.$x \in \{-4\} \cup (7,14)$\\
H.$x \in \{4\} \cup [7,14]$
\testStop
\kluczStart
A
\kluczStop



\zadStart{Zadanie z Wikieł Z 1.62 c) moja wersja nr 277}

Rozwiązać nierówności $(7-x)(x+4)^{2}(15-x)^{3}\le0$.
\zadStop
\rozwStart{Patryk Wirkus}{}
Miejsca zerowe naszego wielomianu to: $7, -4, 15$.\\
Wielomian jest stopnia parzystego, ponadto znak współczynnika przy\linebreak najwyższej potędze x jest ujemny.\\ W związku z tym wykres wielomianu zaczyna się od lewej strony powyżej osi OX.\\
Ponadto w punkcie $-4$ wykres odbija się od osi poziomej.\\
A więc $$x \in \{-4\} \cup [7,15].$$
\rozwStop
\odpStart
$x \in \{-4\} \cup [7,15]$
\odpStop
\testStart
A.$x \in \{-4\} \cup [7,15]$\\
B.$x \in \{4\} \cup (7,15)$\\
C.$x \in \{-4\} \cup (7,15]$\\
D.$x \in \{4\} \cup (7,15]$\\
E.$x \in \{-4\} \cup [7,15)$\\
F.$x \in \{4\} \cup [7,15)$\\
G.$x \in \{-4\} \cup (7,15)$\\
H.$x \in \{4\} \cup [7,15]$
\testStop
\kluczStart
A
\kluczStop



\zadStart{Zadanie z Wikieł Z 1.62 c) moja wersja nr 278}

Rozwiązać nierówności $(7-x)(x+4)^{2}(16-x)^{3}\le0$.
\zadStop
\rozwStart{Patryk Wirkus}{}
Miejsca zerowe naszego wielomianu to: $7, -4, 16$.\\
Wielomian jest stopnia parzystego, ponadto znak współczynnika przy\linebreak najwyższej potędze x jest ujemny.\\ W związku z tym wykres wielomianu zaczyna się od lewej strony powyżej osi OX.\\
Ponadto w punkcie $-4$ wykres odbija się od osi poziomej.\\
A więc $$x \in \{-4\} \cup [7,16].$$
\rozwStop
\odpStart
$x \in \{-4\} \cup [7,16]$
\odpStop
\testStart
A.$x \in \{-4\} \cup [7,16]$\\
B.$x \in \{4\} \cup (7,16)$\\
C.$x \in \{-4\} \cup (7,16]$\\
D.$x \in \{4\} \cup (7,16]$\\
E.$x \in \{-4\} \cup [7,16)$\\
F.$x \in \{4\} \cup [7,16)$\\
G.$x \in \{-4\} \cup (7,16)$\\
H.$x \in \{4\} \cup [7,16]$
\testStop
\kluczStart
A
\kluczStop



\zadStart{Zadanie z Wikieł Z 1.62 c) moja wersja nr 279}

Rozwiązać nierówności $(7-x)(x+4)^{2}(17-x)^{3}\le0$.
\zadStop
\rozwStart{Patryk Wirkus}{}
Miejsca zerowe naszego wielomianu to: $7, -4, 17$.\\
Wielomian jest stopnia parzystego, ponadto znak współczynnika przy\linebreak najwyższej potędze x jest ujemny.\\ W związku z tym wykres wielomianu zaczyna się od lewej strony powyżej osi OX.\\
Ponadto w punkcie $-4$ wykres odbija się od osi poziomej.\\
A więc $$x \in \{-4\} \cup [7,17].$$
\rozwStop
\odpStart
$x \in \{-4\} \cup [7,17]$
\odpStop
\testStart
A.$x \in \{-4\} \cup [7,17]$\\
B.$x \in \{4\} \cup (7,17)$\\
C.$x \in \{-4\} \cup (7,17]$\\
D.$x \in \{4\} \cup (7,17]$\\
E.$x \in \{-4\} \cup [7,17)$\\
F.$x \in \{4\} \cup [7,17)$\\
G.$x \in \{-4\} \cup (7,17)$\\
H.$x \in \{4\} \cup [7,17]$
\testStop
\kluczStart
A
\kluczStop



\zadStart{Zadanie z Wikieł Z 1.62 c) moja wersja nr 280}

Rozwiązać nierówności $(7-x)(x+4)^{2}(18-x)^{3}\le0$.
\zadStop
\rozwStart{Patryk Wirkus}{}
Miejsca zerowe naszego wielomianu to: $7, -4, 18$.\\
Wielomian jest stopnia parzystego, ponadto znak współczynnika przy\linebreak najwyższej potędze x jest ujemny.\\ W związku z tym wykres wielomianu zaczyna się od lewej strony powyżej osi OX.\\
Ponadto w punkcie $-4$ wykres odbija się od osi poziomej.\\
A więc $$x \in \{-4\} \cup [7,18].$$
\rozwStop
\odpStart
$x \in \{-4\} \cup [7,18]$
\odpStop
\testStart
A.$x \in \{-4\} \cup [7,18]$\\
B.$x \in \{4\} \cup (7,18)$\\
C.$x \in \{-4\} \cup (7,18]$\\
D.$x \in \{4\} \cup (7,18]$\\
E.$x \in \{-4\} \cup [7,18)$\\
F.$x \in \{4\} \cup [7,18)$\\
G.$x \in \{-4\} \cup (7,18)$\\
H.$x \in \{4\} \cup [7,18]$
\testStop
\kluczStart
A
\kluczStop



\zadStart{Zadanie z Wikieł Z 1.62 c) moja wersja nr 281}

Rozwiązać nierówności $(7-x)(x+4)^{2}(19-x)^{3}\le0$.
\zadStop
\rozwStart{Patryk Wirkus}{}
Miejsca zerowe naszego wielomianu to: $7, -4, 19$.\\
Wielomian jest stopnia parzystego, ponadto znak współczynnika przy\linebreak najwyższej potędze x jest ujemny.\\ W związku z tym wykres wielomianu zaczyna się od lewej strony powyżej osi OX.\\
Ponadto w punkcie $-4$ wykres odbija się od osi poziomej.\\
A więc $$x \in \{-4\} \cup [7,19].$$
\rozwStop
\odpStart
$x \in \{-4\} \cup [7,19]$
\odpStop
\testStart
A.$x \in \{-4\} \cup [7,19]$\\
B.$x \in \{4\} \cup (7,19)$\\
C.$x \in \{-4\} \cup (7,19]$\\
D.$x \in \{4\} \cup (7,19]$\\
E.$x \in \{-4\} \cup [7,19)$\\
F.$x \in \{4\} \cup [7,19)$\\
G.$x \in \{-4\} \cup (7,19)$\\
H.$x \in \{4\} \cup [7,19]$
\testStop
\kluczStart
A
\kluczStop



\zadStart{Zadanie z Wikieł Z 1.62 c) moja wersja nr 282}

Rozwiązać nierówności $(7-x)(x+4)^{2}(20-x)^{3}\le0$.
\zadStop
\rozwStart{Patryk Wirkus}{}
Miejsca zerowe naszego wielomianu to: $7, -4, 20$.\\
Wielomian jest stopnia parzystego, ponadto znak współczynnika przy\linebreak najwyższej potędze x jest ujemny.\\ W związku z tym wykres wielomianu zaczyna się od lewej strony powyżej osi OX.\\
Ponadto w punkcie $-4$ wykres odbija się od osi poziomej.\\
A więc $$x \in \{-4\} \cup [7,20].$$
\rozwStop
\odpStart
$x \in \{-4\} \cup [7,20]$
\odpStop
\testStart
A.$x \in \{-4\} \cup [7,20]$\\
B.$x \in \{4\} \cup (7,20)$\\
C.$x \in \{-4\} \cup (7,20]$\\
D.$x \in \{4\} \cup (7,20]$\\
E.$x \in \{-4\} \cup [7,20)$\\
F.$x \in \{4\} \cup [7,20)$\\
G.$x \in \{-4\} \cup (7,20)$\\
H.$x \in \{4\} \cup [7,20]$
\testStop
\kluczStart
A
\kluczStop



\zadStart{Zadanie z Wikieł Z 1.62 c) moja wersja nr 283}

Rozwiązać nierówności $(7-x)(x+5)^{2}(8-x)^{3}\le0$.
\zadStop
\rozwStart{Patryk Wirkus}{}
Miejsca zerowe naszego wielomianu to: $7, -5, 8$.\\
Wielomian jest stopnia parzystego, ponadto znak współczynnika przy\linebreak najwyższej potędze x jest ujemny.\\ W związku z tym wykres wielomianu zaczyna się od lewej strony powyżej osi OX.\\
Ponadto w punkcie $-5$ wykres odbija się od osi poziomej.\\
A więc $$x \in \{-5\} \cup [7,8].$$
\rozwStop
\odpStart
$x \in \{-5\} \cup [7,8]$
\odpStop
\testStart
A.$x \in \{-5\} \cup [7,8]$\\
B.$x \in \{5\} \cup (7,8)$\\
C.$x \in \{-5\} \cup (7,8]$\\
D.$x \in \{5\} \cup (7,8]$\\
E.$x \in \{-5\} \cup [7,8)$\\
F.$x \in \{5\} \cup [7,8)$\\
G.$x \in \{-5\} \cup (7,8)$\\
H.$x \in \{5\} \cup [7,8]$
\testStop
\kluczStart
A
\kluczStop



\zadStart{Zadanie z Wikieł Z 1.62 c) moja wersja nr 284}

Rozwiązać nierówności $(7-x)(x+5)^{2}(9-x)^{3}\le0$.
\zadStop
\rozwStart{Patryk Wirkus}{}
Miejsca zerowe naszego wielomianu to: $7, -5, 9$.\\
Wielomian jest stopnia parzystego, ponadto znak współczynnika przy\linebreak najwyższej potędze x jest ujemny.\\ W związku z tym wykres wielomianu zaczyna się od lewej strony powyżej osi OX.\\
Ponadto w punkcie $-5$ wykres odbija się od osi poziomej.\\
A więc $$x \in \{-5\} \cup [7,9].$$
\rozwStop
\odpStart
$x \in \{-5\} \cup [7,9]$
\odpStop
\testStart
A.$x \in \{-5\} \cup [7,9]$\\
B.$x \in \{5\} \cup (7,9)$\\
C.$x \in \{-5\} \cup (7,9]$\\
D.$x \in \{5\} \cup (7,9]$\\
E.$x \in \{-5\} \cup [7,9)$\\
F.$x \in \{5\} \cup [7,9)$\\
G.$x \in \{-5\} \cup (7,9)$\\
H.$x \in \{5\} \cup [7,9]$
\testStop
\kluczStart
A
\kluczStop



\zadStart{Zadanie z Wikieł Z 1.62 c) moja wersja nr 285}

Rozwiązać nierówności $(7-x)(x+5)^{2}(10-x)^{3}\le0$.
\zadStop
\rozwStart{Patryk Wirkus}{}
Miejsca zerowe naszego wielomianu to: $7, -5, 10$.\\
Wielomian jest stopnia parzystego, ponadto znak współczynnika przy\linebreak najwyższej potędze x jest ujemny.\\ W związku z tym wykres wielomianu zaczyna się od lewej strony powyżej osi OX.\\
Ponadto w punkcie $-5$ wykres odbija się od osi poziomej.\\
A więc $$x \in \{-5\} \cup [7,10].$$
\rozwStop
\odpStart
$x \in \{-5\} \cup [7,10]$
\odpStop
\testStart
A.$x \in \{-5\} \cup [7,10]$\\
B.$x \in \{5\} \cup (7,10)$\\
C.$x \in \{-5\} \cup (7,10]$\\
D.$x \in \{5\} \cup (7,10]$\\
E.$x \in \{-5\} \cup [7,10)$\\
F.$x \in \{5\} \cup [7,10)$\\
G.$x \in \{-5\} \cup (7,10)$\\
H.$x \in \{5\} \cup [7,10]$
\testStop
\kluczStart
A
\kluczStop



\zadStart{Zadanie z Wikieł Z 1.62 c) moja wersja nr 286}

Rozwiązać nierówności $(7-x)(x+5)^{2}(11-x)^{3}\le0$.
\zadStop
\rozwStart{Patryk Wirkus}{}
Miejsca zerowe naszego wielomianu to: $7, -5, 11$.\\
Wielomian jest stopnia parzystego, ponadto znak współczynnika przy\linebreak najwyższej potędze x jest ujemny.\\ W związku z tym wykres wielomianu zaczyna się od lewej strony powyżej osi OX.\\
Ponadto w punkcie $-5$ wykres odbija się od osi poziomej.\\
A więc $$x \in \{-5\} \cup [7,11].$$
\rozwStop
\odpStart
$x \in \{-5\} \cup [7,11]$
\odpStop
\testStart
A.$x \in \{-5\} \cup [7,11]$\\
B.$x \in \{5\} \cup (7,11)$\\
C.$x \in \{-5\} \cup (7,11]$\\
D.$x \in \{5\} \cup (7,11]$\\
E.$x \in \{-5\} \cup [7,11)$\\
F.$x \in \{5\} \cup [7,11)$\\
G.$x \in \{-5\} \cup (7,11)$\\
H.$x \in \{5\} \cup [7,11]$
\testStop
\kluczStart
A
\kluczStop



\zadStart{Zadanie z Wikieł Z 1.62 c) moja wersja nr 287}

Rozwiązać nierówności $(7-x)(x+5)^{2}(12-x)^{3}\le0$.
\zadStop
\rozwStart{Patryk Wirkus}{}
Miejsca zerowe naszego wielomianu to: $7, -5, 12$.\\
Wielomian jest stopnia parzystego, ponadto znak współczynnika przy\linebreak najwyższej potędze x jest ujemny.\\ W związku z tym wykres wielomianu zaczyna się od lewej strony powyżej osi OX.\\
Ponadto w punkcie $-5$ wykres odbija się od osi poziomej.\\
A więc $$x \in \{-5\} \cup [7,12].$$
\rozwStop
\odpStart
$x \in \{-5\} \cup [7,12]$
\odpStop
\testStart
A.$x \in \{-5\} \cup [7,12]$\\
B.$x \in \{5\} \cup (7,12)$\\
C.$x \in \{-5\} \cup (7,12]$\\
D.$x \in \{5\} \cup (7,12]$\\
E.$x \in \{-5\} \cup [7,12)$\\
F.$x \in \{5\} \cup [7,12)$\\
G.$x \in \{-5\} \cup (7,12)$\\
H.$x \in \{5\} \cup [7,12]$
\testStop
\kluczStart
A
\kluczStop



\zadStart{Zadanie z Wikieł Z 1.62 c) moja wersja nr 288}

Rozwiązać nierówności $(7-x)(x+5)^{2}(13-x)^{3}\le0$.
\zadStop
\rozwStart{Patryk Wirkus}{}
Miejsca zerowe naszego wielomianu to: $7, -5, 13$.\\
Wielomian jest stopnia parzystego, ponadto znak współczynnika przy\linebreak najwyższej potędze x jest ujemny.\\ W związku z tym wykres wielomianu zaczyna się od lewej strony powyżej osi OX.\\
Ponadto w punkcie $-5$ wykres odbija się od osi poziomej.\\
A więc $$x \in \{-5\} \cup [7,13].$$
\rozwStop
\odpStart
$x \in \{-5\} \cup [7,13]$
\odpStop
\testStart
A.$x \in \{-5\} \cup [7,13]$\\
B.$x \in \{5\} \cup (7,13)$\\
C.$x \in \{-5\} \cup (7,13]$\\
D.$x \in \{5\} \cup (7,13]$\\
E.$x \in \{-5\} \cup [7,13)$\\
F.$x \in \{5\} \cup [7,13)$\\
G.$x \in \{-5\} \cup (7,13)$\\
H.$x \in \{5\} \cup [7,13]$
\testStop
\kluczStart
A
\kluczStop



\zadStart{Zadanie z Wikieł Z 1.62 c) moja wersja nr 289}

Rozwiązać nierówności $(7-x)(x+5)^{2}(14-x)^{3}\le0$.
\zadStop
\rozwStart{Patryk Wirkus}{}
Miejsca zerowe naszego wielomianu to: $7, -5, 14$.\\
Wielomian jest stopnia parzystego, ponadto znak współczynnika przy\linebreak najwyższej potędze x jest ujemny.\\ W związku z tym wykres wielomianu zaczyna się od lewej strony powyżej osi OX.\\
Ponadto w punkcie $-5$ wykres odbija się od osi poziomej.\\
A więc $$x \in \{-5\} \cup [7,14].$$
\rozwStop
\odpStart
$x \in \{-5\} \cup [7,14]$
\odpStop
\testStart
A.$x \in \{-5\} \cup [7,14]$\\
B.$x \in \{5\} \cup (7,14)$\\
C.$x \in \{-5\} \cup (7,14]$\\
D.$x \in \{5\} \cup (7,14]$\\
E.$x \in \{-5\} \cup [7,14)$\\
F.$x \in \{5\} \cup [7,14)$\\
G.$x \in \{-5\} \cup (7,14)$\\
H.$x \in \{5\} \cup [7,14]$
\testStop
\kluczStart
A
\kluczStop



\zadStart{Zadanie z Wikieł Z 1.62 c) moja wersja nr 290}

Rozwiązać nierówności $(7-x)(x+5)^{2}(15-x)^{3}\le0$.
\zadStop
\rozwStart{Patryk Wirkus}{}
Miejsca zerowe naszego wielomianu to: $7, -5, 15$.\\
Wielomian jest stopnia parzystego, ponadto znak współczynnika przy\linebreak najwyższej potędze x jest ujemny.\\ W związku z tym wykres wielomianu zaczyna się od lewej strony powyżej osi OX.\\
Ponadto w punkcie $-5$ wykres odbija się od osi poziomej.\\
A więc $$x \in \{-5\} \cup [7,15].$$
\rozwStop
\odpStart
$x \in \{-5\} \cup [7,15]$
\odpStop
\testStart
A.$x \in \{-5\} \cup [7,15]$\\
B.$x \in \{5\} \cup (7,15)$\\
C.$x \in \{-5\} \cup (7,15]$\\
D.$x \in \{5\} \cup (7,15]$\\
E.$x \in \{-5\} \cup [7,15)$\\
F.$x \in \{5\} \cup [7,15)$\\
G.$x \in \{-5\} \cup (7,15)$\\
H.$x \in \{5\} \cup [7,15]$
\testStop
\kluczStart
A
\kluczStop



\zadStart{Zadanie z Wikieł Z 1.62 c) moja wersja nr 291}

Rozwiązać nierówności $(7-x)(x+5)^{2}(16-x)^{3}\le0$.
\zadStop
\rozwStart{Patryk Wirkus}{}
Miejsca zerowe naszego wielomianu to: $7, -5, 16$.\\
Wielomian jest stopnia parzystego, ponadto znak współczynnika przy\linebreak najwyższej potędze x jest ujemny.\\ W związku z tym wykres wielomianu zaczyna się od lewej strony powyżej osi OX.\\
Ponadto w punkcie $-5$ wykres odbija się od osi poziomej.\\
A więc $$x \in \{-5\} \cup [7,16].$$
\rozwStop
\odpStart
$x \in \{-5\} \cup [7,16]$
\odpStop
\testStart
A.$x \in \{-5\} \cup [7,16]$\\
B.$x \in \{5\} \cup (7,16)$\\
C.$x \in \{-5\} \cup (7,16]$\\
D.$x \in \{5\} \cup (7,16]$\\
E.$x \in \{-5\} \cup [7,16)$\\
F.$x \in \{5\} \cup [7,16)$\\
G.$x \in \{-5\} \cup (7,16)$\\
H.$x \in \{5\} \cup [7,16]$
\testStop
\kluczStart
A
\kluczStop



\zadStart{Zadanie z Wikieł Z 1.62 c) moja wersja nr 292}

Rozwiązać nierówności $(7-x)(x+5)^{2}(17-x)^{3}\le0$.
\zadStop
\rozwStart{Patryk Wirkus}{}
Miejsca zerowe naszego wielomianu to: $7, -5, 17$.\\
Wielomian jest stopnia parzystego, ponadto znak współczynnika przy\linebreak najwyższej potędze x jest ujemny.\\ W związku z tym wykres wielomianu zaczyna się od lewej strony powyżej osi OX.\\
Ponadto w punkcie $-5$ wykres odbija się od osi poziomej.\\
A więc $$x \in \{-5\} \cup [7,17].$$
\rozwStop
\odpStart
$x \in \{-5\} \cup [7,17]$
\odpStop
\testStart
A.$x \in \{-5\} \cup [7,17]$\\
B.$x \in \{5\} \cup (7,17)$\\
C.$x \in \{-5\} \cup (7,17]$\\
D.$x \in \{5\} \cup (7,17]$\\
E.$x \in \{-5\} \cup [7,17)$\\
F.$x \in \{5\} \cup [7,17)$\\
G.$x \in \{-5\} \cup (7,17)$\\
H.$x \in \{5\} \cup [7,17]$
\testStop
\kluczStart
A
\kluczStop



\zadStart{Zadanie z Wikieł Z 1.62 c) moja wersja nr 293}

Rozwiązać nierówności $(7-x)(x+5)^{2}(18-x)^{3}\le0$.
\zadStop
\rozwStart{Patryk Wirkus}{}
Miejsca zerowe naszego wielomianu to: $7, -5, 18$.\\
Wielomian jest stopnia parzystego, ponadto znak współczynnika przy\linebreak najwyższej potędze x jest ujemny.\\ W związku z tym wykres wielomianu zaczyna się od lewej strony powyżej osi OX.\\
Ponadto w punkcie $-5$ wykres odbija się od osi poziomej.\\
A więc $$x \in \{-5\} \cup [7,18].$$
\rozwStop
\odpStart
$x \in \{-5\} \cup [7,18]$
\odpStop
\testStart
A.$x \in \{-5\} \cup [7,18]$\\
B.$x \in \{5\} \cup (7,18)$\\
C.$x \in \{-5\} \cup (7,18]$\\
D.$x \in \{5\} \cup (7,18]$\\
E.$x \in \{-5\} \cup [7,18)$\\
F.$x \in \{5\} \cup [7,18)$\\
G.$x \in \{-5\} \cup (7,18)$\\
H.$x \in \{5\} \cup [7,18]$
\testStop
\kluczStart
A
\kluczStop



\zadStart{Zadanie z Wikieł Z 1.62 c) moja wersja nr 294}

Rozwiązać nierówności $(7-x)(x+5)^{2}(19-x)^{3}\le0$.
\zadStop
\rozwStart{Patryk Wirkus}{}
Miejsca zerowe naszego wielomianu to: $7, -5, 19$.\\
Wielomian jest stopnia parzystego, ponadto znak współczynnika przy\linebreak najwyższej potędze x jest ujemny.\\ W związku z tym wykres wielomianu zaczyna się od lewej strony powyżej osi OX.\\
Ponadto w punkcie $-5$ wykres odbija się od osi poziomej.\\
A więc $$x \in \{-5\} \cup [7,19].$$
\rozwStop
\odpStart
$x \in \{-5\} \cup [7,19]$
\odpStop
\testStart
A.$x \in \{-5\} \cup [7,19]$\\
B.$x \in \{5\} \cup (7,19)$\\
C.$x \in \{-5\} \cup (7,19]$\\
D.$x \in \{5\} \cup (7,19]$\\
E.$x \in \{-5\} \cup [7,19)$\\
F.$x \in \{5\} \cup [7,19)$\\
G.$x \in \{-5\} \cup (7,19)$\\
H.$x \in \{5\} \cup [7,19]$
\testStop
\kluczStart
A
\kluczStop



\zadStart{Zadanie z Wikieł Z 1.62 c) moja wersja nr 295}

Rozwiązać nierówności $(7-x)(x+5)^{2}(20-x)^{3}\le0$.
\zadStop
\rozwStart{Patryk Wirkus}{}
Miejsca zerowe naszego wielomianu to: $7, -5, 20$.\\
Wielomian jest stopnia parzystego, ponadto znak współczynnika przy\linebreak najwyższej potędze x jest ujemny.\\ W związku z tym wykres wielomianu zaczyna się od lewej strony powyżej osi OX.\\
Ponadto w punkcie $-5$ wykres odbija się od osi poziomej.\\
A więc $$x \in \{-5\} \cup [7,20].$$
\rozwStop
\odpStart
$x \in \{-5\} \cup [7,20]$
\odpStop
\testStart
A.$x \in \{-5\} \cup [7,20]$\\
B.$x \in \{5\} \cup (7,20)$\\
C.$x \in \{-5\} \cup (7,20]$\\
D.$x \in \{5\} \cup (7,20]$\\
E.$x \in \{-5\} \cup [7,20)$\\
F.$x \in \{5\} \cup [7,20)$\\
G.$x \in \{-5\} \cup (7,20)$\\
H.$x \in \{5\} \cup [7,20]$
\testStop
\kluczStart
A
\kluczStop



\zadStart{Zadanie z Wikieł Z 1.62 c) moja wersja nr 296}

Rozwiązać nierówności $(7-x)(x+6)^{2}(8-x)^{3}\le0$.
\zadStop
\rozwStart{Patryk Wirkus}{}
Miejsca zerowe naszego wielomianu to: $7, -6, 8$.\\
Wielomian jest stopnia parzystego, ponadto znak współczynnika przy\linebreak najwyższej potędze x jest ujemny.\\ W związku z tym wykres wielomianu zaczyna się od lewej strony powyżej osi OX.\\
Ponadto w punkcie $-6$ wykres odbija się od osi poziomej.\\
A więc $$x \in \{-6\} \cup [7,8].$$
\rozwStop
\odpStart
$x \in \{-6\} \cup [7,8]$
\odpStop
\testStart
A.$x \in \{-6\} \cup [7,8]$\\
B.$x \in \{6\} \cup (7,8)$\\
C.$x \in \{-6\} \cup (7,8]$\\
D.$x \in \{6\} \cup (7,8]$\\
E.$x \in \{-6\} \cup [7,8)$\\
F.$x \in \{6\} \cup [7,8)$\\
G.$x \in \{-6\} \cup (7,8)$\\
H.$x \in \{6\} \cup [7,8]$
\testStop
\kluczStart
A
\kluczStop



\zadStart{Zadanie z Wikieł Z 1.62 c) moja wersja nr 297}

Rozwiązać nierówności $(7-x)(x+6)^{2}(9-x)^{3}\le0$.
\zadStop
\rozwStart{Patryk Wirkus}{}
Miejsca zerowe naszego wielomianu to: $7, -6, 9$.\\
Wielomian jest stopnia parzystego, ponadto znak współczynnika przy\linebreak najwyższej potędze x jest ujemny.\\ W związku z tym wykres wielomianu zaczyna się od lewej strony powyżej osi OX.\\
Ponadto w punkcie $-6$ wykres odbija się od osi poziomej.\\
A więc $$x \in \{-6\} \cup [7,9].$$
\rozwStop
\odpStart
$x \in \{-6\} \cup [7,9]$
\odpStop
\testStart
A.$x \in \{-6\} \cup [7,9]$\\
B.$x \in \{6\} \cup (7,9)$\\
C.$x \in \{-6\} \cup (7,9]$\\
D.$x \in \{6\} \cup (7,9]$\\
E.$x \in \{-6\} \cup [7,9)$\\
F.$x \in \{6\} \cup [7,9)$\\
G.$x \in \{-6\} \cup (7,9)$\\
H.$x \in \{6\} \cup [7,9]$
\testStop
\kluczStart
A
\kluczStop



\zadStart{Zadanie z Wikieł Z 1.62 c) moja wersja nr 298}

Rozwiązać nierówności $(7-x)(x+6)^{2}(10-x)^{3}\le0$.
\zadStop
\rozwStart{Patryk Wirkus}{}
Miejsca zerowe naszego wielomianu to: $7, -6, 10$.\\
Wielomian jest stopnia parzystego, ponadto znak współczynnika przy\linebreak najwyższej potędze x jest ujemny.\\ W związku z tym wykres wielomianu zaczyna się od lewej strony powyżej osi OX.\\
Ponadto w punkcie $-6$ wykres odbija się od osi poziomej.\\
A więc $$x \in \{-6\} \cup [7,10].$$
\rozwStop
\odpStart
$x \in \{-6\} \cup [7,10]$
\odpStop
\testStart
A.$x \in \{-6\} \cup [7,10]$\\
B.$x \in \{6\} \cup (7,10)$\\
C.$x \in \{-6\} \cup (7,10]$\\
D.$x \in \{6\} \cup (7,10]$\\
E.$x \in \{-6\} \cup [7,10)$\\
F.$x \in \{6\} \cup [7,10)$\\
G.$x \in \{-6\} \cup (7,10)$\\
H.$x \in \{6\} \cup [7,10]$
\testStop
\kluczStart
A
\kluczStop



\zadStart{Zadanie z Wikieł Z 1.62 c) moja wersja nr 299}

Rozwiązać nierówności $(7-x)(x+6)^{2}(11-x)^{3}\le0$.
\zadStop
\rozwStart{Patryk Wirkus}{}
Miejsca zerowe naszego wielomianu to: $7, -6, 11$.\\
Wielomian jest stopnia parzystego, ponadto znak współczynnika przy\linebreak najwyższej potędze x jest ujemny.\\ W związku z tym wykres wielomianu zaczyna się od lewej strony powyżej osi OX.\\
Ponadto w punkcie $-6$ wykres odbija się od osi poziomej.\\
A więc $$x \in \{-6\} \cup [7,11].$$
\rozwStop
\odpStart
$x \in \{-6\} \cup [7,11]$
\odpStop
\testStart
A.$x \in \{-6\} \cup [7,11]$\\
B.$x \in \{6\} \cup (7,11)$\\
C.$x \in \{-6\} \cup (7,11]$\\
D.$x \in \{6\} \cup (7,11]$\\
E.$x \in \{-6\} \cup [7,11)$\\
F.$x \in \{6\} \cup [7,11)$\\
G.$x \in \{-6\} \cup (7,11)$\\
H.$x \in \{6\} \cup [7,11]$
\testStop
\kluczStart
A
\kluczStop



\zadStart{Zadanie z Wikieł Z 1.62 c) moja wersja nr 300}

Rozwiązać nierówności $(7-x)(x+6)^{2}(12-x)^{3}\le0$.
\zadStop
\rozwStart{Patryk Wirkus}{}
Miejsca zerowe naszego wielomianu to: $7, -6, 12$.\\
Wielomian jest stopnia parzystego, ponadto znak współczynnika przy\linebreak najwyższej potędze x jest ujemny.\\ W związku z tym wykres wielomianu zaczyna się od lewej strony powyżej osi OX.\\
Ponadto w punkcie $-6$ wykres odbija się od osi poziomej.\\
A więc $$x \in \{-6\} \cup [7,12].$$
\rozwStop
\odpStart
$x \in \{-6\} \cup [7,12]$
\odpStop
\testStart
A.$x \in \{-6\} \cup [7,12]$\\
B.$x \in \{6\} \cup (7,12)$\\
C.$x \in \{-6\} \cup (7,12]$\\
D.$x \in \{6\} \cup (7,12]$\\
E.$x \in \{-6\} \cup [7,12)$\\
F.$x \in \{6\} \cup [7,12)$\\
G.$x \in \{-6\} \cup (7,12)$\\
H.$x \in \{6\} \cup [7,12]$
\testStop
\kluczStart
A
\kluczStop



\zadStart{Zadanie z Wikieł Z 1.62 c) moja wersja nr 301}

Rozwiązać nierówności $(7-x)(x+6)^{2}(13-x)^{3}\le0$.
\zadStop
\rozwStart{Patryk Wirkus}{}
Miejsca zerowe naszego wielomianu to: $7, -6, 13$.\\
Wielomian jest stopnia parzystego, ponadto znak współczynnika przy\linebreak najwyższej potędze x jest ujemny.\\ W związku z tym wykres wielomianu zaczyna się od lewej strony powyżej osi OX.\\
Ponadto w punkcie $-6$ wykres odbija się od osi poziomej.\\
A więc $$x \in \{-6\} \cup [7,13].$$
\rozwStop
\odpStart
$x \in \{-6\} \cup [7,13]$
\odpStop
\testStart
A.$x \in \{-6\} \cup [7,13]$\\
B.$x \in \{6\} \cup (7,13)$\\
C.$x \in \{-6\} \cup (7,13]$\\
D.$x \in \{6\} \cup (7,13]$\\
E.$x \in \{-6\} \cup [7,13)$\\
F.$x \in \{6\} \cup [7,13)$\\
G.$x \in \{-6\} \cup (7,13)$\\
H.$x \in \{6\} \cup [7,13]$
\testStop
\kluczStart
A
\kluczStop



\zadStart{Zadanie z Wikieł Z 1.62 c) moja wersja nr 302}

Rozwiązać nierówności $(7-x)(x+6)^{2}(14-x)^{3}\le0$.
\zadStop
\rozwStart{Patryk Wirkus}{}
Miejsca zerowe naszego wielomianu to: $7, -6, 14$.\\
Wielomian jest stopnia parzystego, ponadto znak współczynnika przy\linebreak najwyższej potędze x jest ujemny.\\ W związku z tym wykres wielomianu zaczyna się od lewej strony powyżej osi OX.\\
Ponadto w punkcie $-6$ wykres odbija się od osi poziomej.\\
A więc $$x \in \{-6\} \cup [7,14].$$
\rozwStop
\odpStart
$x \in \{-6\} \cup [7,14]$
\odpStop
\testStart
A.$x \in \{-6\} \cup [7,14]$\\
B.$x \in \{6\} \cup (7,14)$\\
C.$x \in \{-6\} \cup (7,14]$\\
D.$x \in \{6\} \cup (7,14]$\\
E.$x \in \{-6\} \cup [7,14)$\\
F.$x \in \{6\} \cup [7,14)$\\
G.$x \in \{-6\} \cup (7,14)$\\
H.$x \in \{6\} \cup [7,14]$
\testStop
\kluczStart
A
\kluczStop



\zadStart{Zadanie z Wikieł Z 1.62 c) moja wersja nr 303}

Rozwiązać nierówności $(7-x)(x+6)^{2}(15-x)^{3}\le0$.
\zadStop
\rozwStart{Patryk Wirkus}{}
Miejsca zerowe naszego wielomianu to: $7, -6, 15$.\\
Wielomian jest stopnia parzystego, ponadto znak współczynnika przy\linebreak najwyższej potędze x jest ujemny.\\ W związku z tym wykres wielomianu zaczyna się od lewej strony powyżej osi OX.\\
Ponadto w punkcie $-6$ wykres odbija się od osi poziomej.\\
A więc $$x \in \{-6\} \cup [7,15].$$
\rozwStop
\odpStart
$x \in \{-6\} \cup [7,15]$
\odpStop
\testStart
A.$x \in \{-6\} \cup [7,15]$\\
B.$x \in \{6\} \cup (7,15)$\\
C.$x \in \{-6\} \cup (7,15]$\\
D.$x \in \{6\} \cup (7,15]$\\
E.$x \in \{-6\} \cup [7,15)$\\
F.$x \in \{6\} \cup [7,15)$\\
G.$x \in \{-6\} \cup (7,15)$\\
H.$x \in \{6\} \cup [7,15]$
\testStop
\kluczStart
A
\kluczStop



\zadStart{Zadanie z Wikieł Z 1.62 c) moja wersja nr 304}

Rozwiązać nierówności $(7-x)(x+6)^{2}(16-x)^{3}\le0$.
\zadStop
\rozwStart{Patryk Wirkus}{}
Miejsca zerowe naszego wielomianu to: $7, -6, 16$.\\
Wielomian jest stopnia parzystego, ponadto znak współczynnika przy\linebreak najwyższej potędze x jest ujemny.\\ W związku z tym wykres wielomianu zaczyna się od lewej strony powyżej osi OX.\\
Ponadto w punkcie $-6$ wykres odbija się od osi poziomej.\\
A więc $$x \in \{-6\} \cup [7,16].$$
\rozwStop
\odpStart
$x \in \{-6\} \cup [7,16]$
\odpStop
\testStart
A.$x \in \{-6\} \cup [7,16]$\\
B.$x \in \{6\} \cup (7,16)$\\
C.$x \in \{-6\} \cup (7,16]$\\
D.$x \in \{6\} \cup (7,16]$\\
E.$x \in \{-6\} \cup [7,16)$\\
F.$x \in \{6\} \cup [7,16)$\\
G.$x \in \{-6\} \cup (7,16)$\\
H.$x \in \{6\} \cup [7,16]$
\testStop
\kluczStart
A
\kluczStop



\zadStart{Zadanie z Wikieł Z 1.62 c) moja wersja nr 305}

Rozwiązać nierówności $(7-x)(x+6)^{2}(17-x)^{3}\le0$.
\zadStop
\rozwStart{Patryk Wirkus}{}
Miejsca zerowe naszego wielomianu to: $7, -6, 17$.\\
Wielomian jest stopnia parzystego, ponadto znak współczynnika przy\linebreak najwyższej potędze x jest ujemny.\\ W związku z tym wykres wielomianu zaczyna się od lewej strony powyżej osi OX.\\
Ponadto w punkcie $-6$ wykres odbija się od osi poziomej.\\
A więc $$x \in \{-6\} \cup [7,17].$$
\rozwStop
\odpStart
$x \in \{-6\} \cup [7,17]$
\odpStop
\testStart
A.$x \in \{-6\} \cup [7,17]$\\
B.$x \in \{6\} \cup (7,17)$\\
C.$x \in \{-6\} \cup (7,17]$\\
D.$x \in \{6\} \cup (7,17]$\\
E.$x \in \{-6\} \cup [7,17)$\\
F.$x \in \{6\} \cup [7,17)$\\
G.$x \in \{-6\} \cup (7,17)$\\
H.$x \in \{6\} \cup [7,17]$
\testStop
\kluczStart
A
\kluczStop



\zadStart{Zadanie z Wikieł Z 1.62 c) moja wersja nr 306}

Rozwiązać nierówności $(7-x)(x+6)^{2}(18-x)^{3}\le0$.
\zadStop
\rozwStart{Patryk Wirkus}{}
Miejsca zerowe naszego wielomianu to: $7, -6, 18$.\\
Wielomian jest stopnia parzystego, ponadto znak współczynnika przy\linebreak najwyższej potędze x jest ujemny.\\ W związku z tym wykres wielomianu zaczyna się od lewej strony powyżej osi OX.\\
Ponadto w punkcie $-6$ wykres odbija się od osi poziomej.\\
A więc $$x \in \{-6\} \cup [7,18].$$
\rozwStop
\odpStart
$x \in \{-6\} \cup [7,18]$
\odpStop
\testStart
A.$x \in \{-6\} \cup [7,18]$\\
B.$x \in \{6\} \cup (7,18)$\\
C.$x \in \{-6\} \cup (7,18]$\\
D.$x \in \{6\} \cup (7,18]$\\
E.$x \in \{-6\} \cup [7,18)$\\
F.$x \in \{6\} \cup [7,18)$\\
G.$x \in \{-6\} \cup (7,18)$\\
H.$x \in \{6\} \cup [7,18]$
\testStop
\kluczStart
A
\kluczStop



\zadStart{Zadanie z Wikieł Z 1.62 c) moja wersja nr 307}

Rozwiązać nierówności $(7-x)(x+6)^{2}(19-x)^{3}\le0$.
\zadStop
\rozwStart{Patryk Wirkus}{}
Miejsca zerowe naszego wielomianu to: $7, -6, 19$.\\
Wielomian jest stopnia parzystego, ponadto znak współczynnika przy\linebreak najwyższej potędze x jest ujemny.\\ W związku z tym wykres wielomianu zaczyna się od lewej strony powyżej osi OX.\\
Ponadto w punkcie $-6$ wykres odbija się od osi poziomej.\\
A więc $$x \in \{-6\} \cup [7,19].$$
\rozwStop
\odpStart
$x \in \{-6\} \cup [7,19]$
\odpStop
\testStart
A.$x \in \{-6\} \cup [7,19]$\\
B.$x \in \{6\} \cup (7,19)$\\
C.$x \in \{-6\} \cup (7,19]$\\
D.$x \in \{6\} \cup (7,19]$\\
E.$x \in \{-6\} \cup [7,19)$\\
F.$x \in \{6\} \cup [7,19)$\\
G.$x \in \{-6\} \cup (7,19)$\\
H.$x \in \{6\} \cup [7,19]$
\testStop
\kluczStart
A
\kluczStop



\zadStart{Zadanie z Wikieł Z 1.62 c) moja wersja nr 308}

Rozwiązać nierówności $(7-x)(x+6)^{2}(20-x)^{3}\le0$.
\zadStop
\rozwStart{Patryk Wirkus}{}
Miejsca zerowe naszego wielomianu to: $7, -6, 20$.\\
Wielomian jest stopnia parzystego, ponadto znak współczynnika przy\linebreak najwyższej potędze x jest ujemny.\\ W związku z tym wykres wielomianu zaczyna się od lewej strony powyżej osi OX.\\
Ponadto w punkcie $-6$ wykres odbija się od osi poziomej.\\
A więc $$x \in \{-6\} \cup [7,20].$$
\rozwStop
\odpStart
$x \in \{-6\} \cup [7,20]$
\odpStop
\testStart
A.$x \in \{-6\} \cup [7,20]$\\
B.$x \in \{6\} \cup (7,20)$\\
C.$x \in \{-6\} \cup (7,20]$\\
D.$x \in \{6\} \cup (7,20]$\\
E.$x \in \{-6\} \cup [7,20)$\\
F.$x \in \{6\} \cup [7,20)$\\
G.$x \in \{-6\} \cup (7,20)$\\
H.$x \in \{6\} \cup [7,20]$
\testStop
\kluczStart
A
\kluczStop



\zadStart{Zadanie z Wikieł Z 1.62 c) moja wersja nr 309}

Rozwiązać nierówności $(8-x)(x+1)^{2}(9-x)^{3}\le0$.
\zadStop
\rozwStart{Patryk Wirkus}{}
Miejsca zerowe naszego wielomianu to: $8, -1, 9$.\\
Wielomian jest stopnia parzystego, ponadto znak współczynnika przy\linebreak najwyższej potędze x jest ujemny.\\ W związku z tym wykres wielomianu zaczyna się od lewej strony powyżej osi OX.\\
Ponadto w punkcie $-1$ wykres odbija się od osi poziomej.\\
A więc $$x \in \{-1\} \cup [8,9].$$
\rozwStop
\odpStart
$x \in \{-1\} \cup [8,9]$
\odpStop
\testStart
A.$x \in \{-1\} \cup [8,9]$\\
B.$x \in \{1\} \cup (8,9)$\\
C.$x \in \{-1\} \cup (8,9]$\\
D.$x \in \{1\} \cup (8,9]$\\
E.$x \in \{-1\} \cup [8,9)$\\
F.$x \in \{1\} \cup [8,9)$\\
G.$x \in \{-1\} \cup (8,9)$\\
H.$x \in \{1\} \cup [8,9]$
\testStop
\kluczStart
A
\kluczStop



\zadStart{Zadanie z Wikieł Z 1.62 c) moja wersja nr 310}

Rozwiązać nierówności $(8-x)(x+1)^{2}(10-x)^{3}\le0$.
\zadStop
\rozwStart{Patryk Wirkus}{}
Miejsca zerowe naszego wielomianu to: $8, -1, 10$.\\
Wielomian jest stopnia parzystego, ponadto znak współczynnika przy\linebreak najwyższej potędze x jest ujemny.\\ W związku z tym wykres wielomianu zaczyna się od lewej strony powyżej osi OX.\\
Ponadto w punkcie $-1$ wykres odbija się od osi poziomej.\\
A więc $$x \in \{-1\} \cup [8,10].$$
\rozwStop
\odpStart
$x \in \{-1\} \cup [8,10]$
\odpStop
\testStart
A.$x \in \{-1\} \cup [8,10]$\\
B.$x \in \{1\} \cup (8,10)$\\
C.$x \in \{-1\} \cup (8,10]$\\
D.$x \in \{1\} \cup (8,10]$\\
E.$x \in \{-1\} \cup [8,10)$\\
F.$x \in \{1\} \cup [8,10)$\\
G.$x \in \{-1\} \cup (8,10)$\\
H.$x \in \{1\} \cup [8,10]$
\testStop
\kluczStart
A
\kluczStop



\zadStart{Zadanie z Wikieł Z 1.62 c) moja wersja nr 311}

Rozwiązać nierówności $(8-x)(x+1)^{2}(11-x)^{3}\le0$.
\zadStop
\rozwStart{Patryk Wirkus}{}
Miejsca zerowe naszego wielomianu to: $8, -1, 11$.\\
Wielomian jest stopnia parzystego, ponadto znak współczynnika przy\linebreak najwyższej potędze x jest ujemny.\\ W związku z tym wykres wielomianu zaczyna się od lewej strony powyżej osi OX.\\
Ponadto w punkcie $-1$ wykres odbija się od osi poziomej.\\
A więc $$x \in \{-1\} \cup [8,11].$$
\rozwStop
\odpStart
$x \in \{-1\} \cup [8,11]$
\odpStop
\testStart
A.$x \in \{-1\} \cup [8,11]$\\
B.$x \in \{1\} \cup (8,11)$\\
C.$x \in \{-1\} \cup (8,11]$\\
D.$x \in \{1\} \cup (8,11]$\\
E.$x \in \{-1\} \cup [8,11)$\\
F.$x \in \{1\} \cup [8,11)$\\
G.$x \in \{-1\} \cup (8,11)$\\
H.$x \in \{1\} \cup [8,11]$
\testStop
\kluczStart
A
\kluczStop



\zadStart{Zadanie z Wikieł Z 1.62 c) moja wersja nr 312}

Rozwiązać nierówności $(8-x)(x+1)^{2}(12-x)^{3}\le0$.
\zadStop
\rozwStart{Patryk Wirkus}{}
Miejsca zerowe naszego wielomianu to: $8, -1, 12$.\\
Wielomian jest stopnia parzystego, ponadto znak współczynnika przy\linebreak najwyższej potędze x jest ujemny.\\ W związku z tym wykres wielomianu zaczyna się od lewej strony powyżej osi OX.\\
Ponadto w punkcie $-1$ wykres odbija się od osi poziomej.\\
A więc $$x \in \{-1\} \cup [8,12].$$
\rozwStop
\odpStart
$x \in \{-1\} \cup [8,12]$
\odpStop
\testStart
A.$x \in \{-1\} \cup [8,12]$\\
B.$x \in \{1\} \cup (8,12)$\\
C.$x \in \{-1\} \cup (8,12]$\\
D.$x \in \{1\} \cup (8,12]$\\
E.$x \in \{-1\} \cup [8,12)$\\
F.$x \in \{1\} \cup [8,12)$\\
G.$x \in \{-1\} \cup (8,12)$\\
H.$x \in \{1\} \cup [8,12]$
\testStop
\kluczStart
A
\kluczStop



\zadStart{Zadanie z Wikieł Z 1.62 c) moja wersja nr 313}

Rozwiązać nierówności $(8-x)(x+1)^{2}(13-x)^{3}\le0$.
\zadStop
\rozwStart{Patryk Wirkus}{}
Miejsca zerowe naszego wielomianu to: $8, -1, 13$.\\
Wielomian jest stopnia parzystego, ponadto znak współczynnika przy\linebreak najwyższej potędze x jest ujemny.\\ W związku z tym wykres wielomianu zaczyna się od lewej strony powyżej osi OX.\\
Ponadto w punkcie $-1$ wykres odbija się od osi poziomej.\\
A więc $$x \in \{-1\} \cup [8,13].$$
\rozwStop
\odpStart
$x \in \{-1\} \cup [8,13]$
\odpStop
\testStart
A.$x \in \{-1\} \cup [8,13]$\\
B.$x \in \{1\} \cup (8,13)$\\
C.$x \in \{-1\} \cup (8,13]$\\
D.$x \in \{1\} \cup (8,13]$\\
E.$x \in \{-1\} \cup [8,13)$\\
F.$x \in \{1\} \cup [8,13)$\\
G.$x \in \{-1\} \cup (8,13)$\\
H.$x \in \{1\} \cup [8,13]$
\testStop
\kluczStart
A
\kluczStop



\zadStart{Zadanie z Wikieł Z 1.62 c) moja wersja nr 314}

Rozwiązać nierówności $(8-x)(x+1)^{2}(14-x)^{3}\le0$.
\zadStop
\rozwStart{Patryk Wirkus}{}
Miejsca zerowe naszego wielomianu to: $8, -1, 14$.\\
Wielomian jest stopnia parzystego, ponadto znak współczynnika przy\linebreak najwyższej potędze x jest ujemny.\\ W związku z tym wykres wielomianu zaczyna się od lewej strony powyżej osi OX.\\
Ponadto w punkcie $-1$ wykres odbija się od osi poziomej.\\
A więc $$x \in \{-1\} \cup [8,14].$$
\rozwStop
\odpStart
$x \in \{-1\} \cup [8,14]$
\odpStop
\testStart
A.$x \in \{-1\} \cup [8,14]$\\
B.$x \in \{1\} \cup (8,14)$\\
C.$x \in \{-1\} \cup (8,14]$\\
D.$x \in \{1\} \cup (8,14]$\\
E.$x \in \{-1\} \cup [8,14)$\\
F.$x \in \{1\} \cup [8,14)$\\
G.$x \in \{-1\} \cup (8,14)$\\
H.$x \in \{1\} \cup [8,14]$
\testStop
\kluczStart
A
\kluczStop



\zadStart{Zadanie z Wikieł Z 1.62 c) moja wersja nr 315}

Rozwiązać nierówności $(8-x)(x+1)^{2}(15-x)^{3}\le0$.
\zadStop
\rozwStart{Patryk Wirkus}{}
Miejsca zerowe naszego wielomianu to: $8, -1, 15$.\\
Wielomian jest stopnia parzystego, ponadto znak współczynnika przy\linebreak najwyższej potędze x jest ujemny.\\ W związku z tym wykres wielomianu zaczyna się od lewej strony powyżej osi OX.\\
Ponadto w punkcie $-1$ wykres odbija się od osi poziomej.\\
A więc $$x \in \{-1\} \cup [8,15].$$
\rozwStop
\odpStart
$x \in \{-1\} \cup [8,15]$
\odpStop
\testStart
A.$x \in \{-1\} \cup [8,15]$\\
B.$x \in \{1\} \cup (8,15)$\\
C.$x \in \{-1\} \cup (8,15]$\\
D.$x \in \{1\} \cup (8,15]$\\
E.$x \in \{-1\} \cup [8,15)$\\
F.$x \in \{1\} \cup [8,15)$\\
G.$x \in \{-1\} \cup (8,15)$\\
H.$x \in \{1\} \cup [8,15]$
\testStop
\kluczStart
A
\kluczStop



\zadStart{Zadanie z Wikieł Z 1.62 c) moja wersja nr 316}

Rozwiązać nierówności $(8-x)(x+1)^{2}(16-x)^{3}\le0$.
\zadStop
\rozwStart{Patryk Wirkus}{}
Miejsca zerowe naszego wielomianu to: $8, -1, 16$.\\
Wielomian jest stopnia parzystego, ponadto znak współczynnika przy\linebreak najwyższej potędze x jest ujemny.\\ W związku z tym wykres wielomianu zaczyna się od lewej strony powyżej osi OX.\\
Ponadto w punkcie $-1$ wykres odbija się od osi poziomej.\\
A więc $$x \in \{-1\} \cup [8,16].$$
\rozwStop
\odpStart
$x \in \{-1\} \cup [8,16]$
\odpStop
\testStart
A.$x \in \{-1\} \cup [8,16]$\\
B.$x \in \{1\} \cup (8,16)$\\
C.$x \in \{-1\} \cup (8,16]$\\
D.$x \in \{1\} \cup (8,16]$\\
E.$x \in \{-1\} \cup [8,16)$\\
F.$x \in \{1\} \cup [8,16)$\\
G.$x \in \{-1\} \cup (8,16)$\\
H.$x \in \{1\} \cup [8,16]$
\testStop
\kluczStart
A
\kluczStop



\zadStart{Zadanie z Wikieł Z 1.62 c) moja wersja nr 317}

Rozwiązać nierówności $(8-x)(x+1)^{2}(17-x)^{3}\le0$.
\zadStop
\rozwStart{Patryk Wirkus}{}
Miejsca zerowe naszego wielomianu to: $8, -1, 17$.\\
Wielomian jest stopnia parzystego, ponadto znak współczynnika przy\linebreak najwyższej potędze x jest ujemny.\\ W związku z tym wykres wielomianu zaczyna się od lewej strony powyżej osi OX.\\
Ponadto w punkcie $-1$ wykres odbija się od osi poziomej.\\
A więc $$x \in \{-1\} \cup [8,17].$$
\rozwStop
\odpStart
$x \in \{-1\} \cup [8,17]$
\odpStop
\testStart
A.$x \in \{-1\} \cup [8,17]$\\
B.$x \in \{1\} \cup (8,17)$\\
C.$x \in \{-1\} \cup (8,17]$\\
D.$x \in \{1\} \cup (8,17]$\\
E.$x \in \{-1\} \cup [8,17)$\\
F.$x \in \{1\} \cup [8,17)$\\
G.$x \in \{-1\} \cup (8,17)$\\
H.$x \in \{1\} \cup [8,17]$
\testStop
\kluczStart
A
\kluczStop



\zadStart{Zadanie z Wikieł Z 1.62 c) moja wersja nr 318}

Rozwiązać nierówności $(8-x)(x+1)^{2}(18-x)^{3}\le0$.
\zadStop
\rozwStart{Patryk Wirkus}{}
Miejsca zerowe naszego wielomianu to: $8, -1, 18$.\\
Wielomian jest stopnia parzystego, ponadto znak współczynnika przy\linebreak najwyższej potędze x jest ujemny.\\ W związku z tym wykres wielomianu zaczyna się od lewej strony powyżej osi OX.\\
Ponadto w punkcie $-1$ wykres odbija się od osi poziomej.\\
A więc $$x \in \{-1\} \cup [8,18].$$
\rozwStop
\odpStart
$x \in \{-1\} \cup [8,18]$
\odpStop
\testStart
A.$x \in \{-1\} \cup [8,18]$\\
B.$x \in \{1\} \cup (8,18)$\\
C.$x \in \{-1\} \cup (8,18]$\\
D.$x \in \{1\} \cup (8,18]$\\
E.$x \in \{-1\} \cup [8,18)$\\
F.$x \in \{1\} \cup [8,18)$\\
G.$x \in \{-1\} \cup (8,18)$\\
H.$x \in \{1\} \cup [8,18]$
\testStop
\kluczStart
A
\kluczStop



\zadStart{Zadanie z Wikieł Z 1.62 c) moja wersja nr 319}

Rozwiązać nierówności $(8-x)(x+1)^{2}(19-x)^{3}\le0$.
\zadStop
\rozwStart{Patryk Wirkus}{}
Miejsca zerowe naszego wielomianu to: $8, -1, 19$.\\
Wielomian jest stopnia parzystego, ponadto znak współczynnika przy\linebreak najwyższej potędze x jest ujemny.\\ W związku z tym wykres wielomianu zaczyna się od lewej strony powyżej osi OX.\\
Ponadto w punkcie $-1$ wykres odbija się od osi poziomej.\\
A więc $$x \in \{-1\} \cup [8,19].$$
\rozwStop
\odpStart
$x \in \{-1\} \cup [8,19]$
\odpStop
\testStart
A.$x \in \{-1\} \cup [8,19]$\\
B.$x \in \{1\} \cup (8,19)$\\
C.$x \in \{-1\} \cup (8,19]$\\
D.$x \in \{1\} \cup (8,19]$\\
E.$x \in \{-1\} \cup [8,19)$\\
F.$x \in \{1\} \cup [8,19)$\\
G.$x \in \{-1\} \cup (8,19)$\\
H.$x \in \{1\} \cup [8,19]$
\testStop
\kluczStart
A
\kluczStop



\zadStart{Zadanie z Wikieł Z 1.62 c) moja wersja nr 320}

Rozwiązać nierówności $(8-x)(x+1)^{2}(20-x)^{3}\le0$.
\zadStop
\rozwStart{Patryk Wirkus}{}
Miejsca zerowe naszego wielomianu to: $8, -1, 20$.\\
Wielomian jest stopnia parzystego, ponadto znak współczynnika przy\linebreak najwyższej potędze x jest ujemny.\\ W związku z tym wykres wielomianu zaczyna się od lewej strony powyżej osi OX.\\
Ponadto w punkcie $-1$ wykres odbija się od osi poziomej.\\
A więc $$x \in \{-1\} \cup [8,20].$$
\rozwStop
\odpStart
$x \in \{-1\} \cup [8,20]$
\odpStop
\testStart
A.$x \in \{-1\} \cup [8,20]$\\
B.$x \in \{1\} \cup (8,20)$\\
C.$x \in \{-1\} \cup (8,20]$\\
D.$x \in \{1\} \cup (8,20]$\\
E.$x \in \{-1\} \cup [8,20)$\\
F.$x \in \{1\} \cup [8,20)$\\
G.$x \in \{-1\} \cup (8,20)$\\
H.$x \in \{1\} \cup [8,20]$
\testStop
\kluczStart
A
\kluczStop



\zadStart{Zadanie z Wikieł Z 1.62 c) moja wersja nr 321}

Rozwiązać nierówności $(8-x)(x+2)^{2}(9-x)^{3}\le0$.
\zadStop
\rozwStart{Patryk Wirkus}{}
Miejsca zerowe naszego wielomianu to: $8, -2, 9$.\\
Wielomian jest stopnia parzystego, ponadto znak współczynnika przy\linebreak najwyższej potędze x jest ujemny.\\ W związku z tym wykres wielomianu zaczyna się od lewej strony powyżej osi OX.\\
Ponadto w punkcie $-2$ wykres odbija się od osi poziomej.\\
A więc $$x \in \{-2\} \cup [8,9].$$
\rozwStop
\odpStart
$x \in \{-2\} \cup [8,9]$
\odpStop
\testStart
A.$x \in \{-2\} \cup [8,9]$\\
B.$x \in \{2\} \cup (8,9)$\\
C.$x \in \{-2\} \cup (8,9]$\\
D.$x \in \{2\} \cup (8,9]$\\
E.$x \in \{-2\} \cup [8,9)$\\
F.$x \in \{2\} \cup [8,9)$\\
G.$x \in \{-2\} \cup (8,9)$\\
H.$x \in \{2\} \cup [8,9]$
\testStop
\kluczStart
A
\kluczStop



\zadStart{Zadanie z Wikieł Z 1.62 c) moja wersja nr 322}

Rozwiązać nierówności $(8-x)(x+2)^{2}(10-x)^{3}\le0$.
\zadStop
\rozwStart{Patryk Wirkus}{}
Miejsca zerowe naszego wielomianu to: $8, -2, 10$.\\
Wielomian jest stopnia parzystego, ponadto znak współczynnika przy\linebreak najwyższej potędze x jest ujemny.\\ W związku z tym wykres wielomianu zaczyna się od lewej strony powyżej osi OX.\\
Ponadto w punkcie $-2$ wykres odbija się od osi poziomej.\\
A więc $$x \in \{-2\} \cup [8,10].$$
\rozwStop
\odpStart
$x \in \{-2\} \cup [8,10]$
\odpStop
\testStart
A.$x \in \{-2\} \cup [8,10]$\\
B.$x \in \{2\} \cup (8,10)$\\
C.$x \in \{-2\} \cup (8,10]$\\
D.$x \in \{2\} \cup (8,10]$\\
E.$x \in \{-2\} \cup [8,10)$\\
F.$x \in \{2\} \cup [8,10)$\\
G.$x \in \{-2\} \cup (8,10)$\\
H.$x \in \{2\} \cup [8,10]$
\testStop
\kluczStart
A
\kluczStop



\zadStart{Zadanie z Wikieł Z 1.62 c) moja wersja nr 323}

Rozwiązać nierówności $(8-x)(x+2)^{2}(11-x)^{3}\le0$.
\zadStop
\rozwStart{Patryk Wirkus}{}
Miejsca zerowe naszego wielomianu to: $8, -2, 11$.\\
Wielomian jest stopnia parzystego, ponadto znak współczynnika przy\linebreak najwyższej potędze x jest ujemny.\\ W związku z tym wykres wielomianu zaczyna się od lewej strony powyżej osi OX.\\
Ponadto w punkcie $-2$ wykres odbija się od osi poziomej.\\
A więc $$x \in \{-2\} \cup [8,11].$$
\rozwStop
\odpStart
$x \in \{-2\} \cup [8,11]$
\odpStop
\testStart
A.$x \in \{-2\} \cup [8,11]$\\
B.$x \in \{2\} \cup (8,11)$\\
C.$x \in \{-2\} \cup (8,11]$\\
D.$x \in \{2\} \cup (8,11]$\\
E.$x \in \{-2\} \cup [8,11)$\\
F.$x \in \{2\} \cup [8,11)$\\
G.$x \in \{-2\} \cup (8,11)$\\
H.$x \in \{2\} \cup [8,11]$
\testStop
\kluczStart
A
\kluczStop



\zadStart{Zadanie z Wikieł Z 1.62 c) moja wersja nr 324}

Rozwiązać nierówności $(8-x)(x+2)^{2}(12-x)^{3}\le0$.
\zadStop
\rozwStart{Patryk Wirkus}{}
Miejsca zerowe naszego wielomianu to: $8, -2, 12$.\\
Wielomian jest stopnia parzystego, ponadto znak współczynnika przy\linebreak najwyższej potędze x jest ujemny.\\ W związku z tym wykres wielomianu zaczyna się od lewej strony powyżej osi OX.\\
Ponadto w punkcie $-2$ wykres odbija się od osi poziomej.\\
A więc $$x \in \{-2\} \cup [8,12].$$
\rozwStop
\odpStart
$x \in \{-2\} \cup [8,12]$
\odpStop
\testStart
A.$x \in \{-2\} \cup [8,12]$\\
B.$x \in \{2\} \cup (8,12)$\\
C.$x \in \{-2\} \cup (8,12]$\\
D.$x \in \{2\} \cup (8,12]$\\
E.$x \in \{-2\} \cup [8,12)$\\
F.$x \in \{2\} \cup [8,12)$\\
G.$x \in \{-2\} \cup (8,12)$\\
H.$x \in \{2\} \cup [8,12]$
\testStop
\kluczStart
A
\kluczStop



\zadStart{Zadanie z Wikieł Z 1.62 c) moja wersja nr 325}

Rozwiązać nierówności $(8-x)(x+2)^{2}(13-x)^{3}\le0$.
\zadStop
\rozwStart{Patryk Wirkus}{}
Miejsca zerowe naszego wielomianu to: $8, -2, 13$.\\
Wielomian jest stopnia parzystego, ponadto znak współczynnika przy\linebreak najwyższej potędze x jest ujemny.\\ W związku z tym wykres wielomianu zaczyna się od lewej strony powyżej osi OX.\\
Ponadto w punkcie $-2$ wykres odbija się od osi poziomej.\\
A więc $$x \in \{-2\} \cup [8,13].$$
\rozwStop
\odpStart
$x \in \{-2\} \cup [8,13]$
\odpStop
\testStart
A.$x \in \{-2\} \cup [8,13]$\\
B.$x \in \{2\} \cup (8,13)$\\
C.$x \in \{-2\} \cup (8,13]$\\
D.$x \in \{2\} \cup (8,13]$\\
E.$x \in \{-2\} \cup [8,13)$\\
F.$x \in \{2\} \cup [8,13)$\\
G.$x \in \{-2\} \cup (8,13)$\\
H.$x \in \{2\} \cup [8,13]$
\testStop
\kluczStart
A
\kluczStop



\zadStart{Zadanie z Wikieł Z 1.62 c) moja wersja nr 326}

Rozwiązać nierówności $(8-x)(x+2)^{2}(14-x)^{3}\le0$.
\zadStop
\rozwStart{Patryk Wirkus}{}
Miejsca zerowe naszego wielomianu to: $8, -2, 14$.\\
Wielomian jest stopnia parzystego, ponadto znak współczynnika przy\linebreak najwyższej potędze x jest ujemny.\\ W związku z tym wykres wielomianu zaczyna się od lewej strony powyżej osi OX.\\
Ponadto w punkcie $-2$ wykres odbija się od osi poziomej.\\
A więc $$x \in \{-2\} \cup [8,14].$$
\rozwStop
\odpStart
$x \in \{-2\} \cup [8,14]$
\odpStop
\testStart
A.$x \in \{-2\} \cup [8,14]$\\
B.$x \in \{2\} \cup (8,14)$\\
C.$x \in \{-2\} \cup (8,14]$\\
D.$x \in \{2\} \cup (8,14]$\\
E.$x \in \{-2\} \cup [8,14)$\\
F.$x \in \{2\} \cup [8,14)$\\
G.$x \in \{-2\} \cup (8,14)$\\
H.$x \in \{2\} \cup [8,14]$
\testStop
\kluczStart
A
\kluczStop



\zadStart{Zadanie z Wikieł Z 1.62 c) moja wersja nr 327}

Rozwiązać nierówności $(8-x)(x+2)^{2}(15-x)^{3}\le0$.
\zadStop
\rozwStart{Patryk Wirkus}{}
Miejsca zerowe naszego wielomianu to: $8, -2, 15$.\\
Wielomian jest stopnia parzystego, ponadto znak współczynnika przy\linebreak najwyższej potędze x jest ujemny.\\ W związku z tym wykres wielomianu zaczyna się od lewej strony powyżej osi OX.\\
Ponadto w punkcie $-2$ wykres odbija się od osi poziomej.\\
A więc $$x \in \{-2\} \cup [8,15].$$
\rozwStop
\odpStart
$x \in \{-2\} \cup [8,15]$
\odpStop
\testStart
A.$x \in \{-2\} \cup [8,15]$\\
B.$x \in \{2\} \cup (8,15)$\\
C.$x \in \{-2\} \cup (8,15]$\\
D.$x \in \{2\} \cup (8,15]$\\
E.$x \in \{-2\} \cup [8,15)$\\
F.$x \in \{2\} \cup [8,15)$\\
G.$x \in \{-2\} \cup (8,15)$\\
H.$x \in \{2\} \cup [8,15]$
\testStop
\kluczStart
A
\kluczStop



\zadStart{Zadanie z Wikieł Z 1.62 c) moja wersja nr 328}

Rozwiązać nierówności $(8-x)(x+2)^{2}(16-x)^{3}\le0$.
\zadStop
\rozwStart{Patryk Wirkus}{}
Miejsca zerowe naszego wielomianu to: $8, -2, 16$.\\
Wielomian jest stopnia parzystego, ponadto znak współczynnika przy\linebreak najwyższej potędze x jest ujemny.\\ W związku z tym wykres wielomianu zaczyna się od lewej strony powyżej osi OX.\\
Ponadto w punkcie $-2$ wykres odbija się od osi poziomej.\\
A więc $$x \in \{-2\} \cup [8,16].$$
\rozwStop
\odpStart
$x \in \{-2\} \cup [8,16]$
\odpStop
\testStart
A.$x \in \{-2\} \cup [8,16]$\\
B.$x \in \{2\} \cup (8,16)$\\
C.$x \in \{-2\} \cup (8,16]$\\
D.$x \in \{2\} \cup (8,16]$\\
E.$x \in \{-2\} \cup [8,16)$\\
F.$x \in \{2\} \cup [8,16)$\\
G.$x \in \{-2\} \cup (8,16)$\\
H.$x \in \{2\} \cup [8,16]$
\testStop
\kluczStart
A
\kluczStop



\zadStart{Zadanie z Wikieł Z 1.62 c) moja wersja nr 329}

Rozwiązać nierówności $(8-x)(x+2)^{2}(17-x)^{3}\le0$.
\zadStop
\rozwStart{Patryk Wirkus}{}
Miejsca zerowe naszego wielomianu to: $8, -2, 17$.\\
Wielomian jest stopnia parzystego, ponadto znak współczynnika przy\linebreak najwyższej potędze x jest ujemny.\\ W związku z tym wykres wielomianu zaczyna się od lewej strony powyżej osi OX.\\
Ponadto w punkcie $-2$ wykres odbija się od osi poziomej.\\
A więc $$x \in \{-2\} \cup [8,17].$$
\rozwStop
\odpStart
$x \in \{-2\} \cup [8,17]$
\odpStop
\testStart
A.$x \in \{-2\} \cup [8,17]$\\
B.$x \in \{2\} \cup (8,17)$\\
C.$x \in \{-2\} \cup (8,17]$\\
D.$x \in \{2\} \cup (8,17]$\\
E.$x \in \{-2\} \cup [8,17)$\\
F.$x \in \{2\} \cup [8,17)$\\
G.$x \in \{-2\} \cup (8,17)$\\
H.$x \in \{2\} \cup [8,17]$
\testStop
\kluczStart
A
\kluczStop



\zadStart{Zadanie z Wikieł Z 1.62 c) moja wersja nr 330}

Rozwiązać nierówności $(8-x)(x+2)^{2}(18-x)^{3}\le0$.
\zadStop
\rozwStart{Patryk Wirkus}{}
Miejsca zerowe naszego wielomianu to: $8, -2, 18$.\\
Wielomian jest stopnia parzystego, ponadto znak współczynnika przy\linebreak najwyższej potędze x jest ujemny.\\ W związku z tym wykres wielomianu zaczyna się od lewej strony powyżej osi OX.\\
Ponadto w punkcie $-2$ wykres odbija się od osi poziomej.\\
A więc $$x \in \{-2\} \cup [8,18].$$
\rozwStop
\odpStart
$x \in \{-2\} \cup [8,18]$
\odpStop
\testStart
A.$x \in \{-2\} \cup [8,18]$\\
B.$x \in \{2\} \cup (8,18)$\\
C.$x \in \{-2\} \cup (8,18]$\\
D.$x \in \{2\} \cup (8,18]$\\
E.$x \in \{-2\} \cup [8,18)$\\
F.$x \in \{2\} \cup [8,18)$\\
G.$x \in \{-2\} \cup (8,18)$\\
H.$x \in \{2\} \cup [8,18]$
\testStop
\kluczStart
A
\kluczStop



\zadStart{Zadanie z Wikieł Z 1.62 c) moja wersja nr 331}

Rozwiązać nierówności $(8-x)(x+2)^{2}(19-x)^{3}\le0$.
\zadStop
\rozwStart{Patryk Wirkus}{}
Miejsca zerowe naszego wielomianu to: $8, -2, 19$.\\
Wielomian jest stopnia parzystego, ponadto znak współczynnika przy\linebreak najwyższej potędze x jest ujemny.\\ W związku z tym wykres wielomianu zaczyna się od lewej strony powyżej osi OX.\\
Ponadto w punkcie $-2$ wykres odbija się od osi poziomej.\\
A więc $$x \in \{-2\} \cup [8,19].$$
\rozwStop
\odpStart
$x \in \{-2\} \cup [8,19]$
\odpStop
\testStart
A.$x \in \{-2\} \cup [8,19]$\\
B.$x \in \{2\} \cup (8,19)$\\
C.$x \in \{-2\} \cup (8,19]$\\
D.$x \in \{2\} \cup (8,19]$\\
E.$x \in \{-2\} \cup [8,19)$\\
F.$x \in \{2\} \cup [8,19)$\\
G.$x \in \{-2\} \cup (8,19)$\\
H.$x \in \{2\} \cup [8,19]$
\testStop
\kluczStart
A
\kluczStop



\zadStart{Zadanie z Wikieł Z 1.62 c) moja wersja nr 332}

Rozwiązać nierówności $(8-x)(x+2)^{2}(20-x)^{3}\le0$.
\zadStop
\rozwStart{Patryk Wirkus}{}
Miejsca zerowe naszego wielomianu to: $8, -2, 20$.\\
Wielomian jest stopnia parzystego, ponadto znak współczynnika przy\linebreak najwyższej potędze x jest ujemny.\\ W związku z tym wykres wielomianu zaczyna się od lewej strony powyżej osi OX.\\
Ponadto w punkcie $-2$ wykres odbija się od osi poziomej.\\
A więc $$x \in \{-2\} \cup [8,20].$$
\rozwStop
\odpStart
$x \in \{-2\} \cup [8,20]$
\odpStop
\testStart
A.$x \in \{-2\} \cup [8,20]$\\
B.$x \in \{2\} \cup (8,20)$\\
C.$x \in \{-2\} \cup (8,20]$\\
D.$x \in \{2\} \cup (8,20]$\\
E.$x \in \{-2\} \cup [8,20)$\\
F.$x \in \{2\} \cup [8,20)$\\
G.$x \in \{-2\} \cup (8,20)$\\
H.$x \in \{2\} \cup [8,20]$
\testStop
\kluczStart
A
\kluczStop



\zadStart{Zadanie z Wikieł Z 1.62 c) moja wersja nr 333}

Rozwiązać nierówności $(8-x)(x+3)^{2}(9-x)^{3}\le0$.
\zadStop
\rozwStart{Patryk Wirkus}{}
Miejsca zerowe naszego wielomianu to: $8, -3, 9$.\\
Wielomian jest stopnia parzystego, ponadto znak współczynnika przy\linebreak najwyższej potędze x jest ujemny.\\ W związku z tym wykres wielomianu zaczyna się od lewej strony powyżej osi OX.\\
Ponadto w punkcie $-3$ wykres odbija się od osi poziomej.\\
A więc $$x \in \{-3\} \cup [8,9].$$
\rozwStop
\odpStart
$x \in \{-3\} \cup [8,9]$
\odpStop
\testStart
A.$x \in \{-3\} \cup [8,9]$\\
B.$x \in \{3\} \cup (8,9)$\\
C.$x \in \{-3\} \cup (8,9]$\\
D.$x \in \{3\} \cup (8,9]$\\
E.$x \in \{-3\} \cup [8,9)$\\
F.$x \in \{3\} \cup [8,9)$\\
G.$x \in \{-3\} \cup (8,9)$\\
H.$x \in \{3\} \cup [8,9]$
\testStop
\kluczStart
A
\kluczStop



\zadStart{Zadanie z Wikieł Z 1.62 c) moja wersja nr 334}

Rozwiązać nierówności $(8-x)(x+3)^{2}(10-x)^{3}\le0$.
\zadStop
\rozwStart{Patryk Wirkus}{}
Miejsca zerowe naszego wielomianu to: $8, -3, 10$.\\
Wielomian jest stopnia parzystego, ponadto znak współczynnika przy\linebreak najwyższej potędze x jest ujemny.\\ W związku z tym wykres wielomianu zaczyna się od lewej strony powyżej osi OX.\\
Ponadto w punkcie $-3$ wykres odbija się od osi poziomej.\\
A więc $$x \in \{-3\} \cup [8,10].$$
\rozwStop
\odpStart
$x \in \{-3\} \cup [8,10]$
\odpStop
\testStart
A.$x \in \{-3\} \cup [8,10]$\\
B.$x \in \{3\} \cup (8,10)$\\
C.$x \in \{-3\} \cup (8,10]$\\
D.$x \in \{3\} \cup (8,10]$\\
E.$x \in \{-3\} \cup [8,10)$\\
F.$x \in \{3\} \cup [8,10)$\\
G.$x \in \{-3\} \cup (8,10)$\\
H.$x \in \{3\} \cup [8,10]$
\testStop
\kluczStart
A
\kluczStop



\zadStart{Zadanie z Wikieł Z 1.62 c) moja wersja nr 335}

Rozwiązać nierówności $(8-x)(x+3)^{2}(11-x)^{3}\le0$.
\zadStop
\rozwStart{Patryk Wirkus}{}
Miejsca zerowe naszego wielomianu to: $8, -3, 11$.\\
Wielomian jest stopnia parzystego, ponadto znak współczynnika przy\linebreak najwyższej potędze x jest ujemny.\\ W związku z tym wykres wielomianu zaczyna się od lewej strony powyżej osi OX.\\
Ponadto w punkcie $-3$ wykres odbija się od osi poziomej.\\
A więc $$x \in \{-3\} \cup [8,11].$$
\rozwStop
\odpStart
$x \in \{-3\} \cup [8,11]$
\odpStop
\testStart
A.$x \in \{-3\} \cup [8,11]$\\
B.$x \in \{3\} \cup (8,11)$\\
C.$x \in \{-3\} \cup (8,11]$\\
D.$x \in \{3\} \cup (8,11]$\\
E.$x \in \{-3\} \cup [8,11)$\\
F.$x \in \{3\} \cup [8,11)$\\
G.$x \in \{-3\} \cup (8,11)$\\
H.$x \in \{3\} \cup [8,11]$
\testStop
\kluczStart
A
\kluczStop



\zadStart{Zadanie z Wikieł Z 1.62 c) moja wersja nr 336}

Rozwiązać nierówności $(8-x)(x+3)^{2}(12-x)^{3}\le0$.
\zadStop
\rozwStart{Patryk Wirkus}{}
Miejsca zerowe naszego wielomianu to: $8, -3, 12$.\\
Wielomian jest stopnia parzystego, ponadto znak współczynnika przy\linebreak najwyższej potędze x jest ujemny.\\ W związku z tym wykres wielomianu zaczyna się od lewej strony powyżej osi OX.\\
Ponadto w punkcie $-3$ wykres odbija się od osi poziomej.\\
A więc $$x \in \{-3\} \cup [8,12].$$
\rozwStop
\odpStart
$x \in \{-3\} \cup [8,12]$
\odpStop
\testStart
A.$x \in \{-3\} \cup [8,12]$\\
B.$x \in \{3\} \cup (8,12)$\\
C.$x \in \{-3\} \cup (8,12]$\\
D.$x \in \{3\} \cup (8,12]$\\
E.$x \in \{-3\} \cup [8,12)$\\
F.$x \in \{3\} \cup [8,12)$\\
G.$x \in \{-3\} \cup (8,12)$\\
H.$x \in \{3\} \cup [8,12]$
\testStop
\kluczStart
A
\kluczStop



\zadStart{Zadanie z Wikieł Z 1.62 c) moja wersja nr 337}

Rozwiązać nierówności $(8-x)(x+3)^{2}(13-x)^{3}\le0$.
\zadStop
\rozwStart{Patryk Wirkus}{}
Miejsca zerowe naszego wielomianu to: $8, -3, 13$.\\
Wielomian jest stopnia parzystego, ponadto znak współczynnika przy\linebreak najwyższej potędze x jest ujemny.\\ W związku z tym wykres wielomianu zaczyna się od lewej strony powyżej osi OX.\\
Ponadto w punkcie $-3$ wykres odbija się od osi poziomej.\\
A więc $$x \in \{-3\} \cup [8,13].$$
\rozwStop
\odpStart
$x \in \{-3\} \cup [8,13]$
\odpStop
\testStart
A.$x \in \{-3\} \cup [8,13]$\\
B.$x \in \{3\} \cup (8,13)$\\
C.$x \in \{-3\} \cup (8,13]$\\
D.$x \in \{3\} \cup (8,13]$\\
E.$x \in \{-3\} \cup [8,13)$\\
F.$x \in \{3\} \cup [8,13)$\\
G.$x \in \{-3\} \cup (8,13)$\\
H.$x \in \{3\} \cup [8,13]$
\testStop
\kluczStart
A
\kluczStop



\zadStart{Zadanie z Wikieł Z 1.62 c) moja wersja nr 338}

Rozwiązać nierówności $(8-x)(x+3)^{2}(14-x)^{3}\le0$.
\zadStop
\rozwStart{Patryk Wirkus}{}
Miejsca zerowe naszego wielomianu to: $8, -3, 14$.\\
Wielomian jest stopnia parzystego, ponadto znak współczynnika przy\linebreak najwyższej potędze x jest ujemny.\\ W związku z tym wykres wielomianu zaczyna się od lewej strony powyżej osi OX.\\
Ponadto w punkcie $-3$ wykres odbija się od osi poziomej.\\
A więc $$x \in \{-3\} \cup [8,14].$$
\rozwStop
\odpStart
$x \in \{-3\} \cup [8,14]$
\odpStop
\testStart
A.$x \in \{-3\} \cup [8,14]$\\
B.$x \in \{3\} \cup (8,14)$\\
C.$x \in \{-3\} \cup (8,14]$\\
D.$x \in \{3\} \cup (8,14]$\\
E.$x \in \{-3\} \cup [8,14)$\\
F.$x \in \{3\} \cup [8,14)$\\
G.$x \in \{-3\} \cup (8,14)$\\
H.$x \in \{3\} \cup [8,14]$
\testStop
\kluczStart
A
\kluczStop



\zadStart{Zadanie z Wikieł Z 1.62 c) moja wersja nr 339}

Rozwiązać nierówności $(8-x)(x+3)^{2}(15-x)^{3}\le0$.
\zadStop
\rozwStart{Patryk Wirkus}{}
Miejsca zerowe naszego wielomianu to: $8, -3, 15$.\\
Wielomian jest stopnia parzystego, ponadto znak współczynnika przy\linebreak najwyższej potędze x jest ujemny.\\ W związku z tym wykres wielomianu zaczyna się od lewej strony powyżej osi OX.\\
Ponadto w punkcie $-3$ wykres odbija się od osi poziomej.\\
A więc $$x \in \{-3\} \cup [8,15].$$
\rozwStop
\odpStart
$x \in \{-3\} \cup [8,15]$
\odpStop
\testStart
A.$x \in \{-3\} \cup [8,15]$\\
B.$x \in \{3\} \cup (8,15)$\\
C.$x \in \{-3\} \cup (8,15]$\\
D.$x \in \{3\} \cup (8,15]$\\
E.$x \in \{-3\} \cup [8,15)$\\
F.$x \in \{3\} \cup [8,15)$\\
G.$x \in \{-3\} \cup (8,15)$\\
H.$x \in \{3\} \cup [8,15]$
\testStop
\kluczStart
A
\kluczStop



\zadStart{Zadanie z Wikieł Z 1.62 c) moja wersja nr 340}

Rozwiązać nierówności $(8-x)(x+3)^{2}(16-x)^{3}\le0$.
\zadStop
\rozwStart{Patryk Wirkus}{}
Miejsca zerowe naszego wielomianu to: $8, -3, 16$.\\
Wielomian jest stopnia parzystego, ponadto znak współczynnika przy\linebreak najwyższej potędze x jest ujemny.\\ W związku z tym wykres wielomianu zaczyna się od lewej strony powyżej osi OX.\\
Ponadto w punkcie $-3$ wykres odbija się od osi poziomej.\\
A więc $$x \in \{-3\} \cup [8,16].$$
\rozwStop
\odpStart
$x \in \{-3\} \cup [8,16]$
\odpStop
\testStart
A.$x \in \{-3\} \cup [8,16]$\\
B.$x \in \{3\} \cup (8,16)$\\
C.$x \in \{-3\} \cup (8,16]$\\
D.$x \in \{3\} \cup (8,16]$\\
E.$x \in \{-3\} \cup [8,16)$\\
F.$x \in \{3\} \cup [8,16)$\\
G.$x \in \{-3\} \cup (8,16)$\\
H.$x \in \{3\} \cup [8,16]$
\testStop
\kluczStart
A
\kluczStop



\zadStart{Zadanie z Wikieł Z 1.62 c) moja wersja nr 341}

Rozwiązać nierówności $(8-x)(x+3)^{2}(17-x)^{3}\le0$.
\zadStop
\rozwStart{Patryk Wirkus}{}
Miejsca zerowe naszego wielomianu to: $8, -3, 17$.\\
Wielomian jest stopnia parzystego, ponadto znak współczynnika przy\linebreak najwyższej potędze x jest ujemny.\\ W związku z tym wykres wielomianu zaczyna się od lewej strony powyżej osi OX.\\
Ponadto w punkcie $-3$ wykres odbija się od osi poziomej.\\
A więc $$x \in \{-3\} \cup [8,17].$$
\rozwStop
\odpStart
$x \in \{-3\} \cup [8,17]$
\odpStop
\testStart
A.$x \in \{-3\} \cup [8,17]$\\
B.$x \in \{3\} \cup (8,17)$\\
C.$x \in \{-3\} \cup (8,17]$\\
D.$x \in \{3\} \cup (8,17]$\\
E.$x \in \{-3\} \cup [8,17)$\\
F.$x \in \{3\} \cup [8,17)$\\
G.$x \in \{-3\} \cup (8,17)$\\
H.$x \in \{3\} \cup [8,17]$
\testStop
\kluczStart
A
\kluczStop



\zadStart{Zadanie z Wikieł Z 1.62 c) moja wersja nr 342}

Rozwiązać nierówności $(8-x)(x+3)^{2}(18-x)^{3}\le0$.
\zadStop
\rozwStart{Patryk Wirkus}{}
Miejsca zerowe naszego wielomianu to: $8, -3, 18$.\\
Wielomian jest stopnia parzystego, ponadto znak współczynnika przy\linebreak najwyższej potędze x jest ujemny.\\ W związku z tym wykres wielomianu zaczyna się od lewej strony powyżej osi OX.\\
Ponadto w punkcie $-3$ wykres odbija się od osi poziomej.\\
A więc $$x \in \{-3\} \cup [8,18].$$
\rozwStop
\odpStart
$x \in \{-3\} \cup [8,18]$
\odpStop
\testStart
A.$x \in \{-3\} \cup [8,18]$\\
B.$x \in \{3\} \cup (8,18)$\\
C.$x \in \{-3\} \cup (8,18]$\\
D.$x \in \{3\} \cup (8,18]$\\
E.$x \in \{-3\} \cup [8,18)$\\
F.$x \in \{3\} \cup [8,18)$\\
G.$x \in \{-3\} \cup (8,18)$\\
H.$x \in \{3\} \cup [8,18]$
\testStop
\kluczStart
A
\kluczStop



\zadStart{Zadanie z Wikieł Z 1.62 c) moja wersja nr 343}

Rozwiązać nierówności $(8-x)(x+3)^{2}(19-x)^{3}\le0$.
\zadStop
\rozwStart{Patryk Wirkus}{}
Miejsca zerowe naszego wielomianu to: $8, -3, 19$.\\
Wielomian jest stopnia parzystego, ponadto znak współczynnika przy\linebreak najwyższej potędze x jest ujemny.\\ W związku z tym wykres wielomianu zaczyna się od lewej strony powyżej osi OX.\\
Ponadto w punkcie $-3$ wykres odbija się od osi poziomej.\\
A więc $$x \in \{-3\} \cup [8,19].$$
\rozwStop
\odpStart
$x \in \{-3\} \cup [8,19]$
\odpStop
\testStart
A.$x \in \{-3\} \cup [8,19]$\\
B.$x \in \{3\} \cup (8,19)$\\
C.$x \in \{-3\} \cup (8,19]$\\
D.$x \in \{3\} \cup (8,19]$\\
E.$x \in \{-3\} \cup [8,19)$\\
F.$x \in \{3\} \cup [8,19)$\\
G.$x \in \{-3\} \cup (8,19)$\\
H.$x \in \{3\} \cup [8,19]$
\testStop
\kluczStart
A
\kluczStop



\zadStart{Zadanie z Wikieł Z 1.62 c) moja wersja nr 344}

Rozwiązać nierówności $(8-x)(x+3)^{2}(20-x)^{3}\le0$.
\zadStop
\rozwStart{Patryk Wirkus}{}
Miejsca zerowe naszego wielomianu to: $8, -3, 20$.\\
Wielomian jest stopnia parzystego, ponadto znak współczynnika przy\linebreak najwyższej potędze x jest ujemny.\\ W związku z tym wykres wielomianu zaczyna się od lewej strony powyżej osi OX.\\
Ponadto w punkcie $-3$ wykres odbija się od osi poziomej.\\
A więc $$x \in \{-3\} \cup [8,20].$$
\rozwStop
\odpStart
$x \in \{-3\} \cup [8,20]$
\odpStop
\testStart
A.$x \in \{-3\} \cup [8,20]$\\
B.$x \in \{3\} \cup (8,20)$\\
C.$x \in \{-3\} \cup (8,20]$\\
D.$x \in \{3\} \cup (8,20]$\\
E.$x \in \{-3\} \cup [8,20)$\\
F.$x \in \{3\} \cup [8,20)$\\
G.$x \in \{-3\} \cup (8,20)$\\
H.$x \in \{3\} \cup [8,20]$
\testStop
\kluczStart
A
\kluczStop



\zadStart{Zadanie z Wikieł Z 1.62 c) moja wersja nr 345}

Rozwiązać nierówności $(8-x)(x+4)^{2}(9-x)^{3}\le0$.
\zadStop
\rozwStart{Patryk Wirkus}{}
Miejsca zerowe naszego wielomianu to: $8, -4, 9$.\\
Wielomian jest stopnia parzystego, ponadto znak współczynnika przy\linebreak najwyższej potędze x jest ujemny.\\ W związku z tym wykres wielomianu zaczyna się od lewej strony powyżej osi OX.\\
Ponadto w punkcie $-4$ wykres odbija się od osi poziomej.\\
A więc $$x \in \{-4\} \cup [8,9].$$
\rozwStop
\odpStart
$x \in \{-4\} \cup [8,9]$
\odpStop
\testStart
A.$x \in \{-4\} \cup [8,9]$\\
B.$x \in \{4\} \cup (8,9)$\\
C.$x \in \{-4\} \cup (8,9]$\\
D.$x \in \{4\} \cup (8,9]$\\
E.$x \in \{-4\} \cup [8,9)$\\
F.$x \in \{4\} \cup [8,9)$\\
G.$x \in \{-4\} \cup (8,9)$\\
H.$x \in \{4\} \cup [8,9]$
\testStop
\kluczStart
A
\kluczStop



\zadStart{Zadanie z Wikieł Z 1.62 c) moja wersja nr 346}

Rozwiązać nierówności $(8-x)(x+4)^{2}(10-x)^{3}\le0$.
\zadStop
\rozwStart{Patryk Wirkus}{}
Miejsca zerowe naszego wielomianu to: $8, -4, 10$.\\
Wielomian jest stopnia parzystego, ponadto znak współczynnika przy\linebreak najwyższej potędze x jest ujemny.\\ W związku z tym wykres wielomianu zaczyna się od lewej strony powyżej osi OX.\\
Ponadto w punkcie $-4$ wykres odbija się od osi poziomej.\\
A więc $$x \in \{-4\} \cup [8,10].$$
\rozwStop
\odpStart
$x \in \{-4\} \cup [8,10]$
\odpStop
\testStart
A.$x \in \{-4\} \cup [8,10]$\\
B.$x \in \{4\} \cup (8,10)$\\
C.$x \in \{-4\} \cup (8,10]$\\
D.$x \in \{4\} \cup (8,10]$\\
E.$x \in \{-4\} \cup [8,10)$\\
F.$x \in \{4\} \cup [8,10)$\\
G.$x \in \{-4\} \cup (8,10)$\\
H.$x \in \{4\} \cup [8,10]$
\testStop
\kluczStart
A
\kluczStop



\zadStart{Zadanie z Wikieł Z 1.62 c) moja wersja nr 347}

Rozwiązać nierówności $(8-x)(x+4)^{2}(11-x)^{3}\le0$.
\zadStop
\rozwStart{Patryk Wirkus}{}
Miejsca zerowe naszego wielomianu to: $8, -4, 11$.\\
Wielomian jest stopnia parzystego, ponadto znak współczynnika przy\linebreak najwyższej potędze x jest ujemny.\\ W związku z tym wykres wielomianu zaczyna się od lewej strony powyżej osi OX.\\
Ponadto w punkcie $-4$ wykres odbija się od osi poziomej.\\
A więc $$x \in \{-4\} \cup [8,11].$$
\rozwStop
\odpStart
$x \in \{-4\} \cup [8,11]$
\odpStop
\testStart
A.$x \in \{-4\} \cup [8,11]$\\
B.$x \in \{4\} \cup (8,11)$\\
C.$x \in \{-4\} \cup (8,11]$\\
D.$x \in \{4\} \cup (8,11]$\\
E.$x \in \{-4\} \cup [8,11)$\\
F.$x \in \{4\} \cup [8,11)$\\
G.$x \in \{-4\} \cup (8,11)$\\
H.$x \in \{4\} \cup [8,11]$
\testStop
\kluczStart
A
\kluczStop



\zadStart{Zadanie z Wikieł Z 1.62 c) moja wersja nr 348}

Rozwiązać nierówności $(8-x)(x+4)^{2}(12-x)^{3}\le0$.
\zadStop
\rozwStart{Patryk Wirkus}{}
Miejsca zerowe naszego wielomianu to: $8, -4, 12$.\\
Wielomian jest stopnia parzystego, ponadto znak współczynnika przy\linebreak najwyższej potędze x jest ujemny.\\ W związku z tym wykres wielomianu zaczyna się od lewej strony powyżej osi OX.\\
Ponadto w punkcie $-4$ wykres odbija się od osi poziomej.\\
A więc $$x \in \{-4\} \cup [8,12].$$
\rozwStop
\odpStart
$x \in \{-4\} \cup [8,12]$
\odpStop
\testStart
A.$x \in \{-4\} \cup [8,12]$\\
B.$x \in \{4\} \cup (8,12)$\\
C.$x \in \{-4\} \cup (8,12]$\\
D.$x \in \{4\} \cup (8,12]$\\
E.$x \in \{-4\} \cup [8,12)$\\
F.$x \in \{4\} \cup [8,12)$\\
G.$x \in \{-4\} \cup (8,12)$\\
H.$x \in \{4\} \cup [8,12]$
\testStop
\kluczStart
A
\kluczStop



\zadStart{Zadanie z Wikieł Z 1.62 c) moja wersja nr 349}

Rozwiązać nierówności $(8-x)(x+4)^{2}(13-x)^{3}\le0$.
\zadStop
\rozwStart{Patryk Wirkus}{}
Miejsca zerowe naszego wielomianu to: $8, -4, 13$.\\
Wielomian jest stopnia parzystego, ponadto znak współczynnika przy\linebreak najwyższej potędze x jest ujemny.\\ W związku z tym wykres wielomianu zaczyna się od lewej strony powyżej osi OX.\\
Ponadto w punkcie $-4$ wykres odbija się od osi poziomej.\\
A więc $$x \in \{-4\} \cup [8,13].$$
\rozwStop
\odpStart
$x \in \{-4\} \cup [8,13]$
\odpStop
\testStart
A.$x \in \{-4\} \cup [8,13]$\\
B.$x \in \{4\} \cup (8,13)$\\
C.$x \in \{-4\} \cup (8,13]$\\
D.$x \in \{4\} \cup (8,13]$\\
E.$x \in \{-4\} \cup [8,13)$\\
F.$x \in \{4\} \cup [8,13)$\\
G.$x \in \{-4\} \cup (8,13)$\\
H.$x \in \{4\} \cup [8,13]$
\testStop
\kluczStart
A
\kluczStop



\zadStart{Zadanie z Wikieł Z 1.62 c) moja wersja nr 350}

Rozwiązać nierówności $(8-x)(x+4)^{2}(14-x)^{3}\le0$.
\zadStop
\rozwStart{Patryk Wirkus}{}
Miejsca zerowe naszego wielomianu to: $8, -4, 14$.\\
Wielomian jest stopnia parzystego, ponadto znak współczynnika przy\linebreak najwyższej potędze x jest ujemny.\\ W związku z tym wykres wielomianu zaczyna się od lewej strony powyżej osi OX.\\
Ponadto w punkcie $-4$ wykres odbija się od osi poziomej.\\
A więc $$x \in \{-4\} \cup [8,14].$$
\rozwStop
\odpStart
$x \in \{-4\} \cup [8,14]$
\odpStop
\testStart
A.$x \in \{-4\} \cup [8,14]$\\
B.$x \in \{4\} \cup (8,14)$\\
C.$x \in \{-4\} \cup (8,14]$\\
D.$x \in \{4\} \cup (8,14]$\\
E.$x \in \{-4\} \cup [8,14)$\\
F.$x \in \{4\} \cup [8,14)$\\
G.$x \in \{-4\} \cup (8,14)$\\
H.$x \in \{4\} \cup [8,14]$
\testStop
\kluczStart
A
\kluczStop



\zadStart{Zadanie z Wikieł Z 1.62 c) moja wersja nr 351}

Rozwiązać nierówności $(8-x)(x+4)^{2}(15-x)^{3}\le0$.
\zadStop
\rozwStart{Patryk Wirkus}{}
Miejsca zerowe naszego wielomianu to: $8, -4, 15$.\\
Wielomian jest stopnia parzystego, ponadto znak współczynnika przy\linebreak najwyższej potędze x jest ujemny.\\ W związku z tym wykres wielomianu zaczyna się od lewej strony powyżej osi OX.\\
Ponadto w punkcie $-4$ wykres odbija się od osi poziomej.\\
A więc $$x \in \{-4\} \cup [8,15].$$
\rozwStop
\odpStart
$x \in \{-4\} \cup [8,15]$
\odpStop
\testStart
A.$x \in \{-4\} \cup [8,15]$\\
B.$x \in \{4\} \cup (8,15)$\\
C.$x \in \{-4\} \cup (8,15]$\\
D.$x \in \{4\} \cup (8,15]$\\
E.$x \in \{-4\} \cup [8,15)$\\
F.$x \in \{4\} \cup [8,15)$\\
G.$x \in \{-4\} \cup (8,15)$\\
H.$x \in \{4\} \cup [8,15]$
\testStop
\kluczStart
A
\kluczStop



\zadStart{Zadanie z Wikieł Z 1.62 c) moja wersja nr 352}

Rozwiązać nierówności $(8-x)(x+4)^{2}(16-x)^{3}\le0$.
\zadStop
\rozwStart{Patryk Wirkus}{}
Miejsca zerowe naszego wielomianu to: $8, -4, 16$.\\
Wielomian jest stopnia parzystego, ponadto znak współczynnika przy\linebreak najwyższej potędze x jest ujemny.\\ W związku z tym wykres wielomianu zaczyna się od lewej strony powyżej osi OX.\\
Ponadto w punkcie $-4$ wykres odbija się od osi poziomej.\\
A więc $$x \in \{-4\} \cup [8,16].$$
\rozwStop
\odpStart
$x \in \{-4\} \cup [8,16]$
\odpStop
\testStart
A.$x \in \{-4\} \cup [8,16]$\\
B.$x \in \{4\} \cup (8,16)$\\
C.$x \in \{-4\} \cup (8,16]$\\
D.$x \in \{4\} \cup (8,16]$\\
E.$x \in \{-4\} \cup [8,16)$\\
F.$x \in \{4\} \cup [8,16)$\\
G.$x \in \{-4\} \cup (8,16)$\\
H.$x \in \{4\} \cup [8,16]$
\testStop
\kluczStart
A
\kluczStop



\zadStart{Zadanie z Wikieł Z 1.62 c) moja wersja nr 353}

Rozwiązać nierówności $(8-x)(x+4)^{2}(17-x)^{3}\le0$.
\zadStop
\rozwStart{Patryk Wirkus}{}
Miejsca zerowe naszego wielomianu to: $8, -4, 17$.\\
Wielomian jest stopnia parzystego, ponadto znak współczynnika przy\linebreak najwyższej potędze x jest ujemny.\\ W związku z tym wykres wielomianu zaczyna się od lewej strony powyżej osi OX.\\
Ponadto w punkcie $-4$ wykres odbija się od osi poziomej.\\
A więc $$x \in \{-4\} \cup [8,17].$$
\rozwStop
\odpStart
$x \in \{-4\} \cup [8,17]$
\odpStop
\testStart
A.$x \in \{-4\} \cup [8,17]$\\
B.$x \in \{4\} \cup (8,17)$\\
C.$x \in \{-4\} \cup (8,17]$\\
D.$x \in \{4\} \cup (8,17]$\\
E.$x \in \{-4\} \cup [8,17)$\\
F.$x \in \{4\} \cup [8,17)$\\
G.$x \in \{-4\} \cup (8,17)$\\
H.$x \in \{4\} \cup [8,17]$
\testStop
\kluczStart
A
\kluczStop



\zadStart{Zadanie z Wikieł Z 1.62 c) moja wersja nr 354}

Rozwiązać nierówności $(8-x)(x+4)^{2}(18-x)^{3}\le0$.
\zadStop
\rozwStart{Patryk Wirkus}{}
Miejsca zerowe naszego wielomianu to: $8, -4, 18$.\\
Wielomian jest stopnia parzystego, ponadto znak współczynnika przy\linebreak najwyższej potędze x jest ujemny.\\ W związku z tym wykres wielomianu zaczyna się od lewej strony powyżej osi OX.\\
Ponadto w punkcie $-4$ wykres odbija się od osi poziomej.\\
A więc $$x \in \{-4\} \cup [8,18].$$
\rozwStop
\odpStart
$x \in \{-4\} \cup [8,18]$
\odpStop
\testStart
A.$x \in \{-4\} \cup [8,18]$\\
B.$x \in \{4\} \cup (8,18)$\\
C.$x \in \{-4\} \cup (8,18]$\\
D.$x \in \{4\} \cup (8,18]$\\
E.$x \in \{-4\} \cup [8,18)$\\
F.$x \in \{4\} \cup [8,18)$\\
G.$x \in \{-4\} \cup (8,18)$\\
H.$x \in \{4\} \cup [8,18]$
\testStop
\kluczStart
A
\kluczStop



\zadStart{Zadanie z Wikieł Z 1.62 c) moja wersja nr 355}

Rozwiązać nierówności $(8-x)(x+4)^{2}(19-x)^{3}\le0$.
\zadStop
\rozwStart{Patryk Wirkus}{}
Miejsca zerowe naszego wielomianu to: $8, -4, 19$.\\
Wielomian jest stopnia parzystego, ponadto znak współczynnika przy\linebreak najwyższej potędze x jest ujemny.\\ W związku z tym wykres wielomianu zaczyna się od lewej strony powyżej osi OX.\\
Ponadto w punkcie $-4$ wykres odbija się od osi poziomej.\\
A więc $$x \in \{-4\} \cup [8,19].$$
\rozwStop
\odpStart
$x \in \{-4\} \cup [8,19]$
\odpStop
\testStart
A.$x \in \{-4\} \cup [8,19]$\\
B.$x \in \{4\} \cup (8,19)$\\
C.$x \in \{-4\} \cup (8,19]$\\
D.$x \in \{4\} \cup (8,19]$\\
E.$x \in \{-4\} \cup [8,19)$\\
F.$x \in \{4\} \cup [8,19)$\\
G.$x \in \{-4\} \cup (8,19)$\\
H.$x \in \{4\} \cup [8,19]$
\testStop
\kluczStart
A
\kluczStop



\zadStart{Zadanie z Wikieł Z 1.62 c) moja wersja nr 356}

Rozwiązać nierówności $(8-x)(x+4)^{2}(20-x)^{3}\le0$.
\zadStop
\rozwStart{Patryk Wirkus}{}
Miejsca zerowe naszego wielomianu to: $8, -4, 20$.\\
Wielomian jest stopnia parzystego, ponadto znak współczynnika przy\linebreak najwyższej potędze x jest ujemny.\\ W związku z tym wykres wielomianu zaczyna się od lewej strony powyżej osi OX.\\
Ponadto w punkcie $-4$ wykres odbija się od osi poziomej.\\
A więc $$x \in \{-4\} \cup [8,20].$$
\rozwStop
\odpStart
$x \in \{-4\} \cup [8,20]$
\odpStop
\testStart
A.$x \in \{-4\} \cup [8,20]$\\
B.$x \in \{4\} \cup (8,20)$\\
C.$x \in \{-4\} \cup (8,20]$\\
D.$x \in \{4\} \cup (8,20]$\\
E.$x \in \{-4\} \cup [8,20)$\\
F.$x \in \{4\} \cup [8,20)$\\
G.$x \in \{-4\} \cup (8,20)$\\
H.$x \in \{4\} \cup [8,20]$
\testStop
\kluczStart
A
\kluczStop



\zadStart{Zadanie z Wikieł Z 1.62 c) moja wersja nr 357}

Rozwiązać nierówności $(8-x)(x+5)^{2}(9-x)^{3}\le0$.
\zadStop
\rozwStart{Patryk Wirkus}{}
Miejsca zerowe naszego wielomianu to: $8, -5, 9$.\\
Wielomian jest stopnia parzystego, ponadto znak współczynnika przy\linebreak najwyższej potędze x jest ujemny.\\ W związku z tym wykres wielomianu zaczyna się od lewej strony powyżej osi OX.\\
Ponadto w punkcie $-5$ wykres odbija się od osi poziomej.\\
A więc $$x \in \{-5\} \cup [8,9].$$
\rozwStop
\odpStart
$x \in \{-5\} \cup [8,9]$
\odpStop
\testStart
A.$x \in \{-5\} \cup [8,9]$\\
B.$x \in \{5\} \cup (8,9)$\\
C.$x \in \{-5\} \cup (8,9]$\\
D.$x \in \{5\} \cup (8,9]$\\
E.$x \in \{-5\} \cup [8,9)$\\
F.$x \in \{5\} \cup [8,9)$\\
G.$x \in \{-5\} \cup (8,9)$\\
H.$x \in \{5\} \cup [8,9]$
\testStop
\kluczStart
A
\kluczStop



\zadStart{Zadanie z Wikieł Z 1.62 c) moja wersja nr 358}

Rozwiązać nierówności $(8-x)(x+5)^{2}(10-x)^{3}\le0$.
\zadStop
\rozwStart{Patryk Wirkus}{}
Miejsca zerowe naszego wielomianu to: $8, -5, 10$.\\
Wielomian jest stopnia parzystego, ponadto znak współczynnika przy\linebreak najwyższej potędze x jest ujemny.\\ W związku z tym wykres wielomianu zaczyna się od lewej strony powyżej osi OX.\\
Ponadto w punkcie $-5$ wykres odbija się od osi poziomej.\\
A więc $$x \in \{-5\} \cup [8,10].$$
\rozwStop
\odpStart
$x \in \{-5\} \cup [8,10]$
\odpStop
\testStart
A.$x \in \{-5\} \cup [8,10]$\\
B.$x \in \{5\} \cup (8,10)$\\
C.$x \in \{-5\} \cup (8,10]$\\
D.$x \in \{5\} \cup (8,10]$\\
E.$x \in \{-5\} \cup [8,10)$\\
F.$x \in \{5\} \cup [8,10)$\\
G.$x \in \{-5\} \cup (8,10)$\\
H.$x \in \{5\} \cup [8,10]$
\testStop
\kluczStart
A
\kluczStop



\zadStart{Zadanie z Wikieł Z 1.62 c) moja wersja nr 359}

Rozwiązać nierówności $(8-x)(x+5)^{2}(11-x)^{3}\le0$.
\zadStop
\rozwStart{Patryk Wirkus}{}
Miejsca zerowe naszego wielomianu to: $8, -5, 11$.\\
Wielomian jest stopnia parzystego, ponadto znak współczynnika przy\linebreak najwyższej potędze x jest ujemny.\\ W związku z tym wykres wielomianu zaczyna się od lewej strony powyżej osi OX.\\
Ponadto w punkcie $-5$ wykres odbija się od osi poziomej.\\
A więc $$x \in \{-5\} \cup [8,11].$$
\rozwStop
\odpStart
$x \in \{-5\} \cup [8,11]$
\odpStop
\testStart
A.$x \in \{-5\} \cup [8,11]$\\
B.$x \in \{5\} \cup (8,11)$\\
C.$x \in \{-5\} \cup (8,11]$\\
D.$x \in \{5\} \cup (8,11]$\\
E.$x \in \{-5\} \cup [8,11)$\\
F.$x \in \{5\} \cup [8,11)$\\
G.$x \in \{-5\} \cup (8,11)$\\
H.$x \in \{5\} \cup [8,11]$
\testStop
\kluczStart
A
\kluczStop



\zadStart{Zadanie z Wikieł Z 1.62 c) moja wersja nr 360}

Rozwiązać nierówności $(8-x)(x+5)^{2}(12-x)^{3}\le0$.
\zadStop
\rozwStart{Patryk Wirkus}{}
Miejsca zerowe naszego wielomianu to: $8, -5, 12$.\\
Wielomian jest stopnia parzystego, ponadto znak współczynnika przy\linebreak najwyższej potędze x jest ujemny.\\ W związku z tym wykres wielomianu zaczyna się od lewej strony powyżej osi OX.\\
Ponadto w punkcie $-5$ wykres odbija się od osi poziomej.\\
A więc $$x \in \{-5\} \cup [8,12].$$
\rozwStop
\odpStart
$x \in \{-5\} \cup [8,12]$
\odpStop
\testStart
A.$x \in \{-5\} \cup [8,12]$\\
B.$x \in \{5\} \cup (8,12)$\\
C.$x \in \{-5\} \cup (8,12]$\\
D.$x \in \{5\} \cup (8,12]$\\
E.$x \in \{-5\} \cup [8,12)$\\
F.$x \in \{5\} \cup [8,12)$\\
G.$x \in \{-5\} \cup (8,12)$\\
H.$x \in \{5\} \cup [8,12]$
\testStop
\kluczStart
A
\kluczStop



\zadStart{Zadanie z Wikieł Z 1.62 c) moja wersja nr 361}

Rozwiązać nierówności $(8-x)(x+5)^{2}(13-x)^{3}\le0$.
\zadStop
\rozwStart{Patryk Wirkus}{}
Miejsca zerowe naszego wielomianu to: $8, -5, 13$.\\
Wielomian jest stopnia parzystego, ponadto znak współczynnika przy\linebreak najwyższej potędze x jest ujemny.\\ W związku z tym wykres wielomianu zaczyna się od lewej strony powyżej osi OX.\\
Ponadto w punkcie $-5$ wykres odbija się od osi poziomej.\\
A więc $$x \in \{-5\} \cup [8,13].$$
\rozwStop
\odpStart
$x \in \{-5\} \cup [8,13]$
\odpStop
\testStart
A.$x \in \{-5\} \cup [8,13]$\\
B.$x \in \{5\} \cup (8,13)$\\
C.$x \in \{-5\} \cup (8,13]$\\
D.$x \in \{5\} \cup (8,13]$\\
E.$x \in \{-5\} \cup [8,13)$\\
F.$x \in \{5\} \cup [8,13)$\\
G.$x \in \{-5\} \cup (8,13)$\\
H.$x \in \{5\} \cup [8,13]$
\testStop
\kluczStart
A
\kluczStop



\zadStart{Zadanie z Wikieł Z 1.62 c) moja wersja nr 362}

Rozwiązać nierówności $(8-x)(x+5)^{2}(14-x)^{3}\le0$.
\zadStop
\rozwStart{Patryk Wirkus}{}
Miejsca zerowe naszego wielomianu to: $8, -5, 14$.\\
Wielomian jest stopnia parzystego, ponadto znak współczynnika przy\linebreak najwyższej potędze x jest ujemny.\\ W związku z tym wykres wielomianu zaczyna się od lewej strony powyżej osi OX.\\
Ponadto w punkcie $-5$ wykres odbija się od osi poziomej.\\
A więc $$x \in \{-5\} \cup [8,14].$$
\rozwStop
\odpStart
$x \in \{-5\} \cup [8,14]$
\odpStop
\testStart
A.$x \in \{-5\} \cup [8,14]$\\
B.$x \in \{5\} \cup (8,14)$\\
C.$x \in \{-5\} \cup (8,14]$\\
D.$x \in \{5\} \cup (8,14]$\\
E.$x \in \{-5\} \cup [8,14)$\\
F.$x \in \{5\} \cup [8,14)$\\
G.$x \in \{-5\} \cup (8,14)$\\
H.$x \in \{5\} \cup [8,14]$
\testStop
\kluczStart
A
\kluczStop



\zadStart{Zadanie z Wikieł Z 1.62 c) moja wersja nr 363}

Rozwiązać nierówności $(8-x)(x+5)^{2}(15-x)^{3}\le0$.
\zadStop
\rozwStart{Patryk Wirkus}{}
Miejsca zerowe naszego wielomianu to: $8, -5, 15$.\\
Wielomian jest stopnia parzystego, ponadto znak współczynnika przy\linebreak najwyższej potędze x jest ujemny.\\ W związku z tym wykres wielomianu zaczyna się od lewej strony powyżej osi OX.\\
Ponadto w punkcie $-5$ wykres odbija się od osi poziomej.\\
A więc $$x \in \{-5\} \cup [8,15].$$
\rozwStop
\odpStart
$x \in \{-5\} \cup [8,15]$
\odpStop
\testStart
A.$x \in \{-5\} \cup [8,15]$\\
B.$x \in \{5\} \cup (8,15)$\\
C.$x \in \{-5\} \cup (8,15]$\\
D.$x \in \{5\} \cup (8,15]$\\
E.$x \in \{-5\} \cup [8,15)$\\
F.$x \in \{5\} \cup [8,15)$\\
G.$x \in \{-5\} \cup (8,15)$\\
H.$x \in \{5\} \cup [8,15]$
\testStop
\kluczStart
A
\kluczStop



\zadStart{Zadanie z Wikieł Z 1.62 c) moja wersja nr 364}

Rozwiązać nierówności $(8-x)(x+5)^{2}(16-x)^{3}\le0$.
\zadStop
\rozwStart{Patryk Wirkus}{}
Miejsca zerowe naszego wielomianu to: $8, -5, 16$.\\
Wielomian jest stopnia parzystego, ponadto znak współczynnika przy\linebreak najwyższej potędze x jest ujemny.\\ W związku z tym wykres wielomianu zaczyna się od lewej strony powyżej osi OX.\\
Ponadto w punkcie $-5$ wykres odbija się od osi poziomej.\\
A więc $$x \in \{-5\} \cup [8,16].$$
\rozwStop
\odpStart
$x \in \{-5\} \cup [8,16]$
\odpStop
\testStart
A.$x \in \{-5\} \cup [8,16]$\\
B.$x \in \{5\} \cup (8,16)$\\
C.$x \in \{-5\} \cup (8,16]$\\
D.$x \in \{5\} \cup (8,16]$\\
E.$x \in \{-5\} \cup [8,16)$\\
F.$x \in \{5\} \cup [8,16)$\\
G.$x \in \{-5\} \cup (8,16)$\\
H.$x \in \{5\} \cup [8,16]$
\testStop
\kluczStart
A
\kluczStop



\zadStart{Zadanie z Wikieł Z 1.62 c) moja wersja nr 365}

Rozwiązać nierówności $(8-x)(x+5)^{2}(17-x)^{3}\le0$.
\zadStop
\rozwStart{Patryk Wirkus}{}
Miejsca zerowe naszego wielomianu to: $8, -5, 17$.\\
Wielomian jest stopnia parzystego, ponadto znak współczynnika przy\linebreak najwyższej potędze x jest ujemny.\\ W związku z tym wykres wielomianu zaczyna się od lewej strony powyżej osi OX.\\
Ponadto w punkcie $-5$ wykres odbija się od osi poziomej.\\
A więc $$x \in \{-5\} \cup [8,17].$$
\rozwStop
\odpStart
$x \in \{-5\} \cup [8,17]$
\odpStop
\testStart
A.$x \in \{-5\} \cup [8,17]$\\
B.$x \in \{5\} \cup (8,17)$\\
C.$x \in \{-5\} \cup (8,17]$\\
D.$x \in \{5\} \cup (8,17]$\\
E.$x \in \{-5\} \cup [8,17)$\\
F.$x \in \{5\} \cup [8,17)$\\
G.$x \in \{-5\} \cup (8,17)$\\
H.$x \in \{5\} \cup [8,17]$
\testStop
\kluczStart
A
\kluczStop



\zadStart{Zadanie z Wikieł Z 1.62 c) moja wersja nr 366}

Rozwiązać nierówności $(8-x)(x+5)^{2}(18-x)^{3}\le0$.
\zadStop
\rozwStart{Patryk Wirkus}{}
Miejsca zerowe naszego wielomianu to: $8, -5, 18$.\\
Wielomian jest stopnia parzystego, ponadto znak współczynnika przy\linebreak najwyższej potędze x jest ujemny.\\ W związku z tym wykres wielomianu zaczyna się od lewej strony powyżej osi OX.\\
Ponadto w punkcie $-5$ wykres odbija się od osi poziomej.\\
A więc $$x \in \{-5\} \cup [8,18].$$
\rozwStop
\odpStart
$x \in \{-5\} \cup [8,18]$
\odpStop
\testStart
A.$x \in \{-5\} \cup [8,18]$\\
B.$x \in \{5\} \cup (8,18)$\\
C.$x \in \{-5\} \cup (8,18]$\\
D.$x \in \{5\} \cup (8,18]$\\
E.$x \in \{-5\} \cup [8,18)$\\
F.$x \in \{5\} \cup [8,18)$\\
G.$x \in \{-5\} \cup (8,18)$\\
H.$x \in \{5\} \cup [8,18]$
\testStop
\kluczStart
A
\kluczStop



\zadStart{Zadanie z Wikieł Z 1.62 c) moja wersja nr 367}

Rozwiązać nierówności $(8-x)(x+5)^{2}(19-x)^{3}\le0$.
\zadStop
\rozwStart{Patryk Wirkus}{}
Miejsca zerowe naszego wielomianu to: $8, -5, 19$.\\
Wielomian jest stopnia parzystego, ponadto znak współczynnika przy\linebreak najwyższej potędze x jest ujemny.\\ W związku z tym wykres wielomianu zaczyna się od lewej strony powyżej osi OX.\\
Ponadto w punkcie $-5$ wykres odbija się od osi poziomej.\\
A więc $$x \in \{-5\} \cup [8,19].$$
\rozwStop
\odpStart
$x \in \{-5\} \cup [8,19]$
\odpStop
\testStart
A.$x \in \{-5\} \cup [8,19]$\\
B.$x \in \{5\} \cup (8,19)$\\
C.$x \in \{-5\} \cup (8,19]$\\
D.$x \in \{5\} \cup (8,19]$\\
E.$x \in \{-5\} \cup [8,19)$\\
F.$x \in \{5\} \cup [8,19)$\\
G.$x \in \{-5\} \cup (8,19)$\\
H.$x \in \{5\} \cup [8,19]$
\testStop
\kluczStart
A
\kluczStop



\zadStart{Zadanie z Wikieł Z 1.62 c) moja wersja nr 368}

Rozwiązać nierówności $(8-x)(x+5)^{2}(20-x)^{3}\le0$.
\zadStop
\rozwStart{Patryk Wirkus}{}
Miejsca zerowe naszego wielomianu to: $8, -5, 20$.\\
Wielomian jest stopnia parzystego, ponadto znak współczynnika przy\linebreak najwyższej potędze x jest ujemny.\\ W związku z tym wykres wielomianu zaczyna się od lewej strony powyżej osi OX.\\
Ponadto w punkcie $-5$ wykres odbija się od osi poziomej.\\
A więc $$x \in \{-5\} \cup [8,20].$$
\rozwStop
\odpStart
$x \in \{-5\} \cup [8,20]$
\odpStop
\testStart
A.$x \in \{-5\} \cup [8,20]$\\
B.$x \in \{5\} \cup (8,20)$\\
C.$x \in \{-5\} \cup (8,20]$\\
D.$x \in \{5\} \cup (8,20]$\\
E.$x \in \{-5\} \cup [8,20)$\\
F.$x \in \{5\} \cup [8,20)$\\
G.$x \in \{-5\} \cup (8,20)$\\
H.$x \in \{5\} \cup [8,20]$
\testStop
\kluczStart
A
\kluczStop



\zadStart{Zadanie z Wikieł Z 1.62 c) moja wersja nr 369}

Rozwiązać nierówności $(8-x)(x+6)^{2}(9-x)^{3}\le0$.
\zadStop
\rozwStart{Patryk Wirkus}{}
Miejsca zerowe naszego wielomianu to: $8, -6, 9$.\\
Wielomian jest stopnia parzystego, ponadto znak współczynnika przy\linebreak najwyższej potędze x jest ujemny.\\ W związku z tym wykres wielomianu zaczyna się od lewej strony powyżej osi OX.\\
Ponadto w punkcie $-6$ wykres odbija się od osi poziomej.\\
A więc $$x \in \{-6\} \cup [8,9].$$
\rozwStop
\odpStart
$x \in \{-6\} \cup [8,9]$
\odpStop
\testStart
A.$x \in \{-6\} \cup [8,9]$\\
B.$x \in \{6\} \cup (8,9)$\\
C.$x \in \{-6\} \cup (8,9]$\\
D.$x \in \{6\} \cup (8,9]$\\
E.$x \in \{-6\} \cup [8,9)$\\
F.$x \in \{6\} \cup [8,9)$\\
G.$x \in \{-6\} \cup (8,9)$\\
H.$x \in \{6\} \cup [8,9]$
\testStop
\kluczStart
A
\kluczStop



\zadStart{Zadanie z Wikieł Z 1.62 c) moja wersja nr 370}

Rozwiązać nierówności $(8-x)(x+6)^{2}(10-x)^{3}\le0$.
\zadStop
\rozwStart{Patryk Wirkus}{}
Miejsca zerowe naszego wielomianu to: $8, -6, 10$.\\
Wielomian jest stopnia parzystego, ponadto znak współczynnika przy\linebreak najwyższej potędze x jest ujemny.\\ W związku z tym wykres wielomianu zaczyna się od lewej strony powyżej osi OX.\\
Ponadto w punkcie $-6$ wykres odbija się od osi poziomej.\\
A więc $$x \in \{-6\} \cup [8,10].$$
\rozwStop
\odpStart
$x \in \{-6\} \cup [8,10]$
\odpStop
\testStart
A.$x \in \{-6\} \cup [8,10]$\\
B.$x \in \{6\} \cup (8,10)$\\
C.$x \in \{-6\} \cup (8,10]$\\
D.$x \in \{6\} \cup (8,10]$\\
E.$x \in \{-6\} \cup [8,10)$\\
F.$x \in \{6\} \cup [8,10)$\\
G.$x \in \{-6\} \cup (8,10)$\\
H.$x \in \{6\} \cup [8,10]$
\testStop
\kluczStart
A
\kluczStop



\zadStart{Zadanie z Wikieł Z 1.62 c) moja wersja nr 371}

Rozwiązać nierówności $(8-x)(x+6)^{2}(11-x)^{3}\le0$.
\zadStop
\rozwStart{Patryk Wirkus}{}
Miejsca zerowe naszego wielomianu to: $8, -6, 11$.\\
Wielomian jest stopnia parzystego, ponadto znak współczynnika przy\linebreak najwyższej potędze x jest ujemny.\\ W związku z tym wykres wielomianu zaczyna się od lewej strony powyżej osi OX.\\
Ponadto w punkcie $-6$ wykres odbija się od osi poziomej.\\
A więc $$x \in \{-6\} \cup [8,11].$$
\rozwStop
\odpStart
$x \in \{-6\} \cup [8,11]$
\odpStop
\testStart
A.$x \in \{-6\} \cup [8,11]$\\
B.$x \in \{6\} \cup (8,11)$\\
C.$x \in \{-6\} \cup (8,11]$\\
D.$x \in \{6\} \cup (8,11]$\\
E.$x \in \{-6\} \cup [8,11)$\\
F.$x \in \{6\} \cup [8,11)$\\
G.$x \in \{-6\} \cup (8,11)$\\
H.$x \in \{6\} \cup [8,11]$
\testStop
\kluczStart
A
\kluczStop



\zadStart{Zadanie z Wikieł Z 1.62 c) moja wersja nr 372}

Rozwiązać nierówności $(8-x)(x+6)^{2}(12-x)^{3}\le0$.
\zadStop
\rozwStart{Patryk Wirkus}{}
Miejsca zerowe naszego wielomianu to: $8, -6, 12$.\\
Wielomian jest stopnia parzystego, ponadto znak współczynnika przy\linebreak najwyższej potędze x jest ujemny.\\ W związku z tym wykres wielomianu zaczyna się od lewej strony powyżej osi OX.\\
Ponadto w punkcie $-6$ wykres odbija się od osi poziomej.\\
A więc $$x \in \{-6\} \cup [8,12].$$
\rozwStop
\odpStart
$x \in \{-6\} \cup [8,12]$
\odpStop
\testStart
A.$x \in \{-6\} \cup [8,12]$\\
B.$x \in \{6\} \cup (8,12)$\\
C.$x \in \{-6\} \cup (8,12]$\\
D.$x \in \{6\} \cup (8,12]$\\
E.$x \in \{-6\} \cup [8,12)$\\
F.$x \in \{6\} \cup [8,12)$\\
G.$x \in \{-6\} \cup (8,12)$\\
H.$x \in \{6\} \cup [8,12]$
\testStop
\kluczStart
A
\kluczStop



\zadStart{Zadanie z Wikieł Z 1.62 c) moja wersja nr 373}

Rozwiązać nierówności $(8-x)(x+6)^{2}(13-x)^{3}\le0$.
\zadStop
\rozwStart{Patryk Wirkus}{}
Miejsca zerowe naszego wielomianu to: $8, -6, 13$.\\
Wielomian jest stopnia parzystego, ponadto znak współczynnika przy\linebreak najwyższej potędze x jest ujemny.\\ W związku z tym wykres wielomianu zaczyna się od lewej strony powyżej osi OX.\\
Ponadto w punkcie $-6$ wykres odbija się od osi poziomej.\\
A więc $$x \in \{-6\} \cup [8,13].$$
\rozwStop
\odpStart
$x \in \{-6\} \cup [8,13]$
\odpStop
\testStart
A.$x \in \{-6\} \cup [8,13]$\\
B.$x \in \{6\} \cup (8,13)$\\
C.$x \in \{-6\} \cup (8,13]$\\
D.$x \in \{6\} \cup (8,13]$\\
E.$x \in \{-6\} \cup [8,13)$\\
F.$x \in \{6\} \cup [8,13)$\\
G.$x \in \{-6\} \cup (8,13)$\\
H.$x \in \{6\} \cup [8,13]$
\testStop
\kluczStart
A
\kluczStop



\zadStart{Zadanie z Wikieł Z 1.62 c) moja wersja nr 374}

Rozwiązać nierówności $(8-x)(x+6)^{2}(14-x)^{3}\le0$.
\zadStop
\rozwStart{Patryk Wirkus}{}
Miejsca zerowe naszego wielomianu to: $8, -6, 14$.\\
Wielomian jest stopnia parzystego, ponadto znak współczynnika przy\linebreak najwyższej potędze x jest ujemny.\\ W związku z tym wykres wielomianu zaczyna się od lewej strony powyżej osi OX.\\
Ponadto w punkcie $-6$ wykres odbija się od osi poziomej.\\
A więc $$x \in \{-6\} \cup [8,14].$$
\rozwStop
\odpStart
$x \in \{-6\} \cup [8,14]$
\odpStop
\testStart
A.$x \in \{-6\} \cup [8,14]$\\
B.$x \in \{6\} \cup (8,14)$\\
C.$x \in \{-6\} \cup (8,14]$\\
D.$x \in \{6\} \cup (8,14]$\\
E.$x \in \{-6\} \cup [8,14)$\\
F.$x \in \{6\} \cup [8,14)$\\
G.$x \in \{-6\} \cup (8,14)$\\
H.$x \in \{6\} \cup [8,14]$
\testStop
\kluczStart
A
\kluczStop



\zadStart{Zadanie z Wikieł Z 1.62 c) moja wersja nr 375}

Rozwiązać nierówności $(8-x)(x+6)^{2}(15-x)^{3}\le0$.
\zadStop
\rozwStart{Patryk Wirkus}{}
Miejsca zerowe naszego wielomianu to: $8, -6, 15$.\\
Wielomian jest stopnia parzystego, ponadto znak współczynnika przy\linebreak najwyższej potędze x jest ujemny.\\ W związku z tym wykres wielomianu zaczyna się od lewej strony powyżej osi OX.\\
Ponadto w punkcie $-6$ wykres odbija się od osi poziomej.\\
A więc $$x \in \{-6\} \cup [8,15].$$
\rozwStop
\odpStart
$x \in \{-6\} \cup [8,15]$
\odpStop
\testStart
A.$x \in \{-6\} \cup [8,15]$\\
B.$x \in \{6\} \cup (8,15)$\\
C.$x \in \{-6\} \cup (8,15]$\\
D.$x \in \{6\} \cup (8,15]$\\
E.$x \in \{-6\} \cup [8,15)$\\
F.$x \in \{6\} \cup [8,15)$\\
G.$x \in \{-6\} \cup (8,15)$\\
H.$x \in \{6\} \cup [8,15]$
\testStop
\kluczStart
A
\kluczStop



\zadStart{Zadanie z Wikieł Z 1.62 c) moja wersja nr 376}

Rozwiązać nierówności $(8-x)(x+6)^{2}(16-x)^{3}\le0$.
\zadStop
\rozwStart{Patryk Wirkus}{}
Miejsca zerowe naszego wielomianu to: $8, -6, 16$.\\
Wielomian jest stopnia parzystego, ponadto znak współczynnika przy\linebreak najwyższej potędze x jest ujemny.\\ W związku z tym wykres wielomianu zaczyna się od lewej strony powyżej osi OX.\\
Ponadto w punkcie $-6$ wykres odbija się od osi poziomej.\\
A więc $$x \in \{-6\} \cup [8,16].$$
\rozwStop
\odpStart
$x \in \{-6\} \cup [8,16]$
\odpStop
\testStart
A.$x \in \{-6\} \cup [8,16]$\\
B.$x \in \{6\} \cup (8,16)$\\
C.$x \in \{-6\} \cup (8,16]$\\
D.$x \in \{6\} \cup (8,16]$\\
E.$x \in \{-6\} \cup [8,16)$\\
F.$x \in \{6\} \cup [8,16)$\\
G.$x \in \{-6\} \cup (8,16)$\\
H.$x \in \{6\} \cup [8,16]$
\testStop
\kluczStart
A
\kluczStop



\zadStart{Zadanie z Wikieł Z 1.62 c) moja wersja nr 377}

Rozwiązać nierówności $(8-x)(x+6)^{2}(17-x)^{3}\le0$.
\zadStop
\rozwStart{Patryk Wirkus}{}
Miejsca zerowe naszego wielomianu to: $8, -6, 17$.\\
Wielomian jest stopnia parzystego, ponadto znak współczynnika przy\linebreak najwyższej potędze x jest ujemny.\\ W związku z tym wykres wielomianu zaczyna się od lewej strony powyżej osi OX.\\
Ponadto w punkcie $-6$ wykres odbija się od osi poziomej.\\
A więc $$x \in \{-6\} \cup [8,17].$$
\rozwStop
\odpStart
$x \in \{-6\} \cup [8,17]$
\odpStop
\testStart
A.$x \in \{-6\} \cup [8,17]$\\
B.$x \in \{6\} \cup (8,17)$\\
C.$x \in \{-6\} \cup (8,17]$\\
D.$x \in \{6\} \cup (8,17]$\\
E.$x \in \{-6\} \cup [8,17)$\\
F.$x \in \{6\} \cup [8,17)$\\
G.$x \in \{-6\} \cup (8,17)$\\
H.$x \in \{6\} \cup [8,17]$
\testStop
\kluczStart
A
\kluczStop



\zadStart{Zadanie z Wikieł Z 1.62 c) moja wersja nr 378}

Rozwiązać nierówności $(8-x)(x+6)^{2}(18-x)^{3}\le0$.
\zadStop
\rozwStart{Patryk Wirkus}{}
Miejsca zerowe naszego wielomianu to: $8, -6, 18$.\\
Wielomian jest stopnia parzystego, ponadto znak współczynnika przy\linebreak najwyższej potędze x jest ujemny.\\ W związku z tym wykres wielomianu zaczyna się od lewej strony powyżej osi OX.\\
Ponadto w punkcie $-6$ wykres odbija się od osi poziomej.\\
A więc $$x \in \{-6\} \cup [8,18].$$
\rozwStop
\odpStart
$x \in \{-6\} \cup [8,18]$
\odpStop
\testStart
A.$x \in \{-6\} \cup [8,18]$\\
B.$x \in \{6\} \cup (8,18)$\\
C.$x \in \{-6\} \cup (8,18]$\\
D.$x \in \{6\} \cup (8,18]$\\
E.$x \in \{-6\} \cup [8,18)$\\
F.$x \in \{6\} \cup [8,18)$\\
G.$x \in \{-6\} \cup (8,18)$\\
H.$x \in \{6\} \cup [8,18]$
\testStop
\kluczStart
A
\kluczStop



\zadStart{Zadanie z Wikieł Z 1.62 c) moja wersja nr 379}

Rozwiązać nierówności $(8-x)(x+6)^{2}(19-x)^{3}\le0$.
\zadStop
\rozwStart{Patryk Wirkus}{}
Miejsca zerowe naszego wielomianu to: $8, -6, 19$.\\
Wielomian jest stopnia parzystego, ponadto znak współczynnika przy\linebreak najwyższej potędze x jest ujemny.\\ W związku z tym wykres wielomianu zaczyna się od lewej strony powyżej osi OX.\\
Ponadto w punkcie $-6$ wykres odbija się od osi poziomej.\\
A więc $$x \in \{-6\} \cup [8,19].$$
\rozwStop
\odpStart
$x \in \{-6\} \cup [8,19]$
\odpStop
\testStart
A.$x \in \{-6\} \cup [8,19]$\\
B.$x \in \{6\} \cup (8,19)$\\
C.$x \in \{-6\} \cup (8,19]$\\
D.$x \in \{6\} \cup (8,19]$\\
E.$x \in \{-6\} \cup [8,19)$\\
F.$x \in \{6\} \cup [8,19)$\\
G.$x \in \{-6\} \cup (8,19)$\\
H.$x \in \{6\} \cup [8,19]$
\testStop
\kluczStart
A
\kluczStop



\zadStart{Zadanie z Wikieł Z 1.62 c) moja wersja nr 380}

Rozwiązać nierówności $(8-x)(x+6)^{2}(20-x)^{3}\le0$.
\zadStop
\rozwStart{Patryk Wirkus}{}
Miejsca zerowe naszego wielomianu to: $8, -6, 20$.\\
Wielomian jest stopnia parzystego, ponadto znak współczynnika przy\linebreak najwyższej potędze x jest ujemny.\\ W związku z tym wykres wielomianu zaczyna się od lewej strony powyżej osi OX.\\
Ponadto w punkcie $-6$ wykres odbija się od osi poziomej.\\
A więc $$x \in \{-6\} \cup [8,20].$$
\rozwStop
\odpStart
$x \in \{-6\} \cup [8,20]$
\odpStop
\testStart
A.$x \in \{-6\} \cup [8,20]$\\
B.$x \in \{6\} \cup (8,20)$\\
C.$x \in \{-6\} \cup (8,20]$\\
D.$x \in \{6\} \cup (8,20]$\\
E.$x \in \{-6\} \cup [8,20)$\\
F.$x \in \{6\} \cup [8,20)$\\
G.$x \in \{-6\} \cup (8,20)$\\
H.$x \in \{6\} \cup [8,20]$
\testStop
\kluczStart
A
\kluczStop



\zadStart{Zadanie z Wikieł Z 1.62 c) moja wersja nr 381}

Rozwiązać nierówności $(8-x)(x+7)^{2}(9-x)^{3}\le0$.
\zadStop
\rozwStart{Patryk Wirkus}{}
Miejsca zerowe naszego wielomianu to: $8, -7, 9$.\\
Wielomian jest stopnia parzystego, ponadto znak współczynnika przy\linebreak najwyższej potędze x jest ujemny.\\ W związku z tym wykres wielomianu zaczyna się od lewej strony powyżej osi OX.\\
Ponadto w punkcie $-7$ wykres odbija się od osi poziomej.\\
A więc $$x \in \{-7\} \cup [8,9].$$
\rozwStop
\odpStart
$x \in \{-7\} \cup [8,9]$
\odpStop
\testStart
A.$x \in \{-7\} \cup [8,9]$\\
B.$x \in \{7\} \cup (8,9)$\\
C.$x \in \{-7\} \cup (8,9]$\\
D.$x \in \{7\} \cup (8,9]$\\
E.$x \in \{-7\} \cup [8,9)$\\
F.$x \in \{7\} \cup [8,9)$\\
G.$x \in \{-7\} \cup (8,9)$\\
H.$x \in \{7\} \cup [8,9]$
\testStop
\kluczStart
A
\kluczStop



\zadStart{Zadanie z Wikieł Z 1.62 c) moja wersja nr 382}

Rozwiązać nierówności $(8-x)(x+7)^{2}(10-x)^{3}\le0$.
\zadStop
\rozwStart{Patryk Wirkus}{}
Miejsca zerowe naszego wielomianu to: $8, -7, 10$.\\
Wielomian jest stopnia parzystego, ponadto znak współczynnika przy\linebreak najwyższej potędze x jest ujemny.\\ W związku z tym wykres wielomianu zaczyna się od lewej strony powyżej osi OX.\\
Ponadto w punkcie $-7$ wykres odbija się od osi poziomej.\\
A więc $$x \in \{-7\} \cup [8,10].$$
\rozwStop
\odpStart
$x \in \{-7\} \cup [8,10]$
\odpStop
\testStart
A.$x \in \{-7\} \cup [8,10]$\\
B.$x \in \{7\} \cup (8,10)$\\
C.$x \in \{-7\} \cup (8,10]$\\
D.$x \in \{7\} \cup (8,10]$\\
E.$x \in \{-7\} \cup [8,10)$\\
F.$x \in \{7\} \cup [8,10)$\\
G.$x \in \{-7\} \cup (8,10)$\\
H.$x \in \{7\} \cup [8,10]$
\testStop
\kluczStart
A
\kluczStop



\zadStart{Zadanie z Wikieł Z 1.62 c) moja wersja nr 383}

Rozwiązać nierówności $(8-x)(x+7)^{2}(11-x)^{3}\le0$.
\zadStop
\rozwStart{Patryk Wirkus}{}
Miejsca zerowe naszego wielomianu to: $8, -7, 11$.\\
Wielomian jest stopnia parzystego, ponadto znak współczynnika przy\linebreak najwyższej potędze x jest ujemny.\\ W związku z tym wykres wielomianu zaczyna się od lewej strony powyżej osi OX.\\
Ponadto w punkcie $-7$ wykres odbija się od osi poziomej.\\
A więc $$x \in \{-7\} \cup [8,11].$$
\rozwStop
\odpStart
$x \in \{-7\} \cup [8,11]$
\odpStop
\testStart
A.$x \in \{-7\} \cup [8,11]$\\
B.$x \in \{7\} \cup (8,11)$\\
C.$x \in \{-7\} \cup (8,11]$\\
D.$x \in \{7\} \cup (8,11]$\\
E.$x \in \{-7\} \cup [8,11)$\\
F.$x \in \{7\} \cup [8,11)$\\
G.$x \in \{-7\} \cup (8,11)$\\
H.$x \in \{7\} \cup [8,11]$
\testStop
\kluczStart
A
\kluczStop



\zadStart{Zadanie z Wikieł Z 1.62 c) moja wersja nr 384}

Rozwiązać nierówności $(8-x)(x+7)^{2}(12-x)^{3}\le0$.
\zadStop
\rozwStart{Patryk Wirkus}{}
Miejsca zerowe naszego wielomianu to: $8, -7, 12$.\\
Wielomian jest stopnia parzystego, ponadto znak współczynnika przy\linebreak najwyższej potędze x jest ujemny.\\ W związku z tym wykres wielomianu zaczyna się od lewej strony powyżej osi OX.\\
Ponadto w punkcie $-7$ wykres odbija się od osi poziomej.\\
A więc $$x \in \{-7\} \cup [8,12].$$
\rozwStop
\odpStart
$x \in \{-7\} \cup [8,12]$
\odpStop
\testStart
A.$x \in \{-7\} \cup [8,12]$\\
B.$x \in \{7\} \cup (8,12)$\\
C.$x \in \{-7\} \cup (8,12]$\\
D.$x \in \{7\} \cup (8,12]$\\
E.$x \in \{-7\} \cup [8,12)$\\
F.$x \in \{7\} \cup [8,12)$\\
G.$x \in \{-7\} \cup (8,12)$\\
H.$x \in \{7\} \cup [8,12]$
\testStop
\kluczStart
A
\kluczStop



\zadStart{Zadanie z Wikieł Z 1.62 c) moja wersja nr 385}

Rozwiązać nierówności $(8-x)(x+7)^{2}(13-x)^{3}\le0$.
\zadStop
\rozwStart{Patryk Wirkus}{}
Miejsca zerowe naszego wielomianu to: $8, -7, 13$.\\
Wielomian jest stopnia parzystego, ponadto znak współczynnika przy\linebreak najwyższej potędze x jest ujemny.\\ W związku z tym wykres wielomianu zaczyna się od lewej strony powyżej osi OX.\\
Ponadto w punkcie $-7$ wykres odbija się od osi poziomej.\\
A więc $$x \in \{-7\} \cup [8,13].$$
\rozwStop
\odpStart
$x \in \{-7\} \cup [8,13]$
\odpStop
\testStart
A.$x \in \{-7\} \cup [8,13]$\\
B.$x \in \{7\} \cup (8,13)$\\
C.$x \in \{-7\} \cup (8,13]$\\
D.$x \in \{7\} \cup (8,13]$\\
E.$x \in \{-7\} \cup [8,13)$\\
F.$x \in \{7\} \cup [8,13)$\\
G.$x \in \{-7\} \cup (8,13)$\\
H.$x \in \{7\} \cup [8,13]$
\testStop
\kluczStart
A
\kluczStop



\zadStart{Zadanie z Wikieł Z 1.62 c) moja wersja nr 386}

Rozwiązać nierówności $(8-x)(x+7)^{2}(14-x)^{3}\le0$.
\zadStop
\rozwStart{Patryk Wirkus}{}
Miejsca zerowe naszego wielomianu to: $8, -7, 14$.\\
Wielomian jest stopnia parzystego, ponadto znak współczynnika przy\linebreak najwyższej potędze x jest ujemny.\\ W związku z tym wykres wielomianu zaczyna się od lewej strony powyżej osi OX.\\
Ponadto w punkcie $-7$ wykres odbija się od osi poziomej.\\
A więc $$x \in \{-7\} \cup [8,14].$$
\rozwStop
\odpStart
$x \in \{-7\} \cup [8,14]$
\odpStop
\testStart
A.$x \in \{-7\} \cup [8,14]$\\
B.$x \in \{7\} \cup (8,14)$\\
C.$x \in \{-7\} \cup (8,14]$\\
D.$x \in \{7\} \cup (8,14]$\\
E.$x \in \{-7\} \cup [8,14)$\\
F.$x \in \{7\} \cup [8,14)$\\
G.$x \in \{-7\} \cup (8,14)$\\
H.$x \in \{7\} \cup [8,14]$
\testStop
\kluczStart
A
\kluczStop



\zadStart{Zadanie z Wikieł Z 1.62 c) moja wersja nr 387}

Rozwiązać nierówności $(8-x)(x+7)^{2}(15-x)^{3}\le0$.
\zadStop
\rozwStart{Patryk Wirkus}{}
Miejsca zerowe naszego wielomianu to: $8, -7, 15$.\\
Wielomian jest stopnia parzystego, ponadto znak współczynnika przy\linebreak najwyższej potędze x jest ujemny.\\ W związku z tym wykres wielomianu zaczyna się od lewej strony powyżej osi OX.\\
Ponadto w punkcie $-7$ wykres odbija się od osi poziomej.\\
A więc $$x \in \{-7\} \cup [8,15].$$
\rozwStop
\odpStart
$x \in \{-7\} \cup [8,15]$
\odpStop
\testStart
A.$x \in \{-7\} \cup [8,15]$\\
B.$x \in \{7\} \cup (8,15)$\\
C.$x \in \{-7\} \cup (8,15]$\\
D.$x \in \{7\} \cup (8,15]$\\
E.$x \in \{-7\} \cup [8,15)$\\
F.$x \in \{7\} \cup [8,15)$\\
G.$x \in \{-7\} \cup (8,15)$\\
H.$x \in \{7\} \cup [8,15]$
\testStop
\kluczStart
A
\kluczStop



\zadStart{Zadanie z Wikieł Z 1.62 c) moja wersja nr 388}

Rozwiązać nierówności $(8-x)(x+7)^{2}(16-x)^{3}\le0$.
\zadStop
\rozwStart{Patryk Wirkus}{}
Miejsca zerowe naszego wielomianu to: $8, -7, 16$.\\
Wielomian jest stopnia parzystego, ponadto znak współczynnika przy\linebreak najwyższej potędze x jest ujemny.\\ W związku z tym wykres wielomianu zaczyna się od lewej strony powyżej osi OX.\\
Ponadto w punkcie $-7$ wykres odbija się od osi poziomej.\\
A więc $$x \in \{-7\} \cup [8,16].$$
\rozwStop
\odpStart
$x \in \{-7\} \cup [8,16]$
\odpStop
\testStart
A.$x \in \{-7\} \cup [8,16]$\\
B.$x \in \{7\} \cup (8,16)$\\
C.$x \in \{-7\} \cup (8,16]$\\
D.$x \in \{7\} \cup (8,16]$\\
E.$x \in \{-7\} \cup [8,16)$\\
F.$x \in \{7\} \cup [8,16)$\\
G.$x \in \{-7\} \cup (8,16)$\\
H.$x \in \{7\} \cup [8,16]$
\testStop
\kluczStart
A
\kluczStop



\zadStart{Zadanie z Wikieł Z 1.62 c) moja wersja nr 389}

Rozwiązać nierówności $(8-x)(x+7)^{2}(17-x)^{3}\le0$.
\zadStop
\rozwStart{Patryk Wirkus}{}
Miejsca zerowe naszego wielomianu to: $8, -7, 17$.\\
Wielomian jest stopnia parzystego, ponadto znak współczynnika przy\linebreak najwyższej potędze x jest ujemny.\\ W związku z tym wykres wielomianu zaczyna się od lewej strony powyżej osi OX.\\
Ponadto w punkcie $-7$ wykres odbija się od osi poziomej.\\
A więc $$x \in \{-7\} \cup [8,17].$$
\rozwStop
\odpStart
$x \in \{-7\} \cup [8,17]$
\odpStop
\testStart
A.$x \in \{-7\} \cup [8,17]$\\
B.$x \in \{7\} \cup (8,17)$\\
C.$x \in \{-7\} \cup (8,17]$\\
D.$x \in \{7\} \cup (8,17]$\\
E.$x \in \{-7\} \cup [8,17)$\\
F.$x \in \{7\} \cup [8,17)$\\
G.$x \in \{-7\} \cup (8,17)$\\
H.$x \in \{7\} \cup [8,17]$
\testStop
\kluczStart
A
\kluczStop



\zadStart{Zadanie z Wikieł Z 1.62 c) moja wersja nr 390}

Rozwiązać nierówności $(8-x)(x+7)^{2}(18-x)^{3}\le0$.
\zadStop
\rozwStart{Patryk Wirkus}{}
Miejsca zerowe naszego wielomianu to: $8, -7, 18$.\\
Wielomian jest stopnia parzystego, ponadto znak współczynnika przy\linebreak najwyższej potędze x jest ujemny.\\ W związku z tym wykres wielomianu zaczyna się od lewej strony powyżej osi OX.\\
Ponadto w punkcie $-7$ wykres odbija się od osi poziomej.\\
A więc $$x \in \{-7\} \cup [8,18].$$
\rozwStop
\odpStart
$x \in \{-7\} \cup [8,18]$
\odpStop
\testStart
A.$x \in \{-7\} \cup [8,18]$\\
B.$x \in \{7\} \cup (8,18)$\\
C.$x \in \{-7\} \cup (8,18]$\\
D.$x \in \{7\} \cup (8,18]$\\
E.$x \in \{-7\} \cup [8,18)$\\
F.$x \in \{7\} \cup [8,18)$\\
G.$x \in \{-7\} \cup (8,18)$\\
H.$x \in \{7\} \cup [8,18]$
\testStop
\kluczStart
A
\kluczStop



\zadStart{Zadanie z Wikieł Z 1.62 c) moja wersja nr 391}

Rozwiązać nierówności $(8-x)(x+7)^{2}(19-x)^{3}\le0$.
\zadStop
\rozwStart{Patryk Wirkus}{}
Miejsca zerowe naszego wielomianu to: $8, -7, 19$.\\
Wielomian jest stopnia parzystego, ponadto znak współczynnika przy\linebreak najwyższej potędze x jest ujemny.\\ W związku z tym wykres wielomianu zaczyna się od lewej strony powyżej osi OX.\\
Ponadto w punkcie $-7$ wykres odbija się od osi poziomej.\\
A więc $$x \in \{-7\} \cup [8,19].$$
\rozwStop
\odpStart
$x \in \{-7\} \cup [8,19]$
\odpStop
\testStart
A.$x \in \{-7\} \cup [8,19]$\\
B.$x \in \{7\} \cup (8,19)$\\
C.$x \in \{-7\} \cup (8,19]$\\
D.$x \in \{7\} \cup (8,19]$\\
E.$x \in \{-7\} \cup [8,19)$\\
F.$x \in \{7\} \cup [8,19)$\\
G.$x \in \{-7\} \cup (8,19)$\\
H.$x \in \{7\} \cup [8,19]$
\testStop
\kluczStart
A
\kluczStop



\zadStart{Zadanie z Wikieł Z 1.62 c) moja wersja nr 392}

Rozwiązać nierówności $(8-x)(x+7)^{2}(20-x)^{3}\le0$.
\zadStop
\rozwStart{Patryk Wirkus}{}
Miejsca zerowe naszego wielomianu to: $8, -7, 20$.\\
Wielomian jest stopnia parzystego, ponadto znak współczynnika przy\linebreak najwyższej potędze x jest ujemny.\\ W związku z tym wykres wielomianu zaczyna się od lewej strony powyżej osi OX.\\
Ponadto w punkcie $-7$ wykres odbija się od osi poziomej.\\
A więc $$x \in \{-7\} \cup [8,20].$$
\rozwStop
\odpStart
$x \in \{-7\} \cup [8,20]$
\odpStop
\testStart
A.$x \in \{-7\} \cup [8,20]$\\
B.$x \in \{7\} \cup (8,20)$\\
C.$x \in \{-7\} \cup (8,20]$\\
D.$x \in \{7\} \cup (8,20]$\\
E.$x \in \{-7\} \cup [8,20)$\\
F.$x \in \{7\} \cup [8,20)$\\
G.$x \in \{-7\} \cup (8,20)$\\
H.$x \in \{7\} \cup [8,20]$
\testStop
\kluczStart
A
\kluczStop



\zadStart{Zadanie z Wikieł Z 1.62 c) moja wersja nr 393}

Rozwiązać nierówności $(9-x)(x+1)^{2}(10-x)^{3}\le0$.
\zadStop
\rozwStart{Patryk Wirkus}{}
Miejsca zerowe naszego wielomianu to: $9, -1, 10$.\\
Wielomian jest stopnia parzystego, ponadto znak współczynnika przy\linebreak najwyższej potędze x jest ujemny.\\ W związku z tym wykres wielomianu zaczyna się od lewej strony powyżej osi OX.\\
Ponadto w punkcie $-1$ wykres odbija się od osi poziomej.\\
A więc $$x \in \{-1\} \cup [9,10].$$
\rozwStop
\odpStart
$x \in \{-1\} \cup [9,10]$
\odpStop
\testStart
A.$x \in \{-1\} \cup [9,10]$\\
B.$x \in \{1\} \cup (9,10)$\\
C.$x \in \{-1\} \cup (9,10]$\\
D.$x \in \{1\} \cup (9,10]$\\
E.$x \in \{-1\} \cup [9,10)$\\
F.$x \in \{1\} \cup [9,10)$\\
G.$x \in \{-1\} \cup (9,10)$\\
H.$x \in \{1\} \cup [9,10]$
\testStop
\kluczStart
A
\kluczStop



\zadStart{Zadanie z Wikieł Z 1.62 c) moja wersja nr 394}

Rozwiązać nierówności $(9-x)(x+1)^{2}(11-x)^{3}\le0$.
\zadStop
\rozwStart{Patryk Wirkus}{}
Miejsca zerowe naszego wielomianu to: $9, -1, 11$.\\
Wielomian jest stopnia parzystego, ponadto znak współczynnika przy\linebreak najwyższej potędze x jest ujemny.\\ W związku z tym wykres wielomianu zaczyna się od lewej strony powyżej osi OX.\\
Ponadto w punkcie $-1$ wykres odbija się od osi poziomej.\\
A więc $$x \in \{-1\} \cup [9,11].$$
\rozwStop
\odpStart
$x \in \{-1\} \cup [9,11]$
\odpStop
\testStart
A.$x \in \{-1\} \cup [9,11]$\\
B.$x \in \{1\} \cup (9,11)$\\
C.$x \in \{-1\} \cup (9,11]$\\
D.$x \in \{1\} \cup (9,11]$\\
E.$x \in \{-1\} \cup [9,11)$\\
F.$x \in \{1\} \cup [9,11)$\\
G.$x \in \{-1\} \cup (9,11)$\\
H.$x \in \{1\} \cup [9,11]$
\testStop
\kluczStart
A
\kluczStop



\zadStart{Zadanie z Wikieł Z 1.62 c) moja wersja nr 395}

Rozwiązać nierówności $(9-x)(x+1)^{2}(12-x)^{3}\le0$.
\zadStop
\rozwStart{Patryk Wirkus}{}
Miejsca zerowe naszego wielomianu to: $9, -1, 12$.\\
Wielomian jest stopnia parzystego, ponadto znak współczynnika przy\linebreak najwyższej potędze x jest ujemny.\\ W związku z tym wykres wielomianu zaczyna się od lewej strony powyżej osi OX.\\
Ponadto w punkcie $-1$ wykres odbija się od osi poziomej.\\
A więc $$x \in \{-1\} \cup [9,12].$$
\rozwStop
\odpStart
$x \in \{-1\} \cup [9,12]$
\odpStop
\testStart
A.$x \in \{-1\} \cup [9,12]$\\
B.$x \in \{1\} \cup (9,12)$\\
C.$x \in \{-1\} \cup (9,12]$\\
D.$x \in \{1\} \cup (9,12]$\\
E.$x \in \{-1\} \cup [9,12)$\\
F.$x \in \{1\} \cup [9,12)$\\
G.$x \in \{-1\} \cup (9,12)$\\
H.$x \in \{1\} \cup [9,12]$
\testStop
\kluczStart
A
\kluczStop



\zadStart{Zadanie z Wikieł Z 1.62 c) moja wersja nr 396}

Rozwiązać nierówności $(9-x)(x+1)^{2}(13-x)^{3}\le0$.
\zadStop
\rozwStart{Patryk Wirkus}{}
Miejsca zerowe naszego wielomianu to: $9, -1, 13$.\\
Wielomian jest stopnia parzystego, ponadto znak współczynnika przy\linebreak najwyższej potędze x jest ujemny.\\ W związku z tym wykres wielomianu zaczyna się od lewej strony powyżej osi OX.\\
Ponadto w punkcie $-1$ wykres odbija się od osi poziomej.\\
A więc $$x \in \{-1\} \cup [9,13].$$
\rozwStop
\odpStart
$x \in \{-1\} \cup [9,13]$
\odpStop
\testStart
A.$x \in \{-1\} \cup [9,13]$\\
B.$x \in \{1\} \cup (9,13)$\\
C.$x \in \{-1\} \cup (9,13]$\\
D.$x \in \{1\} \cup (9,13]$\\
E.$x \in \{-1\} \cup [9,13)$\\
F.$x \in \{1\} \cup [9,13)$\\
G.$x \in \{-1\} \cup (9,13)$\\
H.$x \in \{1\} \cup [9,13]$
\testStop
\kluczStart
A
\kluczStop



\zadStart{Zadanie z Wikieł Z 1.62 c) moja wersja nr 397}

Rozwiązać nierówności $(9-x)(x+1)^{2}(14-x)^{3}\le0$.
\zadStop
\rozwStart{Patryk Wirkus}{}
Miejsca zerowe naszego wielomianu to: $9, -1, 14$.\\
Wielomian jest stopnia parzystego, ponadto znak współczynnika przy\linebreak najwyższej potędze x jest ujemny.\\ W związku z tym wykres wielomianu zaczyna się od lewej strony powyżej osi OX.\\
Ponadto w punkcie $-1$ wykres odbija się od osi poziomej.\\
A więc $$x \in \{-1\} \cup [9,14].$$
\rozwStop
\odpStart
$x \in \{-1\} \cup [9,14]$
\odpStop
\testStart
A.$x \in \{-1\} \cup [9,14]$\\
B.$x \in \{1\} \cup (9,14)$\\
C.$x \in \{-1\} \cup (9,14]$\\
D.$x \in \{1\} \cup (9,14]$\\
E.$x \in \{-1\} \cup [9,14)$\\
F.$x \in \{1\} \cup [9,14)$\\
G.$x \in \{-1\} \cup (9,14)$\\
H.$x \in \{1\} \cup [9,14]$
\testStop
\kluczStart
A
\kluczStop



\zadStart{Zadanie z Wikieł Z 1.62 c) moja wersja nr 398}

Rozwiązać nierówności $(9-x)(x+1)^{2}(15-x)^{3}\le0$.
\zadStop
\rozwStart{Patryk Wirkus}{}
Miejsca zerowe naszego wielomianu to: $9, -1, 15$.\\
Wielomian jest stopnia parzystego, ponadto znak współczynnika przy\linebreak najwyższej potędze x jest ujemny.\\ W związku z tym wykres wielomianu zaczyna się od lewej strony powyżej osi OX.\\
Ponadto w punkcie $-1$ wykres odbija się od osi poziomej.\\
A więc $$x \in \{-1\} \cup [9,15].$$
\rozwStop
\odpStart
$x \in \{-1\} \cup [9,15]$
\odpStop
\testStart
A.$x \in \{-1\} \cup [9,15]$\\
B.$x \in \{1\} \cup (9,15)$\\
C.$x \in \{-1\} \cup (9,15]$\\
D.$x \in \{1\} \cup (9,15]$\\
E.$x \in \{-1\} \cup [9,15)$\\
F.$x \in \{1\} \cup [9,15)$\\
G.$x \in \{-1\} \cup (9,15)$\\
H.$x \in \{1\} \cup [9,15]$
\testStop
\kluczStart
A
\kluczStop



\zadStart{Zadanie z Wikieł Z 1.62 c) moja wersja nr 399}

Rozwiązać nierówności $(9-x)(x+1)^{2}(16-x)^{3}\le0$.
\zadStop
\rozwStart{Patryk Wirkus}{}
Miejsca zerowe naszego wielomianu to: $9, -1, 16$.\\
Wielomian jest stopnia parzystego, ponadto znak współczynnika przy\linebreak najwyższej potędze x jest ujemny.\\ W związku z tym wykres wielomianu zaczyna się od lewej strony powyżej osi OX.\\
Ponadto w punkcie $-1$ wykres odbija się od osi poziomej.\\
A więc $$x \in \{-1\} \cup [9,16].$$
\rozwStop
\odpStart
$x \in \{-1\} \cup [9,16]$
\odpStop
\testStart
A.$x \in \{-1\} \cup [9,16]$\\
B.$x \in \{1\} \cup (9,16)$\\
C.$x \in \{-1\} \cup (9,16]$\\
D.$x \in \{1\} \cup (9,16]$\\
E.$x \in \{-1\} \cup [9,16)$\\
F.$x \in \{1\} \cup [9,16)$\\
G.$x \in \{-1\} \cup (9,16)$\\
H.$x \in \{1\} \cup [9,16]$
\testStop
\kluczStart
A
\kluczStop



\zadStart{Zadanie z Wikieł Z 1.62 c) moja wersja nr 400}

Rozwiązać nierówności $(9-x)(x+1)^{2}(17-x)^{3}\le0$.
\zadStop
\rozwStart{Patryk Wirkus}{}
Miejsca zerowe naszego wielomianu to: $9, -1, 17$.\\
Wielomian jest stopnia parzystego, ponadto znak współczynnika przy\linebreak najwyższej potędze x jest ujemny.\\ W związku z tym wykres wielomianu zaczyna się od lewej strony powyżej osi OX.\\
Ponadto w punkcie $-1$ wykres odbija się od osi poziomej.\\
A więc $$x \in \{-1\} \cup [9,17].$$
\rozwStop
\odpStart
$x \in \{-1\} \cup [9,17]$
\odpStop
\testStart
A.$x \in \{-1\} \cup [9,17]$\\
B.$x \in \{1\} \cup (9,17)$\\
C.$x \in \{-1\} \cup (9,17]$\\
D.$x \in \{1\} \cup (9,17]$\\
E.$x \in \{-1\} \cup [9,17)$\\
F.$x \in \{1\} \cup [9,17)$\\
G.$x \in \{-1\} \cup (9,17)$\\
H.$x \in \{1\} \cup [9,17]$
\testStop
\kluczStart
A
\kluczStop



\zadStart{Zadanie z Wikieł Z 1.62 c) moja wersja nr 401}

Rozwiązać nierówności $(9-x)(x+1)^{2}(18-x)^{3}\le0$.
\zadStop
\rozwStart{Patryk Wirkus}{}
Miejsca zerowe naszego wielomianu to: $9, -1, 18$.\\
Wielomian jest stopnia parzystego, ponadto znak współczynnika przy\linebreak najwyższej potędze x jest ujemny.\\ W związku z tym wykres wielomianu zaczyna się od lewej strony powyżej osi OX.\\
Ponadto w punkcie $-1$ wykres odbija się od osi poziomej.\\
A więc $$x \in \{-1\} \cup [9,18].$$
\rozwStop
\odpStart
$x \in \{-1\} \cup [9,18]$
\odpStop
\testStart
A.$x \in \{-1\} \cup [9,18]$\\
B.$x \in \{1\} \cup (9,18)$\\
C.$x \in \{-1\} \cup (9,18]$\\
D.$x \in \{1\} \cup (9,18]$\\
E.$x \in \{-1\} \cup [9,18)$\\
F.$x \in \{1\} \cup [9,18)$\\
G.$x \in \{-1\} \cup (9,18)$\\
H.$x \in \{1\} \cup [9,18]$
\testStop
\kluczStart
A
\kluczStop



\zadStart{Zadanie z Wikieł Z 1.62 c) moja wersja nr 402}

Rozwiązać nierówności $(9-x)(x+1)^{2}(19-x)^{3}\le0$.
\zadStop
\rozwStart{Patryk Wirkus}{}
Miejsca zerowe naszego wielomianu to: $9, -1, 19$.\\
Wielomian jest stopnia parzystego, ponadto znak współczynnika przy\linebreak najwyższej potędze x jest ujemny.\\ W związku z tym wykres wielomianu zaczyna się od lewej strony powyżej osi OX.\\
Ponadto w punkcie $-1$ wykres odbija się od osi poziomej.\\
A więc $$x \in \{-1\} \cup [9,19].$$
\rozwStop
\odpStart
$x \in \{-1\} \cup [9,19]$
\odpStop
\testStart
A.$x \in \{-1\} \cup [9,19]$\\
B.$x \in \{1\} \cup (9,19)$\\
C.$x \in \{-1\} \cup (9,19]$\\
D.$x \in \{1\} \cup (9,19]$\\
E.$x \in \{-1\} \cup [9,19)$\\
F.$x \in \{1\} \cup [9,19)$\\
G.$x \in \{-1\} \cup (9,19)$\\
H.$x \in \{1\} \cup [9,19]$
\testStop
\kluczStart
A
\kluczStop



\zadStart{Zadanie z Wikieł Z 1.62 c) moja wersja nr 403}

Rozwiązać nierówności $(9-x)(x+1)^{2}(20-x)^{3}\le0$.
\zadStop
\rozwStart{Patryk Wirkus}{}
Miejsca zerowe naszego wielomianu to: $9, -1, 20$.\\
Wielomian jest stopnia parzystego, ponadto znak współczynnika przy\linebreak najwyższej potędze x jest ujemny.\\ W związku z tym wykres wielomianu zaczyna się od lewej strony powyżej osi OX.\\
Ponadto w punkcie $-1$ wykres odbija się od osi poziomej.\\
A więc $$x \in \{-1\} \cup [9,20].$$
\rozwStop
\odpStart
$x \in \{-1\} \cup [9,20]$
\odpStop
\testStart
A.$x \in \{-1\} \cup [9,20]$\\
B.$x \in \{1\} \cup (9,20)$\\
C.$x \in \{-1\} \cup (9,20]$\\
D.$x \in \{1\} \cup (9,20]$\\
E.$x \in \{-1\} \cup [9,20)$\\
F.$x \in \{1\} \cup [9,20)$\\
G.$x \in \{-1\} \cup (9,20)$\\
H.$x \in \{1\} \cup [9,20]$
\testStop
\kluczStart
A
\kluczStop



\zadStart{Zadanie z Wikieł Z 1.62 c) moja wersja nr 404}

Rozwiązać nierówności $(9-x)(x+2)^{2}(10-x)^{3}\le0$.
\zadStop
\rozwStart{Patryk Wirkus}{}
Miejsca zerowe naszego wielomianu to: $9, -2, 10$.\\
Wielomian jest stopnia parzystego, ponadto znak współczynnika przy\linebreak najwyższej potędze x jest ujemny.\\ W związku z tym wykres wielomianu zaczyna się od lewej strony powyżej osi OX.\\
Ponadto w punkcie $-2$ wykres odbija się od osi poziomej.\\
A więc $$x \in \{-2\} \cup [9,10].$$
\rozwStop
\odpStart
$x \in \{-2\} \cup [9,10]$
\odpStop
\testStart
A.$x \in \{-2\} \cup [9,10]$\\
B.$x \in \{2\} \cup (9,10)$\\
C.$x \in \{-2\} \cup (9,10]$\\
D.$x \in \{2\} \cup (9,10]$\\
E.$x \in \{-2\} \cup [9,10)$\\
F.$x \in \{2\} \cup [9,10)$\\
G.$x \in \{-2\} \cup (9,10)$\\
H.$x \in \{2\} \cup [9,10]$
\testStop
\kluczStart
A
\kluczStop



\zadStart{Zadanie z Wikieł Z 1.62 c) moja wersja nr 405}

Rozwiązać nierówności $(9-x)(x+2)^{2}(11-x)^{3}\le0$.
\zadStop
\rozwStart{Patryk Wirkus}{}
Miejsca zerowe naszego wielomianu to: $9, -2, 11$.\\
Wielomian jest stopnia parzystego, ponadto znak współczynnika przy\linebreak najwyższej potędze x jest ujemny.\\ W związku z tym wykres wielomianu zaczyna się od lewej strony powyżej osi OX.\\
Ponadto w punkcie $-2$ wykres odbija się od osi poziomej.\\
A więc $$x \in \{-2\} \cup [9,11].$$
\rozwStop
\odpStart
$x \in \{-2\} \cup [9,11]$
\odpStop
\testStart
A.$x \in \{-2\} \cup [9,11]$\\
B.$x \in \{2\} \cup (9,11)$\\
C.$x \in \{-2\} \cup (9,11]$\\
D.$x \in \{2\} \cup (9,11]$\\
E.$x \in \{-2\} \cup [9,11)$\\
F.$x \in \{2\} \cup [9,11)$\\
G.$x \in \{-2\} \cup (9,11)$\\
H.$x \in \{2\} \cup [9,11]$
\testStop
\kluczStart
A
\kluczStop



\zadStart{Zadanie z Wikieł Z 1.62 c) moja wersja nr 406}

Rozwiązać nierówności $(9-x)(x+2)^{2}(12-x)^{3}\le0$.
\zadStop
\rozwStart{Patryk Wirkus}{}
Miejsca zerowe naszego wielomianu to: $9, -2, 12$.\\
Wielomian jest stopnia parzystego, ponadto znak współczynnika przy\linebreak najwyższej potędze x jest ujemny.\\ W związku z tym wykres wielomianu zaczyna się od lewej strony powyżej osi OX.\\
Ponadto w punkcie $-2$ wykres odbija się od osi poziomej.\\
A więc $$x \in \{-2\} \cup [9,12].$$
\rozwStop
\odpStart
$x \in \{-2\} \cup [9,12]$
\odpStop
\testStart
A.$x \in \{-2\} \cup [9,12]$\\
B.$x \in \{2\} \cup (9,12)$\\
C.$x \in \{-2\} \cup (9,12]$\\
D.$x \in \{2\} \cup (9,12]$\\
E.$x \in \{-2\} \cup [9,12)$\\
F.$x \in \{2\} \cup [9,12)$\\
G.$x \in \{-2\} \cup (9,12)$\\
H.$x \in \{2\} \cup [9,12]$
\testStop
\kluczStart
A
\kluczStop



\zadStart{Zadanie z Wikieł Z 1.62 c) moja wersja nr 407}

Rozwiązać nierówności $(9-x)(x+2)^{2}(13-x)^{3}\le0$.
\zadStop
\rozwStart{Patryk Wirkus}{}
Miejsca zerowe naszego wielomianu to: $9, -2, 13$.\\
Wielomian jest stopnia parzystego, ponadto znak współczynnika przy\linebreak najwyższej potędze x jest ujemny.\\ W związku z tym wykres wielomianu zaczyna się od lewej strony powyżej osi OX.\\
Ponadto w punkcie $-2$ wykres odbija się od osi poziomej.\\
A więc $$x \in \{-2\} \cup [9,13].$$
\rozwStop
\odpStart
$x \in \{-2\} \cup [9,13]$
\odpStop
\testStart
A.$x \in \{-2\} \cup [9,13]$\\
B.$x \in \{2\} \cup (9,13)$\\
C.$x \in \{-2\} \cup (9,13]$\\
D.$x \in \{2\} \cup (9,13]$\\
E.$x \in \{-2\} \cup [9,13)$\\
F.$x \in \{2\} \cup [9,13)$\\
G.$x \in \{-2\} \cup (9,13)$\\
H.$x \in \{2\} \cup [9,13]$
\testStop
\kluczStart
A
\kluczStop



\zadStart{Zadanie z Wikieł Z 1.62 c) moja wersja nr 408}

Rozwiązać nierówności $(9-x)(x+2)^{2}(14-x)^{3}\le0$.
\zadStop
\rozwStart{Patryk Wirkus}{}
Miejsca zerowe naszego wielomianu to: $9, -2, 14$.\\
Wielomian jest stopnia parzystego, ponadto znak współczynnika przy\linebreak najwyższej potędze x jest ujemny.\\ W związku z tym wykres wielomianu zaczyna się od lewej strony powyżej osi OX.\\
Ponadto w punkcie $-2$ wykres odbija się od osi poziomej.\\
A więc $$x \in \{-2\} \cup [9,14].$$
\rozwStop
\odpStart
$x \in \{-2\} \cup [9,14]$
\odpStop
\testStart
A.$x \in \{-2\} \cup [9,14]$\\
B.$x \in \{2\} \cup (9,14)$\\
C.$x \in \{-2\} \cup (9,14]$\\
D.$x \in \{2\} \cup (9,14]$\\
E.$x \in \{-2\} \cup [9,14)$\\
F.$x \in \{2\} \cup [9,14)$\\
G.$x \in \{-2\} \cup (9,14)$\\
H.$x \in \{2\} \cup [9,14]$
\testStop
\kluczStart
A
\kluczStop



\zadStart{Zadanie z Wikieł Z 1.62 c) moja wersja nr 409}

Rozwiązać nierówności $(9-x)(x+2)^{2}(15-x)^{3}\le0$.
\zadStop
\rozwStart{Patryk Wirkus}{}
Miejsca zerowe naszego wielomianu to: $9, -2, 15$.\\
Wielomian jest stopnia parzystego, ponadto znak współczynnika przy\linebreak najwyższej potędze x jest ujemny.\\ W związku z tym wykres wielomianu zaczyna się od lewej strony powyżej osi OX.\\
Ponadto w punkcie $-2$ wykres odbija się od osi poziomej.\\
A więc $$x \in \{-2\} \cup [9,15].$$
\rozwStop
\odpStart
$x \in \{-2\} \cup [9,15]$
\odpStop
\testStart
A.$x \in \{-2\} \cup [9,15]$\\
B.$x \in \{2\} \cup (9,15)$\\
C.$x \in \{-2\} \cup (9,15]$\\
D.$x \in \{2\} \cup (9,15]$\\
E.$x \in \{-2\} \cup [9,15)$\\
F.$x \in \{2\} \cup [9,15)$\\
G.$x \in \{-2\} \cup (9,15)$\\
H.$x \in \{2\} \cup [9,15]$
\testStop
\kluczStart
A
\kluczStop



\zadStart{Zadanie z Wikieł Z 1.62 c) moja wersja nr 410}

Rozwiązać nierówności $(9-x)(x+2)^{2}(16-x)^{3}\le0$.
\zadStop
\rozwStart{Patryk Wirkus}{}
Miejsca zerowe naszego wielomianu to: $9, -2, 16$.\\
Wielomian jest stopnia parzystego, ponadto znak współczynnika przy\linebreak najwyższej potędze x jest ujemny.\\ W związku z tym wykres wielomianu zaczyna się od lewej strony powyżej osi OX.\\
Ponadto w punkcie $-2$ wykres odbija się od osi poziomej.\\
A więc $$x \in \{-2\} \cup [9,16].$$
\rozwStop
\odpStart
$x \in \{-2\} \cup [9,16]$
\odpStop
\testStart
A.$x \in \{-2\} \cup [9,16]$\\
B.$x \in \{2\} \cup (9,16)$\\
C.$x \in \{-2\} \cup (9,16]$\\
D.$x \in \{2\} \cup (9,16]$\\
E.$x \in \{-2\} \cup [9,16)$\\
F.$x \in \{2\} \cup [9,16)$\\
G.$x \in \{-2\} \cup (9,16)$\\
H.$x \in \{2\} \cup [9,16]$
\testStop
\kluczStart
A
\kluczStop



\zadStart{Zadanie z Wikieł Z 1.62 c) moja wersja nr 411}

Rozwiązać nierówności $(9-x)(x+2)^{2}(17-x)^{3}\le0$.
\zadStop
\rozwStart{Patryk Wirkus}{}
Miejsca zerowe naszego wielomianu to: $9, -2, 17$.\\
Wielomian jest stopnia parzystego, ponadto znak współczynnika przy\linebreak najwyższej potędze x jest ujemny.\\ W związku z tym wykres wielomianu zaczyna się od lewej strony powyżej osi OX.\\
Ponadto w punkcie $-2$ wykres odbija się od osi poziomej.\\
A więc $$x \in \{-2\} \cup [9,17].$$
\rozwStop
\odpStart
$x \in \{-2\} \cup [9,17]$
\odpStop
\testStart
A.$x \in \{-2\} \cup [9,17]$\\
B.$x \in \{2\} \cup (9,17)$\\
C.$x \in \{-2\} \cup (9,17]$\\
D.$x \in \{2\} \cup (9,17]$\\
E.$x \in \{-2\} \cup [9,17)$\\
F.$x \in \{2\} \cup [9,17)$\\
G.$x \in \{-2\} \cup (9,17)$\\
H.$x \in \{2\} \cup [9,17]$
\testStop
\kluczStart
A
\kluczStop



\zadStart{Zadanie z Wikieł Z 1.62 c) moja wersja nr 412}

Rozwiązać nierówności $(9-x)(x+2)^{2}(18-x)^{3}\le0$.
\zadStop
\rozwStart{Patryk Wirkus}{}
Miejsca zerowe naszego wielomianu to: $9, -2, 18$.\\
Wielomian jest stopnia parzystego, ponadto znak współczynnika przy\linebreak najwyższej potędze x jest ujemny.\\ W związku z tym wykres wielomianu zaczyna się od lewej strony powyżej osi OX.\\
Ponadto w punkcie $-2$ wykres odbija się od osi poziomej.\\
A więc $$x \in \{-2\} \cup [9,18].$$
\rozwStop
\odpStart
$x \in \{-2\} \cup [9,18]$
\odpStop
\testStart
A.$x \in \{-2\} \cup [9,18]$\\
B.$x \in \{2\} \cup (9,18)$\\
C.$x \in \{-2\} \cup (9,18]$\\
D.$x \in \{2\} \cup (9,18]$\\
E.$x \in \{-2\} \cup [9,18)$\\
F.$x \in \{2\} \cup [9,18)$\\
G.$x \in \{-2\} \cup (9,18)$\\
H.$x \in \{2\} \cup [9,18]$
\testStop
\kluczStart
A
\kluczStop



\zadStart{Zadanie z Wikieł Z 1.62 c) moja wersja nr 413}

Rozwiązać nierówności $(9-x)(x+2)^{2}(19-x)^{3}\le0$.
\zadStop
\rozwStart{Patryk Wirkus}{}
Miejsca zerowe naszego wielomianu to: $9, -2, 19$.\\
Wielomian jest stopnia parzystego, ponadto znak współczynnika przy\linebreak najwyższej potędze x jest ujemny.\\ W związku z tym wykres wielomianu zaczyna się od lewej strony powyżej osi OX.\\
Ponadto w punkcie $-2$ wykres odbija się od osi poziomej.\\
A więc $$x \in \{-2\} \cup [9,19].$$
\rozwStop
\odpStart
$x \in \{-2\} \cup [9,19]$
\odpStop
\testStart
A.$x \in \{-2\} \cup [9,19]$\\
B.$x \in \{2\} \cup (9,19)$\\
C.$x \in \{-2\} \cup (9,19]$\\
D.$x \in \{2\} \cup (9,19]$\\
E.$x \in \{-2\} \cup [9,19)$\\
F.$x \in \{2\} \cup [9,19)$\\
G.$x \in \{-2\} \cup (9,19)$\\
H.$x \in \{2\} \cup [9,19]$
\testStop
\kluczStart
A
\kluczStop



\zadStart{Zadanie z Wikieł Z 1.62 c) moja wersja nr 414}

Rozwiązać nierówności $(9-x)(x+2)^{2}(20-x)^{3}\le0$.
\zadStop
\rozwStart{Patryk Wirkus}{}
Miejsca zerowe naszego wielomianu to: $9, -2, 20$.\\
Wielomian jest stopnia parzystego, ponadto znak współczynnika przy\linebreak najwyższej potędze x jest ujemny.\\ W związku z tym wykres wielomianu zaczyna się od lewej strony powyżej osi OX.\\
Ponadto w punkcie $-2$ wykres odbija się od osi poziomej.\\
A więc $$x \in \{-2\} \cup [9,20].$$
\rozwStop
\odpStart
$x \in \{-2\} \cup [9,20]$
\odpStop
\testStart
A.$x \in \{-2\} \cup [9,20]$\\
B.$x \in \{2\} \cup (9,20)$\\
C.$x \in \{-2\} \cup (9,20]$\\
D.$x \in \{2\} \cup (9,20]$\\
E.$x \in \{-2\} \cup [9,20)$\\
F.$x \in \{2\} \cup [9,20)$\\
G.$x \in \{-2\} \cup (9,20)$\\
H.$x \in \{2\} \cup [9,20]$
\testStop
\kluczStart
A
\kluczStop



\zadStart{Zadanie z Wikieł Z 1.62 c) moja wersja nr 415}

Rozwiązać nierówności $(9-x)(x+3)^{2}(10-x)^{3}\le0$.
\zadStop
\rozwStart{Patryk Wirkus}{}
Miejsca zerowe naszego wielomianu to: $9, -3, 10$.\\
Wielomian jest stopnia parzystego, ponadto znak współczynnika przy\linebreak najwyższej potędze x jest ujemny.\\ W związku z tym wykres wielomianu zaczyna się od lewej strony powyżej osi OX.\\
Ponadto w punkcie $-3$ wykres odbija się od osi poziomej.\\
A więc $$x \in \{-3\} \cup [9,10].$$
\rozwStop
\odpStart
$x \in \{-3\} \cup [9,10]$
\odpStop
\testStart
A.$x \in \{-3\} \cup [9,10]$\\
B.$x \in \{3\} \cup (9,10)$\\
C.$x \in \{-3\} \cup (9,10]$\\
D.$x \in \{3\} \cup (9,10]$\\
E.$x \in \{-3\} \cup [9,10)$\\
F.$x \in \{3\} \cup [9,10)$\\
G.$x \in \{-3\} \cup (9,10)$\\
H.$x \in \{3\} \cup [9,10]$
\testStop
\kluczStart
A
\kluczStop



\zadStart{Zadanie z Wikieł Z 1.62 c) moja wersja nr 416}

Rozwiązać nierówności $(9-x)(x+3)^{2}(11-x)^{3}\le0$.
\zadStop
\rozwStart{Patryk Wirkus}{}
Miejsca zerowe naszego wielomianu to: $9, -3, 11$.\\
Wielomian jest stopnia parzystego, ponadto znak współczynnika przy\linebreak najwyższej potędze x jest ujemny.\\ W związku z tym wykres wielomianu zaczyna się od lewej strony powyżej osi OX.\\
Ponadto w punkcie $-3$ wykres odbija się od osi poziomej.\\
A więc $$x \in \{-3\} \cup [9,11].$$
\rozwStop
\odpStart
$x \in \{-3\} \cup [9,11]$
\odpStop
\testStart
A.$x \in \{-3\} \cup [9,11]$\\
B.$x \in \{3\} \cup (9,11)$\\
C.$x \in \{-3\} \cup (9,11]$\\
D.$x \in \{3\} \cup (9,11]$\\
E.$x \in \{-3\} \cup [9,11)$\\
F.$x \in \{3\} \cup [9,11)$\\
G.$x \in \{-3\} \cup (9,11)$\\
H.$x \in \{3\} \cup [9,11]$
\testStop
\kluczStart
A
\kluczStop



\zadStart{Zadanie z Wikieł Z 1.62 c) moja wersja nr 417}

Rozwiązać nierówności $(9-x)(x+3)^{2}(12-x)^{3}\le0$.
\zadStop
\rozwStart{Patryk Wirkus}{}
Miejsca zerowe naszego wielomianu to: $9, -3, 12$.\\
Wielomian jest stopnia parzystego, ponadto znak współczynnika przy\linebreak najwyższej potędze x jest ujemny.\\ W związku z tym wykres wielomianu zaczyna się od lewej strony powyżej osi OX.\\
Ponadto w punkcie $-3$ wykres odbija się od osi poziomej.\\
A więc $$x \in \{-3\} \cup [9,12].$$
\rozwStop
\odpStart
$x \in \{-3\} \cup [9,12]$
\odpStop
\testStart
A.$x \in \{-3\} \cup [9,12]$\\
B.$x \in \{3\} \cup (9,12)$\\
C.$x \in \{-3\} \cup (9,12]$\\
D.$x \in \{3\} \cup (9,12]$\\
E.$x \in \{-3\} \cup [9,12)$\\
F.$x \in \{3\} \cup [9,12)$\\
G.$x \in \{-3\} \cup (9,12)$\\
H.$x \in \{3\} \cup [9,12]$
\testStop
\kluczStart
A
\kluczStop



\zadStart{Zadanie z Wikieł Z 1.62 c) moja wersja nr 418}

Rozwiązać nierówności $(9-x)(x+3)^{2}(13-x)^{3}\le0$.
\zadStop
\rozwStart{Patryk Wirkus}{}
Miejsca zerowe naszego wielomianu to: $9, -3, 13$.\\
Wielomian jest stopnia parzystego, ponadto znak współczynnika przy\linebreak najwyższej potędze x jest ujemny.\\ W związku z tym wykres wielomianu zaczyna się od lewej strony powyżej osi OX.\\
Ponadto w punkcie $-3$ wykres odbija się od osi poziomej.\\
A więc $$x \in \{-3\} \cup [9,13].$$
\rozwStop
\odpStart
$x \in \{-3\} \cup [9,13]$
\odpStop
\testStart
A.$x \in \{-3\} \cup [9,13]$\\
B.$x \in \{3\} \cup (9,13)$\\
C.$x \in \{-3\} \cup (9,13]$\\
D.$x \in \{3\} \cup (9,13]$\\
E.$x \in \{-3\} \cup [9,13)$\\
F.$x \in \{3\} \cup [9,13)$\\
G.$x \in \{-3\} \cup (9,13)$\\
H.$x \in \{3\} \cup [9,13]$
\testStop
\kluczStart
A
\kluczStop



\zadStart{Zadanie z Wikieł Z 1.62 c) moja wersja nr 419}

Rozwiązać nierówności $(9-x)(x+3)^{2}(14-x)^{3}\le0$.
\zadStop
\rozwStart{Patryk Wirkus}{}
Miejsca zerowe naszego wielomianu to: $9, -3, 14$.\\
Wielomian jest stopnia parzystego, ponadto znak współczynnika przy\linebreak najwyższej potędze x jest ujemny.\\ W związku z tym wykres wielomianu zaczyna się od lewej strony powyżej osi OX.\\
Ponadto w punkcie $-3$ wykres odbija się od osi poziomej.\\
A więc $$x \in \{-3\} \cup [9,14].$$
\rozwStop
\odpStart
$x \in \{-3\} \cup [9,14]$
\odpStop
\testStart
A.$x \in \{-3\} \cup [9,14]$\\
B.$x \in \{3\} \cup (9,14)$\\
C.$x \in \{-3\} \cup (9,14]$\\
D.$x \in \{3\} \cup (9,14]$\\
E.$x \in \{-3\} \cup [9,14)$\\
F.$x \in \{3\} \cup [9,14)$\\
G.$x \in \{-3\} \cup (9,14)$\\
H.$x \in \{3\} \cup [9,14]$
\testStop
\kluczStart
A
\kluczStop



\zadStart{Zadanie z Wikieł Z 1.62 c) moja wersja nr 420}

Rozwiązać nierówności $(9-x)(x+3)^{2}(15-x)^{3}\le0$.
\zadStop
\rozwStart{Patryk Wirkus}{}
Miejsca zerowe naszego wielomianu to: $9, -3, 15$.\\
Wielomian jest stopnia parzystego, ponadto znak współczynnika przy\linebreak najwyższej potędze x jest ujemny.\\ W związku z tym wykres wielomianu zaczyna się od lewej strony powyżej osi OX.\\
Ponadto w punkcie $-3$ wykres odbija się od osi poziomej.\\
A więc $$x \in \{-3\} \cup [9,15].$$
\rozwStop
\odpStart
$x \in \{-3\} \cup [9,15]$
\odpStop
\testStart
A.$x \in \{-3\} \cup [9,15]$\\
B.$x \in \{3\} \cup (9,15)$\\
C.$x \in \{-3\} \cup (9,15]$\\
D.$x \in \{3\} \cup (9,15]$\\
E.$x \in \{-3\} \cup [9,15)$\\
F.$x \in \{3\} \cup [9,15)$\\
G.$x \in \{-3\} \cup (9,15)$\\
H.$x \in \{3\} \cup [9,15]$
\testStop
\kluczStart
A
\kluczStop



\zadStart{Zadanie z Wikieł Z 1.62 c) moja wersja nr 421}

Rozwiązać nierówności $(9-x)(x+3)^{2}(16-x)^{3}\le0$.
\zadStop
\rozwStart{Patryk Wirkus}{}
Miejsca zerowe naszego wielomianu to: $9, -3, 16$.\\
Wielomian jest stopnia parzystego, ponadto znak współczynnika przy\linebreak najwyższej potędze x jest ujemny.\\ W związku z tym wykres wielomianu zaczyna się od lewej strony powyżej osi OX.\\
Ponadto w punkcie $-3$ wykres odbija się od osi poziomej.\\
A więc $$x \in \{-3\} \cup [9,16].$$
\rozwStop
\odpStart
$x \in \{-3\} \cup [9,16]$
\odpStop
\testStart
A.$x \in \{-3\} \cup [9,16]$\\
B.$x \in \{3\} \cup (9,16)$\\
C.$x \in \{-3\} \cup (9,16]$\\
D.$x \in \{3\} \cup (9,16]$\\
E.$x \in \{-3\} \cup [9,16)$\\
F.$x \in \{3\} \cup [9,16)$\\
G.$x \in \{-3\} \cup (9,16)$\\
H.$x \in \{3\} \cup [9,16]$
\testStop
\kluczStart
A
\kluczStop



\zadStart{Zadanie z Wikieł Z 1.62 c) moja wersja nr 422}

Rozwiązać nierówności $(9-x)(x+3)^{2}(17-x)^{3}\le0$.
\zadStop
\rozwStart{Patryk Wirkus}{}
Miejsca zerowe naszego wielomianu to: $9, -3, 17$.\\
Wielomian jest stopnia parzystego, ponadto znak współczynnika przy\linebreak najwyższej potędze x jest ujemny.\\ W związku z tym wykres wielomianu zaczyna się od lewej strony powyżej osi OX.\\
Ponadto w punkcie $-3$ wykres odbija się od osi poziomej.\\
A więc $$x \in \{-3\} \cup [9,17].$$
\rozwStop
\odpStart
$x \in \{-3\} \cup [9,17]$
\odpStop
\testStart
A.$x \in \{-3\} \cup [9,17]$\\
B.$x \in \{3\} \cup (9,17)$\\
C.$x \in \{-3\} \cup (9,17]$\\
D.$x \in \{3\} \cup (9,17]$\\
E.$x \in \{-3\} \cup [9,17)$\\
F.$x \in \{3\} \cup [9,17)$\\
G.$x \in \{-3\} \cup (9,17)$\\
H.$x \in \{3\} \cup [9,17]$
\testStop
\kluczStart
A
\kluczStop



\zadStart{Zadanie z Wikieł Z 1.62 c) moja wersja nr 423}

Rozwiązać nierówności $(9-x)(x+3)^{2}(18-x)^{3}\le0$.
\zadStop
\rozwStart{Patryk Wirkus}{}
Miejsca zerowe naszego wielomianu to: $9, -3, 18$.\\
Wielomian jest stopnia parzystego, ponadto znak współczynnika przy\linebreak najwyższej potędze x jest ujemny.\\ W związku z tym wykres wielomianu zaczyna się od lewej strony powyżej osi OX.\\
Ponadto w punkcie $-3$ wykres odbija się od osi poziomej.\\
A więc $$x \in \{-3\} \cup [9,18].$$
\rozwStop
\odpStart
$x \in \{-3\} \cup [9,18]$
\odpStop
\testStart
A.$x \in \{-3\} \cup [9,18]$\\
B.$x \in \{3\} \cup (9,18)$\\
C.$x \in \{-3\} \cup (9,18]$\\
D.$x \in \{3\} \cup (9,18]$\\
E.$x \in \{-3\} \cup [9,18)$\\
F.$x \in \{3\} \cup [9,18)$\\
G.$x \in \{-3\} \cup (9,18)$\\
H.$x \in \{3\} \cup [9,18]$
\testStop
\kluczStart
A
\kluczStop



\zadStart{Zadanie z Wikieł Z 1.62 c) moja wersja nr 424}

Rozwiązać nierówności $(9-x)(x+3)^{2}(19-x)^{3}\le0$.
\zadStop
\rozwStart{Patryk Wirkus}{}
Miejsca zerowe naszego wielomianu to: $9, -3, 19$.\\
Wielomian jest stopnia parzystego, ponadto znak współczynnika przy\linebreak najwyższej potędze x jest ujemny.\\ W związku z tym wykres wielomianu zaczyna się od lewej strony powyżej osi OX.\\
Ponadto w punkcie $-3$ wykres odbija się od osi poziomej.\\
A więc $$x \in \{-3\} \cup [9,19].$$
\rozwStop
\odpStart
$x \in \{-3\} \cup [9,19]$
\odpStop
\testStart
A.$x \in \{-3\} \cup [9,19]$\\
B.$x \in \{3\} \cup (9,19)$\\
C.$x \in \{-3\} \cup (9,19]$\\
D.$x \in \{3\} \cup (9,19]$\\
E.$x \in \{-3\} \cup [9,19)$\\
F.$x \in \{3\} \cup [9,19)$\\
G.$x \in \{-3\} \cup (9,19)$\\
H.$x \in \{3\} \cup [9,19]$
\testStop
\kluczStart
A
\kluczStop



\zadStart{Zadanie z Wikieł Z 1.62 c) moja wersja nr 425}

Rozwiązać nierówności $(9-x)(x+3)^{2}(20-x)^{3}\le0$.
\zadStop
\rozwStart{Patryk Wirkus}{}
Miejsca zerowe naszego wielomianu to: $9, -3, 20$.\\
Wielomian jest stopnia parzystego, ponadto znak współczynnika przy\linebreak najwyższej potędze x jest ujemny.\\ W związku z tym wykres wielomianu zaczyna się od lewej strony powyżej osi OX.\\
Ponadto w punkcie $-3$ wykres odbija się od osi poziomej.\\
A więc $$x \in \{-3\} \cup [9,20].$$
\rozwStop
\odpStart
$x \in \{-3\} \cup [9,20]$
\odpStop
\testStart
A.$x \in \{-3\} \cup [9,20]$\\
B.$x \in \{3\} \cup (9,20)$\\
C.$x \in \{-3\} \cup (9,20]$\\
D.$x \in \{3\} \cup (9,20]$\\
E.$x \in \{-3\} \cup [9,20)$\\
F.$x \in \{3\} \cup [9,20)$\\
G.$x \in \{-3\} \cup (9,20)$\\
H.$x \in \{3\} \cup [9,20]$
\testStop
\kluczStart
A
\kluczStop



\zadStart{Zadanie z Wikieł Z 1.62 c) moja wersja nr 426}

Rozwiązać nierówności $(9-x)(x+4)^{2}(10-x)^{3}\le0$.
\zadStop
\rozwStart{Patryk Wirkus}{}
Miejsca zerowe naszego wielomianu to: $9, -4, 10$.\\
Wielomian jest stopnia parzystego, ponadto znak współczynnika przy\linebreak najwyższej potędze x jest ujemny.\\ W związku z tym wykres wielomianu zaczyna się od lewej strony powyżej osi OX.\\
Ponadto w punkcie $-4$ wykres odbija się od osi poziomej.\\
A więc $$x \in \{-4\} \cup [9,10].$$
\rozwStop
\odpStart
$x \in \{-4\} \cup [9,10]$
\odpStop
\testStart
A.$x \in \{-4\} \cup [9,10]$\\
B.$x \in \{4\} \cup (9,10)$\\
C.$x \in \{-4\} \cup (9,10]$\\
D.$x \in \{4\} \cup (9,10]$\\
E.$x \in \{-4\} \cup [9,10)$\\
F.$x \in \{4\} \cup [9,10)$\\
G.$x \in \{-4\} \cup (9,10)$\\
H.$x \in \{4\} \cup [9,10]$
\testStop
\kluczStart
A
\kluczStop



\zadStart{Zadanie z Wikieł Z 1.62 c) moja wersja nr 427}

Rozwiązać nierówności $(9-x)(x+4)^{2}(11-x)^{3}\le0$.
\zadStop
\rozwStart{Patryk Wirkus}{}
Miejsca zerowe naszego wielomianu to: $9, -4, 11$.\\
Wielomian jest stopnia parzystego, ponadto znak współczynnika przy\linebreak najwyższej potędze x jest ujemny.\\ W związku z tym wykres wielomianu zaczyna się od lewej strony powyżej osi OX.\\
Ponadto w punkcie $-4$ wykres odbija się od osi poziomej.\\
A więc $$x \in \{-4\} \cup [9,11].$$
\rozwStop
\odpStart
$x \in \{-4\} \cup [9,11]$
\odpStop
\testStart
A.$x \in \{-4\} \cup [9,11]$\\
B.$x \in \{4\} \cup (9,11)$\\
C.$x \in \{-4\} \cup (9,11]$\\
D.$x \in \{4\} \cup (9,11]$\\
E.$x \in \{-4\} \cup [9,11)$\\
F.$x \in \{4\} \cup [9,11)$\\
G.$x \in \{-4\} \cup (9,11)$\\
H.$x \in \{4\} \cup [9,11]$
\testStop
\kluczStart
A
\kluczStop



\zadStart{Zadanie z Wikieł Z 1.62 c) moja wersja nr 428}

Rozwiązać nierówności $(9-x)(x+4)^{2}(12-x)^{3}\le0$.
\zadStop
\rozwStart{Patryk Wirkus}{}
Miejsca zerowe naszego wielomianu to: $9, -4, 12$.\\
Wielomian jest stopnia parzystego, ponadto znak współczynnika przy\linebreak najwyższej potędze x jest ujemny.\\ W związku z tym wykres wielomianu zaczyna się od lewej strony powyżej osi OX.\\
Ponadto w punkcie $-4$ wykres odbija się od osi poziomej.\\
A więc $$x \in \{-4\} \cup [9,12].$$
\rozwStop
\odpStart
$x \in \{-4\} \cup [9,12]$
\odpStop
\testStart
A.$x \in \{-4\} \cup [9,12]$\\
B.$x \in \{4\} \cup (9,12)$\\
C.$x \in \{-4\} \cup (9,12]$\\
D.$x \in \{4\} \cup (9,12]$\\
E.$x \in \{-4\} \cup [9,12)$\\
F.$x \in \{4\} \cup [9,12)$\\
G.$x \in \{-4\} \cup (9,12)$\\
H.$x \in \{4\} \cup [9,12]$
\testStop
\kluczStart
A
\kluczStop



\zadStart{Zadanie z Wikieł Z 1.62 c) moja wersja nr 429}

Rozwiązać nierówności $(9-x)(x+4)^{2}(13-x)^{3}\le0$.
\zadStop
\rozwStart{Patryk Wirkus}{}
Miejsca zerowe naszego wielomianu to: $9, -4, 13$.\\
Wielomian jest stopnia parzystego, ponadto znak współczynnika przy\linebreak najwyższej potędze x jest ujemny.\\ W związku z tym wykres wielomianu zaczyna się od lewej strony powyżej osi OX.\\
Ponadto w punkcie $-4$ wykres odbija się od osi poziomej.\\
A więc $$x \in \{-4\} \cup [9,13].$$
\rozwStop
\odpStart
$x \in \{-4\} \cup [9,13]$
\odpStop
\testStart
A.$x \in \{-4\} \cup [9,13]$\\
B.$x \in \{4\} \cup (9,13)$\\
C.$x \in \{-4\} \cup (9,13]$\\
D.$x \in \{4\} \cup (9,13]$\\
E.$x \in \{-4\} \cup [9,13)$\\
F.$x \in \{4\} \cup [9,13)$\\
G.$x \in \{-4\} \cup (9,13)$\\
H.$x \in \{4\} \cup [9,13]$
\testStop
\kluczStart
A
\kluczStop



\zadStart{Zadanie z Wikieł Z 1.62 c) moja wersja nr 430}

Rozwiązać nierówności $(9-x)(x+4)^{2}(14-x)^{3}\le0$.
\zadStop
\rozwStart{Patryk Wirkus}{}
Miejsca zerowe naszego wielomianu to: $9, -4, 14$.\\
Wielomian jest stopnia parzystego, ponadto znak współczynnika przy\linebreak najwyższej potędze x jest ujemny.\\ W związku z tym wykres wielomianu zaczyna się od lewej strony powyżej osi OX.\\
Ponadto w punkcie $-4$ wykres odbija się od osi poziomej.\\
A więc $$x \in \{-4\} \cup [9,14].$$
\rozwStop
\odpStart
$x \in \{-4\} \cup [9,14]$
\odpStop
\testStart
A.$x \in \{-4\} \cup [9,14]$\\
B.$x \in \{4\} \cup (9,14)$\\
C.$x \in \{-4\} \cup (9,14]$\\
D.$x \in \{4\} \cup (9,14]$\\
E.$x \in \{-4\} \cup [9,14)$\\
F.$x \in \{4\} \cup [9,14)$\\
G.$x \in \{-4\} \cup (9,14)$\\
H.$x \in \{4\} \cup [9,14]$
\testStop
\kluczStart
A
\kluczStop



\zadStart{Zadanie z Wikieł Z 1.62 c) moja wersja nr 431}

Rozwiązać nierówności $(9-x)(x+4)^{2}(15-x)^{3}\le0$.
\zadStop
\rozwStart{Patryk Wirkus}{}
Miejsca zerowe naszego wielomianu to: $9, -4, 15$.\\
Wielomian jest stopnia parzystego, ponadto znak współczynnika przy\linebreak najwyższej potędze x jest ujemny.\\ W związku z tym wykres wielomianu zaczyna się od lewej strony powyżej osi OX.\\
Ponadto w punkcie $-4$ wykres odbija się od osi poziomej.\\
A więc $$x \in \{-4\} \cup [9,15].$$
\rozwStop
\odpStart
$x \in \{-4\} \cup [9,15]$
\odpStop
\testStart
A.$x \in \{-4\} \cup [9,15]$\\
B.$x \in \{4\} \cup (9,15)$\\
C.$x \in \{-4\} \cup (9,15]$\\
D.$x \in \{4\} \cup (9,15]$\\
E.$x \in \{-4\} \cup [9,15)$\\
F.$x \in \{4\} \cup [9,15)$\\
G.$x \in \{-4\} \cup (9,15)$\\
H.$x \in \{4\} \cup [9,15]$
\testStop
\kluczStart
A
\kluczStop



\zadStart{Zadanie z Wikieł Z 1.62 c) moja wersja nr 432}

Rozwiązać nierówności $(9-x)(x+4)^{2}(16-x)^{3}\le0$.
\zadStop
\rozwStart{Patryk Wirkus}{}
Miejsca zerowe naszego wielomianu to: $9, -4, 16$.\\
Wielomian jest stopnia parzystego, ponadto znak współczynnika przy\linebreak najwyższej potędze x jest ujemny.\\ W związku z tym wykres wielomianu zaczyna się od lewej strony powyżej osi OX.\\
Ponadto w punkcie $-4$ wykres odbija się od osi poziomej.\\
A więc $$x \in \{-4\} \cup [9,16].$$
\rozwStop
\odpStart
$x \in \{-4\} \cup [9,16]$
\odpStop
\testStart
A.$x \in \{-4\} \cup [9,16]$\\
B.$x \in \{4\} \cup (9,16)$\\
C.$x \in \{-4\} \cup (9,16]$\\
D.$x \in \{4\} \cup (9,16]$\\
E.$x \in \{-4\} \cup [9,16)$\\
F.$x \in \{4\} \cup [9,16)$\\
G.$x \in \{-4\} \cup (9,16)$\\
H.$x \in \{4\} \cup [9,16]$
\testStop
\kluczStart
A
\kluczStop



\zadStart{Zadanie z Wikieł Z 1.62 c) moja wersja nr 433}

Rozwiązać nierówności $(9-x)(x+4)^{2}(17-x)^{3}\le0$.
\zadStop
\rozwStart{Patryk Wirkus}{}
Miejsca zerowe naszego wielomianu to: $9, -4, 17$.\\
Wielomian jest stopnia parzystego, ponadto znak współczynnika przy\linebreak najwyższej potędze x jest ujemny.\\ W związku z tym wykres wielomianu zaczyna się od lewej strony powyżej osi OX.\\
Ponadto w punkcie $-4$ wykres odbija się od osi poziomej.\\
A więc $$x \in \{-4\} \cup [9,17].$$
\rozwStop
\odpStart
$x \in \{-4\} \cup [9,17]$
\odpStop
\testStart
A.$x \in \{-4\} \cup [9,17]$\\
B.$x \in \{4\} \cup (9,17)$\\
C.$x \in \{-4\} \cup (9,17]$\\
D.$x \in \{4\} \cup (9,17]$\\
E.$x \in \{-4\} \cup [9,17)$\\
F.$x \in \{4\} \cup [9,17)$\\
G.$x \in \{-4\} \cup (9,17)$\\
H.$x \in \{4\} \cup [9,17]$
\testStop
\kluczStart
A
\kluczStop



\zadStart{Zadanie z Wikieł Z 1.62 c) moja wersja nr 434}

Rozwiązać nierówności $(9-x)(x+4)^{2}(18-x)^{3}\le0$.
\zadStop
\rozwStart{Patryk Wirkus}{}
Miejsca zerowe naszego wielomianu to: $9, -4, 18$.\\
Wielomian jest stopnia parzystego, ponadto znak współczynnika przy\linebreak najwyższej potędze x jest ujemny.\\ W związku z tym wykres wielomianu zaczyna się od lewej strony powyżej osi OX.\\
Ponadto w punkcie $-4$ wykres odbija się od osi poziomej.\\
A więc $$x \in \{-4\} \cup [9,18].$$
\rozwStop
\odpStart
$x \in \{-4\} \cup [9,18]$
\odpStop
\testStart
A.$x \in \{-4\} \cup [9,18]$\\
B.$x \in \{4\} \cup (9,18)$\\
C.$x \in \{-4\} \cup (9,18]$\\
D.$x \in \{4\} \cup (9,18]$\\
E.$x \in \{-4\} \cup [9,18)$\\
F.$x \in \{4\} \cup [9,18)$\\
G.$x \in \{-4\} \cup (9,18)$\\
H.$x \in \{4\} \cup [9,18]$
\testStop
\kluczStart
A
\kluczStop



\zadStart{Zadanie z Wikieł Z 1.62 c) moja wersja nr 435}

Rozwiązać nierówności $(9-x)(x+4)^{2}(19-x)^{3}\le0$.
\zadStop
\rozwStart{Patryk Wirkus}{}
Miejsca zerowe naszego wielomianu to: $9, -4, 19$.\\
Wielomian jest stopnia parzystego, ponadto znak współczynnika przy\linebreak najwyższej potędze x jest ujemny.\\ W związku z tym wykres wielomianu zaczyna się od lewej strony powyżej osi OX.\\
Ponadto w punkcie $-4$ wykres odbija się od osi poziomej.\\
A więc $$x \in \{-4\} \cup [9,19].$$
\rozwStop
\odpStart
$x \in \{-4\} \cup [9,19]$
\odpStop
\testStart
A.$x \in \{-4\} \cup [9,19]$\\
B.$x \in \{4\} \cup (9,19)$\\
C.$x \in \{-4\} \cup (9,19]$\\
D.$x \in \{4\} \cup (9,19]$\\
E.$x \in \{-4\} \cup [9,19)$\\
F.$x \in \{4\} \cup [9,19)$\\
G.$x \in \{-4\} \cup (9,19)$\\
H.$x \in \{4\} \cup [9,19]$
\testStop
\kluczStart
A
\kluczStop



\zadStart{Zadanie z Wikieł Z 1.62 c) moja wersja nr 436}

Rozwiązać nierówności $(9-x)(x+4)^{2}(20-x)^{3}\le0$.
\zadStop
\rozwStart{Patryk Wirkus}{}
Miejsca zerowe naszego wielomianu to: $9, -4, 20$.\\
Wielomian jest stopnia parzystego, ponadto znak współczynnika przy\linebreak najwyższej potędze x jest ujemny.\\ W związku z tym wykres wielomianu zaczyna się od lewej strony powyżej osi OX.\\
Ponadto w punkcie $-4$ wykres odbija się od osi poziomej.\\
A więc $$x \in \{-4\} \cup [9,20].$$
\rozwStop
\odpStart
$x \in \{-4\} \cup [9,20]$
\odpStop
\testStart
A.$x \in \{-4\} \cup [9,20]$\\
B.$x \in \{4\} \cup (9,20)$\\
C.$x \in \{-4\} \cup (9,20]$\\
D.$x \in \{4\} \cup (9,20]$\\
E.$x \in \{-4\} \cup [9,20)$\\
F.$x \in \{4\} \cup [9,20)$\\
G.$x \in \{-4\} \cup (9,20)$\\
H.$x \in \{4\} \cup [9,20]$
\testStop
\kluczStart
A
\kluczStop



\zadStart{Zadanie z Wikieł Z 1.62 c) moja wersja nr 437}

Rozwiązać nierówności $(9-x)(x+5)^{2}(10-x)^{3}\le0$.
\zadStop
\rozwStart{Patryk Wirkus}{}
Miejsca zerowe naszego wielomianu to: $9, -5, 10$.\\
Wielomian jest stopnia parzystego, ponadto znak współczynnika przy\linebreak najwyższej potędze x jest ujemny.\\ W związku z tym wykres wielomianu zaczyna się od lewej strony powyżej osi OX.\\
Ponadto w punkcie $-5$ wykres odbija się od osi poziomej.\\
A więc $$x \in \{-5\} \cup [9,10].$$
\rozwStop
\odpStart
$x \in \{-5\} \cup [9,10]$
\odpStop
\testStart
A.$x \in \{-5\} \cup [9,10]$\\
B.$x \in \{5\} \cup (9,10)$\\
C.$x \in \{-5\} \cup (9,10]$\\
D.$x \in \{5\} \cup (9,10]$\\
E.$x \in \{-5\} \cup [9,10)$\\
F.$x \in \{5\} \cup [9,10)$\\
G.$x \in \{-5\} \cup (9,10)$\\
H.$x \in \{5\} \cup [9,10]$
\testStop
\kluczStart
A
\kluczStop



\zadStart{Zadanie z Wikieł Z 1.62 c) moja wersja nr 438}

Rozwiązać nierówności $(9-x)(x+5)^{2}(11-x)^{3}\le0$.
\zadStop
\rozwStart{Patryk Wirkus}{}
Miejsca zerowe naszego wielomianu to: $9, -5, 11$.\\
Wielomian jest stopnia parzystego, ponadto znak współczynnika przy\linebreak najwyższej potędze x jest ujemny.\\ W związku z tym wykres wielomianu zaczyna się od lewej strony powyżej osi OX.\\
Ponadto w punkcie $-5$ wykres odbija się od osi poziomej.\\
A więc $$x \in \{-5\} \cup [9,11].$$
\rozwStop
\odpStart
$x \in \{-5\} \cup [9,11]$
\odpStop
\testStart
A.$x \in \{-5\} \cup [9,11]$\\
B.$x \in \{5\} \cup (9,11)$\\
C.$x \in \{-5\} \cup (9,11]$\\
D.$x \in \{5\} \cup (9,11]$\\
E.$x \in \{-5\} \cup [9,11)$\\
F.$x \in \{5\} \cup [9,11)$\\
G.$x \in \{-5\} \cup (9,11)$\\
H.$x \in \{5\} \cup [9,11]$
\testStop
\kluczStart
A
\kluczStop



\zadStart{Zadanie z Wikieł Z 1.62 c) moja wersja nr 439}

Rozwiązać nierówności $(9-x)(x+5)^{2}(12-x)^{3}\le0$.
\zadStop
\rozwStart{Patryk Wirkus}{}
Miejsca zerowe naszego wielomianu to: $9, -5, 12$.\\
Wielomian jest stopnia parzystego, ponadto znak współczynnika przy\linebreak najwyższej potędze x jest ujemny.\\ W związku z tym wykres wielomianu zaczyna się od lewej strony powyżej osi OX.\\
Ponadto w punkcie $-5$ wykres odbija się od osi poziomej.\\
A więc $$x \in \{-5\} \cup [9,12].$$
\rozwStop
\odpStart
$x \in \{-5\} \cup [9,12]$
\odpStop
\testStart
A.$x \in \{-5\} \cup [9,12]$\\
B.$x \in \{5\} \cup (9,12)$\\
C.$x \in \{-5\} \cup (9,12]$\\
D.$x \in \{5\} \cup (9,12]$\\
E.$x \in \{-5\} \cup [9,12)$\\
F.$x \in \{5\} \cup [9,12)$\\
G.$x \in \{-5\} \cup (9,12)$\\
H.$x \in \{5\} \cup [9,12]$
\testStop
\kluczStart
A
\kluczStop



\zadStart{Zadanie z Wikieł Z 1.62 c) moja wersja nr 440}

Rozwiązać nierówności $(9-x)(x+5)^{2}(13-x)^{3}\le0$.
\zadStop
\rozwStart{Patryk Wirkus}{}
Miejsca zerowe naszego wielomianu to: $9, -5, 13$.\\
Wielomian jest stopnia parzystego, ponadto znak współczynnika przy\linebreak najwyższej potędze x jest ujemny.\\ W związku z tym wykres wielomianu zaczyna się od lewej strony powyżej osi OX.\\
Ponadto w punkcie $-5$ wykres odbija się od osi poziomej.\\
A więc $$x \in \{-5\} \cup [9,13].$$
\rozwStop
\odpStart
$x \in \{-5\} \cup [9,13]$
\odpStop
\testStart
A.$x \in \{-5\} \cup [9,13]$\\
B.$x \in \{5\} \cup (9,13)$\\
C.$x \in \{-5\} \cup (9,13]$\\
D.$x \in \{5\} \cup (9,13]$\\
E.$x \in \{-5\} \cup [9,13)$\\
F.$x \in \{5\} \cup [9,13)$\\
G.$x \in \{-5\} \cup (9,13)$\\
H.$x \in \{5\} \cup [9,13]$
\testStop
\kluczStart
A
\kluczStop



\zadStart{Zadanie z Wikieł Z 1.62 c) moja wersja nr 441}

Rozwiązać nierówności $(9-x)(x+5)^{2}(14-x)^{3}\le0$.
\zadStop
\rozwStart{Patryk Wirkus}{}
Miejsca zerowe naszego wielomianu to: $9, -5, 14$.\\
Wielomian jest stopnia parzystego, ponadto znak współczynnika przy\linebreak najwyższej potędze x jest ujemny.\\ W związku z tym wykres wielomianu zaczyna się od lewej strony powyżej osi OX.\\
Ponadto w punkcie $-5$ wykres odbija się od osi poziomej.\\
A więc $$x \in \{-5\} \cup [9,14].$$
\rozwStop
\odpStart
$x \in \{-5\} \cup [9,14]$
\odpStop
\testStart
A.$x \in \{-5\} \cup [9,14]$\\
B.$x \in \{5\} \cup (9,14)$\\
C.$x \in \{-5\} \cup (9,14]$\\
D.$x \in \{5\} \cup (9,14]$\\
E.$x \in \{-5\} \cup [9,14)$\\
F.$x \in \{5\} \cup [9,14)$\\
G.$x \in \{-5\} \cup (9,14)$\\
H.$x \in \{5\} \cup [9,14]$
\testStop
\kluczStart
A
\kluczStop



\zadStart{Zadanie z Wikieł Z 1.62 c) moja wersja nr 442}

Rozwiązać nierówności $(9-x)(x+5)^{2}(15-x)^{3}\le0$.
\zadStop
\rozwStart{Patryk Wirkus}{}
Miejsca zerowe naszego wielomianu to: $9, -5, 15$.\\
Wielomian jest stopnia parzystego, ponadto znak współczynnika przy\linebreak najwyższej potędze x jest ujemny.\\ W związku z tym wykres wielomianu zaczyna się od lewej strony powyżej osi OX.\\
Ponadto w punkcie $-5$ wykres odbija się od osi poziomej.\\
A więc $$x \in \{-5\} \cup [9,15].$$
\rozwStop
\odpStart
$x \in \{-5\} \cup [9,15]$
\odpStop
\testStart
A.$x \in \{-5\} \cup [9,15]$\\
B.$x \in \{5\} \cup (9,15)$\\
C.$x \in \{-5\} \cup (9,15]$\\
D.$x \in \{5\} \cup (9,15]$\\
E.$x \in \{-5\} \cup [9,15)$\\
F.$x \in \{5\} \cup [9,15)$\\
G.$x \in \{-5\} \cup (9,15)$\\
H.$x \in \{5\} \cup [9,15]$
\testStop
\kluczStart
A
\kluczStop



\zadStart{Zadanie z Wikieł Z 1.62 c) moja wersja nr 443}

Rozwiązać nierówności $(9-x)(x+5)^{2}(16-x)^{3}\le0$.
\zadStop
\rozwStart{Patryk Wirkus}{}
Miejsca zerowe naszego wielomianu to: $9, -5, 16$.\\
Wielomian jest stopnia parzystego, ponadto znak współczynnika przy\linebreak najwyższej potędze x jest ujemny.\\ W związku z tym wykres wielomianu zaczyna się od lewej strony powyżej osi OX.\\
Ponadto w punkcie $-5$ wykres odbija się od osi poziomej.\\
A więc $$x \in \{-5\} \cup [9,16].$$
\rozwStop
\odpStart
$x \in \{-5\} \cup [9,16]$
\odpStop
\testStart
A.$x \in \{-5\} \cup [9,16]$\\
B.$x \in \{5\} \cup (9,16)$\\
C.$x \in \{-5\} \cup (9,16]$\\
D.$x \in \{5\} \cup (9,16]$\\
E.$x \in \{-5\} \cup [9,16)$\\
F.$x \in \{5\} \cup [9,16)$\\
G.$x \in \{-5\} \cup (9,16)$\\
H.$x \in \{5\} \cup [9,16]$
\testStop
\kluczStart
A
\kluczStop



\zadStart{Zadanie z Wikieł Z 1.62 c) moja wersja nr 444}

Rozwiązać nierówności $(9-x)(x+5)^{2}(17-x)^{3}\le0$.
\zadStop
\rozwStart{Patryk Wirkus}{}
Miejsca zerowe naszego wielomianu to: $9, -5, 17$.\\
Wielomian jest stopnia parzystego, ponadto znak współczynnika przy\linebreak najwyższej potędze x jest ujemny.\\ W związku z tym wykres wielomianu zaczyna się od lewej strony powyżej osi OX.\\
Ponadto w punkcie $-5$ wykres odbija się od osi poziomej.\\
A więc $$x \in \{-5\} \cup [9,17].$$
\rozwStop
\odpStart
$x \in \{-5\} \cup [9,17]$
\odpStop
\testStart
A.$x \in \{-5\} \cup [9,17]$\\
B.$x \in \{5\} \cup (9,17)$\\
C.$x \in \{-5\} \cup (9,17]$\\
D.$x \in \{5\} \cup (9,17]$\\
E.$x \in \{-5\} \cup [9,17)$\\
F.$x \in \{5\} \cup [9,17)$\\
G.$x \in \{-5\} \cup (9,17)$\\
H.$x \in \{5\} \cup [9,17]$
\testStop
\kluczStart
A
\kluczStop



\zadStart{Zadanie z Wikieł Z 1.62 c) moja wersja nr 445}

Rozwiązać nierówności $(9-x)(x+5)^{2}(18-x)^{3}\le0$.
\zadStop
\rozwStart{Patryk Wirkus}{}
Miejsca zerowe naszego wielomianu to: $9, -5, 18$.\\
Wielomian jest stopnia parzystego, ponadto znak współczynnika przy\linebreak najwyższej potędze x jest ujemny.\\ W związku z tym wykres wielomianu zaczyna się od lewej strony powyżej osi OX.\\
Ponadto w punkcie $-5$ wykres odbija się od osi poziomej.\\
A więc $$x \in \{-5\} \cup [9,18].$$
\rozwStop
\odpStart
$x \in \{-5\} \cup [9,18]$
\odpStop
\testStart
A.$x \in \{-5\} \cup [9,18]$\\
B.$x \in \{5\} \cup (9,18)$\\
C.$x \in \{-5\} \cup (9,18]$\\
D.$x \in \{5\} \cup (9,18]$\\
E.$x \in \{-5\} \cup [9,18)$\\
F.$x \in \{5\} \cup [9,18)$\\
G.$x \in \{-5\} \cup (9,18)$\\
H.$x \in \{5\} \cup [9,18]$
\testStop
\kluczStart
A
\kluczStop



\zadStart{Zadanie z Wikieł Z 1.62 c) moja wersja nr 446}

Rozwiązać nierówności $(9-x)(x+5)^{2}(19-x)^{3}\le0$.
\zadStop
\rozwStart{Patryk Wirkus}{}
Miejsca zerowe naszego wielomianu to: $9, -5, 19$.\\
Wielomian jest stopnia parzystego, ponadto znak współczynnika przy\linebreak najwyższej potędze x jest ujemny.\\ W związku z tym wykres wielomianu zaczyna się od lewej strony powyżej osi OX.\\
Ponadto w punkcie $-5$ wykres odbija się od osi poziomej.\\
A więc $$x \in \{-5\} \cup [9,19].$$
\rozwStop
\odpStart
$x \in \{-5\} \cup [9,19]$
\odpStop
\testStart
A.$x \in \{-5\} \cup [9,19]$\\
B.$x \in \{5\} \cup (9,19)$\\
C.$x \in \{-5\} \cup (9,19]$\\
D.$x \in \{5\} \cup (9,19]$\\
E.$x \in \{-5\} \cup [9,19)$\\
F.$x \in \{5\} \cup [9,19)$\\
G.$x \in \{-5\} \cup (9,19)$\\
H.$x \in \{5\} \cup [9,19]$
\testStop
\kluczStart
A
\kluczStop



\zadStart{Zadanie z Wikieł Z 1.62 c) moja wersja nr 447}

Rozwiązać nierówności $(9-x)(x+5)^{2}(20-x)^{3}\le0$.
\zadStop
\rozwStart{Patryk Wirkus}{}
Miejsca zerowe naszego wielomianu to: $9, -5, 20$.\\
Wielomian jest stopnia parzystego, ponadto znak współczynnika przy\linebreak najwyższej potędze x jest ujemny.\\ W związku z tym wykres wielomianu zaczyna się od lewej strony powyżej osi OX.\\
Ponadto w punkcie $-5$ wykres odbija się od osi poziomej.\\
A więc $$x \in \{-5\} \cup [9,20].$$
\rozwStop
\odpStart
$x \in \{-5\} \cup [9,20]$
\odpStop
\testStart
A.$x \in \{-5\} \cup [9,20]$\\
B.$x \in \{5\} \cup (9,20)$\\
C.$x \in \{-5\} \cup (9,20]$\\
D.$x \in \{5\} \cup (9,20]$\\
E.$x \in \{-5\} \cup [9,20)$\\
F.$x \in \{5\} \cup [9,20)$\\
G.$x \in \{-5\} \cup (9,20)$\\
H.$x \in \{5\} \cup [9,20]$
\testStop
\kluczStart
A
\kluczStop



\zadStart{Zadanie z Wikieł Z 1.62 c) moja wersja nr 448}

Rozwiązać nierówności $(9-x)(x+6)^{2}(10-x)^{3}\le0$.
\zadStop
\rozwStart{Patryk Wirkus}{}
Miejsca zerowe naszego wielomianu to: $9, -6, 10$.\\
Wielomian jest stopnia parzystego, ponadto znak współczynnika przy\linebreak najwyższej potędze x jest ujemny.\\ W związku z tym wykres wielomianu zaczyna się od lewej strony powyżej osi OX.\\
Ponadto w punkcie $-6$ wykres odbija się od osi poziomej.\\
A więc $$x \in \{-6\} \cup [9,10].$$
\rozwStop
\odpStart
$x \in \{-6\} \cup [9,10]$
\odpStop
\testStart
A.$x \in \{-6\} \cup [9,10]$\\
B.$x \in \{6\} \cup (9,10)$\\
C.$x \in \{-6\} \cup (9,10]$\\
D.$x \in \{6\} \cup (9,10]$\\
E.$x \in \{-6\} \cup [9,10)$\\
F.$x \in \{6\} \cup [9,10)$\\
G.$x \in \{-6\} \cup (9,10)$\\
H.$x \in \{6\} \cup [9,10]$
\testStop
\kluczStart
A
\kluczStop



\zadStart{Zadanie z Wikieł Z 1.62 c) moja wersja nr 449}

Rozwiązać nierówności $(9-x)(x+6)^{2}(11-x)^{3}\le0$.
\zadStop
\rozwStart{Patryk Wirkus}{}
Miejsca zerowe naszego wielomianu to: $9, -6, 11$.\\
Wielomian jest stopnia parzystego, ponadto znak współczynnika przy\linebreak najwyższej potędze x jest ujemny.\\ W związku z tym wykres wielomianu zaczyna się od lewej strony powyżej osi OX.\\
Ponadto w punkcie $-6$ wykres odbija się od osi poziomej.\\
A więc $$x \in \{-6\} \cup [9,11].$$
\rozwStop
\odpStart
$x \in \{-6\} \cup [9,11]$
\odpStop
\testStart
A.$x \in \{-6\} \cup [9,11]$\\
B.$x \in \{6\} \cup (9,11)$\\
C.$x \in \{-6\} \cup (9,11]$\\
D.$x \in \{6\} \cup (9,11]$\\
E.$x \in \{-6\} \cup [9,11)$\\
F.$x \in \{6\} \cup [9,11)$\\
G.$x \in \{-6\} \cup (9,11)$\\
H.$x \in \{6\} \cup [9,11]$
\testStop
\kluczStart
A
\kluczStop



\zadStart{Zadanie z Wikieł Z 1.62 c) moja wersja nr 450}

Rozwiązać nierówności $(9-x)(x+6)^{2}(12-x)^{3}\le0$.
\zadStop
\rozwStart{Patryk Wirkus}{}
Miejsca zerowe naszego wielomianu to: $9, -6, 12$.\\
Wielomian jest stopnia parzystego, ponadto znak współczynnika przy\linebreak najwyższej potędze x jest ujemny.\\ W związku z tym wykres wielomianu zaczyna się od lewej strony powyżej osi OX.\\
Ponadto w punkcie $-6$ wykres odbija się od osi poziomej.\\
A więc $$x \in \{-6\} \cup [9,12].$$
\rozwStop
\odpStart
$x \in \{-6\} \cup [9,12]$
\odpStop
\testStart
A.$x \in \{-6\} \cup [9,12]$\\
B.$x \in \{6\} \cup (9,12)$\\
C.$x \in \{-6\} \cup (9,12]$\\
D.$x \in \{6\} \cup (9,12]$\\
E.$x \in \{-6\} \cup [9,12)$\\
F.$x \in \{6\} \cup [9,12)$\\
G.$x \in \{-6\} \cup (9,12)$\\
H.$x \in \{6\} \cup [9,12]$
\testStop
\kluczStart
A
\kluczStop



\zadStart{Zadanie z Wikieł Z 1.62 c) moja wersja nr 451}

Rozwiązać nierówności $(9-x)(x+6)^{2}(13-x)^{3}\le0$.
\zadStop
\rozwStart{Patryk Wirkus}{}
Miejsca zerowe naszego wielomianu to: $9, -6, 13$.\\
Wielomian jest stopnia parzystego, ponadto znak współczynnika przy\linebreak najwyższej potędze x jest ujemny.\\ W związku z tym wykres wielomianu zaczyna się od lewej strony powyżej osi OX.\\
Ponadto w punkcie $-6$ wykres odbija się od osi poziomej.\\
A więc $$x \in \{-6\} \cup [9,13].$$
\rozwStop
\odpStart
$x \in \{-6\} \cup [9,13]$
\odpStop
\testStart
A.$x \in \{-6\} \cup [9,13]$\\
B.$x \in \{6\} \cup (9,13)$\\
C.$x \in \{-6\} \cup (9,13]$\\
D.$x \in \{6\} \cup (9,13]$\\
E.$x \in \{-6\} \cup [9,13)$\\
F.$x \in \{6\} \cup [9,13)$\\
G.$x \in \{-6\} \cup (9,13)$\\
H.$x \in \{6\} \cup [9,13]$
\testStop
\kluczStart
A
\kluczStop



\zadStart{Zadanie z Wikieł Z 1.62 c) moja wersja nr 452}

Rozwiązać nierówności $(9-x)(x+6)^{2}(14-x)^{3}\le0$.
\zadStop
\rozwStart{Patryk Wirkus}{}
Miejsca zerowe naszego wielomianu to: $9, -6, 14$.\\
Wielomian jest stopnia parzystego, ponadto znak współczynnika przy\linebreak najwyższej potędze x jest ujemny.\\ W związku z tym wykres wielomianu zaczyna się od lewej strony powyżej osi OX.\\
Ponadto w punkcie $-6$ wykres odbija się od osi poziomej.\\
A więc $$x \in \{-6\} \cup [9,14].$$
\rozwStop
\odpStart
$x \in \{-6\} \cup [9,14]$
\odpStop
\testStart
A.$x \in \{-6\} \cup [9,14]$\\
B.$x \in \{6\} \cup (9,14)$\\
C.$x \in \{-6\} \cup (9,14]$\\
D.$x \in \{6\} \cup (9,14]$\\
E.$x \in \{-6\} \cup [9,14)$\\
F.$x \in \{6\} \cup [9,14)$\\
G.$x \in \{-6\} \cup (9,14)$\\
H.$x \in \{6\} \cup [9,14]$
\testStop
\kluczStart
A
\kluczStop



\zadStart{Zadanie z Wikieł Z 1.62 c) moja wersja nr 453}

Rozwiązać nierówności $(9-x)(x+6)^{2}(15-x)^{3}\le0$.
\zadStop
\rozwStart{Patryk Wirkus}{}
Miejsca zerowe naszego wielomianu to: $9, -6, 15$.\\
Wielomian jest stopnia parzystego, ponadto znak współczynnika przy\linebreak najwyższej potędze x jest ujemny.\\ W związku z tym wykres wielomianu zaczyna się od lewej strony powyżej osi OX.\\
Ponadto w punkcie $-6$ wykres odbija się od osi poziomej.\\
A więc $$x \in \{-6\} \cup [9,15].$$
\rozwStop
\odpStart
$x \in \{-6\} \cup [9,15]$
\odpStop
\testStart
A.$x \in \{-6\} \cup [9,15]$\\
B.$x \in \{6\} \cup (9,15)$\\
C.$x \in \{-6\} \cup (9,15]$\\
D.$x \in \{6\} \cup (9,15]$\\
E.$x \in \{-6\} \cup [9,15)$\\
F.$x \in \{6\} \cup [9,15)$\\
G.$x \in \{-6\} \cup (9,15)$\\
H.$x \in \{6\} \cup [9,15]$
\testStop
\kluczStart
A
\kluczStop



\zadStart{Zadanie z Wikieł Z 1.62 c) moja wersja nr 454}

Rozwiązać nierówności $(9-x)(x+6)^{2}(16-x)^{3}\le0$.
\zadStop
\rozwStart{Patryk Wirkus}{}
Miejsca zerowe naszego wielomianu to: $9, -6, 16$.\\
Wielomian jest stopnia parzystego, ponadto znak współczynnika przy\linebreak najwyższej potędze x jest ujemny.\\ W związku z tym wykres wielomianu zaczyna się od lewej strony powyżej osi OX.\\
Ponadto w punkcie $-6$ wykres odbija się od osi poziomej.\\
A więc $$x \in \{-6\} \cup [9,16].$$
\rozwStop
\odpStart
$x \in \{-6\} \cup [9,16]$
\odpStop
\testStart
A.$x \in \{-6\} \cup [9,16]$\\
B.$x \in \{6\} \cup (9,16)$\\
C.$x \in \{-6\} \cup (9,16]$\\
D.$x \in \{6\} \cup (9,16]$\\
E.$x \in \{-6\} \cup [9,16)$\\
F.$x \in \{6\} \cup [9,16)$\\
G.$x \in \{-6\} \cup (9,16)$\\
H.$x \in \{6\} \cup [9,16]$
\testStop
\kluczStart
A
\kluczStop



\zadStart{Zadanie z Wikieł Z 1.62 c) moja wersja nr 455}

Rozwiązać nierówności $(9-x)(x+6)^{2}(17-x)^{3}\le0$.
\zadStop
\rozwStart{Patryk Wirkus}{}
Miejsca zerowe naszego wielomianu to: $9, -6, 17$.\\
Wielomian jest stopnia parzystego, ponadto znak współczynnika przy\linebreak najwyższej potędze x jest ujemny.\\ W związku z tym wykres wielomianu zaczyna się od lewej strony powyżej osi OX.\\
Ponadto w punkcie $-6$ wykres odbija się od osi poziomej.\\
A więc $$x \in \{-6\} \cup [9,17].$$
\rozwStop
\odpStart
$x \in \{-6\} \cup [9,17]$
\odpStop
\testStart
A.$x \in \{-6\} \cup [9,17]$\\
B.$x \in \{6\} \cup (9,17)$\\
C.$x \in \{-6\} \cup (9,17]$\\
D.$x \in \{6\} \cup (9,17]$\\
E.$x \in \{-6\} \cup [9,17)$\\
F.$x \in \{6\} \cup [9,17)$\\
G.$x \in \{-6\} \cup (9,17)$\\
H.$x \in \{6\} \cup [9,17]$
\testStop
\kluczStart
A
\kluczStop



\zadStart{Zadanie z Wikieł Z 1.62 c) moja wersja nr 456}

Rozwiązać nierówności $(9-x)(x+6)^{2}(18-x)^{3}\le0$.
\zadStop
\rozwStart{Patryk Wirkus}{}
Miejsca zerowe naszego wielomianu to: $9, -6, 18$.\\
Wielomian jest stopnia parzystego, ponadto znak współczynnika przy\linebreak najwyższej potędze x jest ujemny.\\ W związku z tym wykres wielomianu zaczyna się od lewej strony powyżej osi OX.\\
Ponadto w punkcie $-6$ wykres odbija się od osi poziomej.\\
A więc $$x \in \{-6\} \cup [9,18].$$
\rozwStop
\odpStart
$x \in \{-6\} \cup [9,18]$
\odpStop
\testStart
A.$x \in \{-6\} \cup [9,18]$\\
B.$x \in \{6\} \cup (9,18)$\\
C.$x \in \{-6\} \cup (9,18]$\\
D.$x \in \{6\} \cup (9,18]$\\
E.$x \in \{-6\} \cup [9,18)$\\
F.$x \in \{6\} \cup [9,18)$\\
G.$x \in \{-6\} \cup (9,18)$\\
H.$x \in \{6\} \cup [9,18]$
\testStop
\kluczStart
A
\kluczStop



\zadStart{Zadanie z Wikieł Z 1.62 c) moja wersja nr 457}

Rozwiązać nierówności $(9-x)(x+6)^{2}(19-x)^{3}\le0$.
\zadStop
\rozwStart{Patryk Wirkus}{}
Miejsca zerowe naszego wielomianu to: $9, -6, 19$.\\
Wielomian jest stopnia parzystego, ponadto znak współczynnika przy\linebreak najwyższej potędze x jest ujemny.\\ W związku z tym wykres wielomianu zaczyna się od lewej strony powyżej osi OX.\\
Ponadto w punkcie $-6$ wykres odbija się od osi poziomej.\\
A więc $$x \in \{-6\} \cup [9,19].$$
\rozwStop
\odpStart
$x \in \{-6\} \cup [9,19]$
\odpStop
\testStart
A.$x \in \{-6\} \cup [9,19]$\\
B.$x \in \{6\} \cup (9,19)$\\
C.$x \in \{-6\} \cup (9,19]$\\
D.$x \in \{6\} \cup (9,19]$\\
E.$x \in \{-6\} \cup [9,19)$\\
F.$x \in \{6\} \cup [9,19)$\\
G.$x \in \{-6\} \cup (9,19)$\\
H.$x \in \{6\} \cup [9,19]$
\testStop
\kluczStart
A
\kluczStop



\zadStart{Zadanie z Wikieł Z 1.62 c) moja wersja nr 458}

Rozwiązać nierówności $(9-x)(x+6)^{2}(20-x)^{3}\le0$.
\zadStop
\rozwStart{Patryk Wirkus}{}
Miejsca zerowe naszego wielomianu to: $9, -6, 20$.\\
Wielomian jest stopnia parzystego, ponadto znak współczynnika przy\linebreak najwyższej potędze x jest ujemny.\\ W związku z tym wykres wielomianu zaczyna się od lewej strony powyżej osi OX.\\
Ponadto w punkcie $-6$ wykres odbija się od osi poziomej.\\
A więc $$x \in \{-6\} \cup [9,20].$$
\rozwStop
\odpStart
$x \in \{-6\} \cup [9,20]$
\odpStop
\testStart
A.$x \in \{-6\} \cup [9,20]$\\
B.$x \in \{6\} \cup (9,20)$\\
C.$x \in \{-6\} \cup (9,20]$\\
D.$x \in \{6\} \cup (9,20]$\\
E.$x \in \{-6\} \cup [9,20)$\\
F.$x \in \{6\} \cup [9,20)$\\
G.$x \in \{-6\} \cup (9,20)$\\
H.$x \in \{6\} \cup [9,20]$
\testStop
\kluczStart
A
\kluczStop



\zadStart{Zadanie z Wikieł Z 1.62 c) moja wersja nr 459}

Rozwiązać nierówności $(9-x)(x+7)^{2}(10-x)^{3}\le0$.
\zadStop
\rozwStart{Patryk Wirkus}{}
Miejsca zerowe naszego wielomianu to: $9, -7, 10$.\\
Wielomian jest stopnia parzystego, ponadto znak współczynnika przy\linebreak najwyższej potędze x jest ujemny.\\ W związku z tym wykres wielomianu zaczyna się od lewej strony powyżej osi OX.\\
Ponadto w punkcie $-7$ wykres odbija się od osi poziomej.\\
A więc $$x \in \{-7\} \cup [9,10].$$
\rozwStop
\odpStart
$x \in \{-7\} \cup [9,10]$
\odpStop
\testStart
A.$x \in \{-7\} \cup [9,10]$\\
B.$x \in \{7\} \cup (9,10)$\\
C.$x \in \{-7\} \cup (9,10]$\\
D.$x \in \{7\} \cup (9,10]$\\
E.$x \in \{-7\} \cup [9,10)$\\
F.$x \in \{7\} \cup [9,10)$\\
G.$x \in \{-7\} \cup (9,10)$\\
H.$x \in \{7\} \cup [9,10]$
\testStop
\kluczStart
A
\kluczStop



\zadStart{Zadanie z Wikieł Z 1.62 c) moja wersja nr 460}

Rozwiązać nierówności $(9-x)(x+7)^{2}(11-x)^{3}\le0$.
\zadStop
\rozwStart{Patryk Wirkus}{}
Miejsca zerowe naszego wielomianu to: $9, -7, 11$.\\
Wielomian jest stopnia parzystego, ponadto znak współczynnika przy\linebreak najwyższej potędze x jest ujemny.\\ W związku z tym wykres wielomianu zaczyna się od lewej strony powyżej osi OX.\\
Ponadto w punkcie $-7$ wykres odbija się od osi poziomej.\\
A więc $$x \in \{-7\} \cup [9,11].$$
\rozwStop
\odpStart
$x \in \{-7\} \cup [9,11]$
\odpStop
\testStart
A.$x \in \{-7\} \cup [9,11]$\\
B.$x \in \{7\} \cup (9,11)$\\
C.$x \in \{-7\} \cup (9,11]$\\
D.$x \in \{7\} \cup (9,11]$\\
E.$x \in \{-7\} \cup [9,11)$\\
F.$x \in \{7\} \cup [9,11)$\\
G.$x \in \{-7\} \cup (9,11)$\\
H.$x \in \{7\} \cup [9,11]$
\testStop
\kluczStart
A
\kluczStop



\zadStart{Zadanie z Wikieł Z 1.62 c) moja wersja nr 461}

Rozwiązać nierówności $(9-x)(x+7)^{2}(12-x)^{3}\le0$.
\zadStop
\rozwStart{Patryk Wirkus}{}
Miejsca zerowe naszego wielomianu to: $9, -7, 12$.\\
Wielomian jest stopnia parzystego, ponadto znak współczynnika przy\linebreak najwyższej potędze x jest ujemny.\\ W związku z tym wykres wielomianu zaczyna się od lewej strony powyżej osi OX.\\
Ponadto w punkcie $-7$ wykres odbija się od osi poziomej.\\
A więc $$x \in \{-7\} \cup [9,12].$$
\rozwStop
\odpStart
$x \in \{-7\} \cup [9,12]$
\odpStop
\testStart
A.$x \in \{-7\} \cup [9,12]$\\
B.$x \in \{7\} \cup (9,12)$\\
C.$x \in \{-7\} \cup (9,12]$\\
D.$x \in \{7\} \cup (9,12]$\\
E.$x \in \{-7\} \cup [9,12)$\\
F.$x \in \{7\} \cup [9,12)$\\
G.$x \in \{-7\} \cup (9,12)$\\
H.$x \in \{7\} \cup [9,12]$
\testStop
\kluczStart
A
\kluczStop



\zadStart{Zadanie z Wikieł Z 1.62 c) moja wersja nr 462}

Rozwiązać nierówności $(9-x)(x+7)^{2}(13-x)^{3}\le0$.
\zadStop
\rozwStart{Patryk Wirkus}{}
Miejsca zerowe naszego wielomianu to: $9, -7, 13$.\\
Wielomian jest stopnia parzystego, ponadto znak współczynnika przy\linebreak najwyższej potędze x jest ujemny.\\ W związku z tym wykres wielomianu zaczyna się od lewej strony powyżej osi OX.\\
Ponadto w punkcie $-7$ wykres odbija się od osi poziomej.\\
A więc $$x \in \{-7\} \cup [9,13].$$
\rozwStop
\odpStart
$x \in \{-7\} \cup [9,13]$
\odpStop
\testStart
A.$x \in \{-7\} \cup [9,13]$\\
B.$x \in \{7\} \cup (9,13)$\\
C.$x \in \{-7\} \cup (9,13]$\\
D.$x \in \{7\} \cup (9,13]$\\
E.$x \in \{-7\} \cup [9,13)$\\
F.$x \in \{7\} \cup [9,13)$\\
G.$x \in \{-7\} \cup (9,13)$\\
H.$x \in \{7\} \cup [9,13]$
\testStop
\kluczStart
A
\kluczStop



\zadStart{Zadanie z Wikieł Z 1.62 c) moja wersja nr 463}

Rozwiązać nierówności $(9-x)(x+7)^{2}(14-x)^{3}\le0$.
\zadStop
\rozwStart{Patryk Wirkus}{}
Miejsca zerowe naszego wielomianu to: $9, -7, 14$.\\
Wielomian jest stopnia parzystego, ponadto znak współczynnika przy\linebreak najwyższej potędze x jest ujemny.\\ W związku z tym wykres wielomianu zaczyna się od lewej strony powyżej osi OX.\\
Ponadto w punkcie $-7$ wykres odbija się od osi poziomej.\\
A więc $$x \in \{-7\} \cup [9,14].$$
\rozwStop
\odpStart
$x \in \{-7\} \cup [9,14]$
\odpStop
\testStart
A.$x \in \{-7\} \cup [9,14]$\\
B.$x \in \{7\} \cup (9,14)$\\
C.$x \in \{-7\} \cup (9,14]$\\
D.$x \in \{7\} \cup (9,14]$\\
E.$x \in \{-7\} \cup [9,14)$\\
F.$x \in \{7\} \cup [9,14)$\\
G.$x \in \{-7\} \cup (9,14)$\\
H.$x \in \{7\} \cup [9,14]$
\testStop
\kluczStart
A
\kluczStop



\zadStart{Zadanie z Wikieł Z 1.62 c) moja wersja nr 464}

Rozwiązać nierówności $(9-x)(x+7)^{2}(15-x)^{3}\le0$.
\zadStop
\rozwStart{Patryk Wirkus}{}
Miejsca zerowe naszego wielomianu to: $9, -7, 15$.\\
Wielomian jest stopnia parzystego, ponadto znak współczynnika przy\linebreak najwyższej potędze x jest ujemny.\\ W związku z tym wykres wielomianu zaczyna się od lewej strony powyżej osi OX.\\
Ponadto w punkcie $-7$ wykres odbija się od osi poziomej.\\
A więc $$x \in \{-7\} \cup [9,15].$$
\rozwStop
\odpStart
$x \in \{-7\} \cup [9,15]$
\odpStop
\testStart
A.$x \in \{-7\} \cup [9,15]$\\
B.$x \in \{7\} \cup (9,15)$\\
C.$x \in \{-7\} \cup (9,15]$\\
D.$x \in \{7\} \cup (9,15]$\\
E.$x \in \{-7\} \cup [9,15)$\\
F.$x \in \{7\} \cup [9,15)$\\
G.$x \in \{-7\} \cup (9,15)$\\
H.$x \in \{7\} \cup [9,15]$
\testStop
\kluczStart
A
\kluczStop



\zadStart{Zadanie z Wikieł Z 1.62 c) moja wersja nr 465}

Rozwiązać nierówności $(9-x)(x+7)^{2}(16-x)^{3}\le0$.
\zadStop
\rozwStart{Patryk Wirkus}{}
Miejsca zerowe naszego wielomianu to: $9, -7, 16$.\\
Wielomian jest stopnia parzystego, ponadto znak współczynnika przy\linebreak najwyższej potędze x jest ujemny.\\ W związku z tym wykres wielomianu zaczyna się od lewej strony powyżej osi OX.\\
Ponadto w punkcie $-7$ wykres odbija się od osi poziomej.\\
A więc $$x \in \{-7\} \cup [9,16].$$
\rozwStop
\odpStart
$x \in \{-7\} \cup [9,16]$
\odpStop
\testStart
A.$x \in \{-7\} \cup [9,16]$\\
B.$x \in \{7\} \cup (9,16)$\\
C.$x \in \{-7\} \cup (9,16]$\\
D.$x \in \{7\} \cup (9,16]$\\
E.$x \in \{-7\} \cup [9,16)$\\
F.$x \in \{7\} \cup [9,16)$\\
G.$x \in \{-7\} \cup (9,16)$\\
H.$x \in \{7\} \cup [9,16]$
\testStop
\kluczStart
A
\kluczStop



\zadStart{Zadanie z Wikieł Z 1.62 c) moja wersja nr 466}

Rozwiązać nierówności $(9-x)(x+7)^{2}(17-x)^{3}\le0$.
\zadStop
\rozwStart{Patryk Wirkus}{}
Miejsca zerowe naszego wielomianu to: $9, -7, 17$.\\
Wielomian jest stopnia parzystego, ponadto znak współczynnika przy\linebreak najwyższej potędze x jest ujemny.\\ W związku z tym wykres wielomianu zaczyna się od lewej strony powyżej osi OX.\\
Ponadto w punkcie $-7$ wykres odbija się od osi poziomej.\\
A więc $$x \in \{-7\} \cup [9,17].$$
\rozwStop
\odpStart
$x \in \{-7\} \cup [9,17]$
\odpStop
\testStart
A.$x \in \{-7\} \cup [9,17]$\\
B.$x \in \{7\} \cup (9,17)$\\
C.$x \in \{-7\} \cup (9,17]$\\
D.$x \in \{7\} \cup (9,17]$\\
E.$x \in \{-7\} \cup [9,17)$\\
F.$x \in \{7\} \cup [9,17)$\\
G.$x \in \{-7\} \cup (9,17)$\\
H.$x \in \{7\} \cup [9,17]$
\testStop
\kluczStart
A
\kluczStop



\zadStart{Zadanie z Wikieł Z 1.62 c) moja wersja nr 467}

Rozwiązać nierówności $(9-x)(x+7)^{2}(18-x)^{3}\le0$.
\zadStop
\rozwStart{Patryk Wirkus}{}
Miejsca zerowe naszego wielomianu to: $9, -7, 18$.\\
Wielomian jest stopnia parzystego, ponadto znak współczynnika przy\linebreak najwyższej potędze x jest ujemny.\\ W związku z tym wykres wielomianu zaczyna się od lewej strony powyżej osi OX.\\
Ponadto w punkcie $-7$ wykres odbija się od osi poziomej.\\
A więc $$x \in \{-7\} \cup [9,18].$$
\rozwStop
\odpStart
$x \in \{-7\} \cup [9,18]$
\odpStop
\testStart
A.$x \in \{-7\} \cup [9,18]$\\
B.$x \in \{7\} \cup (9,18)$\\
C.$x \in \{-7\} \cup (9,18]$\\
D.$x \in \{7\} \cup (9,18]$\\
E.$x \in \{-7\} \cup [9,18)$\\
F.$x \in \{7\} \cup [9,18)$\\
G.$x \in \{-7\} \cup (9,18)$\\
H.$x \in \{7\} \cup [9,18]$
\testStop
\kluczStart
A
\kluczStop



\zadStart{Zadanie z Wikieł Z 1.62 c) moja wersja nr 468}

Rozwiązać nierówności $(9-x)(x+7)^{2}(19-x)^{3}\le0$.
\zadStop
\rozwStart{Patryk Wirkus}{}
Miejsca zerowe naszego wielomianu to: $9, -7, 19$.\\
Wielomian jest stopnia parzystego, ponadto znak współczynnika przy\linebreak najwyższej potędze x jest ujemny.\\ W związku z tym wykres wielomianu zaczyna się od lewej strony powyżej osi OX.\\
Ponadto w punkcie $-7$ wykres odbija się od osi poziomej.\\
A więc $$x \in \{-7\} \cup [9,19].$$
\rozwStop
\odpStart
$x \in \{-7\} \cup [9,19]$
\odpStop
\testStart
A.$x \in \{-7\} \cup [9,19]$\\
B.$x \in \{7\} \cup (9,19)$\\
C.$x \in \{-7\} \cup (9,19]$\\
D.$x \in \{7\} \cup (9,19]$\\
E.$x \in \{-7\} \cup [9,19)$\\
F.$x \in \{7\} \cup [9,19)$\\
G.$x \in \{-7\} \cup (9,19)$\\
H.$x \in \{7\} \cup [9,19]$
\testStop
\kluczStart
A
\kluczStop



\zadStart{Zadanie z Wikieł Z 1.62 c) moja wersja nr 469}

Rozwiązać nierówności $(9-x)(x+7)^{2}(20-x)^{3}\le0$.
\zadStop
\rozwStart{Patryk Wirkus}{}
Miejsca zerowe naszego wielomianu to: $9, -7, 20$.\\
Wielomian jest stopnia parzystego, ponadto znak współczynnika przy\linebreak najwyższej potędze x jest ujemny.\\ W związku z tym wykres wielomianu zaczyna się od lewej strony powyżej osi OX.\\
Ponadto w punkcie $-7$ wykres odbija się od osi poziomej.\\
A więc $$x \in \{-7\} \cup [9,20].$$
\rozwStop
\odpStart
$x \in \{-7\} \cup [9,20]$
\odpStop
\testStart
A.$x \in \{-7\} \cup [9,20]$\\
B.$x \in \{7\} \cup (9,20)$\\
C.$x \in \{-7\} \cup (9,20]$\\
D.$x \in \{7\} \cup (9,20]$\\
E.$x \in \{-7\} \cup [9,20)$\\
F.$x \in \{7\} \cup [9,20)$\\
G.$x \in \{-7\} \cup (9,20)$\\
H.$x \in \{7\} \cup [9,20]$
\testStop
\kluczStart
A
\kluczStop



\zadStart{Zadanie z Wikieł Z 1.62 c) moja wersja nr 470}

Rozwiązać nierówności $(9-x)(x+8)^{2}(10-x)^{3}\le0$.
\zadStop
\rozwStart{Patryk Wirkus}{}
Miejsca zerowe naszego wielomianu to: $9, -8, 10$.\\
Wielomian jest stopnia parzystego, ponadto znak współczynnika przy\linebreak najwyższej potędze x jest ujemny.\\ W związku z tym wykres wielomianu zaczyna się od lewej strony powyżej osi OX.\\
Ponadto w punkcie $-8$ wykres odbija się od osi poziomej.\\
A więc $$x \in \{-8\} \cup [9,10].$$
\rozwStop
\odpStart
$x \in \{-8\} \cup [9,10]$
\odpStop
\testStart
A.$x \in \{-8\} \cup [9,10]$\\
B.$x \in \{8\} \cup (9,10)$\\
C.$x \in \{-8\} \cup (9,10]$\\
D.$x \in \{8\} \cup (9,10]$\\
E.$x \in \{-8\} \cup [9,10)$\\
F.$x \in \{8\} \cup [9,10)$\\
G.$x \in \{-8\} \cup (9,10)$\\
H.$x \in \{8\} \cup [9,10]$
\testStop
\kluczStart
A
\kluczStop



\zadStart{Zadanie z Wikieł Z 1.62 c) moja wersja nr 471}

Rozwiązać nierówności $(9-x)(x+8)^{2}(11-x)^{3}\le0$.
\zadStop
\rozwStart{Patryk Wirkus}{}
Miejsca zerowe naszego wielomianu to: $9, -8, 11$.\\
Wielomian jest stopnia parzystego, ponadto znak współczynnika przy\linebreak najwyższej potędze x jest ujemny.\\ W związku z tym wykres wielomianu zaczyna się od lewej strony powyżej osi OX.\\
Ponadto w punkcie $-8$ wykres odbija się od osi poziomej.\\
A więc $$x \in \{-8\} \cup [9,11].$$
\rozwStop
\odpStart
$x \in \{-8\} \cup [9,11]$
\odpStop
\testStart
A.$x \in \{-8\} \cup [9,11]$\\
B.$x \in \{8\} \cup (9,11)$\\
C.$x \in \{-8\} \cup (9,11]$\\
D.$x \in \{8\} \cup (9,11]$\\
E.$x \in \{-8\} \cup [9,11)$\\
F.$x \in \{8\} \cup [9,11)$\\
G.$x \in \{-8\} \cup (9,11)$\\
H.$x \in \{8\} \cup [9,11]$
\testStop
\kluczStart
A
\kluczStop



\zadStart{Zadanie z Wikieł Z 1.62 c) moja wersja nr 472}

Rozwiązać nierówności $(9-x)(x+8)^{2}(12-x)^{3}\le0$.
\zadStop
\rozwStart{Patryk Wirkus}{}
Miejsca zerowe naszego wielomianu to: $9, -8, 12$.\\
Wielomian jest stopnia parzystego, ponadto znak współczynnika przy\linebreak najwyższej potędze x jest ujemny.\\ W związku z tym wykres wielomianu zaczyna się od lewej strony powyżej osi OX.\\
Ponadto w punkcie $-8$ wykres odbija się od osi poziomej.\\
A więc $$x \in \{-8\} \cup [9,12].$$
\rozwStop
\odpStart
$x \in \{-8\} \cup [9,12]$
\odpStop
\testStart
A.$x \in \{-8\} \cup [9,12]$\\
B.$x \in \{8\} \cup (9,12)$\\
C.$x \in \{-8\} \cup (9,12]$\\
D.$x \in \{8\} \cup (9,12]$\\
E.$x \in \{-8\} \cup [9,12)$\\
F.$x \in \{8\} \cup [9,12)$\\
G.$x \in \{-8\} \cup (9,12)$\\
H.$x \in \{8\} \cup [9,12]$
\testStop
\kluczStart
A
\kluczStop



\zadStart{Zadanie z Wikieł Z 1.62 c) moja wersja nr 473}

Rozwiązać nierówności $(9-x)(x+8)^{2}(13-x)^{3}\le0$.
\zadStop
\rozwStart{Patryk Wirkus}{}
Miejsca zerowe naszego wielomianu to: $9, -8, 13$.\\
Wielomian jest stopnia parzystego, ponadto znak współczynnika przy\linebreak najwyższej potędze x jest ujemny.\\ W związku z tym wykres wielomianu zaczyna się od lewej strony powyżej osi OX.\\
Ponadto w punkcie $-8$ wykres odbija się od osi poziomej.\\
A więc $$x \in \{-8\} \cup [9,13].$$
\rozwStop
\odpStart
$x \in \{-8\} \cup [9,13]$
\odpStop
\testStart
A.$x \in \{-8\} \cup [9,13]$\\
B.$x \in \{8\} \cup (9,13)$\\
C.$x \in \{-8\} \cup (9,13]$\\
D.$x \in \{8\} \cup (9,13]$\\
E.$x \in \{-8\} \cup [9,13)$\\
F.$x \in \{8\} \cup [9,13)$\\
G.$x \in \{-8\} \cup (9,13)$\\
H.$x \in \{8\} \cup [9,13]$
\testStop
\kluczStart
A
\kluczStop



\zadStart{Zadanie z Wikieł Z 1.62 c) moja wersja nr 474}

Rozwiązać nierówności $(9-x)(x+8)^{2}(14-x)^{3}\le0$.
\zadStop
\rozwStart{Patryk Wirkus}{}
Miejsca zerowe naszego wielomianu to: $9, -8, 14$.\\
Wielomian jest stopnia parzystego, ponadto znak współczynnika przy\linebreak najwyższej potędze x jest ujemny.\\ W związku z tym wykres wielomianu zaczyna się od lewej strony powyżej osi OX.\\
Ponadto w punkcie $-8$ wykres odbija się od osi poziomej.\\
A więc $$x \in \{-8\} \cup [9,14].$$
\rozwStop
\odpStart
$x \in \{-8\} \cup [9,14]$
\odpStop
\testStart
A.$x \in \{-8\} \cup [9,14]$\\
B.$x \in \{8\} \cup (9,14)$\\
C.$x \in \{-8\} \cup (9,14]$\\
D.$x \in \{8\} \cup (9,14]$\\
E.$x \in \{-8\} \cup [9,14)$\\
F.$x \in \{8\} \cup [9,14)$\\
G.$x \in \{-8\} \cup (9,14)$\\
H.$x \in \{8\} \cup [9,14]$
\testStop
\kluczStart
A
\kluczStop



\zadStart{Zadanie z Wikieł Z 1.62 c) moja wersja nr 475}

Rozwiązać nierówności $(9-x)(x+8)^{2}(15-x)^{3}\le0$.
\zadStop
\rozwStart{Patryk Wirkus}{}
Miejsca zerowe naszego wielomianu to: $9, -8, 15$.\\
Wielomian jest stopnia parzystego, ponadto znak współczynnika przy\linebreak najwyższej potędze x jest ujemny.\\ W związku z tym wykres wielomianu zaczyna się od lewej strony powyżej osi OX.\\
Ponadto w punkcie $-8$ wykres odbija się od osi poziomej.\\
A więc $$x \in \{-8\} \cup [9,15].$$
\rozwStop
\odpStart
$x \in \{-8\} \cup [9,15]$
\odpStop
\testStart
A.$x \in \{-8\} \cup [9,15]$\\
B.$x \in \{8\} \cup (9,15)$\\
C.$x \in \{-8\} \cup (9,15]$\\
D.$x \in \{8\} \cup (9,15]$\\
E.$x \in \{-8\} \cup [9,15)$\\
F.$x \in \{8\} \cup [9,15)$\\
G.$x \in \{-8\} \cup (9,15)$\\
H.$x \in \{8\} \cup [9,15]$
\testStop
\kluczStart
A
\kluczStop



\zadStart{Zadanie z Wikieł Z 1.62 c) moja wersja nr 476}

Rozwiązać nierówności $(9-x)(x+8)^{2}(16-x)^{3}\le0$.
\zadStop
\rozwStart{Patryk Wirkus}{}
Miejsca zerowe naszego wielomianu to: $9, -8, 16$.\\
Wielomian jest stopnia parzystego, ponadto znak współczynnika przy\linebreak najwyższej potędze x jest ujemny.\\ W związku z tym wykres wielomianu zaczyna się od lewej strony powyżej osi OX.\\
Ponadto w punkcie $-8$ wykres odbija się od osi poziomej.\\
A więc $$x \in \{-8\} \cup [9,16].$$
\rozwStop
\odpStart
$x \in \{-8\} \cup [9,16]$
\odpStop
\testStart
A.$x \in \{-8\} \cup [9,16]$\\
B.$x \in \{8\} \cup (9,16)$\\
C.$x \in \{-8\} \cup (9,16]$\\
D.$x \in \{8\} \cup (9,16]$\\
E.$x \in \{-8\} \cup [9,16)$\\
F.$x \in \{8\} \cup [9,16)$\\
G.$x \in \{-8\} \cup (9,16)$\\
H.$x \in \{8\} \cup [9,16]$
\testStop
\kluczStart
A
\kluczStop



\zadStart{Zadanie z Wikieł Z 1.62 c) moja wersja nr 477}

Rozwiązać nierówności $(9-x)(x+8)^{2}(17-x)^{3}\le0$.
\zadStop
\rozwStart{Patryk Wirkus}{}
Miejsca zerowe naszego wielomianu to: $9, -8, 17$.\\
Wielomian jest stopnia parzystego, ponadto znak współczynnika przy\linebreak najwyższej potędze x jest ujemny.\\ W związku z tym wykres wielomianu zaczyna się od lewej strony powyżej osi OX.\\
Ponadto w punkcie $-8$ wykres odbija się od osi poziomej.\\
A więc $$x \in \{-8\} \cup [9,17].$$
\rozwStop
\odpStart
$x \in \{-8\} \cup [9,17]$
\odpStop
\testStart
A.$x \in \{-8\} \cup [9,17]$\\
B.$x \in \{8\} \cup (9,17)$\\
C.$x \in \{-8\} \cup (9,17]$\\
D.$x \in \{8\} \cup (9,17]$\\
E.$x \in \{-8\} \cup [9,17)$\\
F.$x \in \{8\} \cup [9,17)$\\
G.$x \in \{-8\} \cup (9,17)$\\
H.$x \in \{8\} \cup [9,17]$
\testStop
\kluczStart
A
\kluczStop



\zadStart{Zadanie z Wikieł Z 1.62 c) moja wersja nr 478}

Rozwiązać nierówności $(9-x)(x+8)^{2}(18-x)^{3}\le0$.
\zadStop
\rozwStart{Patryk Wirkus}{}
Miejsca zerowe naszego wielomianu to: $9, -8, 18$.\\
Wielomian jest stopnia parzystego, ponadto znak współczynnika przy\linebreak najwyższej potędze x jest ujemny.\\ W związku z tym wykres wielomianu zaczyna się od lewej strony powyżej osi OX.\\
Ponadto w punkcie $-8$ wykres odbija się od osi poziomej.\\
A więc $$x \in \{-8\} \cup [9,18].$$
\rozwStop
\odpStart
$x \in \{-8\} \cup [9,18]$
\odpStop
\testStart
A.$x \in \{-8\} \cup [9,18]$\\
B.$x \in \{8\} \cup (9,18)$\\
C.$x \in \{-8\} \cup (9,18]$\\
D.$x \in \{8\} \cup (9,18]$\\
E.$x \in \{-8\} \cup [9,18)$\\
F.$x \in \{8\} \cup [9,18)$\\
G.$x \in \{-8\} \cup (9,18)$\\
H.$x \in \{8\} \cup [9,18]$
\testStop
\kluczStart
A
\kluczStop



\zadStart{Zadanie z Wikieł Z 1.62 c) moja wersja nr 479}

Rozwiązać nierówności $(9-x)(x+8)^{2}(19-x)^{3}\le0$.
\zadStop
\rozwStart{Patryk Wirkus}{}
Miejsca zerowe naszego wielomianu to: $9, -8, 19$.\\
Wielomian jest stopnia parzystego, ponadto znak współczynnika przy\linebreak najwyższej potędze x jest ujemny.\\ W związku z tym wykres wielomianu zaczyna się od lewej strony powyżej osi OX.\\
Ponadto w punkcie $-8$ wykres odbija się od osi poziomej.\\
A więc $$x \in \{-8\} \cup [9,19].$$
\rozwStop
\odpStart
$x \in \{-8\} \cup [9,19]$
\odpStop
\testStart
A.$x \in \{-8\} \cup [9,19]$\\
B.$x \in \{8\} \cup (9,19)$\\
C.$x \in \{-8\} \cup (9,19]$\\
D.$x \in \{8\} \cup (9,19]$\\
E.$x \in \{-8\} \cup [9,19)$\\
F.$x \in \{8\} \cup [9,19)$\\
G.$x \in \{-8\} \cup (9,19)$\\
H.$x \in \{8\} \cup [9,19]$
\testStop
\kluczStart
A
\kluczStop



\zadStart{Zadanie z Wikieł Z 1.62 c) moja wersja nr 480}

Rozwiązać nierówności $(9-x)(x+8)^{2}(20-x)^{3}\le0$.
\zadStop
\rozwStart{Patryk Wirkus}{}
Miejsca zerowe naszego wielomianu to: $9, -8, 20$.\\
Wielomian jest stopnia parzystego, ponadto znak współczynnika przy\linebreak najwyższej potędze x jest ujemny.\\ W związku z tym wykres wielomianu zaczyna się od lewej strony powyżej osi OX.\\
Ponadto w punkcie $-8$ wykres odbija się od osi poziomej.\\
A więc $$x \in \{-8\} \cup [9,20].$$
\rozwStop
\odpStart
$x \in \{-8\} \cup [9,20]$
\odpStop
\testStart
A.$x \in \{-8\} \cup [9,20]$\\
B.$x \in \{8\} \cup (9,20)$\\
C.$x \in \{-8\} \cup (9,20]$\\
D.$x \in \{8\} \cup (9,20]$\\
E.$x \in \{-8\} \cup [9,20)$\\
F.$x \in \{8\} \cup [9,20)$\\
G.$x \in \{-8\} \cup (9,20)$\\
H.$x \in \{8\} \cup [9,20]$
\testStop
\kluczStart
A
\kluczStop



\zadStart{Zadanie z Wikieł Z 1.62 c) moja wersja nr 481}

Rozwiązać nierówności $(10-x)(x+1)^{2}(11-x)^{3}\le0$.
\zadStop
\rozwStart{Patryk Wirkus}{}
Miejsca zerowe naszego wielomianu to: $10, -1, 11$.\\
Wielomian jest stopnia parzystego, ponadto znak współczynnika przy\linebreak najwyższej potędze x jest ujemny.\\ W związku z tym wykres wielomianu zaczyna się od lewej strony powyżej osi OX.\\
Ponadto w punkcie $-1$ wykres odbija się od osi poziomej.\\
A więc $$x \in \{-1\} \cup [10,11].$$
\rozwStop
\odpStart
$x \in \{-1\} \cup [10,11]$
\odpStop
\testStart
A.$x \in \{-1\} \cup [10,11]$\\
B.$x \in \{1\} \cup (10,11)$\\
C.$x \in \{-1\} \cup (10,11]$\\
D.$x \in \{1\} \cup (10,11]$\\
E.$x \in \{-1\} \cup [10,11)$\\
F.$x \in \{1\} \cup [10,11)$\\
G.$x \in \{-1\} \cup (10,11)$\\
H.$x \in \{1\} \cup [10,11]$
\testStop
\kluczStart
A
\kluczStop



\zadStart{Zadanie z Wikieł Z 1.62 c) moja wersja nr 482}

Rozwiązać nierówności $(10-x)(x+1)^{2}(12-x)^{3}\le0$.
\zadStop
\rozwStart{Patryk Wirkus}{}
Miejsca zerowe naszego wielomianu to: $10, -1, 12$.\\
Wielomian jest stopnia parzystego, ponadto znak współczynnika przy\linebreak najwyższej potędze x jest ujemny.\\ W związku z tym wykres wielomianu zaczyna się od lewej strony powyżej osi OX.\\
Ponadto w punkcie $-1$ wykres odbija się od osi poziomej.\\
A więc $$x \in \{-1\} \cup [10,12].$$
\rozwStop
\odpStart
$x \in \{-1\} \cup [10,12]$
\odpStop
\testStart
A.$x \in \{-1\} \cup [10,12]$\\
B.$x \in \{1\} \cup (10,12)$\\
C.$x \in \{-1\} \cup (10,12]$\\
D.$x \in \{1\} \cup (10,12]$\\
E.$x \in \{-1\} \cup [10,12)$\\
F.$x \in \{1\} \cup [10,12)$\\
G.$x \in \{-1\} \cup (10,12)$\\
H.$x \in \{1\} \cup [10,12]$
\testStop
\kluczStart
A
\kluczStop



\zadStart{Zadanie z Wikieł Z 1.62 c) moja wersja nr 483}

Rozwiązać nierówności $(10-x)(x+1)^{2}(13-x)^{3}\le0$.
\zadStop
\rozwStart{Patryk Wirkus}{}
Miejsca zerowe naszego wielomianu to: $10, -1, 13$.\\
Wielomian jest stopnia parzystego, ponadto znak współczynnika przy\linebreak najwyższej potędze x jest ujemny.\\ W związku z tym wykres wielomianu zaczyna się od lewej strony powyżej osi OX.\\
Ponadto w punkcie $-1$ wykres odbija się od osi poziomej.\\
A więc $$x \in \{-1\} \cup [10,13].$$
\rozwStop
\odpStart
$x \in \{-1\} \cup [10,13]$
\odpStop
\testStart
A.$x \in \{-1\} \cup [10,13]$\\
B.$x \in \{1\} \cup (10,13)$\\
C.$x \in \{-1\} \cup (10,13]$\\
D.$x \in \{1\} \cup (10,13]$\\
E.$x \in \{-1\} \cup [10,13)$\\
F.$x \in \{1\} \cup [10,13)$\\
G.$x \in \{-1\} \cup (10,13)$\\
H.$x \in \{1\} \cup [10,13]$
\testStop
\kluczStart
A
\kluczStop



\zadStart{Zadanie z Wikieł Z 1.62 c) moja wersja nr 484}

Rozwiązać nierówności $(10-x)(x+1)^{2}(14-x)^{3}\le0$.
\zadStop
\rozwStart{Patryk Wirkus}{}
Miejsca zerowe naszego wielomianu to: $10, -1, 14$.\\
Wielomian jest stopnia parzystego, ponadto znak współczynnika przy\linebreak najwyższej potędze x jest ujemny.\\ W związku z tym wykres wielomianu zaczyna się od lewej strony powyżej osi OX.\\
Ponadto w punkcie $-1$ wykres odbija się od osi poziomej.\\
A więc $$x \in \{-1\} \cup [10,14].$$
\rozwStop
\odpStart
$x \in \{-1\} \cup [10,14]$
\odpStop
\testStart
A.$x \in \{-1\} \cup [10,14]$\\
B.$x \in \{1\} \cup (10,14)$\\
C.$x \in \{-1\} \cup (10,14]$\\
D.$x \in \{1\} \cup (10,14]$\\
E.$x \in \{-1\} \cup [10,14)$\\
F.$x \in \{1\} \cup [10,14)$\\
G.$x \in \{-1\} \cup (10,14)$\\
H.$x \in \{1\} \cup [10,14]$
\testStop
\kluczStart
A
\kluczStop



\zadStart{Zadanie z Wikieł Z 1.62 c) moja wersja nr 485}

Rozwiązać nierówności $(10-x)(x+1)^{2}(15-x)^{3}\le0$.
\zadStop
\rozwStart{Patryk Wirkus}{}
Miejsca zerowe naszego wielomianu to: $10, -1, 15$.\\
Wielomian jest stopnia parzystego, ponadto znak współczynnika przy\linebreak najwyższej potędze x jest ujemny.\\ W związku z tym wykres wielomianu zaczyna się od lewej strony powyżej osi OX.\\
Ponadto w punkcie $-1$ wykres odbija się od osi poziomej.\\
A więc $$x \in \{-1\} \cup [10,15].$$
\rozwStop
\odpStart
$x \in \{-1\} \cup [10,15]$
\odpStop
\testStart
A.$x \in \{-1\} \cup [10,15]$\\
B.$x \in \{1\} \cup (10,15)$\\
C.$x \in \{-1\} \cup (10,15]$\\
D.$x \in \{1\} \cup (10,15]$\\
E.$x \in \{-1\} \cup [10,15)$\\
F.$x \in \{1\} \cup [10,15)$\\
G.$x \in \{-1\} \cup (10,15)$\\
H.$x \in \{1\} \cup [10,15]$
\testStop
\kluczStart
A
\kluczStop



\zadStart{Zadanie z Wikieł Z 1.62 c) moja wersja nr 486}

Rozwiązać nierówności $(10-x)(x+1)^{2}(16-x)^{3}\le0$.
\zadStop
\rozwStart{Patryk Wirkus}{}
Miejsca zerowe naszego wielomianu to: $10, -1, 16$.\\
Wielomian jest stopnia parzystego, ponadto znak współczynnika przy\linebreak najwyższej potędze x jest ujemny.\\ W związku z tym wykres wielomianu zaczyna się od lewej strony powyżej osi OX.\\
Ponadto w punkcie $-1$ wykres odbija się od osi poziomej.\\
A więc $$x \in \{-1\} \cup [10,16].$$
\rozwStop
\odpStart
$x \in \{-1\} \cup [10,16]$
\odpStop
\testStart
A.$x \in \{-1\} \cup [10,16]$\\
B.$x \in \{1\} \cup (10,16)$\\
C.$x \in \{-1\} \cup (10,16]$\\
D.$x \in \{1\} \cup (10,16]$\\
E.$x \in \{-1\} \cup [10,16)$\\
F.$x \in \{1\} \cup [10,16)$\\
G.$x \in \{-1\} \cup (10,16)$\\
H.$x \in \{1\} \cup [10,16]$
\testStop
\kluczStart
A
\kluczStop



\zadStart{Zadanie z Wikieł Z 1.62 c) moja wersja nr 487}

Rozwiązać nierówności $(10-x)(x+1)^{2}(17-x)^{3}\le0$.
\zadStop
\rozwStart{Patryk Wirkus}{}
Miejsca zerowe naszego wielomianu to: $10, -1, 17$.\\
Wielomian jest stopnia parzystego, ponadto znak współczynnika przy\linebreak najwyższej potędze x jest ujemny.\\ W związku z tym wykres wielomianu zaczyna się od lewej strony powyżej osi OX.\\
Ponadto w punkcie $-1$ wykres odbija się od osi poziomej.\\
A więc $$x \in \{-1\} \cup [10,17].$$
\rozwStop
\odpStart
$x \in \{-1\} \cup [10,17]$
\odpStop
\testStart
A.$x \in \{-1\} \cup [10,17]$\\
B.$x \in \{1\} \cup (10,17)$\\
C.$x \in \{-1\} \cup (10,17]$\\
D.$x \in \{1\} \cup (10,17]$\\
E.$x \in \{-1\} \cup [10,17)$\\
F.$x \in \{1\} \cup [10,17)$\\
G.$x \in \{-1\} \cup (10,17)$\\
H.$x \in \{1\} \cup [10,17]$
\testStop
\kluczStart
A
\kluczStop



\zadStart{Zadanie z Wikieł Z 1.62 c) moja wersja nr 488}

Rozwiązać nierówności $(10-x)(x+1)^{2}(18-x)^{3}\le0$.
\zadStop
\rozwStart{Patryk Wirkus}{}
Miejsca zerowe naszego wielomianu to: $10, -1, 18$.\\
Wielomian jest stopnia parzystego, ponadto znak współczynnika przy\linebreak najwyższej potędze x jest ujemny.\\ W związku z tym wykres wielomianu zaczyna się od lewej strony powyżej osi OX.\\
Ponadto w punkcie $-1$ wykres odbija się od osi poziomej.\\
A więc $$x \in \{-1\} \cup [10,18].$$
\rozwStop
\odpStart
$x \in \{-1\} \cup [10,18]$
\odpStop
\testStart
A.$x \in \{-1\} \cup [10,18]$\\
B.$x \in \{1\} \cup (10,18)$\\
C.$x \in \{-1\} \cup (10,18]$\\
D.$x \in \{1\} \cup (10,18]$\\
E.$x \in \{-1\} \cup [10,18)$\\
F.$x \in \{1\} \cup [10,18)$\\
G.$x \in \{-1\} \cup (10,18)$\\
H.$x \in \{1\} \cup [10,18]$
\testStop
\kluczStart
A
\kluczStop



\zadStart{Zadanie z Wikieł Z 1.62 c) moja wersja nr 489}

Rozwiązać nierówności $(10-x)(x+1)^{2}(19-x)^{3}\le0$.
\zadStop
\rozwStart{Patryk Wirkus}{}
Miejsca zerowe naszego wielomianu to: $10, -1, 19$.\\
Wielomian jest stopnia parzystego, ponadto znak współczynnika przy\linebreak najwyższej potędze x jest ujemny.\\ W związku z tym wykres wielomianu zaczyna się od lewej strony powyżej osi OX.\\
Ponadto w punkcie $-1$ wykres odbija się od osi poziomej.\\
A więc $$x \in \{-1\} \cup [10,19].$$
\rozwStop
\odpStart
$x \in \{-1\} \cup [10,19]$
\odpStop
\testStart
A.$x \in \{-1\} \cup [10,19]$\\
B.$x \in \{1\} \cup (10,19)$\\
C.$x \in \{-1\} \cup (10,19]$\\
D.$x \in \{1\} \cup (10,19]$\\
E.$x \in \{-1\} \cup [10,19)$\\
F.$x \in \{1\} \cup [10,19)$\\
G.$x \in \{-1\} \cup (10,19)$\\
H.$x \in \{1\} \cup [10,19]$
\testStop
\kluczStart
A
\kluczStop



\zadStart{Zadanie z Wikieł Z 1.62 c) moja wersja nr 490}

Rozwiązać nierówności $(10-x)(x+1)^{2}(20-x)^{3}\le0$.
\zadStop
\rozwStart{Patryk Wirkus}{}
Miejsca zerowe naszego wielomianu to: $10, -1, 20$.\\
Wielomian jest stopnia parzystego, ponadto znak współczynnika przy\linebreak najwyższej potędze x jest ujemny.\\ W związku z tym wykres wielomianu zaczyna się od lewej strony powyżej osi OX.\\
Ponadto w punkcie $-1$ wykres odbija się od osi poziomej.\\
A więc $$x \in \{-1\} \cup [10,20].$$
\rozwStop
\odpStart
$x \in \{-1\} \cup [10,20]$
\odpStop
\testStart
A.$x \in \{-1\} \cup [10,20]$\\
B.$x \in \{1\} \cup (10,20)$\\
C.$x \in \{-1\} \cup (10,20]$\\
D.$x \in \{1\} \cup (10,20]$\\
E.$x \in \{-1\} \cup [10,20)$\\
F.$x \in \{1\} \cup [10,20)$\\
G.$x \in \{-1\} \cup (10,20)$\\
H.$x \in \{1\} \cup [10,20]$
\testStop
\kluczStart
A
\kluczStop



\zadStart{Zadanie z Wikieł Z 1.62 c) moja wersja nr 491}

Rozwiązać nierówności $(10-x)(x+2)^{2}(11-x)^{3}\le0$.
\zadStop
\rozwStart{Patryk Wirkus}{}
Miejsca zerowe naszego wielomianu to: $10, -2, 11$.\\
Wielomian jest stopnia parzystego, ponadto znak współczynnika przy\linebreak najwyższej potędze x jest ujemny.\\ W związku z tym wykres wielomianu zaczyna się od lewej strony powyżej osi OX.\\
Ponadto w punkcie $-2$ wykres odbija się od osi poziomej.\\
A więc $$x \in \{-2\} \cup [10,11].$$
\rozwStop
\odpStart
$x \in \{-2\} \cup [10,11]$
\odpStop
\testStart
A.$x \in \{-2\} \cup [10,11]$\\
B.$x \in \{2\} \cup (10,11)$\\
C.$x \in \{-2\} \cup (10,11]$\\
D.$x \in \{2\} \cup (10,11]$\\
E.$x \in \{-2\} \cup [10,11)$\\
F.$x \in \{2\} \cup [10,11)$\\
G.$x \in \{-2\} \cup (10,11)$\\
H.$x \in \{2\} \cup [10,11]$
\testStop
\kluczStart
A
\kluczStop



\zadStart{Zadanie z Wikieł Z 1.62 c) moja wersja nr 492}

Rozwiązać nierówności $(10-x)(x+2)^{2}(12-x)^{3}\le0$.
\zadStop
\rozwStart{Patryk Wirkus}{}
Miejsca zerowe naszego wielomianu to: $10, -2, 12$.\\
Wielomian jest stopnia parzystego, ponadto znak współczynnika przy\linebreak najwyższej potędze x jest ujemny.\\ W związku z tym wykres wielomianu zaczyna się od lewej strony powyżej osi OX.\\
Ponadto w punkcie $-2$ wykres odbija się od osi poziomej.\\
A więc $$x \in \{-2\} \cup [10,12].$$
\rozwStop
\odpStart
$x \in \{-2\} \cup [10,12]$
\odpStop
\testStart
A.$x \in \{-2\} \cup [10,12]$\\
B.$x \in \{2\} \cup (10,12)$\\
C.$x \in \{-2\} \cup (10,12]$\\
D.$x \in \{2\} \cup (10,12]$\\
E.$x \in \{-2\} \cup [10,12)$\\
F.$x \in \{2\} \cup [10,12)$\\
G.$x \in \{-2\} \cup (10,12)$\\
H.$x \in \{2\} \cup [10,12]$
\testStop
\kluczStart
A
\kluczStop



\zadStart{Zadanie z Wikieł Z 1.62 c) moja wersja nr 493}

Rozwiązać nierówności $(10-x)(x+2)^{2}(13-x)^{3}\le0$.
\zadStop
\rozwStart{Patryk Wirkus}{}
Miejsca zerowe naszego wielomianu to: $10, -2, 13$.\\
Wielomian jest stopnia parzystego, ponadto znak współczynnika przy\linebreak najwyższej potędze x jest ujemny.\\ W związku z tym wykres wielomianu zaczyna się od lewej strony powyżej osi OX.\\
Ponadto w punkcie $-2$ wykres odbija się od osi poziomej.\\
A więc $$x \in \{-2\} \cup [10,13].$$
\rozwStop
\odpStart
$x \in \{-2\} \cup [10,13]$
\odpStop
\testStart
A.$x \in \{-2\} \cup [10,13]$\\
B.$x \in \{2\} \cup (10,13)$\\
C.$x \in \{-2\} \cup (10,13]$\\
D.$x \in \{2\} \cup (10,13]$\\
E.$x \in \{-2\} \cup [10,13)$\\
F.$x \in \{2\} \cup [10,13)$\\
G.$x \in \{-2\} \cup (10,13)$\\
H.$x \in \{2\} \cup [10,13]$
\testStop
\kluczStart
A
\kluczStop



\zadStart{Zadanie z Wikieł Z 1.62 c) moja wersja nr 494}

Rozwiązać nierówności $(10-x)(x+2)^{2}(14-x)^{3}\le0$.
\zadStop
\rozwStart{Patryk Wirkus}{}
Miejsca zerowe naszego wielomianu to: $10, -2, 14$.\\
Wielomian jest stopnia parzystego, ponadto znak współczynnika przy\linebreak najwyższej potędze x jest ujemny.\\ W związku z tym wykres wielomianu zaczyna się od lewej strony powyżej osi OX.\\
Ponadto w punkcie $-2$ wykres odbija się od osi poziomej.\\
A więc $$x \in \{-2\} \cup [10,14].$$
\rozwStop
\odpStart
$x \in \{-2\} \cup [10,14]$
\odpStop
\testStart
A.$x \in \{-2\} \cup [10,14]$\\
B.$x \in \{2\} \cup (10,14)$\\
C.$x \in \{-2\} \cup (10,14]$\\
D.$x \in \{2\} \cup (10,14]$\\
E.$x \in \{-2\} \cup [10,14)$\\
F.$x \in \{2\} \cup [10,14)$\\
G.$x \in \{-2\} \cup (10,14)$\\
H.$x \in \{2\} \cup [10,14]$
\testStop
\kluczStart
A
\kluczStop



\zadStart{Zadanie z Wikieł Z 1.62 c) moja wersja nr 495}

Rozwiązać nierówności $(10-x)(x+2)^{2}(15-x)^{3}\le0$.
\zadStop
\rozwStart{Patryk Wirkus}{}
Miejsca zerowe naszego wielomianu to: $10, -2, 15$.\\
Wielomian jest stopnia parzystego, ponadto znak współczynnika przy\linebreak najwyższej potędze x jest ujemny.\\ W związku z tym wykres wielomianu zaczyna się od lewej strony powyżej osi OX.\\
Ponadto w punkcie $-2$ wykres odbija się od osi poziomej.\\
A więc $$x \in \{-2\} \cup [10,15].$$
\rozwStop
\odpStart
$x \in \{-2\} \cup [10,15]$
\odpStop
\testStart
A.$x \in \{-2\} \cup [10,15]$\\
B.$x \in \{2\} \cup (10,15)$\\
C.$x \in \{-2\} \cup (10,15]$\\
D.$x \in \{2\} \cup (10,15]$\\
E.$x \in \{-2\} \cup [10,15)$\\
F.$x \in \{2\} \cup [10,15)$\\
G.$x \in \{-2\} \cup (10,15)$\\
H.$x \in \{2\} \cup [10,15]$
\testStop
\kluczStart
A
\kluczStop



\zadStart{Zadanie z Wikieł Z 1.62 c) moja wersja nr 496}

Rozwiązać nierówności $(10-x)(x+2)^{2}(16-x)^{3}\le0$.
\zadStop
\rozwStart{Patryk Wirkus}{}
Miejsca zerowe naszego wielomianu to: $10, -2, 16$.\\
Wielomian jest stopnia parzystego, ponadto znak współczynnika przy\linebreak najwyższej potędze x jest ujemny.\\ W związku z tym wykres wielomianu zaczyna się od lewej strony powyżej osi OX.\\
Ponadto w punkcie $-2$ wykres odbija się od osi poziomej.\\
A więc $$x \in \{-2\} \cup [10,16].$$
\rozwStop
\odpStart
$x \in \{-2\} \cup [10,16]$
\odpStop
\testStart
A.$x \in \{-2\} \cup [10,16]$\\
B.$x \in \{2\} \cup (10,16)$\\
C.$x \in \{-2\} \cup (10,16]$\\
D.$x \in \{2\} \cup (10,16]$\\
E.$x \in \{-2\} \cup [10,16)$\\
F.$x \in \{2\} \cup [10,16)$\\
G.$x \in \{-2\} \cup (10,16)$\\
H.$x \in \{2\} \cup [10,16]$
\testStop
\kluczStart
A
\kluczStop



\zadStart{Zadanie z Wikieł Z 1.62 c) moja wersja nr 497}

Rozwiązać nierówności $(10-x)(x+2)^{2}(17-x)^{3}\le0$.
\zadStop
\rozwStart{Patryk Wirkus}{}
Miejsca zerowe naszego wielomianu to: $10, -2, 17$.\\
Wielomian jest stopnia parzystego, ponadto znak współczynnika przy\linebreak najwyższej potędze x jest ujemny.\\ W związku z tym wykres wielomianu zaczyna się od lewej strony powyżej osi OX.\\
Ponadto w punkcie $-2$ wykres odbija się od osi poziomej.\\
A więc $$x \in \{-2\} \cup [10,17].$$
\rozwStop
\odpStart
$x \in \{-2\} \cup [10,17]$
\odpStop
\testStart
A.$x \in \{-2\} \cup [10,17]$\\
B.$x \in \{2\} \cup (10,17)$\\
C.$x \in \{-2\} \cup (10,17]$\\
D.$x \in \{2\} \cup (10,17]$\\
E.$x \in \{-2\} \cup [10,17)$\\
F.$x \in \{2\} \cup [10,17)$\\
G.$x \in \{-2\} \cup (10,17)$\\
H.$x \in \{2\} \cup [10,17]$
\testStop
\kluczStart
A
\kluczStop



\zadStart{Zadanie z Wikieł Z 1.62 c) moja wersja nr 498}

Rozwiązać nierówności $(10-x)(x+2)^{2}(18-x)^{3}\le0$.
\zadStop
\rozwStart{Patryk Wirkus}{}
Miejsca zerowe naszego wielomianu to: $10, -2, 18$.\\
Wielomian jest stopnia parzystego, ponadto znak współczynnika przy\linebreak najwyższej potędze x jest ujemny.\\ W związku z tym wykres wielomianu zaczyna się od lewej strony powyżej osi OX.\\
Ponadto w punkcie $-2$ wykres odbija się od osi poziomej.\\
A więc $$x \in \{-2\} \cup [10,18].$$
\rozwStop
\odpStart
$x \in \{-2\} \cup [10,18]$
\odpStop
\testStart
A.$x \in \{-2\} \cup [10,18]$\\
B.$x \in \{2\} \cup (10,18)$\\
C.$x \in \{-2\} \cup (10,18]$\\
D.$x \in \{2\} \cup (10,18]$\\
E.$x \in \{-2\} \cup [10,18)$\\
F.$x \in \{2\} \cup [10,18)$\\
G.$x \in \{-2\} \cup (10,18)$\\
H.$x \in \{2\} \cup [10,18]$
\testStop
\kluczStart
A
\kluczStop



\zadStart{Zadanie z Wikieł Z 1.62 c) moja wersja nr 499}

Rozwiązać nierówności $(10-x)(x+2)^{2}(19-x)^{3}\le0$.
\zadStop
\rozwStart{Patryk Wirkus}{}
Miejsca zerowe naszego wielomianu to: $10, -2, 19$.\\
Wielomian jest stopnia parzystego, ponadto znak współczynnika przy\linebreak najwyższej potędze x jest ujemny.\\ W związku z tym wykres wielomianu zaczyna się od lewej strony powyżej osi OX.\\
Ponadto w punkcie $-2$ wykres odbija się od osi poziomej.\\
A więc $$x \in \{-2\} \cup [10,19].$$
\rozwStop
\odpStart
$x \in \{-2\} \cup [10,19]$
\odpStop
\testStart
A.$x \in \{-2\} \cup [10,19]$\\
B.$x \in \{2\} \cup (10,19)$\\
C.$x \in \{-2\} \cup (10,19]$\\
D.$x \in \{2\} \cup (10,19]$\\
E.$x \in \{-2\} \cup [10,19)$\\
F.$x \in \{2\} \cup [10,19)$\\
G.$x \in \{-2\} \cup (10,19)$\\
H.$x \in \{2\} \cup [10,19]$
\testStop
\kluczStart
A
\kluczStop



\zadStart{Zadanie z Wikieł Z 1.62 c) moja wersja nr 500}

Rozwiązać nierówności $(10-x)(x+2)^{2}(20-x)^{3}\le0$.
\zadStop
\rozwStart{Patryk Wirkus}{}
Miejsca zerowe naszego wielomianu to: $10, -2, 20$.\\
Wielomian jest stopnia parzystego, ponadto znak współczynnika przy\linebreak najwyższej potędze x jest ujemny.\\ W związku z tym wykres wielomianu zaczyna się od lewej strony powyżej osi OX.\\
Ponadto w punkcie $-2$ wykres odbija się od osi poziomej.\\
A więc $$x \in \{-2\} \cup [10,20].$$
\rozwStop
\odpStart
$x \in \{-2\} \cup [10,20]$
\odpStop
\testStart
A.$x \in \{-2\} \cup [10,20]$\\
B.$x \in \{2\} \cup (10,20)$\\
C.$x \in \{-2\} \cup (10,20]$\\
D.$x \in \{2\} \cup (10,20]$\\
E.$x \in \{-2\} \cup [10,20)$\\
F.$x \in \{2\} \cup [10,20)$\\
G.$x \in \{-2\} \cup (10,20)$\\
H.$x \in \{2\} \cup [10,20]$
\testStop
\kluczStart
A
\kluczStop



\zadStart{Zadanie z Wikieł Z 1.62 c) moja wersja nr 501}

Rozwiązać nierówności $(10-x)(x+3)^{2}(11-x)^{3}\le0$.
\zadStop
\rozwStart{Patryk Wirkus}{}
Miejsca zerowe naszego wielomianu to: $10, -3, 11$.\\
Wielomian jest stopnia parzystego, ponadto znak współczynnika przy\linebreak najwyższej potędze x jest ujemny.\\ W związku z tym wykres wielomianu zaczyna się od lewej strony powyżej osi OX.\\
Ponadto w punkcie $-3$ wykres odbija się od osi poziomej.\\
A więc $$x \in \{-3\} \cup [10,11].$$
\rozwStop
\odpStart
$x \in \{-3\} \cup [10,11]$
\odpStop
\testStart
A.$x \in \{-3\} \cup [10,11]$\\
B.$x \in \{3\} \cup (10,11)$\\
C.$x \in \{-3\} \cup (10,11]$\\
D.$x \in \{3\} \cup (10,11]$\\
E.$x \in \{-3\} \cup [10,11)$\\
F.$x \in \{3\} \cup [10,11)$\\
G.$x \in \{-3\} \cup (10,11)$\\
H.$x \in \{3\} \cup [10,11]$
\testStop
\kluczStart
A
\kluczStop



\zadStart{Zadanie z Wikieł Z 1.62 c) moja wersja nr 502}

Rozwiązać nierówności $(10-x)(x+3)^{2}(12-x)^{3}\le0$.
\zadStop
\rozwStart{Patryk Wirkus}{}
Miejsca zerowe naszego wielomianu to: $10, -3, 12$.\\
Wielomian jest stopnia parzystego, ponadto znak współczynnika przy\linebreak najwyższej potędze x jest ujemny.\\ W związku z tym wykres wielomianu zaczyna się od lewej strony powyżej osi OX.\\
Ponadto w punkcie $-3$ wykres odbija się od osi poziomej.\\
A więc $$x \in \{-3\} \cup [10,12].$$
\rozwStop
\odpStart
$x \in \{-3\} \cup [10,12]$
\odpStop
\testStart
A.$x \in \{-3\} \cup [10,12]$\\
B.$x \in \{3\} \cup (10,12)$\\
C.$x \in \{-3\} \cup (10,12]$\\
D.$x \in \{3\} \cup (10,12]$\\
E.$x \in \{-3\} \cup [10,12)$\\
F.$x \in \{3\} \cup [10,12)$\\
G.$x \in \{-3\} \cup (10,12)$\\
H.$x \in \{3\} \cup [10,12]$
\testStop
\kluczStart
A
\kluczStop



\zadStart{Zadanie z Wikieł Z 1.62 c) moja wersja nr 503}

Rozwiązać nierówności $(10-x)(x+3)^{2}(13-x)^{3}\le0$.
\zadStop
\rozwStart{Patryk Wirkus}{}
Miejsca zerowe naszego wielomianu to: $10, -3, 13$.\\
Wielomian jest stopnia parzystego, ponadto znak współczynnika przy\linebreak najwyższej potędze x jest ujemny.\\ W związku z tym wykres wielomianu zaczyna się od lewej strony powyżej osi OX.\\
Ponadto w punkcie $-3$ wykres odbija się od osi poziomej.\\
A więc $$x \in \{-3\} \cup [10,13].$$
\rozwStop
\odpStart
$x \in \{-3\} \cup [10,13]$
\odpStop
\testStart
A.$x \in \{-3\} \cup [10,13]$\\
B.$x \in \{3\} \cup (10,13)$\\
C.$x \in \{-3\} \cup (10,13]$\\
D.$x \in \{3\} \cup (10,13]$\\
E.$x \in \{-3\} \cup [10,13)$\\
F.$x \in \{3\} \cup [10,13)$\\
G.$x \in \{-3\} \cup (10,13)$\\
H.$x \in \{3\} \cup [10,13]$
\testStop
\kluczStart
A
\kluczStop



\zadStart{Zadanie z Wikieł Z 1.62 c) moja wersja nr 504}

Rozwiązać nierówności $(10-x)(x+3)^{2}(14-x)^{3}\le0$.
\zadStop
\rozwStart{Patryk Wirkus}{}
Miejsca zerowe naszego wielomianu to: $10, -3, 14$.\\
Wielomian jest stopnia parzystego, ponadto znak współczynnika przy\linebreak najwyższej potędze x jest ujemny.\\ W związku z tym wykres wielomianu zaczyna się od lewej strony powyżej osi OX.\\
Ponadto w punkcie $-3$ wykres odbija się od osi poziomej.\\
A więc $$x \in \{-3\} \cup [10,14].$$
\rozwStop
\odpStart
$x \in \{-3\} \cup [10,14]$
\odpStop
\testStart
A.$x \in \{-3\} \cup [10,14]$\\
B.$x \in \{3\} \cup (10,14)$\\
C.$x \in \{-3\} \cup (10,14]$\\
D.$x \in \{3\} \cup (10,14]$\\
E.$x \in \{-3\} \cup [10,14)$\\
F.$x \in \{3\} \cup [10,14)$\\
G.$x \in \{-3\} \cup (10,14)$\\
H.$x \in \{3\} \cup [10,14]$
\testStop
\kluczStart
A
\kluczStop



\zadStart{Zadanie z Wikieł Z 1.62 c) moja wersja nr 505}

Rozwiązać nierówności $(10-x)(x+3)^{2}(15-x)^{3}\le0$.
\zadStop
\rozwStart{Patryk Wirkus}{}
Miejsca zerowe naszego wielomianu to: $10, -3, 15$.\\
Wielomian jest stopnia parzystego, ponadto znak współczynnika przy\linebreak najwyższej potędze x jest ujemny.\\ W związku z tym wykres wielomianu zaczyna się od lewej strony powyżej osi OX.\\
Ponadto w punkcie $-3$ wykres odbija się od osi poziomej.\\
A więc $$x \in \{-3\} \cup [10,15].$$
\rozwStop
\odpStart
$x \in \{-3\} \cup [10,15]$
\odpStop
\testStart
A.$x \in \{-3\} \cup [10,15]$\\
B.$x \in \{3\} \cup (10,15)$\\
C.$x \in \{-3\} \cup (10,15]$\\
D.$x \in \{3\} \cup (10,15]$\\
E.$x \in \{-3\} \cup [10,15)$\\
F.$x \in \{3\} \cup [10,15)$\\
G.$x \in \{-3\} \cup (10,15)$\\
H.$x \in \{3\} \cup [10,15]$
\testStop
\kluczStart
A
\kluczStop



\zadStart{Zadanie z Wikieł Z 1.62 c) moja wersja nr 506}

Rozwiązać nierówności $(10-x)(x+3)^{2}(16-x)^{3}\le0$.
\zadStop
\rozwStart{Patryk Wirkus}{}
Miejsca zerowe naszego wielomianu to: $10, -3, 16$.\\
Wielomian jest stopnia parzystego, ponadto znak współczynnika przy\linebreak najwyższej potędze x jest ujemny.\\ W związku z tym wykres wielomianu zaczyna się od lewej strony powyżej osi OX.\\
Ponadto w punkcie $-3$ wykres odbija się od osi poziomej.\\
A więc $$x \in \{-3\} \cup [10,16].$$
\rozwStop
\odpStart
$x \in \{-3\} \cup [10,16]$
\odpStop
\testStart
A.$x \in \{-3\} \cup [10,16]$\\
B.$x \in \{3\} \cup (10,16)$\\
C.$x \in \{-3\} \cup (10,16]$\\
D.$x \in \{3\} \cup (10,16]$\\
E.$x \in \{-3\} \cup [10,16)$\\
F.$x \in \{3\} \cup [10,16)$\\
G.$x \in \{-3\} \cup (10,16)$\\
H.$x \in \{3\} \cup [10,16]$
\testStop
\kluczStart
A
\kluczStop



\zadStart{Zadanie z Wikieł Z 1.62 c) moja wersja nr 507}

Rozwiązać nierówności $(10-x)(x+3)^{2}(17-x)^{3}\le0$.
\zadStop
\rozwStart{Patryk Wirkus}{}
Miejsca zerowe naszego wielomianu to: $10, -3, 17$.\\
Wielomian jest stopnia parzystego, ponadto znak współczynnika przy\linebreak najwyższej potędze x jest ujemny.\\ W związku z tym wykres wielomianu zaczyna się od lewej strony powyżej osi OX.\\
Ponadto w punkcie $-3$ wykres odbija się od osi poziomej.\\
A więc $$x \in \{-3\} \cup [10,17].$$
\rozwStop
\odpStart
$x \in \{-3\} \cup [10,17]$
\odpStop
\testStart
A.$x \in \{-3\} \cup [10,17]$\\
B.$x \in \{3\} \cup (10,17)$\\
C.$x \in \{-3\} \cup (10,17]$\\
D.$x \in \{3\} \cup (10,17]$\\
E.$x \in \{-3\} \cup [10,17)$\\
F.$x \in \{3\} \cup [10,17)$\\
G.$x \in \{-3\} \cup (10,17)$\\
H.$x \in \{3\} \cup [10,17]$
\testStop
\kluczStart
A
\kluczStop



\zadStart{Zadanie z Wikieł Z 1.62 c) moja wersja nr 508}

Rozwiązać nierówności $(10-x)(x+3)^{2}(18-x)^{3}\le0$.
\zadStop
\rozwStart{Patryk Wirkus}{}
Miejsca zerowe naszego wielomianu to: $10, -3, 18$.\\
Wielomian jest stopnia parzystego, ponadto znak współczynnika przy\linebreak najwyższej potędze x jest ujemny.\\ W związku z tym wykres wielomianu zaczyna się od lewej strony powyżej osi OX.\\
Ponadto w punkcie $-3$ wykres odbija się od osi poziomej.\\
A więc $$x \in \{-3\} \cup [10,18].$$
\rozwStop
\odpStart
$x \in \{-3\} \cup [10,18]$
\odpStop
\testStart
A.$x \in \{-3\} \cup [10,18]$\\
B.$x \in \{3\} \cup (10,18)$\\
C.$x \in \{-3\} \cup (10,18]$\\
D.$x \in \{3\} \cup (10,18]$\\
E.$x \in \{-3\} \cup [10,18)$\\
F.$x \in \{3\} \cup [10,18)$\\
G.$x \in \{-3\} \cup (10,18)$\\
H.$x \in \{3\} \cup [10,18]$
\testStop
\kluczStart
A
\kluczStop



\zadStart{Zadanie z Wikieł Z 1.62 c) moja wersja nr 509}

Rozwiązać nierówności $(10-x)(x+3)^{2}(19-x)^{3}\le0$.
\zadStop
\rozwStart{Patryk Wirkus}{}
Miejsca zerowe naszego wielomianu to: $10, -3, 19$.\\
Wielomian jest stopnia parzystego, ponadto znak współczynnika przy\linebreak najwyższej potędze x jest ujemny.\\ W związku z tym wykres wielomianu zaczyna się od lewej strony powyżej osi OX.\\
Ponadto w punkcie $-3$ wykres odbija się od osi poziomej.\\
A więc $$x \in \{-3\} \cup [10,19].$$
\rozwStop
\odpStart
$x \in \{-3\} \cup [10,19]$
\odpStop
\testStart
A.$x \in \{-3\} \cup [10,19]$\\
B.$x \in \{3\} \cup (10,19)$\\
C.$x \in \{-3\} \cup (10,19]$\\
D.$x \in \{3\} \cup (10,19]$\\
E.$x \in \{-3\} \cup [10,19)$\\
F.$x \in \{3\} \cup [10,19)$\\
G.$x \in \{-3\} \cup (10,19)$\\
H.$x \in \{3\} \cup [10,19]$
\testStop
\kluczStart
A
\kluczStop



\zadStart{Zadanie z Wikieł Z 1.62 c) moja wersja nr 510}

Rozwiązać nierówności $(10-x)(x+3)^{2}(20-x)^{3}\le0$.
\zadStop
\rozwStart{Patryk Wirkus}{}
Miejsca zerowe naszego wielomianu to: $10, -3, 20$.\\
Wielomian jest stopnia parzystego, ponadto znak współczynnika przy\linebreak najwyższej potędze x jest ujemny.\\ W związku z tym wykres wielomianu zaczyna się od lewej strony powyżej osi OX.\\
Ponadto w punkcie $-3$ wykres odbija się od osi poziomej.\\
A więc $$x \in \{-3\} \cup [10,20].$$
\rozwStop
\odpStart
$x \in \{-3\} \cup [10,20]$
\odpStop
\testStart
A.$x \in \{-3\} \cup [10,20]$\\
B.$x \in \{3\} \cup (10,20)$\\
C.$x \in \{-3\} \cup (10,20]$\\
D.$x \in \{3\} \cup (10,20]$\\
E.$x \in \{-3\} \cup [10,20)$\\
F.$x \in \{3\} \cup [10,20)$\\
G.$x \in \{-3\} \cup (10,20)$\\
H.$x \in \{3\} \cup [10,20]$
\testStop
\kluczStart
A
\kluczStop



\zadStart{Zadanie z Wikieł Z 1.62 c) moja wersja nr 511}

Rozwiązać nierówności $(10-x)(x+4)^{2}(11-x)^{3}\le0$.
\zadStop
\rozwStart{Patryk Wirkus}{}
Miejsca zerowe naszego wielomianu to: $10, -4, 11$.\\
Wielomian jest stopnia parzystego, ponadto znak współczynnika przy\linebreak najwyższej potędze x jest ujemny.\\ W związku z tym wykres wielomianu zaczyna się od lewej strony powyżej osi OX.\\
Ponadto w punkcie $-4$ wykres odbija się od osi poziomej.\\
A więc $$x \in \{-4\} \cup [10,11].$$
\rozwStop
\odpStart
$x \in \{-4\} \cup [10,11]$
\odpStop
\testStart
A.$x \in \{-4\} \cup [10,11]$\\
B.$x \in \{4\} \cup (10,11)$\\
C.$x \in \{-4\} \cup (10,11]$\\
D.$x \in \{4\} \cup (10,11]$\\
E.$x \in \{-4\} \cup [10,11)$\\
F.$x \in \{4\} \cup [10,11)$\\
G.$x \in \{-4\} \cup (10,11)$\\
H.$x \in \{4\} \cup [10,11]$
\testStop
\kluczStart
A
\kluczStop



\zadStart{Zadanie z Wikieł Z 1.62 c) moja wersja nr 512}

Rozwiązać nierówności $(10-x)(x+4)^{2}(12-x)^{3}\le0$.
\zadStop
\rozwStart{Patryk Wirkus}{}
Miejsca zerowe naszego wielomianu to: $10, -4, 12$.\\
Wielomian jest stopnia parzystego, ponadto znak współczynnika przy\linebreak najwyższej potędze x jest ujemny.\\ W związku z tym wykres wielomianu zaczyna się od lewej strony powyżej osi OX.\\
Ponadto w punkcie $-4$ wykres odbija się od osi poziomej.\\
A więc $$x \in \{-4\} \cup [10,12].$$
\rozwStop
\odpStart
$x \in \{-4\} \cup [10,12]$
\odpStop
\testStart
A.$x \in \{-4\} \cup [10,12]$\\
B.$x \in \{4\} \cup (10,12)$\\
C.$x \in \{-4\} \cup (10,12]$\\
D.$x \in \{4\} \cup (10,12]$\\
E.$x \in \{-4\} \cup [10,12)$\\
F.$x \in \{4\} \cup [10,12)$\\
G.$x \in \{-4\} \cup (10,12)$\\
H.$x \in \{4\} \cup [10,12]$
\testStop
\kluczStart
A
\kluczStop



\zadStart{Zadanie z Wikieł Z 1.62 c) moja wersja nr 513}

Rozwiązać nierówności $(10-x)(x+4)^{2}(13-x)^{3}\le0$.
\zadStop
\rozwStart{Patryk Wirkus}{}
Miejsca zerowe naszego wielomianu to: $10, -4, 13$.\\
Wielomian jest stopnia parzystego, ponadto znak współczynnika przy\linebreak najwyższej potędze x jest ujemny.\\ W związku z tym wykres wielomianu zaczyna się od lewej strony powyżej osi OX.\\
Ponadto w punkcie $-4$ wykres odbija się od osi poziomej.\\
A więc $$x \in \{-4\} \cup [10,13].$$
\rozwStop
\odpStart
$x \in \{-4\} \cup [10,13]$
\odpStop
\testStart
A.$x \in \{-4\} \cup [10,13]$\\
B.$x \in \{4\} \cup (10,13)$\\
C.$x \in \{-4\} \cup (10,13]$\\
D.$x \in \{4\} \cup (10,13]$\\
E.$x \in \{-4\} \cup [10,13)$\\
F.$x \in \{4\} \cup [10,13)$\\
G.$x \in \{-4\} \cup (10,13)$\\
H.$x \in \{4\} \cup [10,13]$
\testStop
\kluczStart
A
\kluczStop



\zadStart{Zadanie z Wikieł Z 1.62 c) moja wersja nr 514}

Rozwiązać nierówności $(10-x)(x+4)^{2}(14-x)^{3}\le0$.
\zadStop
\rozwStart{Patryk Wirkus}{}
Miejsca zerowe naszego wielomianu to: $10, -4, 14$.\\
Wielomian jest stopnia parzystego, ponadto znak współczynnika przy\linebreak najwyższej potędze x jest ujemny.\\ W związku z tym wykres wielomianu zaczyna się od lewej strony powyżej osi OX.\\
Ponadto w punkcie $-4$ wykres odbija się od osi poziomej.\\
A więc $$x \in \{-4\} \cup [10,14].$$
\rozwStop
\odpStart
$x \in \{-4\} \cup [10,14]$
\odpStop
\testStart
A.$x \in \{-4\} \cup [10,14]$\\
B.$x \in \{4\} \cup (10,14)$\\
C.$x \in \{-4\} \cup (10,14]$\\
D.$x \in \{4\} \cup (10,14]$\\
E.$x \in \{-4\} \cup [10,14)$\\
F.$x \in \{4\} \cup [10,14)$\\
G.$x \in \{-4\} \cup (10,14)$\\
H.$x \in \{4\} \cup [10,14]$
\testStop
\kluczStart
A
\kluczStop



\zadStart{Zadanie z Wikieł Z 1.62 c) moja wersja nr 515}

Rozwiązać nierówności $(10-x)(x+4)^{2}(15-x)^{3}\le0$.
\zadStop
\rozwStart{Patryk Wirkus}{}
Miejsca zerowe naszego wielomianu to: $10, -4, 15$.\\
Wielomian jest stopnia parzystego, ponadto znak współczynnika przy\linebreak najwyższej potędze x jest ujemny.\\ W związku z tym wykres wielomianu zaczyna się od lewej strony powyżej osi OX.\\
Ponadto w punkcie $-4$ wykres odbija się od osi poziomej.\\
A więc $$x \in \{-4\} \cup [10,15].$$
\rozwStop
\odpStart
$x \in \{-4\} \cup [10,15]$
\odpStop
\testStart
A.$x \in \{-4\} \cup [10,15]$\\
B.$x \in \{4\} \cup (10,15)$\\
C.$x \in \{-4\} \cup (10,15]$\\
D.$x \in \{4\} \cup (10,15]$\\
E.$x \in \{-4\} \cup [10,15)$\\
F.$x \in \{4\} \cup [10,15)$\\
G.$x \in \{-4\} \cup (10,15)$\\
H.$x \in \{4\} \cup [10,15]$
\testStop
\kluczStart
A
\kluczStop



\zadStart{Zadanie z Wikieł Z 1.62 c) moja wersja nr 516}

Rozwiązać nierówności $(10-x)(x+4)^{2}(16-x)^{3}\le0$.
\zadStop
\rozwStart{Patryk Wirkus}{}
Miejsca zerowe naszego wielomianu to: $10, -4, 16$.\\
Wielomian jest stopnia parzystego, ponadto znak współczynnika przy\linebreak najwyższej potędze x jest ujemny.\\ W związku z tym wykres wielomianu zaczyna się od lewej strony powyżej osi OX.\\
Ponadto w punkcie $-4$ wykres odbija się od osi poziomej.\\
A więc $$x \in \{-4\} \cup [10,16].$$
\rozwStop
\odpStart
$x \in \{-4\} \cup [10,16]$
\odpStop
\testStart
A.$x \in \{-4\} \cup [10,16]$\\
B.$x \in \{4\} \cup (10,16)$\\
C.$x \in \{-4\} \cup (10,16]$\\
D.$x \in \{4\} \cup (10,16]$\\
E.$x \in \{-4\} \cup [10,16)$\\
F.$x \in \{4\} \cup [10,16)$\\
G.$x \in \{-4\} \cup (10,16)$\\
H.$x \in \{4\} \cup [10,16]$
\testStop
\kluczStart
A
\kluczStop



\zadStart{Zadanie z Wikieł Z 1.62 c) moja wersja nr 517}

Rozwiązać nierówności $(10-x)(x+4)^{2}(17-x)^{3}\le0$.
\zadStop
\rozwStart{Patryk Wirkus}{}
Miejsca zerowe naszego wielomianu to: $10, -4, 17$.\\
Wielomian jest stopnia parzystego, ponadto znak współczynnika przy\linebreak najwyższej potędze x jest ujemny.\\ W związku z tym wykres wielomianu zaczyna się od lewej strony powyżej osi OX.\\
Ponadto w punkcie $-4$ wykres odbija się od osi poziomej.\\
A więc $$x \in \{-4\} \cup [10,17].$$
\rozwStop
\odpStart
$x \in \{-4\} \cup [10,17]$
\odpStop
\testStart
A.$x \in \{-4\} \cup [10,17]$\\
B.$x \in \{4\} \cup (10,17)$\\
C.$x \in \{-4\} \cup (10,17]$\\
D.$x \in \{4\} \cup (10,17]$\\
E.$x \in \{-4\} \cup [10,17)$\\
F.$x \in \{4\} \cup [10,17)$\\
G.$x \in \{-4\} \cup (10,17)$\\
H.$x \in \{4\} \cup [10,17]$
\testStop
\kluczStart
A
\kluczStop



\zadStart{Zadanie z Wikieł Z 1.62 c) moja wersja nr 518}

Rozwiązać nierówności $(10-x)(x+4)^{2}(18-x)^{3}\le0$.
\zadStop
\rozwStart{Patryk Wirkus}{}
Miejsca zerowe naszego wielomianu to: $10, -4, 18$.\\
Wielomian jest stopnia parzystego, ponadto znak współczynnika przy\linebreak najwyższej potędze x jest ujemny.\\ W związku z tym wykres wielomianu zaczyna się od lewej strony powyżej osi OX.\\
Ponadto w punkcie $-4$ wykres odbija się od osi poziomej.\\
A więc $$x \in \{-4\} \cup [10,18].$$
\rozwStop
\odpStart
$x \in \{-4\} \cup [10,18]$
\odpStop
\testStart
A.$x \in \{-4\} \cup [10,18]$\\
B.$x \in \{4\} \cup (10,18)$\\
C.$x \in \{-4\} \cup (10,18]$\\
D.$x \in \{4\} \cup (10,18]$\\
E.$x \in \{-4\} \cup [10,18)$\\
F.$x \in \{4\} \cup [10,18)$\\
G.$x \in \{-4\} \cup (10,18)$\\
H.$x \in \{4\} \cup [10,18]$
\testStop
\kluczStart
A
\kluczStop



\zadStart{Zadanie z Wikieł Z 1.62 c) moja wersja nr 519}

Rozwiązać nierówności $(10-x)(x+4)^{2}(19-x)^{3}\le0$.
\zadStop
\rozwStart{Patryk Wirkus}{}
Miejsca zerowe naszego wielomianu to: $10, -4, 19$.\\
Wielomian jest stopnia parzystego, ponadto znak współczynnika przy\linebreak najwyższej potędze x jest ujemny.\\ W związku z tym wykres wielomianu zaczyna się od lewej strony powyżej osi OX.\\
Ponadto w punkcie $-4$ wykres odbija się od osi poziomej.\\
A więc $$x \in \{-4\} \cup [10,19].$$
\rozwStop
\odpStart
$x \in \{-4\} \cup [10,19]$
\odpStop
\testStart
A.$x \in \{-4\} \cup [10,19]$\\
B.$x \in \{4\} \cup (10,19)$\\
C.$x \in \{-4\} \cup (10,19]$\\
D.$x \in \{4\} \cup (10,19]$\\
E.$x \in \{-4\} \cup [10,19)$\\
F.$x \in \{4\} \cup [10,19)$\\
G.$x \in \{-4\} \cup (10,19)$\\
H.$x \in \{4\} \cup [10,19]$
\testStop
\kluczStart
A
\kluczStop



\zadStart{Zadanie z Wikieł Z 1.62 c) moja wersja nr 520}

Rozwiązać nierówności $(10-x)(x+4)^{2}(20-x)^{3}\le0$.
\zadStop
\rozwStart{Patryk Wirkus}{}
Miejsca zerowe naszego wielomianu to: $10, -4, 20$.\\
Wielomian jest stopnia parzystego, ponadto znak współczynnika przy\linebreak najwyższej potędze x jest ujemny.\\ W związku z tym wykres wielomianu zaczyna się od lewej strony powyżej osi OX.\\
Ponadto w punkcie $-4$ wykres odbija się od osi poziomej.\\
A więc $$x \in \{-4\} \cup [10,20].$$
\rozwStop
\odpStart
$x \in \{-4\} \cup [10,20]$
\odpStop
\testStart
A.$x \in \{-4\} \cup [10,20]$\\
B.$x \in \{4\} \cup (10,20)$\\
C.$x \in \{-4\} \cup (10,20]$\\
D.$x \in \{4\} \cup (10,20]$\\
E.$x \in \{-4\} \cup [10,20)$\\
F.$x \in \{4\} \cup [10,20)$\\
G.$x \in \{-4\} \cup (10,20)$\\
H.$x \in \{4\} \cup [10,20]$
\testStop
\kluczStart
A
\kluczStop



\zadStart{Zadanie z Wikieł Z 1.62 c) moja wersja nr 521}

Rozwiązać nierówności $(10-x)(x+5)^{2}(11-x)^{3}\le0$.
\zadStop
\rozwStart{Patryk Wirkus}{}
Miejsca zerowe naszego wielomianu to: $10, -5, 11$.\\
Wielomian jest stopnia parzystego, ponadto znak współczynnika przy\linebreak najwyższej potędze x jest ujemny.\\ W związku z tym wykres wielomianu zaczyna się od lewej strony powyżej osi OX.\\
Ponadto w punkcie $-5$ wykres odbija się od osi poziomej.\\
A więc $$x \in \{-5\} \cup [10,11].$$
\rozwStop
\odpStart
$x \in \{-5\} \cup [10,11]$
\odpStop
\testStart
A.$x \in \{-5\} \cup [10,11]$\\
B.$x \in \{5\} \cup (10,11)$\\
C.$x \in \{-5\} \cup (10,11]$\\
D.$x \in \{5\} \cup (10,11]$\\
E.$x \in \{-5\} \cup [10,11)$\\
F.$x \in \{5\} \cup [10,11)$\\
G.$x \in \{-5\} \cup (10,11)$\\
H.$x \in \{5\} \cup [10,11]$
\testStop
\kluczStart
A
\kluczStop



\zadStart{Zadanie z Wikieł Z 1.62 c) moja wersja nr 522}

Rozwiązać nierówności $(10-x)(x+5)^{2}(12-x)^{3}\le0$.
\zadStop
\rozwStart{Patryk Wirkus}{}
Miejsca zerowe naszego wielomianu to: $10, -5, 12$.\\
Wielomian jest stopnia parzystego, ponadto znak współczynnika przy\linebreak najwyższej potędze x jest ujemny.\\ W związku z tym wykres wielomianu zaczyna się od lewej strony powyżej osi OX.\\
Ponadto w punkcie $-5$ wykres odbija się od osi poziomej.\\
A więc $$x \in \{-5\} \cup [10,12].$$
\rozwStop
\odpStart
$x \in \{-5\} \cup [10,12]$
\odpStop
\testStart
A.$x \in \{-5\} \cup [10,12]$\\
B.$x \in \{5\} \cup (10,12)$\\
C.$x \in \{-5\} \cup (10,12]$\\
D.$x \in \{5\} \cup (10,12]$\\
E.$x \in \{-5\} \cup [10,12)$\\
F.$x \in \{5\} \cup [10,12)$\\
G.$x \in \{-5\} \cup (10,12)$\\
H.$x \in \{5\} \cup [10,12]$
\testStop
\kluczStart
A
\kluczStop



\zadStart{Zadanie z Wikieł Z 1.62 c) moja wersja nr 523}

Rozwiązać nierówności $(10-x)(x+5)^{2}(13-x)^{3}\le0$.
\zadStop
\rozwStart{Patryk Wirkus}{}
Miejsca zerowe naszego wielomianu to: $10, -5, 13$.\\
Wielomian jest stopnia parzystego, ponadto znak współczynnika przy\linebreak najwyższej potędze x jest ujemny.\\ W związku z tym wykres wielomianu zaczyna się od lewej strony powyżej osi OX.\\
Ponadto w punkcie $-5$ wykres odbija się od osi poziomej.\\
A więc $$x \in \{-5\} \cup [10,13].$$
\rozwStop
\odpStart
$x \in \{-5\} \cup [10,13]$
\odpStop
\testStart
A.$x \in \{-5\} \cup [10,13]$\\
B.$x \in \{5\} \cup (10,13)$\\
C.$x \in \{-5\} \cup (10,13]$\\
D.$x \in \{5\} \cup (10,13]$\\
E.$x \in \{-5\} \cup [10,13)$\\
F.$x \in \{5\} \cup [10,13)$\\
G.$x \in \{-5\} \cup (10,13)$\\
H.$x \in \{5\} \cup [10,13]$
\testStop
\kluczStart
A
\kluczStop



\zadStart{Zadanie z Wikieł Z 1.62 c) moja wersja nr 524}

Rozwiązać nierówności $(10-x)(x+5)^{2}(14-x)^{3}\le0$.
\zadStop
\rozwStart{Patryk Wirkus}{}
Miejsca zerowe naszego wielomianu to: $10, -5, 14$.\\
Wielomian jest stopnia parzystego, ponadto znak współczynnika przy\linebreak najwyższej potędze x jest ujemny.\\ W związku z tym wykres wielomianu zaczyna się od lewej strony powyżej osi OX.\\
Ponadto w punkcie $-5$ wykres odbija się od osi poziomej.\\
A więc $$x \in \{-5\} \cup [10,14].$$
\rozwStop
\odpStart
$x \in \{-5\} \cup [10,14]$
\odpStop
\testStart
A.$x \in \{-5\} \cup [10,14]$\\
B.$x \in \{5\} \cup (10,14)$\\
C.$x \in \{-5\} \cup (10,14]$\\
D.$x \in \{5\} \cup (10,14]$\\
E.$x \in \{-5\} \cup [10,14)$\\
F.$x \in \{5\} \cup [10,14)$\\
G.$x \in \{-5\} \cup (10,14)$\\
H.$x \in \{5\} \cup [10,14]$
\testStop
\kluczStart
A
\kluczStop



\zadStart{Zadanie z Wikieł Z 1.62 c) moja wersja nr 525}

Rozwiązać nierówności $(10-x)(x+5)^{2}(15-x)^{3}\le0$.
\zadStop
\rozwStart{Patryk Wirkus}{}
Miejsca zerowe naszego wielomianu to: $10, -5, 15$.\\
Wielomian jest stopnia parzystego, ponadto znak współczynnika przy\linebreak najwyższej potędze x jest ujemny.\\ W związku z tym wykres wielomianu zaczyna się od lewej strony powyżej osi OX.\\
Ponadto w punkcie $-5$ wykres odbija się od osi poziomej.\\
A więc $$x \in \{-5\} \cup [10,15].$$
\rozwStop
\odpStart
$x \in \{-5\} \cup [10,15]$
\odpStop
\testStart
A.$x \in \{-5\} \cup [10,15]$\\
B.$x \in \{5\} \cup (10,15)$\\
C.$x \in \{-5\} \cup (10,15]$\\
D.$x \in \{5\} \cup (10,15]$\\
E.$x \in \{-5\} \cup [10,15)$\\
F.$x \in \{5\} \cup [10,15)$\\
G.$x \in \{-5\} \cup (10,15)$\\
H.$x \in \{5\} \cup [10,15]$
\testStop
\kluczStart
A
\kluczStop



\zadStart{Zadanie z Wikieł Z 1.62 c) moja wersja nr 526}

Rozwiązać nierówności $(10-x)(x+5)^{2}(16-x)^{3}\le0$.
\zadStop
\rozwStart{Patryk Wirkus}{}
Miejsca zerowe naszego wielomianu to: $10, -5, 16$.\\
Wielomian jest stopnia parzystego, ponadto znak współczynnika przy\linebreak najwyższej potędze x jest ujemny.\\ W związku z tym wykres wielomianu zaczyna się od lewej strony powyżej osi OX.\\
Ponadto w punkcie $-5$ wykres odbija się od osi poziomej.\\
A więc $$x \in \{-5\} \cup [10,16].$$
\rozwStop
\odpStart
$x \in \{-5\} \cup [10,16]$
\odpStop
\testStart
A.$x \in \{-5\} \cup [10,16]$\\
B.$x \in \{5\} \cup (10,16)$\\
C.$x \in \{-5\} \cup (10,16]$\\
D.$x \in \{5\} \cup (10,16]$\\
E.$x \in \{-5\} \cup [10,16)$\\
F.$x \in \{5\} \cup [10,16)$\\
G.$x \in \{-5\} \cup (10,16)$\\
H.$x \in \{5\} \cup [10,16]$
\testStop
\kluczStart
A
\kluczStop



\zadStart{Zadanie z Wikieł Z 1.62 c) moja wersja nr 527}

Rozwiązać nierówności $(10-x)(x+5)^{2}(17-x)^{3}\le0$.
\zadStop
\rozwStart{Patryk Wirkus}{}
Miejsca zerowe naszego wielomianu to: $10, -5, 17$.\\
Wielomian jest stopnia parzystego, ponadto znak współczynnika przy\linebreak najwyższej potędze x jest ujemny.\\ W związku z tym wykres wielomianu zaczyna się od lewej strony powyżej osi OX.\\
Ponadto w punkcie $-5$ wykres odbija się od osi poziomej.\\
A więc $$x \in \{-5\} \cup [10,17].$$
\rozwStop
\odpStart
$x \in \{-5\} \cup [10,17]$
\odpStop
\testStart
A.$x \in \{-5\} \cup [10,17]$\\
B.$x \in \{5\} \cup (10,17)$\\
C.$x \in \{-5\} \cup (10,17]$\\
D.$x \in \{5\} \cup (10,17]$\\
E.$x \in \{-5\} \cup [10,17)$\\
F.$x \in \{5\} \cup [10,17)$\\
G.$x \in \{-5\} \cup (10,17)$\\
H.$x \in \{5\} \cup [10,17]$
\testStop
\kluczStart
A
\kluczStop



\zadStart{Zadanie z Wikieł Z 1.62 c) moja wersja nr 528}

Rozwiązać nierówności $(10-x)(x+5)^{2}(18-x)^{3}\le0$.
\zadStop
\rozwStart{Patryk Wirkus}{}
Miejsca zerowe naszego wielomianu to: $10, -5, 18$.\\
Wielomian jest stopnia parzystego, ponadto znak współczynnika przy\linebreak najwyższej potędze x jest ujemny.\\ W związku z tym wykres wielomianu zaczyna się od lewej strony powyżej osi OX.\\
Ponadto w punkcie $-5$ wykres odbija się od osi poziomej.\\
A więc $$x \in \{-5\} \cup [10,18].$$
\rozwStop
\odpStart
$x \in \{-5\} \cup [10,18]$
\odpStop
\testStart
A.$x \in \{-5\} \cup [10,18]$\\
B.$x \in \{5\} \cup (10,18)$\\
C.$x \in \{-5\} \cup (10,18]$\\
D.$x \in \{5\} \cup (10,18]$\\
E.$x \in \{-5\} \cup [10,18)$\\
F.$x \in \{5\} \cup [10,18)$\\
G.$x \in \{-5\} \cup (10,18)$\\
H.$x \in \{5\} \cup [10,18]$
\testStop
\kluczStart
A
\kluczStop



\zadStart{Zadanie z Wikieł Z 1.62 c) moja wersja nr 529}

Rozwiązać nierówności $(10-x)(x+5)^{2}(19-x)^{3}\le0$.
\zadStop
\rozwStart{Patryk Wirkus}{}
Miejsca zerowe naszego wielomianu to: $10, -5, 19$.\\
Wielomian jest stopnia parzystego, ponadto znak współczynnika przy\linebreak najwyższej potędze x jest ujemny.\\ W związku z tym wykres wielomianu zaczyna się od lewej strony powyżej osi OX.\\
Ponadto w punkcie $-5$ wykres odbija się od osi poziomej.\\
A więc $$x \in \{-5\} \cup [10,19].$$
\rozwStop
\odpStart
$x \in \{-5\} \cup [10,19]$
\odpStop
\testStart
A.$x \in \{-5\} \cup [10,19]$\\
B.$x \in \{5\} \cup (10,19)$\\
C.$x \in \{-5\} \cup (10,19]$\\
D.$x \in \{5\} \cup (10,19]$\\
E.$x \in \{-5\} \cup [10,19)$\\
F.$x \in \{5\} \cup [10,19)$\\
G.$x \in \{-5\} \cup (10,19)$\\
H.$x \in \{5\} \cup [10,19]$
\testStop
\kluczStart
A
\kluczStop



\zadStart{Zadanie z Wikieł Z 1.62 c) moja wersja nr 530}

Rozwiązać nierówności $(10-x)(x+5)^{2}(20-x)^{3}\le0$.
\zadStop
\rozwStart{Patryk Wirkus}{}
Miejsca zerowe naszego wielomianu to: $10, -5, 20$.\\
Wielomian jest stopnia parzystego, ponadto znak współczynnika przy\linebreak najwyższej potędze x jest ujemny.\\ W związku z tym wykres wielomianu zaczyna się od lewej strony powyżej osi OX.\\
Ponadto w punkcie $-5$ wykres odbija się od osi poziomej.\\
A więc $$x \in \{-5\} \cup [10,20].$$
\rozwStop
\odpStart
$x \in \{-5\} \cup [10,20]$
\odpStop
\testStart
A.$x \in \{-5\} \cup [10,20]$\\
B.$x \in \{5\} \cup (10,20)$\\
C.$x \in \{-5\} \cup (10,20]$\\
D.$x \in \{5\} \cup (10,20]$\\
E.$x \in \{-5\} \cup [10,20)$\\
F.$x \in \{5\} \cup [10,20)$\\
G.$x \in \{-5\} \cup (10,20)$\\
H.$x \in \{5\} \cup [10,20]$
\testStop
\kluczStart
A
\kluczStop



\zadStart{Zadanie z Wikieł Z 1.62 c) moja wersja nr 531}

Rozwiązać nierówności $(10-x)(x+6)^{2}(11-x)^{3}\le0$.
\zadStop
\rozwStart{Patryk Wirkus}{}
Miejsca zerowe naszego wielomianu to: $10, -6, 11$.\\
Wielomian jest stopnia parzystego, ponadto znak współczynnika przy\linebreak najwyższej potędze x jest ujemny.\\ W związku z tym wykres wielomianu zaczyna się od lewej strony powyżej osi OX.\\
Ponadto w punkcie $-6$ wykres odbija się od osi poziomej.\\
A więc $$x \in \{-6\} \cup [10,11].$$
\rozwStop
\odpStart
$x \in \{-6\} \cup [10,11]$
\odpStop
\testStart
A.$x \in \{-6\} \cup [10,11]$\\
B.$x \in \{6\} \cup (10,11)$\\
C.$x \in \{-6\} \cup (10,11]$\\
D.$x \in \{6\} \cup (10,11]$\\
E.$x \in \{-6\} \cup [10,11)$\\
F.$x \in \{6\} \cup [10,11)$\\
G.$x \in \{-6\} \cup (10,11)$\\
H.$x \in \{6\} \cup [10,11]$
\testStop
\kluczStart
A
\kluczStop



\zadStart{Zadanie z Wikieł Z 1.62 c) moja wersja nr 532}

Rozwiązać nierówności $(10-x)(x+6)^{2}(12-x)^{3}\le0$.
\zadStop
\rozwStart{Patryk Wirkus}{}
Miejsca zerowe naszego wielomianu to: $10, -6, 12$.\\
Wielomian jest stopnia parzystego, ponadto znak współczynnika przy\linebreak najwyższej potędze x jest ujemny.\\ W związku z tym wykres wielomianu zaczyna się od lewej strony powyżej osi OX.\\
Ponadto w punkcie $-6$ wykres odbija się od osi poziomej.\\
A więc $$x \in \{-6\} \cup [10,12].$$
\rozwStop
\odpStart
$x \in \{-6\} \cup [10,12]$
\odpStop
\testStart
A.$x \in \{-6\} \cup [10,12]$\\
B.$x \in \{6\} \cup (10,12)$\\
C.$x \in \{-6\} \cup (10,12]$\\
D.$x \in \{6\} \cup (10,12]$\\
E.$x \in \{-6\} \cup [10,12)$\\
F.$x \in \{6\} \cup [10,12)$\\
G.$x \in \{-6\} \cup (10,12)$\\
H.$x \in \{6\} \cup [10,12]$
\testStop
\kluczStart
A
\kluczStop



\zadStart{Zadanie z Wikieł Z 1.62 c) moja wersja nr 533}

Rozwiązać nierówności $(10-x)(x+6)^{2}(13-x)^{3}\le0$.
\zadStop
\rozwStart{Patryk Wirkus}{}
Miejsca zerowe naszego wielomianu to: $10, -6, 13$.\\
Wielomian jest stopnia parzystego, ponadto znak współczynnika przy\linebreak najwyższej potędze x jest ujemny.\\ W związku z tym wykres wielomianu zaczyna się od lewej strony powyżej osi OX.\\
Ponadto w punkcie $-6$ wykres odbija się od osi poziomej.\\
A więc $$x \in \{-6\} \cup [10,13].$$
\rozwStop
\odpStart
$x \in \{-6\} \cup [10,13]$
\odpStop
\testStart
A.$x \in \{-6\} \cup [10,13]$\\
B.$x \in \{6\} \cup (10,13)$\\
C.$x \in \{-6\} \cup (10,13]$\\
D.$x \in \{6\} \cup (10,13]$\\
E.$x \in \{-6\} \cup [10,13)$\\
F.$x \in \{6\} \cup [10,13)$\\
G.$x \in \{-6\} \cup (10,13)$\\
H.$x \in \{6\} \cup [10,13]$
\testStop
\kluczStart
A
\kluczStop



\zadStart{Zadanie z Wikieł Z 1.62 c) moja wersja nr 534}

Rozwiązać nierówności $(10-x)(x+6)^{2}(14-x)^{3}\le0$.
\zadStop
\rozwStart{Patryk Wirkus}{}
Miejsca zerowe naszego wielomianu to: $10, -6, 14$.\\
Wielomian jest stopnia parzystego, ponadto znak współczynnika przy\linebreak najwyższej potędze x jest ujemny.\\ W związku z tym wykres wielomianu zaczyna się od lewej strony powyżej osi OX.\\
Ponadto w punkcie $-6$ wykres odbija się od osi poziomej.\\
A więc $$x \in \{-6\} \cup [10,14].$$
\rozwStop
\odpStart
$x \in \{-6\} \cup [10,14]$
\odpStop
\testStart
A.$x \in \{-6\} \cup [10,14]$\\
B.$x \in \{6\} \cup (10,14)$\\
C.$x \in \{-6\} \cup (10,14]$\\
D.$x \in \{6\} \cup (10,14]$\\
E.$x \in \{-6\} \cup [10,14)$\\
F.$x \in \{6\} \cup [10,14)$\\
G.$x \in \{-6\} \cup (10,14)$\\
H.$x \in \{6\} \cup [10,14]$
\testStop
\kluczStart
A
\kluczStop



\zadStart{Zadanie z Wikieł Z 1.62 c) moja wersja nr 535}

Rozwiązać nierówności $(10-x)(x+6)^{2}(15-x)^{3}\le0$.
\zadStop
\rozwStart{Patryk Wirkus}{}
Miejsca zerowe naszego wielomianu to: $10, -6, 15$.\\
Wielomian jest stopnia parzystego, ponadto znak współczynnika przy\linebreak najwyższej potędze x jest ujemny.\\ W związku z tym wykres wielomianu zaczyna się od lewej strony powyżej osi OX.\\
Ponadto w punkcie $-6$ wykres odbija się od osi poziomej.\\
A więc $$x \in \{-6\} \cup [10,15].$$
\rozwStop
\odpStart
$x \in \{-6\} \cup [10,15]$
\odpStop
\testStart
A.$x \in \{-6\} \cup [10,15]$\\
B.$x \in \{6\} \cup (10,15)$\\
C.$x \in \{-6\} \cup (10,15]$\\
D.$x \in \{6\} \cup (10,15]$\\
E.$x \in \{-6\} \cup [10,15)$\\
F.$x \in \{6\} \cup [10,15)$\\
G.$x \in \{-6\} \cup (10,15)$\\
H.$x \in \{6\} \cup [10,15]$
\testStop
\kluczStart
A
\kluczStop



\zadStart{Zadanie z Wikieł Z 1.62 c) moja wersja nr 536}

Rozwiązać nierówności $(10-x)(x+6)^{2}(16-x)^{3}\le0$.
\zadStop
\rozwStart{Patryk Wirkus}{}
Miejsca zerowe naszego wielomianu to: $10, -6, 16$.\\
Wielomian jest stopnia parzystego, ponadto znak współczynnika przy\linebreak najwyższej potędze x jest ujemny.\\ W związku z tym wykres wielomianu zaczyna się od lewej strony powyżej osi OX.\\
Ponadto w punkcie $-6$ wykres odbija się od osi poziomej.\\
A więc $$x \in \{-6\} \cup [10,16].$$
\rozwStop
\odpStart
$x \in \{-6\} \cup [10,16]$
\odpStop
\testStart
A.$x \in \{-6\} \cup [10,16]$\\
B.$x \in \{6\} \cup (10,16)$\\
C.$x \in \{-6\} \cup (10,16]$\\
D.$x \in \{6\} \cup (10,16]$\\
E.$x \in \{-6\} \cup [10,16)$\\
F.$x \in \{6\} \cup [10,16)$\\
G.$x \in \{-6\} \cup (10,16)$\\
H.$x \in \{6\} \cup [10,16]$
\testStop
\kluczStart
A
\kluczStop



\zadStart{Zadanie z Wikieł Z 1.62 c) moja wersja nr 537}

Rozwiązać nierówności $(10-x)(x+6)^{2}(17-x)^{3}\le0$.
\zadStop
\rozwStart{Patryk Wirkus}{}
Miejsca zerowe naszego wielomianu to: $10, -6, 17$.\\
Wielomian jest stopnia parzystego, ponadto znak współczynnika przy\linebreak najwyższej potędze x jest ujemny.\\ W związku z tym wykres wielomianu zaczyna się od lewej strony powyżej osi OX.\\
Ponadto w punkcie $-6$ wykres odbija się od osi poziomej.\\
A więc $$x \in \{-6\} \cup [10,17].$$
\rozwStop
\odpStart
$x \in \{-6\} \cup [10,17]$
\odpStop
\testStart
A.$x \in \{-6\} \cup [10,17]$\\
B.$x \in \{6\} \cup (10,17)$\\
C.$x \in \{-6\} \cup (10,17]$\\
D.$x \in \{6\} \cup (10,17]$\\
E.$x \in \{-6\} \cup [10,17)$\\
F.$x \in \{6\} \cup [10,17)$\\
G.$x \in \{-6\} \cup (10,17)$\\
H.$x \in \{6\} \cup [10,17]$
\testStop
\kluczStart
A
\kluczStop



\zadStart{Zadanie z Wikieł Z 1.62 c) moja wersja nr 538}

Rozwiązać nierówności $(10-x)(x+6)^{2}(18-x)^{3}\le0$.
\zadStop
\rozwStart{Patryk Wirkus}{}
Miejsca zerowe naszego wielomianu to: $10, -6, 18$.\\
Wielomian jest stopnia parzystego, ponadto znak współczynnika przy\linebreak najwyższej potędze x jest ujemny.\\ W związku z tym wykres wielomianu zaczyna się od lewej strony powyżej osi OX.\\
Ponadto w punkcie $-6$ wykres odbija się od osi poziomej.\\
A więc $$x \in \{-6\} \cup [10,18].$$
\rozwStop
\odpStart
$x \in \{-6\} \cup [10,18]$
\odpStop
\testStart
A.$x \in \{-6\} \cup [10,18]$\\
B.$x \in \{6\} \cup (10,18)$\\
C.$x \in \{-6\} \cup (10,18]$\\
D.$x \in \{6\} \cup (10,18]$\\
E.$x \in \{-6\} \cup [10,18)$\\
F.$x \in \{6\} \cup [10,18)$\\
G.$x \in \{-6\} \cup (10,18)$\\
H.$x \in \{6\} \cup [10,18]$
\testStop
\kluczStart
A
\kluczStop



\zadStart{Zadanie z Wikieł Z 1.62 c) moja wersja nr 539}

Rozwiązać nierówności $(10-x)(x+6)^{2}(19-x)^{3}\le0$.
\zadStop
\rozwStart{Patryk Wirkus}{}
Miejsca zerowe naszego wielomianu to: $10, -6, 19$.\\
Wielomian jest stopnia parzystego, ponadto znak współczynnika przy\linebreak najwyższej potędze x jest ujemny.\\ W związku z tym wykres wielomianu zaczyna się od lewej strony powyżej osi OX.\\
Ponadto w punkcie $-6$ wykres odbija się od osi poziomej.\\
A więc $$x \in \{-6\} \cup [10,19].$$
\rozwStop
\odpStart
$x \in \{-6\} \cup [10,19]$
\odpStop
\testStart
A.$x \in \{-6\} \cup [10,19]$\\
B.$x \in \{6\} \cup (10,19)$\\
C.$x \in \{-6\} \cup (10,19]$\\
D.$x \in \{6\} \cup (10,19]$\\
E.$x \in \{-6\} \cup [10,19)$\\
F.$x \in \{6\} \cup [10,19)$\\
G.$x \in \{-6\} \cup (10,19)$\\
H.$x \in \{6\} \cup [10,19]$
\testStop
\kluczStart
A
\kluczStop



\zadStart{Zadanie z Wikieł Z 1.62 c) moja wersja nr 540}

Rozwiązać nierówności $(10-x)(x+6)^{2}(20-x)^{3}\le0$.
\zadStop
\rozwStart{Patryk Wirkus}{}
Miejsca zerowe naszego wielomianu to: $10, -6, 20$.\\
Wielomian jest stopnia parzystego, ponadto znak współczynnika przy\linebreak najwyższej potędze x jest ujemny.\\ W związku z tym wykres wielomianu zaczyna się od lewej strony powyżej osi OX.\\
Ponadto w punkcie $-6$ wykres odbija się od osi poziomej.\\
A więc $$x \in \{-6\} \cup [10,20].$$
\rozwStop
\odpStart
$x \in \{-6\} \cup [10,20]$
\odpStop
\testStart
A.$x \in \{-6\} \cup [10,20]$\\
B.$x \in \{6\} \cup (10,20)$\\
C.$x \in \{-6\} \cup (10,20]$\\
D.$x \in \{6\} \cup (10,20]$\\
E.$x \in \{-6\} \cup [10,20)$\\
F.$x \in \{6\} \cup [10,20)$\\
G.$x \in \{-6\} \cup (10,20)$\\
H.$x \in \{6\} \cup [10,20]$
\testStop
\kluczStart
A
\kluczStop



\zadStart{Zadanie z Wikieł Z 1.62 c) moja wersja nr 541}

Rozwiązać nierówności $(10-x)(x+7)^{2}(11-x)^{3}\le0$.
\zadStop
\rozwStart{Patryk Wirkus}{}
Miejsca zerowe naszego wielomianu to: $10, -7, 11$.\\
Wielomian jest stopnia parzystego, ponadto znak współczynnika przy\linebreak najwyższej potędze x jest ujemny.\\ W związku z tym wykres wielomianu zaczyna się od lewej strony powyżej osi OX.\\
Ponadto w punkcie $-7$ wykres odbija się od osi poziomej.\\
A więc $$x \in \{-7\} \cup [10,11].$$
\rozwStop
\odpStart
$x \in \{-7\} \cup [10,11]$
\odpStop
\testStart
A.$x \in \{-7\} \cup [10,11]$\\
B.$x \in \{7\} \cup (10,11)$\\
C.$x \in \{-7\} \cup (10,11]$\\
D.$x \in \{7\} \cup (10,11]$\\
E.$x \in \{-7\} \cup [10,11)$\\
F.$x \in \{7\} \cup [10,11)$\\
G.$x \in \{-7\} \cup (10,11)$\\
H.$x \in \{7\} \cup [10,11]$
\testStop
\kluczStart
A
\kluczStop



\zadStart{Zadanie z Wikieł Z 1.62 c) moja wersja nr 542}

Rozwiązać nierówności $(10-x)(x+7)^{2}(12-x)^{3}\le0$.
\zadStop
\rozwStart{Patryk Wirkus}{}
Miejsca zerowe naszego wielomianu to: $10, -7, 12$.\\
Wielomian jest stopnia parzystego, ponadto znak współczynnika przy\linebreak najwyższej potędze x jest ujemny.\\ W związku z tym wykres wielomianu zaczyna się od lewej strony powyżej osi OX.\\
Ponadto w punkcie $-7$ wykres odbija się od osi poziomej.\\
A więc $$x \in \{-7\} \cup [10,12].$$
\rozwStop
\odpStart
$x \in \{-7\} \cup [10,12]$
\odpStop
\testStart
A.$x \in \{-7\} \cup [10,12]$\\
B.$x \in \{7\} \cup (10,12)$\\
C.$x \in \{-7\} \cup (10,12]$\\
D.$x \in \{7\} \cup (10,12]$\\
E.$x \in \{-7\} \cup [10,12)$\\
F.$x \in \{7\} \cup [10,12)$\\
G.$x \in \{-7\} \cup (10,12)$\\
H.$x \in \{7\} \cup [10,12]$
\testStop
\kluczStart
A
\kluczStop



\zadStart{Zadanie z Wikieł Z 1.62 c) moja wersja nr 543}

Rozwiązać nierówności $(10-x)(x+7)^{2}(13-x)^{3}\le0$.
\zadStop
\rozwStart{Patryk Wirkus}{}
Miejsca zerowe naszego wielomianu to: $10, -7, 13$.\\
Wielomian jest stopnia parzystego, ponadto znak współczynnika przy\linebreak najwyższej potędze x jest ujemny.\\ W związku z tym wykres wielomianu zaczyna się od lewej strony powyżej osi OX.\\
Ponadto w punkcie $-7$ wykres odbija się od osi poziomej.\\
A więc $$x \in \{-7\} \cup [10,13].$$
\rozwStop
\odpStart
$x \in \{-7\} \cup [10,13]$
\odpStop
\testStart
A.$x \in \{-7\} \cup [10,13]$\\
B.$x \in \{7\} \cup (10,13)$\\
C.$x \in \{-7\} \cup (10,13]$\\
D.$x \in \{7\} \cup (10,13]$\\
E.$x \in \{-7\} \cup [10,13)$\\
F.$x \in \{7\} \cup [10,13)$\\
G.$x \in \{-7\} \cup (10,13)$\\
H.$x \in \{7\} \cup [10,13]$
\testStop
\kluczStart
A
\kluczStop



\zadStart{Zadanie z Wikieł Z 1.62 c) moja wersja nr 544}

Rozwiązać nierówności $(10-x)(x+7)^{2}(14-x)^{3}\le0$.
\zadStop
\rozwStart{Patryk Wirkus}{}
Miejsca zerowe naszego wielomianu to: $10, -7, 14$.\\
Wielomian jest stopnia parzystego, ponadto znak współczynnika przy\linebreak najwyższej potędze x jest ujemny.\\ W związku z tym wykres wielomianu zaczyna się od lewej strony powyżej osi OX.\\
Ponadto w punkcie $-7$ wykres odbija się od osi poziomej.\\
A więc $$x \in \{-7\} \cup [10,14].$$
\rozwStop
\odpStart
$x \in \{-7\} \cup [10,14]$
\odpStop
\testStart
A.$x \in \{-7\} \cup [10,14]$\\
B.$x \in \{7\} \cup (10,14)$\\
C.$x \in \{-7\} \cup (10,14]$\\
D.$x \in \{7\} \cup (10,14]$\\
E.$x \in \{-7\} \cup [10,14)$\\
F.$x \in \{7\} \cup [10,14)$\\
G.$x \in \{-7\} \cup (10,14)$\\
H.$x \in \{7\} \cup [10,14]$
\testStop
\kluczStart
A
\kluczStop



\zadStart{Zadanie z Wikieł Z 1.62 c) moja wersja nr 545}

Rozwiązać nierówności $(10-x)(x+7)^{2}(15-x)^{3}\le0$.
\zadStop
\rozwStart{Patryk Wirkus}{}
Miejsca zerowe naszego wielomianu to: $10, -7, 15$.\\
Wielomian jest stopnia parzystego, ponadto znak współczynnika przy\linebreak najwyższej potędze x jest ujemny.\\ W związku z tym wykres wielomianu zaczyna się od lewej strony powyżej osi OX.\\
Ponadto w punkcie $-7$ wykres odbija się od osi poziomej.\\
A więc $$x \in \{-7\} \cup [10,15].$$
\rozwStop
\odpStart
$x \in \{-7\} \cup [10,15]$
\odpStop
\testStart
A.$x \in \{-7\} \cup [10,15]$\\
B.$x \in \{7\} \cup (10,15)$\\
C.$x \in \{-7\} \cup (10,15]$\\
D.$x \in \{7\} \cup (10,15]$\\
E.$x \in \{-7\} \cup [10,15)$\\
F.$x \in \{7\} \cup [10,15)$\\
G.$x \in \{-7\} \cup (10,15)$\\
H.$x \in \{7\} \cup [10,15]$
\testStop
\kluczStart
A
\kluczStop



\zadStart{Zadanie z Wikieł Z 1.62 c) moja wersja nr 546}

Rozwiązać nierówności $(10-x)(x+7)^{2}(16-x)^{3}\le0$.
\zadStop
\rozwStart{Patryk Wirkus}{}
Miejsca zerowe naszego wielomianu to: $10, -7, 16$.\\
Wielomian jest stopnia parzystego, ponadto znak współczynnika przy\linebreak najwyższej potędze x jest ujemny.\\ W związku z tym wykres wielomianu zaczyna się od lewej strony powyżej osi OX.\\
Ponadto w punkcie $-7$ wykres odbija się od osi poziomej.\\
A więc $$x \in \{-7\} \cup [10,16].$$
\rozwStop
\odpStart
$x \in \{-7\} \cup [10,16]$
\odpStop
\testStart
A.$x \in \{-7\} \cup [10,16]$\\
B.$x \in \{7\} \cup (10,16)$\\
C.$x \in \{-7\} \cup (10,16]$\\
D.$x \in \{7\} \cup (10,16]$\\
E.$x \in \{-7\} \cup [10,16)$\\
F.$x \in \{7\} \cup [10,16)$\\
G.$x \in \{-7\} \cup (10,16)$\\
H.$x \in \{7\} \cup [10,16]$
\testStop
\kluczStart
A
\kluczStop



\zadStart{Zadanie z Wikieł Z 1.62 c) moja wersja nr 547}

Rozwiązać nierówności $(10-x)(x+7)^{2}(17-x)^{3}\le0$.
\zadStop
\rozwStart{Patryk Wirkus}{}
Miejsca zerowe naszego wielomianu to: $10, -7, 17$.\\
Wielomian jest stopnia parzystego, ponadto znak współczynnika przy\linebreak najwyższej potędze x jest ujemny.\\ W związku z tym wykres wielomianu zaczyna się od lewej strony powyżej osi OX.\\
Ponadto w punkcie $-7$ wykres odbija się od osi poziomej.\\
A więc $$x \in \{-7\} \cup [10,17].$$
\rozwStop
\odpStart
$x \in \{-7\} \cup [10,17]$
\odpStop
\testStart
A.$x \in \{-7\} \cup [10,17]$\\
B.$x \in \{7\} \cup (10,17)$\\
C.$x \in \{-7\} \cup (10,17]$\\
D.$x \in \{7\} \cup (10,17]$\\
E.$x \in \{-7\} \cup [10,17)$\\
F.$x \in \{7\} \cup [10,17)$\\
G.$x \in \{-7\} \cup (10,17)$\\
H.$x \in \{7\} \cup [10,17]$
\testStop
\kluczStart
A
\kluczStop



\zadStart{Zadanie z Wikieł Z 1.62 c) moja wersja nr 548}

Rozwiązać nierówności $(10-x)(x+7)^{2}(18-x)^{3}\le0$.
\zadStop
\rozwStart{Patryk Wirkus}{}
Miejsca zerowe naszego wielomianu to: $10, -7, 18$.\\
Wielomian jest stopnia parzystego, ponadto znak współczynnika przy\linebreak najwyższej potędze x jest ujemny.\\ W związku z tym wykres wielomianu zaczyna się od lewej strony powyżej osi OX.\\
Ponadto w punkcie $-7$ wykres odbija się od osi poziomej.\\
A więc $$x \in \{-7\} \cup [10,18].$$
\rozwStop
\odpStart
$x \in \{-7\} \cup [10,18]$
\odpStop
\testStart
A.$x \in \{-7\} \cup [10,18]$\\
B.$x \in \{7\} \cup (10,18)$\\
C.$x \in \{-7\} \cup (10,18]$\\
D.$x \in \{7\} \cup (10,18]$\\
E.$x \in \{-7\} \cup [10,18)$\\
F.$x \in \{7\} \cup [10,18)$\\
G.$x \in \{-7\} \cup (10,18)$\\
H.$x \in \{7\} \cup [10,18]$
\testStop
\kluczStart
A
\kluczStop



\zadStart{Zadanie z Wikieł Z 1.62 c) moja wersja nr 549}

Rozwiązać nierówności $(10-x)(x+7)^{2}(19-x)^{3}\le0$.
\zadStop
\rozwStart{Patryk Wirkus}{}
Miejsca zerowe naszego wielomianu to: $10, -7, 19$.\\
Wielomian jest stopnia parzystego, ponadto znak współczynnika przy\linebreak najwyższej potędze x jest ujemny.\\ W związku z tym wykres wielomianu zaczyna się od lewej strony powyżej osi OX.\\
Ponadto w punkcie $-7$ wykres odbija się od osi poziomej.\\
A więc $$x \in \{-7\} \cup [10,19].$$
\rozwStop
\odpStart
$x \in \{-7\} \cup [10,19]$
\odpStop
\testStart
A.$x \in \{-7\} \cup [10,19]$\\
B.$x \in \{7\} \cup (10,19)$\\
C.$x \in \{-7\} \cup (10,19]$\\
D.$x \in \{7\} \cup (10,19]$\\
E.$x \in \{-7\} \cup [10,19)$\\
F.$x \in \{7\} \cup [10,19)$\\
G.$x \in \{-7\} \cup (10,19)$\\
H.$x \in \{7\} \cup [10,19]$
\testStop
\kluczStart
A
\kluczStop



\zadStart{Zadanie z Wikieł Z 1.62 c) moja wersja nr 550}

Rozwiązać nierówności $(10-x)(x+7)^{2}(20-x)^{3}\le0$.
\zadStop
\rozwStart{Patryk Wirkus}{}
Miejsca zerowe naszego wielomianu to: $10, -7, 20$.\\
Wielomian jest stopnia parzystego, ponadto znak współczynnika przy\linebreak najwyższej potędze x jest ujemny.\\ W związku z tym wykres wielomianu zaczyna się od lewej strony powyżej osi OX.\\
Ponadto w punkcie $-7$ wykres odbija się od osi poziomej.\\
A więc $$x \in \{-7\} \cup [10,20].$$
\rozwStop
\odpStart
$x \in \{-7\} \cup [10,20]$
\odpStop
\testStart
A.$x \in \{-7\} \cup [10,20]$\\
B.$x \in \{7\} \cup (10,20)$\\
C.$x \in \{-7\} \cup (10,20]$\\
D.$x \in \{7\} \cup (10,20]$\\
E.$x \in \{-7\} \cup [10,20)$\\
F.$x \in \{7\} \cup [10,20)$\\
G.$x \in \{-7\} \cup (10,20)$\\
H.$x \in \{7\} \cup [10,20]$
\testStop
\kluczStart
A
\kluczStop



\zadStart{Zadanie z Wikieł Z 1.62 c) moja wersja nr 551}

Rozwiązać nierówności $(10-x)(x+8)^{2}(11-x)^{3}\le0$.
\zadStop
\rozwStart{Patryk Wirkus}{}
Miejsca zerowe naszego wielomianu to: $10, -8, 11$.\\
Wielomian jest stopnia parzystego, ponadto znak współczynnika przy\linebreak najwyższej potędze x jest ujemny.\\ W związku z tym wykres wielomianu zaczyna się od lewej strony powyżej osi OX.\\
Ponadto w punkcie $-8$ wykres odbija się od osi poziomej.\\
A więc $$x \in \{-8\} \cup [10,11].$$
\rozwStop
\odpStart
$x \in \{-8\} \cup [10,11]$
\odpStop
\testStart
A.$x \in \{-8\} \cup [10,11]$\\
B.$x \in \{8\} \cup (10,11)$\\
C.$x \in \{-8\} \cup (10,11]$\\
D.$x \in \{8\} \cup (10,11]$\\
E.$x \in \{-8\} \cup [10,11)$\\
F.$x \in \{8\} \cup [10,11)$\\
G.$x \in \{-8\} \cup (10,11)$\\
H.$x \in \{8\} \cup [10,11]$
\testStop
\kluczStart
A
\kluczStop



\zadStart{Zadanie z Wikieł Z 1.62 c) moja wersja nr 552}

Rozwiązać nierówności $(10-x)(x+8)^{2}(12-x)^{3}\le0$.
\zadStop
\rozwStart{Patryk Wirkus}{}
Miejsca zerowe naszego wielomianu to: $10, -8, 12$.\\
Wielomian jest stopnia parzystego, ponadto znak współczynnika przy\linebreak najwyższej potędze x jest ujemny.\\ W związku z tym wykres wielomianu zaczyna się od lewej strony powyżej osi OX.\\
Ponadto w punkcie $-8$ wykres odbija się od osi poziomej.\\
A więc $$x \in \{-8\} \cup [10,12].$$
\rozwStop
\odpStart
$x \in \{-8\} \cup [10,12]$
\odpStop
\testStart
A.$x \in \{-8\} \cup [10,12]$\\
B.$x \in \{8\} \cup (10,12)$\\
C.$x \in \{-8\} \cup (10,12]$\\
D.$x \in \{8\} \cup (10,12]$\\
E.$x \in \{-8\} \cup [10,12)$\\
F.$x \in \{8\} \cup [10,12)$\\
G.$x \in \{-8\} \cup (10,12)$\\
H.$x \in \{8\} \cup [10,12]$
\testStop
\kluczStart
A
\kluczStop



\zadStart{Zadanie z Wikieł Z 1.62 c) moja wersja nr 553}

Rozwiązać nierówności $(10-x)(x+8)^{2}(13-x)^{3}\le0$.
\zadStop
\rozwStart{Patryk Wirkus}{}
Miejsca zerowe naszego wielomianu to: $10, -8, 13$.\\
Wielomian jest stopnia parzystego, ponadto znak współczynnika przy\linebreak najwyższej potędze x jest ujemny.\\ W związku z tym wykres wielomianu zaczyna się od lewej strony powyżej osi OX.\\
Ponadto w punkcie $-8$ wykres odbija się od osi poziomej.\\
A więc $$x \in \{-8\} \cup [10,13].$$
\rozwStop
\odpStart
$x \in \{-8\} \cup [10,13]$
\odpStop
\testStart
A.$x \in \{-8\} \cup [10,13]$\\
B.$x \in \{8\} \cup (10,13)$\\
C.$x \in \{-8\} \cup (10,13]$\\
D.$x \in \{8\} \cup (10,13]$\\
E.$x \in \{-8\} \cup [10,13)$\\
F.$x \in \{8\} \cup [10,13)$\\
G.$x \in \{-8\} \cup (10,13)$\\
H.$x \in \{8\} \cup [10,13]$
\testStop
\kluczStart
A
\kluczStop



\zadStart{Zadanie z Wikieł Z 1.62 c) moja wersja nr 554}

Rozwiązać nierówności $(10-x)(x+8)^{2}(14-x)^{3}\le0$.
\zadStop
\rozwStart{Patryk Wirkus}{}
Miejsca zerowe naszego wielomianu to: $10, -8, 14$.\\
Wielomian jest stopnia parzystego, ponadto znak współczynnika przy\linebreak najwyższej potędze x jest ujemny.\\ W związku z tym wykres wielomianu zaczyna się od lewej strony powyżej osi OX.\\
Ponadto w punkcie $-8$ wykres odbija się od osi poziomej.\\
A więc $$x \in \{-8\} \cup [10,14].$$
\rozwStop
\odpStart
$x \in \{-8\} \cup [10,14]$
\odpStop
\testStart
A.$x \in \{-8\} \cup [10,14]$\\
B.$x \in \{8\} \cup (10,14)$\\
C.$x \in \{-8\} \cup (10,14]$\\
D.$x \in \{8\} \cup (10,14]$\\
E.$x \in \{-8\} \cup [10,14)$\\
F.$x \in \{8\} \cup [10,14)$\\
G.$x \in \{-8\} \cup (10,14)$\\
H.$x \in \{8\} \cup [10,14]$
\testStop
\kluczStart
A
\kluczStop



\zadStart{Zadanie z Wikieł Z 1.62 c) moja wersja nr 555}

Rozwiązać nierówności $(10-x)(x+8)^{2}(15-x)^{3}\le0$.
\zadStop
\rozwStart{Patryk Wirkus}{}
Miejsca zerowe naszego wielomianu to: $10, -8, 15$.\\
Wielomian jest stopnia parzystego, ponadto znak współczynnika przy\linebreak najwyższej potędze x jest ujemny.\\ W związku z tym wykres wielomianu zaczyna się od lewej strony powyżej osi OX.\\
Ponadto w punkcie $-8$ wykres odbija się od osi poziomej.\\
A więc $$x \in \{-8\} \cup [10,15].$$
\rozwStop
\odpStart
$x \in \{-8\} \cup [10,15]$
\odpStop
\testStart
A.$x \in \{-8\} \cup [10,15]$\\
B.$x \in \{8\} \cup (10,15)$\\
C.$x \in \{-8\} \cup (10,15]$\\
D.$x \in \{8\} \cup (10,15]$\\
E.$x \in \{-8\} \cup [10,15)$\\
F.$x \in \{8\} \cup [10,15)$\\
G.$x \in \{-8\} \cup (10,15)$\\
H.$x \in \{8\} \cup [10,15]$
\testStop
\kluczStart
A
\kluczStop



\zadStart{Zadanie z Wikieł Z 1.62 c) moja wersja nr 556}

Rozwiązać nierówności $(10-x)(x+8)^{2}(16-x)^{3}\le0$.
\zadStop
\rozwStart{Patryk Wirkus}{}
Miejsca zerowe naszego wielomianu to: $10, -8, 16$.\\
Wielomian jest stopnia parzystego, ponadto znak współczynnika przy\linebreak najwyższej potędze x jest ujemny.\\ W związku z tym wykres wielomianu zaczyna się od lewej strony powyżej osi OX.\\
Ponadto w punkcie $-8$ wykres odbija się od osi poziomej.\\
A więc $$x \in \{-8\} \cup [10,16].$$
\rozwStop
\odpStart
$x \in \{-8\} \cup [10,16]$
\odpStop
\testStart
A.$x \in \{-8\} \cup [10,16]$\\
B.$x \in \{8\} \cup (10,16)$\\
C.$x \in \{-8\} \cup (10,16]$\\
D.$x \in \{8\} \cup (10,16]$\\
E.$x \in \{-8\} \cup [10,16)$\\
F.$x \in \{8\} \cup [10,16)$\\
G.$x \in \{-8\} \cup (10,16)$\\
H.$x \in \{8\} \cup [10,16]$
\testStop
\kluczStart
A
\kluczStop



\zadStart{Zadanie z Wikieł Z 1.62 c) moja wersja nr 557}

Rozwiązać nierówności $(10-x)(x+8)^{2}(17-x)^{3}\le0$.
\zadStop
\rozwStart{Patryk Wirkus}{}
Miejsca zerowe naszego wielomianu to: $10, -8, 17$.\\
Wielomian jest stopnia parzystego, ponadto znak współczynnika przy\linebreak najwyższej potędze x jest ujemny.\\ W związku z tym wykres wielomianu zaczyna się od lewej strony powyżej osi OX.\\
Ponadto w punkcie $-8$ wykres odbija się od osi poziomej.\\
A więc $$x \in \{-8\} \cup [10,17].$$
\rozwStop
\odpStart
$x \in \{-8\} \cup [10,17]$
\odpStop
\testStart
A.$x \in \{-8\} \cup [10,17]$\\
B.$x \in \{8\} \cup (10,17)$\\
C.$x \in \{-8\} \cup (10,17]$\\
D.$x \in \{8\} \cup (10,17]$\\
E.$x \in \{-8\} \cup [10,17)$\\
F.$x \in \{8\} \cup [10,17)$\\
G.$x \in \{-8\} \cup (10,17)$\\
H.$x \in \{8\} \cup [10,17]$
\testStop
\kluczStart
A
\kluczStop



\zadStart{Zadanie z Wikieł Z 1.62 c) moja wersja nr 558}

Rozwiązać nierówności $(10-x)(x+8)^{2}(18-x)^{3}\le0$.
\zadStop
\rozwStart{Patryk Wirkus}{}
Miejsca zerowe naszego wielomianu to: $10, -8, 18$.\\
Wielomian jest stopnia parzystego, ponadto znak współczynnika przy\linebreak najwyższej potędze x jest ujemny.\\ W związku z tym wykres wielomianu zaczyna się od lewej strony powyżej osi OX.\\
Ponadto w punkcie $-8$ wykres odbija się od osi poziomej.\\
A więc $$x \in \{-8\} \cup [10,18].$$
\rozwStop
\odpStart
$x \in \{-8\} \cup [10,18]$
\odpStop
\testStart
A.$x \in \{-8\} \cup [10,18]$\\
B.$x \in \{8\} \cup (10,18)$\\
C.$x \in \{-8\} \cup (10,18]$\\
D.$x \in \{8\} \cup (10,18]$\\
E.$x \in \{-8\} \cup [10,18)$\\
F.$x \in \{8\} \cup [10,18)$\\
G.$x \in \{-8\} \cup (10,18)$\\
H.$x \in \{8\} \cup [10,18]$
\testStop
\kluczStart
A
\kluczStop



\zadStart{Zadanie z Wikieł Z 1.62 c) moja wersja nr 559}

Rozwiązać nierówności $(10-x)(x+8)^{2}(19-x)^{3}\le0$.
\zadStop
\rozwStart{Patryk Wirkus}{}
Miejsca zerowe naszego wielomianu to: $10, -8, 19$.\\
Wielomian jest stopnia parzystego, ponadto znak współczynnika przy\linebreak najwyższej potędze x jest ujemny.\\ W związku z tym wykres wielomianu zaczyna się od lewej strony powyżej osi OX.\\
Ponadto w punkcie $-8$ wykres odbija się od osi poziomej.\\
A więc $$x \in \{-8\} \cup [10,19].$$
\rozwStop
\odpStart
$x \in \{-8\} \cup [10,19]$
\odpStop
\testStart
A.$x \in \{-8\} \cup [10,19]$\\
B.$x \in \{8\} \cup (10,19)$\\
C.$x \in \{-8\} \cup (10,19]$\\
D.$x \in \{8\} \cup (10,19]$\\
E.$x \in \{-8\} \cup [10,19)$\\
F.$x \in \{8\} \cup [10,19)$\\
G.$x \in \{-8\} \cup (10,19)$\\
H.$x \in \{8\} \cup [10,19]$
\testStop
\kluczStart
A
\kluczStop



\zadStart{Zadanie z Wikieł Z 1.62 c) moja wersja nr 560}

Rozwiązać nierówności $(10-x)(x+8)^{2}(20-x)^{3}\le0$.
\zadStop
\rozwStart{Patryk Wirkus}{}
Miejsca zerowe naszego wielomianu to: $10, -8, 20$.\\
Wielomian jest stopnia parzystego, ponadto znak współczynnika przy\linebreak najwyższej potędze x jest ujemny.\\ W związku z tym wykres wielomianu zaczyna się od lewej strony powyżej osi OX.\\
Ponadto w punkcie $-8$ wykres odbija się od osi poziomej.\\
A więc $$x \in \{-8\} \cup [10,20].$$
\rozwStop
\odpStart
$x \in \{-8\} \cup [10,20]$
\odpStop
\testStart
A.$x \in \{-8\} \cup [10,20]$\\
B.$x \in \{8\} \cup (10,20)$\\
C.$x \in \{-8\} \cup (10,20]$\\
D.$x \in \{8\} \cup (10,20]$\\
E.$x \in \{-8\} \cup [10,20)$\\
F.$x \in \{8\} \cup [10,20)$\\
G.$x \in \{-8\} \cup (10,20)$\\
H.$x \in \{8\} \cup [10,20]$
\testStop
\kluczStart
A
\kluczStop



\zadStart{Zadanie z Wikieł Z 1.62 c) moja wersja nr 561}

Rozwiązać nierówności $(10-x)(x+9)^{2}(11-x)^{3}\le0$.
\zadStop
\rozwStart{Patryk Wirkus}{}
Miejsca zerowe naszego wielomianu to: $10, -9, 11$.\\
Wielomian jest stopnia parzystego, ponadto znak współczynnika przy\linebreak najwyższej potędze x jest ujemny.\\ W związku z tym wykres wielomianu zaczyna się od lewej strony powyżej osi OX.\\
Ponadto w punkcie $-9$ wykres odbija się od osi poziomej.\\
A więc $$x \in \{-9\} \cup [10,11].$$
\rozwStop
\odpStart
$x \in \{-9\} \cup [10,11]$
\odpStop
\testStart
A.$x \in \{-9\} \cup [10,11]$\\
B.$x \in \{9\} \cup (10,11)$\\
C.$x \in \{-9\} \cup (10,11]$\\
D.$x \in \{9\} \cup (10,11]$\\
E.$x \in \{-9\} \cup [10,11)$\\
F.$x \in \{9\} \cup [10,11)$\\
G.$x \in \{-9\} \cup (10,11)$\\
H.$x \in \{9\} \cup [10,11]$
\testStop
\kluczStart
A
\kluczStop



\zadStart{Zadanie z Wikieł Z 1.62 c) moja wersja nr 562}

Rozwiązać nierówności $(10-x)(x+9)^{2}(12-x)^{3}\le0$.
\zadStop
\rozwStart{Patryk Wirkus}{}
Miejsca zerowe naszego wielomianu to: $10, -9, 12$.\\
Wielomian jest stopnia parzystego, ponadto znak współczynnika przy\linebreak najwyższej potędze x jest ujemny.\\ W związku z tym wykres wielomianu zaczyna się od lewej strony powyżej osi OX.\\
Ponadto w punkcie $-9$ wykres odbija się od osi poziomej.\\
A więc $$x \in \{-9\} \cup [10,12].$$
\rozwStop
\odpStart
$x \in \{-9\} \cup [10,12]$
\odpStop
\testStart
A.$x \in \{-9\} \cup [10,12]$\\
B.$x \in \{9\} \cup (10,12)$\\
C.$x \in \{-9\} \cup (10,12]$\\
D.$x \in \{9\} \cup (10,12]$\\
E.$x \in \{-9\} \cup [10,12)$\\
F.$x \in \{9\} \cup [10,12)$\\
G.$x \in \{-9\} \cup (10,12)$\\
H.$x \in \{9\} \cup [10,12]$
\testStop
\kluczStart
A
\kluczStop



\zadStart{Zadanie z Wikieł Z 1.62 c) moja wersja nr 563}

Rozwiązać nierówności $(10-x)(x+9)^{2}(13-x)^{3}\le0$.
\zadStop
\rozwStart{Patryk Wirkus}{}
Miejsca zerowe naszego wielomianu to: $10, -9, 13$.\\
Wielomian jest stopnia parzystego, ponadto znak współczynnika przy\linebreak najwyższej potędze x jest ujemny.\\ W związku z tym wykres wielomianu zaczyna się od lewej strony powyżej osi OX.\\
Ponadto w punkcie $-9$ wykres odbija się od osi poziomej.\\
A więc $$x \in \{-9\} \cup [10,13].$$
\rozwStop
\odpStart
$x \in \{-9\} \cup [10,13]$
\odpStop
\testStart
A.$x \in \{-9\} \cup [10,13]$\\
B.$x \in \{9\} \cup (10,13)$\\
C.$x \in \{-9\} \cup (10,13]$\\
D.$x \in \{9\} \cup (10,13]$\\
E.$x \in \{-9\} \cup [10,13)$\\
F.$x \in \{9\} \cup [10,13)$\\
G.$x \in \{-9\} \cup (10,13)$\\
H.$x \in \{9\} \cup [10,13]$
\testStop
\kluczStart
A
\kluczStop



\zadStart{Zadanie z Wikieł Z 1.62 c) moja wersja nr 564}

Rozwiązać nierówności $(10-x)(x+9)^{2}(14-x)^{3}\le0$.
\zadStop
\rozwStart{Patryk Wirkus}{}
Miejsca zerowe naszego wielomianu to: $10, -9, 14$.\\
Wielomian jest stopnia parzystego, ponadto znak współczynnika przy\linebreak najwyższej potędze x jest ujemny.\\ W związku z tym wykres wielomianu zaczyna się od lewej strony powyżej osi OX.\\
Ponadto w punkcie $-9$ wykres odbija się od osi poziomej.\\
A więc $$x \in \{-9\} \cup [10,14].$$
\rozwStop
\odpStart
$x \in \{-9\} \cup [10,14]$
\odpStop
\testStart
A.$x \in \{-9\} \cup [10,14]$\\
B.$x \in \{9\} \cup (10,14)$\\
C.$x \in \{-9\} \cup (10,14]$\\
D.$x \in \{9\} \cup (10,14]$\\
E.$x \in \{-9\} \cup [10,14)$\\
F.$x \in \{9\} \cup [10,14)$\\
G.$x \in \{-9\} \cup (10,14)$\\
H.$x \in \{9\} \cup [10,14]$
\testStop
\kluczStart
A
\kluczStop



\zadStart{Zadanie z Wikieł Z 1.62 c) moja wersja nr 565}

Rozwiązać nierówności $(10-x)(x+9)^{2}(15-x)^{3}\le0$.
\zadStop
\rozwStart{Patryk Wirkus}{}
Miejsca zerowe naszego wielomianu to: $10, -9, 15$.\\
Wielomian jest stopnia parzystego, ponadto znak współczynnika przy\linebreak najwyższej potędze x jest ujemny.\\ W związku z tym wykres wielomianu zaczyna się od lewej strony powyżej osi OX.\\
Ponadto w punkcie $-9$ wykres odbija się od osi poziomej.\\
A więc $$x \in \{-9\} \cup [10,15].$$
\rozwStop
\odpStart
$x \in \{-9\} \cup [10,15]$
\odpStop
\testStart
A.$x \in \{-9\} \cup [10,15]$\\
B.$x \in \{9\} \cup (10,15)$\\
C.$x \in \{-9\} \cup (10,15]$\\
D.$x \in \{9\} \cup (10,15]$\\
E.$x \in \{-9\} \cup [10,15)$\\
F.$x \in \{9\} \cup [10,15)$\\
G.$x \in \{-9\} \cup (10,15)$\\
H.$x \in \{9\} \cup [10,15]$
\testStop
\kluczStart
A
\kluczStop



\zadStart{Zadanie z Wikieł Z 1.62 c) moja wersja nr 566}

Rozwiązać nierówności $(10-x)(x+9)^{2}(16-x)^{3}\le0$.
\zadStop
\rozwStart{Patryk Wirkus}{}
Miejsca zerowe naszego wielomianu to: $10, -9, 16$.\\
Wielomian jest stopnia parzystego, ponadto znak współczynnika przy\linebreak najwyższej potędze x jest ujemny.\\ W związku z tym wykres wielomianu zaczyna się od lewej strony powyżej osi OX.\\
Ponadto w punkcie $-9$ wykres odbija się od osi poziomej.\\
A więc $$x \in \{-9\} \cup [10,16].$$
\rozwStop
\odpStart
$x \in \{-9\} \cup [10,16]$
\odpStop
\testStart
A.$x \in \{-9\} \cup [10,16]$\\
B.$x \in \{9\} \cup (10,16)$\\
C.$x \in \{-9\} \cup (10,16]$\\
D.$x \in \{9\} \cup (10,16]$\\
E.$x \in \{-9\} \cup [10,16)$\\
F.$x \in \{9\} \cup [10,16)$\\
G.$x \in \{-9\} \cup (10,16)$\\
H.$x \in \{9\} \cup [10,16]$
\testStop
\kluczStart
A
\kluczStop



\zadStart{Zadanie z Wikieł Z 1.62 c) moja wersja nr 567}

Rozwiązać nierówności $(10-x)(x+9)^{2}(17-x)^{3}\le0$.
\zadStop
\rozwStart{Patryk Wirkus}{}
Miejsca zerowe naszego wielomianu to: $10, -9, 17$.\\
Wielomian jest stopnia parzystego, ponadto znak współczynnika przy\linebreak najwyższej potędze x jest ujemny.\\ W związku z tym wykres wielomianu zaczyna się od lewej strony powyżej osi OX.\\
Ponadto w punkcie $-9$ wykres odbija się od osi poziomej.\\
A więc $$x \in \{-9\} \cup [10,17].$$
\rozwStop
\odpStart
$x \in \{-9\} \cup [10,17]$
\odpStop
\testStart
A.$x \in \{-9\} \cup [10,17]$\\
B.$x \in \{9\} \cup (10,17)$\\
C.$x \in \{-9\} \cup (10,17]$\\
D.$x \in \{9\} \cup (10,17]$\\
E.$x \in \{-9\} \cup [10,17)$\\
F.$x \in \{9\} \cup [10,17)$\\
G.$x \in \{-9\} \cup (10,17)$\\
H.$x \in \{9\} \cup [10,17]$
\testStop
\kluczStart
A
\kluczStop



\zadStart{Zadanie z Wikieł Z 1.62 c) moja wersja nr 568}

Rozwiązać nierówności $(10-x)(x+9)^{2}(18-x)^{3}\le0$.
\zadStop
\rozwStart{Patryk Wirkus}{}
Miejsca zerowe naszego wielomianu to: $10, -9, 18$.\\
Wielomian jest stopnia parzystego, ponadto znak współczynnika przy\linebreak najwyższej potędze x jest ujemny.\\ W związku z tym wykres wielomianu zaczyna się od lewej strony powyżej osi OX.\\
Ponadto w punkcie $-9$ wykres odbija się od osi poziomej.\\
A więc $$x \in \{-9\} \cup [10,18].$$
\rozwStop
\odpStart
$x \in \{-9\} \cup [10,18]$
\odpStop
\testStart
A.$x \in \{-9\} \cup [10,18]$\\
B.$x \in \{9\} \cup (10,18)$\\
C.$x \in \{-9\} \cup (10,18]$\\
D.$x \in \{9\} \cup (10,18]$\\
E.$x \in \{-9\} \cup [10,18)$\\
F.$x \in \{9\} \cup [10,18)$\\
G.$x \in \{-9\} \cup (10,18)$\\
H.$x \in \{9\} \cup [10,18]$
\testStop
\kluczStart
A
\kluczStop



\zadStart{Zadanie z Wikieł Z 1.62 c) moja wersja nr 569}

Rozwiązać nierówności $(10-x)(x+9)^{2}(19-x)^{3}\le0$.
\zadStop
\rozwStart{Patryk Wirkus}{}
Miejsca zerowe naszego wielomianu to: $10, -9, 19$.\\
Wielomian jest stopnia parzystego, ponadto znak współczynnika przy\linebreak najwyższej potędze x jest ujemny.\\ W związku z tym wykres wielomianu zaczyna się od lewej strony powyżej osi OX.\\
Ponadto w punkcie $-9$ wykres odbija się od osi poziomej.\\
A więc $$x \in \{-9\} \cup [10,19].$$
\rozwStop
\odpStart
$x \in \{-9\} \cup [10,19]$
\odpStop
\testStart
A.$x \in \{-9\} \cup [10,19]$\\
B.$x \in \{9\} \cup (10,19)$\\
C.$x \in \{-9\} \cup (10,19]$\\
D.$x \in \{9\} \cup (10,19]$\\
E.$x \in \{-9\} \cup [10,19)$\\
F.$x \in \{9\} \cup [10,19)$\\
G.$x \in \{-9\} \cup (10,19)$\\
H.$x \in \{9\} \cup [10,19]$
\testStop
\kluczStart
A
\kluczStop



\zadStart{Zadanie z Wikieł Z 1.62 c) moja wersja nr 570}

Rozwiązać nierówności $(10-x)(x+9)^{2}(20-x)^{3}\le0$.
\zadStop
\rozwStart{Patryk Wirkus}{}
Miejsca zerowe naszego wielomianu to: $10, -9, 20$.\\
Wielomian jest stopnia parzystego, ponadto znak współczynnika przy\linebreak najwyższej potędze x jest ujemny.\\ W związku z tym wykres wielomianu zaczyna się od lewej strony powyżej osi OX.\\
Ponadto w punkcie $-9$ wykres odbija się od osi poziomej.\\
A więc $$x \in \{-9\} \cup [10,20].$$
\rozwStop
\odpStart
$x \in \{-9\} \cup [10,20]$
\odpStop
\testStart
A.$x \in \{-9\} \cup [10,20]$\\
B.$x \in \{9\} \cup (10,20)$\\
C.$x \in \{-9\} \cup (10,20]$\\
D.$x \in \{9\} \cup (10,20]$\\
E.$x \in \{-9\} \cup [10,20)$\\
F.$x \in \{9\} \cup [10,20)$\\
G.$x \in \{-9\} \cup (10,20)$\\
H.$x \in \{9\} \cup [10,20]$
\testStop
\kluczStart
A
\kluczStop



\zadStart{Zadanie z Wikieł Z 1.62 c) moja wersja nr 571}

Rozwiązać nierówności $(11-x)(x+1)^{2}(12-x)^{3}\le0$.
\zadStop
\rozwStart{Patryk Wirkus}{}
Miejsca zerowe naszego wielomianu to: $11, -1, 12$.\\
Wielomian jest stopnia parzystego, ponadto znak współczynnika przy\linebreak najwyższej potędze x jest ujemny.\\ W związku z tym wykres wielomianu zaczyna się od lewej strony powyżej osi OX.\\
Ponadto w punkcie $-1$ wykres odbija się od osi poziomej.\\
A więc $$x \in \{-1\} \cup [11,12].$$
\rozwStop
\odpStart
$x \in \{-1\} \cup [11,12]$
\odpStop
\testStart
A.$x \in \{-1\} \cup [11,12]$\\
B.$x \in \{1\} \cup (11,12)$\\
C.$x \in \{-1\} \cup (11,12]$\\
D.$x \in \{1\} \cup (11,12]$\\
E.$x \in \{-1\} \cup [11,12)$\\
F.$x \in \{1\} \cup [11,12)$\\
G.$x \in \{-1\} \cup (11,12)$\\
H.$x \in \{1\} \cup [11,12]$
\testStop
\kluczStart
A
\kluczStop



\zadStart{Zadanie z Wikieł Z 1.62 c) moja wersja nr 572}

Rozwiązać nierówności $(11-x)(x+1)^{2}(13-x)^{3}\le0$.
\zadStop
\rozwStart{Patryk Wirkus}{}
Miejsca zerowe naszego wielomianu to: $11, -1, 13$.\\
Wielomian jest stopnia parzystego, ponadto znak współczynnika przy\linebreak najwyższej potędze x jest ujemny.\\ W związku z tym wykres wielomianu zaczyna się od lewej strony powyżej osi OX.\\
Ponadto w punkcie $-1$ wykres odbija się od osi poziomej.\\
A więc $$x \in \{-1\} \cup [11,13].$$
\rozwStop
\odpStart
$x \in \{-1\} \cup [11,13]$
\odpStop
\testStart
A.$x \in \{-1\} \cup [11,13]$\\
B.$x \in \{1\} \cup (11,13)$\\
C.$x \in \{-1\} \cup (11,13]$\\
D.$x \in \{1\} \cup (11,13]$\\
E.$x \in \{-1\} \cup [11,13)$\\
F.$x \in \{1\} \cup [11,13)$\\
G.$x \in \{-1\} \cup (11,13)$\\
H.$x \in \{1\} \cup [11,13]$
\testStop
\kluczStart
A
\kluczStop



\zadStart{Zadanie z Wikieł Z 1.62 c) moja wersja nr 573}

Rozwiązać nierówności $(11-x)(x+1)^{2}(14-x)^{3}\le0$.
\zadStop
\rozwStart{Patryk Wirkus}{}
Miejsca zerowe naszego wielomianu to: $11, -1, 14$.\\
Wielomian jest stopnia parzystego, ponadto znak współczynnika przy\linebreak najwyższej potędze x jest ujemny.\\ W związku z tym wykres wielomianu zaczyna się od lewej strony powyżej osi OX.\\
Ponadto w punkcie $-1$ wykres odbija się od osi poziomej.\\
A więc $$x \in \{-1\} \cup [11,14].$$
\rozwStop
\odpStart
$x \in \{-1\} \cup [11,14]$
\odpStop
\testStart
A.$x \in \{-1\} \cup [11,14]$\\
B.$x \in \{1\} \cup (11,14)$\\
C.$x \in \{-1\} \cup (11,14]$\\
D.$x \in \{1\} \cup (11,14]$\\
E.$x \in \{-1\} \cup [11,14)$\\
F.$x \in \{1\} \cup [11,14)$\\
G.$x \in \{-1\} \cup (11,14)$\\
H.$x \in \{1\} \cup [11,14]$
\testStop
\kluczStart
A
\kluczStop



\zadStart{Zadanie z Wikieł Z 1.62 c) moja wersja nr 574}

Rozwiązać nierówności $(11-x)(x+1)^{2}(15-x)^{3}\le0$.
\zadStop
\rozwStart{Patryk Wirkus}{}
Miejsca zerowe naszego wielomianu to: $11, -1, 15$.\\
Wielomian jest stopnia parzystego, ponadto znak współczynnika przy\linebreak najwyższej potędze x jest ujemny.\\ W związku z tym wykres wielomianu zaczyna się od lewej strony powyżej osi OX.\\
Ponadto w punkcie $-1$ wykres odbija się od osi poziomej.\\
A więc $$x \in \{-1\} \cup [11,15].$$
\rozwStop
\odpStart
$x \in \{-1\} \cup [11,15]$
\odpStop
\testStart
A.$x \in \{-1\} \cup [11,15]$\\
B.$x \in \{1\} \cup (11,15)$\\
C.$x \in \{-1\} \cup (11,15]$\\
D.$x \in \{1\} \cup (11,15]$\\
E.$x \in \{-1\} \cup [11,15)$\\
F.$x \in \{1\} \cup [11,15)$\\
G.$x \in \{-1\} \cup (11,15)$\\
H.$x \in \{1\} \cup [11,15]$
\testStop
\kluczStart
A
\kluczStop



\zadStart{Zadanie z Wikieł Z 1.62 c) moja wersja nr 575}

Rozwiązać nierówności $(11-x)(x+1)^{2}(16-x)^{3}\le0$.
\zadStop
\rozwStart{Patryk Wirkus}{}
Miejsca zerowe naszego wielomianu to: $11, -1, 16$.\\
Wielomian jest stopnia parzystego, ponadto znak współczynnika przy\linebreak najwyższej potędze x jest ujemny.\\ W związku z tym wykres wielomianu zaczyna się od lewej strony powyżej osi OX.\\
Ponadto w punkcie $-1$ wykres odbija się od osi poziomej.\\
A więc $$x \in \{-1\} \cup [11,16].$$
\rozwStop
\odpStart
$x \in \{-1\} \cup [11,16]$
\odpStop
\testStart
A.$x \in \{-1\} \cup [11,16]$\\
B.$x \in \{1\} \cup (11,16)$\\
C.$x \in \{-1\} \cup (11,16]$\\
D.$x \in \{1\} \cup (11,16]$\\
E.$x \in \{-1\} \cup [11,16)$\\
F.$x \in \{1\} \cup [11,16)$\\
G.$x \in \{-1\} \cup (11,16)$\\
H.$x \in \{1\} \cup [11,16]$
\testStop
\kluczStart
A
\kluczStop



\zadStart{Zadanie z Wikieł Z 1.62 c) moja wersja nr 576}

Rozwiązać nierówności $(11-x)(x+1)^{2}(17-x)^{3}\le0$.
\zadStop
\rozwStart{Patryk Wirkus}{}
Miejsca zerowe naszego wielomianu to: $11, -1, 17$.\\
Wielomian jest stopnia parzystego, ponadto znak współczynnika przy\linebreak najwyższej potędze x jest ujemny.\\ W związku z tym wykres wielomianu zaczyna się od lewej strony powyżej osi OX.\\
Ponadto w punkcie $-1$ wykres odbija się od osi poziomej.\\
A więc $$x \in \{-1\} \cup [11,17].$$
\rozwStop
\odpStart
$x \in \{-1\} \cup [11,17]$
\odpStop
\testStart
A.$x \in \{-1\} \cup [11,17]$\\
B.$x \in \{1\} \cup (11,17)$\\
C.$x \in \{-1\} \cup (11,17]$\\
D.$x \in \{1\} \cup (11,17]$\\
E.$x \in \{-1\} \cup [11,17)$\\
F.$x \in \{1\} \cup [11,17)$\\
G.$x \in \{-1\} \cup (11,17)$\\
H.$x \in \{1\} \cup [11,17]$
\testStop
\kluczStart
A
\kluczStop



\zadStart{Zadanie z Wikieł Z 1.62 c) moja wersja nr 577}

Rozwiązać nierówności $(11-x)(x+1)^{2}(18-x)^{3}\le0$.
\zadStop
\rozwStart{Patryk Wirkus}{}
Miejsca zerowe naszego wielomianu to: $11, -1, 18$.\\
Wielomian jest stopnia parzystego, ponadto znak współczynnika przy\linebreak najwyższej potędze x jest ujemny.\\ W związku z tym wykres wielomianu zaczyna się od lewej strony powyżej osi OX.\\
Ponadto w punkcie $-1$ wykres odbija się od osi poziomej.\\
A więc $$x \in \{-1\} \cup [11,18].$$
\rozwStop
\odpStart
$x \in \{-1\} \cup [11,18]$
\odpStop
\testStart
A.$x \in \{-1\} \cup [11,18]$\\
B.$x \in \{1\} \cup (11,18)$\\
C.$x \in \{-1\} \cup (11,18]$\\
D.$x \in \{1\} \cup (11,18]$\\
E.$x \in \{-1\} \cup [11,18)$\\
F.$x \in \{1\} \cup [11,18)$\\
G.$x \in \{-1\} \cup (11,18)$\\
H.$x \in \{1\} \cup [11,18]$
\testStop
\kluczStart
A
\kluczStop



\zadStart{Zadanie z Wikieł Z 1.62 c) moja wersja nr 578}

Rozwiązać nierówności $(11-x)(x+1)^{2}(19-x)^{3}\le0$.
\zadStop
\rozwStart{Patryk Wirkus}{}
Miejsca zerowe naszego wielomianu to: $11, -1, 19$.\\
Wielomian jest stopnia parzystego, ponadto znak współczynnika przy\linebreak najwyższej potędze x jest ujemny.\\ W związku z tym wykres wielomianu zaczyna się od lewej strony powyżej osi OX.\\
Ponadto w punkcie $-1$ wykres odbija się od osi poziomej.\\
A więc $$x \in \{-1\} \cup [11,19].$$
\rozwStop
\odpStart
$x \in \{-1\} \cup [11,19]$
\odpStop
\testStart
A.$x \in \{-1\} \cup [11,19]$\\
B.$x \in \{1\} \cup (11,19)$\\
C.$x \in \{-1\} \cup (11,19]$\\
D.$x \in \{1\} \cup (11,19]$\\
E.$x \in \{-1\} \cup [11,19)$\\
F.$x \in \{1\} \cup [11,19)$\\
G.$x \in \{-1\} \cup (11,19)$\\
H.$x \in \{1\} \cup [11,19]$
\testStop
\kluczStart
A
\kluczStop



\zadStart{Zadanie z Wikieł Z 1.62 c) moja wersja nr 579}

Rozwiązać nierówności $(11-x)(x+1)^{2}(20-x)^{3}\le0$.
\zadStop
\rozwStart{Patryk Wirkus}{}
Miejsca zerowe naszego wielomianu to: $11, -1, 20$.\\
Wielomian jest stopnia parzystego, ponadto znak współczynnika przy\linebreak najwyższej potędze x jest ujemny.\\ W związku z tym wykres wielomianu zaczyna się od lewej strony powyżej osi OX.\\
Ponadto w punkcie $-1$ wykres odbija się od osi poziomej.\\
A więc $$x \in \{-1\} \cup [11,20].$$
\rozwStop
\odpStart
$x \in \{-1\} \cup [11,20]$
\odpStop
\testStart
A.$x \in \{-1\} \cup [11,20]$\\
B.$x \in \{1\} \cup (11,20)$\\
C.$x \in \{-1\} \cup (11,20]$\\
D.$x \in \{1\} \cup (11,20]$\\
E.$x \in \{-1\} \cup [11,20)$\\
F.$x \in \{1\} \cup [11,20)$\\
G.$x \in \{-1\} \cup (11,20)$\\
H.$x \in \{1\} \cup [11,20]$
\testStop
\kluczStart
A
\kluczStop



\zadStart{Zadanie z Wikieł Z 1.62 c) moja wersja nr 580}

Rozwiązać nierówności $(11-x)(x+2)^{2}(12-x)^{3}\le0$.
\zadStop
\rozwStart{Patryk Wirkus}{}
Miejsca zerowe naszego wielomianu to: $11, -2, 12$.\\
Wielomian jest stopnia parzystego, ponadto znak współczynnika przy\linebreak najwyższej potędze x jest ujemny.\\ W związku z tym wykres wielomianu zaczyna się od lewej strony powyżej osi OX.\\
Ponadto w punkcie $-2$ wykres odbija się od osi poziomej.\\
A więc $$x \in \{-2\} \cup [11,12].$$
\rozwStop
\odpStart
$x \in \{-2\} \cup [11,12]$
\odpStop
\testStart
A.$x \in \{-2\} \cup [11,12]$\\
B.$x \in \{2\} \cup (11,12)$\\
C.$x \in \{-2\} \cup (11,12]$\\
D.$x \in \{2\} \cup (11,12]$\\
E.$x \in \{-2\} \cup [11,12)$\\
F.$x \in \{2\} \cup [11,12)$\\
G.$x \in \{-2\} \cup (11,12)$\\
H.$x \in \{2\} \cup [11,12]$
\testStop
\kluczStart
A
\kluczStop



\zadStart{Zadanie z Wikieł Z 1.62 c) moja wersja nr 581}

Rozwiązać nierówności $(11-x)(x+2)^{2}(13-x)^{3}\le0$.
\zadStop
\rozwStart{Patryk Wirkus}{}
Miejsca zerowe naszego wielomianu to: $11, -2, 13$.\\
Wielomian jest stopnia parzystego, ponadto znak współczynnika przy\linebreak najwyższej potędze x jest ujemny.\\ W związku z tym wykres wielomianu zaczyna się od lewej strony powyżej osi OX.\\
Ponadto w punkcie $-2$ wykres odbija się od osi poziomej.\\
A więc $$x \in \{-2\} \cup [11,13].$$
\rozwStop
\odpStart
$x \in \{-2\} \cup [11,13]$
\odpStop
\testStart
A.$x \in \{-2\} \cup [11,13]$\\
B.$x \in \{2\} \cup (11,13)$\\
C.$x \in \{-2\} \cup (11,13]$\\
D.$x \in \{2\} \cup (11,13]$\\
E.$x \in \{-2\} \cup [11,13)$\\
F.$x \in \{2\} \cup [11,13)$\\
G.$x \in \{-2\} \cup (11,13)$\\
H.$x \in \{2\} \cup [11,13]$
\testStop
\kluczStart
A
\kluczStop



\zadStart{Zadanie z Wikieł Z 1.62 c) moja wersja nr 582}

Rozwiązać nierówności $(11-x)(x+2)^{2}(14-x)^{3}\le0$.
\zadStop
\rozwStart{Patryk Wirkus}{}
Miejsca zerowe naszego wielomianu to: $11, -2, 14$.\\
Wielomian jest stopnia parzystego, ponadto znak współczynnika przy\linebreak najwyższej potędze x jest ujemny.\\ W związku z tym wykres wielomianu zaczyna się od lewej strony powyżej osi OX.\\
Ponadto w punkcie $-2$ wykres odbija się od osi poziomej.\\
A więc $$x \in \{-2\} \cup [11,14].$$
\rozwStop
\odpStart
$x \in \{-2\} \cup [11,14]$
\odpStop
\testStart
A.$x \in \{-2\} \cup [11,14]$\\
B.$x \in \{2\} \cup (11,14)$\\
C.$x \in \{-2\} \cup (11,14]$\\
D.$x \in \{2\} \cup (11,14]$\\
E.$x \in \{-2\} \cup [11,14)$\\
F.$x \in \{2\} \cup [11,14)$\\
G.$x \in \{-2\} \cup (11,14)$\\
H.$x \in \{2\} \cup [11,14]$
\testStop
\kluczStart
A
\kluczStop



\zadStart{Zadanie z Wikieł Z 1.62 c) moja wersja nr 583}

Rozwiązać nierówności $(11-x)(x+2)^{2}(15-x)^{3}\le0$.
\zadStop
\rozwStart{Patryk Wirkus}{}
Miejsca zerowe naszego wielomianu to: $11, -2, 15$.\\
Wielomian jest stopnia parzystego, ponadto znak współczynnika przy\linebreak najwyższej potędze x jest ujemny.\\ W związku z tym wykres wielomianu zaczyna się od lewej strony powyżej osi OX.\\
Ponadto w punkcie $-2$ wykres odbija się od osi poziomej.\\
A więc $$x \in \{-2\} \cup [11,15].$$
\rozwStop
\odpStart
$x \in \{-2\} \cup [11,15]$
\odpStop
\testStart
A.$x \in \{-2\} \cup [11,15]$\\
B.$x \in \{2\} \cup (11,15)$\\
C.$x \in \{-2\} \cup (11,15]$\\
D.$x \in \{2\} \cup (11,15]$\\
E.$x \in \{-2\} \cup [11,15)$\\
F.$x \in \{2\} \cup [11,15)$\\
G.$x \in \{-2\} \cup (11,15)$\\
H.$x \in \{2\} \cup [11,15]$
\testStop
\kluczStart
A
\kluczStop



\zadStart{Zadanie z Wikieł Z 1.62 c) moja wersja nr 584}

Rozwiązać nierówności $(11-x)(x+2)^{2}(16-x)^{3}\le0$.
\zadStop
\rozwStart{Patryk Wirkus}{}
Miejsca zerowe naszego wielomianu to: $11, -2, 16$.\\
Wielomian jest stopnia parzystego, ponadto znak współczynnika przy\linebreak najwyższej potędze x jest ujemny.\\ W związku z tym wykres wielomianu zaczyna się od lewej strony powyżej osi OX.\\
Ponadto w punkcie $-2$ wykres odbija się od osi poziomej.\\
A więc $$x \in \{-2\} \cup [11,16].$$
\rozwStop
\odpStart
$x \in \{-2\} \cup [11,16]$
\odpStop
\testStart
A.$x \in \{-2\} \cup [11,16]$\\
B.$x \in \{2\} \cup (11,16)$\\
C.$x \in \{-2\} \cup (11,16]$\\
D.$x \in \{2\} \cup (11,16]$\\
E.$x \in \{-2\} \cup [11,16)$\\
F.$x \in \{2\} \cup [11,16)$\\
G.$x \in \{-2\} \cup (11,16)$\\
H.$x \in \{2\} \cup [11,16]$
\testStop
\kluczStart
A
\kluczStop



\zadStart{Zadanie z Wikieł Z 1.62 c) moja wersja nr 585}

Rozwiązać nierówności $(11-x)(x+2)^{2}(17-x)^{3}\le0$.
\zadStop
\rozwStart{Patryk Wirkus}{}
Miejsca zerowe naszego wielomianu to: $11, -2, 17$.\\
Wielomian jest stopnia parzystego, ponadto znak współczynnika przy\linebreak najwyższej potędze x jest ujemny.\\ W związku z tym wykres wielomianu zaczyna się od lewej strony powyżej osi OX.\\
Ponadto w punkcie $-2$ wykres odbija się od osi poziomej.\\
A więc $$x \in \{-2\} \cup [11,17].$$
\rozwStop
\odpStart
$x \in \{-2\} \cup [11,17]$
\odpStop
\testStart
A.$x \in \{-2\} \cup [11,17]$\\
B.$x \in \{2\} \cup (11,17)$\\
C.$x \in \{-2\} \cup (11,17]$\\
D.$x \in \{2\} \cup (11,17]$\\
E.$x \in \{-2\} \cup [11,17)$\\
F.$x \in \{2\} \cup [11,17)$\\
G.$x \in \{-2\} \cup (11,17)$\\
H.$x \in \{2\} \cup [11,17]$
\testStop
\kluczStart
A
\kluczStop



\zadStart{Zadanie z Wikieł Z 1.62 c) moja wersja nr 586}

Rozwiązać nierówności $(11-x)(x+2)^{2}(18-x)^{3}\le0$.
\zadStop
\rozwStart{Patryk Wirkus}{}
Miejsca zerowe naszego wielomianu to: $11, -2, 18$.\\
Wielomian jest stopnia parzystego, ponadto znak współczynnika przy\linebreak najwyższej potędze x jest ujemny.\\ W związku z tym wykres wielomianu zaczyna się od lewej strony powyżej osi OX.\\
Ponadto w punkcie $-2$ wykres odbija się od osi poziomej.\\
A więc $$x \in \{-2\} \cup [11,18].$$
\rozwStop
\odpStart
$x \in \{-2\} \cup [11,18]$
\odpStop
\testStart
A.$x \in \{-2\} \cup [11,18]$\\
B.$x \in \{2\} \cup (11,18)$\\
C.$x \in \{-2\} \cup (11,18]$\\
D.$x \in \{2\} \cup (11,18]$\\
E.$x \in \{-2\} \cup [11,18)$\\
F.$x \in \{2\} \cup [11,18)$\\
G.$x \in \{-2\} \cup (11,18)$\\
H.$x \in \{2\} \cup [11,18]$
\testStop
\kluczStart
A
\kluczStop



\zadStart{Zadanie z Wikieł Z 1.62 c) moja wersja nr 587}

Rozwiązać nierówności $(11-x)(x+2)^{2}(19-x)^{3}\le0$.
\zadStop
\rozwStart{Patryk Wirkus}{}
Miejsca zerowe naszego wielomianu to: $11, -2, 19$.\\
Wielomian jest stopnia parzystego, ponadto znak współczynnika przy\linebreak najwyższej potędze x jest ujemny.\\ W związku z tym wykres wielomianu zaczyna się od lewej strony powyżej osi OX.\\
Ponadto w punkcie $-2$ wykres odbija się od osi poziomej.\\
A więc $$x \in \{-2\} \cup [11,19].$$
\rozwStop
\odpStart
$x \in \{-2\} \cup [11,19]$
\odpStop
\testStart
A.$x \in \{-2\} \cup [11,19]$\\
B.$x \in \{2\} \cup (11,19)$\\
C.$x \in \{-2\} \cup (11,19]$\\
D.$x \in \{2\} \cup (11,19]$\\
E.$x \in \{-2\} \cup [11,19)$\\
F.$x \in \{2\} \cup [11,19)$\\
G.$x \in \{-2\} \cup (11,19)$\\
H.$x \in \{2\} \cup [11,19]$
\testStop
\kluczStart
A
\kluczStop



\zadStart{Zadanie z Wikieł Z 1.62 c) moja wersja nr 588}

Rozwiązać nierówności $(11-x)(x+2)^{2}(20-x)^{3}\le0$.
\zadStop
\rozwStart{Patryk Wirkus}{}
Miejsca zerowe naszego wielomianu to: $11, -2, 20$.\\
Wielomian jest stopnia parzystego, ponadto znak współczynnika przy\linebreak najwyższej potędze x jest ujemny.\\ W związku z tym wykres wielomianu zaczyna się od lewej strony powyżej osi OX.\\
Ponadto w punkcie $-2$ wykres odbija się od osi poziomej.\\
A więc $$x \in \{-2\} \cup [11,20].$$
\rozwStop
\odpStart
$x \in \{-2\} \cup [11,20]$
\odpStop
\testStart
A.$x \in \{-2\} \cup [11,20]$\\
B.$x \in \{2\} \cup (11,20)$\\
C.$x \in \{-2\} \cup (11,20]$\\
D.$x \in \{2\} \cup (11,20]$\\
E.$x \in \{-2\} \cup [11,20)$\\
F.$x \in \{2\} \cup [11,20)$\\
G.$x \in \{-2\} \cup (11,20)$\\
H.$x \in \{2\} \cup [11,20]$
\testStop
\kluczStart
A
\kluczStop



\zadStart{Zadanie z Wikieł Z 1.62 c) moja wersja nr 589}

Rozwiązać nierówności $(11-x)(x+3)^{2}(12-x)^{3}\le0$.
\zadStop
\rozwStart{Patryk Wirkus}{}
Miejsca zerowe naszego wielomianu to: $11, -3, 12$.\\
Wielomian jest stopnia parzystego, ponadto znak współczynnika przy\linebreak najwyższej potędze x jest ujemny.\\ W związku z tym wykres wielomianu zaczyna się od lewej strony powyżej osi OX.\\
Ponadto w punkcie $-3$ wykres odbija się od osi poziomej.\\
A więc $$x \in \{-3\} \cup [11,12].$$
\rozwStop
\odpStart
$x \in \{-3\} \cup [11,12]$
\odpStop
\testStart
A.$x \in \{-3\} \cup [11,12]$\\
B.$x \in \{3\} \cup (11,12)$\\
C.$x \in \{-3\} \cup (11,12]$\\
D.$x \in \{3\} \cup (11,12]$\\
E.$x \in \{-3\} \cup [11,12)$\\
F.$x \in \{3\} \cup [11,12)$\\
G.$x \in \{-3\} \cup (11,12)$\\
H.$x \in \{3\} \cup [11,12]$
\testStop
\kluczStart
A
\kluczStop



\zadStart{Zadanie z Wikieł Z 1.62 c) moja wersja nr 590}

Rozwiązać nierówności $(11-x)(x+3)^{2}(13-x)^{3}\le0$.
\zadStop
\rozwStart{Patryk Wirkus}{}
Miejsca zerowe naszego wielomianu to: $11, -3, 13$.\\
Wielomian jest stopnia parzystego, ponadto znak współczynnika przy\linebreak najwyższej potędze x jest ujemny.\\ W związku z tym wykres wielomianu zaczyna się od lewej strony powyżej osi OX.\\
Ponadto w punkcie $-3$ wykres odbija się od osi poziomej.\\
A więc $$x \in \{-3\} \cup [11,13].$$
\rozwStop
\odpStart
$x \in \{-3\} \cup [11,13]$
\odpStop
\testStart
A.$x \in \{-3\} \cup [11,13]$\\
B.$x \in \{3\} \cup (11,13)$\\
C.$x \in \{-3\} \cup (11,13]$\\
D.$x \in \{3\} \cup (11,13]$\\
E.$x \in \{-3\} \cup [11,13)$\\
F.$x \in \{3\} \cup [11,13)$\\
G.$x \in \{-3\} \cup (11,13)$\\
H.$x \in \{3\} \cup [11,13]$
\testStop
\kluczStart
A
\kluczStop



\zadStart{Zadanie z Wikieł Z 1.62 c) moja wersja nr 591}

Rozwiązać nierówności $(11-x)(x+3)^{2}(14-x)^{3}\le0$.
\zadStop
\rozwStart{Patryk Wirkus}{}
Miejsca zerowe naszego wielomianu to: $11, -3, 14$.\\
Wielomian jest stopnia parzystego, ponadto znak współczynnika przy\linebreak najwyższej potędze x jest ujemny.\\ W związku z tym wykres wielomianu zaczyna się od lewej strony powyżej osi OX.\\
Ponadto w punkcie $-3$ wykres odbija się od osi poziomej.\\
A więc $$x \in \{-3\} \cup [11,14].$$
\rozwStop
\odpStart
$x \in \{-3\} \cup [11,14]$
\odpStop
\testStart
A.$x \in \{-3\} \cup [11,14]$\\
B.$x \in \{3\} \cup (11,14)$\\
C.$x \in \{-3\} \cup (11,14]$\\
D.$x \in \{3\} \cup (11,14]$\\
E.$x \in \{-3\} \cup [11,14)$\\
F.$x \in \{3\} \cup [11,14)$\\
G.$x \in \{-3\} \cup (11,14)$\\
H.$x \in \{3\} \cup [11,14]$
\testStop
\kluczStart
A
\kluczStop



\zadStart{Zadanie z Wikieł Z 1.62 c) moja wersja nr 592}

Rozwiązać nierówności $(11-x)(x+3)^{2}(15-x)^{3}\le0$.
\zadStop
\rozwStart{Patryk Wirkus}{}
Miejsca zerowe naszego wielomianu to: $11, -3, 15$.\\
Wielomian jest stopnia parzystego, ponadto znak współczynnika przy\linebreak najwyższej potędze x jest ujemny.\\ W związku z tym wykres wielomianu zaczyna się od lewej strony powyżej osi OX.\\
Ponadto w punkcie $-3$ wykres odbija się od osi poziomej.\\
A więc $$x \in \{-3\} \cup [11,15].$$
\rozwStop
\odpStart
$x \in \{-3\} \cup [11,15]$
\odpStop
\testStart
A.$x \in \{-3\} \cup [11,15]$\\
B.$x \in \{3\} \cup (11,15)$\\
C.$x \in \{-3\} \cup (11,15]$\\
D.$x \in \{3\} \cup (11,15]$\\
E.$x \in \{-3\} \cup [11,15)$\\
F.$x \in \{3\} \cup [11,15)$\\
G.$x \in \{-3\} \cup (11,15)$\\
H.$x \in \{3\} \cup [11,15]$
\testStop
\kluczStart
A
\kluczStop



\zadStart{Zadanie z Wikieł Z 1.62 c) moja wersja nr 593}

Rozwiązać nierówności $(11-x)(x+3)^{2}(16-x)^{3}\le0$.
\zadStop
\rozwStart{Patryk Wirkus}{}
Miejsca zerowe naszego wielomianu to: $11, -3, 16$.\\
Wielomian jest stopnia parzystego, ponadto znak współczynnika przy\linebreak najwyższej potędze x jest ujemny.\\ W związku z tym wykres wielomianu zaczyna się od lewej strony powyżej osi OX.\\
Ponadto w punkcie $-3$ wykres odbija się od osi poziomej.\\
A więc $$x \in \{-3\} \cup [11,16].$$
\rozwStop
\odpStart
$x \in \{-3\} \cup [11,16]$
\odpStop
\testStart
A.$x \in \{-3\} \cup [11,16]$\\
B.$x \in \{3\} \cup (11,16)$\\
C.$x \in \{-3\} \cup (11,16]$\\
D.$x \in \{3\} \cup (11,16]$\\
E.$x \in \{-3\} \cup [11,16)$\\
F.$x \in \{3\} \cup [11,16)$\\
G.$x \in \{-3\} \cup (11,16)$\\
H.$x \in \{3\} \cup [11,16]$
\testStop
\kluczStart
A
\kluczStop



\zadStart{Zadanie z Wikieł Z 1.62 c) moja wersja nr 594}

Rozwiązać nierówności $(11-x)(x+3)^{2}(17-x)^{3}\le0$.
\zadStop
\rozwStart{Patryk Wirkus}{}
Miejsca zerowe naszego wielomianu to: $11, -3, 17$.\\
Wielomian jest stopnia parzystego, ponadto znak współczynnika przy\linebreak najwyższej potędze x jest ujemny.\\ W związku z tym wykres wielomianu zaczyna się od lewej strony powyżej osi OX.\\
Ponadto w punkcie $-3$ wykres odbija się od osi poziomej.\\
A więc $$x \in \{-3\} \cup [11,17].$$
\rozwStop
\odpStart
$x \in \{-3\} \cup [11,17]$
\odpStop
\testStart
A.$x \in \{-3\} \cup [11,17]$\\
B.$x \in \{3\} \cup (11,17)$\\
C.$x \in \{-3\} \cup (11,17]$\\
D.$x \in \{3\} \cup (11,17]$\\
E.$x \in \{-3\} \cup [11,17)$\\
F.$x \in \{3\} \cup [11,17)$\\
G.$x \in \{-3\} \cup (11,17)$\\
H.$x \in \{3\} \cup [11,17]$
\testStop
\kluczStart
A
\kluczStop



\zadStart{Zadanie z Wikieł Z 1.62 c) moja wersja nr 595}

Rozwiązać nierówności $(11-x)(x+3)^{2}(18-x)^{3}\le0$.
\zadStop
\rozwStart{Patryk Wirkus}{}
Miejsca zerowe naszego wielomianu to: $11, -3, 18$.\\
Wielomian jest stopnia parzystego, ponadto znak współczynnika przy\linebreak najwyższej potędze x jest ujemny.\\ W związku z tym wykres wielomianu zaczyna się od lewej strony powyżej osi OX.\\
Ponadto w punkcie $-3$ wykres odbija się od osi poziomej.\\
A więc $$x \in \{-3\} \cup [11,18].$$
\rozwStop
\odpStart
$x \in \{-3\} \cup [11,18]$
\odpStop
\testStart
A.$x \in \{-3\} \cup [11,18]$\\
B.$x \in \{3\} \cup (11,18)$\\
C.$x \in \{-3\} \cup (11,18]$\\
D.$x \in \{3\} \cup (11,18]$\\
E.$x \in \{-3\} \cup [11,18)$\\
F.$x \in \{3\} \cup [11,18)$\\
G.$x \in \{-3\} \cup (11,18)$\\
H.$x \in \{3\} \cup [11,18]$
\testStop
\kluczStart
A
\kluczStop



\zadStart{Zadanie z Wikieł Z 1.62 c) moja wersja nr 596}

Rozwiązać nierówności $(11-x)(x+3)^{2}(19-x)^{3}\le0$.
\zadStop
\rozwStart{Patryk Wirkus}{}
Miejsca zerowe naszego wielomianu to: $11, -3, 19$.\\
Wielomian jest stopnia parzystego, ponadto znak współczynnika przy\linebreak najwyższej potędze x jest ujemny.\\ W związku z tym wykres wielomianu zaczyna się od lewej strony powyżej osi OX.\\
Ponadto w punkcie $-3$ wykres odbija się od osi poziomej.\\
A więc $$x \in \{-3\} \cup [11,19].$$
\rozwStop
\odpStart
$x \in \{-3\} \cup [11,19]$
\odpStop
\testStart
A.$x \in \{-3\} \cup [11,19]$\\
B.$x \in \{3\} \cup (11,19)$\\
C.$x \in \{-3\} \cup (11,19]$\\
D.$x \in \{3\} \cup (11,19]$\\
E.$x \in \{-3\} \cup [11,19)$\\
F.$x \in \{3\} \cup [11,19)$\\
G.$x \in \{-3\} \cup (11,19)$\\
H.$x \in \{3\} \cup [11,19]$
\testStop
\kluczStart
A
\kluczStop



\zadStart{Zadanie z Wikieł Z 1.62 c) moja wersja nr 597}

Rozwiązać nierówności $(11-x)(x+3)^{2}(20-x)^{3}\le0$.
\zadStop
\rozwStart{Patryk Wirkus}{}
Miejsca zerowe naszego wielomianu to: $11, -3, 20$.\\
Wielomian jest stopnia parzystego, ponadto znak współczynnika przy\linebreak najwyższej potędze x jest ujemny.\\ W związku z tym wykres wielomianu zaczyna się od lewej strony powyżej osi OX.\\
Ponadto w punkcie $-3$ wykres odbija się od osi poziomej.\\
A więc $$x \in \{-3\} \cup [11,20].$$
\rozwStop
\odpStart
$x \in \{-3\} \cup [11,20]$
\odpStop
\testStart
A.$x \in \{-3\} \cup [11,20]$\\
B.$x \in \{3\} \cup (11,20)$\\
C.$x \in \{-3\} \cup (11,20]$\\
D.$x \in \{3\} \cup (11,20]$\\
E.$x \in \{-3\} \cup [11,20)$\\
F.$x \in \{3\} \cup [11,20)$\\
G.$x \in \{-3\} \cup (11,20)$\\
H.$x \in \{3\} \cup [11,20]$
\testStop
\kluczStart
A
\kluczStop



\zadStart{Zadanie z Wikieł Z 1.62 c) moja wersja nr 598}

Rozwiązać nierówności $(11-x)(x+4)^{2}(12-x)^{3}\le0$.
\zadStop
\rozwStart{Patryk Wirkus}{}
Miejsca zerowe naszego wielomianu to: $11, -4, 12$.\\
Wielomian jest stopnia parzystego, ponadto znak współczynnika przy\linebreak najwyższej potędze x jest ujemny.\\ W związku z tym wykres wielomianu zaczyna się od lewej strony powyżej osi OX.\\
Ponadto w punkcie $-4$ wykres odbija się od osi poziomej.\\
A więc $$x \in \{-4\} \cup [11,12].$$
\rozwStop
\odpStart
$x \in \{-4\} \cup [11,12]$
\odpStop
\testStart
A.$x \in \{-4\} \cup [11,12]$\\
B.$x \in \{4\} \cup (11,12)$\\
C.$x \in \{-4\} \cup (11,12]$\\
D.$x \in \{4\} \cup (11,12]$\\
E.$x \in \{-4\} \cup [11,12)$\\
F.$x \in \{4\} \cup [11,12)$\\
G.$x \in \{-4\} \cup (11,12)$\\
H.$x \in \{4\} \cup [11,12]$
\testStop
\kluczStart
A
\kluczStop



\zadStart{Zadanie z Wikieł Z 1.62 c) moja wersja nr 599}

Rozwiązać nierówności $(11-x)(x+4)^{2}(13-x)^{3}\le0$.
\zadStop
\rozwStart{Patryk Wirkus}{}
Miejsca zerowe naszego wielomianu to: $11, -4, 13$.\\
Wielomian jest stopnia parzystego, ponadto znak współczynnika przy\linebreak najwyższej potędze x jest ujemny.\\ W związku z tym wykres wielomianu zaczyna się od lewej strony powyżej osi OX.\\
Ponadto w punkcie $-4$ wykres odbija się od osi poziomej.\\
A więc $$x \in \{-4\} \cup [11,13].$$
\rozwStop
\odpStart
$x \in \{-4\} \cup [11,13]$
\odpStop
\testStart
A.$x \in \{-4\} \cup [11,13]$\\
B.$x \in \{4\} \cup (11,13)$\\
C.$x \in \{-4\} \cup (11,13]$\\
D.$x \in \{4\} \cup (11,13]$\\
E.$x \in \{-4\} \cup [11,13)$\\
F.$x \in \{4\} \cup [11,13)$\\
G.$x \in \{-4\} \cup (11,13)$\\
H.$x \in \{4\} \cup [11,13]$
\testStop
\kluczStart
A
\kluczStop



\zadStart{Zadanie z Wikieł Z 1.62 c) moja wersja nr 600}

Rozwiązać nierówności $(11-x)(x+4)^{2}(14-x)^{3}\le0$.
\zadStop
\rozwStart{Patryk Wirkus}{}
Miejsca zerowe naszego wielomianu to: $11, -4, 14$.\\
Wielomian jest stopnia parzystego, ponadto znak współczynnika przy\linebreak najwyższej potędze x jest ujemny.\\ W związku z tym wykres wielomianu zaczyna się od lewej strony powyżej osi OX.\\
Ponadto w punkcie $-4$ wykres odbija się od osi poziomej.\\
A więc $$x \in \{-4\} \cup [11,14].$$
\rozwStop
\odpStart
$x \in \{-4\} \cup [11,14]$
\odpStop
\testStart
A.$x \in \{-4\} \cup [11,14]$\\
B.$x \in \{4\} \cup (11,14)$\\
C.$x \in \{-4\} \cup (11,14]$\\
D.$x \in \{4\} \cup (11,14]$\\
E.$x \in \{-4\} \cup [11,14)$\\
F.$x \in \{4\} \cup [11,14)$\\
G.$x \in \{-4\} \cup (11,14)$\\
H.$x \in \{4\} \cup [11,14]$
\testStop
\kluczStart
A
\kluczStop



\zadStart{Zadanie z Wikieł Z 1.62 c) moja wersja nr 601}

Rozwiązać nierówności $(11-x)(x+4)^{2}(15-x)^{3}\le0$.
\zadStop
\rozwStart{Patryk Wirkus}{}
Miejsca zerowe naszego wielomianu to: $11, -4, 15$.\\
Wielomian jest stopnia parzystego, ponadto znak współczynnika przy\linebreak najwyższej potędze x jest ujemny.\\ W związku z tym wykres wielomianu zaczyna się od lewej strony powyżej osi OX.\\
Ponadto w punkcie $-4$ wykres odbija się od osi poziomej.\\
A więc $$x \in \{-4\} \cup [11,15].$$
\rozwStop
\odpStart
$x \in \{-4\} \cup [11,15]$
\odpStop
\testStart
A.$x \in \{-4\} \cup [11,15]$\\
B.$x \in \{4\} \cup (11,15)$\\
C.$x \in \{-4\} \cup (11,15]$\\
D.$x \in \{4\} \cup (11,15]$\\
E.$x \in \{-4\} \cup [11,15)$\\
F.$x \in \{4\} \cup [11,15)$\\
G.$x \in \{-4\} \cup (11,15)$\\
H.$x \in \{4\} \cup [11,15]$
\testStop
\kluczStart
A
\kluczStop



\zadStart{Zadanie z Wikieł Z 1.62 c) moja wersja nr 602}

Rozwiązać nierówności $(11-x)(x+4)^{2}(16-x)^{3}\le0$.
\zadStop
\rozwStart{Patryk Wirkus}{}
Miejsca zerowe naszego wielomianu to: $11, -4, 16$.\\
Wielomian jest stopnia parzystego, ponadto znak współczynnika przy\linebreak najwyższej potędze x jest ujemny.\\ W związku z tym wykres wielomianu zaczyna się od lewej strony powyżej osi OX.\\
Ponadto w punkcie $-4$ wykres odbija się od osi poziomej.\\
A więc $$x \in \{-4\} \cup [11,16].$$
\rozwStop
\odpStart
$x \in \{-4\} \cup [11,16]$
\odpStop
\testStart
A.$x \in \{-4\} \cup [11,16]$\\
B.$x \in \{4\} \cup (11,16)$\\
C.$x \in \{-4\} \cup (11,16]$\\
D.$x \in \{4\} \cup (11,16]$\\
E.$x \in \{-4\} \cup [11,16)$\\
F.$x \in \{4\} \cup [11,16)$\\
G.$x \in \{-4\} \cup (11,16)$\\
H.$x \in \{4\} \cup [11,16]$
\testStop
\kluczStart
A
\kluczStop



\zadStart{Zadanie z Wikieł Z 1.62 c) moja wersja nr 603}

Rozwiązać nierówności $(11-x)(x+4)^{2}(17-x)^{3}\le0$.
\zadStop
\rozwStart{Patryk Wirkus}{}
Miejsca zerowe naszego wielomianu to: $11, -4, 17$.\\
Wielomian jest stopnia parzystego, ponadto znak współczynnika przy\linebreak najwyższej potędze x jest ujemny.\\ W związku z tym wykres wielomianu zaczyna się od lewej strony powyżej osi OX.\\
Ponadto w punkcie $-4$ wykres odbija się od osi poziomej.\\
A więc $$x \in \{-4\} \cup [11,17].$$
\rozwStop
\odpStart
$x \in \{-4\} \cup [11,17]$
\odpStop
\testStart
A.$x \in \{-4\} \cup [11,17]$\\
B.$x \in \{4\} \cup (11,17)$\\
C.$x \in \{-4\} \cup (11,17]$\\
D.$x \in \{4\} \cup (11,17]$\\
E.$x \in \{-4\} \cup [11,17)$\\
F.$x \in \{4\} \cup [11,17)$\\
G.$x \in \{-4\} \cup (11,17)$\\
H.$x \in \{4\} \cup [11,17]$
\testStop
\kluczStart
A
\kluczStop



\zadStart{Zadanie z Wikieł Z 1.62 c) moja wersja nr 604}

Rozwiązać nierówności $(11-x)(x+4)^{2}(18-x)^{3}\le0$.
\zadStop
\rozwStart{Patryk Wirkus}{}
Miejsca zerowe naszego wielomianu to: $11, -4, 18$.\\
Wielomian jest stopnia parzystego, ponadto znak współczynnika przy\linebreak najwyższej potędze x jest ujemny.\\ W związku z tym wykres wielomianu zaczyna się od lewej strony powyżej osi OX.\\
Ponadto w punkcie $-4$ wykres odbija się od osi poziomej.\\
A więc $$x \in \{-4\} \cup [11,18].$$
\rozwStop
\odpStart
$x \in \{-4\} \cup [11,18]$
\odpStop
\testStart
A.$x \in \{-4\} \cup [11,18]$\\
B.$x \in \{4\} \cup (11,18)$\\
C.$x \in \{-4\} \cup (11,18]$\\
D.$x \in \{4\} \cup (11,18]$\\
E.$x \in \{-4\} \cup [11,18)$\\
F.$x \in \{4\} \cup [11,18)$\\
G.$x \in \{-4\} \cup (11,18)$\\
H.$x \in \{4\} \cup [11,18]$
\testStop
\kluczStart
A
\kluczStop



\zadStart{Zadanie z Wikieł Z 1.62 c) moja wersja nr 605}

Rozwiązać nierówności $(11-x)(x+4)^{2}(19-x)^{3}\le0$.
\zadStop
\rozwStart{Patryk Wirkus}{}
Miejsca zerowe naszego wielomianu to: $11, -4, 19$.\\
Wielomian jest stopnia parzystego, ponadto znak współczynnika przy\linebreak najwyższej potędze x jest ujemny.\\ W związku z tym wykres wielomianu zaczyna się od lewej strony powyżej osi OX.\\
Ponadto w punkcie $-4$ wykres odbija się od osi poziomej.\\
A więc $$x \in \{-4\} \cup [11,19].$$
\rozwStop
\odpStart
$x \in \{-4\} \cup [11,19]$
\odpStop
\testStart
A.$x \in \{-4\} \cup [11,19]$\\
B.$x \in \{4\} \cup (11,19)$\\
C.$x \in \{-4\} \cup (11,19]$\\
D.$x \in \{4\} \cup (11,19]$\\
E.$x \in \{-4\} \cup [11,19)$\\
F.$x \in \{4\} \cup [11,19)$\\
G.$x \in \{-4\} \cup (11,19)$\\
H.$x \in \{4\} \cup [11,19]$
\testStop
\kluczStart
A
\kluczStop



\zadStart{Zadanie z Wikieł Z 1.62 c) moja wersja nr 606}

Rozwiązać nierówności $(11-x)(x+4)^{2}(20-x)^{3}\le0$.
\zadStop
\rozwStart{Patryk Wirkus}{}
Miejsca zerowe naszego wielomianu to: $11, -4, 20$.\\
Wielomian jest stopnia parzystego, ponadto znak współczynnika przy\linebreak najwyższej potędze x jest ujemny.\\ W związku z tym wykres wielomianu zaczyna się od lewej strony powyżej osi OX.\\
Ponadto w punkcie $-4$ wykres odbija się od osi poziomej.\\
A więc $$x \in \{-4\} \cup [11,20].$$
\rozwStop
\odpStart
$x \in \{-4\} \cup [11,20]$
\odpStop
\testStart
A.$x \in \{-4\} \cup [11,20]$\\
B.$x \in \{4\} \cup (11,20)$\\
C.$x \in \{-4\} \cup (11,20]$\\
D.$x \in \{4\} \cup (11,20]$\\
E.$x \in \{-4\} \cup [11,20)$\\
F.$x \in \{4\} \cup [11,20)$\\
G.$x \in \{-4\} \cup (11,20)$\\
H.$x \in \{4\} \cup [11,20]$
\testStop
\kluczStart
A
\kluczStop



\zadStart{Zadanie z Wikieł Z 1.62 c) moja wersja nr 607}

Rozwiązać nierówności $(11-x)(x+5)^{2}(12-x)^{3}\le0$.
\zadStop
\rozwStart{Patryk Wirkus}{}
Miejsca zerowe naszego wielomianu to: $11, -5, 12$.\\
Wielomian jest stopnia parzystego, ponadto znak współczynnika przy\linebreak najwyższej potędze x jest ujemny.\\ W związku z tym wykres wielomianu zaczyna się od lewej strony powyżej osi OX.\\
Ponadto w punkcie $-5$ wykres odbija się od osi poziomej.\\
A więc $$x \in \{-5\} \cup [11,12].$$
\rozwStop
\odpStart
$x \in \{-5\} \cup [11,12]$
\odpStop
\testStart
A.$x \in \{-5\} \cup [11,12]$\\
B.$x \in \{5\} \cup (11,12)$\\
C.$x \in \{-5\} \cup (11,12]$\\
D.$x \in \{5\} \cup (11,12]$\\
E.$x \in \{-5\} \cup [11,12)$\\
F.$x \in \{5\} \cup [11,12)$\\
G.$x \in \{-5\} \cup (11,12)$\\
H.$x \in \{5\} \cup [11,12]$
\testStop
\kluczStart
A
\kluczStop



\zadStart{Zadanie z Wikieł Z 1.62 c) moja wersja nr 608}

Rozwiązać nierówności $(11-x)(x+5)^{2}(13-x)^{3}\le0$.
\zadStop
\rozwStart{Patryk Wirkus}{}
Miejsca zerowe naszego wielomianu to: $11, -5, 13$.\\
Wielomian jest stopnia parzystego, ponadto znak współczynnika przy\linebreak najwyższej potędze x jest ujemny.\\ W związku z tym wykres wielomianu zaczyna się od lewej strony powyżej osi OX.\\
Ponadto w punkcie $-5$ wykres odbija się od osi poziomej.\\
A więc $$x \in \{-5\} \cup [11,13].$$
\rozwStop
\odpStart
$x \in \{-5\} \cup [11,13]$
\odpStop
\testStart
A.$x \in \{-5\} \cup [11,13]$\\
B.$x \in \{5\} \cup (11,13)$\\
C.$x \in \{-5\} \cup (11,13]$\\
D.$x \in \{5\} \cup (11,13]$\\
E.$x \in \{-5\} \cup [11,13)$\\
F.$x \in \{5\} \cup [11,13)$\\
G.$x \in \{-5\} \cup (11,13)$\\
H.$x \in \{5\} \cup [11,13]$
\testStop
\kluczStart
A
\kluczStop



\zadStart{Zadanie z Wikieł Z 1.62 c) moja wersja nr 609}

Rozwiązać nierówności $(11-x)(x+5)^{2}(14-x)^{3}\le0$.
\zadStop
\rozwStart{Patryk Wirkus}{}
Miejsca zerowe naszego wielomianu to: $11, -5, 14$.\\
Wielomian jest stopnia parzystego, ponadto znak współczynnika przy\linebreak najwyższej potędze x jest ujemny.\\ W związku z tym wykres wielomianu zaczyna się od lewej strony powyżej osi OX.\\
Ponadto w punkcie $-5$ wykres odbija się od osi poziomej.\\
A więc $$x \in \{-5\} \cup [11,14].$$
\rozwStop
\odpStart
$x \in \{-5\} \cup [11,14]$
\odpStop
\testStart
A.$x \in \{-5\} \cup [11,14]$\\
B.$x \in \{5\} \cup (11,14)$\\
C.$x \in \{-5\} \cup (11,14]$\\
D.$x \in \{5\} \cup (11,14]$\\
E.$x \in \{-5\} \cup [11,14)$\\
F.$x \in \{5\} \cup [11,14)$\\
G.$x \in \{-5\} \cup (11,14)$\\
H.$x \in \{5\} \cup [11,14]$
\testStop
\kluczStart
A
\kluczStop



\zadStart{Zadanie z Wikieł Z 1.62 c) moja wersja nr 610}

Rozwiązać nierówności $(11-x)(x+5)^{2}(15-x)^{3}\le0$.
\zadStop
\rozwStart{Patryk Wirkus}{}
Miejsca zerowe naszego wielomianu to: $11, -5, 15$.\\
Wielomian jest stopnia parzystego, ponadto znak współczynnika przy\linebreak najwyższej potędze x jest ujemny.\\ W związku z tym wykres wielomianu zaczyna się od lewej strony powyżej osi OX.\\
Ponadto w punkcie $-5$ wykres odbija się od osi poziomej.\\
A więc $$x \in \{-5\} \cup [11,15].$$
\rozwStop
\odpStart
$x \in \{-5\} \cup [11,15]$
\odpStop
\testStart
A.$x \in \{-5\} \cup [11,15]$\\
B.$x \in \{5\} \cup (11,15)$\\
C.$x \in \{-5\} \cup (11,15]$\\
D.$x \in \{5\} \cup (11,15]$\\
E.$x \in \{-5\} \cup [11,15)$\\
F.$x \in \{5\} \cup [11,15)$\\
G.$x \in \{-5\} \cup (11,15)$\\
H.$x \in \{5\} \cup [11,15]$
\testStop
\kluczStart
A
\kluczStop



\zadStart{Zadanie z Wikieł Z 1.62 c) moja wersja nr 611}

Rozwiązać nierówności $(11-x)(x+5)^{2}(16-x)^{3}\le0$.
\zadStop
\rozwStart{Patryk Wirkus}{}
Miejsca zerowe naszego wielomianu to: $11, -5, 16$.\\
Wielomian jest stopnia parzystego, ponadto znak współczynnika przy\linebreak najwyższej potędze x jest ujemny.\\ W związku z tym wykres wielomianu zaczyna się od lewej strony powyżej osi OX.\\
Ponadto w punkcie $-5$ wykres odbija się od osi poziomej.\\
A więc $$x \in \{-5\} \cup [11,16].$$
\rozwStop
\odpStart
$x \in \{-5\} \cup [11,16]$
\odpStop
\testStart
A.$x \in \{-5\} \cup [11,16]$\\
B.$x \in \{5\} \cup (11,16)$\\
C.$x \in \{-5\} \cup (11,16]$\\
D.$x \in \{5\} \cup (11,16]$\\
E.$x \in \{-5\} \cup [11,16)$\\
F.$x \in \{5\} \cup [11,16)$\\
G.$x \in \{-5\} \cup (11,16)$\\
H.$x \in \{5\} \cup [11,16]$
\testStop
\kluczStart
A
\kluczStop



\zadStart{Zadanie z Wikieł Z 1.62 c) moja wersja nr 612}

Rozwiązać nierówności $(11-x)(x+5)^{2}(17-x)^{3}\le0$.
\zadStop
\rozwStart{Patryk Wirkus}{}
Miejsca zerowe naszego wielomianu to: $11, -5, 17$.\\
Wielomian jest stopnia parzystego, ponadto znak współczynnika przy\linebreak najwyższej potędze x jest ujemny.\\ W związku z tym wykres wielomianu zaczyna się od lewej strony powyżej osi OX.\\
Ponadto w punkcie $-5$ wykres odbija się od osi poziomej.\\
A więc $$x \in \{-5\} \cup [11,17].$$
\rozwStop
\odpStart
$x \in \{-5\} \cup [11,17]$
\odpStop
\testStart
A.$x \in \{-5\} \cup [11,17]$\\
B.$x \in \{5\} \cup (11,17)$\\
C.$x \in \{-5\} \cup (11,17]$\\
D.$x \in \{5\} \cup (11,17]$\\
E.$x \in \{-5\} \cup [11,17)$\\
F.$x \in \{5\} \cup [11,17)$\\
G.$x \in \{-5\} \cup (11,17)$\\
H.$x \in \{5\} \cup [11,17]$
\testStop
\kluczStart
A
\kluczStop



\zadStart{Zadanie z Wikieł Z 1.62 c) moja wersja nr 613}

Rozwiązać nierówności $(11-x)(x+5)^{2}(18-x)^{3}\le0$.
\zadStop
\rozwStart{Patryk Wirkus}{}
Miejsca zerowe naszego wielomianu to: $11, -5, 18$.\\
Wielomian jest stopnia parzystego, ponadto znak współczynnika przy\linebreak najwyższej potędze x jest ujemny.\\ W związku z tym wykres wielomianu zaczyna się od lewej strony powyżej osi OX.\\
Ponadto w punkcie $-5$ wykres odbija się od osi poziomej.\\
A więc $$x \in \{-5\} \cup [11,18].$$
\rozwStop
\odpStart
$x \in \{-5\} \cup [11,18]$
\odpStop
\testStart
A.$x \in \{-5\} \cup [11,18]$\\
B.$x \in \{5\} \cup (11,18)$\\
C.$x \in \{-5\} \cup (11,18]$\\
D.$x \in \{5\} \cup (11,18]$\\
E.$x \in \{-5\} \cup [11,18)$\\
F.$x \in \{5\} \cup [11,18)$\\
G.$x \in \{-5\} \cup (11,18)$\\
H.$x \in \{5\} \cup [11,18]$
\testStop
\kluczStart
A
\kluczStop



\zadStart{Zadanie z Wikieł Z 1.62 c) moja wersja nr 614}

Rozwiązać nierówności $(11-x)(x+5)^{2}(19-x)^{3}\le0$.
\zadStop
\rozwStart{Patryk Wirkus}{}
Miejsca zerowe naszego wielomianu to: $11, -5, 19$.\\
Wielomian jest stopnia parzystego, ponadto znak współczynnika przy\linebreak najwyższej potędze x jest ujemny.\\ W związku z tym wykres wielomianu zaczyna się od lewej strony powyżej osi OX.\\
Ponadto w punkcie $-5$ wykres odbija się od osi poziomej.\\
A więc $$x \in \{-5\} \cup [11,19].$$
\rozwStop
\odpStart
$x \in \{-5\} \cup [11,19]$
\odpStop
\testStart
A.$x \in \{-5\} \cup [11,19]$\\
B.$x \in \{5\} \cup (11,19)$\\
C.$x \in \{-5\} \cup (11,19]$\\
D.$x \in \{5\} \cup (11,19]$\\
E.$x \in \{-5\} \cup [11,19)$\\
F.$x \in \{5\} \cup [11,19)$\\
G.$x \in \{-5\} \cup (11,19)$\\
H.$x \in \{5\} \cup [11,19]$
\testStop
\kluczStart
A
\kluczStop



\zadStart{Zadanie z Wikieł Z 1.62 c) moja wersja nr 615}

Rozwiązać nierówności $(11-x)(x+5)^{2}(20-x)^{3}\le0$.
\zadStop
\rozwStart{Patryk Wirkus}{}
Miejsca zerowe naszego wielomianu to: $11, -5, 20$.\\
Wielomian jest stopnia parzystego, ponadto znak współczynnika przy\linebreak najwyższej potędze x jest ujemny.\\ W związku z tym wykres wielomianu zaczyna się od lewej strony powyżej osi OX.\\
Ponadto w punkcie $-5$ wykres odbija się od osi poziomej.\\
A więc $$x \in \{-5\} \cup [11,20].$$
\rozwStop
\odpStart
$x \in \{-5\} \cup [11,20]$
\odpStop
\testStart
A.$x \in \{-5\} \cup [11,20]$\\
B.$x \in \{5\} \cup (11,20)$\\
C.$x \in \{-5\} \cup (11,20]$\\
D.$x \in \{5\} \cup (11,20]$\\
E.$x \in \{-5\} \cup [11,20)$\\
F.$x \in \{5\} \cup [11,20)$\\
G.$x \in \{-5\} \cup (11,20)$\\
H.$x \in \{5\} \cup [11,20]$
\testStop
\kluczStart
A
\kluczStop



\zadStart{Zadanie z Wikieł Z 1.62 c) moja wersja nr 616}

Rozwiązać nierówności $(11-x)(x+6)^{2}(12-x)^{3}\le0$.
\zadStop
\rozwStart{Patryk Wirkus}{}
Miejsca zerowe naszego wielomianu to: $11, -6, 12$.\\
Wielomian jest stopnia parzystego, ponadto znak współczynnika przy\linebreak najwyższej potędze x jest ujemny.\\ W związku z tym wykres wielomianu zaczyna się od lewej strony powyżej osi OX.\\
Ponadto w punkcie $-6$ wykres odbija się od osi poziomej.\\
A więc $$x \in \{-6\} \cup [11,12].$$
\rozwStop
\odpStart
$x \in \{-6\} \cup [11,12]$
\odpStop
\testStart
A.$x \in \{-6\} \cup [11,12]$\\
B.$x \in \{6\} \cup (11,12)$\\
C.$x \in \{-6\} \cup (11,12]$\\
D.$x \in \{6\} \cup (11,12]$\\
E.$x \in \{-6\} \cup [11,12)$\\
F.$x \in \{6\} \cup [11,12)$\\
G.$x \in \{-6\} \cup (11,12)$\\
H.$x \in \{6\} \cup [11,12]$
\testStop
\kluczStart
A
\kluczStop



\zadStart{Zadanie z Wikieł Z 1.62 c) moja wersja nr 617}

Rozwiązać nierówności $(11-x)(x+6)^{2}(13-x)^{3}\le0$.
\zadStop
\rozwStart{Patryk Wirkus}{}
Miejsca zerowe naszego wielomianu to: $11, -6, 13$.\\
Wielomian jest stopnia parzystego, ponadto znak współczynnika przy\linebreak najwyższej potędze x jest ujemny.\\ W związku z tym wykres wielomianu zaczyna się od lewej strony powyżej osi OX.\\
Ponadto w punkcie $-6$ wykres odbija się od osi poziomej.\\
A więc $$x \in \{-6\} \cup [11,13].$$
\rozwStop
\odpStart
$x \in \{-6\} \cup [11,13]$
\odpStop
\testStart
A.$x \in \{-6\} \cup [11,13]$\\
B.$x \in \{6\} \cup (11,13)$\\
C.$x \in \{-6\} \cup (11,13]$\\
D.$x \in \{6\} \cup (11,13]$\\
E.$x \in \{-6\} \cup [11,13)$\\
F.$x \in \{6\} \cup [11,13)$\\
G.$x \in \{-6\} \cup (11,13)$\\
H.$x \in \{6\} \cup [11,13]$
\testStop
\kluczStart
A
\kluczStop



\zadStart{Zadanie z Wikieł Z 1.62 c) moja wersja nr 618}

Rozwiązać nierówności $(11-x)(x+6)^{2}(14-x)^{3}\le0$.
\zadStop
\rozwStart{Patryk Wirkus}{}
Miejsca zerowe naszego wielomianu to: $11, -6, 14$.\\
Wielomian jest stopnia parzystego, ponadto znak współczynnika przy\linebreak najwyższej potędze x jest ujemny.\\ W związku z tym wykres wielomianu zaczyna się od lewej strony powyżej osi OX.\\
Ponadto w punkcie $-6$ wykres odbija się od osi poziomej.\\
A więc $$x \in \{-6\} \cup [11,14].$$
\rozwStop
\odpStart
$x \in \{-6\} \cup [11,14]$
\odpStop
\testStart
A.$x \in \{-6\} \cup [11,14]$\\
B.$x \in \{6\} \cup (11,14)$\\
C.$x \in \{-6\} \cup (11,14]$\\
D.$x \in \{6\} \cup (11,14]$\\
E.$x \in \{-6\} \cup [11,14)$\\
F.$x \in \{6\} \cup [11,14)$\\
G.$x \in \{-6\} \cup (11,14)$\\
H.$x \in \{6\} \cup [11,14]$
\testStop
\kluczStart
A
\kluczStop



\zadStart{Zadanie z Wikieł Z 1.62 c) moja wersja nr 619}

Rozwiązać nierówności $(11-x)(x+6)^{2}(15-x)^{3}\le0$.
\zadStop
\rozwStart{Patryk Wirkus}{}
Miejsca zerowe naszego wielomianu to: $11, -6, 15$.\\
Wielomian jest stopnia parzystego, ponadto znak współczynnika przy\linebreak najwyższej potędze x jest ujemny.\\ W związku z tym wykres wielomianu zaczyna się od lewej strony powyżej osi OX.\\
Ponadto w punkcie $-6$ wykres odbija się od osi poziomej.\\
A więc $$x \in \{-6\} \cup [11,15].$$
\rozwStop
\odpStart
$x \in \{-6\} \cup [11,15]$
\odpStop
\testStart
A.$x \in \{-6\} \cup [11,15]$\\
B.$x \in \{6\} \cup (11,15)$\\
C.$x \in \{-6\} \cup (11,15]$\\
D.$x \in \{6\} \cup (11,15]$\\
E.$x \in \{-6\} \cup [11,15)$\\
F.$x \in \{6\} \cup [11,15)$\\
G.$x \in \{-6\} \cup (11,15)$\\
H.$x \in \{6\} \cup [11,15]$
\testStop
\kluczStart
A
\kluczStop



\zadStart{Zadanie z Wikieł Z 1.62 c) moja wersja nr 620}

Rozwiązać nierówności $(11-x)(x+6)^{2}(16-x)^{3}\le0$.
\zadStop
\rozwStart{Patryk Wirkus}{}
Miejsca zerowe naszego wielomianu to: $11, -6, 16$.\\
Wielomian jest stopnia parzystego, ponadto znak współczynnika przy\linebreak najwyższej potędze x jest ujemny.\\ W związku z tym wykres wielomianu zaczyna się od lewej strony powyżej osi OX.\\
Ponadto w punkcie $-6$ wykres odbija się od osi poziomej.\\
A więc $$x \in \{-6\} \cup [11,16].$$
\rozwStop
\odpStart
$x \in \{-6\} \cup [11,16]$
\odpStop
\testStart
A.$x \in \{-6\} \cup [11,16]$\\
B.$x \in \{6\} \cup (11,16)$\\
C.$x \in \{-6\} \cup (11,16]$\\
D.$x \in \{6\} \cup (11,16]$\\
E.$x \in \{-6\} \cup [11,16)$\\
F.$x \in \{6\} \cup [11,16)$\\
G.$x \in \{-6\} \cup (11,16)$\\
H.$x \in \{6\} \cup [11,16]$
\testStop
\kluczStart
A
\kluczStop



\zadStart{Zadanie z Wikieł Z 1.62 c) moja wersja nr 621}

Rozwiązać nierówności $(11-x)(x+6)^{2}(17-x)^{3}\le0$.
\zadStop
\rozwStart{Patryk Wirkus}{}
Miejsca zerowe naszego wielomianu to: $11, -6, 17$.\\
Wielomian jest stopnia parzystego, ponadto znak współczynnika przy\linebreak najwyższej potędze x jest ujemny.\\ W związku z tym wykres wielomianu zaczyna się od lewej strony powyżej osi OX.\\
Ponadto w punkcie $-6$ wykres odbija się od osi poziomej.\\
A więc $$x \in \{-6\} \cup [11,17].$$
\rozwStop
\odpStart
$x \in \{-6\} \cup [11,17]$
\odpStop
\testStart
A.$x \in \{-6\} \cup [11,17]$\\
B.$x \in \{6\} \cup (11,17)$\\
C.$x \in \{-6\} \cup (11,17]$\\
D.$x \in \{6\} \cup (11,17]$\\
E.$x \in \{-6\} \cup [11,17)$\\
F.$x \in \{6\} \cup [11,17)$\\
G.$x \in \{-6\} \cup (11,17)$\\
H.$x \in \{6\} \cup [11,17]$
\testStop
\kluczStart
A
\kluczStop



\zadStart{Zadanie z Wikieł Z 1.62 c) moja wersja nr 622}

Rozwiązać nierówności $(11-x)(x+6)^{2}(18-x)^{3}\le0$.
\zadStop
\rozwStart{Patryk Wirkus}{}
Miejsca zerowe naszego wielomianu to: $11, -6, 18$.\\
Wielomian jest stopnia parzystego, ponadto znak współczynnika przy\linebreak najwyższej potędze x jest ujemny.\\ W związku z tym wykres wielomianu zaczyna się od lewej strony powyżej osi OX.\\
Ponadto w punkcie $-6$ wykres odbija się od osi poziomej.\\
A więc $$x \in \{-6\} \cup [11,18].$$
\rozwStop
\odpStart
$x \in \{-6\} \cup [11,18]$
\odpStop
\testStart
A.$x \in \{-6\} \cup [11,18]$\\
B.$x \in \{6\} \cup (11,18)$\\
C.$x \in \{-6\} \cup (11,18]$\\
D.$x \in \{6\} \cup (11,18]$\\
E.$x \in \{-6\} \cup [11,18)$\\
F.$x \in \{6\} \cup [11,18)$\\
G.$x \in \{-6\} \cup (11,18)$\\
H.$x \in \{6\} \cup [11,18]$
\testStop
\kluczStart
A
\kluczStop



\zadStart{Zadanie z Wikieł Z 1.62 c) moja wersja nr 623}

Rozwiązać nierówności $(11-x)(x+6)^{2}(19-x)^{3}\le0$.
\zadStop
\rozwStart{Patryk Wirkus}{}
Miejsca zerowe naszego wielomianu to: $11, -6, 19$.\\
Wielomian jest stopnia parzystego, ponadto znak współczynnika przy\linebreak najwyższej potędze x jest ujemny.\\ W związku z tym wykres wielomianu zaczyna się od lewej strony powyżej osi OX.\\
Ponadto w punkcie $-6$ wykres odbija się od osi poziomej.\\
A więc $$x \in \{-6\} \cup [11,19].$$
\rozwStop
\odpStart
$x \in \{-6\} \cup [11,19]$
\odpStop
\testStart
A.$x \in \{-6\} \cup [11,19]$\\
B.$x \in \{6\} \cup (11,19)$\\
C.$x \in \{-6\} \cup (11,19]$\\
D.$x \in \{6\} \cup (11,19]$\\
E.$x \in \{-6\} \cup [11,19)$\\
F.$x \in \{6\} \cup [11,19)$\\
G.$x \in \{-6\} \cup (11,19)$\\
H.$x \in \{6\} \cup [11,19]$
\testStop
\kluczStart
A
\kluczStop



\zadStart{Zadanie z Wikieł Z 1.62 c) moja wersja nr 624}

Rozwiązać nierówności $(11-x)(x+6)^{2}(20-x)^{3}\le0$.
\zadStop
\rozwStart{Patryk Wirkus}{}
Miejsca zerowe naszego wielomianu to: $11, -6, 20$.\\
Wielomian jest stopnia parzystego, ponadto znak współczynnika przy\linebreak najwyższej potędze x jest ujemny.\\ W związku z tym wykres wielomianu zaczyna się od lewej strony powyżej osi OX.\\
Ponadto w punkcie $-6$ wykres odbija się od osi poziomej.\\
A więc $$x \in \{-6\} \cup [11,20].$$
\rozwStop
\odpStart
$x \in \{-6\} \cup [11,20]$
\odpStop
\testStart
A.$x \in \{-6\} \cup [11,20]$\\
B.$x \in \{6\} \cup (11,20)$\\
C.$x \in \{-6\} \cup (11,20]$\\
D.$x \in \{6\} \cup (11,20]$\\
E.$x \in \{-6\} \cup [11,20)$\\
F.$x \in \{6\} \cup [11,20)$\\
G.$x \in \{-6\} \cup (11,20)$\\
H.$x \in \{6\} \cup [11,20]$
\testStop
\kluczStart
A
\kluczStop



\zadStart{Zadanie z Wikieł Z 1.62 c) moja wersja nr 625}

Rozwiązać nierówności $(11-x)(x+7)^{2}(12-x)^{3}\le0$.
\zadStop
\rozwStart{Patryk Wirkus}{}
Miejsca zerowe naszego wielomianu to: $11, -7, 12$.\\
Wielomian jest stopnia parzystego, ponadto znak współczynnika przy\linebreak najwyższej potędze x jest ujemny.\\ W związku z tym wykres wielomianu zaczyna się od lewej strony powyżej osi OX.\\
Ponadto w punkcie $-7$ wykres odbija się od osi poziomej.\\
A więc $$x \in \{-7\} \cup [11,12].$$
\rozwStop
\odpStart
$x \in \{-7\} \cup [11,12]$
\odpStop
\testStart
A.$x \in \{-7\} \cup [11,12]$\\
B.$x \in \{7\} \cup (11,12)$\\
C.$x \in \{-7\} \cup (11,12]$\\
D.$x \in \{7\} \cup (11,12]$\\
E.$x \in \{-7\} \cup [11,12)$\\
F.$x \in \{7\} \cup [11,12)$\\
G.$x \in \{-7\} \cup (11,12)$\\
H.$x \in \{7\} \cup [11,12]$
\testStop
\kluczStart
A
\kluczStop



\zadStart{Zadanie z Wikieł Z 1.62 c) moja wersja nr 626}

Rozwiązać nierówności $(11-x)(x+7)^{2}(13-x)^{3}\le0$.
\zadStop
\rozwStart{Patryk Wirkus}{}
Miejsca zerowe naszego wielomianu to: $11, -7, 13$.\\
Wielomian jest stopnia parzystego, ponadto znak współczynnika przy\linebreak najwyższej potędze x jest ujemny.\\ W związku z tym wykres wielomianu zaczyna się od lewej strony powyżej osi OX.\\
Ponadto w punkcie $-7$ wykres odbija się od osi poziomej.\\
A więc $$x \in \{-7\} \cup [11,13].$$
\rozwStop
\odpStart
$x \in \{-7\} \cup [11,13]$
\odpStop
\testStart
A.$x \in \{-7\} \cup [11,13]$\\
B.$x \in \{7\} \cup (11,13)$\\
C.$x \in \{-7\} \cup (11,13]$\\
D.$x \in \{7\} \cup (11,13]$\\
E.$x \in \{-7\} \cup [11,13)$\\
F.$x \in \{7\} \cup [11,13)$\\
G.$x \in \{-7\} \cup (11,13)$\\
H.$x \in \{7\} \cup [11,13]$
\testStop
\kluczStart
A
\kluczStop



\zadStart{Zadanie z Wikieł Z 1.62 c) moja wersja nr 627}

Rozwiązać nierówności $(11-x)(x+7)^{2}(14-x)^{3}\le0$.
\zadStop
\rozwStart{Patryk Wirkus}{}
Miejsca zerowe naszego wielomianu to: $11, -7, 14$.\\
Wielomian jest stopnia parzystego, ponadto znak współczynnika przy\linebreak najwyższej potędze x jest ujemny.\\ W związku z tym wykres wielomianu zaczyna się od lewej strony powyżej osi OX.\\
Ponadto w punkcie $-7$ wykres odbija się od osi poziomej.\\
A więc $$x \in \{-7\} \cup [11,14].$$
\rozwStop
\odpStart
$x \in \{-7\} \cup [11,14]$
\odpStop
\testStart
A.$x \in \{-7\} \cup [11,14]$\\
B.$x \in \{7\} \cup (11,14)$\\
C.$x \in \{-7\} \cup (11,14]$\\
D.$x \in \{7\} \cup (11,14]$\\
E.$x \in \{-7\} \cup [11,14)$\\
F.$x \in \{7\} \cup [11,14)$\\
G.$x \in \{-7\} \cup (11,14)$\\
H.$x \in \{7\} \cup [11,14]$
\testStop
\kluczStart
A
\kluczStop



\zadStart{Zadanie z Wikieł Z 1.62 c) moja wersja nr 628}

Rozwiązać nierówności $(11-x)(x+7)^{2}(15-x)^{3}\le0$.
\zadStop
\rozwStart{Patryk Wirkus}{}
Miejsca zerowe naszego wielomianu to: $11, -7, 15$.\\
Wielomian jest stopnia parzystego, ponadto znak współczynnika przy\linebreak najwyższej potędze x jest ujemny.\\ W związku z tym wykres wielomianu zaczyna się od lewej strony powyżej osi OX.\\
Ponadto w punkcie $-7$ wykres odbija się od osi poziomej.\\
A więc $$x \in \{-7\} \cup [11,15].$$
\rozwStop
\odpStart
$x \in \{-7\} \cup [11,15]$
\odpStop
\testStart
A.$x \in \{-7\} \cup [11,15]$\\
B.$x \in \{7\} \cup (11,15)$\\
C.$x \in \{-7\} \cup (11,15]$\\
D.$x \in \{7\} \cup (11,15]$\\
E.$x \in \{-7\} \cup [11,15)$\\
F.$x \in \{7\} \cup [11,15)$\\
G.$x \in \{-7\} \cup (11,15)$\\
H.$x \in \{7\} \cup [11,15]$
\testStop
\kluczStart
A
\kluczStop



\zadStart{Zadanie z Wikieł Z 1.62 c) moja wersja nr 629}

Rozwiązać nierówności $(11-x)(x+7)^{2}(16-x)^{3}\le0$.
\zadStop
\rozwStart{Patryk Wirkus}{}
Miejsca zerowe naszego wielomianu to: $11, -7, 16$.\\
Wielomian jest stopnia parzystego, ponadto znak współczynnika przy\linebreak najwyższej potędze x jest ujemny.\\ W związku z tym wykres wielomianu zaczyna się od lewej strony powyżej osi OX.\\
Ponadto w punkcie $-7$ wykres odbija się od osi poziomej.\\
A więc $$x \in \{-7\} \cup [11,16].$$
\rozwStop
\odpStart
$x \in \{-7\} \cup [11,16]$
\odpStop
\testStart
A.$x \in \{-7\} \cup [11,16]$\\
B.$x \in \{7\} \cup (11,16)$\\
C.$x \in \{-7\} \cup (11,16]$\\
D.$x \in \{7\} \cup (11,16]$\\
E.$x \in \{-7\} \cup [11,16)$\\
F.$x \in \{7\} \cup [11,16)$\\
G.$x \in \{-7\} \cup (11,16)$\\
H.$x \in \{7\} \cup [11,16]$
\testStop
\kluczStart
A
\kluczStop



\zadStart{Zadanie z Wikieł Z 1.62 c) moja wersja nr 630}

Rozwiązać nierówności $(11-x)(x+7)^{2}(17-x)^{3}\le0$.
\zadStop
\rozwStart{Patryk Wirkus}{}
Miejsca zerowe naszego wielomianu to: $11, -7, 17$.\\
Wielomian jest stopnia parzystego, ponadto znak współczynnika przy\linebreak najwyższej potędze x jest ujemny.\\ W związku z tym wykres wielomianu zaczyna się od lewej strony powyżej osi OX.\\
Ponadto w punkcie $-7$ wykres odbija się od osi poziomej.\\
A więc $$x \in \{-7\} \cup [11,17].$$
\rozwStop
\odpStart
$x \in \{-7\} \cup [11,17]$
\odpStop
\testStart
A.$x \in \{-7\} \cup [11,17]$\\
B.$x \in \{7\} \cup (11,17)$\\
C.$x \in \{-7\} \cup (11,17]$\\
D.$x \in \{7\} \cup (11,17]$\\
E.$x \in \{-7\} \cup [11,17)$\\
F.$x \in \{7\} \cup [11,17)$\\
G.$x \in \{-7\} \cup (11,17)$\\
H.$x \in \{7\} \cup [11,17]$
\testStop
\kluczStart
A
\kluczStop



\zadStart{Zadanie z Wikieł Z 1.62 c) moja wersja nr 631}

Rozwiązać nierówności $(11-x)(x+7)^{2}(18-x)^{3}\le0$.
\zadStop
\rozwStart{Patryk Wirkus}{}
Miejsca zerowe naszego wielomianu to: $11, -7, 18$.\\
Wielomian jest stopnia parzystego, ponadto znak współczynnika przy\linebreak najwyższej potędze x jest ujemny.\\ W związku z tym wykres wielomianu zaczyna się od lewej strony powyżej osi OX.\\
Ponadto w punkcie $-7$ wykres odbija się od osi poziomej.\\
A więc $$x \in \{-7\} \cup [11,18].$$
\rozwStop
\odpStart
$x \in \{-7\} \cup [11,18]$
\odpStop
\testStart
A.$x \in \{-7\} \cup [11,18]$\\
B.$x \in \{7\} \cup (11,18)$\\
C.$x \in \{-7\} \cup (11,18]$\\
D.$x \in \{7\} \cup (11,18]$\\
E.$x \in \{-7\} \cup [11,18)$\\
F.$x \in \{7\} \cup [11,18)$\\
G.$x \in \{-7\} \cup (11,18)$\\
H.$x \in \{7\} \cup [11,18]$
\testStop
\kluczStart
A
\kluczStop



\zadStart{Zadanie z Wikieł Z 1.62 c) moja wersja nr 632}

Rozwiązać nierówności $(11-x)(x+7)^{2}(19-x)^{3}\le0$.
\zadStop
\rozwStart{Patryk Wirkus}{}
Miejsca zerowe naszego wielomianu to: $11, -7, 19$.\\
Wielomian jest stopnia parzystego, ponadto znak współczynnika przy\linebreak najwyższej potędze x jest ujemny.\\ W związku z tym wykres wielomianu zaczyna się od lewej strony powyżej osi OX.\\
Ponadto w punkcie $-7$ wykres odbija się od osi poziomej.\\
A więc $$x \in \{-7\} \cup [11,19].$$
\rozwStop
\odpStart
$x \in \{-7\} \cup [11,19]$
\odpStop
\testStart
A.$x \in \{-7\} \cup [11,19]$\\
B.$x \in \{7\} \cup (11,19)$\\
C.$x \in \{-7\} \cup (11,19]$\\
D.$x \in \{7\} \cup (11,19]$\\
E.$x \in \{-7\} \cup [11,19)$\\
F.$x \in \{7\} \cup [11,19)$\\
G.$x \in \{-7\} \cup (11,19)$\\
H.$x \in \{7\} \cup [11,19]$
\testStop
\kluczStart
A
\kluczStop



\zadStart{Zadanie z Wikieł Z 1.62 c) moja wersja nr 633}

Rozwiązać nierówności $(11-x)(x+7)^{2}(20-x)^{3}\le0$.
\zadStop
\rozwStart{Patryk Wirkus}{}
Miejsca zerowe naszego wielomianu to: $11, -7, 20$.\\
Wielomian jest stopnia parzystego, ponadto znak współczynnika przy\linebreak najwyższej potędze x jest ujemny.\\ W związku z tym wykres wielomianu zaczyna się od lewej strony powyżej osi OX.\\
Ponadto w punkcie $-7$ wykres odbija się od osi poziomej.\\
A więc $$x \in \{-7\} \cup [11,20].$$
\rozwStop
\odpStart
$x \in \{-7\} \cup [11,20]$
\odpStop
\testStart
A.$x \in \{-7\} \cup [11,20]$\\
B.$x \in \{7\} \cup (11,20)$\\
C.$x \in \{-7\} \cup (11,20]$\\
D.$x \in \{7\} \cup (11,20]$\\
E.$x \in \{-7\} \cup [11,20)$\\
F.$x \in \{7\} \cup [11,20)$\\
G.$x \in \{-7\} \cup (11,20)$\\
H.$x \in \{7\} \cup [11,20]$
\testStop
\kluczStart
A
\kluczStop



\zadStart{Zadanie z Wikieł Z 1.62 c) moja wersja nr 634}

Rozwiązać nierówności $(11-x)(x+8)^{2}(12-x)^{3}\le0$.
\zadStop
\rozwStart{Patryk Wirkus}{}
Miejsca zerowe naszego wielomianu to: $11, -8, 12$.\\
Wielomian jest stopnia parzystego, ponadto znak współczynnika przy\linebreak najwyższej potędze x jest ujemny.\\ W związku z tym wykres wielomianu zaczyna się od lewej strony powyżej osi OX.\\
Ponadto w punkcie $-8$ wykres odbija się od osi poziomej.\\
A więc $$x \in \{-8\} \cup [11,12].$$
\rozwStop
\odpStart
$x \in \{-8\} \cup [11,12]$
\odpStop
\testStart
A.$x \in \{-8\} \cup [11,12]$\\
B.$x \in \{8\} \cup (11,12)$\\
C.$x \in \{-8\} \cup (11,12]$\\
D.$x \in \{8\} \cup (11,12]$\\
E.$x \in \{-8\} \cup [11,12)$\\
F.$x \in \{8\} \cup [11,12)$\\
G.$x \in \{-8\} \cup (11,12)$\\
H.$x \in \{8\} \cup [11,12]$
\testStop
\kluczStart
A
\kluczStop



\zadStart{Zadanie z Wikieł Z 1.62 c) moja wersja nr 635}

Rozwiązać nierówności $(11-x)(x+8)^{2}(13-x)^{3}\le0$.
\zadStop
\rozwStart{Patryk Wirkus}{}
Miejsca zerowe naszego wielomianu to: $11, -8, 13$.\\
Wielomian jest stopnia parzystego, ponadto znak współczynnika przy\linebreak najwyższej potędze x jest ujemny.\\ W związku z tym wykres wielomianu zaczyna się od lewej strony powyżej osi OX.\\
Ponadto w punkcie $-8$ wykres odbija się od osi poziomej.\\
A więc $$x \in \{-8\} \cup [11,13].$$
\rozwStop
\odpStart
$x \in \{-8\} \cup [11,13]$
\odpStop
\testStart
A.$x \in \{-8\} \cup [11,13]$\\
B.$x \in \{8\} \cup (11,13)$\\
C.$x \in \{-8\} \cup (11,13]$\\
D.$x \in \{8\} \cup (11,13]$\\
E.$x \in \{-8\} \cup [11,13)$\\
F.$x \in \{8\} \cup [11,13)$\\
G.$x \in \{-8\} \cup (11,13)$\\
H.$x \in \{8\} \cup [11,13]$
\testStop
\kluczStart
A
\kluczStop



\zadStart{Zadanie z Wikieł Z 1.62 c) moja wersja nr 636}

Rozwiązać nierówności $(11-x)(x+8)^{2}(14-x)^{3}\le0$.
\zadStop
\rozwStart{Patryk Wirkus}{}
Miejsca zerowe naszego wielomianu to: $11, -8, 14$.\\
Wielomian jest stopnia parzystego, ponadto znak współczynnika przy\linebreak najwyższej potędze x jest ujemny.\\ W związku z tym wykres wielomianu zaczyna się od lewej strony powyżej osi OX.\\
Ponadto w punkcie $-8$ wykres odbija się od osi poziomej.\\
A więc $$x \in \{-8\} \cup [11,14].$$
\rozwStop
\odpStart
$x \in \{-8\} \cup [11,14]$
\odpStop
\testStart
A.$x \in \{-8\} \cup [11,14]$\\
B.$x \in \{8\} \cup (11,14)$\\
C.$x \in \{-8\} \cup (11,14]$\\
D.$x \in \{8\} \cup (11,14]$\\
E.$x \in \{-8\} \cup [11,14)$\\
F.$x \in \{8\} \cup [11,14)$\\
G.$x \in \{-8\} \cup (11,14)$\\
H.$x \in \{8\} \cup [11,14]$
\testStop
\kluczStart
A
\kluczStop



\zadStart{Zadanie z Wikieł Z 1.62 c) moja wersja nr 637}

Rozwiązać nierówności $(11-x)(x+8)^{2}(15-x)^{3}\le0$.
\zadStop
\rozwStart{Patryk Wirkus}{}
Miejsca zerowe naszego wielomianu to: $11, -8, 15$.\\
Wielomian jest stopnia parzystego, ponadto znak współczynnika przy\linebreak najwyższej potędze x jest ujemny.\\ W związku z tym wykres wielomianu zaczyna się od lewej strony powyżej osi OX.\\
Ponadto w punkcie $-8$ wykres odbija się od osi poziomej.\\
A więc $$x \in \{-8\} \cup [11,15].$$
\rozwStop
\odpStart
$x \in \{-8\} \cup [11,15]$
\odpStop
\testStart
A.$x \in \{-8\} \cup [11,15]$\\
B.$x \in \{8\} \cup (11,15)$\\
C.$x \in \{-8\} \cup (11,15]$\\
D.$x \in \{8\} \cup (11,15]$\\
E.$x \in \{-8\} \cup [11,15)$\\
F.$x \in \{8\} \cup [11,15)$\\
G.$x \in \{-8\} \cup (11,15)$\\
H.$x \in \{8\} \cup [11,15]$
\testStop
\kluczStart
A
\kluczStop



\zadStart{Zadanie z Wikieł Z 1.62 c) moja wersja nr 638}

Rozwiązać nierówności $(11-x)(x+8)^{2}(16-x)^{3}\le0$.
\zadStop
\rozwStart{Patryk Wirkus}{}
Miejsca zerowe naszego wielomianu to: $11, -8, 16$.\\
Wielomian jest stopnia parzystego, ponadto znak współczynnika przy\linebreak najwyższej potędze x jest ujemny.\\ W związku z tym wykres wielomianu zaczyna się od lewej strony powyżej osi OX.\\
Ponadto w punkcie $-8$ wykres odbija się od osi poziomej.\\
A więc $$x \in \{-8\} \cup [11,16].$$
\rozwStop
\odpStart
$x \in \{-8\} \cup [11,16]$
\odpStop
\testStart
A.$x \in \{-8\} \cup [11,16]$\\
B.$x \in \{8\} \cup (11,16)$\\
C.$x \in \{-8\} \cup (11,16]$\\
D.$x \in \{8\} \cup (11,16]$\\
E.$x \in \{-8\} \cup [11,16)$\\
F.$x \in \{8\} \cup [11,16)$\\
G.$x \in \{-8\} \cup (11,16)$\\
H.$x \in \{8\} \cup [11,16]$
\testStop
\kluczStart
A
\kluczStop



\zadStart{Zadanie z Wikieł Z 1.62 c) moja wersja nr 639}

Rozwiązać nierówności $(11-x)(x+8)^{2}(17-x)^{3}\le0$.
\zadStop
\rozwStart{Patryk Wirkus}{}
Miejsca zerowe naszego wielomianu to: $11, -8, 17$.\\
Wielomian jest stopnia parzystego, ponadto znak współczynnika przy\linebreak najwyższej potędze x jest ujemny.\\ W związku z tym wykres wielomianu zaczyna się od lewej strony powyżej osi OX.\\
Ponadto w punkcie $-8$ wykres odbija się od osi poziomej.\\
A więc $$x \in \{-8\} \cup [11,17].$$
\rozwStop
\odpStart
$x \in \{-8\} \cup [11,17]$
\odpStop
\testStart
A.$x \in \{-8\} \cup [11,17]$\\
B.$x \in \{8\} \cup (11,17)$\\
C.$x \in \{-8\} \cup (11,17]$\\
D.$x \in \{8\} \cup (11,17]$\\
E.$x \in \{-8\} \cup [11,17)$\\
F.$x \in \{8\} \cup [11,17)$\\
G.$x \in \{-8\} \cup (11,17)$\\
H.$x \in \{8\} \cup [11,17]$
\testStop
\kluczStart
A
\kluczStop



\zadStart{Zadanie z Wikieł Z 1.62 c) moja wersja nr 640}

Rozwiązać nierówności $(11-x)(x+8)^{2}(18-x)^{3}\le0$.
\zadStop
\rozwStart{Patryk Wirkus}{}
Miejsca zerowe naszego wielomianu to: $11, -8, 18$.\\
Wielomian jest stopnia parzystego, ponadto znak współczynnika przy\linebreak najwyższej potędze x jest ujemny.\\ W związku z tym wykres wielomianu zaczyna się od lewej strony powyżej osi OX.\\
Ponadto w punkcie $-8$ wykres odbija się od osi poziomej.\\
A więc $$x \in \{-8\} \cup [11,18].$$
\rozwStop
\odpStart
$x \in \{-8\} \cup [11,18]$
\odpStop
\testStart
A.$x \in \{-8\} \cup [11,18]$\\
B.$x \in \{8\} \cup (11,18)$\\
C.$x \in \{-8\} \cup (11,18]$\\
D.$x \in \{8\} \cup (11,18]$\\
E.$x \in \{-8\} \cup [11,18)$\\
F.$x \in \{8\} \cup [11,18)$\\
G.$x \in \{-8\} \cup (11,18)$\\
H.$x \in \{8\} \cup [11,18]$
\testStop
\kluczStart
A
\kluczStop



\zadStart{Zadanie z Wikieł Z 1.62 c) moja wersja nr 641}

Rozwiązać nierówności $(11-x)(x+8)^{2}(19-x)^{3}\le0$.
\zadStop
\rozwStart{Patryk Wirkus}{}
Miejsca zerowe naszego wielomianu to: $11, -8, 19$.\\
Wielomian jest stopnia parzystego, ponadto znak współczynnika przy\linebreak najwyższej potędze x jest ujemny.\\ W związku z tym wykres wielomianu zaczyna się od lewej strony powyżej osi OX.\\
Ponadto w punkcie $-8$ wykres odbija się od osi poziomej.\\
A więc $$x \in \{-8\} \cup [11,19].$$
\rozwStop
\odpStart
$x \in \{-8\} \cup [11,19]$
\odpStop
\testStart
A.$x \in \{-8\} \cup [11,19]$\\
B.$x \in \{8\} \cup (11,19)$\\
C.$x \in \{-8\} \cup (11,19]$\\
D.$x \in \{8\} \cup (11,19]$\\
E.$x \in \{-8\} \cup [11,19)$\\
F.$x \in \{8\} \cup [11,19)$\\
G.$x \in \{-8\} \cup (11,19)$\\
H.$x \in \{8\} \cup [11,19]$
\testStop
\kluczStart
A
\kluczStop



\zadStart{Zadanie z Wikieł Z 1.62 c) moja wersja nr 642}

Rozwiązać nierówności $(11-x)(x+8)^{2}(20-x)^{3}\le0$.
\zadStop
\rozwStart{Patryk Wirkus}{}
Miejsca zerowe naszego wielomianu to: $11, -8, 20$.\\
Wielomian jest stopnia parzystego, ponadto znak współczynnika przy\linebreak najwyższej potędze x jest ujemny.\\ W związku z tym wykres wielomianu zaczyna się od lewej strony powyżej osi OX.\\
Ponadto w punkcie $-8$ wykres odbija się od osi poziomej.\\
A więc $$x \in \{-8\} \cup [11,20].$$
\rozwStop
\odpStart
$x \in \{-8\} \cup [11,20]$
\odpStop
\testStart
A.$x \in \{-8\} \cup [11,20]$\\
B.$x \in \{8\} \cup (11,20)$\\
C.$x \in \{-8\} \cup (11,20]$\\
D.$x \in \{8\} \cup (11,20]$\\
E.$x \in \{-8\} \cup [11,20)$\\
F.$x \in \{8\} \cup [11,20)$\\
G.$x \in \{-8\} \cup (11,20)$\\
H.$x \in \{8\} \cup [11,20]$
\testStop
\kluczStart
A
\kluczStop



\zadStart{Zadanie z Wikieł Z 1.62 c) moja wersja nr 643}

Rozwiązać nierówności $(11-x)(x+9)^{2}(12-x)^{3}\le0$.
\zadStop
\rozwStart{Patryk Wirkus}{}
Miejsca zerowe naszego wielomianu to: $11, -9, 12$.\\
Wielomian jest stopnia parzystego, ponadto znak współczynnika przy\linebreak najwyższej potędze x jest ujemny.\\ W związku z tym wykres wielomianu zaczyna się od lewej strony powyżej osi OX.\\
Ponadto w punkcie $-9$ wykres odbija się od osi poziomej.\\
A więc $$x \in \{-9\} \cup [11,12].$$
\rozwStop
\odpStart
$x \in \{-9\} \cup [11,12]$
\odpStop
\testStart
A.$x \in \{-9\} \cup [11,12]$\\
B.$x \in \{9\} \cup (11,12)$\\
C.$x \in \{-9\} \cup (11,12]$\\
D.$x \in \{9\} \cup (11,12]$\\
E.$x \in \{-9\} \cup [11,12)$\\
F.$x \in \{9\} \cup [11,12)$\\
G.$x \in \{-9\} \cup (11,12)$\\
H.$x \in \{9\} \cup [11,12]$
\testStop
\kluczStart
A
\kluczStop



\zadStart{Zadanie z Wikieł Z 1.62 c) moja wersja nr 644}

Rozwiązać nierówności $(11-x)(x+9)^{2}(13-x)^{3}\le0$.
\zadStop
\rozwStart{Patryk Wirkus}{}
Miejsca zerowe naszego wielomianu to: $11, -9, 13$.\\
Wielomian jest stopnia parzystego, ponadto znak współczynnika przy\linebreak najwyższej potędze x jest ujemny.\\ W związku z tym wykres wielomianu zaczyna się od lewej strony powyżej osi OX.\\
Ponadto w punkcie $-9$ wykres odbija się od osi poziomej.\\
A więc $$x \in \{-9\} \cup [11,13].$$
\rozwStop
\odpStart
$x \in \{-9\} \cup [11,13]$
\odpStop
\testStart
A.$x \in \{-9\} \cup [11,13]$\\
B.$x \in \{9\} \cup (11,13)$\\
C.$x \in \{-9\} \cup (11,13]$\\
D.$x \in \{9\} \cup (11,13]$\\
E.$x \in \{-9\} \cup [11,13)$\\
F.$x \in \{9\} \cup [11,13)$\\
G.$x \in \{-9\} \cup (11,13)$\\
H.$x \in \{9\} \cup [11,13]$
\testStop
\kluczStart
A
\kluczStop



\zadStart{Zadanie z Wikieł Z 1.62 c) moja wersja nr 645}

Rozwiązać nierówności $(11-x)(x+9)^{2}(14-x)^{3}\le0$.
\zadStop
\rozwStart{Patryk Wirkus}{}
Miejsca zerowe naszego wielomianu to: $11, -9, 14$.\\
Wielomian jest stopnia parzystego, ponadto znak współczynnika przy\linebreak najwyższej potędze x jest ujemny.\\ W związku z tym wykres wielomianu zaczyna się od lewej strony powyżej osi OX.\\
Ponadto w punkcie $-9$ wykres odbija się od osi poziomej.\\
A więc $$x \in \{-9\} \cup [11,14].$$
\rozwStop
\odpStart
$x \in \{-9\} \cup [11,14]$
\odpStop
\testStart
A.$x \in \{-9\} \cup [11,14]$\\
B.$x \in \{9\} \cup (11,14)$\\
C.$x \in \{-9\} \cup (11,14]$\\
D.$x \in \{9\} \cup (11,14]$\\
E.$x \in \{-9\} \cup [11,14)$\\
F.$x \in \{9\} \cup [11,14)$\\
G.$x \in \{-9\} \cup (11,14)$\\
H.$x \in \{9\} \cup [11,14]$
\testStop
\kluczStart
A
\kluczStop



\zadStart{Zadanie z Wikieł Z 1.62 c) moja wersja nr 646}

Rozwiązać nierówności $(11-x)(x+9)^{2}(15-x)^{3}\le0$.
\zadStop
\rozwStart{Patryk Wirkus}{}
Miejsca zerowe naszego wielomianu to: $11, -9, 15$.\\
Wielomian jest stopnia parzystego, ponadto znak współczynnika przy\linebreak najwyższej potędze x jest ujemny.\\ W związku z tym wykres wielomianu zaczyna się od lewej strony powyżej osi OX.\\
Ponadto w punkcie $-9$ wykres odbija się od osi poziomej.\\
A więc $$x \in \{-9\} \cup [11,15].$$
\rozwStop
\odpStart
$x \in \{-9\} \cup [11,15]$
\odpStop
\testStart
A.$x \in \{-9\} \cup [11,15]$\\
B.$x \in \{9\} \cup (11,15)$\\
C.$x \in \{-9\} \cup (11,15]$\\
D.$x \in \{9\} \cup (11,15]$\\
E.$x \in \{-9\} \cup [11,15)$\\
F.$x \in \{9\} \cup [11,15)$\\
G.$x \in \{-9\} \cup (11,15)$\\
H.$x \in \{9\} \cup [11,15]$
\testStop
\kluczStart
A
\kluczStop



\zadStart{Zadanie z Wikieł Z 1.62 c) moja wersja nr 647}

Rozwiązać nierówności $(11-x)(x+9)^{2}(16-x)^{3}\le0$.
\zadStop
\rozwStart{Patryk Wirkus}{}
Miejsca zerowe naszego wielomianu to: $11, -9, 16$.\\
Wielomian jest stopnia parzystego, ponadto znak współczynnika przy\linebreak najwyższej potędze x jest ujemny.\\ W związku z tym wykres wielomianu zaczyna się od lewej strony powyżej osi OX.\\
Ponadto w punkcie $-9$ wykres odbija się od osi poziomej.\\
A więc $$x \in \{-9\} \cup [11,16].$$
\rozwStop
\odpStart
$x \in \{-9\} \cup [11,16]$
\odpStop
\testStart
A.$x \in \{-9\} \cup [11,16]$\\
B.$x \in \{9\} \cup (11,16)$\\
C.$x \in \{-9\} \cup (11,16]$\\
D.$x \in \{9\} \cup (11,16]$\\
E.$x \in \{-9\} \cup [11,16)$\\
F.$x \in \{9\} \cup [11,16)$\\
G.$x \in \{-9\} \cup (11,16)$\\
H.$x \in \{9\} \cup [11,16]$
\testStop
\kluczStart
A
\kluczStop



\zadStart{Zadanie z Wikieł Z 1.62 c) moja wersja nr 648}

Rozwiązać nierówności $(11-x)(x+9)^{2}(17-x)^{3}\le0$.
\zadStop
\rozwStart{Patryk Wirkus}{}
Miejsca zerowe naszego wielomianu to: $11, -9, 17$.\\
Wielomian jest stopnia parzystego, ponadto znak współczynnika przy\linebreak najwyższej potędze x jest ujemny.\\ W związku z tym wykres wielomianu zaczyna się od lewej strony powyżej osi OX.\\
Ponadto w punkcie $-9$ wykres odbija się od osi poziomej.\\
A więc $$x \in \{-9\} \cup [11,17].$$
\rozwStop
\odpStart
$x \in \{-9\} \cup [11,17]$
\odpStop
\testStart
A.$x \in \{-9\} \cup [11,17]$\\
B.$x \in \{9\} \cup (11,17)$\\
C.$x \in \{-9\} \cup (11,17]$\\
D.$x \in \{9\} \cup (11,17]$\\
E.$x \in \{-9\} \cup [11,17)$\\
F.$x \in \{9\} \cup [11,17)$\\
G.$x \in \{-9\} \cup (11,17)$\\
H.$x \in \{9\} \cup [11,17]$
\testStop
\kluczStart
A
\kluczStop



\zadStart{Zadanie z Wikieł Z 1.62 c) moja wersja nr 649}

Rozwiązać nierówności $(11-x)(x+9)^{2}(18-x)^{3}\le0$.
\zadStop
\rozwStart{Patryk Wirkus}{}
Miejsca zerowe naszego wielomianu to: $11, -9, 18$.\\
Wielomian jest stopnia parzystego, ponadto znak współczynnika przy\linebreak najwyższej potędze x jest ujemny.\\ W związku z tym wykres wielomianu zaczyna się od lewej strony powyżej osi OX.\\
Ponadto w punkcie $-9$ wykres odbija się od osi poziomej.\\
A więc $$x \in \{-9\} \cup [11,18].$$
\rozwStop
\odpStart
$x \in \{-9\} \cup [11,18]$
\odpStop
\testStart
A.$x \in \{-9\} \cup [11,18]$\\
B.$x \in \{9\} \cup (11,18)$\\
C.$x \in \{-9\} \cup (11,18]$\\
D.$x \in \{9\} \cup (11,18]$\\
E.$x \in \{-9\} \cup [11,18)$\\
F.$x \in \{9\} \cup [11,18)$\\
G.$x \in \{-9\} \cup (11,18)$\\
H.$x \in \{9\} \cup [11,18]$
\testStop
\kluczStart
A
\kluczStop



\zadStart{Zadanie z Wikieł Z 1.62 c) moja wersja nr 650}

Rozwiązać nierówności $(11-x)(x+9)^{2}(19-x)^{3}\le0$.
\zadStop
\rozwStart{Patryk Wirkus}{}
Miejsca zerowe naszego wielomianu to: $11, -9, 19$.\\
Wielomian jest stopnia parzystego, ponadto znak współczynnika przy\linebreak najwyższej potędze x jest ujemny.\\ W związku z tym wykres wielomianu zaczyna się od lewej strony powyżej osi OX.\\
Ponadto w punkcie $-9$ wykres odbija się od osi poziomej.\\
A więc $$x \in \{-9\} \cup [11,19].$$
\rozwStop
\odpStart
$x \in \{-9\} \cup [11,19]$
\odpStop
\testStart
A.$x \in \{-9\} \cup [11,19]$\\
B.$x \in \{9\} \cup (11,19)$\\
C.$x \in \{-9\} \cup (11,19]$\\
D.$x \in \{9\} \cup (11,19]$\\
E.$x \in \{-9\} \cup [11,19)$\\
F.$x \in \{9\} \cup [11,19)$\\
G.$x \in \{-9\} \cup (11,19)$\\
H.$x \in \{9\} \cup [11,19]$
\testStop
\kluczStart
A
\kluczStop



\zadStart{Zadanie z Wikieł Z 1.62 c) moja wersja nr 651}

Rozwiązać nierówności $(11-x)(x+9)^{2}(20-x)^{3}\le0$.
\zadStop
\rozwStart{Patryk Wirkus}{}
Miejsca zerowe naszego wielomianu to: $11, -9, 20$.\\
Wielomian jest stopnia parzystego, ponadto znak współczynnika przy\linebreak najwyższej potędze x jest ujemny.\\ W związku z tym wykres wielomianu zaczyna się od lewej strony powyżej osi OX.\\
Ponadto w punkcie $-9$ wykres odbija się od osi poziomej.\\
A więc $$x \in \{-9\} \cup [11,20].$$
\rozwStop
\odpStart
$x \in \{-9\} \cup [11,20]$
\odpStop
\testStart
A.$x \in \{-9\} \cup [11,20]$\\
B.$x \in \{9\} \cup (11,20)$\\
C.$x \in \{-9\} \cup (11,20]$\\
D.$x \in \{9\} \cup (11,20]$\\
E.$x \in \{-9\} \cup [11,20)$\\
F.$x \in \{9\} \cup [11,20)$\\
G.$x \in \{-9\} \cup (11,20)$\\
H.$x \in \{9\} \cup [11,20]$
\testStop
\kluczStart
A
\kluczStop



\zadStart{Zadanie z Wikieł Z 1.62 c) moja wersja nr 652}

Rozwiązać nierówności $(11-x)(x+10)^{2}(12-x)^{3}\le0$.
\zadStop
\rozwStart{Patryk Wirkus}{}
Miejsca zerowe naszego wielomianu to: $11, -10, 12$.\\
Wielomian jest stopnia parzystego, ponadto znak współczynnika przy\linebreak najwyższej potędze x jest ujemny.\\ W związku z tym wykres wielomianu zaczyna się od lewej strony powyżej osi OX.\\
Ponadto w punkcie $-10$ wykres odbija się od osi poziomej.\\
A więc $$x \in \{-10\} \cup [11,12].$$
\rozwStop
\odpStart
$x \in \{-10\} \cup [11,12]$
\odpStop
\testStart
A.$x \in \{-10\} \cup [11,12]$\\
B.$x \in \{10\} \cup (11,12)$\\
C.$x \in \{-10\} \cup (11,12]$\\
D.$x \in \{10\} \cup (11,12]$\\
E.$x \in \{-10\} \cup [11,12)$\\
F.$x \in \{10\} \cup [11,12)$\\
G.$x \in \{-10\} \cup (11,12)$\\
H.$x \in \{10\} \cup [11,12]$
\testStop
\kluczStart
A
\kluczStop



\zadStart{Zadanie z Wikieł Z 1.62 c) moja wersja nr 653}

Rozwiązać nierówności $(11-x)(x+10)^{2}(13-x)^{3}\le0$.
\zadStop
\rozwStart{Patryk Wirkus}{}
Miejsca zerowe naszego wielomianu to: $11, -10, 13$.\\
Wielomian jest stopnia parzystego, ponadto znak współczynnika przy\linebreak najwyższej potędze x jest ujemny.\\ W związku z tym wykres wielomianu zaczyna się od lewej strony powyżej osi OX.\\
Ponadto w punkcie $-10$ wykres odbija się od osi poziomej.\\
A więc $$x \in \{-10\} \cup [11,13].$$
\rozwStop
\odpStart
$x \in \{-10\} \cup [11,13]$
\odpStop
\testStart
A.$x \in \{-10\} \cup [11,13]$\\
B.$x \in \{10\} \cup (11,13)$\\
C.$x \in \{-10\} \cup (11,13]$\\
D.$x \in \{10\} \cup (11,13]$\\
E.$x \in \{-10\} \cup [11,13)$\\
F.$x \in \{10\} \cup [11,13)$\\
G.$x \in \{-10\} \cup (11,13)$\\
H.$x \in \{10\} \cup [11,13]$
\testStop
\kluczStart
A
\kluczStop



\zadStart{Zadanie z Wikieł Z 1.62 c) moja wersja nr 654}

Rozwiązać nierówności $(11-x)(x+10)^{2}(14-x)^{3}\le0$.
\zadStop
\rozwStart{Patryk Wirkus}{}
Miejsca zerowe naszego wielomianu to: $11, -10, 14$.\\
Wielomian jest stopnia parzystego, ponadto znak współczynnika przy\linebreak najwyższej potędze x jest ujemny.\\ W związku z tym wykres wielomianu zaczyna się od lewej strony powyżej osi OX.\\
Ponadto w punkcie $-10$ wykres odbija się od osi poziomej.\\
A więc $$x \in \{-10\} \cup [11,14].$$
\rozwStop
\odpStart
$x \in \{-10\} \cup [11,14]$
\odpStop
\testStart
A.$x \in \{-10\} \cup [11,14]$\\
B.$x \in \{10\} \cup (11,14)$\\
C.$x \in \{-10\} \cup (11,14]$\\
D.$x \in \{10\} \cup (11,14]$\\
E.$x \in \{-10\} \cup [11,14)$\\
F.$x \in \{10\} \cup [11,14)$\\
G.$x \in \{-10\} \cup (11,14)$\\
H.$x \in \{10\} \cup [11,14]$
\testStop
\kluczStart
A
\kluczStop



\zadStart{Zadanie z Wikieł Z 1.62 c) moja wersja nr 655}

Rozwiązać nierówności $(11-x)(x+10)^{2}(15-x)^{3}\le0$.
\zadStop
\rozwStart{Patryk Wirkus}{}
Miejsca zerowe naszego wielomianu to: $11, -10, 15$.\\
Wielomian jest stopnia parzystego, ponadto znak współczynnika przy\linebreak najwyższej potędze x jest ujemny.\\ W związku z tym wykres wielomianu zaczyna się od lewej strony powyżej osi OX.\\
Ponadto w punkcie $-10$ wykres odbija się od osi poziomej.\\
A więc $$x \in \{-10\} \cup [11,15].$$
\rozwStop
\odpStart
$x \in \{-10\} \cup [11,15]$
\odpStop
\testStart
A.$x \in \{-10\} \cup [11,15]$\\
B.$x \in \{10\} \cup (11,15)$\\
C.$x \in \{-10\} \cup (11,15]$\\
D.$x \in \{10\} \cup (11,15]$\\
E.$x \in \{-10\} \cup [11,15)$\\
F.$x \in \{10\} \cup [11,15)$\\
G.$x \in \{-10\} \cup (11,15)$\\
H.$x \in \{10\} \cup [11,15]$
\testStop
\kluczStart
A
\kluczStop



\zadStart{Zadanie z Wikieł Z 1.62 c) moja wersja nr 656}

Rozwiązać nierówności $(11-x)(x+10)^{2}(16-x)^{3}\le0$.
\zadStop
\rozwStart{Patryk Wirkus}{}
Miejsca zerowe naszego wielomianu to: $11, -10, 16$.\\
Wielomian jest stopnia parzystego, ponadto znak współczynnika przy\linebreak najwyższej potędze x jest ujemny.\\ W związku z tym wykres wielomianu zaczyna się od lewej strony powyżej osi OX.\\
Ponadto w punkcie $-10$ wykres odbija się od osi poziomej.\\
A więc $$x \in \{-10\} \cup [11,16].$$
\rozwStop
\odpStart
$x \in \{-10\} \cup [11,16]$
\odpStop
\testStart
A.$x \in \{-10\} \cup [11,16]$\\
B.$x \in \{10\} \cup (11,16)$\\
C.$x \in \{-10\} \cup (11,16]$\\
D.$x \in \{10\} \cup (11,16]$\\
E.$x \in \{-10\} \cup [11,16)$\\
F.$x \in \{10\} \cup [11,16)$\\
G.$x \in \{-10\} \cup (11,16)$\\
H.$x \in \{10\} \cup [11,16]$
\testStop
\kluczStart
A
\kluczStop



\zadStart{Zadanie z Wikieł Z 1.62 c) moja wersja nr 657}

Rozwiązać nierówności $(11-x)(x+10)^{2}(17-x)^{3}\le0$.
\zadStop
\rozwStart{Patryk Wirkus}{}
Miejsca zerowe naszego wielomianu to: $11, -10, 17$.\\
Wielomian jest stopnia parzystego, ponadto znak współczynnika przy\linebreak najwyższej potędze x jest ujemny.\\ W związku z tym wykres wielomianu zaczyna się od lewej strony powyżej osi OX.\\
Ponadto w punkcie $-10$ wykres odbija się od osi poziomej.\\
A więc $$x \in \{-10\} \cup [11,17].$$
\rozwStop
\odpStart
$x \in \{-10\} \cup [11,17]$
\odpStop
\testStart
A.$x \in \{-10\} \cup [11,17]$\\
B.$x \in \{10\} \cup (11,17)$\\
C.$x \in \{-10\} \cup (11,17]$\\
D.$x \in \{10\} \cup (11,17]$\\
E.$x \in \{-10\} \cup [11,17)$\\
F.$x \in \{10\} \cup [11,17)$\\
G.$x \in \{-10\} \cup (11,17)$\\
H.$x \in \{10\} \cup [11,17]$
\testStop
\kluczStart
A
\kluczStop



\zadStart{Zadanie z Wikieł Z 1.62 c) moja wersja nr 658}

Rozwiązać nierówności $(11-x)(x+10)^{2}(18-x)^{3}\le0$.
\zadStop
\rozwStart{Patryk Wirkus}{}
Miejsca zerowe naszego wielomianu to: $11, -10, 18$.\\
Wielomian jest stopnia parzystego, ponadto znak współczynnika przy\linebreak najwyższej potędze x jest ujemny.\\ W związku z tym wykres wielomianu zaczyna się od lewej strony powyżej osi OX.\\
Ponadto w punkcie $-10$ wykres odbija się od osi poziomej.\\
A więc $$x \in \{-10\} \cup [11,18].$$
\rozwStop
\odpStart
$x \in \{-10\} \cup [11,18]$
\odpStop
\testStart
A.$x \in \{-10\} \cup [11,18]$\\
B.$x \in \{10\} \cup (11,18)$\\
C.$x \in \{-10\} \cup (11,18]$\\
D.$x \in \{10\} \cup (11,18]$\\
E.$x \in \{-10\} \cup [11,18)$\\
F.$x \in \{10\} \cup [11,18)$\\
G.$x \in \{-10\} \cup (11,18)$\\
H.$x \in \{10\} \cup [11,18]$
\testStop
\kluczStart
A
\kluczStop



\zadStart{Zadanie z Wikieł Z 1.62 c) moja wersja nr 659}

Rozwiązać nierówności $(11-x)(x+10)^{2}(19-x)^{3}\le0$.
\zadStop
\rozwStart{Patryk Wirkus}{}
Miejsca zerowe naszego wielomianu to: $11, -10, 19$.\\
Wielomian jest stopnia parzystego, ponadto znak współczynnika przy\linebreak najwyższej potędze x jest ujemny.\\ W związku z tym wykres wielomianu zaczyna się od lewej strony powyżej osi OX.\\
Ponadto w punkcie $-10$ wykres odbija się od osi poziomej.\\
A więc $$x \in \{-10\} \cup [11,19].$$
\rozwStop
\odpStart
$x \in \{-10\} \cup [11,19]$
\odpStop
\testStart
A.$x \in \{-10\} \cup [11,19]$\\
B.$x \in \{10\} \cup (11,19)$\\
C.$x \in \{-10\} \cup (11,19]$\\
D.$x \in \{10\} \cup (11,19]$\\
E.$x \in \{-10\} \cup [11,19)$\\
F.$x \in \{10\} \cup [11,19)$\\
G.$x \in \{-10\} \cup (11,19)$\\
H.$x \in \{10\} \cup [11,19]$
\testStop
\kluczStart
A
\kluczStop



\zadStart{Zadanie z Wikieł Z 1.62 c) moja wersja nr 660}

Rozwiązać nierówności $(11-x)(x+10)^{2}(20-x)^{3}\le0$.
\zadStop
\rozwStart{Patryk Wirkus}{}
Miejsca zerowe naszego wielomianu to: $11, -10, 20$.\\
Wielomian jest stopnia parzystego, ponadto znak współczynnika przy\linebreak najwyższej potędze x jest ujemny.\\ W związku z tym wykres wielomianu zaczyna się od lewej strony powyżej osi OX.\\
Ponadto w punkcie $-10$ wykres odbija się od osi poziomej.\\
A więc $$x \in \{-10\} \cup [11,20].$$
\rozwStop
\odpStart
$x \in \{-10\} \cup [11,20]$
\odpStop
\testStart
A.$x \in \{-10\} \cup [11,20]$\\
B.$x \in \{10\} \cup (11,20)$\\
C.$x \in \{-10\} \cup (11,20]$\\
D.$x \in \{10\} \cup (11,20]$\\
E.$x \in \{-10\} \cup [11,20)$\\
F.$x \in \{10\} \cup [11,20)$\\
G.$x \in \{-10\} \cup (11,20)$\\
H.$x \in \{10\} \cup [11,20]$
\testStop
\kluczStart
A
\kluczStop



\zadStart{Zadanie z Wikieł Z 1.62 c) moja wersja nr 661}

Rozwiązać nierówności $(12-x)(x+1)^{2}(13-x)^{3}\le0$.
\zadStop
\rozwStart{Patryk Wirkus}{}
Miejsca zerowe naszego wielomianu to: $12, -1, 13$.\\
Wielomian jest stopnia parzystego, ponadto znak współczynnika przy\linebreak najwyższej potędze x jest ujemny.\\ W związku z tym wykres wielomianu zaczyna się od lewej strony powyżej osi OX.\\
Ponadto w punkcie $-1$ wykres odbija się od osi poziomej.\\
A więc $$x \in \{-1\} \cup [12,13].$$
\rozwStop
\odpStart
$x \in \{-1\} \cup [12,13]$
\odpStop
\testStart
A.$x \in \{-1\} \cup [12,13]$\\
B.$x \in \{1\} \cup (12,13)$\\
C.$x \in \{-1\} \cup (12,13]$\\
D.$x \in \{1\} \cup (12,13]$\\
E.$x \in \{-1\} \cup [12,13)$\\
F.$x \in \{1\} \cup [12,13)$\\
G.$x \in \{-1\} \cup (12,13)$\\
H.$x \in \{1\} \cup [12,13]$
\testStop
\kluczStart
A
\kluczStop



\zadStart{Zadanie z Wikieł Z 1.62 c) moja wersja nr 662}

Rozwiązać nierówności $(12-x)(x+1)^{2}(14-x)^{3}\le0$.
\zadStop
\rozwStart{Patryk Wirkus}{}
Miejsca zerowe naszego wielomianu to: $12, -1, 14$.\\
Wielomian jest stopnia parzystego, ponadto znak współczynnika przy\linebreak najwyższej potędze x jest ujemny.\\ W związku z tym wykres wielomianu zaczyna się od lewej strony powyżej osi OX.\\
Ponadto w punkcie $-1$ wykres odbija się od osi poziomej.\\
A więc $$x \in \{-1\} \cup [12,14].$$
\rozwStop
\odpStart
$x \in \{-1\} \cup [12,14]$
\odpStop
\testStart
A.$x \in \{-1\} \cup [12,14]$\\
B.$x \in \{1\} \cup (12,14)$\\
C.$x \in \{-1\} \cup (12,14]$\\
D.$x \in \{1\} \cup (12,14]$\\
E.$x \in \{-1\} \cup [12,14)$\\
F.$x \in \{1\} \cup [12,14)$\\
G.$x \in \{-1\} \cup (12,14)$\\
H.$x \in \{1\} \cup [12,14]$
\testStop
\kluczStart
A
\kluczStop



\zadStart{Zadanie z Wikieł Z 1.62 c) moja wersja nr 663}

Rozwiązać nierówności $(12-x)(x+1)^{2}(15-x)^{3}\le0$.
\zadStop
\rozwStart{Patryk Wirkus}{}
Miejsca zerowe naszego wielomianu to: $12, -1, 15$.\\
Wielomian jest stopnia parzystego, ponadto znak współczynnika przy\linebreak najwyższej potędze x jest ujemny.\\ W związku z tym wykres wielomianu zaczyna się od lewej strony powyżej osi OX.\\
Ponadto w punkcie $-1$ wykres odbija się od osi poziomej.\\
A więc $$x \in \{-1\} \cup [12,15].$$
\rozwStop
\odpStart
$x \in \{-1\} \cup [12,15]$
\odpStop
\testStart
A.$x \in \{-1\} \cup [12,15]$\\
B.$x \in \{1\} \cup (12,15)$\\
C.$x \in \{-1\} \cup (12,15]$\\
D.$x \in \{1\} \cup (12,15]$\\
E.$x \in \{-1\} \cup [12,15)$\\
F.$x \in \{1\} \cup [12,15)$\\
G.$x \in \{-1\} \cup (12,15)$\\
H.$x \in \{1\} \cup [12,15]$
\testStop
\kluczStart
A
\kluczStop



\zadStart{Zadanie z Wikieł Z 1.62 c) moja wersja nr 664}

Rozwiązać nierówności $(12-x)(x+1)^{2}(16-x)^{3}\le0$.
\zadStop
\rozwStart{Patryk Wirkus}{}
Miejsca zerowe naszego wielomianu to: $12, -1, 16$.\\
Wielomian jest stopnia parzystego, ponadto znak współczynnika przy\linebreak najwyższej potędze x jest ujemny.\\ W związku z tym wykres wielomianu zaczyna się od lewej strony powyżej osi OX.\\
Ponadto w punkcie $-1$ wykres odbija się od osi poziomej.\\
A więc $$x \in \{-1\} \cup [12,16].$$
\rozwStop
\odpStart
$x \in \{-1\} \cup [12,16]$
\odpStop
\testStart
A.$x \in \{-1\} \cup [12,16]$\\
B.$x \in \{1\} \cup (12,16)$\\
C.$x \in \{-1\} \cup (12,16]$\\
D.$x \in \{1\} \cup (12,16]$\\
E.$x \in \{-1\} \cup [12,16)$\\
F.$x \in \{1\} \cup [12,16)$\\
G.$x \in \{-1\} \cup (12,16)$\\
H.$x \in \{1\} \cup [12,16]$
\testStop
\kluczStart
A
\kluczStop



\zadStart{Zadanie z Wikieł Z 1.62 c) moja wersja nr 665}

Rozwiązać nierówności $(12-x)(x+1)^{2}(17-x)^{3}\le0$.
\zadStop
\rozwStart{Patryk Wirkus}{}
Miejsca zerowe naszego wielomianu to: $12, -1, 17$.\\
Wielomian jest stopnia parzystego, ponadto znak współczynnika przy\linebreak najwyższej potędze x jest ujemny.\\ W związku z tym wykres wielomianu zaczyna się od lewej strony powyżej osi OX.\\
Ponadto w punkcie $-1$ wykres odbija się od osi poziomej.\\
A więc $$x \in \{-1\} \cup [12,17].$$
\rozwStop
\odpStart
$x \in \{-1\} \cup [12,17]$
\odpStop
\testStart
A.$x \in \{-1\} \cup [12,17]$\\
B.$x \in \{1\} \cup (12,17)$\\
C.$x \in \{-1\} \cup (12,17]$\\
D.$x \in \{1\} \cup (12,17]$\\
E.$x \in \{-1\} \cup [12,17)$\\
F.$x \in \{1\} \cup [12,17)$\\
G.$x \in \{-1\} \cup (12,17)$\\
H.$x \in \{1\} \cup [12,17]$
\testStop
\kluczStart
A
\kluczStop



\zadStart{Zadanie z Wikieł Z 1.62 c) moja wersja nr 666}

Rozwiązać nierówności $(12-x)(x+1)^{2}(18-x)^{3}\le0$.
\zadStop
\rozwStart{Patryk Wirkus}{}
Miejsca zerowe naszego wielomianu to: $12, -1, 18$.\\
Wielomian jest stopnia parzystego, ponadto znak współczynnika przy\linebreak najwyższej potędze x jest ujemny.\\ W związku z tym wykres wielomianu zaczyna się od lewej strony powyżej osi OX.\\
Ponadto w punkcie $-1$ wykres odbija się od osi poziomej.\\
A więc $$x \in \{-1\} \cup [12,18].$$
\rozwStop
\odpStart
$x \in \{-1\} \cup [12,18]$
\odpStop
\testStart
A.$x \in \{-1\} \cup [12,18]$\\
B.$x \in \{1\} \cup (12,18)$\\
C.$x \in \{-1\} \cup (12,18]$\\
D.$x \in \{1\} \cup (12,18]$\\
E.$x \in \{-1\} \cup [12,18)$\\
F.$x \in \{1\} \cup [12,18)$\\
G.$x \in \{-1\} \cup (12,18)$\\
H.$x \in \{1\} \cup [12,18]$
\testStop
\kluczStart
A
\kluczStop



\zadStart{Zadanie z Wikieł Z 1.62 c) moja wersja nr 667}

Rozwiązać nierówności $(12-x)(x+1)^{2}(19-x)^{3}\le0$.
\zadStop
\rozwStart{Patryk Wirkus}{}
Miejsca zerowe naszego wielomianu to: $12, -1, 19$.\\
Wielomian jest stopnia parzystego, ponadto znak współczynnika przy\linebreak najwyższej potędze x jest ujemny.\\ W związku z tym wykres wielomianu zaczyna się od lewej strony powyżej osi OX.\\
Ponadto w punkcie $-1$ wykres odbija się od osi poziomej.\\
A więc $$x \in \{-1\} \cup [12,19].$$
\rozwStop
\odpStart
$x \in \{-1\} \cup [12,19]$
\odpStop
\testStart
A.$x \in \{-1\} \cup [12,19]$\\
B.$x \in \{1\} \cup (12,19)$\\
C.$x \in \{-1\} \cup (12,19]$\\
D.$x \in \{1\} \cup (12,19]$\\
E.$x \in \{-1\} \cup [12,19)$\\
F.$x \in \{1\} \cup [12,19)$\\
G.$x \in \{-1\} \cup (12,19)$\\
H.$x \in \{1\} \cup [12,19]$
\testStop
\kluczStart
A
\kluczStop



\zadStart{Zadanie z Wikieł Z 1.62 c) moja wersja nr 668}

Rozwiązać nierówności $(12-x)(x+1)^{2}(20-x)^{3}\le0$.
\zadStop
\rozwStart{Patryk Wirkus}{}
Miejsca zerowe naszego wielomianu to: $12, -1, 20$.\\
Wielomian jest stopnia parzystego, ponadto znak współczynnika przy\linebreak najwyższej potędze x jest ujemny.\\ W związku z tym wykres wielomianu zaczyna się od lewej strony powyżej osi OX.\\
Ponadto w punkcie $-1$ wykres odbija się od osi poziomej.\\
A więc $$x \in \{-1\} \cup [12,20].$$
\rozwStop
\odpStart
$x \in \{-1\} \cup [12,20]$
\odpStop
\testStart
A.$x \in \{-1\} \cup [12,20]$\\
B.$x \in \{1\} \cup (12,20)$\\
C.$x \in \{-1\} \cup (12,20]$\\
D.$x \in \{1\} \cup (12,20]$\\
E.$x \in \{-1\} \cup [12,20)$\\
F.$x \in \{1\} \cup [12,20)$\\
G.$x \in \{-1\} \cup (12,20)$\\
H.$x \in \{1\} \cup [12,20]$
\testStop
\kluczStart
A
\kluczStop



\zadStart{Zadanie z Wikieł Z 1.62 c) moja wersja nr 669}

Rozwiązać nierówności $(12-x)(x+2)^{2}(13-x)^{3}\le0$.
\zadStop
\rozwStart{Patryk Wirkus}{}
Miejsca zerowe naszego wielomianu to: $12, -2, 13$.\\
Wielomian jest stopnia parzystego, ponadto znak współczynnika przy\linebreak najwyższej potędze x jest ujemny.\\ W związku z tym wykres wielomianu zaczyna się od lewej strony powyżej osi OX.\\
Ponadto w punkcie $-2$ wykres odbija się od osi poziomej.\\
A więc $$x \in \{-2\} \cup [12,13].$$
\rozwStop
\odpStart
$x \in \{-2\} \cup [12,13]$
\odpStop
\testStart
A.$x \in \{-2\} \cup [12,13]$\\
B.$x \in \{2\} \cup (12,13)$\\
C.$x \in \{-2\} \cup (12,13]$\\
D.$x \in \{2\} \cup (12,13]$\\
E.$x \in \{-2\} \cup [12,13)$\\
F.$x \in \{2\} \cup [12,13)$\\
G.$x \in \{-2\} \cup (12,13)$\\
H.$x \in \{2\} \cup [12,13]$
\testStop
\kluczStart
A
\kluczStop



\zadStart{Zadanie z Wikieł Z 1.62 c) moja wersja nr 670}

Rozwiązać nierówności $(12-x)(x+2)^{2}(14-x)^{3}\le0$.
\zadStop
\rozwStart{Patryk Wirkus}{}
Miejsca zerowe naszego wielomianu to: $12, -2, 14$.\\
Wielomian jest stopnia parzystego, ponadto znak współczynnika przy\linebreak najwyższej potędze x jest ujemny.\\ W związku z tym wykres wielomianu zaczyna się od lewej strony powyżej osi OX.\\
Ponadto w punkcie $-2$ wykres odbija się od osi poziomej.\\
A więc $$x \in \{-2\} \cup [12,14].$$
\rozwStop
\odpStart
$x \in \{-2\} \cup [12,14]$
\odpStop
\testStart
A.$x \in \{-2\} \cup [12,14]$\\
B.$x \in \{2\} \cup (12,14)$\\
C.$x \in \{-2\} \cup (12,14]$\\
D.$x \in \{2\} \cup (12,14]$\\
E.$x \in \{-2\} \cup [12,14)$\\
F.$x \in \{2\} \cup [12,14)$\\
G.$x \in \{-2\} \cup (12,14)$\\
H.$x \in \{2\} \cup [12,14]$
\testStop
\kluczStart
A
\kluczStop



\zadStart{Zadanie z Wikieł Z 1.62 c) moja wersja nr 671}

Rozwiązać nierówności $(12-x)(x+2)^{2}(15-x)^{3}\le0$.
\zadStop
\rozwStart{Patryk Wirkus}{}
Miejsca zerowe naszego wielomianu to: $12, -2, 15$.\\
Wielomian jest stopnia parzystego, ponadto znak współczynnika przy\linebreak najwyższej potędze x jest ujemny.\\ W związku z tym wykres wielomianu zaczyna się od lewej strony powyżej osi OX.\\
Ponadto w punkcie $-2$ wykres odbija się od osi poziomej.\\
A więc $$x \in \{-2\} \cup [12,15].$$
\rozwStop
\odpStart
$x \in \{-2\} \cup [12,15]$
\odpStop
\testStart
A.$x \in \{-2\} \cup [12,15]$\\
B.$x \in \{2\} \cup (12,15)$\\
C.$x \in \{-2\} \cup (12,15]$\\
D.$x \in \{2\} \cup (12,15]$\\
E.$x \in \{-2\} \cup [12,15)$\\
F.$x \in \{2\} \cup [12,15)$\\
G.$x \in \{-2\} \cup (12,15)$\\
H.$x \in \{2\} \cup [12,15]$
\testStop
\kluczStart
A
\kluczStop



\zadStart{Zadanie z Wikieł Z 1.62 c) moja wersja nr 672}

Rozwiązać nierówności $(12-x)(x+2)^{2}(16-x)^{3}\le0$.
\zadStop
\rozwStart{Patryk Wirkus}{}
Miejsca zerowe naszego wielomianu to: $12, -2, 16$.\\
Wielomian jest stopnia parzystego, ponadto znak współczynnika przy\linebreak najwyższej potędze x jest ujemny.\\ W związku z tym wykres wielomianu zaczyna się od lewej strony powyżej osi OX.\\
Ponadto w punkcie $-2$ wykres odbija się od osi poziomej.\\
A więc $$x \in \{-2\} \cup [12,16].$$
\rozwStop
\odpStart
$x \in \{-2\} \cup [12,16]$
\odpStop
\testStart
A.$x \in \{-2\} \cup [12,16]$\\
B.$x \in \{2\} \cup (12,16)$\\
C.$x \in \{-2\} \cup (12,16]$\\
D.$x \in \{2\} \cup (12,16]$\\
E.$x \in \{-2\} \cup [12,16)$\\
F.$x \in \{2\} \cup [12,16)$\\
G.$x \in \{-2\} \cup (12,16)$\\
H.$x \in \{2\} \cup [12,16]$
\testStop
\kluczStart
A
\kluczStop



\zadStart{Zadanie z Wikieł Z 1.62 c) moja wersja nr 673}

Rozwiązać nierówności $(12-x)(x+2)^{2}(17-x)^{3}\le0$.
\zadStop
\rozwStart{Patryk Wirkus}{}
Miejsca zerowe naszego wielomianu to: $12, -2, 17$.\\
Wielomian jest stopnia parzystego, ponadto znak współczynnika przy\linebreak najwyższej potędze x jest ujemny.\\ W związku z tym wykres wielomianu zaczyna się od lewej strony powyżej osi OX.\\
Ponadto w punkcie $-2$ wykres odbija się od osi poziomej.\\
A więc $$x \in \{-2\} \cup [12,17].$$
\rozwStop
\odpStart
$x \in \{-2\} \cup [12,17]$
\odpStop
\testStart
A.$x \in \{-2\} \cup [12,17]$\\
B.$x \in \{2\} \cup (12,17)$\\
C.$x \in \{-2\} \cup (12,17]$\\
D.$x \in \{2\} \cup (12,17]$\\
E.$x \in \{-2\} \cup [12,17)$\\
F.$x \in \{2\} \cup [12,17)$\\
G.$x \in \{-2\} \cup (12,17)$\\
H.$x \in \{2\} \cup [12,17]$
\testStop
\kluczStart
A
\kluczStop



\zadStart{Zadanie z Wikieł Z 1.62 c) moja wersja nr 674}

Rozwiązać nierówności $(12-x)(x+2)^{2}(18-x)^{3}\le0$.
\zadStop
\rozwStart{Patryk Wirkus}{}
Miejsca zerowe naszego wielomianu to: $12, -2, 18$.\\
Wielomian jest stopnia parzystego, ponadto znak współczynnika przy\linebreak najwyższej potędze x jest ujemny.\\ W związku z tym wykres wielomianu zaczyna się od lewej strony powyżej osi OX.\\
Ponadto w punkcie $-2$ wykres odbija się od osi poziomej.\\
A więc $$x \in \{-2\} \cup [12,18].$$
\rozwStop
\odpStart
$x \in \{-2\} \cup [12,18]$
\odpStop
\testStart
A.$x \in \{-2\} \cup [12,18]$\\
B.$x \in \{2\} \cup (12,18)$\\
C.$x \in \{-2\} \cup (12,18]$\\
D.$x \in \{2\} \cup (12,18]$\\
E.$x \in \{-2\} \cup [12,18)$\\
F.$x \in \{2\} \cup [12,18)$\\
G.$x \in \{-2\} \cup (12,18)$\\
H.$x \in \{2\} \cup [12,18]$
\testStop
\kluczStart
A
\kluczStop



\zadStart{Zadanie z Wikieł Z 1.62 c) moja wersja nr 675}

Rozwiązać nierówności $(12-x)(x+2)^{2}(19-x)^{3}\le0$.
\zadStop
\rozwStart{Patryk Wirkus}{}
Miejsca zerowe naszego wielomianu to: $12, -2, 19$.\\
Wielomian jest stopnia parzystego, ponadto znak współczynnika przy\linebreak najwyższej potędze x jest ujemny.\\ W związku z tym wykres wielomianu zaczyna się od lewej strony powyżej osi OX.\\
Ponadto w punkcie $-2$ wykres odbija się od osi poziomej.\\
A więc $$x \in \{-2\} \cup [12,19].$$
\rozwStop
\odpStart
$x \in \{-2\} \cup [12,19]$
\odpStop
\testStart
A.$x \in \{-2\} \cup [12,19]$\\
B.$x \in \{2\} \cup (12,19)$\\
C.$x \in \{-2\} \cup (12,19]$\\
D.$x \in \{2\} \cup (12,19]$\\
E.$x \in \{-2\} \cup [12,19)$\\
F.$x \in \{2\} \cup [12,19)$\\
G.$x \in \{-2\} \cup (12,19)$\\
H.$x \in \{2\} \cup [12,19]$
\testStop
\kluczStart
A
\kluczStop



\zadStart{Zadanie z Wikieł Z 1.62 c) moja wersja nr 676}

Rozwiązać nierówności $(12-x)(x+2)^{2}(20-x)^{3}\le0$.
\zadStop
\rozwStart{Patryk Wirkus}{}
Miejsca zerowe naszego wielomianu to: $12, -2, 20$.\\
Wielomian jest stopnia parzystego, ponadto znak współczynnika przy\linebreak najwyższej potędze x jest ujemny.\\ W związku z tym wykres wielomianu zaczyna się od lewej strony powyżej osi OX.\\
Ponadto w punkcie $-2$ wykres odbija się od osi poziomej.\\
A więc $$x \in \{-2\} \cup [12,20].$$
\rozwStop
\odpStart
$x \in \{-2\} \cup [12,20]$
\odpStop
\testStart
A.$x \in \{-2\} \cup [12,20]$\\
B.$x \in \{2\} \cup (12,20)$\\
C.$x \in \{-2\} \cup (12,20]$\\
D.$x \in \{2\} \cup (12,20]$\\
E.$x \in \{-2\} \cup [12,20)$\\
F.$x \in \{2\} \cup [12,20)$\\
G.$x \in \{-2\} \cup (12,20)$\\
H.$x \in \{2\} \cup [12,20]$
\testStop
\kluczStart
A
\kluczStop



\zadStart{Zadanie z Wikieł Z 1.62 c) moja wersja nr 677}

Rozwiązać nierówności $(12-x)(x+3)^{2}(13-x)^{3}\le0$.
\zadStop
\rozwStart{Patryk Wirkus}{}
Miejsca zerowe naszego wielomianu to: $12, -3, 13$.\\
Wielomian jest stopnia parzystego, ponadto znak współczynnika przy\linebreak najwyższej potędze x jest ujemny.\\ W związku z tym wykres wielomianu zaczyna się od lewej strony powyżej osi OX.\\
Ponadto w punkcie $-3$ wykres odbija się od osi poziomej.\\
A więc $$x \in \{-3\} \cup [12,13].$$
\rozwStop
\odpStart
$x \in \{-3\} \cup [12,13]$
\odpStop
\testStart
A.$x \in \{-3\} \cup [12,13]$\\
B.$x \in \{3\} \cup (12,13)$\\
C.$x \in \{-3\} \cup (12,13]$\\
D.$x \in \{3\} \cup (12,13]$\\
E.$x \in \{-3\} \cup [12,13)$\\
F.$x \in \{3\} \cup [12,13)$\\
G.$x \in \{-3\} \cup (12,13)$\\
H.$x \in \{3\} \cup [12,13]$
\testStop
\kluczStart
A
\kluczStop



\zadStart{Zadanie z Wikieł Z 1.62 c) moja wersja nr 678}

Rozwiązać nierówności $(12-x)(x+3)^{2}(14-x)^{3}\le0$.
\zadStop
\rozwStart{Patryk Wirkus}{}
Miejsca zerowe naszego wielomianu to: $12, -3, 14$.\\
Wielomian jest stopnia parzystego, ponadto znak współczynnika przy\linebreak najwyższej potędze x jest ujemny.\\ W związku z tym wykres wielomianu zaczyna się od lewej strony powyżej osi OX.\\
Ponadto w punkcie $-3$ wykres odbija się od osi poziomej.\\
A więc $$x \in \{-3\} \cup [12,14].$$
\rozwStop
\odpStart
$x \in \{-3\} \cup [12,14]$
\odpStop
\testStart
A.$x \in \{-3\} \cup [12,14]$\\
B.$x \in \{3\} \cup (12,14)$\\
C.$x \in \{-3\} \cup (12,14]$\\
D.$x \in \{3\} \cup (12,14]$\\
E.$x \in \{-3\} \cup [12,14)$\\
F.$x \in \{3\} \cup [12,14)$\\
G.$x \in \{-3\} \cup (12,14)$\\
H.$x \in \{3\} \cup [12,14]$
\testStop
\kluczStart
A
\kluczStop



\zadStart{Zadanie z Wikieł Z 1.62 c) moja wersja nr 679}

Rozwiązać nierówności $(12-x)(x+3)^{2}(15-x)^{3}\le0$.
\zadStop
\rozwStart{Patryk Wirkus}{}
Miejsca zerowe naszego wielomianu to: $12, -3, 15$.\\
Wielomian jest stopnia parzystego, ponadto znak współczynnika przy\linebreak najwyższej potędze x jest ujemny.\\ W związku z tym wykres wielomianu zaczyna się od lewej strony powyżej osi OX.\\
Ponadto w punkcie $-3$ wykres odbija się od osi poziomej.\\
A więc $$x \in \{-3\} \cup [12,15].$$
\rozwStop
\odpStart
$x \in \{-3\} \cup [12,15]$
\odpStop
\testStart
A.$x \in \{-3\} \cup [12,15]$\\
B.$x \in \{3\} \cup (12,15)$\\
C.$x \in \{-3\} \cup (12,15]$\\
D.$x \in \{3\} \cup (12,15]$\\
E.$x \in \{-3\} \cup [12,15)$\\
F.$x \in \{3\} \cup [12,15)$\\
G.$x \in \{-3\} \cup (12,15)$\\
H.$x \in \{3\} \cup [12,15]$
\testStop
\kluczStart
A
\kluczStop



\zadStart{Zadanie z Wikieł Z 1.62 c) moja wersja nr 680}

Rozwiązać nierówności $(12-x)(x+3)^{2}(16-x)^{3}\le0$.
\zadStop
\rozwStart{Patryk Wirkus}{}
Miejsca zerowe naszego wielomianu to: $12, -3, 16$.\\
Wielomian jest stopnia parzystego, ponadto znak współczynnika przy\linebreak najwyższej potędze x jest ujemny.\\ W związku z tym wykres wielomianu zaczyna się od lewej strony powyżej osi OX.\\
Ponadto w punkcie $-3$ wykres odbija się od osi poziomej.\\
A więc $$x \in \{-3\} \cup [12,16].$$
\rozwStop
\odpStart
$x \in \{-3\} \cup [12,16]$
\odpStop
\testStart
A.$x \in \{-3\} \cup [12,16]$\\
B.$x \in \{3\} \cup (12,16)$\\
C.$x \in \{-3\} \cup (12,16]$\\
D.$x \in \{3\} \cup (12,16]$\\
E.$x \in \{-3\} \cup [12,16)$\\
F.$x \in \{3\} \cup [12,16)$\\
G.$x \in \{-3\} \cup (12,16)$\\
H.$x \in \{3\} \cup [12,16]$
\testStop
\kluczStart
A
\kluczStop



\zadStart{Zadanie z Wikieł Z 1.62 c) moja wersja nr 681}

Rozwiązać nierówności $(12-x)(x+3)^{2}(17-x)^{3}\le0$.
\zadStop
\rozwStart{Patryk Wirkus}{}
Miejsca zerowe naszego wielomianu to: $12, -3, 17$.\\
Wielomian jest stopnia parzystego, ponadto znak współczynnika przy\linebreak najwyższej potędze x jest ujemny.\\ W związku z tym wykres wielomianu zaczyna się od lewej strony powyżej osi OX.\\
Ponadto w punkcie $-3$ wykres odbija się od osi poziomej.\\
A więc $$x \in \{-3\} \cup [12,17].$$
\rozwStop
\odpStart
$x \in \{-3\} \cup [12,17]$
\odpStop
\testStart
A.$x \in \{-3\} \cup [12,17]$\\
B.$x \in \{3\} \cup (12,17)$\\
C.$x \in \{-3\} \cup (12,17]$\\
D.$x \in \{3\} \cup (12,17]$\\
E.$x \in \{-3\} \cup [12,17)$\\
F.$x \in \{3\} \cup [12,17)$\\
G.$x \in \{-3\} \cup (12,17)$\\
H.$x \in \{3\} \cup [12,17]$
\testStop
\kluczStart
A
\kluczStop



\zadStart{Zadanie z Wikieł Z 1.62 c) moja wersja nr 682}

Rozwiązać nierówności $(12-x)(x+3)^{2}(18-x)^{3}\le0$.
\zadStop
\rozwStart{Patryk Wirkus}{}
Miejsca zerowe naszego wielomianu to: $12, -3, 18$.\\
Wielomian jest stopnia parzystego, ponadto znak współczynnika przy\linebreak najwyższej potędze x jest ujemny.\\ W związku z tym wykres wielomianu zaczyna się od lewej strony powyżej osi OX.\\
Ponadto w punkcie $-3$ wykres odbija się od osi poziomej.\\
A więc $$x \in \{-3\} \cup [12,18].$$
\rozwStop
\odpStart
$x \in \{-3\} \cup [12,18]$
\odpStop
\testStart
A.$x \in \{-3\} \cup [12,18]$\\
B.$x \in \{3\} \cup (12,18)$\\
C.$x \in \{-3\} \cup (12,18]$\\
D.$x \in \{3\} \cup (12,18]$\\
E.$x \in \{-3\} \cup [12,18)$\\
F.$x \in \{3\} \cup [12,18)$\\
G.$x \in \{-3\} \cup (12,18)$\\
H.$x \in \{3\} \cup [12,18]$
\testStop
\kluczStart
A
\kluczStop



\zadStart{Zadanie z Wikieł Z 1.62 c) moja wersja nr 683}

Rozwiązać nierówności $(12-x)(x+3)^{2}(19-x)^{3}\le0$.
\zadStop
\rozwStart{Patryk Wirkus}{}
Miejsca zerowe naszego wielomianu to: $12, -3, 19$.\\
Wielomian jest stopnia parzystego, ponadto znak współczynnika przy\linebreak najwyższej potędze x jest ujemny.\\ W związku z tym wykres wielomianu zaczyna się od lewej strony powyżej osi OX.\\
Ponadto w punkcie $-3$ wykres odbija się od osi poziomej.\\
A więc $$x \in \{-3\} \cup [12,19].$$
\rozwStop
\odpStart
$x \in \{-3\} \cup [12,19]$
\odpStop
\testStart
A.$x \in \{-3\} \cup [12,19]$\\
B.$x \in \{3\} \cup (12,19)$\\
C.$x \in \{-3\} \cup (12,19]$\\
D.$x \in \{3\} \cup (12,19]$\\
E.$x \in \{-3\} \cup [12,19)$\\
F.$x \in \{3\} \cup [12,19)$\\
G.$x \in \{-3\} \cup (12,19)$\\
H.$x \in \{3\} \cup [12,19]$
\testStop
\kluczStart
A
\kluczStop



\zadStart{Zadanie z Wikieł Z 1.62 c) moja wersja nr 684}

Rozwiązać nierówności $(12-x)(x+3)^{2}(20-x)^{3}\le0$.
\zadStop
\rozwStart{Patryk Wirkus}{}
Miejsca zerowe naszego wielomianu to: $12, -3, 20$.\\
Wielomian jest stopnia parzystego, ponadto znak współczynnika przy\linebreak najwyższej potędze x jest ujemny.\\ W związku z tym wykres wielomianu zaczyna się od lewej strony powyżej osi OX.\\
Ponadto w punkcie $-3$ wykres odbija się od osi poziomej.\\
A więc $$x \in \{-3\} \cup [12,20].$$
\rozwStop
\odpStart
$x \in \{-3\} \cup [12,20]$
\odpStop
\testStart
A.$x \in \{-3\} \cup [12,20]$\\
B.$x \in \{3\} \cup (12,20)$\\
C.$x \in \{-3\} \cup (12,20]$\\
D.$x \in \{3\} \cup (12,20]$\\
E.$x \in \{-3\} \cup [12,20)$\\
F.$x \in \{3\} \cup [12,20)$\\
G.$x \in \{-3\} \cup (12,20)$\\
H.$x \in \{3\} \cup [12,20]$
\testStop
\kluczStart
A
\kluczStop



\zadStart{Zadanie z Wikieł Z 1.62 c) moja wersja nr 685}

Rozwiązać nierówności $(12-x)(x+4)^{2}(13-x)^{3}\le0$.
\zadStop
\rozwStart{Patryk Wirkus}{}
Miejsca zerowe naszego wielomianu to: $12, -4, 13$.\\
Wielomian jest stopnia parzystego, ponadto znak współczynnika przy\linebreak najwyższej potędze x jest ujemny.\\ W związku z tym wykres wielomianu zaczyna się od lewej strony powyżej osi OX.\\
Ponadto w punkcie $-4$ wykres odbija się od osi poziomej.\\
A więc $$x \in \{-4\} \cup [12,13].$$
\rozwStop
\odpStart
$x \in \{-4\} \cup [12,13]$
\odpStop
\testStart
A.$x \in \{-4\} \cup [12,13]$\\
B.$x \in \{4\} \cup (12,13)$\\
C.$x \in \{-4\} \cup (12,13]$\\
D.$x \in \{4\} \cup (12,13]$\\
E.$x \in \{-4\} \cup [12,13)$\\
F.$x \in \{4\} \cup [12,13)$\\
G.$x \in \{-4\} \cup (12,13)$\\
H.$x \in \{4\} \cup [12,13]$
\testStop
\kluczStart
A
\kluczStop



\zadStart{Zadanie z Wikieł Z 1.62 c) moja wersja nr 686}

Rozwiązać nierówności $(12-x)(x+4)^{2}(14-x)^{3}\le0$.
\zadStop
\rozwStart{Patryk Wirkus}{}
Miejsca zerowe naszego wielomianu to: $12, -4, 14$.\\
Wielomian jest stopnia parzystego, ponadto znak współczynnika przy\linebreak najwyższej potędze x jest ujemny.\\ W związku z tym wykres wielomianu zaczyna się od lewej strony powyżej osi OX.\\
Ponadto w punkcie $-4$ wykres odbija się od osi poziomej.\\
A więc $$x \in \{-4\} \cup [12,14].$$
\rozwStop
\odpStart
$x \in \{-4\} \cup [12,14]$
\odpStop
\testStart
A.$x \in \{-4\} \cup [12,14]$\\
B.$x \in \{4\} \cup (12,14)$\\
C.$x \in \{-4\} \cup (12,14]$\\
D.$x \in \{4\} \cup (12,14]$\\
E.$x \in \{-4\} \cup [12,14)$\\
F.$x \in \{4\} \cup [12,14)$\\
G.$x \in \{-4\} \cup (12,14)$\\
H.$x \in \{4\} \cup [12,14]$
\testStop
\kluczStart
A
\kluczStop



\zadStart{Zadanie z Wikieł Z 1.62 c) moja wersja nr 687}

Rozwiązać nierówności $(12-x)(x+4)^{2}(15-x)^{3}\le0$.
\zadStop
\rozwStart{Patryk Wirkus}{}
Miejsca zerowe naszego wielomianu to: $12, -4, 15$.\\
Wielomian jest stopnia parzystego, ponadto znak współczynnika przy\linebreak najwyższej potędze x jest ujemny.\\ W związku z tym wykres wielomianu zaczyna się od lewej strony powyżej osi OX.\\
Ponadto w punkcie $-4$ wykres odbija się od osi poziomej.\\
A więc $$x \in \{-4\} \cup [12,15].$$
\rozwStop
\odpStart
$x \in \{-4\} \cup [12,15]$
\odpStop
\testStart
A.$x \in \{-4\} \cup [12,15]$\\
B.$x \in \{4\} \cup (12,15)$\\
C.$x \in \{-4\} \cup (12,15]$\\
D.$x \in \{4\} \cup (12,15]$\\
E.$x \in \{-4\} \cup [12,15)$\\
F.$x \in \{4\} \cup [12,15)$\\
G.$x \in \{-4\} \cup (12,15)$\\
H.$x \in \{4\} \cup [12,15]$
\testStop
\kluczStart
A
\kluczStop



\zadStart{Zadanie z Wikieł Z 1.62 c) moja wersja nr 688}

Rozwiązać nierówności $(12-x)(x+4)^{2}(16-x)^{3}\le0$.
\zadStop
\rozwStart{Patryk Wirkus}{}
Miejsca zerowe naszego wielomianu to: $12, -4, 16$.\\
Wielomian jest stopnia parzystego, ponadto znak współczynnika przy\linebreak najwyższej potędze x jest ujemny.\\ W związku z tym wykres wielomianu zaczyna się od lewej strony powyżej osi OX.\\
Ponadto w punkcie $-4$ wykres odbija się od osi poziomej.\\
A więc $$x \in \{-4\} \cup [12,16].$$
\rozwStop
\odpStart
$x \in \{-4\} \cup [12,16]$
\odpStop
\testStart
A.$x \in \{-4\} \cup [12,16]$\\
B.$x \in \{4\} \cup (12,16)$\\
C.$x \in \{-4\} \cup (12,16]$\\
D.$x \in \{4\} \cup (12,16]$\\
E.$x \in \{-4\} \cup [12,16)$\\
F.$x \in \{4\} \cup [12,16)$\\
G.$x \in \{-4\} \cup (12,16)$\\
H.$x \in \{4\} \cup [12,16]$
\testStop
\kluczStart
A
\kluczStop



\zadStart{Zadanie z Wikieł Z 1.62 c) moja wersja nr 689}

Rozwiązać nierówności $(12-x)(x+4)^{2}(17-x)^{3}\le0$.
\zadStop
\rozwStart{Patryk Wirkus}{}
Miejsca zerowe naszego wielomianu to: $12, -4, 17$.\\
Wielomian jest stopnia parzystego, ponadto znak współczynnika przy\linebreak najwyższej potędze x jest ujemny.\\ W związku z tym wykres wielomianu zaczyna się od lewej strony powyżej osi OX.\\
Ponadto w punkcie $-4$ wykres odbija się od osi poziomej.\\
A więc $$x \in \{-4\} \cup [12,17].$$
\rozwStop
\odpStart
$x \in \{-4\} \cup [12,17]$
\odpStop
\testStart
A.$x \in \{-4\} \cup [12,17]$\\
B.$x \in \{4\} \cup (12,17)$\\
C.$x \in \{-4\} \cup (12,17]$\\
D.$x \in \{4\} \cup (12,17]$\\
E.$x \in \{-4\} \cup [12,17)$\\
F.$x \in \{4\} \cup [12,17)$\\
G.$x \in \{-4\} \cup (12,17)$\\
H.$x \in \{4\} \cup [12,17]$
\testStop
\kluczStart
A
\kluczStop



\zadStart{Zadanie z Wikieł Z 1.62 c) moja wersja nr 690}

Rozwiązać nierówności $(12-x)(x+4)^{2}(18-x)^{3}\le0$.
\zadStop
\rozwStart{Patryk Wirkus}{}
Miejsca zerowe naszego wielomianu to: $12, -4, 18$.\\
Wielomian jest stopnia parzystego, ponadto znak współczynnika przy\linebreak najwyższej potędze x jest ujemny.\\ W związku z tym wykres wielomianu zaczyna się od lewej strony powyżej osi OX.\\
Ponadto w punkcie $-4$ wykres odbija się od osi poziomej.\\
A więc $$x \in \{-4\} \cup [12,18].$$
\rozwStop
\odpStart
$x \in \{-4\} \cup [12,18]$
\odpStop
\testStart
A.$x \in \{-4\} \cup [12,18]$\\
B.$x \in \{4\} \cup (12,18)$\\
C.$x \in \{-4\} \cup (12,18]$\\
D.$x \in \{4\} \cup (12,18]$\\
E.$x \in \{-4\} \cup [12,18)$\\
F.$x \in \{4\} \cup [12,18)$\\
G.$x \in \{-4\} \cup (12,18)$\\
H.$x \in \{4\} \cup [12,18]$
\testStop
\kluczStart
A
\kluczStop



\zadStart{Zadanie z Wikieł Z 1.62 c) moja wersja nr 691}

Rozwiązać nierówności $(12-x)(x+4)^{2}(19-x)^{3}\le0$.
\zadStop
\rozwStart{Patryk Wirkus}{}
Miejsca zerowe naszego wielomianu to: $12, -4, 19$.\\
Wielomian jest stopnia parzystego, ponadto znak współczynnika przy\linebreak najwyższej potędze x jest ujemny.\\ W związku z tym wykres wielomianu zaczyna się od lewej strony powyżej osi OX.\\
Ponadto w punkcie $-4$ wykres odbija się od osi poziomej.\\
A więc $$x \in \{-4\} \cup [12,19].$$
\rozwStop
\odpStart
$x \in \{-4\} \cup [12,19]$
\odpStop
\testStart
A.$x \in \{-4\} \cup [12,19]$\\
B.$x \in \{4\} \cup (12,19)$\\
C.$x \in \{-4\} \cup (12,19]$\\
D.$x \in \{4\} \cup (12,19]$\\
E.$x \in \{-4\} \cup [12,19)$\\
F.$x \in \{4\} \cup [12,19)$\\
G.$x \in \{-4\} \cup (12,19)$\\
H.$x \in \{4\} \cup [12,19]$
\testStop
\kluczStart
A
\kluczStop



\zadStart{Zadanie z Wikieł Z 1.62 c) moja wersja nr 692}

Rozwiązać nierówności $(12-x)(x+4)^{2}(20-x)^{3}\le0$.
\zadStop
\rozwStart{Patryk Wirkus}{}
Miejsca zerowe naszego wielomianu to: $12, -4, 20$.\\
Wielomian jest stopnia parzystego, ponadto znak współczynnika przy\linebreak najwyższej potędze x jest ujemny.\\ W związku z tym wykres wielomianu zaczyna się od lewej strony powyżej osi OX.\\
Ponadto w punkcie $-4$ wykres odbija się od osi poziomej.\\
A więc $$x \in \{-4\} \cup [12,20].$$
\rozwStop
\odpStart
$x \in \{-4\} \cup [12,20]$
\odpStop
\testStart
A.$x \in \{-4\} \cup [12,20]$\\
B.$x \in \{4\} \cup (12,20)$\\
C.$x \in \{-4\} \cup (12,20]$\\
D.$x \in \{4\} \cup (12,20]$\\
E.$x \in \{-4\} \cup [12,20)$\\
F.$x \in \{4\} \cup [12,20)$\\
G.$x \in \{-4\} \cup (12,20)$\\
H.$x \in \{4\} \cup [12,20]$
\testStop
\kluczStart
A
\kluczStop



\zadStart{Zadanie z Wikieł Z 1.62 c) moja wersja nr 693}

Rozwiązać nierówności $(12-x)(x+5)^{2}(13-x)^{3}\le0$.
\zadStop
\rozwStart{Patryk Wirkus}{}
Miejsca zerowe naszego wielomianu to: $12, -5, 13$.\\
Wielomian jest stopnia parzystego, ponadto znak współczynnika przy\linebreak najwyższej potędze x jest ujemny.\\ W związku z tym wykres wielomianu zaczyna się od lewej strony powyżej osi OX.\\
Ponadto w punkcie $-5$ wykres odbija się od osi poziomej.\\
A więc $$x \in \{-5\} \cup [12,13].$$
\rozwStop
\odpStart
$x \in \{-5\} \cup [12,13]$
\odpStop
\testStart
A.$x \in \{-5\} \cup [12,13]$\\
B.$x \in \{5\} \cup (12,13)$\\
C.$x \in \{-5\} \cup (12,13]$\\
D.$x \in \{5\} \cup (12,13]$\\
E.$x \in \{-5\} \cup [12,13)$\\
F.$x \in \{5\} \cup [12,13)$\\
G.$x \in \{-5\} \cup (12,13)$\\
H.$x \in \{5\} \cup [12,13]$
\testStop
\kluczStart
A
\kluczStop



\zadStart{Zadanie z Wikieł Z 1.62 c) moja wersja nr 694}

Rozwiązać nierówności $(12-x)(x+5)^{2}(14-x)^{3}\le0$.
\zadStop
\rozwStart{Patryk Wirkus}{}
Miejsca zerowe naszego wielomianu to: $12, -5, 14$.\\
Wielomian jest stopnia parzystego, ponadto znak współczynnika przy\linebreak najwyższej potędze x jest ujemny.\\ W związku z tym wykres wielomianu zaczyna się od lewej strony powyżej osi OX.\\
Ponadto w punkcie $-5$ wykres odbija się od osi poziomej.\\
A więc $$x \in \{-5\} \cup [12,14].$$
\rozwStop
\odpStart
$x \in \{-5\} \cup [12,14]$
\odpStop
\testStart
A.$x \in \{-5\} \cup [12,14]$\\
B.$x \in \{5\} \cup (12,14)$\\
C.$x \in \{-5\} \cup (12,14]$\\
D.$x \in \{5\} \cup (12,14]$\\
E.$x \in \{-5\} \cup [12,14)$\\
F.$x \in \{5\} \cup [12,14)$\\
G.$x \in \{-5\} \cup (12,14)$\\
H.$x \in \{5\} \cup [12,14]$
\testStop
\kluczStart
A
\kluczStop



\zadStart{Zadanie z Wikieł Z 1.62 c) moja wersja nr 695}

Rozwiązać nierówności $(12-x)(x+5)^{2}(15-x)^{3}\le0$.
\zadStop
\rozwStart{Patryk Wirkus}{}
Miejsca zerowe naszego wielomianu to: $12, -5, 15$.\\
Wielomian jest stopnia parzystego, ponadto znak współczynnika przy\linebreak najwyższej potędze x jest ujemny.\\ W związku z tym wykres wielomianu zaczyna się od lewej strony powyżej osi OX.\\
Ponadto w punkcie $-5$ wykres odbija się od osi poziomej.\\
A więc $$x \in \{-5\} \cup [12,15].$$
\rozwStop
\odpStart
$x \in \{-5\} \cup [12,15]$
\odpStop
\testStart
A.$x \in \{-5\} \cup [12,15]$\\
B.$x \in \{5\} \cup (12,15)$\\
C.$x \in \{-5\} \cup (12,15]$\\
D.$x \in \{5\} \cup (12,15]$\\
E.$x \in \{-5\} \cup [12,15)$\\
F.$x \in \{5\} \cup [12,15)$\\
G.$x \in \{-5\} \cup (12,15)$\\
H.$x \in \{5\} \cup [12,15]$
\testStop
\kluczStart
A
\kluczStop



\zadStart{Zadanie z Wikieł Z 1.62 c) moja wersja nr 696}

Rozwiązać nierówności $(12-x)(x+5)^{2}(16-x)^{3}\le0$.
\zadStop
\rozwStart{Patryk Wirkus}{}
Miejsca zerowe naszego wielomianu to: $12, -5, 16$.\\
Wielomian jest stopnia parzystego, ponadto znak współczynnika przy\linebreak najwyższej potędze x jest ujemny.\\ W związku z tym wykres wielomianu zaczyna się od lewej strony powyżej osi OX.\\
Ponadto w punkcie $-5$ wykres odbija się od osi poziomej.\\
A więc $$x \in \{-5\} \cup [12,16].$$
\rozwStop
\odpStart
$x \in \{-5\} \cup [12,16]$
\odpStop
\testStart
A.$x \in \{-5\} \cup [12,16]$\\
B.$x \in \{5\} \cup (12,16)$\\
C.$x \in \{-5\} \cup (12,16]$\\
D.$x \in \{5\} \cup (12,16]$\\
E.$x \in \{-5\} \cup [12,16)$\\
F.$x \in \{5\} \cup [12,16)$\\
G.$x \in \{-5\} \cup (12,16)$\\
H.$x \in \{5\} \cup [12,16]$
\testStop
\kluczStart
A
\kluczStop



\zadStart{Zadanie z Wikieł Z 1.62 c) moja wersja nr 697}

Rozwiązać nierówności $(12-x)(x+5)^{2}(17-x)^{3}\le0$.
\zadStop
\rozwStart{Patryk Wirkus}{}
Miejsca zerowe naszego wielomianu to: $12, -5, 17$.\\
Wielomian jest stopnia parzystego, ponadto znak współczynnika przy\linebreak najwyższej potędze x jest ujemny.\\ W związku z tym wykres wielomianu zaczyna się od lewej strony powyżej osi OX.\\
Ponadto w punkcie $-5$ wykres odbija się od osi poziomej.\\
A więc $$x \in \{-5\} \cup [12,17].$$
\rozwStop
\odpStart
$x \in \{-5\} \cup [12,17]$
\odpStop
\testStart
A.$x \in \{-5\} \cup [12,17]$\\
B.$x \in \{5\} \cup (12,17)$\\
C.$x \in \{-5\} \cup (12,17]$\\
D.$x \in \{5\} \cup (12,17]$\\
E.$x \in \{-5\} \cup [12,17)$\\
F.$x \in \{5\} \cup [12,17)$\\
G.$x \in \{-5\} \cup (12,17)$\\
H.$x \in \{5\} \cup [12,17]$
\testStop
\kluczStart
A
\kluczStop



\zadStart{Zadanie z Wikieł Z 1.62 c) moja wersja nr 698}

Rozwiązać nierówności $(12-x)(x+5)^{2}(18-x)^{3}\le0$.
\zadStop
\rozwStart{Patryk Wirkus}{}
Miejsca zerowe naszego wielomianu to: $12, -5, 18$.\\
Wielomian jest stopnia parzystego, ponadto znak współczynnika przy\linebreak najwyższej potędze x jest ujemny.\\ W związku z tym wykres wielomianu zaczyna się od lewej strony powyżej osi OX.\\
Ponadto w punkcie $-5$ wykres odbija się od osi poziomej.\\
A więc $$x \in \{-5\} \cup [12,18].$$
\rozwStop
\odpStart
$x \in \{-5\} \cup [12,18]$
\odpStop
\testStart
A.$x \in \{-5\} \cup [12,18]$\\
B.$x \in \{5\} \cup (12,18)$\\
C.$x \in \{-5\} \cup (12,18]$\\
D.$x \in \{5\} \cup (12,18]$\\
E.$x \in \{-5\} \cup [12,18)$\\
F.$x \in \{5\} \cup [12,18)$\\
G.$x \in \{-5\} \cup (12,18)$\\
H.$x \in \{5\} \cup [12,18]$
\testStop
\kluczStart
A
\kluczStop



\zadStart{Zadanie z Wikieł Z 1.62 c) moja wersja nr 699}

Rozwiązać nierówności $(12-x)(x+5)^{2}(19-x)^{3}\le0$.
\zadStop
\rozwStart{Patryk Wirkus}{}
Miejsca zerowe naszego wielomianu to: $12, -5, 19$.\\
Wielomian jest stopnia parzystego, ponadto znak współczynnika przy\linebreak najwyższej potędze x jest ujemny.\\ W związku z tym wykres wielomianu zaczyna się od lewej strony powyżej osi OX.\\
Ponadto w punkcie $-5$ wykres odbija się od osi poziomej.\\
A więc $$x \in \{-5\} \cup [12,19].$$
\rozwStop
\odpStart
$x \in \{-5\} \cup [12,19]$
\odpStop
\testStart
A.$x \in \{-5\} \cup [12,19]$\\
B.$x \in \{5\} \cup (12,19)$\\
C.$x \in \{-5\} \cup (12,19]$\\
D.$x \in \{5\} \cup (12,19]$\\
E.$x \in \{-5\} \cup [12,19)$\\
F.$x \in \{5\} \cup [12,19)$\\
G.$x \in \{-5\} \cup (12,19)$\\
H.$x \in \{5\} \cup [12,19]$
\testStop
\kluczStart
A
\kluczStop



\zadStart{Zadanie z Wikieł Z 1.62 c) moja wersja nr 700}

Rozwiązać nierówności $(12-x)(x+5)^{2}(20-x)^{3}\le0$.
\zadStop
\rozwStart{Patryk Wirkus}{}
Miejsca zerowe naszego wielomianu to: $12, -5, 20$.\\
Wielomian jest stopnia parzystego, ponadto znak współczynnika przy\linebreak najwyższej potędze x jest ujemny.\\ W związku z tym wykres wielomianu zaczyna się od lewej strony powyżej osi OX.\\
Ponadto w punkcie $-5$ wykres odbija się od osi poziomej.\\
A więc $$x \in \{-5\} \cup [12,20].$$
\rozwStop
\odpStart
$x \in \{-5\} \cup [12,20]$
\odpStop
\testStart
A.$x \in \{-5\} \cup [12,20]$\\
B.$x \in \{5\} \cup (12,20)$\\
C.$x \in \{-5\} \cup (12,20]$\\
D.$x \in \{5\} \cup (12,20]$\\
E.$x \in \{-5\} \cup [12,20)$\\
F.$x \in \{5\} \cup [12,20)$\\
G.$x \in \{-5\} \cup (12,20)$\\
H.$x \in \{5\} \cup [12,20]$
\testStop
\kluczStart
A
\kluczStop



\zadStart{Zadanie z Wikieł Z 1.62 c) moja wersja nr 701}

Rozwiązać nierówności $(12-x)(x+6)^{2}(13-x)^{3}\le0$.
\zadStop
\rozwStart{Patryk Wirkus}{}
Miejsca zerowe naszego wielomianu to: $12, -6, 13$.\\
Wielomian jest stopnia parzystego, ponadto znak współczynnika przy\linebreak najwyższej potędze x jest ujemny.\\ W związku z tym wykres wielomianu zaczyna się od lewej strony powyżej osi OX.\\
Ponadto w punkcie $-6$ wykres odbija się od osi poziomej.\\
A więc $$x \in \{-6\} \cup [12,13].$$
\rozwStop
\odpStart
$x \in \{-6\} \cup [12,13]$
\odpStop
\testStart
A.$x \in \{-6\} \cup [12,13]$\\
B.$x \in \{6\} \cup (12,13)$\\
C.$x \in \{-6\} \cup (12,13]$\\
D.$x \in \{6\} \cup (12,13]$\\
E.$x \in \{-6\} \cup [12,13)$\\
F.$x \in \{6\} \cup [12,13)$\\
G.$x \in \{-6\} \cup (12,13)$\\
H.$x \in \{6\} \cup [12,13]$
\testStop
\kluczStart
A
\kluczStop



\zadStart{Zadanie z Wikieł Z 1.62 c) moja wersja nr 702}

Rozwiązać nierówności $(12-x)(x+6)^{2}(14-x)^{3}\le0$.
\zadStop
\rozwStart{Patryk Wirkus}{}
Miejsca zerowe naszego wielomianu to: $12, -6, 14$.\\
Wielomian jest stopnia parzystego, ponadto znak współczynnika przy\linebreak najwyższej potędze x jest ujemny.\\ W związku z tym wykres wielomianu zaczyna się od lewej strony powyżej osi OX.\\
Ponadto w punkcie $-6$ wykres odbija się od osi poziomej.\\
A więc $$x \in \{-6\} \cup [12,14].$$
\rozwStop
\odpStart
$x \in \{-6\} \cup [12,14]$
\odpStop
\testStart
A.$x \in \{-6\} \cup [12,14]$\\
B.$x \in \{6\} \cup (12,14)$\\
C.$x \in \{-6\} \cup (12,14]$\\
D.$x \in \{6\} \cup (12,14]$\\
E.$x \in \{-6\} \cup [12,14)$\\
F.$x \in \{6\} \cup [12,14)$\\
G.$x \in \{-6\} \cup (12,14)$\\
H.$x \in \{6\} \cup [12,14]$
\testStop
\kluczStart
A
\kluczStop



\zadStart{Zadanie z Wikieł Z 1.62 c) moja wersja nr 703}

Rozwiązać nierówności $(12-x)(x+6)^{2}(15-x)^{3}\le0$.
\zadStop
\rozwStart{Patryk Wirkus}{}
Miejsca zerowe naszego wielomianu to: $12, -6, 15$.\\
Wielomian jest stopnia parzystego, ponadto znak współczynnika przy\linebreak najwyższej potędze x jest ujemny.\\ W związku z tym wykres wielomianu zaczyna się od lewej strony powyżej osi OX.\\
Ponadto w punkcie $-6$ wykres odbija się od osi poziomej.\\
A więc $$x \in \{-6\} \cup [12,15].$$
\rozwStop
\odpStart
$x \in \{-6\} \cup [12,15]$
\odpStop
\testStart
A.$x \in \{-6\} \cup [12,15]$\\
B.$x \in \{6\} \cup (12,15)$\\
C.$x \in \{-6\} \cup (12,15]$\\
D.$x \in \{6\} \cup (12,15]$\\
E.$x \in \{-6\} \cup [12,15)$\\
F.$x \in \{6\} \cup [12,15)$\\
G.$x \in \{-6\} \cup (12,15)$\\
H.$x \in \{6\} \cup [12,15]$
\testStop
\kluczStart
A
\kluczStop



\zadStart{Zadanie z Wikieł Z 1.62 c) moja wersja nr 704}

Rozwiązać nierówności $(12-x)(x+6)^{2}(16-x)^{3}\le0$.
\zadStop
\rozwStart{Patryk Wirkus}{}
Miejsca zerowe naszego wielomianu to: $12, -6, 16$.\\
Wielomian jest stopnia parzystego, ponadto znak współczynnika przy\linebreak najwyższej potędze x jest ujemny.\\ W związku z tym wykres wielomianu zaczyna się od lewej strony powyżej osi OX.\\
Ponadto w punkcie $-6$ wykres odbija się od osi poziomej.\\
A więc $$x \in \{-6\} \cup [12,16].$$
\rozwStop
\odpStart
$x \in \{-6\} \cup [12,16]$
\odpStop
\testStart
A.$x \in \{-6\} \cup [12,16]$\\
B.$x \in \{6\} \cup (12,16)$\\
C.$x \in \{-6\} \cup (12,16]$\\
D.$x \in \{6\} \cup (12,16]$\\
E.$x \in \{-6\} \cup [12,16)$\\
F.$x \in \{6\} \cup [12,16)$\\
G.$x \in \{-6\} \cup (12,16)$\\
H.$x \in \{6\} \cup [12,16]$
\testStop
\kluczStart
A
\kluczStop



\zadStart{Zadanie z Wikieł Z 1.62 c) moja wersja nr 705}

Rozwiązać nierówności $(12-x)(x+6)^{2}(17-x)^{3}\le0$.
\zadStop
\rozwStart{Patryk Wirkus}{}
Miejsca zerowe naszego wielomianu to: $12, -6, 17$.\\
Wielomian jest stopnia parzystego, ponadto znak współczynnika przy\linebreak najwyższej potędze x jest ujemny.\\ W związku z tym wykres wielomianu zaczyna się od lewej strony powyżej osi OX.\\
Ponadto w punkcie $-6$ wykres odbija się od osi poziomej.\\
A więc $$x \in \{-6\} \cup [12,17].$$
\rozwStop
\odpStart
$x \in \{-6\} \cup [12,17]$
\odpStop
\testStart
A.$x \in \{-6\} \cup [12,17]$\\
B.$x \in \{6\} \cup (12,17)$\\
C.$x \in \{-6\} \cup (12,17]$\\
D.$x \in \{6\} \cup (12,17]$\\
E.$x \in \{-6\} \cup [12,17)$\\
F.$x \in \{6\} \cup [12,17)$\\
G.$x \in \{-6\} \cup (12,17)$\\
H.$x \in \{6\} \cup [12,17]$
\testStop
\kluczStart
A
\kluczStop



\zadStart{Zadanie z Wikieł Z 1.62 c) moja wersja nr 706}

Rozwiązać nierówności $(12-x)(x+6)^{2}(18-x)^{3}\le0$.
\zadStop
\rozwStart{Patryk Wirkus}{}
Miejsca zerowe naszego wielomianu to: $12, -6, 18$.\\
Wielomian jest stopnia parzystego, ponadto znak współczynnika przy\linebreak najwyższej potędze x jest ujemny.\\ W związku z tym wykres wielomianu zaczyna się od lewej strony powyżej osi OX.\\
Ponadto w punkcie $-6$ wykres odbija się od osi poziomej.\\
A więc $$x \in \{-6\} \cup [12,18].$$
\rozwStop
\odpStart
$x \in \{-6\} \cup [12,18]$
\odpStop
\testStart
A.$x \in \{-6\} \cup [12,18]$\\
B.$x \in \{6\} \cup (12,18)$\\
C.$x \in \{-6\} \cup (12,18]$\\
D.$x \in \{6\} \cup (12,18]$\\
E.$x \in \{-6\} \cup [12,18)$\\
F.$x \in \{6\} \cup [12,18)$\\
G.$x \in \{-6\} \cup (12,18)$\\
H.$x \in \{6\} \cup [12,18]$
\testStop
\kluczStart
A
\kluczStop



\zadStart{Zadanie z Wikieł Z 1.62 c) moja wersja nr 707}

Rozwiązać nierówności $(12-x)(x+6)^{2}(19-x)^{3}\le0$.
\zadStop
\rozwStart{Patryk Wirkus}{}
Miejsca zerowe naszego wielomianu to: $12, -6, 19$.\\
Wielomian jest stopnia parzystego, ponadto znak współczynnika przy\linebreak najwyższej potędze x jest ujemny.\\ W związku z tym wykres wielomianu zaczyna się od lewej strony powyżej osi OX.\\
Ponadto w punkcie $-6$ wykres odbija się od osi poziomej.\\
A więc $$x \in \{-6\} \cup [12,19].$$
\rozwStop
\odpStart
$x \in \{-6\} \cup [12,19]$
\odpStop
\testStart
A.$x \in \{-6\} \cup [12,19]$\\
B.$x \in \{6\} \cup (12,19)$\\
C.$x \in \{-6\} \cup (12,19]$\\
D.$x \in \{6\} \cup (12,19]$\\
E.$x \in \{-6\} \cup [12,19)$\\
F.$x \in \{6\} \cup [12,19)$\\
G.$x \in \{-6\} \cup (12,19)$\\
H.$x \in \{6\} \cup [12,19]$
\testStop
\kluczStart
A
\kluczStop



\zadStart{Zadanie z Wikieł Z 1.62 c) moja wersja nr 708}

Rozwiązać nierówności $(12-x)(x+6)^{2}(20-x)^{3}\le0$.
\zadStop
\rozwStart{Patryk Wirkus}{}
Miejsca zerowe naszego wielomianu to: $12, -6, 20$.\\
Wielomian jest stopnia parzystego, ponadto znak współczynnika przy\linebreak najwyższej potędze x jest ujemny.\\ W związku z tym wykres wielomianu zaczyna się od lewej strony powyżej osi OX.\\
Ponadto w punkcie $-6$ wykres odbija się od osi poziomej.\\
A więc $$x \in \{-6\} \cup [12,20].$$
\rozwStop
\odpStart
$x \in \{-6\} \cup [12,20]$
\odpStop
\testStart
A.$x \in \{-6\} \cup [12,20]$\\
B.$x \in \{6\} \cup (12,20)$\\
C.$x \in \{-6\} \cup (12,20]$\\
D.$x \in \{6\} \cup (12,20]$\\
E.$x \in \{-6\} \cup [12,20)$\\
F.$x \in \{6\} \cup [12,20)$\\
G.$x \in \{-6\} \cup (12,20)$\\
H.$x \in \{6\} \cup [12,20]$
\testStop
\kluczStart
A
\kluczStop



\zadStart{Zadanie z Wikieł Z 1.62 c) moja wersja nr 709}

Rozwiązać nierówności $(12-x)(x+7)^{2}(13-x)^{3}\le0$.
\zadStop
\rozwStart{Patryk Wirkus}{}
Miejsca zerowe naszego wielomianu to: $12, -7, 13$.\\
Wielomian jest stopnia parzystego, ponadto znak współczynnika przy\linebreak najwyższej potędze x jest ujemny.\\ W związku z tym wykres wielomianu zaczyna się od lewej strony powyżej osi OX.\\
Ponadto w punkcie $-7$ wykres odbija się od osi poziomej.\\
A więc $$x \in \{-7\} \cup [12,13].$$
\rozwStop
\odpStart
$x \in \{-7\} \cup [12,13]$
\odpStop
\testStart
A.$x \in \{-7\} \cup [12,13]$\\
B.$x \in \{7\} \cup (12,13)$\\
C.$x \in \{-7\} \cup (12,13]$\\
D.$x \in \{7\} \cup (12,13]$\\
E.$x \in \{-7\} \cup [12,13)$\\
F.$x \in \{7\} \cup [12,13)$\\
G.$x \in \{-7\} \cup (12,13)$\\
H.$x \in \{7\} \cup [12,13]$
\testStop
\kluczStart
A
\kluczStop



\zadStart{Zadanie z Wikieł Z 1.62 c) moja wersja nr 710}

Rozwiązać nierówności $(12-x)(x+7)^{2}(14-x)^{3}\le0$.
\zadStop
\rozwStart{Patryk Wirkus}{}
Miejsca zerowe naszego wielomianu to: $12, -7, 14$.\\
Wielomian jest stopnia parzystego, ponadto znak współczynnika przy\linebreak najwyższej potędze x jest ujemny.\\ W związku z tym wykres wielomianu zaczyna się od lewej strony powyżej osi OX.\\
Ponadto w punkcie $-7$ wykres odbija się od osi poziomej.\\
A więc $$x \in \{-7\} \cup [12,14].$$
\rozwStop
\odpStart
$x \in \{-7\} \cup [12,14]$
\odpStop
\testStart
A.$x \in \{-7\} \cup [12,14]$\\
B.$x \in \{7\} \cup (12,14)$\\
C.$x \in \{-7\} \cup (12,14]$\\
D.$x \in \{7\} \cup (12,14]$\\
E.$x \in \{-7\} \cup [12,14)$\\
F.$x \in \{7\} \cup [12,14)$\\
G.$x \in \{-7\} \cup (12,14)$\\
H.$x \in \{7\} \cup [12,14]$
\testStop
\kluczStart
A
\kluczStop



\zadStart{Zadanie z Wikieł Z 1.62 c) moja wersja nr 711}

Rozwiązać nierówności $(12-x)(x+7)^{2}(15-x)^{3}\le0$.
\zadStop
\rozwStart{Patryk Wirkus}{}
Miejsca zerowe naszego wielomianu to: $12, -7, 15$.\\
Wielomian jest stopnia parzystego, ponadto znak współczynnika przy\linebreak najwyższej potędze x jest ujemny.\\ W związku z tym wykres wielomianu zaczyna się od lewej strony powyżej osi OX.\\
Ponadto w punkcie $-7$ wykres odbija się od osi poziomej.\\
A więc $$x \in \{-7\} \cup [12,15].$$
\rozwStop
\odpStart
$x \in \{-7\} \cup [12,15]$
\odpStop
\testStart
A.$x \in \{-7\} \cup [12,15]$\\
B.$x \in \{7\} \cup (12,15)$\\
C.$x \in \{-7\} \cup (12,15]$\\
D.$x \in \{7\} \cup (12,15]$\\
E.$x \in \{-7\} \cup [12,15)$\\
F.$x \in \{7\} \cup [12,15)$\\
G.$x \in \{-7\} \cup (12,15)$\\
H.$x \in \{7\} \cup [12,15]$
\testStop
\kluczStart
A
\kluczStop



\zadStart{Zadanie z Wikieł Z 1.62 c) moja wersja nr 712}

Rozwiązać nierówności $(12-x)(x+7)^{2}(16-x)^{3}\le0$.
\zadStop
\rozwStart{Patryk Wirkus}{}
Miejsca zerowe naszego wielomianu to: $12, -7, 16$.\\
Wielomian jest stopnia parzystego, ponadto znak współczynnika przy\linebreak najwyższej potędze x jest ujemny.\\ W związku z tym wykres wielomianu zaczyna się od lewej strony powyżej osi OX.\\
Ponadto w punkcie $-7$ wykres odbija się od osi poziomej.\\
A więc $$x \in \{-7\} \cup [12,16].$$
\rozwStop
\odpStart
$x \in \{-7\} \cup [12,16]$
\odpStop
\testStart
A.$x \in \{-7\} \cup [12,16]$\\
B.$x \in \{7\} \cup (12,16)$\\
C.$x \in \{-7\} \cup (12,16]$\\
D.$x \in \{7\} \cup (12,16]$\\
E.$x \in \{-7\} \cup [12,16)$\\
F.$x \in \{7\} \cup [12,16)$\\
G.$x \in \{-7\} \cup (12,16)$\\
H.$x \in \{7\} \cup [12,16]$
\testStop
\kluczStart
A
\kluczStop



\zadStart{Zadanie z Wikieł Z 1.62 c) moja wersja nr 713}

Rozwiązać nierówności $(12-x)(x+7)^{2}(17-x)^{3}\le0$.
\zadStop
\rozwStart{Patryk Wirkus}{}
Miejsca zerowe naszego wielomianu to: $12, -7, 17$.\\
Wielomian jest stopnia parzystego, ponadto znak współczynnika przy\linebreak najwyższej potędze x jest ujemny.\\ W związku z tym wykres wielomianu zaczyna się od lewej strony powyżej osi OX.\\
Ponadto w punkcie $-7$ wykres odbija się od osi poziomej.\\
A więc $$x \in \{-7\} \cup [12,17].$$
\rozwStop
\odpStart
$x \in \{-7\} \cup [12,17]$
\odpStop
\testStart
A.$x \in \{-7\} \cup [12,17]$\\
B.$x \in \{7\} \cup (12,17)$\\
C.$x \in \{-7\} \cup (12,17]$\\
D.$x \in \{7\} \cup (12,17]$\\
E.$x \in \{-7\} \cup [12,17)$\\
F.$x \in \{7\} \cup [12,17)$\\
G.$x \in \{-7\} \cup (12,17)$\\
H.$x \in \{7\} \cup [12,17]$
\testStop
\kluczStart
A
\kluczStop



\zadStart{Zadanie z Wikieł Z 1.62 c) moja wersja nr 714}

Rozwiązać nierówności $(12-x)(x+7)^{2}(18-x)^{3}\le0$.
\zadStop
\rozwStart{Patryk Wirkus}{}
Miejsca zerowe naszego wielomianu to: $12, -7, 18$.\\
Wielomian jest stopnia parzystego, ponadto znak współczynnika przy\linebreak najwyższej potędze x jest ujemny.\\ W związku z tym wykres wielomianu zaczyna się od lewej strony powyżej osi OX.\\
Ponadto w punkcie $-7$ wykres odbija się od osi poziomej.\\
A więc $$x \in \{-7\} \cup [12,18].$$
\rozwStop
\odpStart
$x \in \{-7\} \cup [12,18]$
\odpStop
\testStart
A.$x \in \{-7\} \cup [12,18]$\\
B.$x \in \{7\} \cup (12,18)$\\
C.$x \in \{-7\} \cup (12,18]$\\
D.$x \in \{7\} \cup (12,18]$\\
E.$x \in \{-7\} \cup [12,18)$\\
F.$x \in \{7\} \cup [12,18)$\\
G.$x \in \{-7\} \cup (12,18)$\\
H.$x \in \{7\} \cup [12,18]$
\testStop
\kluczStart
A
\kluczStop



\zadStart{Zadanie z Wikieł Z 1.62 c) moja wersja nr 715}

Rozwiązać nierówności $(12-x)(x+7)^{2}(19-x)^{3}\le0$.
\zadStop
\rozwStart{Patryk Wirkus}{}
Miejsca zerowe naszego wielomianu to: $12, -7, 19$.\\
Wielomian jest stopnia parzystego, ponadto znak współczynnika przy\linebreak najwyższej potędze x jest ujemny.\\ W związku z tym wykres wielomianu zaczyna się od lewej strony powyżej osi OX.\\
Ponadto w punkcie $-7$ wykres odbija się od osi poziomej.\\
A więc $$x \in \{-7\} \cup [12,19].$$
\rozwStop
\odpStart
$x \in \{-7\} \cup [12,19]$
\odpStop
\testStart
A.$x \in \{-7\} \cup [12,19]$\\
B.$x \in \{7\} \cup (12,19)$\\
C.$x \in \{-7\} \cup (12,19]$\\
D.$x \in \{7\} \cup (12,19]$\\
E.$x \in \{-7\} \cup [12,19)$\\
F.$x \in \{7\} \cup [12,19)$\\
G.$x \in \{-7\} \cup (12,19)$\\
H.$x \in \{7\} \cup [12,19]$
\testStop
\kluczStart
A
\kluczStop



\zadStart{Zadanie z Wikieł Z 1.62 c) moja wersja nr 716}

Rozwiązać nierówności $(12-x)(x+7)^{2}(20-x)^{3}\le0$.
\zadStop
\rozwStart{Patryk Wirkus}{}
Miejsca zerowe naszego wielomianu to: $12, -7, 20$.\\
Wielomian jest stopnia parzystego, ponadto znak współczynnika przy\linebreak najwyższej potędze x jest ujemny.\\ W związku z tym wykres wielomianu zaczyna się od lewej strony powyżej osi OX.\\
Ponadto w punkcie $-7$ wykres odbija się od osi poziomej.\\
A więc $$x \in \{-7\} \cup [12,20].$$
\rozwStop
\odpStart
$x \in \{-7\} \cup [12,20]$
\odpStop
\testStart
A.$x \in \{-7\} \cup [12,20]$\\
B.$x \in \{7\} \cup (12,20)$\\
C.$x \in \{-7\} \cup (12,20]$\\
D.$x \in \{7\} \cup (12,20]$\\
E.$x \in \{-7\} \cup [12,20)$\\
F.$x \in \{7\} \cup [12,20)$\\
G.$x \in \{-7\} \cup (12,20)$\\
H.$x \in \{7\} \cup [12,20]$
\testStop
\kluczStart
A
\kluczStop



\zadStart{Zadanie z Wikieł Z 1.62 c) moja wersja nr 717}

Rozwiązać nierówności $(12-x)(x+8)^{2}(13-x)^{3}\le0$.
\zadStop
\rozwStart{Patryk Wirkus}{}
Miejsca zerowe naszego wielomianu to: $12, -8, 13$.\\
Wielomian jest stopnia parzystego, ponadto znak współczynnika przy\linebreak najwyższej potędze x jest ujemny.\\ W związku z tym wykres wielomianu zaczyna się od lewej strony powyżej osi OX.\\
Ponadto w punkcie $-8$ wykres odbija się od osi poziomej.\\
A więc $$x \in \{-8\} \cup [12,13].$$
\rozwStop
\odpStart
$x \in \{-8\} \cup [12,13]$
\odpStop
\testStart
A.$x \in \{-8\} \cup [12,13]$\\
B.$x \in \{8\} \cup (12,13)$\\
C.$x \in \{-8\} \cup (12,13]$\\
D.$x \in \{8\} \cup (12,13]$\\
E.$x \in \{-8\} \cup [12,13)$\\
F.$x \in \{8\} \cup [12,13)$\\
G.$x \in \{-8\} \cup (12,13)$\\
H.$x \in \{8\} \cup [12,13]$
\testStop
\kluczStart
A
\kluczStop



\zadStart{Zadanie z Wikieł Z 1.62 c) moja wersja nr 718}

Rozwiązać nierówności $(12-x)(x+8)^{2}(14-x)^{3}\le0$.
\zadStop
\rozwStart{Patryk Wirkus}{}
Miejsca zerowe naszego wielomianu to: $12, -8, 14$.\\
Wielomian jest stopnia parzystego, ponadto znak współczynnika przy\linebreak najwyższej potędze x jest ujemny.\\ W związku z tym wykres wielomianu zaczyna się od lewej strony powyżej osi OX.\\
Ponadto w punkcie $-8$ wykres odbija się od osi poziomej.\\
A więc $$x \in \{-8\} \cup [12,14].$$
\rozwStop
\odpStart
$x \in \{-8\} \cup [12,14]$
\odpStop
\testStart
A.$x \in \{-8\} \cup [12,14]$\\
B.$x \in \{8\} \cup (12,14)$\\
C.$x \in \{-8\} \cup (12,14]$\\
D.$x \in \{8\} \cup (12,14]$\\
E.$x \in \{-8\} \cup [12,14)$\\
F.$x \in \{8\} \cup [12,14)$\\
G.$x \in \{-8\} \cup (12,14)$\\
H.$x \in \{8\} \cup [12,14]$
\testStop
\kluczStart
A
\kluczStop



\zadStart{Zadanie z Wikieł Z 1.62 c) moja wersja nr 719}

Rozwiązać nierówności $(12-x)(x+8)^{2}(15-x)^{3}\le0$.
\zadStop
\rozwStart{Patryk Wirkus}{}
Miejsca zerowe naszego wielomianu to: $12, -8, 15$.\\
Wielomian jest stopnia parzystego, ponadto znak współczynnika przy\linebreak najwyższej potędze x jest ujemny.\\ W związku z tym wykres wielomianu zaczyna się od lewej strony powyżej osi OX.\\
Ponadto w punkcie $-8$ wykres odbija się od osi poziomej.\\
A więc $$x \in \{-8\} \cup [12,15].$$
\rozwStop
\odpStart
$x \in \{-8\} \cup [12,15]$
\odpStop
\testStart
A.$x \in \{-8\} \cup [12,15]$\\
B.$x \in \{8\} \cup (12,15)$\\
C.$x \in \{-8\} \cup (12,15]$\\
D.$x \in \{8\} \cup (12,15]$\\
E.$x \in \{-8\} \cup [12,15)$\\
F.$x \in \{8\} \cup [12,15)$\\
G.$x \in \{-8\} \cup (12,15)$\\
H.$x \in \{8\} \cup [12,15]$
\testStop
\kluczStart
A
\kluczStop



\zadStart{Zadanie z Wikieł Z 1.62 c) moja wersja nr 720}

Rozwiązać nierówności $(12-x)(x+8)^{2}(16-x)^{3}\le0$.
\zadStop
\rozwStart{Patryk Wirkus}{}
Miejsca zerowe naszego wielomianu to: $12, -8, 16$.\\
Wielomian jest stopnia parzystego, ponadto znak współczynnika przy\linebreak najwyższej potędze x jest ujemny.\\ W związku z tym wykres wielomianu zaczyna się od lewej strony powyżej osi OX.\\
Ponadto w punkcie $-8$ wykres odbija się od osi poziomej.\\
A więc $$x \in \{-8\} \cup [12,16].$$
\rozwStop
\odpStart
$x \in \{-8\} \cup [12,16]$
\odpStop
\testStart
A.$x \in \{-8\} \cup [12,16]$\\
B.$x \in \{8\} \cup (12,16)$\\
C.$x \in \{-8\} \cup (12,16]$\\
D.$x \in \{8\} \cup (12,16]$\\
E.$x \in \{-8\} \cup [12,16)$\\
F.$x \in \{8\} \cup [12,16)$\\
G.$x \in \{-8\} \cup (12,16)$\\
H.$x \in \{8\} \cup [12,16]$
\testStop
\kluczStart
A
\kluczStop



\zadStart{Zadanie z Wikieł Z 1.62 c) moja wersja nr 721}

Rozwiązać nierówności $(12-x)(x+8)^{2}(17-x)^{3}\le0$.
\zadStop
\rozwStart{Patryk Wirkus}{}
Miejsca zerowe naszego wielomianu to: $12, -8, 17$.\\
Wielomian jest stopnia parzystego, ponadto znak współczynnika przy\linebreak najwyższej potędze x jest ujemny.\\ W związku z tym wykres wielomianu zaczyna się od lewej strony powyżej osi OX.\\
Ponadto w punkcie $-8$ wykres odbija się od osi poziomej.\\
A więc $$x \in \{-8\} \cup [12,17].$$
\rozwStop
\odpStart
$x \in \{-8\} \cup [12,17]$
\odpStop
\testStart
A.$x \in \{-8\} \cup [12,17]$\\
B.$x \in \{8\} \cup (12,17)$\\
C.$x \in \{-8\} \cup (12,17]$\\
D.$x \in \{8\} \cup (12,17]$\\
E.$x \in \{-8\} \cup [12,17)$\\
F.$x \in \{8\} \cup [12,17)$\\
G.$x \in \{-8\} \cup (12,17)$\\
H.$x \in \{8\} \cup [12,17]$
\testStop
\kluczStart
A
\kluczStop



\zadStart{Zadanie z Wikieł Z 1.62 c) moja wersja nr 722}

Rozwiązać nierówności $(12-x)(x+8)^{2}(18-x)^{3}\le0$.
\zadStop
\rozwStart{Patryk Wirkus}{}
Miejsca zerowe naszego wielomianu to: $12, -8, 18$.\\
Wielomian jest stopnia parzystego, ponadto znak współczynnika przy\linebreak najwyższej potędze x jest ujemny.\\ W związku z tym wykres wielomianu zaczyna się od lewej strony powyżej osi OX.\\
Ponadto w punkcie $-8$ wykres odbija się od osi poziomej.\\
A więc $$x \in \{-8\} \cup [12,18].$$
\rozwStop
\odpStart
$x \in \{-8\} \cup [12,18]$
\odpStop
\testStart
A.$x \in \{-8\} \cup [12,18]$\\
B.$x \in \{8\} \cup (12,18)$\\
C.$x \in \{-8\} \cup (12,18]$\\
D.$x \in \{8\} \cup (12,18]$\\
E.$x \in \{-8\} \cup [12,18)$\\
F.$x \in \{8\} \cup [12,18)$\\
G.$x \in \{-8\} \cup (12,18)$\\
H.$x \in \{8\} \cup [12,18]$
\testStop
\kluczStart
A
\kluczStop



\zadStart{Zadanie z Wikieł Z 1.62 c) moja wersja nr 723}

Rozwiązać nierówności $(12-x)(x+8)^{2}(19-x)^{3}\le0$.
\zadStop
\rozwStart{Patryk Wirkus}{}
Miejsca zerowe naszego wielomianu to: $12, -8, 19$.\\
Wielomian jest stopnia parzystego, ponadto znak współczynnika przy\linebreak najwyższej potędze x jest ujemny.\\ W związku z tym wykres wielomianu zaczyna się od lewej strony powyżej osi OX.\\
Ponadto w punkcie $-8$ wykres odbija się od osi poziomej.\\
A więc $$x \in \{-8\} \cup [12,19].$$
\rozwStop
\odpStart
$x \in \{-8\} \cup [12,19]$
\odpStop
\testStart
A.$x \in \{-8\} \cup [12,19]$\\
B.$x \in \{8\} \cup (12,19)$\\
C.$x \in \{-8\} \cup (12,19]$\\
D.$x \in \{8\} \cup (12,19]$\\
E.$x \in \{-8\} \cup [12,19)$\\
F.$x \in \{8\} \cup [12,19)$\\
G.$x \in \{-8\} \cup (12,19)$\\
H.$x \in \{8\} \cup [12,19]$
\testStop
\kluczStart
A
\kluczStop



\zadStart{Zadanie z Wikieł Z 1.62 c) moja wersja nr 724}

Rozwiązać nierówności $(12-x)(x+8)^{2}(20-x)^{3}\le0$.
\zadStop
\rozwStart{Patryk Wirkus}{}
Miejsca zerowe naszego wielomianu to: $12, -8, 20$.\\
Wielomian jest stopnia parzystego, ponadto znak współczynnika przy\linebreak najwyższej potędze x jest ujemny.\\ W związku z tym wykres wielomianu zaczyna się od lewej strony powyżej osi OX.\\
Ponadto w punkcie $-8$ wykres odbija się od osi poziomej.\\
A więc $$x \in \{-8\} \cup [12,20].$$
\rozwStop
\odpStart
$x \in \{-8\} \cup [12,20]$
\odpStop
\testStart
A.$x \in \{-8\} \cup [12,20]$\\
B.$x \in \{8\} \cup (12,20)$\\
C.$x \in \{-8\} \cup (12,20]$\\
D.$x \in \{8\} \cup (12,20]$\\
E.$x \in \{-8\} \cup [12,20)$\\
F.$x \in \{8\} \cup [12,20)$\\
G.$x \in \{-8\} \cup (12,20)$\\
H.$x \in \{8\} \cup [12,20]$
\testStop
\kluczStart
A
\kluczStop



\zadStart{Zadanie z Wikieł Z 1.62 c) moja wersja nr 725}

Rozwiązać nierówności $(12-x)(x+9)^{2}(13-x)^{3}\le0$.
\zadStop
\rozwStart{Patryk Wirkus}{}
Miejsca zerowe naszego wielomianu to: $12, -9, 13$.\\
Wielomian jest stopnia parzystego, ponadto znak współczynnika przy\linebreak najwyższej potędze x jest ujemny.\\ W związku z tym wykres wielomianu zaczyna się od lewej strony powyżej osi OX.\\
Ponadto w punkcie $-9$ wykres odbija się od osi poziomej.\\
A więc $$x \in \{-9\} \cup [12,13].$$
\rozwStop
\odpStart
$x \in \{-9\} \cup [12,13]$
\odpStop
\testStart
A.$x \in \{-9\} \cup [12,13]$\\
B.$x \in \{9\} \cup (12,13)$\\
C.$x \in \{-9\} \cup (12,13]$\\
D.$x \in \{9\} \cup (12,13]$\\
E.$x \in \{-9\} \cup [12,13)$\\
F.$x \in \{9\} \cup [12,13)$\\
G.$x \in \{-9\} \cup (12,13)$\\
H.$x \in \{9\} \cup [12,13]$
\testStop
\kluczStart
A
\kluczStop



\zadStart{Zadanie z Wikieł Z 1.62 c) moja wersja nr 726}

Rozwiązać nierówności $(12-x)(x+9)^{2}(14-x)^{3}\le0$.
\zadStop
\rozwStart{Patryk Wirkus}{}
Miejsca zerowe naszego wielomianu to: $12, -9, 14$.\\
Wielomian jest stopnia parzystego, ponadto znak współczynnika przy\linebreak najwyższej potędze x jest ujemny.\\ W związku z tym wykres wielomianu zaczyna się od lewej strony powyżej osi OX.\\
Ponadto w punkcie $-9$ wykres odbija się od osi poziomej.\\
A więc $$x \in \{-9\} \cup [12,14].$$
\rozwStop
\odpStart
$x \in \{-9\} \cup [12,14]$
\odpStop
\testStart
A.$x \in \{-9\} \cup [12,14]$\\
B.$x \in \{9\} \cup (12,14)$\\
C.$x \in \{-9\} \cup (12,14]$\\
D.$x \in \{9\} \cup (12,14]$\\
E.$x \in \{-9\} \cup [12,14)$\\
F.$x \in \{9\} \cup [12,14)$\\
G.$x \in \{-9\} \cup (12,14)$\\
H.$x \in \{9\} \cup [12,14]$
\testStop
\kluczStart
A
\kluczStop



\zadStart{Zadanie z Wikieł Z 1.62 c) moja wersja nr 727}

Rozwiązać nierówności $(12-x)(x+9)^{2}(15-x)^{3}\le0$.
\zadStop
\rozwStart{Patryk Wirkus}{}
Miejsca zerowe naszego wielomianu to: $12, -9, 15$.\\
Wielomian jest stopnia parzystego, ponadto znak współczynnika przy\linebreak najwyższej potędze x jest ujemny.\\ W związku z tym wykres wielomianu zaczyna się od lewej strony powyżej osi OX.\\
Ponadto w punkcie $-9$ wykres odbija się od osi poziomej.\\
A więc $$x \in \{-9\} \cup [12,15].$$
\rozwStop
\odpStart
$x \in \{-9\} \cup [12,15]$
\odpStop
\testStart
A.$x \in \{-9\} \cup [12,15]$\\
B.$x \in \{9\} \cup (12,15)$\\
C.$x \in \{-9\} \cup (12,15]$\\
D.$x \in \{9\} \cup (12,15]$\\
E.$x \in \{-9\} \cup [12,15)$\\
F.$x \in \{9\} \cup [12,15)$\\
G.$x \in \{-9\} \cup (12,15)$\\
H.$x \in \{9\} \cup [12,15]$
\testStop
\kluczStart
A
\kluczStop



\zadStart{Zadanie z Wikieł Z 1.62 c) moja wersja nr 728}

Rozwiązać nierówności $(12-x)(x+9)^{2}(16-x)^{3}\le0$.
\zadStop
\rozwStart{Patryk Wirkus}{}
Miejsca zerowe naszego wielomianu to: $12, -9, 16$.\\
Wielomian jest stopnia parzystego, ponadto znak współczynnika przy\linebreak najwyższej potędze x jest ujemny.\\ W związku z tym wykres wielomianu zaczyna się od lewej strony powyżej osi OX.\\
Ponadto w punkcie $-9$ wykres odbija się od osi poziomej.\\
A więc $$x \in \{-9\} \cup [12,16].$$
\rozwStop
\odpStart
$x \in \{-9\} \cup [12,16]$
\odpStop
\testStart
A.$x \in \{-9\} \cup [12,16]$\\
B.$x \in \{9\} \cup (12,16)$\\
C.$x \in \{-9\} \cup (12,16]$\\
D.$x \in \{9\} \cup (12,16]$\\
E.$x \in \{-9\} \cup [12,16)$\\
F.$x \in \{9\} \cup [12,16)$\\
G.$x \in \{-9\} \cup (12,16)$\\
H.$x \in \{9\} \cup [12,16]$
\testStop
\kluczStart
A
\kluczStop



\zadStart{Zadanie z Wikieł Z 1.62 c) moja wersja nr 729}

Rozwiązać nierówności $(12-x)(x+9)^{2}(17-x)^{3}\le0$.
\zadStop
\rozwStart{Patryk Wirkus}{}
Miejsca zerowe naszego wielomianu to: $12, -9, 17$.\\
Wielomian jest stopnia parzystego, ponadto znak współczynnika przy\linebreak najwyższej potędze x jest ujemny.\\ W związku z tym wykres wielomianu zaczyna się od lewej strony powyżej osi OX.\\
Ponadto w punkcie $-9$ wykres odbija się od osi poziomej.\\
A więc $$x \in \{-9\} \cup [12,17].$$
\rozwStop
\odpStart
$x \in \{-9\} \cup [12,17]$
\odpStop
\testStart
A.$x \in \{-9\} \cup [12,17]$\\
B.$x \in \{9\} \cup (12,17)$\\
C.$x \in \{-9\} \cup (12,17]$\\
D.$x \in \{9\} \cup (12,17]$\\
E.$x \in \{-9\} \cup [12,17)$\\
F.$x \in \{9\} \cup [12,17)$\\
G.$x \in \{-9\} \cup (12,17)$\\
H.$x \in \{9\} \cup [12,17]$
\testStop
\kluczStart
A
\kluczStop



\zadStart{Zadanie z Wikieł Z 1.62 c) moja wersja nr 730}

Rozwiązać nierówności $(12-x)(x+9)^{2}(18-x)^{3}\le0$.
\zadStop
\rozwStart{Patryk Wirkus}{}
Miejsca zerowe naszego wielomianu to: $12, -9, 18$.\\
Wielomian jest stopnia parzystego, ponadto znak współczynnika przy\linebreak najwyższej potędze x jest ujemny.\\ W związku z tym wykres wielomianu zaczyna się od lewej strony powyżej osi OX.\\
Ponadto w punkcie $-9$ wykres odbija się od osi poziomej.\\
A więc $$x \in \{-9\} \cup [12,18].$$
\rozwStop
\odpStart
$x \in \{-9\} \cup [12,18]$
\odpStop
\testStart
A.$x \in \{-9\} \cup [12,18]$\\
B.$x \in \{9\} \cup (12,18)$\\
C.$x \in \{-9\} \cup (12,18]$\\
D.$x \in \{9\} \cup (12,18]$\\
E.$x \in \{-9\} \cup [12,18)$\\
F.$x \in \{9\} \cup [12,18)$\\
G.$x \in \{-9\} \cup (12,18)$\\
H.$x \in \{9\} \cup [12,18]$
\testStop
\kluczStart
A
\kluczStop



\zadStart{Zadanie z Wikieł Z 1.62 c) moja wersja nr 731}

Rozwiązać nierówności $(12-x)(x+9)^{2}(19-x)^{3}\le0$.
\zadStop
\rozwStart{Patryk Wirkus}{}
Miejsca zerowe naszego wielomianu to: $12, -9, 19$.\\
Wielomian jest stopnia parzystego, ponadto znak współczynnika przy\linebreak najwyższej potędze x jest ujemny.\\ W związku z tym wykres wielomianu zaczyna się od lewej strony powyżej osi OX.\\
Ponadto w punkcie $-9$ wykres odbija się od osi poziomej.\\
A więc $$x \in \{-9\} \cup [12,19].$$
\rozwStop
\odpStart
$x \in \{-9\} \cup [12,19]$
\odpStop
\testStart
A.$x \in \{-9\} \cup [12,19]$\\
B.$x \in \{9\} \cup (12,19)$\\
C.$x \in \{-9\} \cup (12,19]$\\
D.$x \in \{9\} \cup (12,19]$\\
E.$x \in \{-9\} \cup [12,19)$\\
F.$x \in \{9\} \cup [12,19)$\\
G.$x \in \{-9\} \cup (12,19)$\\
H.$x \in \{9\} \cup [12,19]$
\testStop
\kluczStart
A
\kluczStop



\zadStart{Zadanie z Wikieł Z 1.62 c) moja wersja nr 732}

Rozwiązać nierówności $(12-x)(x+9)^{2}(20-x)^{3}\le0$.
\zadStop
\rozwStart{Patryk Wirkus}{}
Miejsca zerowe naszego wielomianu to: $12, -9, 20$.\\
Wielomian jest stopnia parzystego, ponadto znak współczynnika przy\linebreak najwyższej potędze x jest ujemny.\\ W związku z tym wykres wielomianu zaczyna się od lewej strony powyżej osi OX.\\
Ponadto w punkcie $-9$ wykres odbija się od osi poziomej.\\
A więc $$x \in \{-9\} \cup [12,20].$$
\rozwStop
\odpStart
$x \in \{-9\} \cup [12,20]$
\odpStop
\testStart
A.$x \in \{-9\} \cup [12,20]$\\
B.$x \in \{9\} \cup (12,20)$\\
C.$x \in \{-9\} \cup (12,20]$\\
D.$x \in \{9\} \cup (12,20]$\\
E.$x \in \{-9\} \cup [12,20)$\\
F.$x \in \{9\} \cup [12,20)$\\
G.$x \in \{-9\} \cup (12,20)$\\
H.$x \in \{9\} \cup [12,20]$
\testStop
\kluczStart
A
\kluczStop



\zadStart{Zadanie z Wikieł Z 1.62 c) moja wersja nr 733}

Rozwiązać nierówności $(12-x)(x+10)^{2}(13-x)^{3}\le0$.
\zadStop
\rozwStart{Patryk Wirkus}{}
Miejsca zerowe naszego wielomianu to: $12, -10, 13$.\\
Wielomian jest stopnia parzystego, ponadto znak współczynnika przy\linebreak najwyższej potędze x jest ujemny.\\ W związku z tym wykres wielomianu zaczyna się od lewej strony powyżej osi OX.\\
Ponadto w punkcie $-10$ wykres odbija się od osi poziomej.\\
A więc $$x \in \{-10\} \cup [12,13].$$
\rozwStop
\odpStart
$x \in \{-10\} \cup [12,13]$
\odpStop
\testStart
A.$x \in \{-10\} \cup [12,13]$\\
B.$x \in \{10\} \cup (12,13)$\\
C.$x \in \{-10\} \cup (12,13]$\\
D.$x \in \{10\} \cup (12,13]$\\
E.$x \in \{-10\} \cup [12,13)$\\
F.$x \in \{10\} \cup [12,13)$\\
G.$x \in \{-10\} \cup (12,13)$\\
H.$x \in \{10\} \cup [12,13]$
\testStop
\kluczStart
A
\kluczStop



\zadStart{Zadanie z Wikieł Z 1.62 c) moja wersja nr 734}

Rozwiązać nierówności $(12-x)(x+10)^{2}(14-x)^{3}\le0$.
\zadStop
\rozwStart{Patryk Wirkus}{}
Miejsca zerowe naszego wielomianu to: $12, -10, 14$.\\
Wielomian jest stopnia parzystego, ponadto znak współczynnika przy\linebreak najwyższej potędze x jest ujemny.\\ W związku z tym wykres wielomianu zaczyna się od lewej strony powyżej osi OX.\\
Ponadto w punkcie $-10$ wykres odbija się od osi poziomej.\\
A więc $$x \in \{-10\} \cup [12,14].$$
\rozwStop
\odpStart
$x \in \{-10\} \cup [12,14]$
\odpStop
\testStart
A.$x \in \{-10\} \cup [12,14]$\\
B.$x \in \{10\} \cup (12,14)$\\
C.$x \in \{-10\} \cup (12,14]$\\
D.$x \in \{10\} \cup (12,14]$\\
E.$x \in \{-10\} \cup [12,14)$\\
F.$x \in \{10\} \cup [12,14)$\\
G.$x \in \{-10\} \cup (12,14)$\\
H.$x \in \{10\} \cup [12,14]$
\testStop
\kluczStart
A
\kluczStop



\zadStart{Zadanie z Wikieł Z 1.62 c) moja wersja nr 735}

Rozwiązać nierówności $(12-x)(x+10)^{2}(15-x)^{3}\le0$.
\zadStop
\rozwStart{Patryk Wirkus}{}
Miejsca zerowe naszego wielomianu to: $12, -10, 15$.\\
Wielomian jest stopnia parzystego, ponadto znak współczynnika przy\linebreak najwyższej potędze x jest ujemny.\\ W związku z tym wykres wielomianu zaczyna się od lewej strony powyżej osi OX.\\
Ponadto w punkcie $-10$ wykres odbija się od osi poziomej.\\
A więc $$x \in \{-10\} \cup [12,15].$$
\rozwStop
\odpStart
$x \in \{-10\} \cup [12,15]$
\odpStop
\testStart
A.$x \in \{-10\} \cup [12,15]$\\
B.$x \in \{10\} \cup (12,15)$\\
C.$x \in \{-10\} \cup (12,15]$\\
D.$x \in \{10\} \cup (12,15]$\\
E.$x \in \{-10\} \cup [12,15)$\\
F.$x \in \{10\} \cup [12,15)$\\
G.$x \in \{-10\} \cup (12,15)$\\
H.$x \in \{10\} \cup [12,15]$
\testStop
\kluczStart
A
\kluczStop



\zadStart{Zadanie z Wikieł Z 1.62 c) moja wersja nr 736}

Rozwiązać nierówności $(12-x)(x+10)^{2}(16-x)^{3}\le0$.
\zadStop
\rozwStart{Patryk Wirkus}{}
Miejsca zerowe naszego wielomianu to: $12, -10, 16$.\\
Wielomian jest stopnia parzystego, ponadto znak współczynnika przy\linebreak najwyższej potędze x jest ujemny.\\ W związku z tym wykres wielomianu zaczyna się od lewej strony powyżej osi OX.\\
Ponadto w punkcie $-10$ wykres odbija się od osi poziomej.\\
A więc $$x \in \{-10\} \cup [12,16].$$
\rozwStop
\odpStart
$x \in \{-10\} \cup [12,16]$
\odpStop
\testStart
A.$x \in \{-10\} \cup [12,16]$\\
B.$x \in \{10\} \cup (12,16)$\\
C.$x \in \{-10\} \cup (12,16]$\\
D.$x \in \{10\} \cup (12,16]$\\
E.$x \in \{-10\} \cup [12,16)$\\
F.$x \in \{10\} \cup [12,16)$\\
G.$x \in \{-10\} \cup (12,16)$\\
H.$x \in \{10\} \cup [12,16]$
\testStop
\kluczStart
A
\kluczStop



\zadStart{Zadanie z Wikieł Z 1.62 c) moja wersja nr 737}

Rozwiązać nierówności $(12-x)(x+10)^{2}(17-x)^{3}\le0$.
\zadStop
\rozwStart{Patryk Wirkus}{}
Miejsca zerowe naszego wielomianu to: $12, -10, 17$.\\
Wielomian jest stopnia parzystego, ponadto znak współczynnika przy\linebreak najwyższej potędze x jest ujemny.\\ W związku z tym wykres wielomianu zaczyna się od lewej strony powyżej osi OX.\\
Ponadto w punkcie $-10$ wykres odbija się od osi poziomej.\\
A więc $$x \in \{-10\} \cup [12,17].$$
\rozwStop
\odpStart
$x \in \{-10\} \cup [12,17]$
\odpStop
\testStart
A.$x \in \{-10\} \cup [12,17]$\\
B.$x \in \{10\} \cup (12,17)$\\
C.$x \in \{-10\} \cup (12,17]$\\
D.$x \in \{10\} \cup (12,17]$\\
E.$x \in \{-10\} \cup [12,17)$\\
F.$x \in \{10\} \cup [12,17)$\\
G.$x \in \{-10\} \cup (12,17)$\\
H.$x \in \{10\} \cup [12,17]$
\testStop
\kluczStart
A
\kluczStop



\zadStart{Zadanie z Wikieł Z 1.62 c) moja wersja nr 738}

Rozwiązać nierówności $(12-x)(x+10)^{2}(18-x)^{3}\le0$.
\zadStop
\rozwStart{Patryk Wirkus}{}
Miejsca zerowe naszego wielomianu to: $12, -10, 18$.\\
Wielomian jest stopnia parzystego, ponadto znak współczynnika przy\linebreak najwyższej potędze x jest ujemny.\\ W związku z tym wykres wielomianu zaczyna się od lewej strony powyżej osi OX.\\
Ponadto w punkcie $-10$ wykres odbija się od osi poziomej.\\
A więc $$x \in \{-10\} \cup [12,18].$$
\rozwStop
\odpStart
$x \in \{-10\} \cup [12,18]$
\odpStop
\testStart
A.$x \in \{-10\} \cup [12,18]$\\
B.$x \in \{10\} \cup (12,18)$\\
C.$x \in \{-10\} \cup (12,18]$\\
D.$x \in \{10\} \cup (12,18]$\\
E.$x \in \{-10\} \cup [12,18)$\\
F.$x \in \{10\} \cup [12,18)$\\
G.$x \in \{-10\} \cup (12,18)$\\
H.$x \in \{10\} \cup [12,18]$
\testStop
\kluczStart
A
\kluczStop



\zadStart{Zadanie z Wikieł Z 1.62 c) moja wersja nr 739}

Rozwiązać nierówności $(12-x)(x+10)^{2}(19-x)^{3}\le0$.
\zadStop
\rozwStart{Patryk Wirkus}{}
Miejsca zerowe naszego wielomianu to: $12, -10, 19$.\\
Wielomian jest stopnia parzystego, ponadto znak współczynnika przy\linebreak najwyższej potędze x jest ujemny.\\ W związku z tym wykres wielomianu zaczyna się od lewej strony powyżej osi OX.\\
Ponadto w punkcie $-10$ wykres odbija się od osi poziomej.\\
A więc $$x \in \{-10\} \cup [12,19].$$
\rozwStop
\odpStart
$x \in \{-10\} \cup [12,19]$
\odpStop
\testStart
A.$x \in \{-10\} \cup [12,19]$\\
B.$x \in \{10\} \cup (12,19)$\\
C.$x \in \{-10\} \cup (12,19]$\\
D.$x \in \{10\} \cup (12,19]$\\
E.$x \in \{-10\} \cup [12,19)$\\
F.$x \in \{10\} \cup [12,19)$\\
G.$x \in \{-10\} \cup (12,19)$\\
H.$x \in \{10\} \cup [12,19]$
\testStop
\kluczStart
A
\kluczStop



\zadStart{Zadanie z Wikieł Z 1.62 c) moja wersja nr 740}

Rozwiązać nierówności $(12-x)(x+10)^{2}(20-x)^{3}\le0$.
\zadStop
\rozwStart{Patryk Wirkus}{}
Miejsca zerowe naszego wielomianu to: $12, -10, 20$.\\
Wielomian jest stopnia parzystego, ponadto znak współczynnika przy\linebreak najwyższej potędze x jest ujemny.\\ W związku z tym wykres wielomianu zaczyna się od lewej strony powyżej osi OX.\\
Ponadto w punkcie $-10$ wykres odbija się od osi poziomej.\\
A więc $$x \in \{-10\} \cup [12,20].$$
\rozwStop
\odpStart
$x \in \{-10\} \cup [12,20]$
\odpStop
\testStart
A.$x \in \{-10\} \cup [12,20]$\\
B.$x \in \{10\} \cup (12,20)$\\
C.$x \in \{-10\} \cup (12,20]$\\
D.$x \in \{10\} \cup (12,20]$\\
E.$x \in \{-10\} \cup [12,20)$\\
F.$x \in \{10\} \cup [12,20)$\\
G.$x \in \{-10\} \cup (12,20)$\\
H.$x \in \{10\} \cup [12,20]$
\testStop
\kluczStart
A
\kluczStop



\zadStart{Zadanie z Wikieł Z 1.62 c) moja wersja nr 741}

Rozwiązać nierówności $(12-x)(x+11)^{2}(13-x)^{3}\le0$.
\zadStop
\rozwStart{Patryk Wirkus}{}
Miejsca zerowe naszego wielomianu to: $12, -11, 13$.\\
Wielomian jest stopnia parzystego, ponadto znak współczynnika przy\linebreak najwyższej potędze x jest ujemny.\\ W związku z tym wykres wielomianu zaczyna się od lewej strony powyżej osi OX.\\
Ponadto w punkcie $-11$ wykres odbija się od osi poziomej.\\
A więc $$x \in \{-11\} \cup [12,13].$$
\rozwStop
\odpStart
$x \in \{-11\} \cup [12,13]$
\odpStop
\testStart
A.$x \in \{-11\} \cup [12,13]$\\
B.$x \in \{11\} \cup (12,13)$\\
C.$x \in \{-11\} \cup (12,13]$\\
D.$x \in \{11\} \cup (12,13]$\\
E.$x \in \{-11\} \cup [12,13)$\\
F.$x \in \{11\} \cup [12,13)$\\
G.$x \in \{-11\} \cup (12,13)$\\
H.$x \in \{11\} \cup [12,13]$
\testStop
\kluczStart
A
\kluczStop



\zadStart{Zadanie z Wikieł Z 1.62 c) moja wersja nr 742}

Rozwiązać nierówności $(12-x)(x+11)^{2}(14-x)^{3}\le0$.
\zadStop
\rozwStart{Patryk Wirkus}{}
Miejsca zerowe naszego wielomianu to: $12, -11, 14$.\\
Wielomian jest stopnia parzystego, ponadto znak współczynnika przy\linebreak najwyższej potędze x jest ujemny.\\ W związku z tym wykres wielomianu zaczyna się od lewej strony powyżej osi OX.\\
Ponadto w punkcie $-11$ wykres odbija się od osi poziomej.\\
A więc $$x \in \{-11\} \cup [12,14].$$
\rozwStop
\odpStart
$x \in \{-11\} \cup [12,14]$
\odpStop
\testStart
A.$x \in \{-11\} \cup [12,14]$\\
B.$x \in \{11\} \cup (12,14)$\\
C.$x \in \{-11\} \cup (12,14]$\\
D.$x \in \{11\} \cup (12,14]$\\
E.$x \in \{-11\} \cup [12,14)$\\
F.$x \in \{11\} \cup [12,14)$\\
G.$x \in \{-11\} \cup (12,14)$\\
H.$x \in \{11\} \cup [12,14]$
\testStop
\kluczStart
A
\kluczStop



\zadStart{Zadanie z Wikieł Z 1.62 c) moja wersja nr 743}

Rozwiązać nierówności $(12-x)(x+11)^{2}(15-x)^{3}\le0$.
\zadStop
\rozwStart{Patryk Wirkus}{}
Miejsca zerowe naszego wielomianu to: $12, -11, 15$.\\
Wielomian jest stopnia parzystego, ponadto znak współczynnika przy\linebreak najwyższej potędze x jest ujemny.\\ W związku z tym wykres wielomianu zaczyna się od lewej strony powyżej osi OX.\\
Ponadto w punkcie $-11$ wykres odbija się od osi poziomej.\\
A więc $$x \in \{-11\} \cup [12,15].$$
\rozwStop
\odpStart
$x \in \{-11\} \cup [12,15]$
\odpStop
\testStart
A.$x \in \{-11\} \cup [12,15]$\\
B.$x \in \{11\} \cup (12,15)$\\
C.$x \in \{-11\} \cup (12,15]$\\
D.$x \in \{11\} \cup (12,15]$\\
E.$x \in \{-11\} \cup [12,15)$\\
F.$x \in \{11\} \cup [12,15)$\\
G.$x \in \{-11\} \cup (12,15)$\\
H.$x \in \{11\} \cup [12,15]$
\testStop
\kluczStart
A
\kluczStop



\zadStart{Zadanie z Wikieł Z 1.62 c) moja wersja nr 744}

Rozwiązać nierówności $(12-x)(x+11)^{2}(16-x)^{3}\le0$.
\zadStop
\rozwStart{Patryk Wirkus}{}
Miejsca zerowe naszego wielomianu to: $12, -11, 16$.\\
Wielomian jest stopnia parzystego, ponadto znak współczynnika przy\linebreak najwyższej potędze x jest ujemny.\\ W związku z tym wykres wielomianu zaczyna się od lewej strony powyżej osi OX.\\
Ponadto w punkcie $-11$ wykres odbija się od osi poziomej.\\
A więc $$x \in \{-11\} \cup [12,16].$$
\rozwStop
\odpStart
$x \in \{-11\} \cup [12,16]$
\odpStop
\testStart
A.$x \in \{-11\} \cup [12,16]$\\
B.$x \in \{11\} \cup (12,16)$\\
C.$x \in \{-11\} \cup (12,16]$\\
D.$x \in \{11\} \cup (12,16]$\\
E.$x \in \{-11\} \cup [12,16)$\\
F.$x \in \{11\} \cup [12,16)$\\
G.$x \in \{-11\} \cup (12,16)$\\
H.$x \in \{11\} \cup [12,16]$
\testStop
\kluczStart
A
\kluczStop



\zadStart{Zadanie z Wikieł Z 1.62 c) moja wersja nr 745}

Rozwiązać nierówności $(12-x)(x+11)^{2}(17-x)^{3}\le0$.
\zadStop
\rozwStart{Patryk Wirkus}{}
Miejsca zerowe naszego wielomianu to: $12, -11, 17$.\\
Wielomian jest stopnia parzystego, ponadto znak współczynnika przy\linebreak najwyższej potędze x jest ujemny.\\ W związku z tym wykres wielomianu zaczyna się od lewej strony powyżej osi OX.\\
Ponadto w punkcie $-11$ wykres odbija się od osi poziomej.\\
A więc $$x \in \{-11\} \cup [12,17].$$
\rozwStop
\odpStart
$x \in \{-11\} \cup [12,17]$
\odpStop
\testStart
A.$x \in \{-11\} \cup [12,17]$\\
B.$x \in \{11\} \cup (12,17)$\\
C.$x \in \{-11\} \cup (12,17]$\\
D.$x \in \{11\} \cup (12,17]$\\
E.$x \in \{-11\} \cup [12,17)$\\
F.$x \in \{11\} \cup [12,17)$\\
G.$x \in \{-11\} \cup (12,17)$\\
H.$x \in \{11\} \cup [12,17]$
\testStop
\kluczStart
A
\kluczStop



\zadStart{Zadanie z Wikieł Z 1.62 c) moja wersja nr 746}

Rozwiązać nierówności $(12-x)(x+11)^{2}(18-x)^{3}\le0$.
\zadStop
\rozwStart{Patryk Wirkus}{}
Miejsca zerowe naszego wielomianu to: $12, -11, 18$.\\
Wielomian jest stopnia parzystego, ponadto znak współczynnika przy\linebreak najwyższej potędze x jest ujemny.\\ W związku z tym wykres wielomianu zaczyna się od lewej strony powyżej osi OX.\\
Ponadto w punkcie $-11$ wykres odbija się od osi poziomej.\\
A więc $$x \in \{-11\} \cup [12,18].$$
\rozwStop
\odpStart
$x \in \{-11\} \cup [12,18]$
\odpStop
\testStart
A.$x \in \{-11\} \cup [12,18]$\\
B.$x \in \{11\} \cup (12,18)$\\
C.$x \in \{-11\} \cup (12,18]$\\
D.$x \in \{11\} \cup (12,18]$\\
E.$x \in \{-11\} \cup [12,18)$\\
F.$x \in \{11\} \cup [12,18)$\\
G.$x \in \{-11\} \cup (12,18)$\\
H.$x \in \{11\} \cup [12,18]$
\testStop
\kluczStart
A
\kluczStop



\zadStart{Zadanie z Wikieł Z 1.62 c) moja wersja nr 747}

Rozwiązać nierówności $(12-x)(x+11)^{2}(19-x)^{3}\le0$.
\zadStop
\rozwStart{Patryk Wirkus}{}
Miejsca zerowe naszego wielomianu to: $12, -11, 19$.\\
Wielomian jest stopnia parzystego, ponadto znak współczynnika przy\linebreak najwyższej potędze x jest ujemny.\\ W związku z tym wykres wielomianu zaczyna się od lewej strony powyżej osi OX.\\
Ponadto w punkcie $-11$ wykres odbija się od osi poziomej.\\
A więc $$x \in \{-11\} \cup [12,19].$$
\rozwStop
\odpStart
$x \in \{-11\} \cup [12,19]$
\odpStop
\testStart
A.$x \in \{-11\} \cup [12,19]$\\
B.$x \in \{11\} \cup (12,19)$\\
C.$x \in \{-11\} \cup (12,19]$\\
D.$x \in \{11\} \cup (12,19]$\\
E.$x \in \{-11\} \cup [12,19)$\\
F.$x \in \{11\} \cup [12,19)$\\
G.$x \in \{-11\} \cup (12,19)$\\
H.$x \in \{11\} \cup [12,19]$
\testStop
\kluczStart
A
\kluczStop



\zadStart{Zadanie z Wikieł Z 1.62 c) moja wersja nr 748}

Rozwiązać nierówności $(12-x)(x+11)^{2}(20-x)^{3}\le0$.
\zadStop
\rozwStart{Patryk Wirkus}{}
Miejsca zerowe naszego wielomianu to: $12, -11, 20$.\\
Wielomian jest stopnia parzystego, ponadto znak współczynnika przy\linebreak najwyższej potędze x jest ujemny.\\ W związku z tym wykres wielomianu zaczyna się od lewej strony powyżej osi OX.\\
Ponadto w punkcie $-11$ wykres odbija się od osi poziomej.\\
A więc $$x \in \{-11\} \cup [12,20].$$
\rozwStop
\odpStart
$x \in \{-11\} \cup [12,20]$
\odpStop
\testStart
A.$x \in \{-11\} \cup [12,20]$\\
B.$x \in \{11\} \cup (12,20)$\\
C.$x \in \{-11\} \cup (12,20]$\\
D.$x \in \{11\} \cup (12,20]$\\
E.$x \in \{-11\} \cup [12,20)$\\
F.$x \in \{11\} \cup [12,20)$\\
G.$x \in \{-11\} \cup (12,20)$\\
H.$x \in \{11\} \cup [12,20]$
\testStop
\kluczStart
A
\kluczStop



\zadStart{Zadanie z Wikieł Z 1.62 c) moja wersja nr 749}

Rozwiązać nierówności $(13-x)(x+1)^{2}(14-x)^{3}\le0$.
\zadStop
\rozwStart{Patryk Wirkus}{}
Miejsca zerowe naszego wielomianu to: $13, -1, 14$.\\
Wielomian jest stopnia parzystego, ponadto znak współczynnika przy\linebreak najwyższej potędze x jest ujemny.\\ W związku z tym wykres wielomianu zaczyna się od lewej strony powyżej osi OX.\\
Ponadto w punkcie $-1$ wykres odbija się od osi poziomej.\\
A więc $$x \in \{-1\} \cup [13,14].$$
\rozwStop
\odpStart
$x \in \{-1\} \cup [13,14]$
\odpStop
\testStart
A.$x \in \{-1\} \cup [13,14]$\\
B.$x \in \{1\} \cup (13,14)$\\
C.$x \in \{-1\} \cup (13,14]$\\
D.$x \in \{1\} \cup (13,14]$\\
E.$x \in \{-1\} \cup [13,14)$\\
F.$x \in \{1\} \cup [13,14)$\\
G.$x \in \{-1\} \cup (13,14)$\\
H.$x \in \{1\} \cup [13,14]$
\testStop
\kluczStart
A
\kluczStop



\zadStart{Zadanie z Wikieł Z 1.62 c) moja wersja nr 750}

Rozwiązać nierówności $(13-x)(x+1)^{2}(15-x)^{3}\le0$.
\zadStop
\rozwStart{Patryk Wirkus}{}
Miejsca zerowe naszego wielomianu to: $13, -1, 15$.\\
Wielomian jest stopnia parzystego, ponadto znak współczynnika przy\linebreak najwyższej potędze x jest ujemny.\\ W związku z tym wykres wielomianu zaczyna się od lewej strony powyżej osi OX.\\
Ponadto w punkcie $-1$ wykres odbija się od osi poziomej.\\
A więc $$x \in \{-1\} \cup [13,15].$$
\rozwStop
\odpStart
$x \in \{-1\} \cup [13,15]$
\odpStop
\testStart
A.$x \in \{-1\} \cup [13,15]$\\
B.$x \in \{1\} \cup (13,15)$\\
C.$x \in \{-1\} \cup (13,15]$\\
D.$x \in \{1\} \cup (13,15]$\\
E.$x \in \{-1\} \cup [13,15)$\\
F.$x \in \{1\} \cup [13,15)$\\
G.$x \in \{-1\} \cup (13,15)$\\
H.$x \in \{1\} \cup [13,15]$
\testStop
\kluczStart
A
\kluczStop



\zadStart{Zadanie z Wikieł Z 1.62 c) moja wersja nr 751}

Rozwiązać nierówności $(13-x)(x+1)^{2}(16-x)^{3}\le0$.
\zadStop
\rozwStart{Patryk Wirkus}{}
Miejsca zerowe naszego wielomianu to: $13, -1, 16$.\\
Wielomian jest stopnia parzystego, ponadto znak współczynnika przy\linebreak najwyższej potędze x jest ujemny.\\ W związku z tym wykres wielomianu zaczyna się od lewej strony powyżej osi OX.\\
Ponadto w punkcie $-1$ wykres odbija się od osi poziomej.\\
A więc $$x \in \{-1\} \cup [13,16].$$
\rozwStop
\odpStart
$x \in \{-1\} \cup [13,16]$
\odpStop
\testStart
A.$x \in \{-1\} \cup [13,16]$\\
B.$x \in \{1\} \cup (13,16)$\\
C.$x \in \{-1\} \cup (13,16]$\\
D.$x \in \{1\} \cup (13,16]$\\
E.$x \in \{-1\} \cup [13,16)$\\
F.$x \in \{1\} \cup [13,16)$\\
G.$x \in \{-1\} \cup (13,16)$\\
H.$x \in \{1\} \cup [13,16]$
\testStop
\kluczStart
A
\kluczStop



\zadStart{Zadanie z Wikieł Z 1.62 c) moja wersja nr 752}

Rozwiązać nierówności $(13-x)(x+1)^{2}(17-x)^{3}\le0$.
\zadStop
\rozwStart{Patryk Wirkus}{}
Miejsca zerowe naszego wielomianu to: $13, -1, 17$.\\
Wielomian jest stopnia parzystego, ponadto znak współczynnika przy\linebreak najwyższej potędze x jest ujemny.\\ W związku z tym wykres wielomianu zaczyna się od lewej strony powyżej osi OX.\\
Ponadto w punkcie $-1$ wykres odbija się od osi poziomej.\\
A więc $$x \in \{-1\} \cup [13,17].$$
\rozwStop
\odpStart
$x \in \{-1\} \cup [13,17]$
\odpStop
\testStart
A.$x \in \{-1\} \cup [13,17]$\\
B.$x \in \{1\} \cup (13,17)$\\
C.$x \in \{-1\} \cup (13,17]$\\
D.$x \in \{1\} \cup (13,17]$\\
E.$x \in \{-1\} \cup [13,17)$\\
F.$x \in \{1\} \cup [13,17)$\\
G.$x \in \{-1\} \cup (13,17)$\\
H.$x \in \{1\} \cup [13,17]$
\testStop
\kluczStart
A
\kluczStop



\zadStart{Zadanie z Wikieł Z 1.62 c) moja wersja nr 753}

Rozwiązać nierówności $(13-x)(x+1)^{2}(18-x)^{3}\le0$.
\zadStop
\rozwStart{Patryk Wirkus}{}
Miejsca zerowe naszego wielomianu to: $13, -1, 18$.\\
Wielomian jest stopnia parzystego, ponadto znak współczynnika przy\linebreak najwyższej potędze x jest ujemny.\\ W związku z tym wykres wielomianu zaczyna się od lewej strony powyżej osi OX.\\
Ponadto w punkcie $-1$ wykres odbija się od osi poziomej.\\
A więc $$x \in \{-1\} \cup [13,18].$$
\rozwStop
\odpStart
$x \in \{-1\} \cup [13,18]$
\odpStop
\testStart
A.$x \in \{-1\} \cup [13,18]$\\
B.$x \in \{1\} \cup (13,18)$\\
C.$x \in \{-1\} \cup (13,18]$\\
D.$x \in \{1\} \cup (13,18]$\\
E.$x \in \{-1\} \cup [13,18)$\\
F.$x \in \{1\} \cup [13,18)$\\
G.$x \in \{-1\} \cup (13,18)$\\
H.$x \in \{1\} \cup [13,18]$
\testStop
\kluczStart
A
\kluczStop



\zadStart{Zadanie z Wikieł Z 1.62 c) moja wersja nr 754}

Rozwiązać nierówności $(13-x)(x+1)^{2}(19-x)^{3}\le0$.
\zadStop
\rozwStart{Patryk Wirkus}{}
Miejsca zerowe naszego wielomianu to: $13, -1, 19$.\\
Wielomian jest stopnia parzystego, ponadto znak współczynnika przy\linebreak najwyższej potędze x jest ujemny.\\ W związku z tym wykres wielomianu zaczyna się od lewej strony powyżej osi OX.\\
Ponadto w punkcie $-1$ wykres odbija się od osi poziomej.\\
A więc $$x \in \{-1\} \cup [13,19].$$
\rozwStop
\odpStart
$x \in \{-1\} \cup [13,19]$
\odpStop
\testStart
A.$x \in \{-1\} \cup [13,19]$\\
B.$x \in \{1\} \cup (13,19)$\\
C.$x \in \{-1\} \cup (13,19]$\\
D.$x \in \{1\} \cup (13,19]$\\
E.$x \in \{-1\} \cup [13,19)$\\
F.$x \in \{1\} \cup [13,19)$\\
G.$x \in \{-1\} \cup (13,19)$\\
H.$x \in \{1\} \cup [13,19]$
\testStop
\kluczStart
A
\kluczStop



\zadStart{Zadanie z Wikieł Z 1.62 c) moja wersja nr 755}

Rozwiązać nierówności $(13-x)(x+1)^{2}(20-x)^{3}\le0$.
\zadStop
\rozwStart{Patryk Wirkus}{}
Miejsca zerowe naszego wielomianu to: $13, -1, 20$.\\
Wielomian jest stopnia parzystego, ponadto znak współczynnika przy\linebreak najwyższej potędze x jest ujemny.\\ W związku z tym wykres wielomianu zaczyna się od lewej strony powyżej osi OX.\\
Ponadto w punkcie $-1$ wykres odbija się od osi poziomej.\\
A więc $$x \in \{-1\} \cup [13,20].$$
\rozwStop
\odpStart
$x \in \{-1\} \cup [13,20]$
\odpStop
\testStart
A.$x \in \{-1\} \cup [13,20]$\\
B.$x \in \{1\} \cup (13,20)$\\
C.$x \in \{-1\} \cup (13,20]$\\
D.$x \in \{1\} \cup (13,20]$\\
E.$x \in \{-1\} \cup [13,20)$\\
F.$x \in \{1\} \cup [13,20)$\\
G.$x \in \{-1\} \cup (13,20)$\\
H.$x \in \{1\} \cup [13,20]$
\testStop
\kluczStart
A
\kluczStop



\zadStart{Zadanie z Wikieł Z 1.62 c) moja wersja nr 756}

Rozwiązać nierówności $(13-x)(x+2)^{2}(14-x)^{3}\le0$.
\zadStop
\rozwStart{Patryk Wirkus}{}
Miejsca zerowe naszego wielomianu to: $13, -2, 14$.\\
Wielomian jest stopnia parzystego, ponadto znak współczynnika przy\linebreak najwyższej potędze x jest ujemny.\\ W związku z tym wykres wielomianu zaczyna się od lewej strony powyżej osi OX.\\
Ponadto w punkcie $-2$ wykres odbija się od osi poziomej.\\
A więc $$x \in \{-2\} \cup [13,14].$$
\rozwStop
\odpStart
$x \in \{-2\} \cup [13,14]$
\odpStop
\testStart
A.$x \in \{-2\} \cup [13,14]$\\
B.$x \in \{2\} \cup (13,14)$\\
C.$x \in \{-2\} \cup (13,14]$\\
D.$x \in \{2\} \cup (13,14]$\\
E.$x \in \{-2\} \cup [13,14)$\\
F.$x \in \{2\} \cup [13,14)$\\
G.$x \in \{-2\} \cup (13,14)$\\
H.$x \in \{2\} \cup [13,14]$
\testStop
\kluczStart
A
\kluczStop



\zadStart{Zadanie z Wikieł Z 1.62 c) moja wersja nr 757}

Rozwiązać nierówności $(13-x)(x+2)^{2}(15-x)^{3}\le0$.
\zadStop
\rozwStart{Patryk Wirkus}{}
Miejsca zerowe naszego wielomianu to: $13, -2, 15$.\\
Wielomian jest stopnia parzystego, ponadto znak współczynnika przy\linebreak najwyższej potędze x jest ujemny.\\ W związku z tym wykres wielomianu zaczyna się od lewej strony powyżej osi OX.\\
Ponadto w punkcie $-2$ wykres odbija się od osi poziomej.\\
A więc $$x \in \{-2\} \cup [13,15].$$
\rozwStop
\odpStart
$x \in \{-2\} \cup [13,15]$
\odpStop
\testStart
A.$x \in \{-2\} \cup [13,15]$\\
B.$x \in \{2\} \cup (13,15)$\\
C.$x \in \{-2\} \cup (13,15]$\\
D.$x \in \{2\} \cup (13,15]$\\
E.$x \in \{-2\} \cup [13,15)$\\
F.$x \in \{2\} \cup [13,15)$\\
G.$x \in \{-2\} \cup (13,15)$\\
H.$x \in \{2\} \cup [13,15]$
\testStop
\kluczStart
A
\kluczStop



\zadStart{Zadanie z Wikieł Z 1.62 c) moja wersja nr 758}

Rozwiązać nierówności $(13-x)(x+2)^{2}(16-x)^{3}\le0$.
\zadStop
\rozwStart{Patryk Wirkus}{}
Miejsca zerowe naszego wielomianu to: $13, -2, 16$.\\
Wielomian jest stopnia parzystego, ponadto znak współczynnika przy\linebreak najwyższej potędze x jest ujemny.\\ W związku z tym wykres wielomianu zaczyna się od lewej strony powyżej osi OX.\\
Ponadto w punkcie $-2$ wykres odbija się od osi poziomej.\\
A więc $$x \in \{-2\} \cup [13,16].$$
\rozwStop
\odpStart
$x \in \{-2\} \cup [13,16]$
\odpStop
\testStart
A.$x \in \{-2\} \cup [13,16]$\\
B.$x \in \{2\} \cup (13,16)$\\
C.$x \in \{-2\} \cup (13,16]$\\
D.$x \in \{2\} \cup (13,16]$\\
E.$x \in \{-2\} \cup [13,16)$\\
F.$x \in \{2\} \cup [13,16)$\\
G.$x \in \{-2\} \cup (13,16)$\\
H.$x \in \{2\} \cup [13,16]$
\testStop
\kluczStart
A
\kluczStop



\zadStart{Zadanie z Wikieł Z 1.62 c) moja wersja nr 759}

Rozwiązać nierówności $(13-x)(x+2)^{2}(17-x)^{3}\le0$.
\zadStop
\rozwStart{Patryk Wirkus}{}
Miejsca zerowe naszego wielomianu to: $13, -2, 17$.\\
Wielomian jest stopnia parzystego, ponadto znak współczynnika przy\linebreak najwyższej potędze x jest ujemny.\\ W związku z tym wykres wielomianu zaczyna się od lewej strony powyżej osi OX.\\
Ponadto w punkcie $-2$ wykres odbija się od osi poziomej.\\
A więc $$x \in \{-2\} \cup [13,17].$$
\rozwStop
\odpStart
$x \in \{-2\} \cup [13,17]$
\odpStop
\testStart
A.$x \in \{-2\} \cup [13,17]$\\
B.$x \in \{2\} \cup (13,17)$\\
C.$x \in \{-2\} \cup (13,17]$\\
D.$x \in \{2\} \cup (13,17]$\\
E.$x \in \{-2\} \cup [13,17)$\\
F.$x \in \{2\} \cup [13,17)$\\
G.$x \in \{-2\} \cup (13,17)$\\
H.$x \in \{2\} \cup [13,17]$
\testStop
\kluczStart
A
\kluczStop



\zadStart{Zadanie z Wikieł Z 1.62 c) moja wersja nr 760}

Rozwiązać nierówności $(13-x)(x+2)^{2}(18-x)^{3}\le0$.
\zadStop
\rozwStart{Patryk Wirkus}{}
Miejsca zerowe naszego wielomianu to: $13, -2, 18$.\\
Wielomian jest stopnia parzystego, ponadto znak współczynnika przy\linebreak najwyższej potędze x jest ujemny.\\ W związku z tym wykres wielomianu zaczyna się od lewej strony powyżej osi OX.\\
Ponadto w punkcie $-2$ wykres odbija się od osi poziomej.\\
A więc $$x \in \{-2\} \cup [13,18].$$
\rozwStop
\odpStart
$x \in \{-2\} \cup [13,18]$
\odpStop
\testStart
A.$x \in \{-2\} \cup [13,18]$\\
B.$x \in \{2\} \cup (13,18)$\\
C.$x \in \{-2\} \cup (13,18]$\\
D.$x \in \{2\} \cup (13,18]$\\
E.$x \in \{-2\} \cup [13,18)$\\
F.$x \in \{2\} \cup [13,18)$\\
G.$x \in \{-2\} \cup (13,18)$\\
H.$x \in \{2\} \cup [13,18]$
\testStop
\kluczStart
A
\kluczStop



\zadStart{Zadanie z Wikieł Z 1.62 c) moja wersja nr 761}

Rozwiązać nierówności $(13-x)(x+2)^{2}(19-x)^{3}\le0$.
\zadStop
\rozwStart{Patryk Wirkus}{}
Miejsca zerowe naszego wielomianu to: $13, -2, 19$.\\
Wielomian jest stopnia parzystego, ponadto znak współczynnika przy\linebreak najwyższej potędze x jest ujemny.\\ W związku z tym wykres wielomianu zaczyna się od lewej strony powyżej osi OX.\\
Ponadto w punkcie $-2$ wykres odbija się od osi poziomej.\\
A więc $$x \in \{-2\} \cup [13,19].$$
\rozwStop
\odpStart
$x \in \{-2\} \cup [13,19]$
\odpStop
\testStart
A.$x \in \{-2\} \cup [13,19]$\\
B.$x \in \{2\} \cup (13,19)$\\
C.$x \in \{-2\} \cup (13,19]$\\
D.$x \in \{2\} \cup (13,19]$\\
E.$x \in \{-2\} \cup [13,19)$\\
F.$x \in \{2\} \cup [13,19)$\\
G.$x \in \{-2\} \cup (13,19)$\\
H.$x \in \{2\} \cup [13,19]$
\testStop
\kluczStart
A
\kluczStop



\zadStart{Zadanie z Wikieł Z 1.62 c) moja wersja nr 762}

Rozwiązać nierówności $(13-x)(x+2)^{2}(20-x)^{3}\le0$.
\zadStop
\rozwStart{Patryk Wirkus}{}
Miejsca zerowe naszego wielomianu to: $13, -2, 20$.\\
Wielomian jest stopnia parzystego, ponadto znak współczynnika przy\linebreak najwyższej potędze x jest ujemny.\\ W związku z tym wykres wielomianu zaczyna się od lewej strony powyżej osi OX.\\
Ponadto w punkcie $-2$ wykres odbija się od osi poziomej.\\
A więc $$x \in \{-2\} \cup [13,20].$$
\rozwStop
\odpStart
$x \in \{-2\} \cup [13,20]$
\odpStop
\testStart
A.$x \in \{-2\} \cup [13,20]$\\
B.$x \in \{2\} \cup (13,20)$\\
C.$x \in \{-2\} \cup (13,20]$\\
D.$x \in \{2\} \cup (13,20]$\\
E.$x \in \{-2\} \cup [13,20)$\\
F.$x \in \{2\} \cup [13,20)$\\
G.$x \in \{-2\} \cup (13,20)$\\
H.$x \in \{2\} \cup [13,20]$
\testStop
\kluczStart
A
\kluczStop



\zadStart{Zadanie z Wikieł Z 1.62 c) moja wersja nr 763}

Rozwiązać nierówności $(13-x)(x+3)^{2}(14-x)^{3}\le0$.
\zadStop
\rozwStart{Patryk Wirkus}{}
Miejsca zerowe naszego wielomianu to: $13, -3, 14$.\\
Wielomian jest stopnia parzystego, ponadto znak współczynnika przy\linebreak najwyższej potędze x jest ujemny.\\ W związku z tym wykres wielomianu zaczyna się od lewej strony powyżej osi OX.\\
Ponadto w punkcie $-3$ wykres odbija się od osi poziomej.\\
A więc $$x \in \{-3\} \cup [13,14].$$
\rozwStop
\odpStart
$x \in \{-3\} \cup [13,14]$
\odpStop
\testStart
A.$x \in \{-3\} \cup [13,14]$\\
B.$x \in \{3\} \cup (13,14)$\\
C.$x \in \{-3\} \cup (13,14]$\\
D.$x \in \{3\} \cup (13,14]$\\
E.$x \in \{-3\} \cup [13,14)$\\
F.$x \in \{3\} \cup [13,14)$\\
G.$x \in \{-3\} \cup (13,14)$\\
H.$x \in \{3\} \cup [13,14]$
\testStop
\kluczStart
A
\kluczStop



\zadStart{Zadanie z Wikieł Z 1.62 c) moja wersja nr 764}

Rozwiązać nierówności $(13-x)(x+3)^{2}(15-x)^{3}\le0$.
\zadStop
\rozwStart{Patryk Wirkus}{}
Miejsca zerowe naszego wielomianu to: $13, -3, 15$.\\
Wielomian jest stopnia parzystego, ponadto znak współczynnika przy\linebreak najwyższej potędze x jest ujemny.\\ W związku z tym wykres wielomianu zaczyna się od lewej strony powyżej osi OX.\\
Ponadto w punkcie $-3$ wykres odbija się od osi poziomej.\\
A więc $$x \in \{-3\} \cup [13,15].$$
\rozwStop
\odpStart
$x \in \{-3\} \cup [13,15]$
\odpStop
\testStart
A.$x \in \{-3\} \cup [13,15]$\\
B.$x \in \{3\} \cup (13,15)$\\
C.$x \in \{-3\} \cup (13,15]$\\
D.$x \in \{3\} \cup (13,15]$\\
E.$x \in \{-3\} \cup [13,15)$\\
F.$x \in \{3\} \cup [13,15)$\\
G.$x \in \{-3\} \cup (13,15)$\\
H.$x \in \{3\} \cup [13,15]$
\testStop
\kluczStart
A
\kluczStop



\zadStart{Zadanie z Wikieł Z 1.62 c) moja wersja nr 765}

Rozwiązać nierówności $(13-x)(x+3)^{2}(16-x)^{3}\le0$.
\zadStop
\rozwStart{Patryk Wirkus}{}
Miejsca zerowe naszego wielomianu to: $13, -3, 16$.\\
Wielomian jest stopnia parzystego, ponadto znak współczynnika przy\linebreak najwyższej potędze x jest ujemny.\\ W związku z tym wykres wielomianu zaczyna się od lewej strony powyżej osi OX.\\
Ponadto w punkcie $-3$ wykres odbija się od osi poziomej.\\
A więc $$x \in \{-3\} \cup [13,16].$$
\rozwStop
\odpStart
$x \in \{-3\} \cup [13,16]$
\odpStop
\testStart
A.$x \in \{-3\} \cup [13,16]$\\
B.$x \in \{3\} \cup (13,16)$\\
C.$x \in \{-3\} \cup (13,16]$\\
D.$x \in \{3\} \cup (13,16]$\\
E.$x \in \{-3\} \cup [13,16)$\\
F.$x \in \{3\} \cup [13,16)$\\
G.$x \in \{-3\} \cup (13,16)$\\
H.$x \in \{3\} \cup [13,16]$
\testStop
\kluczStart
A
\kluczStop



\zadStart{Zadanie z Wikieł Z 1.62 c) moja wersja nr 766}

Rozwiązać nierówności $(13-x)(x+3)^{2}(17-x)^{3}\le0$.
\zadStop
\rozwStart{Patryk Wirkus}{}
Miejsca zerowe naszego wielomianu to: $13, -3, 17$.\\
Wielomian jest stopnia parzystego, ponadto znak współczynnika przy\linebreak najwyższej potędze x jest ujemny.\\ W związku z tym wykres wielomianu zaczyna się od lewej strony powyżej osi OX.\\
Ponadto w punkcie $-3$ wykres odbija się od osi poziomej.\\
A więc $$x \in \{-3\} \cup [13,17].$$
\rozwStop
\odpStart
$x \in \{-3\} \cup [13,17]$
\odpStop
\testStart
A.$x \in \{-3\} \cup [13,17]$\\
B.$x \in \{3\} \cup (13,17)$\\
C.$x \in \{-3\} \cup (13,17]$\\
D.$x \in \{3\} \cup (13,17]$\\
E.$x \in \{-3\} \cup [13,17)$\\
F.$x \in \{3\} \cup [13,17)$\\
G.$x \in \{-3\} \cup (13,17)$\\
H.$x \in \{3\} \cup [13,17]$
\testStop
\kluczStart
A
\kluczStop



\zadStart{Zadanie z Wikieł Z 1.62 c) moja wersja nr 767}

Rozwiązać nierówności $(13-x)(x+3)^{2}(18-x)^{3}\le0$.
\zadStop
\rozwStart{Patryk Wirkus}{}
Miejsca zerowe naszego wielomianu to: $13, -3, 18$.\\
Wielomian jest stopnia parzystego, ponadto znak współczynnika przy\linebreak najwyższej potędze x jest ujemny.\\ W związku z tym wykres wielomianu zaczyna się od lewej strony powyżej osi OX.\\
Ponadto w punkcie $-3$ wykres odbija się od osi poziomej.\\
A więc $$x \in \{-3\} \cup [13,18].$$
\rozwStop
\odpStart
$x \in \{-3\} \cup [13,18]$
\odpStop
\testStart
A.$x \in \{-3\} \cup [13,18]$\\
B.$x \in \{3\} \cup (13,18)$\\
C.$x \in \{-3\} \cup (13,18]$\\
D.$x \in \{3\} \cup (13,18]$\\
E.$x \in \{-3\} \cup [13,18)$\\
F.$x \in \{3\} \cup [13,18)$\\
G.$x \in \{-3\} \cup (13,18)$\\
H.$x \in \{3\} \cup [13,18]$
\testStop
\kluczStart
A
\kluczStop



\zadStart{Zadanie z Wikieł Z 1.62 c) moja wersja nr 768}

Rozwiązać nierówności $(13-x)(x+3)^{2}(19-x)^{3}\le0$.
\zadStop
\rozwStart{Patryk Wirkus}{}
Miejsca zerowe naszego wielomianu to: $13, -3, 19$.\\
Wielomian jest stopnia parzystego, ponadto znak współczynnika przy\linebreak najwyższej potędze x jest ujemny.\\ W związku z tym wykres wielomianu zaczyna się od lewej strony powyżej osi OX.\\
Ponadto w punkcie $-3$ wykres odbija się od osi poziomej.\\
A więc $$x \in \{-3\} \cup [13,19].$$
\rozwStop
\odpStart
$x \in \{-3\} \cup [13,19]$
\odpStop
\testStart
A.$x \in \{-3\} \cup [13,19]$\\
B.$x \in \{3\} \cup (13,19)$\\
C.$x \in \{-3\} \cup (13,19]$\\
D.$x \in \{3\} \cup (13,19]$\\
E.$x \in \{-3\} \cup [13,19)$\\
F.$x \in \{3\} \cup [13,19)$\\
G.$x \in \{-3\} \cup (13,19)$\\
H.$x \in \{3\} \cup [13,19]$
\testStop
\kluczStart
A
\kluczStop



\zadStart{Zadanie z Wikieł Z 1.62 c) moja wersja nr 769}

Rozwiązać nierówności $(13-x)(x+3)^{2}(20-x)^{3}\le0$.
\zadStop
\rozwStart{Patryk Wirkus}{}
Miejsca zerowe naszego wielomianu to: $13, -3, 20$.\\
Wielomian jest stopnia parzystego, ponadto znak współczynnika przy\linebreak najwyższej potędze x jest ujemny.\\ W związku z tym wykres wielomianu zaczyna się od lewej strony powyżej osi OX.\\
Ponadto w punkcie $-3$ wykres odbija się od osi poziomej.\\
A więc $$x \in \{-3\} \cup [13,20].$$
\rozwStop
\odpStart
$x \in \{-3\} \cup [13,20]$
\odpStop
\testStart
A.$x \in \{-3\} \cup [13,20]$\\
B.$x \in \{3\} \cup (13,20)$\\
C.$x \in \{-3\} \cup (13,20]$\\
D.$x \in \{3\} \cup (13,20]$\\
E.$x \in \{-3\} \cup [13,20)$\\
F.$x \in \{3\} \cup [13,20)$\\
G.$x \in \{-3\} \cup (13,20)$\\
H.$x \in \{3\} \cup [13,20]$
\testStop
\kluczStart
A
\kluczStop



\zadStart{Zadanie z Wikieł Z 1.62 c) moja wersja nr 770}

Rozwiązać nierówności $(13-x)(x+4)^{2}(14-x)^{3}\le0$.
\zadStop
\rozwStart{Patryk Wirkus}{}
Miejsca zerowe naszego wielomianu to: $13, -4, 14$.\\
Wielomian jest stopnia parzystego, ponadto znak współczynnika przy\linebreak najwyższej potędze x jest ujemny.\\ W związku z tym wykres wielomianu zaczyna się od lewej strony powyżej osi OX.\\
Ponadto w punkcie $-4$ wykres odbija się od osi poziomej.\\
A więc $$x \in \{-4\} \cup [13,14].$$
\rozwStop
\odpStart
$x \in \{-4\} \cup [13,14]$
\odpStop
\testStart
A.$x \in \{-4\} \cup [13,14]$\\
B.$x \in \{4\} \cup (13,14)$\\
C.$x \in \{-4\} \cup (13,14]$\\
D.$x \in \{4\} \cup (13,14]$\\
E.$x \in \{-4\} \cup [13,14)$\\
F.$x \in \{4\} \cup [13,14)$\\
G.$x \in \{-4\} \cup (13,14)$\\
H.$x \in \{4\} \cup [13,14]$
\testStop
\kluczStart
A
\kluczStop



\zadStart{Zadanie z Wikieł Z 1.62 c) moja wersja nr 771}

Rozwiązać nierówności $(13-x)(x+4)^{2}(15-x)^{3}\le0$.
\zadStop
\rozwStart{Patryk Wirkus}{}
Miejsca zerowe naszego wielomianu to: $13, -4, 15$.\\
Wielomian jest stopnia parzystego, ponadto znak współczynnika przy\linebreak najwyższej potędze x jest ujemny.\\ W związku z tym wykres wielomianu zaczyna się od lewej strony powyżej osi OX.\\
Ponadto w punkcie $-4$ wykres odbija się od osi poziomej.\\
A więc $$x \in \{-4\} \cup [13,15].$$
\rozwStop
\odpStart
$x \in \{-4\} \cup [13,15]$
\odpStop
\testStart
A.$x \in \{-4\} \cup [13,15]$\\
B.$x \in \{4\} \cup (13,15)$\\
C.$x \in \{-4\} \cup (13,15]$\\
D.$x \in \{4\} \cup (13,15]$\\
E.$x \in \{-4\} \cup [13,15)$\\
F.$x \in \{4\} \cup [13,15)$\\
G.$x \in \{-4\} \cup (13,15)$\\
H.$x \in \{4\} \cup [13,15]$
\testStop
\kluczStart
A
\kluczStop



\zadStart{Zadanie z Wikieł Z 1.62 c) moja wersja nr 772}

Rozwiązać nierówności $(13-x)(x+4)^{2}(16-x)^{3}\le0$.
\zadStop
\rozwStart{Patryk Wirkus}{}
Miejsca zerowe naszego wielomianu to: $13, -4, 16$.\\
Wielomian jest stopnia parzystego, ponadto znak współczynnika przy\linebreak najwyższej potędze x jest ujemny.\\ W związku z tym wykres wielomianu zaczyna się od lewej strony powyżej osi OX.\\
Ponadto w punkcie $-4$ wykres odbija się od osi poziomej.\\
A więc $$x \in \{-4\} \cup [13,16].$$
\rozwStop
\odpStart
$x \in \{-4\} \cup [13,16]$
\odpStop
\testStart
A.$x \in \{-4\} \cup [13,16]$\\
B.$x \in \{4\} \cup (13,16)$\\
C.$x \in \{-4\} \cup (13,16]$\\
D.$x \in \{4\} \cup (13,16]$\\
E.$x \in \{-4\} \cup [13,16)$\\
F.$x \in \{4\} \cup [13,16)$\\
G.$x \in \{-4\} \cup (13,16)$\\
H.$x \in \{4\} \cup [13,16]$
\testStop
\kluczStart
A
\kluczStop



\zadStart{Zadanie z Wikieł Z 1.62 c) moja wersja nr 773}

Rozwiązać nierówności $(13-x)(x+4)^{2}(17-x)^{3}\le0$.
\zadStop
\rozwStart{Patryk Wirkus}{}
Miejsca zerowe naszego wielomianu to: $13, -4, 17$.\\
Wielomian jest stopnia parzystego, ponadto znak współczynnika przy\linebreak najwyższej potędze x jest ujemny.\\ W związku z tym wykres wielomianu zaczyna się od lewej strony powyżej osi OX.\\
Ponadto w punkcie $-4$ wykres odbija się od osi poziomej.\\
A więc $$x \in \{-4\} \cup [13,17].$$
\rozwStop
\odpStart
$x \in \{-4\} \cup [13,17]$
\odpStop
\testStart
A.$x \in \{-4\} \cup [13,17]$\\
B.$x \in \{4\} \cup (13,17)$\\
C.$x \in \{-4\} \cup (13,17]$\\
D.$x \in \{4\} \cup (13,17]$\\
E.$x \in \{-4\} \cup [13,17)$\\
F.$x \in \{4\} \cup [13,17)$\\
G.$x \in \{-4\} \cup (13,17)$\\
H.$x \in \{4\} \cup [13,17]$
\testStop
\kluczStart
A
\kluczStop



\zadStart{Zadanie z Wikieł Z 1.62 c) moja wersja nr 774}

Rozwiązać nierówności $(13-x)(x+4)^{2}(18-x)^{3}\le0$.
\zadStop
\rozwStart{Patryk Wirkus}{}
Miejsca zerowe naszego wielomianu to: $13, -4, 18$.\\
Wielomian jest stopnia parzystego, ponadto znak współczynnika przy\linebreak najwyższej potędze x jest ujemny.\\ W związku z tym wykres wielomianu zaczyna się od lewej strony powyżej osi OX.\\
Ponadto w punkcie $-4$ wykres odbija się od osi poziomej.\\
A więc $$x \in \{-4\} \cup [13,18].$$
\rozwStop
\odpStart
$x \in \{-4\} \cup [13,18]$
\odpStop
\testStart
A.$x \in \{-4\} \cup [13,18]$\\
B.$x \in \{4\} \cup (13,18)$\\
C.$x \in \{-4\} \cup (13,18]$\\
D.$x \in \{4\} \cup (13,18]$\\
E.$x \in \{-4\} \cup [13,18)$\\
F.$x \in \{4\} \cup [13,18)$\\
G.$x \in \{-4\} \cup (13,18)$\\
H.$x \in \{4\} \cup [13,18]$
\testStop
\kluczStart
A
\kluczStop



\zadStart{Zadanie z Wikieł Z 1.62 c) moja wersja nr 775}

Rozwiązać nierówności $(13-x)(x+4)^{2}(19-x)^{3}\le0$.
\zadStop
\rozwStart{Patryk Wirkus}{}
Miejsca zerowe naszego wielomianu to: $13, -4, 19$.\\
Wielomian jest stopnia parzystego, ponadto znak współczynnika przy\linebreak najwyższej potędze x jest ujemny.\\ W związku z tym wykres wielomianu zaczyna się od lewej strony powyżej osi OX.\\
Ponadto w punkcie $-4$ wykres odbija się od osi poziomej.\\
A więc $$x \in \{-4\} \cup [13,19].$$
\rozwStop
\odpStart
$x \in \{-4\} \cup [13,19]$
\odpStop
\testStart
A.$x \in \{-4\} \cup [13,19]$\\
B.$x \in \{4\} \cup (13,19)$\\
C.$x \in \{-4\} \cup (13,19]$\\
D.$x \in \{4\} \cup (13,19]$\\
E.$x \in \{-4\} \cup [13,19)$\\
F.$x \in \{4\} \cup [13,19)$\\
G.$x \in \{-4\} \cup (13,19)$\\
H.$x \in \{4\} \cup [13,19]$
\testStop
\kluczStart
A
\kluczStop



\zadStart{Zadanie z Wikieł Z 1.62 c) moja wersja nr 776}

Rozwiązać nierówności $(13-x)(x+4)^{2}(20-x)^{3}\le0$.
\zadStop
\rozwStart{Patryk Wirkus}{}
Miejsca zerowe naszego wielomianu to: $13, -4, 20$.\\
Wielomian jest stopnia parzystego, ponadto znak współczynnika przy\linebreak najwyższej potędze x jest ujemny.\\ W związku z tym wykres wielomianu zaczyna się od lewej strony powyżej osi OX.\\
Ponadto w punkcie $-4$ wykres odbija się od osi poziomej.\\
A więc $$x \in \{-4\} \cup [13,20].$$
\rozwStop
\odpStart
$x \in \{-4\} \cup [13,20]$
\odpStop
\testStart
A.$x \in \{-4\} \cup [13,20]$\\
B.$x \in \{4\} \cup (13,20)$\\
C.$x \in \{-4\} \cup (13,20]$\\
D.$x \in \{4\} \cup (13,20]$\\
E.$x \in \{-4\} \cup [13,20)$\\
F.$x \in \{4\} \cup [13,20)$\\
G.$x \in \{-4\} \cup (13,20)$\\
H.$x \in \{4\} \cup [13,20]$
\testStop
\kluczStart
A
\kluczStop



\zadStart{Zadanie z Wikieł Z 1.62 c) moja wersja nr 777}

Rozwiązać nierówności $(13-x)(x+5)^{2}(14-x)^{3}\le0$.
\zadStop
\rozwStart{Patryk Wirkus}{}
Miejsca zerowe naszego wielomianu to: $13, -5, 14$.\\
Wielomian jest stopnia parzystego, ponadto znak współczynnika przy\linebreak najwyższej potędze x jest ujemny.\\ W związku z tym wykres wielomianu zaczyna się od lewej strony powyżej osi OX.\\
Ponadto w punkcie $-5$ wykres odbija się od osi poziomej.\\
A więc $$x \in \{-5\} \cup [13,14].$$
\rozwStop
\odpStart
$x \in \{-5\} \cup [13,14]$
\odpStop
\testStart
A.$x \in \{-5\} \cup [13,14]$\\
B.$x \in \{5\} \cup (13,14)$\\
C.$x \in \{-5\} \cup (13,14]$\\
D.$x \in \{5\} \cup (13,14]$\\
E.$x \in \{-5\} \cup [13,14)$\\
F.$x \in \{5\} \cup [13,14)$\\
G.$x \in \{-5\} \cup (13,14)$\\
H.$x \in \{5\} \cup [13,14]$
\testStop
\kluczStart
A
\kluczStop



\zadStart{Zadanie z Wikieł Z 1.62 c) moja wersja nr 778}

Rozwiązać nierówności $(13-x)(x+5)^{2}(15-x)^{3}\le0$.
\zadStop
\rozwStart{Patryk Wirkus}{}
Miejsca zerowe naszego wielomianu to: $13, -5, 15$.\\
Wielomian jest stopnia parzystego, ponadto znak współczynnika przy\linebreak najwyższej potędze x jest ujemny.\\ W związku z tym wykres wielomianu zaczyna się od lewej strony powyżej osi OX.\\
Ponadto w punkcie $-5$ wykres odbija się od osi poziomej.\\
A więc $$x \in \{-5\} \cup [13,15].$$
\rozwStop
\odpStart
$x \in \{-5\} \cup [13,15]$
\odpStop
\testStart
A.$x \in \{-5\} \cup [13,15]$\\
B.$x \in \{5\} \cup (13,15)$\\
C.$x \in \{-5\} \cup (13,15]$\\
D.$x \in \{5\} \cup (13,15]$\\
E.$x \in \{-5\} \cup [13,15)$\\
F.$x \in \{5\} \cup [13,15)$\\
G.$x \in \{-5\} \cup (13,15)$\\
H.$x \in \{5\} \cup [13,15]$
\testStop
\kluczStart
A
\kluczStop



\zadStart{Zadanie z Wikieł Z 1.62 c) moja wersja nr 779}

Rozwiązać nierówności $(13-x)(x+5)^{2}(16-x)^{3}\le0$.
\zadStop
\rozwStart{Patryk Wirkus}{}
Miejsca zerowe naszego wielomianu to: $13, -5, 16$.\\
Wielomian jest stopnia parzystego, ponadto znak współczynnika przy\linebreak najwyższej potędze x jest ujemny.\\ W związku z tym wykres wielomianu zaczyna się od lewej strony powyżej osi OX.\\
Ponadto w punkcie $-5$ wykres odbija się od osi poziomej.\\
A więc $$x \in \{-5\} \cup [13,16].$$
\rozwStop
\odpStart
$x \in \{-5\} \cup [13,16]$
\odpStop
\testStart
A.$x \in \{-5\} \cup [13,16]$\\
B.$x \in \{5\} \cup (13,16)$\\
C.$x \in \{-5\} \cup (13,16]$\\
D.$x \in \{5\} \cup (13,16]$\\
E.$x \in \{-5\} \cup [13,16)$\\
F.$x \in \{5\} \cup [13,16)$\\
G.$x \in \{-5\} \cup (13,16)$\\
H.$x \in \{5\} \cup [13,16]$
\testStop
\kluczStart
A
\kluczStop



\zadStart{Zadanie z Wikieł Z 1.62 c) moja wersja nr 780}

Rozwiązać nierówności $(13-x)(x+5)^{2}(17-x)^{3}\le0$.
\zadStop
\rozwStart{Patryk Wirkus}{}
Miejsca zerowe naszego wielomianu to: $13, -5, 17$.\\
Wielomian jest stopnia parzystego, ponadto znak współczynnika przy\linebreak najwyższej potędze x jest ujemny.\\ W związku z tym wykres wielomianu zaczyna się od lewej strony powyżej osi OX.\\
Ponadto w punkcie $-5$ wykres odbija się od osi poziomej.\\
A więc $$x \in \{-5\} \cup [13,17].$$
\rozwStop
\odpStart
$x \in \{-5\} \cup [13,17]$
\odpStop
\testStart
A.$x \in \{-5\} \cup [13,17]$\\
B.$x \in \{5\} \cup (13,17)$\\
C.$x \in \{-5\} \cup (13,17]$\\
D.$x \in \{5\} \cup (13,17]$\\
E.$x \in \{-5\} \cup [13,17)$\\
F.$x \in \{5\} \cup [13,17)$\\
G.$x \in \{-5\} \cup (13,17)$\\
H.$x \in \{5\} \cup [13,17]$
\testStop
\kluczStart
A
\kluczStop



\zadStart{Zadanie z Wikieł Z 1.62 c) moja wersja nr 781}

Rozwiązać nierówności $(13-x)(x+5)^{2}(18-x)^{3}\le0$.
\zadStop
\rozwStart{Patryk Wirkus}{}
Miejsca zerowe naszego wielomianu to: $13, -5, 18$.\\
Wielomian jest stopnia parzystego, ponadto znak współczynnika przy\linebreak najwyższej potędze x jest ujemny.\\ W związku z tym wykres wielomianu zaczyna się od lewej strony powyżej osi OX.\\
Ponadto w punkcie $-5$ wykres odbija się od osi poziomej.\\
A więc $$x \in \{-5\} \cup [13,18].$$
\rozwStop
\odpStart
$x \in \{-5\} \cup [13,18]$
\odpStop
\testStart
A.$x \in \{-5\} \cup [13,18]$\\
B.$x \in \{5\} \cup (13,18)$\\
C.$x \in \{-5\} \cup (13,18]$\\
D.$x \in \{5\} \cup (13,18]$\\
E.$x \in \{-5\} \cup [13,18)$\\
F.$x \in \{5\} \cup [13,18)$\\
G.$x \in \{-5\} \cup (13,18)$\\
H.$x \in \{5\} \cup [13,18]$
\testStop
\kluczStart
A
\kluczStop



\zadStart{Zadanie z Wikieł Z 1.62 c) moja wersja nr 782}

Rozwiązać nierówności $(13-x)(x+5)^{2}(19-x)^{3}\le0$.
\zadStop
\rozwStart{Patryk Wirkus}{}
Miejsca zerowe naszego wielomianu to: $13, -5, 19$.\\
Wielomian jest stopnia parzystego, ponadto znak współczynnika przy\linebreak najwyższej potędze x jest ujemny.\\ W związku z tym wykres wielomianu zaczyna się od lewej strony powyżej osi OX.\\
Ponadto w punkcie $-5$ wykres odbija się od osi poziomej.\\
A więc $$x \in \{-5\} \cup [13,19].$$
\rozwStop
\odpStart
$x \in \{-5\} \cup [13,19]$
\odpStop
\testStart
A.$x \in \{-5\} \cup [13,19]$\\
B.$x \in \{5\} \cup (13,19)$\\
C.$x \in \{-5\} \cup (13,19]$\\
D.$x \in \{5\} \cup (13,19]$\\
E.$x \in \{-5\} \cup [13,19)$\\
F.$x \in \{5\} \cup [13,19)$\\
G.$x \in \{-5\} \cup (13,19)$\\
H.$x \in \{5\} \cup [13,19]$
\testStop
\kluczStart
A
\kluczStop



\zadStart{Zadanie z Wikieł Z 1.62 c) moja wersja nr 783}

Rozwiązać nierówności $(13-x)(x+5)^{2}(20-x)^{3}\le0$.
\zadStop
\rozwStart{Patryk Wirkus}{}
Miejsca zerowe naszego wielomianu to: $13, -5, 20$.\\
Wielomian jest stopnia parzystego, ponadto znak współczynnika przy\linebreak najwyższej potędze x jest ujemny.\\ W związku z tym wykres wielomianu zaczyna się od lewej strony powyżej osi OX.\\
Ponadto w punkcie $-5$ wykres odbija się od osi poziomej.\\
A więc $$x \in \{-5\} \cup [13,20].$$
\rozwStop
\odpStart
$x \in \{-5\} \cup [13,20]$
\odpStop
\testStart
A.$x \in \{-5\} \cup [13,20]$\\
B.$x \in \{5\} \cup (13,20)$\\
C.$x \in \{-5\} \cup (13,20]$\\
D.$x \in \{5\} \cup (13,20]$\\
E.$x \in \{-5\} \cup [13,20)$\\
F.$x \in \{5\} \cup [13,20)$\\
G.$x \in \{-5\} \cup (13,20)$\\
H.$x \in \{5\} \cup [13,20]$
\testStop
\kluczStart
A
\kluczStop



\zadStart{Zadanie z Wikieł Z 1.62 c) moja wersja nr 784}

Rozwiązać nierówności $(13-x)(x+6)^{2}(14-x)^{3}\le0$.
\zadStop
\rozwStart{Patryk Wirkus}{}
Miejsca zerowe naszego wielomianu to: $13, -6, 14$.\\
Wielomian jest stopnia parzystego, ponadto znak współczynnika przy\linebreak najwyższej potędze x jest ujemny.\\ W związku z tym wykres wielomianu zaczyna się od lewej strony powyżej osi OX.\\
Ponadto w punkcie $-6$ wykres odbija się od osi poziomej.\\
A więc $$x \in \{-6\} \cup [13,14].$$
\rozwStop
\odpStart
$x \in \{-6\} \cup [13,14]$
\odpStop
\testStart
A.$x \in \{-6\} \cup [13,14]$\\
B.$x \in \{6\} \cup (13,14)$\\
C.$x \in \{-6\} \cup (13,14]$\\
D.$x \in \{6\} \cup (13,14]$\\
E.$x \in \{-6\} \cup [13,14)$\\
F.$x \in \{6\} \cup [13,14)$\\
G.$x \in \{-6\} \cup (13,14)$\\
H.$x \in \{6\} \cup [13,14]$
\testStop
\kluczStart
A
\kluczStop



\zadStart{Zadanie z Wikieł Z 1.62 c) moja wersja nr 785}

Rozwiązać nierówności $(13-x)(x+6)^{2}(15-x)^{3}\le0$.
\zadStop
\rozwStart{Patryk Wirkus}{}
Miejsca zerowe naszego wielomianu to: $13, -6, 15$.\\
Wielomian jest stopnia parzystego, ponadto znak współczynnika przy\linebreak najwyższej potędze x jest ujemny.\\ W związku z tym wykres wielomianu zaczyna się od lewej strony powyżej osi OX.\\
Ponadto w punkcie $-6$ wykres odbija się od osi poziomej.\\
A więc $$x \in \{-6\} \cup [13,15].$$
\rozwStop
\odpStart
$x \in \{-6\} \cup [13,15]$
\odpStop
\testStart
A.$x \in \{-6\} \cup [13,15]$\\
B.$x \in \{6\} \cup (13,15)$\\
C.$x \in \{-6\} \cup (13,15]$\\
D.$x \in \{6\} \cup (13,15]$\\
E.$x \in \{-6\} \cup [13,15)$\\
F.$x \in \{6\} \cup [13,15)$\\
G.$x \in \{-6\} \cup (13,15)$\\
H.$x \in \{6\} \cup [13,15]$
\testStop
\kluczStart
A
\kluczStop



\zadStart{Zadanie z Wikieł Z 1.62 c) moja wersja nr 786}

Rozwiązać nierówności $(13-x)(x+6)^{2}(16-x)^{3}\le0$.
\zadStop
\rozwStart{Patryk Wirkus}{}
Miejsca zerowe naszego wielomianu to: $13, -6, 16$.\\
Wielomian jest stopnia parzystego, ponadto znak współczynnika przy\linebreak najwyższej potędze x jest ujemny.\\ W związku z tym wykres wielomianu zaczyna się od lewej strony powyżej osi OX.\\
Ponadto w punkcie $-6$ wykres odbija się od osi poziomej.\\
A więc $$x \in \{-6\} \cup [13,16].$$
\rozwStop
\odpStart
$x \in \{-6\} \cup [13,16]$
\odpStop
\testStart
A.$x \in \{-6\} \cup [13,16]$\\
B.$x \in \{6\} \cup (13,16)$\\
C.$x \in \{-6\} \cup (13,16]$\\
D.$x \in \{6\} \cup (13,16]$\\
E.$x \in \{-6\} \cup [13,16)$\\
F.$x \in \{6\} \cup [13,16)$\\
G.$x \in \{-6\} \cup (13,16)$\\
H.$x \in \{6\} \cup [13,16]$
\testStop
\kluczStart
A
\kluczStop



\zadStart{Zadanie z Wikieł Z 1.62 c) moja wersja nr 787}

Rozwiązać nierówności $(13-x)(x+6)^{2}(17-x)^{3}\le0$.
\zadStop
\rozwStart{Patryk Wirkus}{}
Miejsca zerowe naszego wielomianu to: $13, -6, 17$.\\
Wielomian jest stopnia parzystego, ponadto znak współczynnika przy\linebreak najwyższej potędze x jest ujemny.\\ W związku z tym wykres wielomianu zaczyna się od lewej strony powyżej osi OX.\\
Ponadto w punkcie $-6$ wykres odbija się od osi poziomej.\\
A więc $$x \in \{-6\} \cup [13,17].$$
\rozwStop
\odpStart
$x \in \{-6\} \cup [13,17]$
\odpStop
\testStart
A.$x \in \{-6\} \cup [13,17]$\\
B.$x \in \{6\} \cup (13,17)$\\
C.$x \in \{-6\} \cup (13,17]$\\
D.$x \in \{6\} \cup (13,17]$\\
E.$x \in \{-6\} \cup [13,17)$\\
F.$x \in \{6\} \cup [13,17)$\\
G.$x \in \{-6\} \cup (13,17)$\\
H.$x \in \{6\} \cup [13,17]$
\testStop
\kluczStart
A
\kluczStop



\zadStart{Zadanie z Wikieł Z 1.62 c) moja wersja nr 788}

Rozwiązać nierówności $(13-x)(x+6)^{2}(18-x)^{3}\le0$.
\zadStop
\rozwStart{Patryk Wirkus}{}
Miejsca zerowe naszego wielomianu to: $13, -6, 18$.\\
Wielomian jest stopnia parzystego, ponadto znak współczynnika przy\linebreak najwyższej potędze x jest ujemny.\\ W związku z tym wykres wielomianu zaczyna się od lewej strony powyżej osi OX.\\
Ponadto w punkcie $-6$ wykres odbija się od osi poziomej.\\
A więc $$x \in \{-6\} \cup [13,18].$$
\rozwStop
\odpStart
$x \in \{-6\} \cup [13,18]$
\odpStop
\testStart
A.$x \in \{-6\} \cup [13,18]$\\
B.$x \in \{6\} \cup (13,18)$\\
C.$x \in \{-6\} \cup (13,18]$\\
D.$x \in \{6\} \cup (13,18]$\\
E.$x \in \{-6\} \cup [13,18)$\\
F.$x \in \{6\} \cup [13,18)$\\
G.$x \in \{-6\} \cup (13,18)$\\
H.$x \in \{6\} \cup [13,18]$
\testStop
\kluczStart
A
\kluczStop



\zadStart{Zadanie z Wikieł Z 1.62 c) moja wersja nr 789}

Rozwiązać nierówności $(13-x)(x+6)^{2}(19-x)^{3}\le0$.
\zadStop
\rozwStart{Patryk Wirkus}{}
Miejsca zerowe naszego wielomianu to: $13, -6, 19$.\\
Wielomian jest stopnia parzystego, ponadto znak współczynnika przy\linebreak najwyższej potędze x jest ujemny.\\ W związku z tym wykres wielomianu zaczyna się od lewej strony powyżej osi OX.\\
Ponadto w punkcie $-6$ wykres odbija się od osi poziomej.\\
A więc $$x \in \{-6\} \cup [13,19].$$
\rozwStop
\odpStart
$x \in \{-6\} \cup [13,19]$
\odpStop
\testStart
A.$x \in \{-6\} \cup [13,19]$\\
B.$x \in \{6\} \cup (13,19)$\\
C.$x \in \{-6\} \cup (13,19]$\\
D.$x \in \{6\} \cup (13,19]$\\
E.$x \in \{-6\} \cup [13,19)$\\
F.$x \in \{6\} \cup [13,19)$\\
G.$x \in \{-6\} \cup (13,19)$\\
H.$x \in \{6\} \cup [13,19]$
\testStop
\kluczStart
A
\kluczStop



\zadStart{Zadanie z Wikieł Z 1.62 c) moja wersja nr 790}

Rozwiązać nierówności $(13-x)(x+6)^{2}(20-x)^{3}\le0$.
\zadStop
\rozwStart{Patryk Wirkus}{}
Miejsca zerowe naszego wielomianu to: $13, -6, 20$.\\
Wielomian jest stopnia parzystego, ponadto znak współczynnika przy\linebreak najwyższej potędze x jest ujemny.\\ W związku z tym wykres wielomianu zaczyna się od lewej strony powyżej osi OX.\\
Ponadto w punkcie $-6$ wykres odbija się od osi poziomej.\\
A więc $$x \in \{-6\} \cup [13,20].$$
\rozwStop
\odpStart
$x \in \{-6\} \cup [13,20]$
\odpStop
\testStart
A.$x \in \{-6\} \cup [13,20]$\\
B.$x \in \{6\} \cup (13,20)$\\
C.$x \in \{-6\} \cup (13,20]$\\
D.$x \in \{6\} \cup (13,20]$\\
E.$x \in \{-6\} \cup [13,20)$\\
F.$x \in \{6\} \cup [13,20)$\\
G.$x \in \{-6\} \cup (13,20)$\\
H.$x \in \{6\} \cup [13,20]$
\testStop
\kluczStart
A
\kluczStop



\zadStart{Zadanie z Wikieł Z 1.62 c) moja wersja nr 791}

Rozwiązać nierówności $(13-x)(x+7)^{2}(14-x)^{3}\le0$.
\zadStop
\rozwStart{Patryk Wirkus}{}
Miejsca zerowe naszego wielomianu to: $13, -7, 14$.\\
Wielomian jest stopnia parzystego, ponadto znak współczynnika przy\linebreak najwyższej potędze x jest ujemny.\\ W związku z tym wykres wielomianu zaczyna się od lewej strony powyżej osi OX.\\
Ponadto w punkcie $-7$ wykres odbija się od osi poziomej.\\
A więc $$x \in \{-7\} \cup [13,14].$$
\rozwStop
\odpStart
$x \in \{-7\} \cup [13,14]$
\odpStop
\testStart
A.$x \in \{-7\} \cup [13,14]$\\
B.$x \in \{7\} \cup (13,14)$\\
C.$x \in \{-7\} \cup (13,14]$\\
D.$x \in \{7\} \cup (13,14]$\\
E.$x \in \{-7\} \cup [13,14)$\\
F.$x \in \{7\} \cup [13,14)$\\
G.$x \in \{-7\} \cup (13,14)$\\
H.$x \in \{7\} \cup [13,14]$
\testStop
\kluczStart
A
\kluczStop



\zadStart{Zadanie z Wikieł Z 1.62 c) moja wersja nr 792}

Rozwiązać nierówności $(13-x)(x+7)^{2}(15-x)^{3}\le0$.
\zadStop
\rozwStart{Patryk Wirkus}{}
Miejsca zerowe naszego wielomianu to: $13, -7, 15$.\\
Wielomian jest stopnia parzystego, ponadto znak współczynnika przy\linebreak najwyższej potędze x jest ujemny.\\ W związku z tym wykres wielomianu zaczyna się od lewej strony powyżej osi OX.\\
Ponadto w punkcie $-7$ wykres odbija się od osi poziomej.\\
A więc $$x \in \{-7\} \cup [13,15].$$
\rozwStop
\odpStart
$x \in \{-7\} \cup [13,15]$
\odpStop
\testStart
A.$x \in \{-7\} \cup [13,15]$\\
B.$x \in \{7\} \cup (13,15)$\\
C.$x \in \{-7\} \cup (13,15]$\\
D.$x \in \{7\} \cup (13,15]$\\
E.$x \in \{-7\} \cup [13,15)$\\
F.$x \in \{7\} \cup [13,15)$\\
G.$x \in \{-7\} \cup (13,15)$\\
H.$x \in \{7\} \cup [13,15]$
\testStop
\kluczStart
A
\kluczStop



\zadStart{Zadanie z Wikieł Z 1.62 c) moja wersja nr 793}

Rozwiązać nierówności $(13-x)(x+7)^{2}(16-x)^{3}\le0$.
\zadStop
\rozwStart{Patryk Wirkus}{}
Miejsca zerowe naszego wielomianu to: $13, -7, 16$.\\
Wielomian jest stopnia parzystego, ponadto znak współczynnika przy\linebreak najwyższej potędze x jest ujemny.\\ W związku z tym wykres wielomianu zaczyna się od lewej strony powyżej osi OX.\\
Ponadto w punkcie $-7$ wykres odbija się od osi poziomej.\\
A więc $$x \in \{-7\} \cup [13,16].$$
\rozwStop
\odpStart
$x \in \{-7\} \cup [13,16]$
\odpStop
\testStart
A.$x \in \{-7\} \cup [13,16]$\\
B.$x \in \{7\} \cup (13,16)$\\
C.$x \in \{-7\} \cup (13,16]$\\
D.$x \in \{7\} \cup (13,16]$\\
E.$x \in \{-7\} \cup [13,16)$\\
F.$x \in \{7\} \cup [13,16)$\\
G.$x \in \{-7\} \cup (13,16)$\\
H.$x \in \{7\} \cup [13,16]$
\testStop
\kluczStart
A
\kluczStop



\zadStart{Zadanie z Wikieł Z 1.62 c) moja wersja nr 794}

Rozwiązać nierówności $(13-x)(x+7)^{2}(17-x)^{3}\le0$.
\zadStop
\rozwStart{Patryk Wirkus}{}
Miejsca zerowe naszego wielomianu to: $13, -7, 17$.\\
Wielomian jest stopnia parzystego, ponadto znak współczynnika przy\linebreak najwyższej potędze x jest ujemny.\\ W związku z tym wykres wielomianu zaczyna się od lewej strony powyżej osi OX.\\
Ponadto w punkcie $-7$ wykres odbija się od osi poziomej.\\
A więc $$x \in \{-7\} \cup [13,17].$$
\rozwStop
\odpStart
$x \in \{-7\} \cup [13,17]$
\odpStop
\testStart
A.$x \in \{-7\} \cup [13,17]$\\
B.$x \in \{7\} \cup (13,17)$\\
C.$x \in \{-7\} \cup (13,17]$\\
D.$x \in \{7\} \cup (13,17]$\\
E.$x \in \{-7\} \cup [13,17)$\\
F.$x \in \{7\} \cup [13,17)$\\
G.$x \in \{-7\} \cup (13,17)$\\
H.$x \in \{7\} \cup [13,17]$
\testStop
\kluczStart
A
\kluczStop



\zadStart{Zadanie z Wikieł Z 1.62 c) moja wersja nr 795}

Rozwiązać nierówności $(13-x)(x+7)^{2}(18-x)^{3}\le0$.
\zadStop
\rozwStart{Patryk Wirkus}{}
Miejsca zerowe naszego wielomianu to: $13, -7, 18$.\\
Wielomian jest stopnia parzystego, ponadto znak współczynnika przy\linebreak najwyższej potędze x jest ujemny.\\ W związku z tym wykres wielomianu zaczyna się od lewej strony powyżej osi OX.\\
Ponadto w punkcie $-7$ wykres odbija się od osi poziomej.\\
A więc $$x \in \{-7\} \cup [13,18].$$
\rozwStop
\odpStart
$x \in \{-7\} \cup [13,18]$
\odpStop
\testStart
A.$x \in \{-7\} \cup [13,18]$\\
B.$x \in \{7\} \cup (13,18)$\\
C.$x \in \{-7\} \cup (13,18]$\\
D.$x \in \{7\} \cup (13,18]$\\
E.$x \in \{-7\} \cup [13,18)$\\
F.$x \in \{7\} \cup [13,18)$\\
G.$x \in \{-7\} \cup (13,18)$\\
H.$x \in \{7\} \cup [13,18]$
\testStop
\kluczStart
A
\kluczStop



\zadStart{Zadanie z Wikieł Z 1.62 c) moja wersja nr 796}

Rozwiązać nierówności $(13-x)(x+7)^{2}(19-x)^{3}\le0$.
\zadStop
\rozwStart{Patryk Wirkus}{}
Miejsca zerowe naszego wielomianu to: $13, -7, 19$.\\
Wielomian jest stopnia parzystego, ponadto znak współczynnika przy\linebreak najwyższej potędze x jest ujemny.\\ W związku z tym wykres wielomianu zaczyna się od lewej strony powyżej osi OX.\\
Ponadto w punkcie $-7$ wykres odbija się od osi poziomej.\\
A więc $$x \in \{-7\} \cup [13,19].$$
\rozwStop
\odpStart
$x \in \{-7\} \cup [13,19]$
\odpStop
\testStart
A.$x \in \{-7\} \cup [13,19]$\\
B.$x \in \{7\} \cup (13,19)$\\
C.$x \in \{-7\} \cup (13,19]$\\
D.$x \in \{7\} \cup (13,19]$\\
E.$x \in \{-7\} \cup [13,19)$\\
F.$x \in \{7\} \cup [13,19)$\\
G.$x \in \{-7\} \cup (13,19)$\\
H.$x \in \{7\} \cup [13,19]$
\testStop
\kluczStart
A
\kluczStop



\zadStart{Zadanie z Wikieł Z 1.62 c) moja wersja nr 797}

Rozwiązać nierówności $(13-x)(x+7)^{2}(20-x)^{3}\le0$.
\zadStop
\rozwStart{Patryk Wirkus}{}
Miejsca zerowe naszego wielomianu to: $13, -7, 20$.\\
Wielomian jest stopnia parzystego, ponadto znak współczynnika przy\linebreak najwyższej potędze x jest ujemny.\\ W związku z tym wykres wielomianu zaczyna się od lewej strony powyżej osi OX.\\
Ponadto w punkcie $-7$ wykres odbija się od osi poziomej.\\
A więc $$x \in \{-7\} \cup [13,20].$$
\rozwStop
\odpStart
$x \in \{-7\} \cup [13,20]$
\odpStop
\testStart
A.$x \in \{-7\} \cup [13,20]$\\
B.$x \in \{7\} \cup (13,20)$\\
C.$x \in \{-7\} \cup (13,20]$\\
D.$x \in \{7\} \cup (13,20]$\\
E.$x \in \{-7\} \cup [13,20)$\\
F.$x \in \{7\} \cup [13,20)$\\
G.$x \in \{-7\} \cup (13,20)$\\
H.$x \in \{7\} \cup [13,20]$
\testStop
\kluczStart
A
\kluczStop



\zadStart{Zadanie z Wikieł Z 1.62 c) moja wersja nr 798}

Rozwiązać nierówności $(13-x)(x+8)^{2}(14-x)^{3}\le0$.
\zadStop
\rozwStart{Patryk Wirkus}{}
Miejsca zerowe naszego wielomianu to: $13, -8, 14$.\\
Wielomian jest stopnia parzystego, ponadto znak współczynnika przy\linebreak najwyższej potędze x jest ujemny.\\ W związku z tym wykres wielomianu zaczyna się od lewej strony powyżej osi OX.\\
Ponadto w punkcie $-8$ wykres odbija się od osi poziomej.\\
A więc $$x \in \{-8\} \cup [13,14].$$
\rozwStop
\odpStart
$x \in \{-8\} \cup [13,14]$
\odpStop
\testStart
A.$x \in \{-8\} \cup [13,14]$\\
B.$x \in \{8\} \cup (13,14)$\\
C.$x \in \{-8\} \cup (13,14]$\\
D.$x \in \{8\} \cup (13,14]$\\
E.$x \in \{-8\} \cup [13,14)$\\
F.$x \in \{8\} \cup [13,14)$\\
G.$x \in \{-8\} \cup (13,14)$\\
H.$x \in \{8\} \cup [13,14]$
\testStop
\kluczStart
A
\kluczStop



\zadStart{Zadanie z Wikieł Z 1.62 c) moja wersja nr 799}

Rozwiązać nierówności $(13-x)(x+8)^{2}(15-x)^{3}\le0$.
\zadStop
\rozwStart{Patryk Wirkus}{}
Miejsca zerowe naszego wielomianu to: $13, -8, 15$.\\
Wielomian jest stopnia parzystego, ponadto znak współczynnika przy\linebreak najwyższej potędze x jest ujemny.\\ W związku z tym wykres wielomianu zaczyna się od lewej strony powyżej osi OX.\\
Ponadto w punkcie $-8$ wykres odbija się od osi poziomej.\\
A więc $$x \in \{-8\} \cup [13,15].$$
\rozwStop
\odpStart
$x \in \{-8\} \cup [13,15]$
\odpStop
\testStart
A.$x \in \{-8\} \cup [13,15]$\\
B.$x \in \{8\} \cup (13,15)$\\
C.$x \in \{-8\} \cup (13,15]$\\
D.$x \in \{8\} \cup (13,15]$\\
E.$x \in \{-8\} \cup [13,15)$\\
F.$x \in \{8\} \cup [13,15)$\\
G.$x \in \{-8\} \cup (13,15)$\\
H.$x \in \{8\} \cup [13,15]$
\testStop
\kluczStart
A
\kluczStop



\zadStart{Zadanie z Wikieł Z 1.62 c) moja wersja nr 800}

Rozwiązać nierówności $(13-x)(x+8)^{2}(16-x)^{3}\le0$.
\zadStop
\rozwStart{Patryk Wirkus}{}
Miejsca zerowe naszego wielomianu to: $13, -8, 16$.\\
Wielomian jest stopnia parzystego, ponadto znak współczynnika przy\linebreak najwyższej potędze x jest ujemny.\\ W związku z tym wykres wielomianu zaczyna się od lewej strony powyżej osi OX.\\
Ponadto w punkcie $-8$ wykres odbija się od osi poziomej.\\
A więc $$x \in \{-8\} \cup [13,16].$$
\rozwStop
\odpStart
$x \in \{-8\} \cup [13,16]$
\odpStop
\testStart
A.$x \in \{-8\} \cup [13,16]$\\
B.$x \in \{8\} \cup (13,16)$\\
C.$x \in \{-8\} \cup (13,16]$\\
D.$x \in \{8\} \cup (13,16]$\\
E.$x \in \{-8\} \cup [13,16)$\\
F.$x \in \{8\} \cup [13,16)$\\
G.$x \in \{-8\} \cup (13,16)$\\
H.$x \in \{8\} \cup [13,16]$
\testStop
\kluczStart
A
\kluczStop



\zadStart{Zadanie z Wikieł Z 1.62 c) moja wersja nr 801}

Rozwiązać nierówności $(13-x)(x+8)^{2}(17-x)^{3}\le0$.
\zadStop
\rozwStart{Patryk Wirkus}{}
Miejsca zerowe naszego wielomianu to: $13, -8, 17$.\\
Wielomian jest stopnia parzystego, ponadto znak współczynnika przy\linebreak najwyższej potędze x jest ujemny.\\ W związku z tym wykres wielomianu zaczyna się od lewej strony powyżej osi OX.\\
Ponadto w punkcie $-8$ wykres odbija się od osi poziomej.\\
A więc $$x \in \{-8\} \cup [13,17].$$
\rozwStop
\odpStart
$x \in \{-8\} \cup [13,17]$
\odpStop
\testStart
A.$x \in \{-8\} \cup [13,17]$\\
B.$x \in \{8\} \cup (13,17)$\\
C.$x \in \{-8\} \cup (13,17]$\\
D.$x \in \{8\} \cup (13,17]$\\
E.$x \in \{-8\} \cup [13,17)$\\
F.$x \in \{8\} \cup [13,17)$\\
G.$x \in \{-8\} \cup (13,17)$\\
H.$x \in \{8\} \cup [13,17]$
\testStop
\kluczStart
A
\kluczStop



\zadStart{Zadanie z Wikieł Z 1.62 c) moja wersja nr 802}

Rozwiązać nierówności $(13-x)(x+8)^{2}(18-x)^{3}\le0$.
\zadStop
\rozwStart{Patryk Wirkus}{}
Miejsca zerowe naszego wielomianu to: $13, -8, 18$.\\
Wielomian jest stopnia parzystego, ponadto znak współczynnika przy\linebreak najwyższej potędze x jest ujemny.\\ W związku z tym wykres wielomianu zaczyna się od lewej strony powyżej osi OX.\\
Ponadto w punkcie $-8$ wykres odbija się od osi poziomej.\\
A więc $$x \in \{-8\} \cup [13,18].$$
\rozwStop
\odpStart
$x \in \{-8\} \cup [13,18]$
\odpStop
\testStart
A.$x \in \{-8\} \cup [13,18]$\\
B.$x \in \{8\} \cup (13,18)$\\
C.$x \in \{-8\} \cup (13,18]$\\
D.$x \in \{8\} \cup (13,18]$\\
E.$x \in \{-8\} \cup [13,18)$\\
F.$x \in \{8\} \cup [13,18)$\\
G.$x \in \{-8\} \cup (13,18)$\\
H.$x \in \{8\} \cup [13,18]$
\testStop
\kluczStart
A
\kluczStop



\zadStart{Zadanie z Wikieł Z 1.62 c) moja wersja nr 803}

Rozwiązać nierówności $(13-x)(x+8)^{2}(19-x)^{3}\le0$.
\zadStop
\rozwStart{Patryk Wirkus}{}
Miejsca zerowe naszego wielomianu to: $13, -8, 19$.\\
Wielomian jest stopnia parzystego, ponadto znak współczynnika przy\linebreak najwyższej potędze x jest ujemny.\\ W związku z tym wykres wielomianu zaczyna się od lewej strony powyżej osi OX.\\
Ponadto w punkcie $-8$ wykres odbija się od osi poziomej.\\
A więc $$x \in \{-8\} \cup [13,19].$$
\rozwStop
\odpStart
$x \in \{-8\} \cup [13,19]$
\odpStop
\testStart
A.$x \in \{-8\} \cup [13,19]$\\
B.$x \in \{8\} \cup (13,19)$\\
C.$x \in \{-8\} \cup (13,19]$\\
D.$x \in \{8\} \cup (13,19]$\\
E.$x \in \{-8\} \cup [13,19)$\\
F.$x \in \{8\} \cup [13,19)$\\
G.$x \in \{-8\} \cup (13,19)$\\
H.$x \in \{8\} \cup [13,19]$
\testStop
\kluczStart
A
\kluczStop



\zadStart{Zadanie z Wikieł Z 1.62 c) moja wersja nr 804}

Rozwiązać nierówności $(13-x)(x+8)^{2}(20-x)^{3}\le0$.
\zadStop
\rozwStart{Patryk Wirkus}{}
Miejsca zerowe naszego wielomianu to: $13, -8, 20$.\\
Wielomian jest stopnia parzystego, ponadto znak współczynnika przy\linebreak najwyższej potędze x jest ujemny.\\ W związku z tym wykres wielomianu zaczyna się od lewej strony powyżej osi OX.\\
Ponadto w punkcie $-8$ wykres odbija się od osi poziomej.\\
A więc $$x \in \{-8\} \cup [13,20].$$
\rozwStop
\odpStart
$x \in \{-8\} \cup [13,20]$
\odpStop
\testStart
A.$x \in \{-8\} \cup [13,20]$\\
B.$x \in \{8\} \cup (13,20)$\\
C.$x \in \{-8\} \cup (13,20]$\\
D.$x \in \{8\} \cup (13,20]$\\
E.$x \in \{-8\} \cup [13,20)$\\
F.$x \in \{8\} \cup [13,20)$\\
G.$x \in \{-8\} \cup (13,20)$\\
H.$x \in \{8\} \cup [13,20]$
\testStop
\kluczStart
A
\kluczStop



\zadStart{Zadanie z Wikieł Z 1.62 c) moja wersja nr 805}

Rozwiązać nierówności $(13-x)(x+9)^{2}(14-x)^{3}\le0$.
\zadStop
\rozwStart{Patryk Wirkus}{}
Miejsca zerowe naszego wielomianu to: $13, -9, 14$.\\
Wielomian jest stopnia parzystego, ponadto znak współczynnika przy\linebreak najwyższej potędze x jest ujemny.\\ W związku z tym wykres wielomianu zaczyna się od lewej strony powyżej osi OX.\\
Ponadto w punkcie $-9$ wykres odbija się od osi poziomej.\\
A więc $$x \in \{-9\} \cup [13,14].$$
\rozwStop
\odpStart
$x \in \{-9\} \cup [13,14]$
\odpStop
\testStart
A.$x \in \{-9\} \cup [13,14]$\\
B.$x \in \{9\} \cup (13,14)$\\
C.$x \in \{-9\} \cup (13,14]$\\
D.$x \in \{9\} \cup (13,14]$\\
E.$x \in \{-9\} \cup [13,14)$\\
F.$x \in \{9\} \cup [13,14)$\\
G.$x \in \{-9\} \cup (13,14)$\\
H.$x \in \{9\} \cup [13,14]$
\testStop
\kluczStart
A
\kluczStop



\zadStart{Zadanie z Wikieł Z 1.62 c) moja wersja nr 806}

Rozwiązać nierówności $(13-x)(x+9)^{2}(15-x)^{3}\le0$.
\zadStop
\rozwStart{Patryk Wirkus}{}
Miejsca zerowe naszego wielomianu to: $13, -9, 15$.\\
Wielomian jest stopnia parzystego, ponadto znak współczynnika przy\linebreak najwyższej potędze x jest ujemny.\\ W związku z tym wykres wielomianu zaczyna się od lewej strony powyżej osi OX.\\
Ponadto w punkcie $-9$ wykres odbija się od osi poziomej.\\
A więc $$x \in \{-9\} \cup [13,15].$$
\rozwStop
\odpStart
$x \in \{-9\} \cup [13,15]$
\odpStop
\testStart
A.$x \in \{-9\} \cup [13,15]$\\
B.$x \in \{9\} \cup (13,15)$\\
C.$x \in \{-9\} \cup (13,15]$\\
D.$x \in \{9\} \cup (13,15]$\\
E.$x \in \{-9\} \cup [13,15)$\\
F.$x \in \{9\} \cup [13,15)$\\
G.$x \in \{-9\} \cup (13,15)$\\
H.$x \in \{9\} \cup [13,15]$
\testStop
\kluczStart
A
\kluczStop



\zadStart{Zadanie z Wikieł Z 1.62 c) moja wersja nr 807}

Rozwiązać nierówności $(13-x)(x+9)^{2}(16-x)^{3}\le0$.
\zadStop
\rozwStart{Patryk Wirkus}{}
Miejsca zerowe naszego wielomianu to: $13, -9, 16$.\\
Wielomian jest stopnia parzystego, ponadto znak współczynnika przy\linebreak najwyższej potędze x jest ujemny.\\ W związku z tym wykres wielomianu zaczyna się od lewej strony powyżej osi OX.\\
Ponadto w punkcie $-9$ wykres odbija się od osi poziomej.\\
A więc $$x \in \{-9\} \cup [13,16].$$
\rozwStop
\odpStart
$x \in \{-9\} \cup [13,16]$
\odpStop
\testStart
A.$x \in \{-9\} \cup [13,16]$\\
B.$x \in \{9\} \cup (13,16)$\\
C.$x \in \{-9\} \cup (13,16]$\\
D.$x \in \{9\} \cup (13,16]$\\
E.$x \in \{-9\} \cup [13,16)$\\
F.$x \in \{9\} \cup [13,16)$\\
G.$x \in \{-9\} \cup (13,16)$\\
H.$x \in \{9\} \cup [13,16]$
\testStop
\kluczStart
A
\kluczStop



\zadStart{Zadanie z Wikieł Z 1.62 c) moja wersja nr 808}

Rozwiązać nierówności $(13-x)(x+9)^{2}(17-x)^{3}\le0$.
\zadStop
\rozwStart{Patryk Wirkus}{}
Miejsca zerowe naszego wielomianu to: $13, -9, 17$.\\
Wielomian jest stopnia parzystego, ponadto znak współczynnika przy\linebreak najwyższej potędze x jest ujemny.\\ W związku z tym wykres wielomianu zaczyna się od lewej strony powyżej osi OX.\\
Ponadto w punkcie $-9$ wykres odbija się od osi poziomej.\\
A więc $$x \in \{-9\} \cup [13,17].$$
\rozwStop
\odpStart
$x \in \{-9\} \cup [13,17]$
\odpStop
\testStart
A.$x \in \{-9\} \cup [13,17]$\\
B.$x \in \{9\} \cup (13,17)$\\
C.$x \in \{-9\} \cup (13,17]$\\
D.$x \in \{9\} \cup (13,17]$\\
E.$x \in \{-9\} \cup [13,17)$\\
F.$x \in \{9\} \cup [13,17)$\\
G.$x \in \{-9\} \cup (13,17)$\\
H.$x \in \{9\} \cup [13,17]$
\testStop
\kluczStart
A
\kluczStop



\zadStart{Zadanie z Wikieł Z 1.62 c) moja wersja nr 809}

Rozwiązać nierówności $(13-x)(x+9)^{2}(18-x)^{3}\le0$.
\zadStop
\rozwStart{Patryk Wirkus}{}
Miejsca zerowe naszego wielomianu to: $13, -9, 18$.\\
Wielomian jest stopnia parzystego, ponadto znak współczynnika przy\linebreak najwyższej potędze x jest ujemny.\\ W związku z tym wykres wielomianu zaczyna się od lewej strony powyżej osi OX.\\
Ponadto w punkcie $-9$ wykres odbija się od osi poziomej.\\
A więc $$x \in \{-9\} \cup [13,18].$$
\rozwStop
\odpStart
$x \in \{-9\} \cup [13,18]$
\odpStop
\testStart
A.$x \in \{-9\} \cup [13,18]$\\
B.$x \in \{9\} \cup (13,18)$\\
C.$x \in \{-9\} \cup (13,18]$\\
D.$x \in \{9\} \cup (13,18]$\\
E.$x \in \{-9\} \cup [13,18)$\\
F.$x \in \{9\} \cup [13,18)$\\
G.$x \in \{-9\} \cup (13,18)$\\
H.$x \in \{9\} \cup [13,18]$
\testStop
\kluczStart
A
\kluczStop



\zadStart{Zadanie z Wikieł Z 1.62 c) moja wersja nr 810}

Rozwiązać nierówności $(13-x)(x+9)^{2}(19-x)^{3}\le0$.
\zadStop
\rozwStart{Patryk Wirkus}{}
Miejsca zerowe naszego wielomianu to: $13, -9, 19$.\\
Wielomian jest stopnia parzystego, ponadto znak współczynnika przy\linebreak najwyższej potędze x jest ujemny.\\ W związku z tym wykres wielomianu zaczyna się od lewej strony powyżej osi OX.\\
Ponadto w punkcie $-9$ wykres odbija się od osi poziomej.\\
A więc $$x \in \{-9\} \cup [13,19].$$
\rozwStop
\odpStart
$x \in \{-9\} \cup [13,19]$
\odpStop
\testStart
A.$x \in \{-9\} \cup [13,19]$\\
B.$x \in \{9\} \cup (13,19)$\\
C.$x \in \{-9\} \cup (13,19]$\\
D.$x \in \{9\} \cup (13,19]$\\
E.$x \in \{-9\} \cup [13,19)$\\
F.$x \in \{9\} \cup [13,19)$\\
G.$x \in \{-9\} \cup (13,19)$\\
H.$x \in \{9\} \cup [13,19]$
\testStop
\kluczStart
A
\kluczStop



\zadStart{Zadanie z Wikieł Z 1.62 c) moja wersja nr 811}

Rozwiązać nierówności $(13-x)(x+9)^{2}(20-x)^{3}\le0$.
\zadStop
\rozwStart{Patryk Wirkus}{}
Miejsca zerowe naszego wielomianu to: $13, -9, 20$.\\
Wielomian jest stopnia parzystego, ponadto znak współczynnika przy\linebreak najwyższej potędze x jest ujemny.\\ W związku z tym wykres wielomianu zaczyna się od lewej strony powyżej osi OX.\\
Ponadto w punkcie $-9$ wykres odbija się od osi poziomej.\\
A więc $$x \in \{-9\} \cup [13,20].$$
\rozwStop
\odpStart
$x \in \{-9\} \cup [13,20]$
\odpStop
\testStart
A.$x \in \{-9\} \cup [13,20]$\\
B.$x \in \{9\} \cup (13,20)$\\
C.$x \in \{-9\} \cup (13,20]$\\
D.$x \in \{9\} \cup (13,20]$\\
E.$x \in \{-9\} \cup [13,20)$\\
F.$x \in \{9\} \cup [13,20)$\\
G.$x \in \{-9\} \cup (13,20)$\\
H.$x \in \{9\} \cup [13,20]$
\testStop
\kluczStart
A
\kluczStop



\zadStart{Zadanie z Wikieł Z 1.62 c) moja wersja nr 812}

Rozwiązać nierówności $(13-x)(x+10)^{2}(14-x)^{3}\le0$.
\zadStop
\rozwStart{Patryk Wirkus}{}
Miejsca zerowe naszego wielomianu to: $13, -10, 14$.\\
Wielomian jest stopnia parzystego, ponadto znak współczynnika przy\linebreak najwyższej potędze x jest ujemny.\\ W związku z tym wykres wielomianu zaczyna się od lewej strony powyżej osi OX.\\
Ponadto w punkcie $-10$ wykres odbija się od osi poziomej.\\
A więc $$x \in \{-10\} \cup [13,14].$$
\rozwStop
\odpStart
$x \in \{-10\} \cup [13,14]$
\odpStop
\testStart
A.$x \in \{-10\} \cup [13,14]$\\
B.$x \in \{10\} \cup (13,14)$\\
C.$x \in \{-10\} \cup (13,14]$\\
D.$x \in \{10\} \cup (13,14]$\\
E.$x \in \{-10\} \cup [13,14)$\\
F.$x \in \{10\} \cup [13,14)$\\
G.$x \in \{-10\} \cup (13,14)$\\
H.$x \in \{10\} \cup [13,14]$
\testStop
\kluczStart
A
\kluczStop



\zadStart{Zadanie z Wikieł Z 1.62 c) moja wersja nr 813}

Rozwiązać nierówności $(13-x)(x+10)^{2}(15-x)^{3}\le0$.
\zadStop
\rozwStart{Patryk Wirkus}{}
Miejsca zerowe naszego wielomianu to: $13, -10, 15$.\\
Wielomian jest stopnia parzystego, ponadto znak współczynnika przy\linebreak najwyższej potędze x jest ujemny.\\ W związku z tym wykres wielomianu zaczyna się od lewej strony powyżej osi OX.\\
Ponadto w punkcie $-10$ wykres odbija się od osi poziomej.\\
A więc $$x \in \{-10\} \cup [13,15].$$
\rozwStop
\odpStart
$x \in \{-10\} \cup [13,15]$
\odpStop
\testStart
A.$x \in \{-10\} \cup [13,15]$\\
B.$x \in \{10\} \cup (13,15)$\\
C.$x \in \{-10\} \cup (13,15]$\\
D.$x \in \{10\} \cup (13,15]$\\
E.$x \in \{-10\} \cup [13,15)$\\
F.$x \in \{10\} \cup [13,15)$\\
G.$x \in \{-10\} \cup (13,15)$\\
H.$x \in \{10\} \cup [13,15]$
\testStop
\kluczStart
A
\kluczStop



\zadStart{Zadanie z Wikieł Z 1.62 c) moja wersja nr 814}

Rozwiązać nierówności $(13-x)(x+10)^{2}(16-x)^{3}\le0$.
\zadStop
\rozwStart{Patryk Wirkus}{}
Miejsca zerowe naszego wielomianu to: $13, -10, 16$.\\
Wielomian jest stopnia parzystego, ponadto znak współczynnika przy\linebreak najwyższej potędze x jest ujemny.\\ W związku z tym wykres wielomianu zaczyna się od lewej strony powyżej osi OX.\\
Ponadto w punkcie $-10$ wykres odbija się od osi poziomej.\\
A więc $$x \in \{-10\} \cup [13,16].$$
\rozwStop
\odpStart
$x \in \{-10\} \cup [13,16]$
\odpStop
\testStart
A.$x \in \{-10\} \cup [13,16]$\\
B.$x \in \{10\} \cup (13,16)$\\
C.$x \in \{-10\} \cup (13,16]$\\
D.$x \in \{10\} \cup (13,16]$\\
E.$x \in \{-10\} \cup [13,16)$\\
F.$x \in \{10\} \cup [13,16)$\\
G.$x \in \{-10\} \cup (13,16)$\\
H.$x \in \{10\} \cup [13,16]$
\testStop
\kluczStart
A
\kluczStop



\zadStart{Zadanie z Wikieł Z 1.62 c) moja wersja nr 815}

Rozwiązać nierówności $(13-x)(x+10)^{2}(17-x)^{3}\le0$.
\zadStop
\rozwStart{Patryk Wirkus}{}
Miejsca zerowe naszego wielomianu to: $13, -10, 17$.\\
Wielomian jest stopnia parzystego, ponadto znak współczynnika przy\linebreak najwyższej potędze x jest ujemny.\\ W związku z tym wykres wielomianu zaczyna się od lewej strony powyżej osi OX.\\
Ponadto w punkcie $-10$ wykres odbija się od osi poziomej.\\
A więc $$x \in \{-10\} \cup [13,17].$$
\rozwStop
\odpStart
$x \in \{-10\} \cup [13,17]$
\odpStop
\testStart
A.$x \in \{-10\} \cup [13,17]$\\
B.$x \in \{10\} \cup (13,17)$\\
C.$x \in \{-10\} \cup (13,17]$\\
D.$x \in \{10\} \cup (13,17]$\\
E.$x \in \{-10\} \cup [13,17)$\\
F.$x \in \{10\} \cup [13,17)$\\
G.$x \in \{-10\} \cup (13,17)$\\
H.$x \in \{10\} \cup [13,17]$
\testStop
\kluczStart
A
\kluczStop



\zadStart{Zadanie z Wikieł Z 1.62 c) moja wersja nr 816}

Rozwiązać nierówności $(13-x)(x+10)^{2}(18-x)^{3}\le0$.
\zadStop
\rozwStart{Patryk Wirkus}{}
Miejsca zerowe naszego wielomianu to: $13, -10, 18$.\\
Wielomian jest stopnia parzystego, ponadto znak współczynnika przy\linebreak najwyższej potędze x jest ujemny.\\ W związku z tym wykres wielomianu zaczyna się od lewej strony powyżej osi OX.\\
Ponadto w punkcie $-10$ wykres odbija się od osi poziomej.\\
A więc $$x \in \{-10\} \cup [13,18].$$
\rozwStop
\odpStart
$x \in \{-10\} \cup [13,18]$
\odpStop
\testStart
A.$x \in \{-10\} \cup [13,18]$\\
B.$x \in \{10\} \cup (13,18)$\\
C.$x \in \{-10\} \cup (13,18]$\\
D.$x \in \{10\} \cup (13,18]$\\
E.$x \in \{-10\} \cup [13,18)$\\
F.$x \in \{10\} \cup [13,18)$\\
G.$x \in \{-10\} \cup (13,18)$\\
H.$x \in \{10\} \cup [13,18]$
\testStop
\kluczStart
A
\kluczStop



\zadStart{Zadanie z Wikieł Z 1.62 c) moja wersja nr 817}

Rozwiązać nierówności $(13-x)(x+10)^{2}(19-x)^{3}\le0$.
\zadStop
\rozwStart{Patryk Wirkus}{}
Miejsca zerowe naszego wielomianu to: $13, -10, 19$.\\
Wielomian jest stopnia parzystego, ponadto znak współczynnika przy\linebreak najwyższej potędze x jest ujemny.\\ W związku z tym wykres wielomianu zaczyna się od lewej strony powyżej osi OX.\\
Ponadto w punkcie $-10$ wykres odbija się od osi poziomej.\\
A więc $$x \in \{-10\} \cup [13,19].$$
\rozwStop
\odpStart
$x \in \{-10\} \cup [13,19]$
\odpStop
\testStart
A.$x \in \{-10\} \cup [13,19]$\\
B.$x \in \{10\} \cup (13,19)$\\
C.$x \in \{-10\} \cup (13,19]$\\
D.$x \in \{10\} \cup (13,19]$\\
E.$x \in \{-10\} \cup [13,19)$\\
F.$x \in \{10\} \cup [13,19)$\\
G.$x \in \{-10\} \cup (13,19)$\\
H.$x \in \{10\} \cup [13,19]$
\testStop
\kluczStart
A
\kluczStop



\zadStart{Zadanie z Wikieł Z 1.62 c) moja wersja nr 818}

Rozwiązać nierówności $(13-x)(x+10)^{2}(20-x)^{3}\le0$.
\zadStop
\rozwStart{Patryk Wirkus}{}
Miejsca zerowe naszego wielomianu to: $13, -10, 20$.\\
Wielomian jest stopnia parzystego, ponadto znak współczynnika przy\linebreak najwyższej potędze x jest ujemny.\\ W związku z tym wykres wielomianu zaczyna się od lewej strony powyżej osi OX.\\
Ponadto w punkcie $-10$ wykres odbija się od osi poziomej.\\
A więc $$x \in \{-10\} \cup [13,20].$$
\rozwStop
\odpStart
$x \in \{-10\} \cup [13,20]$
\odpStop
\testStart
A.$x \in \{-10\} \cup [13,20]$\\
B.$x \in \{10\} \cup (13,20)$\\
C.$x \in \{-10\} \cup (13,20]$\\
D.$x \in \{10\} \cup (13,20]$\\
E.$x \in \{-10\} \cup [13,20)$\\
F.$x \in \{10\} \cup [13,20)$\\
G.$x \in \{-10\} \cup (13,20)$\\
H.$x \in \{10\} \cup [13,20]$
\testStop
\kluczStart
A
\kluczStop



\zadStart{Zadanie z Wikieł Z 1.62 c) moja wersja nr 819}

Rozwiązać nierówności $(13-x)(x+11)^{2}(14-x)^{3}\le0$.
\zadStop
\rozwStart{Patryk Wirkus}{}
Miejsca zerowe naszego wielomianu to: $13, -11, 14$.\\
Wielomian jest stopnia parzystego, ponadto znak współczynnika przy\linebreak najwyższej potędze x jest ujemny.\\ W związku z tym wykres wielomianu zaczyna się od lewej strony powyżej osi OX.\\
Ponadto w punkcie $-11$ wykres odbija się od osi poziomej.\\
A więc $$x \in \{-11\} \cup [13,14].$$
\rozwStop
\odpStart
$x \in \{-11\} \cup [13,14]$
\odpStop
\testStart
A.$x \in \{-11\} \cup [13,14]$\\
B.$x \in \{11\} \cup (13,14)$\\
C.$x \in \{-11\} \cup (13,14]$\\
D.$x \in \{11\} \cup (13,14]$\\
E.$x \in \{-11\} \cup [13,14)$\\
F.$x \in \{11\} \cup [13,14)$\\
G.$x \in \{-11\} \cup (13,14)$\\
H.$x \in \{11\} \cup [13,14]$
\testStop
\kluczStart
A
\kluczStop



\zadStart{Zadanie z Wikieł Z 1.62 c) moja wersja nr 820}

Rozwiązać nierówności $(13-x)(x+11)^{2}(15-x)^{3}\le0$.
\zadStop
\rozwStart{Patryk Wirkus}{}
Miejsca zerowe naszego wielomianu to: $13, -11, 15$.\\
Wielomian jest stopnia parzystego, ponadto znak współczynnika przy\linebreak najwyższej potędze x jest ujemny.\\ W związku z tym wykres wielomianu zaczyna się od lewej strony powyżej osi OX.\\
Ponadto w punkcie $-11$ wykres odbija się od osi poziomej.\\
A więc $$x \in \{-11\} \cup [13,15].$$
\rozwStop
\odpStart
$x \in \{-11\} \cup [13,15]$
\odpStop
\testStart
A.$x \in \{-11\} \cup [13,15]$\\
B.$x \in \{11\} \cup (13,15)$\\
C.$x \in \{-11\} \cup (13,15]$\\
D.$x \in \{11\} \cup (13,15]$\\
E.$x \in \{-11\} \cup [13,15)$\\
F.$x \in \{11\} \cup [13,15)$\\
G.$x \in \{-11\} \cup (13,15)$\\
H.$x \in \{11\} \cup [13,15]$
\testStop
\kluczStart
A
\kluczStop



\zadStart{Zadanie z Wikieł Z 1.62 c) moja wersja nr 821}

Rozwiązać nierówności $(13-x)(x+11)^{2}(16-x)^{3}\le0$.
\zadStop
\rozwStart{Patryk Wirkus}{}
Miejsca zerowe naszego wielomianu to: $13, -11, 16$.\\
Wielomian jest stopnia parzystego, ponadto znak współczynnika przy\linebreak najwyższej potędze x jest ujemny.\\ W związku z tym wykres wielomianu zaczyna się od lewej strony powyżej osi OX.\\
Ponadto w punkcie $-11$ wykres odbija się od osi poziomej.\\
A więc $$x \in \{-11\} \cup [13,16].$$
\rozwStop
\odpStart
$x \in \{-11\} \cup [13,16]$
\odpStop
\testStart
A.$x \in \{-11\} \cup [13,16]$\\
B.$x \in \{11\} \cup (13,16)$\\
C.$x \in \{-11\} \cup (13,16]$\\
D.$x \in \{11\} \cup (13,16]$\\
E.$x \in \{-11\} \cup [13,16)$\\
F.$x \in \{11\} \cup [13,16)$\\
G.$x \in \{-11\} \cup (13,16)$\\
H.$x \in \{11\} \cup [13,16]$
\testStop
\kluczStart
A
\kluczStop



\zadStart{Zadanie z Wikieł Z 1.62 c) moja wersja nr 822}

Rozwiązać nierówności $(13-x)(x+11)^{2}(17-x)^{3}\le0$.
\zadStop
\rozwStart{Patryk Wirkus}{}
Miejsca zerowe naszego wielomianu to: $13, -11, 17$.\\
Wielomian jest stopnia parzystego, ponadto znak współczynnika przy\linebreak najwyższej potędze x jest ujemny.\\ W związku z tym wykres wielomianu zaczyna się od lewej strony powyżej osi OX.\\
Ponadto w punkcie $-11$ wykres odbija się od osi poziomej.\\
A więc $$x \in \{-11\} \cup [13,17].$$
\rozwStop
\odpStart
$x \in \{-11\} \cup [13,17]$
\odpStop
\testStart
A.$x \in \{-11\} \cup [13,17]$\\
B.$x \in \{11\} \cup (13,17)$\\
C.$x \in \{-11\} \cup (13,17]$\\
D.$x \in \{11\} \cup (13,17]$\\
E.$x \in \{-11\} \cup [13,17)$\\
F.$x \in \{11\} \cup [13,17)$\\
G.$x \in \{-11\} \cup (13,17)$\\
H.$x \in \{11\} \cup [13,17]$
\testStop
\kluczStart
A
\kluczStop



\zadStart{Zadanie z Wikieł Z 1.62 c) moja wersja nr 823}

Rozwiązać nierówności $(13-x)(x+11)^{2}(18-x)^{3}\le0$.
\zadStop
\rozwStart{Patryk Wirkus}{}
Miejsca zerowe naszego wielomianu to: $13, -11, 18$.\\
Wielomian jest stopnia parzystego, ponadto znak współczynnika przy\linebreak najwyższej potędze x jest ujemny.\\ W związku z tym wykres wielomianu zaczyna się od lewej strony powyżej osi OX.\\
Ponadto w punkcie $-11$ wykres odbija się od osi poziomej.\\
A więc $$x \in \{-11\} \cup [13,18].$$
\rozwStop
\odpStart
$x \in \{-11\} \cup [13,18]$
\odpStop
\testStart
A.$x \in \{-11\} \cup [13,18]$\\
B.$x \in \{11\} \cup (13,18)$\\
C.$x \in \{-11\} \cup (13,18]$\\
D.$x \in \{11\} \cup (13,18]$\\
E.$x \in \{-11\} \cup [13,18)$\\
F.$x \in \{11\} \cup [13,18)$\\
G.$x \in \{-11\} \cup (13,18)$\\
H.$x \in \{11\} \cup [13,18]$
\testStop
\kluczStart
A
\kluczStop



\zadStart{Zadanie z Wikieł Z 1.62 c) moja wersja nr 824}

Rozwiązać nierówności $(13-x)(x+11)^{2}(19-x)^{3}\le0$.
\zadStop
\rozwStart{Patryk Wirkus}{}
Miejsca zerowe naszego wielomianu to: $13, -11, 19$.\\
Wielomian jest stopnia parzystego, ponadto znak współczynnika przy\linebreak najwyższej potędze x jest ujemny.\\ W związku z tym wykres wielomianu zaczyna się od lewej strony powyżej osi OX.\\
Ponadto w punkcie $-11$ wykres odbija się od osi poziomej.\\
A więc $$x \in \{-11\} \cup [13,19].$$
\rozwStop
\odpStart
$x \in \{-11\} \cup [13,19]$
\odpStop
\testStart
A.$x \in \{-11\} \cup [13,19]$\\
B.$x \in \{11\} \cup (13,19)$\\
C.$x \in \{-11\} \cup (13,19]$\\
D.$x \in \{11\} \cup (13,19]$\\
E.$x \in \{-11\} \cup [13,19)$\\
F.$x \in \{11\} \cup [13,19)$\\
G.$x \in \{-11\} \cup (13,19)$\\
H.$x \in \{11\} \cup [13,19]$
\testStop
\kluczStart
A
\kluczStop



\zadStart{Zadanie z Wikieł Z 1.62 c) moja wersja nr 825}

Rozwiązać nierówności $(13-x)(x+11)^{2}(20-x)^{3}\le0$.
\zadStop
\rozwStart{Patryk Wirkus}{}
Miejsca zerowe naszego wielomianu to: $13, -11, 20$.\\
Wielomian jest stopnia parzystego, ponadto znak współczynnika przy\linebreak najwyższej potędze x jest ujemny.\\ W związku z tym wykres wielomianu zaczyna się od lewej strony powyżej osi OX.\\
Ponadto w punkcie $-11$ wykres odbija się od osi poziomej.\\
A więc $$x \in \{-11\} \cup [13,20].$$
\rozwStop
\odpStart
$x \in \{-11\} \cup [13,20]$
\odpStop
\testStart
A.$x \in \{-11\} \cup [13,20]$\\
B.$x \in \{11\} \cup (13,20)$\\
C.$x \in \{-11\} \cup (13,20]$\\
D.$x \in \{11\} \cup (13,20]$\\
E.$x \in \{-11\} \cup [13,20)$\\
F.$x \in \{11\} \cup [13,20)$\\
G.$x \in \{-11\} \cup (13,20)$\\
H.$x \in \{11\} \cup [13,20]$
\testStop
\kluczStart
A
\kluczStop



\zadStart{Zadanie z Wikieł Z 1.62 c) moja wersja nr 826}

Rozwiązać nierówności $(13-x)(x+12)^{2}(14-x)^{3}\le0$.
\zadStop
\rozwStart{Patryk Wirkus}{}
Miejsca zerowe naszego wielomianu to: $13, -12, 14$.\\
Wielomian jest stopnia parzystego, ponadto znak współczynnika przy\linebreak najwyższej potędze x jest ujemny.\\ W związku z tym wykres wielomianu zaczyna się od lewej strony powyżej osi OX.\\
Ponadto w punkcie $-12$ wykres odbija się od osi poziomej.\\
A więc $$x \in \{-12\} \cup [13,14].$$
\rozwStop
\odpStart
$x \in \{-12\} \cup [13,14]$
\odpStop
\testStart
A.$x \in \{-12\} \cup [13,14]$\\
B.$x \in \{12\} \cup (13,14)$\\
C.$x \in \{-12\} \cup (13,14]$\\
D.$x \in \{12\} \cup (13,14]$\\
E.$x \in \{-12\} \cup [13,14)$\\
F.$x \in \{12\} \cup [13,14)$\\
G.$x \in \{-12\} \cup (13,14)$\\
H.$x \in \{12\} \cup [13,14]$
\testStop
\kluczStart
A
\kluczStop



\zadStart{Zadanie z Wikieł Z 1.62 c) moja wersja nr 827}

Rozwiązać nierówności $(13-x)(x+12)^{2}(15-x)^{3}\le0$.
\zadStop
\rozwStart{Patryk Wirkus}{}
Miejsca zerowe naszego wielomianu to: $13, -12, 15$.\\
Wielomian jest stopnia parzystego, ponadto znak współczynnika przy\linebreak najwyższej potędze x jest ujemny.\\ W związku z tym wykres wielomianu zaczyna się od lewej strony powyżej osi OX.\\
Ponadto w punkcie $-12$ wykres odbija się od osi poziomej.\\
A więc $$x \in \{-12\} \cup [13,15].$$
\rozwStop
\odpStart
$x \in \{-12\} \cup [13,15]$
\odpStop
\testStart
A.$x \in \{-12\} \cup [13,15]$\\
B.$x \in \{12\} \cup (13,15)$\\
C.$x \in \{-12\} \cup (13,15]$\\
D.$x \in \{12\} \cup (13,15]$\\
E.$x \in \{-12\} \cup [13,15)$\\
F.$x \in \{12\} \cup [13,15)$\\
G.$x \in \{-12\} \cup (13,15)$\\
H.$x \in \{12\} \cup [13,15]$
\testStop
\kluczStart
A
\kluczStop



\zadStart{Zadanie z Wikieł Z 1.62 c) moja wersja nr 828}

Rozwiązać nierówności $(13-x)(x+12)^{2}(16-x)^{3}\le0$.
\zadStop
\rozwStart{Patryk Wirkus}{}
Miejsca zerowe naszego wielomianu to: $13, -12, 16$.\\
Wielomian jest stopnia parzystego, ponadto znak współczynnika przy\linebreak najwyższej potędze x jest ujemny.\\ W związku z tym wykres wielomianu zaczyna się od lewej strony powyżej osi OX.\\
Ponadto w punkcie $-12$ wykres odbija się od osi poziomej.\\
A więc $$x \in \{-12\} \cup [13,16].$$
\rozwStop
\odpStart
$x \in \{-12\} \cup [13,16]$
\odpStop
\testStart
A.$x \in \{-12\} \cup [13,16]$\\
B.$x \in \{12\} \cup (13,16)$\\
C.$x \in \{-12\} \cup (13,16]$\\
D.$x \in \{12\} \cup (13,16]$\\
E.$x \in \{-12\} \cup [13,16)$\\
F.$x \in \{12\} \cup [13,16)$\\
G.$x \in \{-12\} \cup (13,16)$\\
H.$x \in \{12\} \cup [13,16]$
\testStop
\kluczStart
A
\kluczStop



\zadStart{Zadanie z Wikieł Z 1.62 c) moja wersja nr 829}

Rozwiązać nierówności $(13-x)(x+12)^{2}(17-x)^{3}\le0$.
\zadStop
\rozwStart{Patryk Wirkus}{}
Miejsca zerowe naszego wielomianu to: $13, -12, 17$.\\
Wielomian jest stopnia parzystego, ponadto znak współczynnika przy\linebreak najwyższej potędze x jest ujemny.\\ W związku z tym wykres wielomianu zaczyna się od lewej strony powyżej osi OX.\\
Ponadto w punkcie $-12$ wykres odbija się od osi poziomej.\\
A więc $$x \in \{-12\} \cup [13,17].$$
\rozwStop
\odpStart
$x \in \{-12\} \cup [13,17]$
\odpStop
\testStart
A.$x \in \{-12\} \cup [13,17]$\\
B.$x \in \{12\} \cup (13,17)$\\
C.$x \in \{-12\} \cup (13,17]$\\
D.$x \in \{12\} \cup (13,17]$\\
E.$x \in \{-12\} \cup [13,17)$\\
F.$x \in \{12\} \cup [13,17)$\\
G.$x \in \{-12\} \cup (13,17)$\\
H.$x \in \{12\} \cup [13,17]$
\testStop
\kluczStart
A
\kluczStop



\zadStart{Zadanie z Wikieł Z 1.62 c) moja wersja nr 830}

Rozwiązać nierówności $(13-x)(x+12)^{2}(18-x)^{3}\le0$.
\zadStop
\rozwStart{Patryk Wirkus}{}
Miejsca zerowe naszego wielomianu to: $13, -12, 18$.\\
Wielomian jest stopnia parzystego, ponadto znak współczynnika przy\linebreak najwyższej potędze x jest ujemny.\\ W związku z tym wykres wielomianu zaczyna się od lewej strony powyżej osi OX.\\
Ponadto w punkcie $-12$ wykres odbija się od osi poziomej.\\
A więc $$x \in \{-12\} \cup [13,18].$$
\rozwStop
\odpStart
$x \in \{-12\} \cup [13,18]$
\odpStop
\testStart
A.$x \in \{-12\} \cup [13,18]$\\
B.$x \in \{12\} \cup (13,18)$\\
C.$x \in \{-12\} \cup (13,18]$\\
D.$x \in \{12\} \cup (13,18]$\\
E.$x \in \{-12\} \cup [13,18)$\\
F.$x \in \{12\} \cup [13,18)$\\
G.$x \in \{-12\} \cup (13,18)$\\
H.$x \in \{12\} \cup [13,18]$
\testStop
\kluczStart
A
\kluczStop



\zadStart{Zadanie z Wikieł Z 1.62 c) moja wersja nr 831}

Rozwiązać nierówności $(13-x)(x+12)^{2}(19-x)^{3}\le0$.
\zadStop
\rozwStart{Patryk Wirkus}{}
Miejsca zerowe naszego wielomianu to: $13, -12, 19$.\\
Wielomian jest stopnia parzystego, ponadto znak współczynnika przy\linebreak najwyższej potędze x jest ujemny.\\ W związku z tym wykres wielomianu zaczyna się od lewej strony powyżej osi OX.\\
Ponadto w punkcie $-12$ wykres odbija się od osi poziomej.\\
A więc $$x \in \{-12\} \cup [13,19].$$
\rozwStop
\odpStart
$x \in \{-12\} \cup [13,19]$
\odpStop
\testStart
A.$x \in \{-12\} \cup [13,19]$\\
B.$x \in \{12\} \cup (13,19)$\\
C.$x \in \{-12\} \cup (13,19]$\\
D.$x \in \{12\} \cup (13,19]$\\
E.$x \in \{-12\} \cup [13,19)$\\
F.$x \in \{12\} \cup [13,19)$\\
G.$x \in \{-12\} \cup (13,19)$\\
H.$x \in \{12\} \cup [13,19]$
\testStop
\kluczStart
A
\kluczStop



\zadStart{Zadanie z Wikieł Z 1.62 c) moja wersja nr 832}

Rozwiązać nierówności $(13-x)(x+12)^{2}(20-x)^{3}\le0$.
\zadStop
\rozwStart{Patryk Wirkus}{}
Miejsca zerowe naszego wielomianu to: $13, -12, 20$.\\
Wielomian jest stopnia parzystego, ponadto znak współczynnika przy\linebreak najwyższej potędze x jest ujemny.\\ W związku z tym wykres wielomianu zaczyna się od lewej strony powyżej osi OX.\\
Ponadto w punkcie $-12$ wykres odbija się od osi poziomej.\\
A więc $$x \in \{-12\} \cup [13,20].$$
\rozwStop
\odpStart
$x \in \{-12\} \cup [13,20]$
\odpStop
\testStart
A.$x \in \{-12\} \cup [13,20]$\\
B.$x \in \{12\} \cup (13,20)$\\
C.$x \in \{-12\} \cup (13,20]$\\
D.$x \in \{12\} \cup (13,20]$\\
E.$x \in \{-12\} \cup [13,20)$\\
F.$x \in \{12\} \cup [13,20)$\\
G.$x \in \{-12\} \cup (13,20)$\\
H.$x \in \{12\} \cup [13,20]$
\testStop
\kluczStart
A
\kluczStop



\zadStart{Zadanie z Wikieł Z 1.62 c) moja wersja nr 833}

Rozwiązać nierówności $(14-x)(x+1)^{2}(15-x)^{3}\le0$.
\zadStop
\rozwStart{Patryk Wirkus}{}
Miejsca zerowe naszego wielomianu to: $14, -1, 15$.\\
Wielomian jest stopnia parzystego, ponadto znak współczynnika przy\linebreak najwyższej potędze x jest ujemny.\\ W związku z tym wykres wielomianu zaczyna się od lewej strony powyżej osi OX.\\
Ponadto w punkcie $-1$ wykres odbija się od osi poziomej.\\
A więc $$x \in \{-1\} \cup [14,15].$$
\rozwStop
\odpStart
$x \in \{-1\} \cup [14,15]$
\odpStop
\testStart
A.$x \in \{-1\} \cup [14,15]$\\
B.$x \in \{1\} \cup (14,15)$\\
C.$x \in \{-1\} \cup (14,15]$\\
D.$x \in \{1\} \cup (14,15]$\\
E.$x \in \{-1\} \cup [14,15)$\\
F.$x \in \{1\} \cup [14,15)$\\
G.$x \in \{-1\} \cup (14,15)$\\
H.$x \in \{1\} \cup [14,15]$
\testStop
\kluczStart
A
\kluczStop



\zadStart{Zadanie z Wikieł Z 1.62 c) moja wersja nr 834}

Rozwiązać nierówności $(14-x)(x+1)^{2}(16-x)^{3}\le0$.
\zadStop
\rozwStart{Patryk Wirkus}{}
Miejsca zerowe naszego wielomianu to: $14, -1, 16$.\\
Wielomian jest stopnia parzystego, ponadto znak współczynnika przy\linebreak najwyższej potędze x jest ujemny.\\ W związku z tym wykres wielomianu zaczyna się od lewej strony powyżej osi OX.\\
Ponadto w punkcie $-1$ wykres odbija się od osi poziomej.\\
A więc $$x \in \{-1\} \cup [14,16].$$
\rozwStop
\odpStart
$x \in \{-1\} \cup [14,16]$
\odpStop
\testStart
A.$x \in \{-1\} \cup [14,16]$\\
B.$x \in \{1\} \cup (14,16)$\\
C.$x \in \{-1\} \cup (14,16]$\\
D.$x \in \{1\} \cup (14,16]$\\
E.$x \in \{-1\} \cup [14,16)$\\
F.$x \in \{1\} \cup [14,16)$\\
G.$x \in \{-1\} \cup (14,16)$\\
H.$x \in \{1\} \cup [14,16]$
\testStop
\kluczStart
A
\kluczStop



\zadStart{Zadanie z Wikieł Z 1.62 c) moja wersja nr 835}

Rozwiązać nierówności $(14-x)(x+1)^{2}(17-x)^{3}\le0$.
\zadStop
\rozwStart{Patryk Wirkus}{}
Miejsca zerowe naszego wielomianu to: $14, -1, 17$.\\
Wielomian jest stopnia parzystego, ponadto znak współczynnika przy\linebreak najwyższej potędze x jest ujemny.\\ W związku z tym wykres wielomianu zaczyna się od lewej strony powyżej osi OX.\\
Ponadto w punkcie $-1$ wykres odbija się od osi poziomej.\\
A więc $$x \in \{-1\} \cup [14,17].$$
\rozwStop
\odpStart
$x \in \{-1\} \cup [14,17]$
\odpStop
\testStart
A.$x \in \{-1\} \cup [14,17]$\\
B.$x \in \{1\} \cup (14,17)$\\
C.$x \in \{-1\} \cup (14,17]$\\
D.$x \in \{1\} \cup (14,17]$\\
E.$x \in \{-1\} \cup [14,17)$\\
F.$x \in \{1\} \cup [14,17)$\\
G.$x \in \{-1\} \cup (14,17)$\\
H.$x \in \{1\} \cup [14,17]$
\testStop
\kluczStart
A
\kluczStop



\zadStart{Zadanie z Wikieł Z 1.62 c) moja wersja nr 836}

Rozwiązać nierówności $(14-x)(x+1)^{2}(18-x)^{3}\le0$.
\zadStop
\rozwStart{Patryk Wirkus}{}
Miejsca zerowe naszego wielomianu to: $14, -1, 18$.\\
Wielomian jest stopnia parzystego, ponadto znak współczynnika przy\linebreak najwyższej potędze x jest ujemny.\\ W związku z tym wykres wielomianu zaczyna się od lewej strony powyżej osi OX.\\
Ponadto w punkcie $-1$ wykres odbija się od osi poziomej.\\
A więc $$x \in \{-1\} \cup [14,18].$$
\rozwStop
\odpStart
$x \in \{-1\} \cup [14,18]$
\odpStop
\testStart
A.$x \in \{-1\} \cup [14,18]$\\
B.$x \in \{1\} \cup (14,18)$\\
C.$x \in \{-1\} \cup (14,18]$\\
D.$x \in \{1\} \cup (14,18]$\\
E.$x \in \{-1\} \cup [14,18)$\\
F.$x \in \{1\} \cup [14,18)$\\
G.$x \in \{-1\} \cup (14,18)$\\
H.$x \in \{1\} \cup [14,18]$
\testStop
\kluczStart
A
\kluczStop



\zadStart{Zadanie z Wikieł Z 1.62 c) moja wersja nr 837}

Rozwiązać nierówności $(14-x)(x+1)^{2}(19-x)^{3}\le0$.
\zadStop
\rozwStart{Patryk Wirkus}{}
Miejsca zerowe naszego wielomianu to: $14, -1, 19$.\\
Wielomian jest stopnia parzystego, ponadto znak współczynnika przy\linebreak najwyższej potędze x jest ujemny.\\ W związku z tym wykres wielomianu zaczyna się od lewej strony powyżej osi OX.\\
Ponadto w punkcie $-1$ wykres odbija się od osi poziomej.\\
A więc $$x \in \{-1\} \cup [14,19].$$
\rozwStop
\odpStart
$x \in \{-1\} \cup [14,19]$
\odpStop
\testStart
A.$x \in \{-1\} \cup [14,19]$\\
B.$x \in \{1\} \cup (14,19)$\\
C.$x \in \{-1\} \cup (14,19]$\\
D.$x \in \{1\} \cup (14,19]$\\
E.$x \in \{-1\} \cup [14,19)$\\
F.$x \in \{1\} \cup [14,19)$\\
G.$x \in \{-1\} \cup (14,19)$\\
H.$x \in \{1\} \cup [14,19]$
\testStop
\kluczStart
A
\kluczStop



\zadStart{Zadanie z Wikieł Z 1.62 c) moja wersja nr 838}

Rozwiązać nierówności $(14-x)(x+1)^{2}(20-x)^{3}\le0$.
\zadStop
\rozwStart{Patryk Wirkus}{}
Miejsca zerowe naszego wielomianu to: $14, -1, 20$.\\
Wielomian jest stopnia parzystego, ponadto znak współczynnika przy\linebreak najwyższej potędze x jest ujemny.\\ W związku z tym wykres wielomianu zaczyna się od lewej strony powyżej osi OX.\\
Ponadto w punkcie $-1$ wykres odbija się od osi poziomej.\\
A więc $$x \in \{-1\} \cup [14,20].$$
\rozwStop
\odpStart
$x \in \{-1\} \cup [14,20]$
\odpStop
\testStart
A.$x \in \{-1\} \cup [14,20]$\\
B.$x \in \{1\} \cup (14,20)$\\
C.$x \in \{-1\} \cup (14,20]$\\
D.$x \in \{1\} \cup (14,20]$\\
E.$x \in \{-1\} \cup [14,20)$\\
F.$x \in \{1\} \cup [14,20)$\\
G.$x \in \{-1\} \cup (14,20)$\\
H.$x \in \{1\} \cup [14,20]$
\testStop
\kluczStart
A
\kluczStop



\zadStart{Zadanie z Wikieł Z 1.62 c) moja wersja nr 839}

Rozwiązać nierówności $(14-x)(x+2)^{2}(15-x)^{3}\le0$.
\zadStop
\rozwStart{Patryk Wirkus}{}
Miejsca zerowe naszego wielomianu to: $14, -2, 15$.\\
Wielomian jest stopnia parzystego, ponadto znak współczynnika przy\linebreak najwyższej potędze x jest ujemny.\\ W związku z tym wykres wielomianu zaczyna się od lewej strony powyżej osi OX.\\
Ponadto w punkcie $-2$ wykres odbija się od osi poziomej.\\
A więc $$x \in \{-2\} \cup [14,15].$$
\rozwStop
\odpStart
$x \in \{-2\} \cup [14,15]$
\odpStop
\testStart
A.$x \in \{-2\} \cup [14,15]$\\
B.$x \in \{2\} \cup (14,15)$\\
C.$x \in \{-2\} \cup (14,15]$\\
D.$x \in \{2\} \cup (14,15]$\\
E.$x \in \{-2\} \cup [14,15)$\\
F.$x \in \{2\} \cup [14,15)$\\
G.$x \in \{-2\} \cup (14,15)$\\
H.$x \in \{2\} \cup [14,15]$
\testStop
\kluczStart
A
\kluczStop



\zadStart{Zadanie z Wikieł Z 1.62 c) moja wersja nr 840}

Rozwiązać nierówności $(14-x)(x+2)^{2}(16-x)^{3}\le0$.
\zadStop
\rozwStart{Patryk Wirkus}{}
Miejsca zerowe naszego wielomianu to: $14, -2, 16$.\\
Wielomian jest stopnia parzystego, ponadto znak współczynnika przy\linebreak najwyższej potędze x jest ujemny.\\ W związku z tym wykres wielomianu zaczyna się od lewej strony powyżej osi OX.\\
Ponadto w punkcie $-2$ wykres odbija się od osi poziomej.\\
A więc $$x \in \{-2\} \cup [14,16].$$
\rozwStop
\odpStart
$x \in \{-2\} \cup [14,16]$
\odpStop
\testStart
A.$x \in \{-2\} \cup [14,16]$\\
B.$x \in \{2\} \cup (14,16)$\\
C.$x \in \{-2\} \cup (14,16]$\\
D.$x \in \{2\} \cup (14,16]$\\
E.$x \in \{-2\} \cup [14,16)$\\
F.$x \in \{2\} \cup [14,16)$\\
G.$x \in \{-2\} \cup (14,16)$\\
H.$x \in \{2\} \cup [14,16]$
\testStop
\kluczStart
A
\kluczStop



\zadStart{Zadanie z Wikieł Z 1.62 c) moja wersja nr 841}

Rozwiązać nierówności $(14-x)(x+2)^{2}(17-x)^{3}\le0$.
\zadStop
\rozwStart{Patryk Wirkus}{}
Miejsca zerowe naszego wielomianu to: $14, -2, 17$.\\
Wielomian jest stopnia parzystego, ponadto znak współczynnika przy\linebreak najwyższej potędze x jest ujemny.\\ W związku z tym wykres wielomianu zaczyna się od lewej strony powyżej osi OX.\\
Ponadto w punkcie $-2$ wykres odbija się od osi poziomej.\\
A więc $$x \in \{-2\} \cup [14,17].$$
\rozwStop
\odpStart
$x \in \{-2\} \cup [14,17]$
\odpStop
\testStart
A.$x \in \{-2\} \cup [14,17]$\\
B.$x \in \{2\} \cup (14,17)$\\
C.$x \in \{-2\} \cup (14,17]$\\
D.$x \in \{2\} \cup (14,17]$\\
E.$x \in \{-2\} \cup [14,17)$\\
F.$x \in \{2\} \cup [14,17)$\\
G.$x \in \{-2\} \cup (14,17)$\\
H.$x \in \{2\} \cup [14,17]$
\testStop
\kluczStart
A
\kluczStop



\zadStart{Zadanie z Wikieł Z 1.62 c) moja wersja nr 842}

Rozwiązać nierówności $(14-x)(x+2)^{2}(18-x)^{3}\le0$.
\zadStop
\rozwStart{Patryk Wirkus}{}
Miejsca zerowe naszego wielomianu to: $14, -2, 18$.\\
Wielomian jest stopnia parzystego, ponadto znak współczynnika przy\linebreak najwyższej potędze x jest ujemny.\\ W związku z tym wykres wielomianu zaczyna się od lewej strony powyżej osi OX.\\
Ponadto w punkcie $-2$ wykres odbija się od osi poziomej.\\
A więc $$x \in \{-2\} \cup [14,18].$$
\rozwStop
\odpStart
$x \in \{-2\} \cup [14,18]$
\odpStop
\testStart
A.$x \in \{-2\} \cup [14,18]$\\
B.$x \in \{2\} \cup (14,18)$\\
C.$x \in \{-2\} \cup (14,18]$\\
D.$x \in \{2\} \cup (14,18]$\\
E.$x \in \{-2\} \cup [14,18)$\\
F.$x \in \{2\} \cup [14,18)$\\
G.$x \in \{-2\} \cup (14,18)$\\
H.$x \in \{2\} \cup [14,18]$
\testStop
\kluczStart
A
\kluczStop



\zadStart{Zadanie z Wikieł Z 1.62 c) moja wersja nr 843}

Rozwiązać nierówności $(14-x)(x+2)^{2}(19-x)^{3}\le0$.
\zadStop
\rozwStart{Patryk Wirkus}{}
Miejsca zerowe naszego wielomianu to: $14, -2, 19$.\\
Wielomian jest stopnia parzystego, ponadto znak współczynnika przy\linebreak najwyższej potędze x jest ujemny.\\ W związku z tym wykres wielomianu zaczyna się od lewej strony powyżej osi OX.\\
Ponadto w punkcie $-2$ wykres odbija się od osi poziomej.\\
A więc $$x \in \{-2\} \cup [14,19].$$
\rozwStop
\odpStart
$x \in \{-2\} \cup [14,19]$
\odpStop
\testStart
A.$x \in \{-2\} \cup [14,19]$\\
B.$x \in \{2\} \cup (14,19)$\\
C.$x \in \{-2\} \cup (14,19]$\\
D.$x \in \{2\} \cup (14,19]$\\
E.$x \in \{-2\} \cup [14,19)$\\
F.$x \in \{2\} \cup [14,19)$\\
G.$x \in \{-2\} \cup (14,19)$\\
H.$x \in \{2\} \cup [14,19]$
\testStop
\kluczStart
A
\kluczStop



\zadStart{Zadanie z Wikieł Z 1.62 c) moja wersja nr 844}

Rozwiązać nierówności $(14-x)(x+2)^{2}(20-x)^{3}\le0$.
\zadStop
\rozwStart{Patryk Wirkus}{}
Miejsca zerowe naszego wielomianu to: $14, -2, 20$.\\
Wielomian jest stopnia parzystego, ponadto znak współczynnika przy\linebreak najwyższej potędze x jest ujemny.\\ W związku z tym wykres wielomianu zaczyna się od lewej strony powyżej osi OX.\\
Ponadto w punkcie $-2$ wykres odbija się od osi poziomej.\\
A więc $$x \in \{-2\} \cup [14,20].$$
\rozwStop
\odpStart
$x \in \{-2\} \cup [14,20]$
\odpStop
\testStart
A.$x \in \{-2\} \cup [14,20]$\\
B.$x \in \{2\} \cup (14,20)$\\
C.$x \in \{-2\} \cup (14,20]$\\
D.$x \in \{2\} \cup (14,20]$\\
E.$x \in \{-2\} \cup [14,20)$\\
F.$x \in \{2\} \cup [14,20)$\\
G.$x \in \{-2\} \cup (14,20)$\\
H.$x \in \{2\} \cup [14,20]$
\testStop
\kluczStart
A
\kluczStop



\zadStart{Zadanie z Wikieł Z 1.62 c) moja wersja nr 845}

Rozwiązać nierówności $(14-x)(x+3)^{2}(15-x)^{3}\le0$.
\zadStop
\rozwStart{Patryk Wirkus}{}
Miejsca zerowe naszego wielomianu to: $14, -3, 15$.\\
Wielomian jest stopnia parzystego, ponadto znak współczynnika przy\linebreak najwyższej potędze x jest ujemny.\\ W związku z tym wykres wielomianu zaczyna się od lewej strony powyżej osi OX.\\
Ponadto w punkcie $-3$ wykres odbija się od osi poziomej.\\
A więc $$x \in \{-3\} \cup [14,15].$$
\rozwStop
\odpStart
$x \in \{-3\} \cup [14,15]$
\odpStop
\testStart
A.$x \in \{-3\} \cup [14,15]$\\
B.$x \in \{3\} \cup (14,15)$\\
C.$x \in \{-3\} \cup (14,15]$\\
D.$x \in \{3\} \cup (14,15]$\\
E.$x \in \{-3\} \cup [14,15)$\\
F.$x \in \{3\} \cup [14,15)$\\
G.$x \in \{-3\} \cup (14,15)$\\
H.$x \in \{3\} \cup [14,15]$
\testStop
\kluczStart
A
\kluczStop



\zadStart{Zadanie z Wikieł Z 1.62 c) moja wersja nr 846}

Rozwiązać nierówności $(14-x)(x+3)^{2}(16-x)^{3}\le0$.
\zadStop
\rozwStart{Patryk Wirkus}{}
Miejsca zerowe naszego wielomianu to: $14, -3, 16$.\\
Wielomian jest stopnia parzystego, ponadto znak współczynnika przy\linebreak najwyższej potędze x jest ujemny.\\ W związku z tym wykres wielomianu zaczyna się od lewej strony powyżej osi OX.\\
Ponadto w punkcie $-3$ wykres odbija się od osi poziomej.\\
A więc $$x \in \{-3\} \cup [14,16].$$
\rozwStop
\odpStart
$x \in \{-3\} \cup [14,16]$
\odpStop
\testStart
A.$x \in \{-3\} \cup [14,16]$\\
B.$x \in \{3\} \cup (14,16)$\\
C.$x \in \{-3\} \cup (14,16]$\\
D.$x \in \{3\} \cup (14,16]$\\
E.$x \in \{-3\} \cup [14,16)$\\
F.$x \in \{3\} \cup [14,16)$\\
G.$x \in \{-3\} \cup (14,16)$\\
H.$x \in \{3\} \cup [14,16]$
\testStop
\kluczStart
A
\kluczStop



\zadStart{Zadanie z Wikieł Z 1.62 c) moja wersja nr 847}

Rozwiązać nierówności $(14-x)(x+3)^{2}(17-x)^{3}\le0$.
\zadStop
\rozwStart{Patryk Wirkus}{}
Miejsca zerowe naszego wielomianu to: $14, -3, 17$.\\
Wielomian jest stopnia parzystego, ponadto znak współczynnika przy\linebreak najwyższej potędze x jest ujemny.\\ W związku z tym wykres wielomianu zaczyna się od lewej strony powyżej osi OX.\\
Ponadto w punkcie $-3$ wykres odbija się od osi poziomej.\\
A więc $$x \in \{-3\} \cup [14,17].$$
\rozwStop
\odpStart
$x \in \{-3\} \cup [14,17]$
\odpStop
\testStart
A.$x \in \{-3\} \cup [14,17]$\\
B.$x \in \{3\} \cup (14,17)$\\
C.$x \in \{-3\} \cup (14,17]$\\
D.$x \in \{3\} \cup (14,17]$\\
E.$x \in \{-3\} \cup [14,17)$\\
F.$x \in \{3\} \cup [14,17)$\\
G.$x \in \{-3\} \cup (14,17)$\\
H.$x \in \{3\} \cup [14,17]$
\testStop
\kluczStart
A
\kluczStop



\zadStart{Zadanie z Wikieł Z 1.62 c) moja wersja nr 848}

Rozwiązać nierówności $(14-x)(x+3)^{2}(18-x)^{3}\le0$.
\zadStop
\rozwStart{Patryk Wirkus}{}
Miejsca zerowe naszego wielomianu to: $14, -3, 18$.\\
Wielomian jest stopnia parzystego, ponadto znak współczynnika przy\linebreak najwyższej potędze x jest ujemny.\\ W związku z tym wykres wielomianu zaczyna się od lewej strony powyżej osi OX.\\
Ponadto w punkcie $-3$ wykres odbija się od osi poziomej.\\
A więc $$x \in \{-3\} \cup [14,18].$$
\rozwStop
\odpStart
$x \in \{-3\} \cup [14,18]$
\odpStop
\testStart
A.$x \in \{-3\} \cup [14,18]$\\
B.$x \in \{3\} \cup (14,18)$\\
C.$x \in \{-3\} \cup (14,18]$\\
D.$x \in \{3\} \cup (14,18]$\\
E.$x \in \{-3\} \cup [14,18)$\\
F.$x \in \{3\} \cup [14,18)$\\
G.$x \in \{-3\} \cup (14,18)$\\
H.$x \in \{3\} \cup [14,18]$
\testStop
\kluczStart
A
\kluczStop



\zadStart{Zadanie z Wikieł Z 1.62 c) moja wersja nr 849}

Rozwiązać nierówności $(14-x)(x+3)^{2}(19-x)^{3}\le0$.
\zadStop
\rozwStart{Patryk Wirkus}{}
Miejsca zerowe naszego wielomianu to: $14, -3, 19$.\\
Wielomian jest stopnia parzystego, ponadto znak współczynnika przy\linebreak najwyższej potędze x jest ujemny.\\ W związku z tym wykres wielomianu zaczyna się od lewej strony powyżej osi OX.\\
Ponadto w punkcie $-3$ wykres odbija się od osi poziomej.\\
A więc $$x \in \{-3\} \cup [14,19].$$
\rozwStop
\odpStart
$x \in \{-3\} \cup [14,19]$
\odpStop
\testStart
A.$x \in \{-3\} \cup [14,19]$\\
B.$x \in \{3\} \cup (14,19)$\\
C.$x \in \{-3\} \cup (14,19]$\\
D.$x \in \{3\} \cup (14,19]$\\
E.$x \in \{-3\} \cup [14,19)$\\
F.$x \in \{3\} \cup [14,19)$\\
G.$x \in \{-3\} \cup (14,19)$\\
H.$x \in \{3\} \cup [14,19]$
\testStop
\kluczStart
A
\kluczStop



\zadStart{Zadanie z Wikieł Z 1.62 c) moja wersja nr 850}

Rozwiązać nierówności $(14-x)(x+3)^{2}(20-x)^{3}\le0$.
\zadStop
\rozwStart{Patryk Wirkus}{}
Miejsca zerowe naszego wielomianu to: $14, -3, 20$.\\
Wielomian jest stopnia parzystego, ponadto znak współczynnika przy\linebreak najwyższej potędze x jest ujemny.\\ W związku z tym wykres wielomianu zaczyna się od lewej strony powyżej osi OX.\\
Ponadto w punkcie $-3$ wykres odbija się od osi poziomej.\\
A więc $$x \in \{-3\} \cup [14,20].$$
\rozwStop
\odpStart
$x \in \{-3\} \cup [14,20]$
\odpStop
\testStart
A.$x \in \{-3\} \cup [14,20]$\\
B.$x \in \{3\} \cup (14,20)$\\
C.$x \in \{-3\} \cup (14,20]$\\
D.$x \in \{3\} \cup (14,20]$\\
E.$x \in \{-3\} \cup [14,20)$\\
F.$x \in \{3\} \cup [14,20)$\\
G.$x \in \{-3\} \cup (14,20)$\\
H.$x \in \{3\} \cup [14,20]$
\testStop
\kluczStart
A
\kluczStop



\zadStart{Zadanie z Wikieł Z 1.62 c) moja wersja nr 851}

Rozwiązać nierówności $(14-x)(x+4)^{2}(15-x)^{3}\le0$.
\zadStop
\rozwStart{Patryk Wirkus}{}
Miejsca zerowe naszego wielomianu to: $14, -4, 15$.\\
Wielomian jest stopnia parzystego, ponadto znak współczynnika przy\linebreak najwyższej potędze x jest ujemny.\\ W związku z tym wykres wielomianu zaczyna się od lewej strony powyżej osi OX.\\
Ponadto w punkcie $-4$ wykres odbija się od osi poziomej.\\
A więc $$x \in \{-4\} \cup [14,15].$$
\rozwStop
\odpStart
$x \in \{-4\} \cup [14,15]$
\odpStop
\testStart
A.$x \in \{-4\} \cup [14,15]$\\
B.$x \in \{4\} \cup (14,15)$\\
C.$x \in \{-4\} \cup (14,15]$\\
D.$x \in \{4\} \cup (14,15]$\\
E.$x \in \{-4\} \cup [14,15)$\\
F.$x \in \{4\} \cup [14,15)$\\
G.$x \in \{-4\} \cup (14,15)$\\
H.$x \in \{4\} \cup [14,15]$
\testStop
\kluczStart
A
\kluczStop



\zadStart{Zadanie z Wikieł Z 1.62 c) moja wersja nr 852}

Rozwiązać nierówności $(14-x)(x+4)^{2}(16-x)^{3}\le0$.
\zadStop
\rozwStart{Patryk Wirkus}{}
Miejsca zerowe naszego wielomianu to: $14, -4, 16$.\\
Wielomian jest stopnia parzystego, ponadto znak współczynnika przy\linebreak najwyższej potędze x jest ujemny.\\ W związku z tym wykres wielomianu zaczyna się od lewej strony powyżej osi OX.\\
Ponadto w punkcie $-4$ wykres odbija się od osi poziomej.\\
A więc $$x \in \{-4\} \cup [14,16].$$
\rozwStop
\odpStart
$x \in \{-4\} \cup [14,16]$
\odpStop
\testStart
A.$x \in \{-4\} \cup [14,16]$\\
B.$x \in \{4\} \cup (14,16)$\\
C.$x \in \{-4\} \cup (14,16]$\\
D.$x \in \{4\} \cup (14,16]$\\
E.$x \in \{-4\} \cup [14,16)$\\
F.$x \in \{4\} \cup [14,16)$\\
G.$x \in \{-4\} \cup (14,16)$\\
H.$x \in \{4\} \cup [14,16]$
\testStop
\kluczStart
A
\kluczStop



\zadStart{Zadanie z Wikieł Z 1.62 c) moja wersja nr 853}

Rozwiązać nierówności $(14-x)(x+4)^{2}(17-x)^{3}\le0$.
\zadStop
\rozwStart{Patryk Wirkus}{}
Miejsca zerowe naszego wielomianu to: $14, -4, 17$.\\
Wielomian jest stopnia parzystego, ponadto znak współczynnika przy\linebreak najwyższej potędze x jest ujemny.\\ W związku z tym wykres wielomianu zaczyna się od lewej strony powyżej osi OX.\\
Ponadto w punkcie $-4$ wykres odbija się od osi poziomej.\\
A więc $$x \in \{-4\} \cup [14,17].$$
\rozwStop
\odpStart
$x \in \{-4\} \cup [14,17]$
\odpStop
\testStart
A.$x \in \{-4\} \cup [14,17]$\\
B.$x \in \{4\} \cup (14,17)$\\
C.$x \in \{-4\} \cup (14,17]$\\
D.$x \in \{4\} \cup (14,17]$\\
E.$x \in \{-4\} \cup [14,17)$\\
F.$x \in \{4\} \cup [14,17)$\\
G.$x \in \{-4\} \cup (14,17)$\\
H.$x \in \{4\} \cup [14,17]$
\testStop
\kluczStart
A
\kluczStop



\zadStart{Zadanie z Wikieł Z 1.62 c) moja wersja nr 854}

Rozwiązać nierówności $(14-x)(x+4)^{2}(18-x)^{3}\le0$.
\zadStop
\rozwStart{Patryk Wirkus}{}
Miejsca zerowe naszego wielomianu to: $14, -4, 18$.\\
Wielomian jest stopnia parzystego, ponadto znak współczynnika przy\linebreak najwyższej potędze x jest ujemny.\\ W związku z tym wykres wielomianu zaczyna się od lewej strony powyżej osi OX.\\
Ponadto w punkcie $-4$ wykres odbija się od osi poziomej.\\
A więc $$x \in \{-4\} \cup [14,18].$$
\rozwStop
\odpStart
$x \in \{-4\} \cup [14,18]$
\odpStop
\testStart
A.$x \in \{-4\} \cup [14,18]$\\
B.$x \in \{4\} \cup (14,18)$\\
C.$x \in \{-4\} \cup (14,18]$\\
D.$x \in \{4\} \cup (14,18]$\\
E.$x \in \{-4\} \cup [14,18)$\\
F.$x \in \{4\} \cup [14,18)$\\
G.$x \in \{-4\} \cup (14,18)$\\
H.$x \in \{4\} \cup [14,18]$
\testStop
\kluczStart
A
\kluczStop



\zadStart{Zadanie z Wikieł Z 1.62 c) moja wersja nr 855}

Rozwiązać nierówności $(14-x)(x+4)^{2}(19-x)^{3}\le0$.
\zadStop
\rozwStart{Patryk Wirkus}{}
Miejsca zerowe naszego wielomianu to: $14, -4, 19$.\\
Wielomian jest stopnia parzystego, ponadto znak współczynnika przy\linebreak najwyższej potędze x jest ujemny.\\ W związku z tym wykres wielomianu zaczyna się od lewej strony powyżej osi OX.\\
Ponadto w punkcie $-4$ wykres odbija się od osi poziomej.\\
A więc $$x \in \{-4\} \cup [14,19].$$
\rozwStop
\odpStart
$x \in \{-4\} \cup [14,19]$
\odpStop
\testStart
A.$x \in \{-4\} \cup [14,19]$\\
B.$x \in \{4\} \cup (14,19)$\\
C.$x \in \{-4\} \cup (14,19]$\\
D.$x \in \{4\} \cup (14,19]$\\
E.$x \in \{-4\} \cup [14,19)$\\
F.$x \in \{4\} \cup [14,19)$\\
G.$x \in \{-4\} \cup (14,19)$\\
H.$x \in \{4\} \cup [14,19]$
\testStop
\kluczStart
A
\kluczStop



\zadStart{Zadanie z Wikieł Z 1.62 c) moja wersja nr 856}

Rozwiązać nierówności $(14-x)(x+4)^{2}(20-x)^{3}\le0$.
\zadStop
\rozwStart{Patryk Wirkus}{}
Miejsca zerowe naszego wielomianu to: $14, -4, 20$.\\
Wielomian jest stopnia parzystego, ponadto znak współczynnika przy\linebreak najwyższej potędze x jest ujemny.\\ W związku z tym wykres wielomianu zaczyna się od lewej strony powyżej osi OX.\\
Ponadto w punkcie $-4$ wykres odbija się od osi poziomej.\\
A więc $$x \in \{-4\} \cup [14,20].$$
\rozwStop
\odpStart
$x \in \{-4\} \cup [14,20]$
\odpStop
\testStart
A.$x \in \{-4\} \cup [14,20]$\\
B.$x \in \{4\} \cup (14,20)$\\
C.$x \in \{-4\} \cup (14,20]$\\
D.$x \in \{4\} \cup (14,20]$\\
E.$x \in \{-4\} \cup [14,20)$\\
F.$x \in \{4\} \cup [14,20)$\\
G.$x \in \{-4\} \cup (14,20)$\\
H.$x \in \{4\} \cup [14,20]$
\testStop
\kluczStart
A
\kluczStop



\zadStart{Zadanie z Wikieł Z 1.62 c) moja wersja nr 857}

Rozwiązać nierówności $(14-x)(x+5)^{2}(15-x)^{3}\le0$.
\zadStop
\rozwStart{Patryk Wirkus}{}
Miejsca zerowe naszego wielomianu to: $14, -5, 15$.\\
Wielomian jest stopnia parzystego, ponadto znak współczynnika przy\linebreak najwyższej potędze x jest ujemny.\\ W związku z tym wykres wielomianu zaczyna się od lewej strony powyżej osi OX.\\
Ponadto w punkcie $-5$ wykres odbija się od osi poziomej.\\
A więc $$x \in \{-5\} \cup [14,15].$$
\rozwStop
\odpStart
$x \in \{-5\} \cup [14,15]$
\odpStop
\testStart
A.$x \in \{-5\} \cup [14,15]$\\
B.$x \in \{5\} \cup (14,15)$\\
C.$x \in \{-5\} \cup (14,15]$\\
D.$x \in \{5\} \cup (14,15]$\\
E.$x \in \{-5\} \cup [14,15)$\\
F.$x \in \{5\} \cup [14,15)$\\
G.$x \in \{-5\} \cup (14,15)$\\
H.$x \in \{5\} \cup [14,15]$
\testStop
\kluczStart
A
\kluczStop



\zadStart{Zadanie z Wikieł Z 1.62 c) moja wersja nr 858}

Rozwiązać nierówności $(14-x)(x+5)^{2}(16-x)^{3}\le0$.
\zadStop
\rozwStart{Patryk Wirkus}{}
Miejsca zerowe naszego wielomianu to: $14, -5, 16$.\\
Wielomian jest stopnia parzystego, ponadto znak współczynnika przy\linebreak najwyższej potędze x jest ujemny.\\ W związku z tym wykres wielomianu zaczyna się od lewej strony powyżej osi OX.\\
Ponadto w punkcie $-5$ wykres odbija się od osi poziomej.\\
A więc $$x \in \{-5\} \cup [14,16].$$
\rozwStop
\odpStart
$x \in \{-5\} \cup [14,16]$
\odpStop
\testStart
A.$x \in \{-5\} \cup [14,16]$\\
B.$x \in \{5\} \cup (14,16)$\\
C.$x \in \{-5\} \cup (14,16]$\\
D.$x \in \{5\} \cup (14,16]$\\
E.$x \in \{-5\} \cup [14,16)$\\
F.$x \in \{5\} \cup [14,16)$\\
G.$x \in \{-5\} \cup (14,16)$\\
H.$x \in \{5\} \cup [14,16]$
\testStop
\kluczStart
A
\kluczStop



\zadStart{Zadanie z Wikieł Z 1.62 c) moja wersja nr 859}

Rozwiązać nierówności $(14-x)(x+5)^{2}(17-x)^{3}\le0$.
\zadStop
\rozwStart{Patryk Wirkus}{}
Miejsca zerowe naszego wielomianu to: $14, -5, 17$.\\
Wielomian jest stopnia parzystego, ponadto znak współczynnika przy\linebreak najwyższej potędze x jest ujemny.\\ W związku z tym wykres wielomianu zaczyna się od lewej strony powyżej osi OX.\\
Ponadto w punkcie $-5$ wykres odbija się od osi poziomej.\\
A więc $$x \in \{-5\} \cup [14,17].$$
\rozwStop
\odpStart
$x \in \{-5\} \cup [14,17]$
\odpStop
\testStart
A.$x \in \{-5\} \cup [14,17]$\\
B.$x \in \{5\} \cup (14,17)$\\
C.$x \in \{-5\} \cup (14,17]$\\
D.$x \in \{5\} \cup (14,17]$\\
E.$x \in \{-5\} \cup [14,17)$\\
F.$x \in \{5\} \cup [14,17)$\\
G.$x \in \{-5\} \cup (14,17)$\\
H.$x \in \{5\} \cup [14,17]$
\testStop
\kluczStart
A
\kluczStop



\zadStart{Zadanie z Wikieł Z 1.62 c) moja wersja nr 860}

Rozwiązać nierówności $(14-x)(x+5)^{2}(18-x)^{3}\le0$.
\zadStop
\rozwStart{Patryk Wirkus}{}
Miejsca zerowe naszego wielomianu to: $14, -5, 18$.\\
Wielomian jest stopnia parzystego, ponadto znak współczynnika przy\linebreak najwyższej potędze x jest ujemny.\\ W związku z tym wykres wielomianu zaczyna się od lewej strony powyżej osi OX.\\
Ponadto w punkcie $-5$ wykres odbija się od osi poziomej.\\
A więc $$x \in \{-5\} \cup [14,18].$$
\rozwStop
\odpStart
$x \in \{-5\} \cup [14,18]$
\odpStop
\testStart
A.$x \in \{-5\} \cup [14,18]$\\
B.$x \in \{5\} \cup (14,18)$\\
C.$x \in \{-5\} \cup (14,18]$\\
D.$x \in \{5\} \cup (14,18]$\\
E.$x \in \{-5\} \cup [14,18)$\\
F.$x \in \{5\} \cup [14,18)$\\
G.$x \in \{-5\} \cup (14,18)$\\
H.$x \in \{5\} \cup [14,18]$
\testStop
\kluczStart
A
\kluczStop



\zadStart{Zadanie z Wikieł Z 1.62 c) moja wersja nr 861}

Rozwiązać nierówności $(14-x)(x+5)^{2}(19-x)^{3}\le0$.
\zadStop
\rozwStart{Patryk Wirkus}{}
Miejsca zerowe naszego wielomianu to: $14, -5, 19$.\\
Wielomian jest stopnia parzystego, ponadto znak współczynnika przy\linebreak najwyższej potędze x jest ujemny.\\ W związku z tym wykres wielomianu zaczyna się od lewej strony powyżej osi OX.\\
Ponadto w punkcie $-5$ wykres odbija się od osi poziomej.\\
A więc $$x \in \{-5\} \cup [14,19].$$
\rozwStop
\odpStart
$x \in \{-5\} \cup [14,19]$
\odpStop
\testStart
A.$x \in \{-5\} \cup [14,19]$\\
B.$x \in \{5\} \cup (14,19)$\\
C.$x \in \{-5\} \cup (14,19]$\\
D.$x \in \{5\} \cup (14,19]$\\
E.$x \in \{-5\} \cup [14,19)$\\
F.$x \in \{5\} \cup [14,19)$\\
G.$x \in \{-5\} \cup (14,19)$\\
H.$x \in \{5\} \cup [14,19]$
\testStop
\kluczStart
A
\kluczStop



\zadStart{Zadanie z Wikieł Z 1.62 c) moja wersja nr 862}

Rozwiązać nierówności $(14-x)(x+5)^{2}(20-x)^{3}\le0$.
\zadStop
\rozwStart{Patryk Wirkus}{}
Miejsca zerowe naszego wielomianu to: $14, -5, 20$.\\
Wielomian jest stopnia parzystego, ponadto znak współczynnika przy\linebreak najwyższej potędze x jest ujemny.\\ W związku z tym wykres wielomianu zaczyna się od lewej strony powyżej osi OX.\\
Ponadto w punkcie $-5$ wykres odbija się od osi poziomej.\\
A więc $$x \in \{-5\} \cup [14,20].$$
\rozwStop
\odpStart
$x \in \{-5\} \cup [14,20]$
\odpStop
\testStart
A.$x \in \{-5\} \cup [14,20]$\\
B.$x \in \{5\} \cup (14,20)$\\
C.$x \in \{-5\} \cup (14,20]$\\
D.$x \in \{5\} \cup (14,20]$\\
E.$x \in \{-5\} \cup [14,20)$\\
F.$x \in \{5\} \cup [14,20)$\\
G.$x \in \{-5\} \cup (14,20)$\\
H.$x \in \{5\} \cup [14,20]$
\testStop
\kluczStart
A
\kluczStop



\zadStart{Zadanie z Wikieł Z 1.62 c) moja wersja nr 863}

Rozwiązać nierówności $(14-x)(x+6)^{2}(15-x)^{3}\le0$.
\zadStop
\rozwStart{Patryk Wirkus}{}
Miejsca zerowe naszego wielomianu to: $14, -6, 15$.\\
Wielomian jest stopnia parzystego, ponadto znak współczynnika przy\linebreak najwyższej potędze x jest ujemny.\\ W związku z tym wykres wielomianu zaczyna się od lewej strony powyżej osi OX.\\
Ponadto w punkcie $-6$ wykres odbija się od osi poziomej.\\
A więc $$x \in \{-6\} \cup [14,15].$$
\rozwStop
\odpStart
$x \in \{-6\} \cup [14,15]$
\odpStop
\testStart
A.$x \in \{-6\} \cup [14,15]$\\
B.$x \in \{6\} \cup (14,15)$\\
C.$x \in \{-6\} \cup (14,15]$\\
D.$x \in \{6\} \cup (14,15]$\\
E.$x \in \{-6\} \cup [14,15)$\\
F.$x \in \{6\} \cup [14,15)$\\
G.$x \in \{-6\} \cup (14,15)$\\
H.$x \in \{6\} \cup [14,15]$
\testStop
\kluczStart
A
\kluczStop



\zadStart{Zadanie z Wikieł Z 1.62 c) moja wersja nr 864}

Rozwiązać nierówności $(14-x)(x+6)^{2}(16-x)^{3}\le0$.
\zadStop
\rozwStart{Patryk Wirkus}{}
Miejsca zerowe naszego wielomianu to: $14, -6, 16$.\\
Wielomian jest stopnia parzystego, ponadto znak współczynnika przy\linebreak najwyższej potędze x jest ujemny.\\ W związku z tym wykres wielomianu zaczyna się od lewej strony powyżej osi OX.\\
Ponadto w punkcie $-6$ wykres odbija się od osi poziomej.\\
A więc $$x \in \{-6\} \cup [14,16].$$
\rozwStop
\odpStart
$x \in \{-6\} \cup [14,16]$
\odpStop
\testStart
A.$x \in \{-6\} \cup [14,16]$\\
B.$x \in \{6\} \cup (14,16)$\\
C.$x \in \{-6\} \cup (14,16]$\\
D.$x \in \{6\} \cup (14,16]$\\
E.$x \in \{-6\} \cup [14,16)$\\
F.$x \in \{6\} \cup [14,16)$\\
G.$x \in \{-6\} \cup (14,16)$\\
H.$x \in \{6\} \cup [14,16]$
\testStop
\kluczStart
A
\kluczStop



\zadStart{Zadanie z Wikieł Z 1.62 c) moja wersja nr 865}

Rozwiązać nierówności $(14-x)(x+6)^{2}(17-x)^{3}\le0$.
\zadStop
\rozwStart{Patryk Wirkus}{}
Miejsca zerowe naszego wielomianu to: $14, -6, 17$.\\
Wielomian jest stopnia parzystego, ponadto znak współczynnika przy\linebreak najwyższej potędze x jest ujemny.\\ W związku z tym wykres wielomianu zaczyna się od lewej strony powyżej osi OX.\\
Ponadto w punkcie $-6$ wykres odbija się od osi poziomej.\\
A więc $$x \in \{-6\} \cup [14,17].$$
\rozwStop
\odpStart
$x \in \{-6\} \cup [14,17]$
\odpStop
\testStart
A.$x \in \{-6\} \cup [14,17]$\\
B.$x \in \{6\} \cup (14,17)$\\
C.$x \in \{-6\} \cup (14,17]$\\
D.$x \in \{6\} \cup (14,17]$\\
E.$x \in \{-6\} \cup [14,17)$\\
F.$x \in \{6\} \cup [14,17)$\\
G.$x \in \{-6\} \cup (14,17)$\\
H.$x \in \{6\} \cup [14,17]$
\testStop
\kluczStart
A
\kluczStop



\zadStart{Zadanie z Wikieł Z 1.62 c) moja wersja nr 866}

Rozwiązać nierówności $(14-x)(x+6)^{2}(18-x)^{3}\le0$.
\zadStop
\rozwStart{Patryk Wirkus}{}
Miejsca zerowe naszego wielomianu to: $14, -6, 18$.\\
Wielomian jest stopnia parzystego, ponadto znak współczynnika przy\linebreak najwyższej potędze x jest ujemny.\\ W związku z tym wykres wielomianu zaczyna się od lewej strony powyżej osi OX.\\
Ponadto w punkcie $-6$ wykres odbija się od osi poziomej.\\
A więc $$x \in \{-6\} \cup [14,18].$$
\rozwStop
\odpStart
$x \in \{-6\} \cup [14,18]$
\odpStop
\testStart
A.$x \in \{-6\} \cup [14,18]$\\
B.$x \in \{6\} \cup (14,18)$\\
C.$x \in \{-6\} \cup (14,18]$\\
D.$x \in \{6\} \cup (14,18]$\\
E.$x \in \{-6\} \cup [14,18)$\\
F.$x \in \{6\} \cup [14,18)$\\
G.$x \in \{-6\} \cup (14,18)$\\
H.$x \in \{6\} \cup [14,18]$
\testStop
\kluczStart
A
\kluczStop



\zadStart{Zadanie z Wikieł Z 1.62 c) moja wersja nr 867}

Rozwiązać nierówności $(14-x)(x+6)^{2}(19-x)^{3}\le0$.
\zadStop
\rozwStart{Patryk Wirkus}{}
Miejsca zerowe naszego wielomianu to: $14, -6, 19$.\\
Wielomian jest stopnia parzystego, ponadto znak współczynnika przy\linebreak najwyższej potędze x jest ujemny.\\ W związku z tym wykres wielomianu zaczyna się od lewej strony powyżej osi OX.\\
Ponadto w punkcie $-6$ wykres odbija się od osi poziomej.\\
A więc $$x \in \{-6\} \cup [14,19].$$
\rozwStop
\odpStart
$x \in \{-6\} \cup [14,19]$
\odpStop
\testStart
A.$x \in \{-6\} \cup [14,19]$\\
B.$x \in \{6\} \cup (14,19)$\\
C.$x \in \{-6\} \cup (14,19]$\\
D.$x \in \{6\} \cup (14,19]$\\
E.$x \in \{-6\} \cup [14,19)$\\
F.$x \in \{6\} \cup [14,19)$\\
G.$x \in \{-6\} \cup (14,19)$\\
H.$x \in \{6\} \cup [14,19]$
\testStop
\kluczStart
A
\kluczStop



\zadStart{Zadanie z Wikieł Z 1.62 c) moja wersja nr 868}

Rozwiązać nierówności $(14-x)(x+6)^{2}(20-x)^{3}\le0$.
\zadStop
\rozwStart{Patryk Wirkus}{}
Miejsca zerowe naszego wielomianu to: $14, -6, 20$.\\
Wielomian jest stopnia parzystego, ponadto znak współczynnika przy\linebreak najwyższej potędze x jest ujemny.\\ W związku z tym wykres wielomianu zaczyna się od lewej strony powyżej osi OX.\\
Ponadto w punkcie $-6$ wykres odbija się od osi poziomej.\\
A więc $$x \in \{-6\} \cup [14,20].$$
\rozwStop
\odpStart
$x \in \{-6\} \cup [14,20]$
\odpStop
\testStart
A.$x \in \{-6\} \cup [14,20]$\\
B.$x \in \{6\} \cup (14,20)$\\
C.$x \in \{-6\} \cup (14,20]$\\
D.$x \in \{6\} \cup (14,20]$\\
E.$x \in \{-6\} \cup [14,20)$\\
F.$x \in \{6\} \cup [14,20)$\\
G.$x \in \{-6\} \cup (14,20)$\\
H.$x \in \{6\} \cup [14,20]$
\testStop
\kluczStart
A
\kluczStop



\zadStart{Zadanie z Wikieł Z 1.62 c) moja wersja nr 869}

Rozwiązać nierówności $(14-x)(x+7)^{2}(15-x)^{3}\le0$.
\zadStop
\rozwStart{Patryk Wirkus}{}
Miejsca zerowe naszego wielomianu to: $14, -7, 15$.\\
Wielomian jest stopnia parzystego, ponadto znak współczynnika przy\linebreak najwyższej potędze x jest ujemny.\\ W związku z tym wykres wielomianu zaczyna się od lewej strony powyżej osi OX.\\
Ponadto w punkcie $-7$ wykres odbija się od osi poziomej.\\
A więc $$x \in \{-7\} \cup [14,15].$$
\rozwStop
\odpStart
$x \in \{-7\} \cup [14,15]$
\odpStop
\testStart
A.$x \in \{-7\} \cup [14,15]$\\
B.$x \in \{7\} \cup (14,15)$\\
C.$x \in \{-7\} \cup (14,15]$\\
D.$x \in \{7\} \cup (14,15]$\\
E.$x \in \{-7\} \cup [14,15)$\\
F.$x \in \{7\} \cup [14,15)$\\
G.$x \in \{-7\} \cup (14,15)$\\
H.$x \in \{7\} \cup [14,15]$
\testStop
\kluczStart
A
\kluczStop



\zadStart{Zadanie z Wikieł Z 1.62 c) moja wersja nr 870}

Rozwiązać nierówności $(14-x)(x+7)^{2}(16-x)^{3}\le0$.
\zadStop
\rozwStart{Patryk Wirkus}{}
Miejsca zerowe naszego wielomianu to: $14, -7, 16$.\\
Wielomian jest stopnia parzystego, ponadto znak współczynnika przy\linebreak najwyższej potędze x jest ujemny.\\ W związku z tym wykres wielomianu zaczyna się od lewej strony powyżej osi OX.\\
Ponadto w punkcie $-7$ wykres odbija się od osi poziomej.\\
A więc $$x \in \{-7\} \cup [14,16].$$
\rozwStop
\odpStart
$x \in \{-7\} \cup [14,16]$
\odpStop
\testStart
A.$x \in \{-7\} \cup [14,16]$\\
B.$x \in \{7\} \cup (14,16)$\\
C.$x \in \{-7\} \cup (14,16]$\\
D.$x \in \{7\} \cup (14,16]$\\
E.$x \in \{-7\} \cup [14,16)$\\
F.$x \in \{7\} \cup [14,16)$\\
G.$x \in \{-7\} \cup (14,16)$\\
H.$x \in \{7\} \cup [14,16]$
\testStop
\kluczStart
A
\kluczStop



\zadStart{Zadanie z Wikieł Z 1.62 c) moja wersja nr 871}

Rozwiązać nierówności $(14-x)(x+7)^{2}(17-x)^{3}\le0$.
\zadStop
\rozwStart{Patryk Wirkus}{}
Miejsca zerowe naszego wielomianu to: $14, -7, 17$.\\
Wielomian jest stopnia parzystego, ponadto znak współczynnika przy\linebreak najwyższej potędze x jest ujemny.\\ W związku z tym wykres wielomianu zaczyna się od lewej strony powyżej osi OX.\\
Ponadto w punkcie $-7$ wykres odbija się od osi poziomej.\\
A więc $$x \in \{-7\} \cup [14,17].$$
\rozwStop
\odpStart
$x \in \{-7\} \cup [14,17]$
\odpStop
\testStart
A.$x \in \{-7\} \cup [14,17]$\\
B.$x \in \{7\} \cup (14,17)$\\
C.$x \in \{-7\} \cup (14,17]$\\
D.$x \in \{7\} \cup (14,17]$\\
E.$x \in \{-7\} \cup [14,17)$\\
F.$x \in \{7\} \cup [14,17)$\\
G.$x \in \{-7\} \cup (14,17)$\\
H.$x \in \{7\} \cup [14,17]$
\testStop
\kluczStart
A
\kluczStop



\zadStart{Zadanie z Wikieł Z 1.62 c) moja wersja nr 872}

Rozwiązać nierówności $(14-x)(x+7)^{2}(18-x)^{3}\le0$.
\zadStop
\rozwStart{Patryk Wirkus}{}
Miejsca zerowe naszego wielomianu to: $14, -7, 18$.\\
Wielomian jest stopnia parzystego, ponadto znak współczynnika przy\linebreak najwyższej potędze x jest ujemny.\\ W związku z tym wykres wielomianu zaczyna się od lewej strony powyżej osi OX.\\
Ponadto w punkcie $-7$ wykres odbija się od osi poziomej.\\
A więc $$x \in \{-7\} \cup [14,18].$$
\rozwStop
\odpStart
$x \in \{-7\} \cup [14,18]$
\odpStop
\testStart
A.$x \in \{-7\} \cup [14,18]$\\
B.$x \in \{7\} \cup (14,18)$\\
C.$x \in \{-7\} \cup (14,18]$\\
D.$x \in \{7\} \cup (14,18]$\\
E.$x \in \{-7\} \cup [14,18)$\\
F.$x \in \{7\} \cup [14,18)$\\
G.$x \in \{-7\} \cup (14,18)$\\
H.$x \in \{7\} \cup [14,18]$
\testStop
\kluczStart
A
\kluczStop



\zadStart{Zadanie z Wikieł Z 1.62 c) moja wersja nr 873}

Rozwiązać nierówności $(14-x)(x+7)^{2}(19-x)^{3}\le0$.
\zadStop
\rozwStart{Patryk Wirkus}{}
Miejsca zerowe naszego wielomianu to: $14, -7, 19$.\\
Wielomian jest stopnia parzystego, ponadto znak współczynnika przy\linebreak najwyższej potędze x jest ujemny.\\ W związku z tym wykres wielomianu zaczyna się od lewej strony powyżej osi OX.\\
Ponadto w punkcie $-7$ wykres odbija się od osi poziomej.\\
A więc $$x \in \{-7\} \cup [14,19].$$
\rozwStop
\odpStart
$x \in \{-7\} \cup [14,19]$
\odpStop
\testStart
A.$x \in \{-7\} \cup [14,19]$\\
B.$x \in \{7\} \cup (14,19)$\\
C.$x \in \{-7\} \cup (14,19]$\\
D.$x \in \{7\} \cup (14,19]$\\
E.$x \in \{-7\} \cup [14,19)$\\
F.$x \in \{7\} \cup [14,19)$\\
G.$x \in \{-7\} \cup (14,19)$\\
H.$x \in \{7\} \cup [14,19]$
\testStop
\kluczStart
A
\kluczStop



\zadStart{Zadanie z Wikieł Z 1.62 c) moja wersja nr 874}

Rozwiązać nierówności $(14-x)(x+7)^{2}(20-x)^{3}\le0$.
\zadStop
\rozwStart{Patryk Wirkus}{}
Miejsca zerowe naszego wielomianu to: $14, -7, 20$.\\
Wielomian jest stopnia parzystego, ponadto znak współczynnika przy\linebreak najwyższej potędze x jest ujemny.\\ W związku z tym wykres wielomianu zaczyna się od lewej strony powyżej osi OX.\\
Ponadto w punkcie $-7$ wykres odbija się od osi poziomej.\\
A więc $$x \in \{-7\} \cup [14,20].$$
\rozwStop
\odpStart
$x \in \{-7\} \cup [14,20]$
\odpStop
\testStart
A.$x \in \{-7\} \cup [14,20]$\\
B.$x \in \{7\} \cup (14,20)$\\
C.$x \in \{-7\} \cup (14,20]$\\
D.$x \in \{7\} \cup (14,20]$\\
E.$x \in \{-7\} \cup [14,20)$\\
F.$x \in \{7\} \cup [14,20)$\\
G.$x \in \{-7\} \cup (14,20)$\\
H.$x \in \{7\} \cup [14,20]$
\testStop
\kluczStart
A
\kluczStop



\zadStart{Zadanie z Wikieł Z 1.62 c) moja wersja nr 875}

Rozwiązać nierówności $(14-x)(x+8)^{2}(15-x)^{3}\le0$.
\zadStop
\rozwStart{Patryk Wirkus}{}
Miejsca zerowe naszego wielomianu to: $14, -8, 15$.\\
Wielomian jest stopnia parzystego, ponadto znak współczynnika przy\linebreak najwyższej potędze x jest ujemny.\\ W związku z tym wykres wielomianu zaczyna się od lewej strony powyżej osi OX.\\
Ponadto w punkcie $-8$ wykres odbija się od osi poziomej.\\
A więc $$x \in \{-8\} \cup [14,15].$$
\rozwStop
\odpStart
$x \in \{-8\} \cup [14,15]$
\odpStop
\testStart
A.$x \in \{-8\} \cup [14,15]$\\
B.$x \in \{8\} \cup (14,15)$\\
C.$x \in \{-8\} \cup (14,15]$\\
D.$x \in \{8\} \cup (14,15]$\\
E.$x \in \{-8\} \cup [14,15)$\\
F.$x \in \{8\} \cup [14,15)$\\
G.$x \in \{-8\} \cup (14,15)$\\
H.$x \in \{8\} \cup [14,15]$
\testStop
\kluczStart
A
\kluczStop



\zadStart{Zadanie z Wikieł Z 1.62 c) moja wersja nr 876}

Rozwiązać nierówności $(14-x)(x+8)^{2}(16-x)^{3}\le0$.
\zadStop
\rozwStart{Patryk Wirkus}{}
Miejsca zerowe naszego wielomianu to: $14, -8, 16$.\\
Wielomian jest stopnia parzystego, ponadto znak współczynnika przy\linebreak najwyższej potędze x jest ujemny.\\ W związku z tym wykres wielomianu zaczyna się od lewej strony powyżej osi OX.\\
Ponadto w punkcie $-8$ wykres odbija się od osi poziomej.\\
A więc $$x \in \{-8\} \cup [14,16].$$
\rozwStop
\odpStart
$x \in \{-8\} \cup [14,16]$
\odpStop
\testStart
A.$x \in \{-8\} \cup [14,16]$\\
B.$x \in \{8\} \cup (14,16)$\\
C.$x \in \{-8\} \cup (14,16]$\\
D.$x \in \{8\} \cup (14,16]$\\
E.$x \in \{-8\} \cup [14,16)$\\
F.$x \in \{8\} \cup [14,16)$\\
G.$x \in \{-8\} \cup (14,16)$\\
H.$x \in \{8\} \cup [14,16]$
\testStop
\kluczStart
A
\kluczStop



\zadStart{Zadanie z Wikieł Z 1.62 c) moja wersja nr 877}

Rozwiązać nierówności $(14-x)(x+8)^{2}(17-x)^{3}\le0$.
\zadStop
\rozwStart{Patryk Wirkus}{}
Miejsca zerowe naszego wielomianu to: $14, -8, 17$.\\
Wielomian jest stopnia parzystego, ponadto znak współczynnika przy\linebreak najwyższej potędze x jest ujemny.\\ W związku z tym wykres wielomianu zaczyna się od lewej strony powyżej osi OX.\\
Ponadto w punkcie $-8$ wykres odbija się od osi poziomej.\\
A więc $$x \in \{-8\} \cup [14,17].$$
\rozwStop
\odpStart
$x \in \{-8\} \cup [14,17]$
\odpStop
\testStart
A.$x \in \{-8\} \cup [14,17]$\\
B.$x \in \{8\} \cup (14,17)$\\
C.$x \in \{-8\} \cup (14,17]$\\
D.$x \in \{8\} \cup (14,17]$\\
E.$x \in \{-8\} \cup [14,17)$\\
F.$x \in \{8\} \cup [14,17)$\\
G.$x \in \{-8\} \cup (14,17)$\\
H.$x \in \{8\} \cup [14,17]$
\testStop
\kluczStart
A
\kluczStop



\zadStart{Zadanie z Wikieł Z 1.62 c) moja wersja nr 878}

Rozwiązać nierówności $(14-x)(x+8)^{2}(18-x)^{3}\le0$.
\zadStop
\rozwStart{Patryk Wirkus}{}
Miejsca zerowe naszego wielomianu to: $14, -8, 18$.\\
Wielomian jest stopnia parzystego, ponadto znak współczynnika przy\linebreak najwyższej potędze x jest ujemny.\\ W związku z tym wykres wielomianu zaczyna się od lewej strony powyżej osi OX.\\
Ponadto w punkcie $-8$ wykres odbija się od osi poziomej.\\
A więc $$x \in \{-8\} \cup [14,18].$$
\rozwStop
\odpStart
$x \in \{-8\} \cup [14,18]$
\odpStop
\testStart
A.$x \in \{-8\} \cup [14,18]$\\
B.$x \in \{8\} \cup (14,18)$\\
C.$x \in \{-8\} \cup (14,18]$\\
D.$x \in \{8\} \cup (14,18]$\\
E.$x \in \{-8\} \cup [14,18)$\\
F.$x \in \{8\} \cup [14,18)$\\
G.$x \in \{-8\} \cup (14,18)$\\
H.$x \in \{8\} \cup [14,18]$
\testStop
\kluczStart
A
\kluczStop



\zadStart{Zadanie z Wikieł Z 1.62 c) moja wersja nr 879}

Rozwiązać nierówności $(14-x)(x+8)^{2}(19-x)^{3}\le0$.
\zadStop
\rozwStart{Patryk Wirkus}{}
Miejsca zerowe naszego wielomianu to: $14, -8, 19$.\\
Wielomian jest stopnia parzystego, ponadto znak współczynnika przy\linebreak najwyższej potędze x jest ujemny.\\ W związku z tym wykres wielomianu zaczyna się od lewej strony powyżej osi OX.\\
Ponadto w punkcie $-8$ wykres odbija się od osi poziomej.\\
A więc $$x \in \{-8\} \cup [14,19].$$
\rozwStop
\odpStart
$x \in \{-8\} \cup [14,19]$
\odpStop
\testStart
A.$x \in \{-8\} \cup [14,19]$\\
B.$x \in \{8\} \cup (14,19)$\\
C.$x \in \{-8\} \cup (14,19]$\\
D.$x \in \{8\} \cup (14,19]$\\
E.$x \in \{-8\} \cup [14,19)$\\
F.$x \in \{8\} \cup [14,19)$\\
G.$x \in \{-8\} \cup (14,19)$\\
H.$x \in \{8\} \cup [14,19]$
\testStop
\kluczStart
A
\kluczStop



\zadStart{Zadanie z Wikieł Z 1.62 c) moja wersja nr 880}

Rozwiązać nierówności $(14-x)(x+8)^{2}(20-x)^{3}\le0$.
\zadStop
\rozwStart{Patryk Wirkus}{}
Miejsca zerowe naszego wielomianu to: $14, -8, 20$.\\
Wielomian jest stopnia parzystego, ponadto znak współczynnika przy\linebreak najwyższej potędze x jest ujemny.\\ W związku z tym wykres wielomianu zaczyna się od lewej strony powyżej osi OX.\\
Ponadto w punkcie $-8$ wykres odbija się od osi poziomej.\\
A więc $$x \in \{-8\} \cup [14,20].$$
\rozwStop
\odpStart
$x \in \{-8\} \cup [14,20]$
\odpStop
\testStart
A.$x \in \{-8\} \cup [14,20]$\\
B.$x \in \{8\} \cup (14,20)$\\
C.$x \in \{-8\} \cup (14,20]$\\
D.$x \in \{8\} \cup (14,20]$\\
E.$x \in \{-8\} \cup [14,20)$\\
F.$x \in \{8\} \cup [14,20)$\\
G.$x \in \{-8\} \cup (14,20)$\\
H.$x \in \{8\} \cup [14,20]$
\testStop
\kluczStart
A
\kluczStop



\zadStart{Zadanie z Wikieł Z 1.62 c) moja wersja nr 881}

Rozwiązać nierówności $(14-x)(x+9)^{2}(15-x)^{3}\le0$.
\zadStop
\rozwStart{Patryk Wirkus}{}
Miejsca zerowe naszego wielomianu to: $14, -9, 15$.\\
Wielomian jest stopnia parzystego, ponadto znak współczynnika przy\linebreak najwyższej potędze x jest ujemny.\\ W związku z tym wykres wielomianu zaczyna się od lewej strony powyżej osi OX.\\
Ponadto w punkcie $-9$ wykres odbija się od osi poziomej.\\
A więc $$x \in \{-9\} \cup [14,15].$$
\rozwStop
\odpStart
$x \in \{-9\} \cup [14,15]$
\odpStop
\testStart
A.$x \in \{-9\} \cup [14,15]$\\
B.$x \in \{9\} \cup (14,15)$\\
C.$x \in \{-9\} \cup (14,15]$\\
D.$x \in \{9\} \cup (14,15]$\\
E.$x \in \{-9\} \cup [14,15)$\\
F.$x \in \{9\} \cup [14,15)$\\
G.$x \in \{-9\} \cup (14,15)$\\
H.$x \in \{9\} \cup [14,15]$
\testStop
\kluczStart
A
\kluczStop



\zadStart{Zadanie z Wikieł Z 1.62 c) moja wersja nr 882}

Rozwiązać nierówności $(14-x)(x+9)^{2}(16-x)^{3}\le0$.
\zadStop
\rozwStart{Patryk Wirkus}{}
Miejsca zerowe naszego wielomianu to: $14, -9, 16$.\\
Wielomian jest stopnia parzystego, ponadto znak współczynnika przy\linebreak najwyższej potędze x jest ujemny.\\ W związku z tym wykres wielomianu zaczyna się od lewej strony powyżej osi OX.\\
Ponadto w punkcie $-9$ wykres odbija się od osi poziomej.\\
A więc $$x \in \{-9\} \cup [14,16].$$
\rozwStop
\odpStart
$x \in \{-9\} \cup [14,16]$
\odpStop
\testStart
A.$x \in \{-9\} \cup [14,16]$\\
B.$x \in \{9\} \cup (14,16)$\\
C.$x \in \{-9\} \cup (14,16]$\\
D.$x \in \{9\} \cup (14,16]$\\
E.$x \in \{-9\} \cup [14,16)$\\
F.$x \in \{9\} \cup [14,16)$\\
G.$x \in \{-9\} \cup (14,16)$\\
H.$x \in \{9\} \cup [14,16]$
\testStop
\kluczStart
A
\kluczStop



\zadStart{Zadanie z Wikieł Z 1.62 c) moja wersja nr 883}

Rozwiązać nierówności $(14-x)(x+9)^{2}(17-x)^{3}\le0$.
\zadStop
\rozwStart{Patryk Wirkus}{}
Miejsca zerowe naszego wielomianu to: $14, -9, 17$.\\
Wielomian jest stopnia parzystego, ponadto znak współczynnika przy\linebreak najwyższej potędze x jest ujemny.\\ W związku z tym wykres wielomianu zaczyna się od lewej strony powyżej osi OX.\\
Ponadto w punkcie $-9$ wykres odbija się od osi poziomej.\\
A więc $$x \in \{-9\} \cup [14,17].$$
\rozwStop
\odpStart
$x \in \{-9\} \cup [14,17]$
\odpStop
\testStart
A.$x \in \{-9\} \cup [14,17]$\\
B.$x \in \{9\} \cup (14,17)$\\
C.$x \in \{-9\} \cup (14,17]$\\
D.$x \in \{9\} \cup (14,17]$\\
E.$x \in \{-9\} \cup [14,17)$\\
F.$x \in \{9\} \cup [14,17)$\\
G.$x \in \{-9\} \cup (14,17)$\\
H.$x \in \{9\} \cup [14,17]$
\testStop
\kluczStart
A
\kluczStop



\zadStart{Zadanie z Wikieł Z 1.62 c) moja wersja nr 884}

Rozwiązać nierówności $(14-x)(x+9)^{2}(18-x)^{3}\le0$.
\zadStop
\rozwStart{Patryk Wirkus}{}
Miejsca zerowe naszego wielomianu to: $14, -9, 18$.\\
Wielomian jest stopnia parzystego, ponadto znak współczynnika przy\linebreak najwyższej potędze x jest ujemny.\\ W związku z tym wykres wielomianu zaczyna się od lewej strony powyżej osi OX.\\
Ponadto w punkcie $-9$ wykres odbija się od osi poziomej.\\
A więc $$x \in \{-9\} \cup [14,18].$$
\rozwStop
\odpStart
$x \in \{-9\} \cup [14,18]$
\odpStop
\testStart
A.$x \in \{-9\} \cup [14,18]$\\
B.$x \in \{9\} \cup (14,18)$\\
C.$x \in \{-9\} \cup (14,18]$\\
D.$x \in \{9\} \cup (14,18]$\\
E.$x \in \{-9\} \cup [14,18)$\\
F.$x \in \{9\} \cup [14,18)$\\
G.$x \in \{-9\} \cup (14,18)$\\
H.$x \in \{9\} \cup [14,18]$
\testStop
\kluczStart
A
\kluczStop



\zadStart{Zadanie z Wikieł Z 1.62 c) moja wersja nr 885}

Rozwiązać nierówności $(14-x)(x+9)^{2}(19-x)^{3}\le0$.
\zadStop
\rozwStart{Patryk Wirkus}{}
Miejsca zerowe naszego wielomianu to: $14, -9, 19$.\\
Wielomian jest stopnia parzystego, ponadto znak współczynnika przy\linebreak najwyższej potędze x jest ujemny.\\ W związku z tym wykres wielomianu zaczyna się od lewej strony powyżej osi OX.\\
Ponadto w punkcie $-9$ wykres odbija się od osi poziomej.\\
A więc $$x \in \{-9\} \cup [14,19].$$
\rozwStop
\odpStart
$x \in \{-9\} \cup [14,19]$
\odpStop
\testStart
A.$x \in \{-9\} \cup [14,19]$\\
B.$x \in \{9\} \cup (14,19)$\\
C.$x \in \{-9\} \cup (14,19]$\\
D.$x \in \{9\} \cup (14,19]$\\
E.$x \in \{-9\} \cup [14,19)$\\
F.$x \in \{9\} \cup [14,19)$\\
G.$x \in \{-9\} \cup (14,19)$\\
H.$x \in \{9\} \cup [14,19]$
\testStop
\kluczStart
A
\kluczStop



\zadStart{Zadanie z Wikieł Z 1.62 c) moja wersja nr 886}

Rozwiązać nierówności $(14-x)(x+9)^{2}(20-x)^{3}\le0$.
\zadStop
\rozwStart{Patryk Wirkus}{}
Miejsca zerowe naszego wielomianu to: $14, -9, 20$.\\
Wielomian jest stopnia parzystego, ponadto znak współczynnika przy\linebreak najwyższej potędze x jest ujemny.\\ W związku z tym wykres wielomianu zaczyna się od lewej strony powyżej osi OX.\\
Ponadto w punkcie $-9$ wykres odbija się od osi poziomej.\\
A więc $$x \in \{-9\} \cup [14,20].$$
\rozwStop
\odpStart
$x \in \{-9\} \cup [14,20]$
\odpStop
\testStart
A.$x \in \{-9\} \cup [14,20]$\\
B.$x \in \{9\} \cup (14,20)$\\
C.$x \in \{-9\} \cup (14,20]$\\
D.$x \in \{9\} \cup (14,20]$\\
E.$x \in \{-9\} \cup [14,20)$\\
F.$x \in \{9\} \cup [14,20)$\\
G.$x \in \{-9\} \cup (14,20)$\\
H.$x \in \{9\} \cup [14,20]$
\testStop
\kluczStart
A
\kluczStop



\zadStart{Zadanie z Wikieł Z 1.62 c) moja wersja nr 887}

Rozwiązać nierówności $(14-x)(x+10)^{2}(15-x)^{3}\le0$.
\zadStop
\rozwStart{Patryk Wirkus}{}
Miejsca zerowe naszego wielomianu to: $14, -10, 15$.\\
Wielomian jest stopnia parzystego, ponadto znak współczynnika przy\linebreak najwyższej potędze x jest ujemny.\\ W związku z tym wykres wielomianu zaczyna się od lewej strony powyżej osi OX.\\
Ponadto w punkcie $-10$ wykres odbija się od osi poziomej.\\
A więc $$x \in \{-10\} \cup [14,15].$$
\rozwStop
\odpStart
$x \in \{-10\} \cup [14,15]$
\odpStop
\testStart
A.$x \in \{-10\} \cup [14,15]$\\
B.$x \in \{10\} \cup (14,15)$\\
C.$x \in \{-10\} \cup (14,15]$\\
D.$x \in \{10\} \cup (14,15]$\\
E.$x \in \{-10\} \cup [14,15)$\\
F.$x \in \{10\} \cup [14,15)$\\
G.$x \in \{-10\} \cup (14,15)$\\
H.$x \in \{10\} \cup [14,15]$
\testStop
\kluczStart
A
\kluczStop



\zadStart{Zadanie z Wikieł Z 1.62 c) moja wersja nr 888}

Rozwiązać nierówności $(14-x)(x+10)^{2}(16-x)^{3}\le0$.
\zadStop
\rozwStart{Patryk Wirkus}{}
Miejsca zerowe naszego wielomianu to: $14, -10, 16$.\\
Wielomian jest stopnia parzystego, ponadto znak współczynnika przy\linebreak najwyższej potędze x jest ujemny.\\ W związku z tym wykres wielomianu zaczyna się od lewej strony powyżej osi OX.\\
Ponadto w punkcie $-10$ wykres odbija się od osi poziomej.\\
A więc $$x \in \{-10\} \cup [14,16].$$
\rozwStop
\odpStart
$x \in \{-10\} \cup [14,16]$
\odpStop
\testStart
A.$x \in \{-10\} \cup [14,16]$\\
B.$x \in \{10\} \cup (14,16)$\\
C.$x \in \{-10\} \cup (14,16]$\\
D.$x \in \{10\} \cup (14,16]$\\
E.$x \in \{-10\} \cup [14,16)$\\
F.$x \in \{10\} \cup [14,16)$\\
G.$x \in \{-10\} \cup (14,16)$\\
H.$x \in \{10\} \cup [14,16]$
\testStop
\kluczStart
A
\kluczStop



\zadStart{Zadanie z Wikieł Z 1.62 c) moja wersja nr 889}

Rozwiązać nierówności $(14-x)(x+10)^{2}(17-x)^{3}\le0$.
\zadStop
\rozwStart{Patryk Wirkus}{}
Miejsca zerowe naszego wielomianu to: $14, -10, 17$.\\
Wielomian jest stopnia parzystego, ponadto znak współczynnika przy\linebreak najwyższej potędze x jest ujemny.\\ W związku z tym wykres wielomianu zaczyna się od lewej strony powyżej osi OX.\\
Ponadto w punkcie $-10$ wykres odbija się od osi poziomej.\\
A więc $$x \in \{-10\} \cup [14,17].$$
\rozwStop
\odpStart
$x \in \{-10\} \cup [14,17]$
\odpStop
\testStart
A.$x \in \{-10\} \cup [14,17]$\\
B.$x \in \{10\} \cup (14,17)$\\
C.$x \in \{-10\} \cup (14,17]$\\
D.$x \in \{10\} \cup (14,17]$\\
E.$x \in \{-10\} \cup [14,17)$\\
F.$x \in \{10\} \cup [14,17)$\\
G.$x \in \{-10\} \cup (14,17)$\\
H.$x \in \{10\} \cup [14,17]$
\testStop
\kluczStart
A
\kluczStop



\zadStart{Zadanie z Wikieł Z 1.62 c) moja wersja nr 890}

Rozwiązać nierówności $(14-x)(x+10)^{2}(18-x)^{3}\le0$.
\zadStop
\rozwStart{Patryk Wirkus}{}
Miejsca zerowe naszego wielomianu to: $14, -10, 18$.\\
Wielomian jest stopnia parzystego, ponadto znak współczynnika przy\linebreak najwyższej potędze x jest ujemny.\\ W związku z tym wykres wielomianu zaczyna się od lewej strony powyżej osi OX.\\
Ponadto w punkcie $-10$ wykres odbija się od osi poziomej.\\
A więc $$x \in \{-10\} \cup [14,18].$$
\rozwStop
\odpStart
$x \in \{-10\} \cup [14,18]$
\odpStop
\testStart
A.$x \in \{-10\} \cup [14,18]$\\
B.$x \in \{10\} \cup (14,18)$\\
C.$x \in \{-10\} \cup (14,18]$\\
D.$x \in \{10\} \cup (14,18]$\\
E.$x \in \{-10\} \cup [14,18)$\\
F.$x \in \{10\} \cup [14,18)$\\
G.$x \in \{-10\} \cup (14,18)$\\
H.$x \in \{10\} \cup [14,18]$
\testStop
\kluczStart
A
\kluczStop



\zadStart{Zadanie z Wikieł Z 1.62 c) moja wersja nr 891}

Rozwiązać nierówności $(14-x)(x+10)^{2}(19-x)^{3}\le0$.
\zadStop
\rozwStart{Patryk Wirkus}{}
Miejsca zerowe naszego wielomianu to: $14, -10, 19$.\\
Wielomian jest stopnia parzystego, ponadto znak współczynnika przy\linebreak najwyższej potędze x jest ujemny.\\ W związku z tym wykres wielomianu zaczyna się od lewej strony powyżej osi OX.\\
Ponadto w punkcie $-10$ wykres odbija się od osi poziomej.\\
A więc $$x \in \{-10\} \cup [14,19].$$
\rozwStop
\odpStart
$x \in \{-10\} \cup [14,19]$
\odpStop
\testStart
A.$x \in \{-10\} \cup [14,19]$\\
B.$x \in \{10\} \cup (14,19)$\\
C.$x \in \{-10\} \cup (14,19]$\\
D.$x \in \{10\} \cup (14,19]$\\
E.$x \in \{-10\} \cup [14,19)$\\
F.$x \in \{10\} \cup [14,19)$\\
G.$x \in \{-10\} \cup (14,19)$\\
H.$x \in \{10\} \cup [14,19]$
\testStop
\kluczStart
A
\kluczStop



\zadStart{Zadanie z Wikieł Z 1.62 c) moja wersja nr 892}

Rozwiązać nierówności $(14-x)(x+10)^{2}(20-x)^{3}\le0$.
\zadStop
\rozwStart{Patryk Wirkus}{}
Miejsca zerowe naszego wielomianu to: $14, -10, 20$.\\
Wielomian jest stopnia parzystego, ponadto znak współczynnika przy\linebreak najwyższej potędze x jest ujemny.\\ W związku z tym wykres wielomianu zaczyna się od lewej strony powyżej osi OX.\\
Ponadto w punkcie $-10$ wykres odbija się od osi poziomej.\\
A więc $$x \in \{-10\} \cup [14,20].$$
\rozwStop
\odpStart
$x \in \{-10\} \cup [14,20]$
\odpStop
\testStart
A.$x \in \{-10\} \cup [14,20]$\\
B.$x \in \{10\} \cup (14,20)$\\
C.$x \in \{-10\} \cup (14,20]$\\
D.$x \in \{10\} \cup (14,20]$\\
E.$x \in \{-10\} \cup [14,20)$\\
F.$x \in \{10\} \cup [14,20)$\\
G.$x \in \{-10\} \cup (14,20)$\\
H.$x \in \{10\} \cup [14,20]$
\testStop
\kluczStart
A
\kluczStop



\zadStart{Zadanie z Wikieł Z 1.62 c) moja wersja nr 893}

Rozwiązać nierówności $(14-x)(x+11)^{2}(15-x)^{3}\le0$.
\zadStop
\rozwStart{Patryk Wirkus}{}
Miejsca zerowe naszego wielomianu to: $14, -11, 15$.\\
Wielomian jest stopnia parzystego, ponadto znak współczynnika przy\linebreak najwyższej potędze x jest ujemny.\\ W związku z tym wykres wielomianu zaczyna się od lewej strony powyżej osi OX.\\
Ponadto w punkcie $-11$ wykres odbija się od osi poziomej.\\
A więc $$x \in \{-11\} \cup [14,15].$$
\rozwStop
\odpStart
$x \in \{-11\} \cup [14,15]$
\odpStop
\testStart
A.$x \in \{-11\} \cup [14,15]$\\
B.$x \in \{11\} \cup (14,15)$\\
C.$x \in \{-11\} \cup (14,15]$\\
D.$x \in \{11\} \cup (14,15]$\\
E.$x \in \{-11\} \cup [14,15)$\\
F.$x \in \{11\} \cup [14,15)$\\
G.$x \in \{-11\} \cup (14,15)$\\
H.$x \in \{11\} \cup [14,15]$
\testStop
\kluczStart
A
\kluczStop



\zadStart{Zadanie z Wikieł Z 1.62 c) moja wersja nr 894}

Rozwiązać nierówności $(14-x)(x+11)^{2}(16-x)^{3}\le0$.
\zadStop
\rozwStart{Patryk Wirkus}{}
Miejsca zerowe naszego wielomianu to: $14, -11, 16$.\\
Wielomian jest stopnia parzystego, ponadto znak współczynnika przy\linebreak najwyższej potędze x jest ujemny.\\ W związku z tym wykres wielomianu zaczyna się od lewej strony powyżej osi OX.\\
Ponadto w punkcie $-11$ wykres odbija się od osi poziomej.\\
A więc $$x \in \{-11\} \cup [14,16].$$
\rozwStop
\odpStart
$x \in \{-11\} \cup [14,16]$
\odpStop
\testStart
A.$x \in \{-11\} \cup [14,16]$\\
B.$x \in \{11\} \cup (14,16)$\\
C.$x \in \{-11\} \cup (14,16]$\\
D.$x \in \{11\} \cup (14,16]$\\
E.$x \in \{-11\} \cup [14,16)$\\
F.$x \in \{11\} \cup [14,16)$\\
G.$x \in \{-11\} \cup (14,16)$\\
H.$x \in \{11\} \cup [14,16]$
\testStop
\kluczStart
A
\kluczStop



\zadStart{Zadanie z Wikieł Z 1.62 c) moja wersja nr 895}

Rozwiązać nierówności $(14-x)(x+11)^{2}(17-x)^{3}\le0$.
\zadStop
\rozwStart{Patryk Wirkus}{}
Miejsca zerowe naszego wielomianu to: $14, -11, 17$.\\
Wielomian jest stopnia parzystego, ponadto znak współczynnika przy\linebreak najwyższej potędze x jest ujemny.\\ W związku z tym wykres wielomianu zaczyna się od lewej strony powyżej osi OX.\\
Ponadto w punkcie $-11$ wykres odbija się od osi poziomej.\\
A więc $$x \in \{-11\} \cup [14,17].$$
\rozwStop
\odpStart
$x \in \{-11\} \cup [14,17]$
\odpStop
\testStart
A.$x \in \{-11\} \cup [14,17]$\\
B.$x \in \{11\} \cup (14,17)$\\
C.$x \in \{-11\} \cup (14,17]$\\
D.$x \in \{11\} \cup (14,17]$\\
E.$x \in \{-11\} \cup [14,17)$\\
F.$x \in \{11\} \cup [14,17)$\\
G.$x \in \{-11\} \cup (14,17)$\\
H.$x \in \{11\} \cup [14,17]$
\testStop
\kluczStart
A
\kluczStop



\zadStart{Zadanie z Wikieł Z 1.62 c) moja wersja nr 896}

Rozwiązać nierówności $(14-x)(x+11)^{2}(18-x)^{3}\le0$.
\zadStop
\rozwStart{Patryk Wirkus}{}
Miejsca zerowe naszego wielomianu to: $14, -11, 18$.\\
Wielomian jest stopnia parzystego, ponadto znak współczynnika przy\linebreak najwyższej potędze x jest ujemny.\\ W związku z tym wykres wielomianu zaczyna się od lewej strony powyżej osi OX.\\
Ponadto w punkcie $-11$ wykres odbija się od osi poziomej.\\
A więc $$x \in \{-11\} \cup [14,18].$$
\rozwStop
\odpStart
$x \in \{-11\} \cup [14,18]$
\odpStop
\testStart
A.$x \in \{-11\} \cup [14,18]$\\
B.$x \in \{11\} \cup (14,18)$\\
C.$x \in \{-11\} \cup (14,18]$\\
D.$x \in \{11\} \cup (14,18]$\\
E.$x \in \{-11\} \cup [14,18)$\\
F.$x \in \{11\} \cup [14,18)$\\
G.$x \in \{-11\} \cup (14,18)$\\
H.$x \in \{11\} \cup [14,18]$
\testStop
\kluczStart
A
\kluczStop



\zadStart{Zadanie z Wikieł Z 1.62 c) moja wersja nr 897}

Rozwiązać nierówności $(14-x)(x+11)^{2}(19-x)^{3}\le0$.
\zadStop
\rozwStart{Patryk Wirkus}{}
Miejsca zerowe naszego wielomianu to: $14, -11, 19$.\\
Wielomian jest stopnia parzystego, ponadto znak współczynnika przy\linebreak najwyższej potędze x jest ujemny.\\ W związku z tym wykres wielomianu zaczyna się od lewej strony powyżej osi OX.\\
Ponadto w punkcie $-11$ wykres odbija się od osi poziomej.\\
A więc $$x \in \{-11\} \cup [14,19].$$
\rozwStop
\odpStart
$x \in \{-11\} \cup [14,19]$
\odpStop
\testStart
A.$x \in \{-11\} \cup [14,19]$\\
B.$x \in \{11\} \cup (14,19)$\\
C.$x \in \{-11\} \cup (14,19]$\\
D.$x \in \{11\} \cup (14,19]$\\
E.$x \in \{-11\} \cup [14,19)$\\
F.$x \in \{11\} \cup [14,19)$\\
G.$x \in \{-11\} \cup (14,19)$\\
H.$x \in \{11\} \cup [14,19]$
\testStop
\kluczStart
A
\kluczStop



\zadStart{Zadanie z Wikieł Z 1.62 c) moja wersja nr 898}

Rozwiązać nierówności $(14-x)(x+11)^{2}(20-x)^{3}\le0$.
\zadStop
\rozwStart{Patryk Wirkus}{}
Miejsca zerowe naszego wielomianu to: $14, -11, 20$.\\
Wielomian jest stopnia parzystego, ponadto znak współczynnika przy\linebreak najwyższej potędze x jest ujemny.\\ W związku z tym wykres wielomianu zaczyna się od lewej strony powyżej osi OX.\\
Ponadto w punkcie $-11$ wykres odbija się od osi poziomej.\\
A więc $$x \in \{-11\} \cup [14,20].$$
\rozwStop
\odpStart
$x \in \{-11\} \cup [14,20]$
\odpStop
\testStart
A.$x \in \{-11\} \cup [14,20]$\\
B.$x \in \{11\} \cup (14,20)$\\
C.$x \in \{-11\} \cup (14,20]$\\
D.$x \in \{11\} \cup (14,20]$\\
E.$x \in \{-11\} \cup [14,20)$\\
F.$x \in \{11\} \cup [14,20)$\\
G.$x \in \{-11\} \cup (14,20)$\\
H.$x \in \{11\} \cup [14,20]$
\testStop
\kluczStart
A
\kluczStop



\zadStart{Zadanie z Wikieł Z 1.62 c) moja wersja nr 899}

Rozwiązać nierówności $(14-x)(x+12)^{2}(15-x)^{3}\le0$.
\zadStop
\rozwStart{Patryk Wirkus}{}
Miejsca zerowe naszego wielomianu to: $14, -12, 15$.\\
Wielomian jest stopnia parzystego, ponadto znak współczynnika przy\linebreak najwyższej potędze x jest ujemny.\\ W związku z tym wykres wielomianu zaczyna się od lewej strony powyżej osi OX.\\
Ponadto w punkcie $-12$ wykres odbija się od osi poziomej.\\
A więc $$x \in \{-12\} \cup [14,15].$$
\rozwStop
\odpStart
$x \in \{-12\} \cup [14,15]$
\odpStop
\testStart
A.$x \in \{-12\} \cup [14,15]$\\
B.$x \in \{12\} \cup (14,15)$\\
C.$x \in \{-12\} \cup (14,15]$\\
D.$x \in \{12\} \cup (14,15]$\\
E.$x \in \{-12\} \cup [14,15)$\\
F.$x \in \{12\} \cup [14,15)$\\
G.$x \in \{-12\} \cup (14,15)$\\
H.$x \in \{12\} \cup [14,15]$
\testStop
\kluczStart
A
\kluczStop



\zadStart{Zadanie z Wikieł Z 1.62 c) moja wersja nr 900}

Rozwiązać nierówności $(14-x)(x+12)^{2}(16-x)^{3}\le0$.
\zadStop
\rozwStart{Patryk Wirkus}{}
Miejsca zerowe naszego wielomianu to: $14, -12, 16$.\\
Wielomian jest stopnia parzystego, ponadto znak współczynnika przy\linebreak najwyższej potędze x jest ujemny.\\ W związku z tym wykres wielomianu zaczyna się od lewej strony powyżej osi OX.\\
Ponadto w punkcie $-12$ wykres odbija się od osi poziomej.\\
A więc $$x \in \{-12\} \cup [14,16].$$
\rozwStop
\odpStart
$x \in \{-12\} \cup [14,16]$
\odpStop
\testStart
A.$x \in \{-12\} \cup [14,16]$\\
B.$x \in \{12\} \cup (14,16)$\\
C.$x \in \{-12\} \cup (14,16]$\\
D.$x \in \{12\} \cup (14,16]$\\
E.$x \in \{-12\} \cup [14,16)$\\
F.$x \in \{12\} \cup [14,16)$\\
G.$x \in \{-12\} \cup (14,16)$\\
H.$x \in \{12\} \cup [14,16]$
\testStop
\kluczStart
A
\kluczStop



\zadStart{Zadanie z Wikieł Z 1.62 c) moja wersja nr 901}

Rozwiązać nierówności $(14-x)(x+12)^{2}(17-x)^{3}\le0$.
\zadStop
\rozwStart{Patryk Wirkus}{}
Miejsca zerowe naszego wielomianu to: $14, -12, 17$.\\
Wielomian jest stopnia parzystego, ponadto znak współczynnika przy\linebreak najwyższej potędze x jest ujemny.\\ W związku z tym wykres wielomianu zaczyna się od lewej strony powyżej osi OX.\\
Ponadto w punkcie $-12$ wykres odbija się od osi poziomej.\\
A więc $$x \in \{-12\} \cup [14,17].$$
\rozwStop
\odpStart
$x \in \{-12\} \cup [14,17]$
\odpStop
\testStart
A.$x \in \{-12\} \cup [14,17]$\\
B.$x \in \{12\} \cup (14,17)$\\
C.$x \in \{-12\} \cup (14,17]$\\
D.$x \in \{12\} \cup (14,17]$\\
E.$x \in \{-12\} \cup [14,17)$\\
F.$x \in \{12\} \cup [14,17)$\\
G.$x \in \{-12\} \cup (14,17)$\\
H.$x \in \{12\} \cup [14,17]$
\testStop
\kluczStart
A
\kluczStop



\zadStart{Zadanie z Wikieł Z 1.62 c) moja wersja nr 902}

Rozwiązać nierówności $(14-x)(x+12)^{2}(18-x)^{3}\le0$.
\zadStop
\rozwStart{Patryk Wirkus}{}
Miejsca zerowe naszego wielomianu to: $14, -12, 18$.\\
Wielomian jest stopnia parzystego, ponadto znak współczynnika przy\linebreak najwyższej potędze x jest ujemny.\\ W związku z tym wykres wielomianu zaczyna się od lewej strony powyżej osi OX.\\
Ponadto w punkcie $-12$ wykres odbija się od osi poziomej.\\
A więc $$x \in \{-12\} \cup [14,18].$$
\rozwStop
\odpStart
$x \in \{-12\} \cup [14,18]$
\odpStop
\testStart
A.$x \in \{-12\} \cup [14,18]$\\
B.$x \in \{12\} \cup (14,18)$\\
C.$x \in \{-12\} \cup (14,18]$\\
D.$x \in \{12\} \cup (14,18]$\\
E.$x \in \{-12\} \cup [14,18)$\\
F.$x \in \{12\} \cup [14,18)$\\
G.$x \in \{-12\} \cup (14,18)$\\
H.$x \in \{12\} \cup [14,18]$
\testStop
\kluczStart
A
\kluczStop



\zadStart{Zadanie z Wikieł Z 1.62 c) moja wersja nr 903}

Rozwiązać nierówności $(14-x)(x+12)^{2}(19-x)^{3}\le0$.
\zadStop
\rozwStart{Patryk Wirkus}{}
Miejsca zerowe naszego wielomianu to: $14, -12, 19$.\\
Wielomian jest stopnia parzystego, ponadto znak współczynnika przy\linebreak najwyższej potędze x jest ujemny.\\ W związku z tym wykres wielomianu zaczyna się od lewej strony powyżej osi OX.\\
Ponadto w punkcie $-12$ wykres odbija się od osi poziomej.\\
A więc $$x \in \{-12\} \cup [14,19].$$
\rozwStop
\odpStart
$x \in \{-12\} \cup [14,19]$
\odpStop
\testStart
A.$x \in \{-12\} \cup [14,19]$\\
B.$x \in \{12\} \cup (14,19)$\\
C.$x \in \{-12\} \cup (14,19]$\\
D.$x \in \{12\} \cup (14,19]$\\
E.$x \in \{-12\} \cup [14,19)$\\
F.$x \in \{12\} \cup [14,19)$\\
G.$x \in \{-12\} \cup (14,19)$\\
H.$x \in \{12\} \cup [14,19]$
\testStop
\kluczStart
A
\kluczStop



\zadStart{Zadanie z Wikieł Z 1.62 c) moja wersja nr 904}

Rozwiązać nierówności $(14-x)(x+12)^{2}(20-x)^{3}\le0$.
\zadStop
\rozwStart{Patryk Wirkus}{}
Miejsca zerowe naszego wielomianu to: $14, -12, 20$.\\
Wielomian jest stopnia parzystego, ponadto znak współczynnika przy\linebreak najwyższej potędze x jest ujemny.\\ W związku z tym wykres wielomianu zaczyna się od lewej strony powyżej osi OX.\\
Ponadto w punkcie $-12$ wykres odbija się od osi poziomej.\\
A więc $$x \in \{-12\} \cup [14,20].$$
\rozwStop
\odpStart
$x \in \{-12\} \cup [14,20]$
\odpStop
\testStart
A.$x \in \{-12\} \cup [14,20]$\\
B.$x \in \{12\} \cup (14,20)$\\
C.$x \in \{-12\} \cup (14,20]$\\
D.$x \in \{12\} \cup (14,20]$\\
E.$x \in \{-12\} \cup [14,20)$\\
F.$x \in \{12\} \cup [14,20)$\\
G.$x \in \{-12\} \cup (14,20)$\\
H.$x \in \{12\} \cup [14,20]$
\testStop
\kluczStart
A
\kluczStop



\zadStart{Zadanie z Wikieł Z 1.62 c) moja wersja nr 905}

Rozwiązać nierówności $(14-x)(x+13)^{2}(15-x)^{3}\le0$.
\zadStop
\rozwStart{Patryk Wirkus}{}
Miejsca zerowe naszego wielomianu to: $14, -13, 15$.\\
Wielomian jest stopnia parzystego, ponadto znak współczynnika przy\linebreak najwyższej potędze x jest ujemny.\\ W związku z tym wykres wielomianu zaczyna się od lewej strony powyżej osi OX.\\
Ponadto w punkcie $-13$ wykres odbija się od osi poziomej.\\
A więc $$x \in \{-13\} \cup [14,15].$$
\rozwStop
\odpStart
$x \in \{-13\} \cup [14,15]$
\odpStop
\testStart
A.$x \in \{-13\} \cup [14,15]$\\
B.$x \in \{13\} \cup (14,15)$\\
C.$x \in \{-13\} \cup (14,15]$\\
D.$x \in \{13\} \cup (14,15]$\\
E.$x \in \{-13\} \cup [14,15)$\\
F.$x \in \{13\} \cup [14,15)$\\
G.$x \in \{-13\} \cup (14,15)$\\
H.$x \in \{13\} \cup [14,15]$
\testStop
\kluczStart
A
\kluczStop



\zadStart{Zadanie z Wikieł Z 1.62 c) moja wersja nr 906}

Rozwiązać nierówności $(14-x)(x+13)^{2}(16-x)^{3}\le0$.
\zadStop
\rozwStart{Patryk Wirkus}{}
Miejsca zerowe naszego wielomianu to: $14, -13, 16$.\\
Wielomian jest stopnia parzystego, ponadto znak współczynnika przy\linebreak najwyższej potędze x jest ujemny.\\ W związku z tym wykres wielomianu zaczyna się od lewej strony powyżej osi OX.\\
Ponadto w punkcie $-13$ wykres odbija się od osi poziomej.\\
A więc $$x \in \{-13\} \cup [14,16].$$
\rozwStop
\odpStart
$x \in \{-13\} \cup [14,16]$
\odpStop
\testStart
A.$x \in \{-13\} \cup [14,16]$\\
B.$x \in \{13\} \cup (14,16)$\\
C.$x \in \{-13\} \cup (14,16]$\\
D.$x \in \{13\} \cup (14,16]$\\
E.$x \in \{-13\} \cup [14,16)$\\
F.$x \in \{13\} \cup [14,16)$\\
G.$x \in \{-13\} \cup (14,16)$\\
H.$x \in \{13\} \cup [14,16]$
\testStop
\kluczStart
A
\kluczStop



\zadStart{Zadanie z Wikieł Z 1.62 c) moja wersja nr 907}

Rozwiązać nierówności $(14-x)(x+13)^{2}(17-x)^{3}\le0$.
\zadStop
\rozwStart{Patryk Wirkus}{}
Miejsca zerowe naszego wielomianu to: $14, -13, 17$.\\
Wielomian jest stopnia parzystego, ponadto znak współczynnika przy\linebreak najwyższej potędze x jest ujemny.\\ W związku z tym wykres wielomianu zaczyna się od lewej strony powyżej osi OX.\\
Ponadto w punkcie $-13$ wykres odbija się od osi poziomej.\\
A więc $$x \in \{-13\} \cup [14,17].$$
\rozwStop
\odpStart
$x \in \{-13\} \cup [14,17]$
\odpStop
\testStart
A.$x \in \{-13\} \cup [14,17]$\\
B.$x \in \{13\} \cup (14,17)$\\
C.$x \in \{-13\} \cup (14,17]$\\
D.$x \in \{13\} \cup (14,17]$\\
E.$x \in \{-13\} \cup [14,17)$\\
F.$x \in \{13\} \cup [14,17)$\\
G.$x \in \{-13\} \cup (14,17)$\\
H.$x \in \{13\} \cup [14,17]$
\testStop
\kluczStart
A
\kluczStop



\zadStart{Zadanie z Wikieł Z 1.62 c) moja wersja nr 908}

Rozwiązać nierówności $(14-x)(x+13)^{2}(18-x)^{3}\le0$.
\zadStop
\rozwStart{Patryk Wirkus}{}
Miejsca zerowe naszego wielomianu to: $14, -13, 18$.\\
Wielomian jest stopnia parzystego, ponadto znak współczynnika przy\linebreak najwyższej potędze x jest ujemny.\\ W związku z tym wykres wielomianu zaczyna się od lewej strony powyżej osi OX.\\
Ponadto w punkcie $-13$ wykres odbija się od osi poziomej.\\
A więc $$x \in \{-13\} \cup [14,18].$$
\rozwStop
\odpStart
$x \in \{-13\} \cup [14,18]$
\odpStop
\testStart
A.$x \in \{-13\} \cup [14,18]$\\
B.$x \in \{13\} \cup (14,18)$\\
C.$x \in \{-13\} \cup (14,18]$\\
D.$x \in \{13\} \cup (14,18]$\\
E.$x \in \{-13\} \cup [14,18)$\\
F.$x \in \{13\} \cup [14,18)$\\
G.$x \in \{-13\} \cup (14,18)$\\
H.$x \in \{13\} \cup [14,18]$
\testStop
\kluczStart
A
\kluczStop



\zadStart{Zadanie z Wikieł Z 1.62 c) moja wersja nr 909}

Rozwiązać nierówności $(14-x)(x+13)^{2}(19-x)^{3}\le0$.
\zadStop
\rozwStart{Patryk Wirkus}{}
Miejsca zerowe naszego wielomianu to: $14, -13, 19$.\\
Wielomian jest stopnia parzystego, ponadto znak współczynnika przy\linebreak najwyższej potędze x jest ujemny.\\ W związku z tym wykres wielomianu zaczyna się od lewej strony powyżej osi OX.\\
Ponadto w punkcie $-13$ wykres odbija się od osi poziomej.\\
A więc $$x \in \{-13\} \cup [14,19].$$
\rozwStop
\odpStart
$x \in \{-13\} \cup [14,19]$
\odpStop
\testStart
A.$x \in \{-13\} \cup [14,19]$\\
B.$x \in \{13\} \cup (14,19)$\\
C.$x \in \{-13\} \cup (14,19]$\\
D.$x \in \{13\} \cup (14,19]$\\
E.$x \in \{-13\} \cup [14,19)$\\
F.$x \in \{13\} \cup [14,19)$\\
G.$x \in \{-13\} \cup (14,19)$\\
H.$x \in \{13\} \cup [14,19]$
\testStop
\kluczStart
A
\kluczStop



\zadStart{Zadanie z Wikieł Z 1.62 c) moja wersja nr 910}

Rozwiązać nierówności $(14-x)(x+13)^{2}(20-x)^{3}\le0$.
\zadStop
\rozwStart{Patryk Wirkus}{}
Miejsca zerowe naszego wielomianu to: $14, -13, 20$.\\
Wielomian jest stopnia parzystego, ponadto znak współczynnika przy\linebreak najwyższej potędze x jest ujemny.\\ W związku z tym wykres wielomianu zaczyna się od lewej strony powyżej osi OX.\\
Ponadto w punkcie $-13$ wykres odbija się od osi poziomej.\\
A więc $$x \in \{-13\} \cup [14,20].$$
\rozwStop
\odpStart
$x \in \{-13\} \cup [14,20]$
\odpStop
\testStart
A.$x \in \{-13\} \cup [14,20]$\\
B.$x \in \{13\} \cup (14,20)$\\
C.$x \in \{-13\} \cup (14,20]$\\
D.$x \in \{13\} \cup (14,20]$\\
E.$x \in \{-13\} \cup [14,20)$\\
F.$x \in \{13\} \cup [14,20)$\\
G.$x \in \{-13\} \cup (14,20)$\\
H.$x \in \{13\} \cup [14,20]$
\testStop
\kluczStart
A
\kluczStop



\zadStart{Zadanie z Wikieł Z 1.62 c) moja wersja nr 911}

Rozwiązać nierówności $(15-x)(x+1)^{2}(16-x)^{3}\le0$.
\zadStop
\rozwStart{Patryk Wirkus}{}
Miejsca zerowe naszego wielomianu to: $15, -1, 16$.\\
Wielomian jest stopnia parzystego, ponadto znak współczynnika przy\linebreak najwyższej potędze x jest ujemny.\\ W związku z tym wykres wielomianu zaczyna się od lewej strony powyżej osi OX.\\
Ponadto w punkcie $-1$ wykres odbija się od osi poziomej.\\
A więc $$x \in \{-1\} \cup [15,16].$$
\rozwStop
\odpStart
$x \in \{-1\} \cup [15,16]$
\odpStop
\testStart
A.$x \in \{-1\} \cup [15,16]$\\
B.$x \in \{1\} \cup (15,16)$\\
C.$x \in \{-1\} \cup (15,16]$\\
D.$x \in \{1\} \cup (15,16]$\\
E.$x \in \{-1\} \cup [15,16)$\\
F.$x \in \{1\} \cup [15,16)$\\
G.$x \in \{-1\} \cup (15,16)$\\
H.$x \in \{1\} \cup [15,16]$
\testStop
\kluczStart
A
\kluczStop



\zadStart{Zadanie z Wikieł Z 1.62 c) moja wersja nr 912}

Rozwiązać nierówności $(15-x)(x+1)^{2}(17-x)^{3}\le0$.
\zadStop
\rozwStart{Patryk Wirkus}{}
Miejsca zerowe naszego wielomianu to: $15, -1, 17$.\\
Wielomian jest stopnia parzystego, ponadto znak współczynnika przy\linebreak najwyższej potędze x jest ujemny.\\ W związku z tym wykres wielomianu zaczyna się od lewej strony powyżej osi OX.\\
Ponadto w punkcie $-1$ wykres odbija się od osi poziomej.\\
A więc $$x \in \{-1\} \cup [15,17].$$
\rozwStop
\odpStart
$x \in \{-1\} \cup [15,17]$
\odpStop
\testStart
A.$x \in \{-1\} \cup [15,17]$\\
B.$x \in \{1\} \cup (15,17)$\\
C.$x \in \{-1\} \cup (15,17]$\\
D.$x \in \{1\} \cup (15,17]$\\
E.$x \in \{-1\} \cup [15,17)$\\
F.$x \in \{1\} \cup [15,17)$\\
G.$x \in \{-1\} \cup (15,17)$\\
H.$x \in \{1\} \cup [15,17]$
\testStop
\kluczStart
A
\kluczStop



\zadStart{Zadanie z Wikieł Z 1.62 c) moja wersja nr 913}

Rozwiązać nierówności $(15-x)(x+1)^{2}(18-x)^{3}\le0$.
\zadStop
\rozwStart{Patryk Wirkus}{}
Miejsca zerowe naszego wielomianu to: $15, -1, 18$.\\
Wielomian jest stopnia parzystego, ponadto znak współczynnika przy\linebreak najwyższej potędze x jest ujemny.\\ W związku z tym wykres wielomianu zaczyna się od lewej strony powyżej osi OX.\\
Ponadto w punkcie $-1$ wykres odbija się od osi poziomej.\\
A więc $$x \in \{-1\} \cup [15,18].$$
\rozwStop
\odpStart
$x \in \{-1\} \cup [15,18]$
\odpStop
\testStart
A.$x \in \{-1\} \cup [15,18]$\\
B.$x \in \{1\} \cup (15,18)$\\
C.$x \in \{-1\} \cup (15,18]$\\
D.$x \in \{1\} \cup (15,18]$\\
E.$x \in \{-1\} \cup [15,18)$\\
F.$x \in \{1\} \cup [15,18)$\\
G.$x \in \{-1\} \cup (15,18)$\\
H.$x \in \{1\} \cup [15,18]$
\testStop
\kluczStart
A
\kluczStop



\zadStart{Zadanie z Wikieł Z 1.62 c) moja wersja nr 914}

Rozwiązać nierówności $(15-x)(x+1)^{2}(19-x)^{3}\le0$.
\zadStop
\rozwStart{Patryk Wirkus}{}
Miejsca zerowe naszego wielomianu to: $15, -1, 19$.\\
Wielomian jest stopnia parzystego, ponadto znak współczynnika przy\linebreak najwyższej potędze x jest ujemny.\\ W związku z tym wykres wielomianu zaczyna się od lewej strony powyżej osi OX.\\
Ponadto w punkcie $-1$ wykres odbija się od osi poziomej.\\
A więc $$x \in \{-1\} \cup [15,19].$$
\rozwStop
\odpStart
$x \in \{-1\} \cup [15,19]$
\odpStop
\testStart
A.$x \in \{-1\} \cup [15,19]$\\
B.$x \in \{1\} \cup (15,19)$\\
C.$x \in \{-1\} \cup (15,19]$\\
D.$x \in \{1\} \cup (15,19]$\\
E.$x \in \{-1\} \cup [15,19)$\\
F.$x \in \{1\} \cup [15,19)$\\
G.$x \in \{-1\} \cup (15,19)$\\
H.$x \in \{1\} \cup [15,19]$
\testStop
\kluczStart
A
\kluczStop



\zadStart{Zadanie z Wikieł Z 1.62 c) moja wersja nr 915}

Rozwiązać nierówności $(15-x)(x+1)^{2}(20-x)^{3}\le0$.
\zadStop
\rozwStart{Patryk Wirkus}{}
Miejsca zerowe naszego wielomianu to: $15, -1, 20$.\\
Wielomian jest stopnia parzystego, ponadto znak współczynnika przy\linebreak najwyższej potędze x jest ujemny.\\ W związku z tym wykres wielomianu zaczyna się od lewej strony powyżej osi OX.\\
Ponadto w punkcie $-1$ wykres odbija się od osi poziomej.\\
A więc $$x \in \{-1\} \cup [15,20].$$
\rozwStop
\odpStart
$x \in \{-1\} \cup [15,20]$
\odpStop
\testStart
A.$x \in \{-1\} \cup [15,20]$\\
B.$x \in \{1\} \cup (15,20)$\\
C.$x \in \{-1\} \cup (15,20]$\\
D.$x \in \{1\} \cup (15,20]$\\
E.$x \in \{-1\} \cup [15,20)$\\
F.$x \in \{1\} \cup [15,20)$\\
G.$x \in \{-1\} \cup (15,20)$\\
H.$x \in \{1\} \cup [15,20]$
\testStop
\kluczStart
A
\kluczStop



\zadStart{Zadanie z Wikieł Z 1.62 c) moja wersja nr 916}

Rozwiązać nierówności $(15-x)(x+2)^{2}(16-x)^{3}\le0$.
\zadStop
\rozwStart{Patryk Wirkus}{}
Miejsca zerowe naszego wielomianu to: $15, -2, 16$.\\
Wielomian jest stopnia parzystego, ponadto znak współczynnika przy\linebreak najwyższej potędze x jest ujemny.\\ W związku z tym wykres wielomianu zaczyna się od lewej strony powyżej osi OX.\\
Ponadto w punkcie $-2$ wykres odbija się od osi poziomej.\\
A więc $$x \in \{-2\} \cup [15,16].$$
\rozwStop
\odpStart
$x \in \{-2\} \cup [15,16]$
\odpStop
\testStart
A.$x \in \{-2\} \cup [15,16]$\\
B.$x \in \{2\} \cup (15,16)$\\
C.$x \in \{-2\} \cup (15,16]$\\
D.$x \in \{2\} \cup (15,16]$\\
E.$x \in \{-2\} \cup [15,16)$\\
F.$x \in \{2\} \cup [15,16)$\\
G.$x \in \{-2\} \cup (15,16)$\\
H.$x \in \{2\} \cup [15,16]$
\testStop
\kluczStart
A
\kluczStop



\zadStart{Zadanie z Wikieł Z 1.62 c) moja wersja nr 917}

Rozwiązać nierówności $(15-x)(x+2)^{2}(17-x)^{3}\le0$.
\zadStop
\rozwStart{Patryk Wirkus}{}
Miejsca zerowe naszego wielomianu to: $15, -2, 17$.\\
Wielomian jest stopnia parzystego, ponadto znak współczynnika przy\linebreak najwyższej potędze x jest ujemny.\\ W związku z tym wykres wielomianu zaczyna się od lewej strony powyżej osi OX.\\
Ponadto w punkcie $-2$ wykres odbija się od osi poziomej.\\
A więc $$x \in \{-2\} \cup [15,17].$$
\rozwStop
\odpStart
$x \in \{-2\} \cup [15,17]$
\odpStop
\testStart
A.$x \in \{-2\} \cup [15,17]$\\
B.$x \in \{2\} \cup (15,17)$\\
C.$x \in \{-2\} \cup (15,17]$\\
D.$x \in \{2\} \cup (15,17]$\\
E.$x \in \{-2\} \cup [15,17)$\\
F.$x \in \{2\} \cup [15,17)$\\
G.$x \in \{-2\} \cup (15,17)$\\
H.$x \in \{2\} \cup [15,17]$
\testStop
\kluczStart
A
\kluczStop



\zadStart{Zadanie z Wikieł Z 1.62 c) moja wersja nr 918}

Rozwiązać nierówności $(15-x)(x+2)^{2}(18-x)^{3}\le0$.
\zadStop
\rozwStart{Patryk Wirkus}{}
Miejsca zerowe naszego wielomianu to: $15, -2, 18$.\\
Wielomian jest stopnia parzystego, ponadto znak współczynnika przy\linebreak najwyższej potędze x jest ujemny.\\ W związku z tym wykres wielomianu zaczyna się od lewej strony powyżej osi OX.\\
Ponadto w punkcie $-2$ wykres odbija się od osi poziomej.\\
A więc $$x \in \{-2\} \cup [15,18].$$
\rozwStop
\odpStart
$x \in \{-2\} \cup [15,18]$
\odpStop
\testStart
A.$x \in \{-2\} \cup [15,18]$\\
B.$x \in \{2\} \cup (15,18)$\\
C.$x \in \{-2\} \cup (15,18]$\\
D.$x \in \{2\} \cup (15,18]$\\
E.$x \in \{-2\} \cup [15,18)$\\
F.$x \in \{2\} \cup [15,18)$\\
G.$x \in \{-2\} \cup (15,18)$\\
H.$x \in \{2\} \cup [15,18]$
\testStop
\kluczStart
A
\kluczStop



\zadStart{Zadanie z Wikieł Z 1.62 c) moja wersja nr 919}

Rozwiązać nierówności $(15-x)(x+2)^{2}(19-x)^{3}\le0$.
\zadStop
\rozwStart{Patryk Wirkus}{}
Miejsca zerowe naszego wielomianu to: $15, -2, 19$.\\
Wielomian jest stopnia parzystego, ponadto znak współczynnika przy\linebreak najwyższej potędze x jest ujemny.\\ W związku z tym wykres wielomianu zaczyna się od lewej strony powyżej osi OX.\\
Ponadto w punkcie $-2$ wykres odbija się od osi poziomej.\\
A więc $$x \in \{-2\} \cup [15,19].$$
\rozwStop
\odpStart
$x \in \{-2\} \cup [15,19]$
\odpStop
\testStart
A.$x \in \{-2\} \cup [15,19]$\\
B.$x \in \{2\} \cup (15,19)$\\
C.$x \in \{-2\} \cup (15,19]$\\
D.$x \in \{2\} \cup (15,19]$\\
E.$x \in \{-2\} \cup [15,19)$\\
F.$x \in \{2\} \cup [15,19)$\\
G.$x \in \{-2\} \cup (15,19)$\\
H.$x \in \{2\} \cup [15,19]$
\testStop
\kluczStart
A
\kluczStop



\zadStart{Zadanie z Wikieł Z 1.62 c) moja wersja nr 920}

Rozwiązać nierówności $(15-x)(x+2)^{2}(20-x)^{3}\le0$.
\zadStop
\rozwStart{Patryk Wirkus}{}
Miejsca zerowe naszego wielomianu to: $15, -2, 20$.\\
Wielomian jest stopnia parzystego, ponadto znak współczynnika przy\linebreak najwyższej potędze x jest ujemny.\\ W związku z tym wykres wielomianu zaczyna się od lewej strony powyżej osi OX.\\
Ponadto w punkcie $-2$ wykres odbija się od osi poziomej.\\
A więc $$x \in \{-2\} \cup [15,20].$$
\rozwStop
\odpStart
$x \in \{-2\} \cup [15,20]$
\odpStop
\testStart
A.$x \in \{-2\} \cup [15,20]$\\
B.$x \in \{2\} \cup (15,20)$\\
C.$x \in \{-2\} \cup (15,20]$\\
D.$x \in \{2\} \cup (15,20]$\\
E.$x \in \{-2\} \cup [15,20)$\\
F.$x \in \{2\} \cup [15,20)$\\
G.$x \in \{-2\} \cup (15,20)$\\
H.$x \in \{2\} \cup [15,20]$
\testStop
\kluczStart
A
\kluczStop



\zadStart{Zadanie z Wikieł Z 1.62 c) moja wersja nr 921}

Rozwiązać nierówności $(15-x)(x+3)^{2}(16-x)^{3}\le0$.
\zadStop
\rozwStart{Patryk Wirkus}{}
Miejsca zerowe naszego wielomianu to: $15, -3, 16$.\\
Wielomian jest stopnia parzystego, ponadto znak współczynnika przy\linebreak najwyższej potędze x jest ujemny.\\ W związku z tym wykres wielomianu zaczyna się od lewej strony powyżej osi OX.\\
Ponadto w punkcie $-3$ wykres odbija się od osi poziomej.\\
A więc $$x \in \{-3\} \cup [15,16].$$
\rozwStop
\odpStart
$x \in \{-3\} \cup [15,16]$
\odpStop
\testStart
A.$x \in \{-3\} \cup [15,16]$\\
B.$x \in \{3\} \cup (15,16)$\\
C.$x \in \{-3\} \cup (15,16]$\\
D.$x \in \{3\} \cup (15,16]$\\
E.$x \in \{-3\} \cup [15,16)$\\
F.$x \in \{3\} \cup [15,16)$\\
G.$x \in \{-3\} \cup (15,16)$\\
H.$x \in \{3\} \cup [15,16]$
\testStop
\kluczStart
A
\kluczStop



\zadStart{Zadanie z Wikieł Z 1.62 c) moja wersja nr 922}

Rozwiązać nierówności $(15-x)(x+3)^{2}(17-x)^{3}\le0$.
\zadStop
\rozwStart{Patryk Wirkus}{}
Miejsca zerowe naszego wielomianu to: $15, -3, 17$.\\
Wielomian jest stopnia parzystego, ponadto znak współczynnika przy\linebreak najwyższej potędze x jest ujemny.\\ W związku z tym wykres wielomianu zaczyna się od lewej strony powyżej osi OX.\\
Ponadto w punkcie $-3$ wykres odbija się od osi poziomej.\\
A więc $$x \in \{-3\} \cup [15,17].$$
\rozwStop
\odpStart
$x \in \{-3\} \cup [15,17]$
\odpStop
\testStart
A.$x \in \{-3\} \cup [15,17]$\\
B.$x \in \{3\} \cup (15,17)$\\
C.$x \in \{-3\} \cup (15,17]$\\
D.$x \in \{3\} \cup (15,17]$\\
E.$x \in \{-3\} \cup [15,17)$\\
F.$x \in \{3\} \cup [15,17)$\\
G.$x \in \{-3\} \cup (15,17)$\\
H.$x \in \{3\} \cup [15,17]$
\testStop
\kluczStart
A
\kluczStop



\zadStart{Zadanie z Wikieł Z 1.62 c) moja wersja nr 923}

Rozwiązać nierówności $(15-x)(x+3)^{2}(18-x)^{3}\le0$.
\zadStop
\rozwStart{Patryk Wirkus}{}
Miejsca zerowe naszego wielomianu to: $15, -3, 18$.\\
Wielomian jest stopnia parzystego, ponadto znak współczynnika przy\linebreak najwyższej potędze x jest ujemny.\\ W związku z tym wykres wielomianu zaczyna się od lewej strony powyżej osi OX.\\
Ponadto w punkcie $-3$ wykres odbija się od osi poziomej.\\
A więc $$x \in \{-3\} \cup [15,18].$$
\rozwStop
\odpStart
$x \in \{-3\} \cup [15,18]$
\odpStop
\testStart
A.$x \in \{-3\} \cup [15,18]$\\
B.$x \in \{3\} \cup (15,18)$\\
C.$x \in \{-3\} \cup (15,18]$\\
D.$x \in \{3\} \cup (15,18]$\\
E.$x \in \{-3\} \cup [15,18)$\\
F.$x \in \{3\} \cup [15,18)$\\
G.$x \in \{-3\} \cup (15,18)$\\
H.$x \in \{3\} \cup [15,18]$
\testStop
\kluczStart
A
\kluczStop



\zadStart{Zadanie z Wikieł Z 1.62 c) moja wersja nr 924}

Rozwiązać nierówności $(15-x)(x+3)^{2}(19-x)^{3}\le0$.
\zadStop
\rozwStart{Patryk Wirkus}{}
Miejsca zerowe naszego wielomianu to: $15, -3, 19$.\\
Wielomian jest stopnia parzystego, ponadto znak współczynnika przy\linebreak najwyższej potędze x jest ujemny.\\ W związku z tym wykres wielomianu zaczyna się od lewej strony powyżej osi OX.\\
Ponadto w punkcie $-3$ wykres odbija się od osi poziomej.\\
A więc $$x \in \{-3\} \cup [15,19].$$
\rozwStop
\odpStart
$x \in \{-3\} \cup [15,19]$
\odpStop
\testStart
A.$x \in \{-3\} \cup [15,19]$\\
B.$x \in \{3\} \cup (15,19)$\\
C.$x \in \{-3\} \cup (15,19]$\\
D.$x \in \{3\} \cup (15,19]$\\
E.$x \in \{-3\} \cup [15,19)$\\
F.$x \in \{3\} \cup [15,19)$\\
G.$x \in \{-3\} \cup (15,19)$\\
H.$x \in \{3\} \cup [15,19]$
\testStop
\kluczStart
A
\kluczStop



\zadStart{Zadanie z Wikieł Z 1.62 c) moja wersja nr 925}

Rozwiązać nierówności $(15-x)(x+3)^{2}(20-x)^{3}\le0$.
\zadStop
\rozwStart{Patryk Wirkus}{}
Miejsca zerowe naszego wielomianu to: $15, -3, 20$.\\
Wielomian jest stopnia parzystego, ponadto znak współczynnika przy\linebreak najwyższej potędze x jest ujemny.\\ W związku z tym wykres wielomianu zaczyna się od lewej strony powyżej osi OX.\\
Ponadto w punkcie $-3$ wykres odbija się od osi poziomej.\\
A więc $$x \in \{-3\} \cup [15,20].$$
\rozwStop
\odpStart
$x \in \{-3\} \cup [15,20]$
\odpStop
\testStart
A.$x \in \{-3\} \cup [15,20]$\\
B.$x \in \{3\} \cup (15,20)$\\
C.$x \in \{-3\} \cup (15,20]$\\
D.$x \in \{3\} \cup (15,20]$\\
E.$x \in \{-3\} \cup [15,20)$\\
F.$x \in \{3\} \cup [15,20)$\\
G.$x \in \{-3\} \cup (15,20)$\\
H.$x \in \{3\} \cup [15,20]$
\testStop
\kluczStart
A
\kluczStop



\zadStart{Zadanie z Wikieł Z 1.62 c) moja wersja nr 926}

Rozwiązać nierówności $(15-x)(x+4)^{2}(16-x)^{3}\le0$.
\zadStop
\rozwStart{Patryk Wirkus}{}
Miejsca zerowe naszego wielomianu to: $15, -4, 16$.\\
Wielomian jest stopnia parzystego, ponadto znak współczynnika przy\linebreak najwyższej potędze x jest ujemny.\\ W związku z tym wykres wielomianu zaczyna się od lewej strony powyżej osi OX.\\
Ponadto w punkcie $-4$ wykres odbija się od osi poziomej.\\
A więc $$x \in \{-4\} \cup [15,16].$$
\rozwStop
\odpStart
$x \in \{-4\} \cup [15,16]$
\odpStop
\testStart
A.$x \in \{-4\} \cup [15,16]$\\
B.$x \in \{4\} \cup (15,16)$\\
C.$x \in \{-4\} \cup (15,16]$\\
D.$x \in \{4\} \cup (15,16]$\\
E.$x \in \{-4\} \cup [15,16)$\\
F.$x \in \{4\} \cup [15,16)$\\
G.$x \in \{-4\} \cup (15,16)$\\
H.$x \in \{4\} \cup [15,16]$
\testStop
\kluczStart
A
\kluczStop



\zadStart{Zadanie z Wikieł Z 1.62 c) moja wersja nr 927}

Rozwiązać nierówności $(15-x)(x+4)^{2}(17-x)^{3}\le0$.
\zadStop
\rozwStart{Patryk Wirkus}{}
Miejsca zerowe naszego wielomianu to: $15, -4, 17$.\\
Wielomian jest stopnia parzystego, ponadto znak współczynnika przy\linebreak najwyższej potędze x jest ujemny.\\ W związku z tym wykres wielomianu zaczyna się od lewej strony powyżej osi OX.\\
Ponadto w punkcie $-4$ wykres odbija się od osi poziomej.\\
A więc $$x \in \{-4\} \cup [15,17].$$
\rozwStop
\odpStart
$x \in \{-4\} \cup [15,17]$
\odpStop
\testStart
A.$x \in \{-4\} \cup [15,17]$\\
B.$x \in \{4\} \cup (15,17)$\\
C.$x \in \{-4\} \cup (15,17]$\\
D.$x \in \{4\} \cup (15,17]$\\
E.$x \in \{-4\} \cup [15,17)$\\
F.$x \in \{4\} \cup [15,17)$\\
G.$x \in \{-4\} \cup (15,17)$\\
H.$x \in \{4\} \cup [15,17]$
\testStop
\kluczStart
A
\kluczStop



\zadStart{Zadanie z Wikieł Z 1.62 c) moja wersja nr 928}

Rozwiązać nierówności $(15-x)(x+4)^{2}(18-x)^{3}\le0$.
\zadStop
\rozwStart{Patryk Wirkus}{}
Miejsca zerowe naszego wielomianu to: $15, -4, 18$.\\
Wielomian jest stopnia parzystego, ponadto znak współczynnika przy\linebreak najwyższej potędze x jest ujemny.\\ W związku z tym wykres wielomianu zaczyna się od lewej strony powyżej osi OX.\\
Ponadto w punkcie $-4$ wykres odbija się od osi poziomej.\\
A więc $$x \in \{-4\} \cup [15,18].$$
\rozwStop
\odpStart
$x \in \{-4\} \cup [15,18]$
\odpStop
\testStart
A.$x \in \{-4\} \cup [15,18]$\\
B.$x \in \{4\} \cup (15,18)$\\
C.$x \in \{-4\} \cup (15,18]$\\
D.$x \in \{4\} \cup (15,18]$\\
E.$x \in \{-4\} \cup [15,18)$\\
F.$x \in \{4\} \cup [15,18)$\\
G.$x \in \{-4\} \cup (15,18)$\\
H.$x \in \{4\} \cup [15,18]$
\testStop
\kluczStart
A
\kluczStop



\zadStart{Zadanie z Wikieł Z 1.62 c) moja wersja nr 929}

Rozwiązać nierówności $(15-x)(x+4)^{2}(19-x)^{3}\le0$.
\zadStop
\rozwStart{Patryk Wirkus}{}
Miejsca zerowe naszego wielomianu to: $15, -4, 19$.\\
Wielomian jest stopnia parzystego, ponadto znak współczynnika przy\linebreak najwyższej potędze x jest ujemny.\\ W związku z tym wykres wielomianu zaczyna się od lewej strony powyżej osi OX.\\
Ponadto w punkcie $-4$ wykres odbija się od osi poziomej.\\
A więc $$x \in \{-4\} \cup [15,19].$$
\rozwStop
\odpStart
$x \in \{-4\} \cup [15,19]$
\odpStop
\testStart
A.$x \in \{-4\} \cup [15,19]$\\
B.$x \in \{4\} \cup (15,19)$\\
C.$x \in \{-4\} \cup (15,19]$\\
D.$x \in \{4\} \cup (15,19]$\\
E.$x \in \{-4\} \cup [15,19)$\\
F.$x \in \{4\} \cup [15,19)$\\
G.$x \in \{-4\} \cup (15,19)$\\
H.$x \in \{4\} \cup [15,19]$
\testStop
\kluczStart
A
\kluczStop



\zadStart{Zadanie z Wikieł Z 1.62 c) moja wersja nr 930}

Rozwiązać nierówności $(15-x)(x+4)^{2}(20-x)^{3}\le0$.
\zadStop
\rozwStart{Patryk Wirkus}{}
Miejsca zerowe naszego wielomianu to: $15, -4, 20$.\\
Wielomian jest stopnia parzystego, ponadto znak współczynnika przy\linebreak najwyższej potędze x jest ujemny.\\ W związku z tym wykres wielomianu zaczyna się od lewej strony powyżej osi OX.\\
Ponadto w punkcie $-4$ wykres odbija się od osi poziomej.\\
A więc $$x \in \{-4\} \cup [15,20].$$
\rozwStop
\odpStart
$x \in \{-4\} \cup [15,20]$
\odpStop
\testStart
A.$x \in \{-4\} \cup [15,20]$\\
B.$x \in \{4\} \cup (15,20)$\\
C.$x \in \{-4\} \cup (15,20]$\\
D.$x \in \{4\} \cup (15,20]$\\
E.$x \in \{-4\} \cup [15,20)$\\
F.$x \in \{4\} \cup [15,20)$\\
G.$x \in \{-4\} \cup (15,20)$\\
H.$x \in \{4\} \cup [15,20]$
\testStop
\kluczStart
A
\kluczStop



\zadStart{Zadanie z Wikieł Z 1.62 c) moja wersja nr 931}

Rozwiązać nierówności $(15-x)(x+5)^{2}(16-x)^{3}\le0$.
\zadStop
\rozwStart{Patryk Wirkus}{}
Miejsca zerowe naszego wielomianu to: $15, -5, 16$.\\
Wielomian jest stopnia parzystego, ponadto znak współczynnika przy\linebreak najwyższej potędze x jest ujemny.\\ W związku z tym wykres wielomianu zaczyna się od lewej strony powyżej osi OX.\\
Ponadto w punkcie $-5$ wykres odbija się od osi poziomej.\\
A więc $$x \in \{-5\} \cup [15,16].$$
\rozwStop
\odpStart
$x \in \{-5\} \cup [15,16]$
\odpStop
\testStart
A.$x \in \{-5\} \cup [15,16]$\\
B.$x \in \{5\} \cup (15,16)$\\
C.$x \in \{-5\} \cup (15,16]$\\
D.$x \in \{5\} \cup (15,16]$\\
E.$x \in \{-5\} \cup [15,16)$\\
F.$x \in \{5\} \cup [15,16)$\\
G.$x \in \{-5\} \cup (15,16)$\\
H.$x \in \{5\} \cup [15,16]$
\testStop
\kluczStart
A
\kluczStop



\zadStart{Zadanie z Wikieł Z 1.62 c) moja wersja nr 932}

Rozwiązać nierówności $(15-x)(x+5)^{2}(17-x)^{3}\le0$.
\zadStop
\rozwStart{Patryk Wirkus}{}
Miejsca zerowe naszego wielomianu to: $15, -5, 17$.\\
Wielomian jest stopnia parzystego, ponadto znak współczynnika przy\linebreak najwyższej potędze x jest ujemny.\\ W związku z tym wykres wielomianu zaczyna się od lewej strony powyżej osi OX.\\
Ponadto w punkcie $-5$ wykres odbija się od osi poziomej.\\
A więc $$x \in \{-5\} \cup [15,17].$$
\rozwStop
\odpStart
$x \in \{-5\} \cup [15,17]$
\odpStop
\testStart
A.$x \in \{-5\} \cup [15,17]$\\
B.$x \in \{5\} \cup (15,17)$\\
C.$x \in \{-5\} \cup (15,17]$\\
D.$x \in \{5\} \cup (15,17]$\\
E.$x \in \{-5\} \cup [15,17)$\\
F.$x \in \{5\} \cup [15,17)$\\
G.$x \in \{-5\} \cup (15,17)$\\
H.$x \in \{5\} \cup [15,17]$
\testStop
\kluczStart
A
\kluczStop



\zadStart{Zadanie z Wikieł Z 1.62 c) moja wersja nr 933}

Rozwiązać nierówności $(15-x)(x+5)^{2}(18-x)^{3}\le0$.
\zadStop
\rozwStart{Patryk Wirkus}{}
Miejsca zerowe naszego wielomianu to: $15, -5, 18$.\\
Wielomian jest stopnia parzystego, ponadto znak współczynnika przy\linebreak najwyższej potędze x jest ujemny.\\ W związku z tym wykres wielomianu zaczyna się od lewej strony powyżej osi OX.\\
Ponadto w punkcie $-5$ wykres odbija się od osi poziomej.\\
A więc $$x \in \{-5\} \cup [15,18].$$
\rozwStop
\odpStart
$x \in \{-5\} \cup [15,18]$
\odpStop
\testStart
A.$x \in \{-5\} \cup [15,18]$\\
B.$x \in \{5\} \cup (15,18)$\\
C.$x \in \{-5\} \cup (15,18]$\\
D.$x \in \{5\} \cup (15,18]$\\
E.$x \in \{-5\} \cup [15,18)$\\
F.$x \in \{5\} \cup [15,18)$\\
G.$x \in \{-5\} \cup (15,18)$\\
H.$x \in \{5\} \cup [15,18]$
\testStop
\kluczStart
A
\kluczStop



\zadStart{Zadanie z Wikieł Z 1.62 c) moja wersja nr 934}

Rozwiązać nierówności $(15-x)(x+5)^{2}(19-x)^{3}\le0$.
\zadStop
\rozwStart{Patryk Wirkus}{}
Miejsca zerowe naszego wielomianu to: $15, -5, 19$.\\
Wielomian jest stopnia parzystego, ponadto znak współczynnika przy\linebreak najwyższej potędze x jest ujemny.\\ W związku z tym wykres wielomianu zaczyna się od lewej strony powyżej osi OX.\\
Ponadto w punkcie $-5$ wykres odbija się od osi poziomej.\\
A więc $$x \in \{-5\} \cup [15,19].$$
\rozwStop
\odpStart
$x \in \{-5\} \cup [15,19]$
\odpStop
\testStart
A.$x \in \{-5\} \cup [15,19]$\\
B.$x \in \{5\} \cup (15,19)$\\
C.$x \in \{-5\} \cup (15,19]$\\
D.$x \in \{5\} \cup (15,19]$\\
E.$x \in \{-5\} \cup [15,19)$\\
F.$x \in \{5\} \cup [15,19)$\\
G.$x \in \{-5\} \cup (15,19)$\\
H.$x \in \{5\} \cup [15,19]$
\testStop
\kluczStart
A
\kluczStop



\zadStart{Zadanie z Wikieł Z 1.62 c) moja wersja nr 935}

Rozwiązać nierówności $(15-x)(x+5)^{2}(20-x)^{3}\le0$.
\zadStop
\rozwStart{Patryk Wirkus}{}
Miejsca zerowe naszego wielomianu to: $15, -5, 20$.\\
Wielomian jest stopnia parzystego, ponadto znak współczynnika przy\linebreak najwyższej potędze x jest ujemny.\\ W związku z tym wykres wielomianu zaczyna się od lewej strony powyżej osi OX.\\
Ponadto w punkcie $-5$ wykres odbija się od osi poziomej.\\
A więc $$x \in \{-5\} \cup [15,20].$$
\rozwStop
\odpStart
$x \in \{-5\} \cup [15,20]$
\odpStop
\testStart
A.$x \in \{-5\} \cup [15,20]$\\
B.$x \in \{5\} \cup (15,20)$\\
C.$x \in \{-5\} \cup (15,20]$\\
D.$x \in \{5\} \cup (15,20]$\\
E.$x \in \{-5\} \cup [15,20)$\\
F.$x \in \{5\} \cup [15,20)$\\
G.$x \in \{-5\} \cup (15,20)$\\
H.$x \in \{5\} \cup [15,20]$
\testStop
\kluczStart
A
\kluczStop



\zadStart{Zadanie z Wikieł Z 1.62 c) moja wersja nr 936}

Rozwiązać nierówności $(15-x)(x+6)^{2}(16-x)^{3}\le0$.
\zadStop
\rozwStart{Patryk Wirkus}{}
Miejsca zerowe naszego wielomianu to: $15, -6, 16$.\\
Wielomian jest stopnia parzystego, ponadto znak współczynnika przy\linebreak najwyższej potędze x jest ujemny.\\ W związku z tym wykres wielomianu zaczyna się od lewej strony powyżej osi OX.\\
Ponadto w punkcie $-6$ wykres odbija się od osi poziomej.\\
A więc $$x \in \{-6\} \cup [15,16].$$
\rozwStop
\odpStart
$x \in \{-6\} \cup [15,16]$
\odpStop
\testStart
A.$x \in \{-6\} \cup [15,16]$\\
B.$x \in \{6\} \cup (15,16)$\\
C.$x \in \{-6\} \cup (15,16]$\\
D.$x \in \{6\} \cup (15,16]$\\
E.$x \in \{-6\} \cup [15,16)$\\
F.$x \in \{6\} \cup [15,16)$\\
G.$x \in \{-6\} \cup (15,16)$\\
H.$x \in \{6\} \cup [15,16]$
\testStop
\kluczStart
A
\kluczStop



\zadStart{Zadanie z Wikieł Z 1.62 c) moja wersja nr 937}

Rozwiązać nierówności $(15-x)(x+6)^{2}(17-x)^{3}\le0$.
\zadStop
\rozwStart{Patryk Wirkus}{}
Miejsca zerowe naszego wielomianu to: $15, -6, 17$.\\
Wielomian jest stopnia parzystego, ponadto znak współczynnika przy\linebreak najwyższej potędze x jest ujemny.\\ W związku z tym wykres wielomianu zaczyna się od lewej strony powyżej osi OX.\\
Ponadto w punkcie $-6$ wykres odbija się od osi poziomej.\\
A więc $$x \in \{-6\} \cup [15,17].$$
\rozwStop
\odpStart
$x \in \{-6\} \cup [15,17]$
\odpStop
\testStart
A.$x \in \{-6\} \cup [15,17]$\\
B.$x \in \{6\} \cup (15,17)$\\
C.$x \in \{-6\} \cup (15,17]$\\
D.$x \in \{6\} \cup (15,17]$\\
E.$x \in \{-6\} \cup [15,17)$\\
F.$x \in \{6\} \cup [15,17)$\\
G.$x \in \{-6\} \cup (15,17)$\\
H.$x \in \{6\} \cup [15,17]$
\testStop
\kluczStart
A
\kluczStop



\zadStart{Zadanie z Wikieł Z 1.62 c) moja wersja nr 938}

Rozwiązać nierówności $(15-x)(x+6)^{2}(18-x)^{3}\le0$.
\zadStop
\rozwStart{Patryk Wirkus}{}
Miejsca zerowe naszego wielomianu to: $15, -6, 18$.\\
Wielomian jest stopnia parzystego, ponadto znak współczynnika przy\linebreak najwyższej potędze x jest ujemny.\\ W związku z tym wykres wielomianu zaczyna się od lewej strony powyżej osi OX.\\
Ponadto w punkcie $-6$ wykres odbija się od osi poziomej.\\
A więc $$x \in \{-6\} \cup [15,18].$$
\rozwStop
\odpStart
$x \in \{-6\} \cup [15,18]$
\odpStop
\testStart
A.$x \in \{-6\} \cup [15,18]$\\
B.$x \in \{6\} \cup (15,18)$\\
C.$x \in \{-6\} \cup (15,18]$\\
D.$x \in \{6\} \cup (15,18]$\\
E.$x \in \{-6\} \cup [15,18)$\\
F.$x \in \{6\} \cup [15,18)$\\
G.$x \in \{-6\} \cup (15,18)$\\
H.$x \in \{6\} \cup [15,18]$
\testStop
\kluczStart
A
\kluczStop



\zadStart{Zadanie z Wikieł Z 1.62 c) moja wersja nr 939}

Rozwiązać nierówności $(15-x)(x+6)^{2}(19-x)^{3}\le0$.
\zadStop
\rozwStart{Patryk Wirkus}{}
Miejsca zerowe naszego wielomianu to: $15, -6, 19$.\\
Wielomian jest stopnia parzystego, ponadto znak współczynnika przy\linebreak najwyższej potędze x jest ujemny.\\ W związku z tym wykres wielomianu zaczyna się od lewej strony powyżej osi OX.\\
Ponadto w punkcie $-6$ wykres odbija się od osi poziomej.\\
A więc $$x \in \{-6\} \cup [15,19].$$
\rozwStop
\odpStart
$x \in \{-6\} \cup [15,19]$
\odpStop
\testStart
A.$x \in \{-6\} \cup [15,19]$\\
B.$x \in \{6\} \cup (15,19)$\\
C.$x \in \{-6\} \cup (15,19]$\\
D.$x \in \{6\} \cup (15,19]$\\
E.$x \in \{-6\} \cup [15,19)$\\
F.$x \in \{6\} \cup [15,19)$\\
G.$x \in \{-6\} \cup (15,19)$\\
H.$x \in \{6\} \cup [15,19]$
\testStop
\kluczStart
A
\kluczStop



\zadStart{Zadanie z Wikieł Z 1.62 c) moja wersja nr 940}

Rozwiązać nierówności $(15-x)(x+6)^{2}(20-x)^{3}\le0$.
\zadStop
\rozwStart{Patryk Wirkus}{}
Miejsca zerowe naszego wielomianu to: $15, -6, 20$.\\
Wielomian jest stopnia parzystego, ponadto znak współczynnika przy\linebreak najwyższej potędze x jest ujemny.\\ W związku z tym wykres wielomianu zaczyna się od lewej strony powyżej osi OX.\\
Ponadto w punkcie $-6$ wykres odbija się od osi poziomej.\\
A więc $$x \in \{-6\} \cup [15,20].$$
\rozwStop
\odpStart
$x \in \{-6\} \cup [15,20]$
\odpStop
\testStart
A.$x \in \{-6\} \cup [15,20]$\\
B.$x \in \{6\} \cup (15,20)$\\
C.$x \in \{-6\} \cup (15,20]$\\
D.$x \in \{6\} \cup (15,20]$\\
E.$x \in \{-6\} \cup [15,20)$\\
F.$x \in \{6\} \cup [15,20)$\\
G.$x \in \{-6\} \cup (15,20)$\\
H.$x \in \{6\} \cup [15,20]$
\testStop
\kluczStart
A
\kluczStop



\zadStart{Zadanie z Wikieł Z 1.62 c) moja wersja nr 941}

Rozwiązać nierówności $(15-x)(x+7)^{2}(16-x)^{3}\le0$.
\zadStop
\rozwStart{Patryk Wirkus}{}
Miejsca zerowe naszego wielomianu to: $15, -7, 16$.\\
Wielomian jest stopnia parzystego, ponadto znak współczynnika przy\linebreak najwyższej potędze x jest ujemny.\\ W związku z tym wykres wielomianu zaczyna się od lewej strony powyżej osi OX.\\
Ponadto w punkcie $-7$ wykres odbija się od osi poziomej.\\
A więc $$x \in \{-7\} \cup [15,16].$$
\rozwStop
\odpStart
$x \in \{-7\} \cup [15,16]$
\odpStop
\testStart
A.$x \in \{-7\} \cup [15,16]$\\
B.$x \in \{7\} \cup (15,16)$\\
C.$x \in \{-7\} \cup (15,16]$\\
D.$x \in \{7\} \cup (15,16]$\\
E.$x \in \{-7\} \cup [15,16)$\\
F.$x \in \{7\} \cup [15,16)$\\
G.$x \in \{-7\} \cup (15,16)$\\
H.$x \in \{7\} \cup [15,16]$
\testStop
\kluczStart
A
\kluczStop



\zadStart{Zadanie z Wikieł Z 1.62 c) moja wersja nr 942}

Rozwiązać nierówności $(15-x)(x+7)^{2}(17-x)^{3}\le0$.
\zadStop
\rozwStart{Patryk Wirkus}{}
Miejsca zerowe naszego wielomianu to: $15, -7, 17$.\\
Wielomian jest stopnia parzystego, ponadto znak współczynnika przy\linebreak najwyższej potędze x jest ujemny.\\ W związku z tym wykres wielomianu zaczyna się od lewej strony powyżej osi OX.\\
Ponadto w punkcie $-7$ wykres odbija się od osi poziomej.\\
A więc $$x \in \{-7\} \cup [15,17].$$
\rozwStop
\odpStart
$x \in \{-7\} \cup [15,17]$
\odpStop
\testStart
A.$x \in \{-7\} \cup [15,17]$\\
B.$x \in \{7\} \cup (15,17)$\\
C.$x \in \{-7\} \cup (15,17]$\\
D.$x \in \{7\} \cup (15,17]$\\
E.$x \in \{-7\} \cup [15,17)$\\
F.$x \in \{7\} \cup [15,17)$\\
G.$x \in \{-7\} \cup (15,17)$\\
H.$x \in \{7\} \cup [15,17]$
\testStop
\kluczStart
A
\kluczStop



\zadStart{Zadanie z Wikieł Z 1.62 c) moja wersja nr 943}

Rozwiązać nierówności $(15-x)(x+7)^{2}(18-x)^{3}\le0$.
\zadStop
\rozwStart{Patryk Wirkus}{}
Miejsca zerowe naszego wielomianu to: $15, -7, 18$.\\
Wielomian jest stopnia parzystego, ponadto znak współczynnika przy\linebreak najwyższej potędze x jest ujemny.\\ W związku z tym wykres wielomianu zaczyna się od lewej strony powyżej osi OX.\\
Ponadto w punkcie $-7$ wykres odbija się od osi poziomej.\\
A więc $$x \in \{-7\} \cup [15,18].$$
\rozwStop
\odpStart
$x \in \{-7\} \cup [15,18]$
\odpStop
\testStart
A.$x \in \{-7\} \cup [15,18]$\\
B.$x \in \{7\} \cup (15,18)$\\
C.$x \in \{-7\} \cup (15,18]$\\
D.$x \in \{7\} \cup (15,18]$\\
E.$x \in \{-7\} \cup [15,18)$\\
F.$x \in \{7\} \cup [15,18)$\\
G.$x \in \{-7\} \cup (15,18)$\\
H.$x \in \{7\} \cup [15,18]$
\testStop
\kluczStart
A
\kluczStop



\zadStart{Zadanie z Wikieł Z 1.62 c) moja wersja nr 944}

Rozwiązać nierówności $(15-x)(x+7)^{2}(19-x)^{3}\le0$.
\zadStop
\rozwStart{Patryk Wirkus}{}
Miejsca zerowe naszego wielomianu to: $15, -7, 19$.\\
Wielomian jest stopnia parzystego, ponadto znak współczynnika przy\linebreak najwyższej potędze x jest ujemny.\\ W związku z tym wykres wielomianu zaczyna się od lewej strony powyżej osi OX.\\
Ponadto w punkcie $-7$ wykres odbija się od osi poziomej.\\
A więc $$x \in \{-7\} \cup [15,19].$$
\rozwStop
\odpStart
$x \in \{-7\} \cup [15,19]$
\odpStop
\testStart
A.$x \in \{-7\} \cup [15,19]$\\
B.$x \in \{7\} \cup (15,19)$\\
C.$x \in \{-7\} \cup (15,19]$\\
D.$x \in \{7\} \cup (15,19]$\\
E.$x \in \{-7\} \cup [15,19)$\\
F.$x \in \{7\} \cup [15,19)$\\
G.$x \in \{-7\} \cup (15,19)$\\
H.$x \in \{7\} \cup [15,19]$
\testStop
\kluczStart
A
\kluczStop



\zadStart{Zadanie z Wikieł Z 1.62 c) moja wersja nr 945}

Rozwiązać nierówności $(15-x)(x+7)^{2}(20-x)^{3}\le0$.
\zadStop
\rozwStart{Patryk Wirkus}{}
Miejsca zerowe naszego wielomianu to: $15, -7, 20$.\\
Wielomian jest stopnia parzystego, ponadto znak współczynnika przy\linebreak najwyższej potędze x jest ujemny.\\ W związku z tym wykres wielomianu zaczyna się od lewej strony powyżej osi OX.\\
Ponadto w punkcie $-7$ wykres odbija się od osi poziomej.\\
A więc $$x \in \{-7\} \cup [15,20].$$
\rozwStop
\odpStart
$x \in \{-7\} \cup [15,20]$
\odpStop
\testStart
A.$x \in \{-7\} \cup [15,20]$\\
B.$x \in \{7\} \cup (15,20)$\\
C.$x \in \{-7\} \cup (15,20]$\\
D.$x \in \{7\} \cup (15,20]$\\
E.$x \in \{-7\} \cup [15,20)$\\
F.$x \in \{7\} \cup [15,20)$\\
G.$x \in \{-7\} \cup (15,20)$\\
H.$x \in \{7\} \cup [15,20]$
\testStop
\kluczStart
A
\kluczStop



\zadStart{Zadanie z Wikieł Z 1.62 c) moja wersja nr 946}

Rozwiązać nierówności $(15-x)(x+8)^{2}(16-x)^{3}\le0$.
\zadStop
\rozwStart{Patryk Wirkus}{}
Miejsca zerowe naszego wielomianu to: $15, -8, 16$.\\
Wielomian jest stopnia parzystego, ponadto znak współczynnika przy\linebreak najwyższej potędze x jest ujemny.\\ W związku z tym wykres wielomianu zaczyna się od lewej strony powyżej osi OX.\\
Ponadto w punkcie $-8$ wykres odbija się od osi poziomej.\\
A więc $$x \in \{-8\} \cup [15,16].$$
\rozwStop
\odpStart
$x \in \{-8\} \cup [15,16]$
\odpStop
\testStart
A.$x \in \{-8\} \cup [15,16]$\\
B.$x \in \{8\} \cup (15,16)$\\
C.$x \in \{-8\} \cup (15,16]$\\
D.$x \in \{8\} \cup (15,16]$\\
E.$x \in \{-8\} \cup [15,16)$\\
F.$x \in \{8\} \cup [15,16)$\\
G.$x \in \{-8\} \cup (15,16)$\\
H.$x \in \{8\} \cup [15,16]$
\testStop
\kluczStart
A
\kluczStop



\zadStart{Zadanie z Wikieł Z 1.62 c) moja wersja nr 947}

Rozwiązać nierówności $(15-x)(x+8)^{2}(17-x)^{3}\le0$.
\zadStop
\rozwStart{Patryk Wirkus}{}
Miejsca zerowe naszego wielomianu to: $15, -8, 17$.\\
Wielomian jest stopnia parzystego, ponadto znak współczynnika przy\linebreak najwyższej potędze x jest ujemny.\\ W związku z tym wykres wielomianu zaczyna się od lewej strony powyżej osi OX.\\
Ponadto w punkcie $-8$ wykres odbija się od osi poziomej.\\
A więc $$x \in \{-8\} \cup [15,17].$$
\rozwStop
\odpStart
$x \in \{-8\} \cup [15,17]$
\odpStop
\testStart
A.$x \in \{-8\} \cup [15,17]$\\
B.$x \in \{8\} \cup (15,17)$\\
C.$x \in \{-8\} \cup (15,17]$\\
D.$x \in \{8\} \cup (15,17]$\\
E.$x \in \{-8\} \cup [15,17)$\\
F.$x \in \{8\} \cup [15,17)$\\
G.$x \in \{-8\} \cup (15,17)$\\
H.$x \in \{8\} \cup [15,17]$
\testStop
\kluczStart
A
\kluczStop



\zadStart{Zadanie z Wikieł Z 1.62 c) moja wersja nr 948}

Rozwiązać nierówności $(15-x)(x+8)^{2}(18-x)^{3}\le0$.
\zadStop
\rozwStart{Patryk Wirkus}{}
Miejsca zerowe naszego wielomianu to: $15, -8, 18$.\\
Wielomian jest stopnia parzystego, ponadto znak współczynnika przy\linebreak najwyższej potędze x jest ujemny.\\ W związku z tym wykres wielomianu zaczyna się od lewej strony powyżej osi OX.\\
Ponadto w punkcie $-8$ wykres odbija się od osi poziomej.\\
A więc $$x \in \{-8\} \cup [15,18].$$
\rozwStop
\odpStart
$x \in \{-8\} \cup [15,18]$
\odpStop
\testStart
A.$x \in \{-8\} \cup [15,18]$\\
B.$x \in \{8\} \cup (15,18)$\\
C.$x \in \{-8\} \cup (15,18]$\\
D.$x \in \{8\} \cup (15,18]$\\
E.$x \in \{-8\} \cup [15,18)$\\
F.$x \in \{8\} \cup [15,18)$\\
G.$x \in \{-8\} \cup (15,18)$\\
H.$x \in \{8\} \cup [15,18]$
\testStop
\kluczStart
A
\kluczStop



\zadStart{Zadanie z Wikieł Z 1.62 c) moja wersja nr 949}

Rozwiązać nierówności $(15-x)(x+8)^{2}(19-x)^{3}\le0$.
\zadStop
\rozwStart{Patryk Wirkus}{}
Miejsca zerowe naszego wielomianu to: $15, -8, 19$.\\
Wielomian jest stopnia parzystego, ponadto znak współczynnika przy\linebreak najwyższej potędze x jest ujemny.\\ W związku z tym wykres wielomianu zaczyna się od lewej strony powyżej osi OX.\\
Ponadto w punkcie $-8$ wykres odbija się od osi poziomej.\\
A więc $$x \in \{-8\} \cup [15,19].$$
\rozwStop
\odpStart
$x \in \{-8\} \cup [15,19]$
\odpStop
\testStart
A.$x \in \{-8\} \cup [15,19]$\\
B.$x \in \{8\} \cup (15,19)$\\
C.$x \in \{-8\} \cup (15,19]$\\
D.$x \in \{8\} \cup (15,19]$\\
E.$x \in \{-8\} \cup [15,19)$\\
F.$x \in \{8\} \cup [15,19)$\\
G.$x \in \{-8\} \cup (15,19)$\\
H.$x \in \{8\} \cup [15,19]$
\testStop
\kluczStart
A
\kluczStop



\zadStart{Zadanie z Wikieł Z 1.62 c) moja wersja nr 950}

Rozwiązać nierówności $(15-x)(x+8)^{2}(20-x)^{3}\le0$.
\zadStop
\rozwStart{Patryk Wirkus}{}
Miejsca zerowe naszego wielomianu to: $15, -8, 20$.\\
Wielomian jest stopnia parzystego, ponadto znak współczynnika przy\linebreak najwyższej potędze x jest ujemny.\\ W związku z tym wykres wielomianu zaczyna się od lewej strony powyżej osi OX.\\
Ponadto w punkcie $-8$ wykres odbija się od osi poziomej.\\
A więc $$x \in \{-8\} \cup [15,20].$$
\rozwStop
\odpStart
$x \in \{-8\} \cup [15,20]$
\odpStop
\testStart
A.$x \in \{-8\} \cup [15,20]$\\
B.$x \in \{8\} \cup (15,20)$\\
C.$x \in \{-8\} \cup (15,20]$\\
D.$x \in \{8\} \cup (15,20]$\\
E.$x \in \{-8\} \cup [15,20)$\\
F.$x \in \{8\} \cup [15,20)$\\
G.$x \in \{-8\} \cup (15,20)$\\
H.$x \in \{8\} \cup [15,20]$
\testStop
\kluczStart
A
\kluczStop



\zadStart{Zadanie z Wikieł Z 1.62 c) moja wersja nr 951}

Rozwiązać nierówności $(15-x)(x+9)^{2}(16-x)^{3}\le0$.
\zadStop
\rozwStart{Patryk Wirkus}{}
Miejsca zerowe naszego wielomianu to: $15, -9, 16$.\\
Wielomian jest stopnia parzystego, ponadto znak współczynnika przy\linebreak najwyższej potędze x jest ujemny.\\ W związku z tym wykres wielomianu zaczyna się od lewej strony powyżej osi OX.\\
Ponadto w punkcie $-9$ wykres odbija się od osi poziomej.\\
A więc $$x \in \{-9\} \cup [15,16].$$
\rozwStop
\odpStart
$x \in \{-9\} \cup [15,16]$
\odpStop
\testStart
A.$x \in \{-9\} \cup [15,16]$\\
B.$x \in \{9\} \cup (15,16)$\\
C.$x \in \{-9\} \cup (15,16]$\\
D.$x \in \{9\} \cup (15,16]$\\
E.$x \in \{-9\} \cup [15,16)$\\
F.$x \in \{9\} \cup [15,16)$\\
G.$x \in \{-9\} \cup (15,16)$\\
H.$x \in \{9\} \cup [15,16]$
\testStop
\kluczStart
A
\kluczStop



\zadStart{Zadanie z Wikieł Z 1.62 c) moja wersja nr 952}

Rozwiązać nierówności $(15-x)(x+9)^{2}(17-x)^{3}\le0$.
\zadStop
\rozwStart{Patryk Wirkus}{}
Miejsca zerowe naszego wielomianu to: $15, -9, 17$.\\
Wielomian jest stopnia parzystego, ponadto znak współczynnika przy\linebreak najwyższej potędze x jest ujemny.\\ W związku z tym wykres wielomianu zaczyna się od lewej strony powyżej osi OX.\\
Ponadto w punkcie $-9$ wykres odbija się od osi poziomej.\\
A więc $$x \in \{-9\} \cup [15,17].$$
\rozwStop
\odpStart
$x \in \{-9\} \cup [15,17]$
\odpStop
\testStart
A.$x \in \{-9\} \cup [15,17]$\\
B.$x \in \{9\} \cup (15,17)$\\
C.$x \in \{-9\} \cup (15,17]$\\
D.$x \in \{9\} \cup (15,17]$\\
E.$x \in \{-9\} \cup [15,17)$\\
F.$x \in \{9\} \cup [15,17)$\\
G.$x \in \{-9\} \cup (15,17)$\\
H.$x \in \{9\} \cup [15,17]$
\testStop
\kluczStart
A
\kluczStop



\zadStart{Zadanie z Wikieł Z 1.62 c) moja wersja nr 953}

Rozwiązać nierówności $(15-x)(x+9)^{2}(18-x)^{3}\le0$.
\zadStop
\rozwStart{Patryk Wirkus}{}
Miejsca zerowe naszego wielomianu to: $15, -9, 18$.\\
Wielomian jest stopnia parzystego, ponadto znak współczynnika przy\linebreak najwyższej potędze x jest ujemny.\\ W związku z tym wykres wielomianu zaczyna się od lewej strony powyżej osi OX.\\
Ponadto w punkcie $-9$ wykres odbija się od osi poziomej.\\
A więc $$x \in \{-9\} \cup [15,18].$$
\rozwStop
\odpStart
$x \in \{-9\} \cup [15,18]$
\odpStop
\testStart
A.$x \in \{-9\} \cup [15,18]$\\
B.$x \in \{9\} \cup (15,18)$\\
C.$x \in \{-9\} \cup (15,18]$\\
D.$x \in \{9\} \cup (15,18]$\\
E.$x \in \{-9\} \cup [15,18)$\\
F.$x \in \{9\} \cup [15,18)$\\
G.$x \in \{-9\} \cup (15,18)$\\
H.$x \in \{9\} \cup [15,18]$
\testStop
\kluczStart
A
\kluczStop



\zadStart{Zadanie z Wikieł Z 1.62 c) moja wersja nr 954}

Rozwiązać nierówności $(15-x)(x+9)^{2}(19-x)^{3}\le0$.
\zadStop
\rozwStart{Patryk Wirkus}{}
Miejsca zerowe naszego wielomianu to: $15, -9, 19$.\\
Wielomian jest stopnia parzystego, ponadto znak współczynnika przy\linebreak najwyższej potędze x jest ujemny.\\ W związku z tym wykres wielomianu zaczyna się od lewej strony powyżej osi OX.\\
Ponadto w punkcie $-9$ wykres odbija się od osi poziomej.\\
A więc $$x \in \{-9\} \cup [15,19].$$
\rozwStop
\odpStart
$x \in \{-9\} \cup [15,19]$
\odpStop
\testStart
A.$x \in \{-9\} \cup [15,19]$\\
B.$x \in \{9\} \cup (15,19)$\\
C.$x \in \{-9\} \cup (15,19]$\\
D.$x \in \{9\} \cup (15,19]$\\
E.$x \in \{-9\} \cup [15,19)$\\
F.$x \in \{9\} \cup [15,19)$\\
G.$x \in \{-9\} \cup (15,19)$\\
H.$x \in \{9\} \cup [15,19]$
\testStop
\kluczStart
A
\kluczStop



\zadStart{Zadanie z Wikieł Z 1.62 c) moja wersja nr 955}

Rozwiązać nierówności $(15-x)(x+9)^{2}(20-x)^{3}\le0$.
\zadStop
\rozwStart{Patryk Wirkus}{}
Miejsca zerowe naszego wielomianu to: $15, -9, 20$.\\
Wielomian jest stopnia parzystego, ponadto znak współczynnika przy\linebreak najwyższej potędze x jest ujemny.\\ W związku z tym wykres wielomianu zaczyna się od lewej strony powyżej osi OX.\\
Ponadto w punkcie $-9$ wykres odbija się od osi poziomej.\\
A więc $$x \in \{-9\} \cup [15,20].$$
\rozwStop
\odpStart
$x \in \{-9\} \cup [15,20]$
\odpStop
\testStart
A.$x \in \{-9\} \cup [15,20]$\\
B.$x \in \{9\} \cup (15,20)$\\
C.$x \in \{-9\} \cup (15,20]$\\
D.$x \in \{9\} \cup (15,20]$\\
E.$x \in \{-9\} \cup [15,20)$\\
F.$x \in \{9\} \cup [15,20)$\\
G.$x \in \{-9\} \cup (15,20)$\\
H.$x \in \{9\} \cup [15,20]$
\testStop
\kluczStart
A
\kluczStop



\zadStart{Zadanie z Wikieł Z 1.62 c) moja wersja nr 956}

Rozwiązać nierówności $(15-x)(x+10)^{2}(16-x)^{3}\le0$.
\zadStop
\rozwStart{Patryk Wirkus}{}
Miejsca zerowe naszego wielomianu to: $15, -10, 16$.\\
Wielomian jest stopnia parzystego, ponadto znak współczynnika przy\linebreak najwyższej potędze x jest ujemny.\\ W związku z tym wykres wielomianu zaczyna się od lewej strony powyżej osi OX.\\
Ponadto w punkcie $-10$ wykres odbija się od osi poziomej.\\
A więc $$x \in \{-10\} \cup [15,16].$$
\rozwStop
\odpStart
$x \in \{-10\} \cup [15,16]$
\odpStop
\testStart
A.$x \in \{-10\} \cup [15,16]$\\
B.$x \in \{10\} \cup (15,16)$\\
C.$x \in \{-10\} \cup (15,16]$\\
D.$x \in \{10\} \cup (15,16]$\\
E.$x \in \{-10\} \cup [15,16)$\\
F.$x \in \{10\} \cup [15,16)$\\
G.$x \in \{-10\} \cup (15,16)$\\
H.$x \in \{10\} \cup [15,16]$
\testStop
\kluczStart
A
\kluczStop



\zadStart{Zadanie z Wikieł Z 1.62 c) moja wersja nr 957}

Rozwiązać nierówności $(15-x)(x+10)^{2}(17-x)^{3}\le0$.
\zadStop
\rozwStart{Patryk Wirkus}{}
Miejsca zerowe naszego wielomianu to: $15, -10, 17$.\\
Wielomian jest stopnia parzystego, ponadto znak współczynnika przy\linebreak najwyższej potędze x jest ujemny.\\ W związku z tym wykres wielomianu zaczyna się od lewej strony powyżej osi OX.\\
Ponadto w punkcie $-10$ wykres odbija się od osi poziomej.\\
A więc $$x \in \{-10\} \cup [15,17].$$
\rozwStop
\odpStart
$x \in \{-10\} \cup [15,17]$
\odpStop
\testStart
A.$x \in \{-10\} \cup [15,17]$\\
B.$x \in \{10\} \cup (15,17)$\\
C.$x \in \{-10\} \cup (15,17]$\\
D.$x \in \{10\} \cup (15,17]$\\
E.$x \in \{-10\} \cup [15,17)$\\
F.$x \in \{10\} \cup [15,17)$\\
G.$x \in \{-10\} \cup (15,17)$\\
H.$x \in \{10\} \cup [15,17]$
\testStop
\kluczStart
A
\kluczStop



\zadStart{Zadanie z Wikieł Z 1.62 c) moja wersja nr 958}

Rozwiązać nierówności $(15-x)(x+10)^{2}(18-x)^{3}\le0$.
\zadStop
\rozwStart{Patryk Wirkus}{}
Miejsca zerowe naszego wielomianu to: $15, -10, 18$.\\
Wielomian jest stopnia parzystego, ponadto znak współczynnika przy\linebreak najwyższej potędze x jest ujemny.\\ W związku z tym wykres wielomianu zaczyna się od lewej strony powyżej osi OX.\\
Ponadto w punkcie $-10$ wykres odbija się od osi poziomej.\\
A więc $$x \in \{-10\} \cup [15,18].$$
\rozwStop
\odpStart
$x \in \{-10\} \cup [15,18]$
\odpStop
\testStart
A.$x \in \{-10\} \cup [15,18]$\\
B.$x \in \{10\} \cup (15,18)$\\
C.$x \in \{-10\} \cup (15,18]$\\
D.$x \in \{10\} \cup (15,18]$\\
E.$x \in \{-10\} \cup [15,18)$\\
F.$x \in \{10\} \cup [15,18)$\\
G.$x \in \{-10\} \cup (15,18)$\\
H.$x \in \{10\} \cup [15,18]$
\testStop
\kluczStart
A
\kluczStop



\zadStart{Zadanie z Wikieł Z 1.62 c) moja wersja nr 959}

Rozwiązać nierówności $(15-x)(x+10)^{2}(19-x)^{3}\le0$.
\zadStop
\rozwStart{Patryk Wirkus}{}
Miejsca zerowe naszego wielomianu to: $15, -10, 19$.\\
Wielomian jest stopnia parzystego, ponadto znak współczynnika przy\linebreak najwyższej potędze x jest ujemny.\\ W związku z tym wykres wielomianu zaczyna się od lewej strony powyżej osi OX.\\
Ponadto w punkcie $-10$ wykres odbija się od osi poziomej.\\
A więc $$x \in \{-10\} \cup [15,19].$$
\rozwStop
\odpStart
$x \in \{-10\} \cup [15,19]$
\odpStop
\testStart
A.$x \in \{-10\} \cup [15,19]$\\
B.$x \in \{10\} \cup (15,19)$\\
C.$x \in \{-10\} \cup (15,19]$\\
D.$x \in \{10\} \cup (15,19]$\\
E.$x \in \{-10\} \cup [15,19)$\\
F.$x \in \{10\} \cup [15,19)$\\
G.$x \in \{-10\} \cup (15,19)$\\
H.$x \in \{10\} \cup [15,19]$
\testStop
\kluczStart
A
\kluczStop



\zadStart{Zadanie z Wikieł Z 1.62 c) moja wersja nr 960}

Rozwiązać nierówności $(15-x)(x+10)^{2}(20-x)^{3}\le0$.
\zadStop
\rozwStart{Patryk Wirkus}{}
Miejsca zerowe naszego wielomianu to: $15, -10, 20$.\\
Wielomian jest stopnia parzystego, ponadto znak współczynnika przy\linebreak najwyższej potędze x jest ujemny.\\ W związku z tym wykres wielomianu zaczyna się od lewej strony powyżej osi OX.\\
Ponadto w punkcie $-10$ wykres odbija się od osi poziomej.\\
A więc $$x \in \{-10\} \cup [15,20].$$
\rozwStop
\odpStart
$x \in \{-10\} \cup [15,20]$
\odpStop
\testStart
A.$x \in \{-10\} \cup [15,20]$\\
B.$x \in \{10\} \cup (15,20)$\\
C.$x \in \{-10\} \cup (15,20]$\\
D.$x \in \{10\} \cup (15,20]$\\
E.$x \in \{-10\} \cup [15,20)$\\
F.$x \in \{10\} \cup [15,20)$\\
G.$x \in \{-10\} \cup (15,20)$\\
H.$x \in \{10\} \cup [15,20]$
\testStop
\kluczStart
A
\kluczStop



\zadStart{Zadanie z Wikieł Z 1.62 c) moja wersja nr 961}

Rozwiązać nierówności $(15-x)(x+11)^{2}(16-x)^{3}\le0$.
\zadStop
\rozwStart{Patryk Wirkus}{}
Miejsca zerowe naszego wielomianu to: $15, -11, 16$.\\
Wielomian jest stopnia parzystego, ponadto znak współczynnika przy\linebreak najwyższej potędze x jest ujemny.\\ W związku z tym wykres wielomianu zaczyna się od lewej strony powyżej osi OX.\\
Ponadto w punkcie $-11$ wykres odbija się od osi poziomej.\\
A więc $$x \in \{-11\} \cup [15,16].$$
\rozwStop
\odpStart
$x \in \{-11\} \cup [15,16]$
\odpStop
\testStart
A.$x \in \{-11\} \cup [15,16]$\\
B.$x \in \{11\} \cup (15,16)$\\
C.$x \in \{-11\} \cup (15,16]$\\
D.$x \in \{11\} \cup (15,16]$\\
E.$x \in \{-11\} \cup [15,16)$\\
F.$x \in \{11\} \cup [15,16)$\\
G.$x \in \{-11\} \cup (15,16)$\\
H.$x \in \{11\} \cup [15,16]$
\testStop
\kluczStart
A
\kluczStop



\zadStart{Zadanie z Wikieł Z 1.62 c) moja wersja nr 962}

Rozwiązać nierówności $(15-x)(x+11)^{2}(17-x)^{3}\le0$.
\zadStop
\rozwStart{Patryk Wirkus}{}
Miejsca zerowe naszego wielomianu to: $15, -11, 17$.\\
Wielomian jest stopnia parzystego, ponadto znak współczynnika przy\linebreak najwyższej potędze x jest ujemny.\\ W związku z tym wykres wielomianu zaczyna się od lewej strony powyżej osi OX.\\
Ponadto w punkcie $-11$ wykres odbija się od osi poziomej.\\
A więc $$x \in \{-11\} \cup [15,17].$$
\rozwStop
\odpStart
$x \in \{-11\} \cup [15,17]$
\odpStop
\testStart
A.$x \in \{-11\} \cup [15,17]$\\
B.$x \in \{11\} \cup (15,17)$\\
C.$x \in \{-11\} \cup (15,17]$\\
D.$x \in \{11\} \cup (15,17]$\\
E.$x \in \{-11\} \cup [15,17)$\\
F.$x \in \{11\} \cup [15,17)$\\
G.$x \in \{-11\} \cup (15,17)$\\
H.$x \in \{11\} \cup [15,17]$
\testStop
\kluczStart
A
\kluczStop



\zadStart{Zadanie z Wikieł Z 1.62 c) moja wersja nr 963}

Rozwiązać nierówności $(15-x)(x+11)^{2}(18-x)^{3}\le0$.
\zadStop
\rozwStart{Patryk Wirkus}{}
Miejsca zerowe naszego wielomianu to: $15, -11, 18$.\\
Wielomian jest stopnia parzystego, ponadto znak współczynnika przy\linebreak najwyższej potędze x jest ujemny.\\ W związku z tym wykres wielomianu zaczyna się od lewej strony powyżej osi OX.\\
Ponadto w punkcie $-11$ wykres odbija się od osi poziomej.\\
A więc $$x \in \{-11\} \cup [15,18].$$
\rozwStop
\odpStart
$x \in \{-11\} \cup [15,18]$
\odpStop
\testStart
A.$x \in \{-11\} \cup [15,18]$\\
B.$x \in \{11\} \cup (15,18)$\\
C.$x \in \{-11\} \cup (15,18]$\\
D.$x \in \{11\} \cup (15,18]$\\
E.$x \in \{-11\} \cup [15,18)$\\
F.$x \in \{11\} \cup [15,18)$\\
G.$x \in \{-11\} \cup (15,18)$\\
H.$x \in \{11\} \cup [15,18]$
\testStop
\kluczStart
A
\kluczStop



\zadStart{Zadanie z Wikieł Z 1.62 c) moja wersja nr 964}

Rozwiązać nierówności $(15-x)(x+11)^{2}(19-x)^{3}\le0$.
\zadStop
\rozwStart{Patryk Wirkus}{}
Miejsca zerowe naszego wielomianu to: $15, -11, 19$.\\
Wielomian jest stopnia parzystego, ponadto znak współczynnika przy\linebreak najwyższej potędze x jest ujemny.\\ W związku z tym wykres wielomianu zaczyna się od lewej strony powyżej osi OX.\\
Ponadto w punkcie $-11$ wykres odbija się od osi poziomej.\\
A więc $$x \in \{-11\} \cup [15,19].$$
\rozwStop
\odpStart
$x \in \{-11\} \cup [15,19]$
\odpStop
\testStart
A.$x \in \{-11\} \cup [15,19]$\\
B.$x \in \{11\} \cup (15,19)$\\
C.$x \in \{-11\} \cup (15,19]$\\
D.$x \in \{11\} \cup (15,19]$\\
E.$x \in \{-11\} \cup [15,19)$\\
F.$x \in \{11\} \cup [15,19)$\\
G.$x \in \{-11\} \cup (15,19)$\\
H.$x \in \{11\} \cup [15,19]$
\testStop
\kluczStart
A
\kluczStop



\zadStart{Zadanie z Wikieł Z 1.62 c) moja wersja nr 965}

Rozwiązać nierówności $(15-x)(x+11)^{2}(20-x)^{3}\le0$.
\zadStop
\rozwStart{Patryk Wirkus}{}
Miejsca zerowe naszego wielomianu to: $15, -11, 20$.\\
Wielomian jest stopnia parzystego, ponadto znak współczynnika przy\linebreak najwyższej potędze x jest ujemny.\\ W związku z tym wykres wielomianu zaczyna się od lewej strony powyżej osi OX.\\
Ponadto w punkcie $-11$ wykres odbija się od osi poziomej.\\
A więc $$x \in \{-11\} \cup [15,20].$$
\rozwStop
\odpStart
$x \in \{-11\} \cup [15,20]$
\odpStop
\testStart
A.$x \in \{-11\} \cup [15,20]$\\
B.$x \in \{11\} \cup (15,20)$\\
C.$x \in \{-11\} \cup (15,20]$\\
D.$x \in \{11\} \cup (15,20]$\\
E.$x \in \{-11\} \cup [15,20)$\\
F.$x \in \{11\} \cup [15,20)$\\
G.$x \in \{-11\} \cup (15,20)$\\
H.$x \in \{11\} \cup [15,20]$
\testStop
\kluczStart
A
\kluczStop



\zadStart{Zadanie z Wikieł Z 1.62 c) moja wersja nr 966}

Rozwiązać nierówności $(15-x)(x+12)^{2}(16-x)^{3}\le0$.
\zadStop
\rozwStart{Patryk Wirkus}{}
Miejsca zerowe naszego wielomianu to: $15, -12, 16$.\\
Wielomian jest stopnia parzystego, ponadto znak współczynnika przy\linebreak najwyższej potędze x jest ujemny.\\ W związku z tym wykres wielomianu zaczyna się od lewej strony powyżej osi OX.\\
Ponadto w punkcie $-12$ wykres odbija się od osi poziomej.\\
A więc $$x \in \{-12\} \cup [15,16].$$
\rozwStop
\odpStart
$x \in \{-12\} \cup [15,16]$
\odpStop
\testStart
A.$x \in \{-12\} \cup [15,16]$\\
B.$x \in \{12\} \cup (15,16)$\\
C.$x \in \{-12\} \cup (15,16]$\\
D.$x \in \{12\} \cup (15,16]$\\
E.$x \in \{-12\} \cup [15,16)$\\
F.$x \in \{12\} \cup [15,16)$\\
G.$x \in \{-12\} \cup (15,16)$\\
H.$x \in \{12\} \cup [15,16]$
\testStop
\kluczStart
A
\kluczStop



\zadStart{Zadanie z Wikieł Z 1.62 c) moja wersja nr 967}

Rozwiązać nierówności $(15-x)(x+12)^{2}(17-x)^{3}\le0$.
\zadStop
\rozwStart{Patryk Wirkus}{}
Miejsca zerowe naszego wielomianu to: $15, -12, 17$.\\
Wielomian jest stopnia parzystego, ponadto znak współczynnika przy\linebreak najwyższej potędze x jest ujemny.\\ W związku z tym wykres wielomianu zaczyna się od lewej strony powyżej osi OX.\\
Ponadto w punkcie $-12$ wykres odbija się od osi poziomej.\\
A więc $$x \in \{-12\} \cup [15,17].$$
\rozwStop
\odpStart
$x \in \{-12\} \cup [15,17]$
\odpStop
\testStart
A.$x \in \{-12\} \cup [15,17]$\\
B.$x \in \{12\} \cup (15,17)$\\
C.$x \in \{-12\} \cup (15,17]$\\
D.$x \in \{12\} \cup (15,17]$\\
E.$x \in \{-12\} \cup [15,17)$\\
F.$x \in \{12\} \cup [15,17)$\\
G.$x \in \{-12\} \cup (15,17)$\\
H.$x \in \{12\} \cup [15,17]$
\testStop
\kluczStart
A
\kluczStop



\zadStart{Zadanie z Wikieł Z 1.62 c) moja wersja nr 968}

Rozwiązać nierówności $(15-x)(x+12)^{2}(18-x)^{3}\le0$.
\zadStop
\rozwStart{Patryk Wirkus}{}
Miejsca zerowe naszego wielomianu to: $15, -12, 18$.\\
Wielomian jest stopnia parzystego, ponadto znak współczynnika przy\linebreak najwyższej potędze x jest ujemny.\\ W związku z tym wykres wielomianu zaczyna się od lewej strony powyżej osi OX.\\
Ponadto w punkcie $-12$ wykres odbija się od osi poziomej.\\
A więc $$x \in \{-12\} \cup [15,18].$$
\rozwStop
\odpStart
$x \in \{-12\} \cup [15,18]$
\odpStop
\testStart
A.$x \in \{-12\} \cup [15,18]$\\
B.$x \in \{12\} \cup (15,18)$\\
C.$x \in \{-12\} \cup (15,18]$\\
D.$x \in \{12\} \cup (15,18]$\\
E.$x \in \{-12\} \cup [15,18)$\\
F.$x \in \{12\} \cup [15,18)$\\
G.$x \in \{-12\} \cup (15,18)$\\
H.$x \in \{12\} \cup [15,18]$
\testStop
\kluczStart
A
\kluczStop



\zadStart{Zadanie z Wikieł Z 1.62 c) moja wersja nr 969}

Rozwiązać nierówności $(15-x)(x+12)^{2}(19-x)^{3}\le0$.
\zadStop
\rozwStart{Patryk Wirkus}{}
Miejsca zerowe naszego wielomianu to: $15, -12, 19$.\\
Wielomian jest stopnia parzystego, ponadto znak współczynnika przy\linebreak najwyższej potędze x jest ujemny.\\ W związku z tym wykres wielomianu zaczyna się od lewej strony powyżej osi OX.\\
Ponadto w punkcie $-12$ wykres odbija się od osi poziomej.\\
A więc $$x \in \{-12\} \cup [15,19].$$
\rozwStop
\odpStart
$x \in \{-12\} \cup [15,19]$
\odpStop
\testStart
A.$x \in \{-12\} \cup [15,19]$\\
B.$x \in \{12\} \cup (15,19)$\\
C.$x \in \{-12\} \cup (15,19]$\\
D.$x \in \{12\} \cup (15,19]$\\
E.$x \in \{-12\} \cup [15,19)$\\
F.$x \in \{12\} \cup [15,19)$\\
G.$x \in \{-12\} \cup (15,19)$\\
H.$x \in \{12\} \cup [15,19]$
\testStop
\kluczStart
A
\kluczStop



\zadStart{Zadanie z Wikieł Z 1.62 c) moja wersja nr 970}

Rozwiązać nierówności $(15-x)(x+12)^{2}(20-x)^{3}\le0$.
\zadStop
\rozwStart{Patryk Wirkus}{}
Miejsca zerowe naszego wielomianu to: $15, -12, 20$.\\
Wielomian jest stopnia parzystego, ponadto znak współczynnika przy\linebreak najwyższej potędze x jest ujemny.\\ W związku z tym wykres wielomianu zaczyna się od lewej strony powyżej osi OX.\\
Ponadto w punkcie $-12$ wykres odbija się od osi poziomej.\\
A więc $$x \in \{-12\} \cup [15,20].$$
\rozwStop
\odpStart
$x \in \{-12\} \cup [15,20]$
\odpStop
\testStart
A.$x \in \{-12\} \cup [15,20]$\\
B.$x \in \{12\} \cup (15,20)$\\
C.$x \in \{-12\} \cup (15,20]$\\
D.$x \in \{12\} \cup (15,20]$\\
E.$x \in \{-12\} \cup [15,20)$\\
F.$x \in \{12\} \cup [15,20)$\\
G.$x \in \{-12\} \cup (15,20)$\\
H.$x \in \{12\} \cup [15,20]$
\testStop
\kluczStart
A
\kluczStop



\zadStart{Zadanie z Wikieł Z 1.62 c) moja wersja nr 971}

Rozwiązać nierówności $(15-x)(x+13)^{2}(16-x)^{3}\le0$.
\zadStop
\rozwStart{Patryk Wirkus}{}
Miejsca zerowe naszego wielomianu to: $15, -13, 16$.\\
Wielomian jest stopnia parzystego, ponadto znak współczynnika przy\linebreak najwyższej potędze x jest ujemny.\\ W związku z tym wykres wielomianu zaczyna się od lewej strony powyżej osi OX.\\
Ponadto w punkcie $-13$ wykres odbija się od osi poziomej.\\
A więc $$x \in \{-13\} \cup [15,16].$$
\rozwStop
\odpStart
$x \in \{-13\} \cup [15,16]$
\odpStop
\testStart
A.$x \in \{-13\} \cup [15,16]$\\
B.$x \in \{13\} \cup (15,16)$\\
C.$x \in \{-13\} \cup (15,16]$\\
D.$x \in \{13\} \cup (15,16]$\\
E.$x \in \{-13\} \cup [15,16)$\\
F.$x \in \{13\} \cup [15,16)$\\
G.$x \in \{-13\} \cup (15,16)$\\
H.$x \in \{13\} \cup [15,16]$
\testStop
\kluczStart
A
\kluczStop



\zadStart{Zadanie z Wikieł Z 1.62 c) moja wersja nr 972}

Rozwiązać nierówności $(15-x)(x+13)^{2}(17-x)^{3}\le0$.
\zadStop
\rozwStart{Patryk Wirkus}{}
Miejsca zerowe naszego wielomianu to: $15, -13, 17$.\\
Wielomian jest stopnia parzystego, ponadto znak współczynnika przy\linebreak najwyższej potędze x jest ujemny.\\ W związku z tym wykres wielomianu zaczyna się od lewej strony powyżej osi OX.\\
Ponadto w punkcie $-13$ wykres odbija się od osi poziomej.\\
A więc $$x \in \{-13\} \cup [15,17].$$
\rozwStop
\odpStart
$x \in \{-13\} \cup [15,17]$
\odpStop
\testStart
A.$x \in \{-13\} \cup [15,17]$\\
B.$x \in \{13\} \cup (15,17)$\\
C.$x \in \{-13\} \cup (15,17]$\\
D.$x \in \{13\} \cup (15,17]$\\
E.$x \in \{-13\} \cup [15,17)$\\
F.$x \in \{13\} \cup [15,17)$\\
G.$x \in \{-13\} \cup (15,17)$\\
H.$x \in \{13\} \cup [15,17]$
\testStop
\kluczStart
A
\kluczStop



\zadStart{Zadanie z Wikieł Z 1.62 c) moja wersja nr 973}

Rozwiązać nierówności $(15-x)(x+13)^{2}(18-x)^{3}\le0$.
\zadStop
\rozwStart{Patryk Wirkus}{}
Miejsca zerowe naszego wielomianu to: $15, -13, 18$.\\
Wielomian jest stopnia parzystego, ponadto znak współczynnika przy\linebreak najwyższej potędze x jest ujemny.\\ W związku z tym wykres wielomianu zaczyna się od lewej strony powyżej osi OX.\\
Ponadto w punkcie $-13$ wykres odbija się od osi poziomej.\\
A więc $$x \in \{-13\} \cup [15,18].$$
\rozwStop
\odpStart
$x \in \{-13\} \cup [15,18]$
\odpStop
\testStart
A.$x \in \{-13\} \cup [15,18]$\\
B.$x \in \{13\} \cup (15,18)$\\
C.$x \in \{-13\} \cup (15,18]$\\
D.$x \in \{13\} \cup (15,18]$\\
E.$x \in \{-13\} \cup [15,18)$\\
F.$x \in \{13\} \cup [15,18)$\\
G.$x \in \{-13\} \cup (15,18)$\\
H.$x \in \{13\} \cup [15,18]$
\testStop
\kluczStart
A
\kluczStop



\zadStart{Zadanie z Wikieł Z 1.62 c) moja wersja nr 974}

Rozwiązać nierówności $(15-x)(x+13)^{2}(19-x)^{3}\le0$.
\zadStop
\rozwStart{Patryk Wirkus}{}
Miejsca zerowe naszego wielomianu to: $15, -13, 19$.\\
Wielomian jest stopnia parzystego, ponadto znak współczynnika przy\linebreak najwyższej potędze x jest ujemny.\\ W związku z tym wykres wielomianu zaczyna się od lewej strony powyżej osi OX.\\
Ponadto w punkcie $-13$ wykres odbija się od osi poziomej.\\
A więc $$x \in \{-13\} \cup [15,19].$$
\rozwStop
\odpStart
$x \in \{-13\} \cup [15,19]$
\odpStop
\testStart
A.$x \in \{-13\} \cup [15,19]$\\
B.$x \in \{13\} \cup (15,19)$\\
C.$x \in \{-13\} \cup (15,19]$\\
D.$x \in \{13\} \cup (15,19]$\\
E.$x \in \{-13\} \cup [15,19)$\\
F.$x \in \{13\} \cup [15,19)$\\
G.$x \in \{-13\} \cup (15,19)$\\
H.$x \in \{13\} \cup [15,19]$
\testStop
\kluczStart
A
\kluczStop



\zadStart{Zadanie z Wikieł Z 1.62 c) moja wersja nr 975}

Rozwiązać nierówności $(15-x)(x+13)^{2}(20-x)^{3}\le0$.
\zadStop
\rozwStart{Patryk Wirkus}{}
Miejsca zerowe naszego wielomianu to: $15, -13, 20$.\\
Wielomian jest stopnia parzystego, ponadto znak współczynnika przy\linebreak najwyższej potędze x jest ujemny.\\ W związku z tym wykres wielomianu zaczyna się od lewej strony powyżej osi OX.\\
Ponadto w punkcie $-13$ wykres odbija się od osi poziomej.\\
A więc $$x \in \{-13\} \cup [15,20].$$
\rozwStop
\odpStart
$x \in \{-13\} \cup [15,20]$
\odpStop
\testStart
A.$x \in \{-13\} \cup [15,20]$\\
B.$x \in \{13\} \cup (15,20)$\\
C.$x \in \{-13\} \cup (15,20]$\\
D.$x \in \{13\} \cup (15,20]$\\
E.$x \in \{-13\} \cup [15,20)$\\
F.$x \in \{13\} \cup [15,20)$\\
G.$x \in \{-13\} \cup (15,20)$\\
H.$x \in \{13\} \cup [15,20]$
\testStop
\kluczStart
A
\kluczStop



\zadStart{Zadanie z Wikieł Z 1.62 c) moja wersja nr 976}

Rozwiązać nierówności $(15-x)(x+14)^{2}(16-x)^{3}\le0$.
\zadStop
\rozwStart{Patryk Wirkus}{}
Miejsca zerowe naszego wielomianu to: $15, -14, 16$.\\
Wielomian jest stopnia parzystego, ponadto znak współczynnika przy\linebreak najwyższej potędze x jest ujemny.\\ W związku z tym wykres wielomianu zaczyna się od lewej strony powyżej osi OX.\\
Ponadto w punkcie $-14$ wykres odbija się od osi poziomej.\\
A więc $$x \in \{-14\} \cup [15,16].$$
\rozwStop
\odpStart
$x \in \{-14\} \cup [15,16]$
\odpStop
\testStart
A.$x \in \{-14\} \cup [15,16]$\\
B.$x \in \{14\} \cup (15,16)$\\
C.$x \in \{-14\} \cup (15,16]$\\
D.$x \in \{14\} \cup (15,16]$\\
E.$x \in \{-14\} \cup [15,16)$\\
F.$x \in \{14\} \cup [15,16)$\\
G.$x \in \{-14\} \cup (15,16)$\\
H.$x \in \{14\} \cup [15,16]$
\testStop
\kluczStart
A
\kluczStop



\zadStart{Zadanie z Wikieł Z 1.62 c) moja wersja nr 977}

Rozwiązać nierówności $(15-x)(x+14)^{2}(17-x)^{3}\le0$.
\zadStop
\rozwStart{Patryk Wirkus}{}
Miejsca zerowe naszego wielomianu to: $15, -14, 17$.\\
Wielomian jest stopnia parzystego, ponadto znak współczynnika przy\linebreak najwyższej potędze x jest ujemny.\\ W związku z tym wykres wielomianu zaczyna się od lewej strony powyżej osi OX.\\
Ponadto w punkcie $-14$ wykres odbija się od osi poziomej.\\
A więc $$x \in \{-14\} \cup [15,17].$$
\rozwStop
\odpStart
$x \in \{-14\} \cup [15,17]$
\odpStop
\testStart
A.$x \in \{-14\} \cup [15,17]$\\
B.$x \in \{14\} \cup (15,17)$\\
C.$x \in \{-14\} \cup (15,17]$\\
D.$x \in \{14\} \cup (15,17]$\\
E.$x \in \{-14\} \cup [15,17)$\\
F.$x \in \{14\} \cup [15,17)$\\
G.$x \in \{-14\} \cup (15,17)$\\
H.$x \in \{14\} \cup [15,17]$
\testStop
\kluczStart
A
\kluczStop



\zadStart{Zadanie z Wikieł Z 1.62 c) moja wersja nr 978}

Rozwiązać nierówności $(15-x)(x+14)^{2}(18-x)^{3}\le0$.
\zadStop
\rozwStart{Patryk Wirkus}{}
Miejsca zerowe naszego wielomianu to: $15, -14, 18$.\\
Wielomian jest stopnia parzystego, ponadto znak współczynnika przy\linebreak najwyższej potędze x jest ujemny.\\ W związku z tym wykres wielomianu zaczyna się od lewej strony powyżej osi OX.\\
Ponadto w punkcie $-14$ wykres odbija się od osi poziomej.\\
A więc $$x \in \{-14\} \cup [15,18].$$
\rozwStop
\odpStart
$x \in \{-14\} \cup [15,18]$
\odpStop
\testStart
A.$x \in \{-14\} \cup [15,18]$\\
B.$x \in \{14\} \cup (15,18)$\\
C.$x \in \{-14\} \cup (15,18]$\\
D.$x \in \{14\} \cup (15,18]$\\
E.$x \in \{-14\} \cup [15,18)$\\
F.$x \in \{14\} \cup [15,18)$\\
G.$x \in \{-14\} \cup (15,18)$\\
H.$x \in \{14\} \cup [15,18]$
\testStop
\kluczStart
A
\kluczStop



\zadStart{Zadanie z Wikieł Z 1.62 c) moja wersja nr 979}

Rozwiązać nierówności $(15-x)(x+14)^{2}(19-x)^{3}\le0$.
\zadStop
\rozwStart{Patryk Wirkus}{}
Miejsca zerowe naszego wielomianu to: $15, -14, 19$.\\
Wielomian jest stopnia parzystego, ponadto znak współczynnika przy\linebreak najwyższej potędze x jest ujemny.\\ W związku z tym wykres wielomianu zaczyna się od lewej strony powyżej osi OX.\\
Ponadto w punkcie $-14$ wykres odbija się od osi poziomej.\\
A więc $$x \in \{-14\} \cup [15,19].$$
\rozwStop
\odpStart
$x \in \{-14\} \cup [15,19]$
\odpStop
\testStart
A.$x \in \{-14\} \cup [15,19]$\\
B.$x \in \{14\} \cup (15,19)$\\
C.$x \in \{-14\} \cup (15,19]$\\
D.$x \in \{14\} \cup (15,19]$\\
E.$x \in \{-14\} \cup [15,19)$\\
F.$x \in \{14\} \cup [15,19)$\\
G.$x \in \{-14\} \cup (15,19)$\\
H.$x \in \{14\} \cup [15,19]$
\testStop
\kluczStart
A
\kluczStop



\zadStart{Zadanie z Wikieł Z 1.62 c) moja wersja nr 980}

Rozwiązać nierówności $(15-x)(x+14)^{2}(20-x)^{3}\le0$.
\zadStop
\rozwStart{Patryk Wirkus}{}
Miejsca zerowe naszego wielomianu to: $15, -14, 20$.\\
Wielomian jest stopnia parzystego, ponadto znak współczynnika przy\linebreak najwyższej potędze x jest ujemny.\\ W związku z tym wykres wielomianu zaczyna się od lewej strony powyżej osi OX.\\
Ponadto w punkcie $-14$ wykres odbija się od osi poziomej.\\
A więc $$x \in \{-14\} \cup [15,20].$$
\rozwStop
\odpStart
$x \in \{-14\} \cup [15,20]$
\odpStop
\testStart
A.$x \in \{-14\} \cup [15,20]$\\
B.$x \in \{14\} \cup (15,20)$\\
C.$x \in \{-14\} \cup (15,20]$\\
D.$x \in \{14\} \cup (15,20]$\\
E.$x \in \{-14\} \cup [15,20)$\\
F.$x \in \{14\} \cup [15,20)$\\
G.$x \in \{-14\} \cup (15,20)$\\
H.$x \in \{14\} \cup [15,20]$
\testStop
\kluczStart
A
\kluczStop



\zadStart{Zadanie z Wikieł Z 1.62 c) moja wersja nr 981}

Rozwiązać nierówności $(16-x)(x+1)^{2}(17-x)^{3}\le0$.
\zadStop
\rozwStart{Patryk Wirkus}{}
Miejsca zerowe naszego wielomianu to: $16, -1, 17$.\\
Wielomian jest stopnia parzystego, ponadto znak współczynnika przy\linebreak najwyższej potędze x jest ujemny.\\ W związku z tym wykres wielomianu zaczyna się od lewej strony powyżej osi OX.\\
Ponadto w punkcie $-1$ wykres odbija się od osi poziomej.\\
A więc $$x \in \{-1\} \cup [16,17].$$
\rozwStop
\odpStart
$x \in \{-1\} \cup [16,17]$
\odpStop
\testStart
A.$x \in \{-1\} \cup [16,17]$\\
B.$x \in \{1\} \cup (16,17)$\\
C.$x \in \{-1\} \cup (16,17]$\\
D.$x \in \{1\} \cup (16,17]$\\
E.$x \in \{-1\} \cup [16,17)$\\
F.$x \in \{1\} \cup [16,17)$\\
G.$x \in \{-1\} \cup (16,17)$\\
H.$x \in \{1\} \cup [16,17]$
\testStop
\kluczStart
A
\kluczStop



\zadStart{Zadanie z Wikieł Z 1.62 c) moja wersja nr 982}

Rozwiązać nierówności $(16-x)(x+1)^{2}(18-x)^{3}\le0$.
\zadStop
\rozwStart{Patryk Wirkus}{}
Miejsca zerowe naszego wielomianu to: $16, -1, 18$.\\
Wielomian jest stopnia parzystego, ponadto znak współczynnika przy\linebreak najwyższej potędze x jest ujemny.\\ W związku z tym wykres wielomianu zaczyna się od lewej strony powyżej osi OX.\\
Ponadto w punkcie $-1$ wykres odbija się od osi poziomej.\\
A więc $$x \in \{-1\} \cup [16,18].$$
\rozwStop
\odpStart
$x \in \{-1\} \cup [16,18]$
\odpStop
\testStart
A.$x \in \{-1\} \cup [16,18]$\\
B.$x \in \{1\} \cup (16,18)$\\
C.$x \in \{-1\} \cup (16,18]$\\
D.$x \in \{1\} \cup (16,18]$\\
E.$x \in \{-1\} \cup [16,18)$\\
F.$x \in \{1\} \cup [16,18)$\\
G.$x \in \{-1\} \cup (16,18)$\\
H.$x \in \{1\} \cup [16,18]$
\testStop
\kluczStart
A
\kluczStop



\zadStart{Zadanie z Wikieł Z 1.62 c) moja wersja nr 983}

Rozwiązać nierówności $(16-x)(x+1)^{2}(19-x)^{3}\le0$.
\zadStop
\rozwStart{Patryk Wirkus}{}
Miejsca zerowe naszego wielomianu to: $16, -1, 19$.\\
Wielomian jest stopnia parzystego, ponadto znak współczynnika przy\linebreak najwyższej potędze x jest ujemny.\\ W związku z tym wykres wielomianu zaczyna się od lewej strony powyżej osi OX.\\
Ponadto w punkcie $-1$ wykres odbija się od osi poziomej.\\
A więc $$x \in \{-1\} \cup [16,19].$$
\rozwStop
\odpStart
$x \in \{-1\} \cup [16,19]$
\odpStop
\testStart
A.$x \in \{-1\} \cup [16,19]$\\
B.$x \in \{1\} \cup (16,19)$\\
C.$x \in \{-1\} \cup (16,19]$\\
D.$x \in \{1\} \cup (16,19]$\\
E.$x \in \{-1\} \cup [16,19)$\\
F.$x \in \{1\} \cup [16,19)$\\
G.$x \in \{-1\} \cup (16,19)$\\
H.$x \in \{1\} \cup [16,19]$
\testStop
\kluczStart
A
\kluczStop



\zadStart{Zadanie z Wikieł Z 1.62 c) moja wersja nr 984}

Rozwiązać nierówności $(16-x)(x+1)^{2}(20-x)^{3}\le0$.
\zadStop
\rozwStart{Patryk Wirkus}{}
Miejsca zerowe naszego wielomianu to: $16, -1, 20$.\\
Wielomian jest stopnia parzystego, ponadto znak współczynnika przy\linebreak najwyższej potędze x jest ujemny.\\ W związku z tym wykres wielomianu zaczyna się od lewej strony powyżej osi OX.\\
Ponadto w punkcie $-1$ wykres odbija się od osi poziomej.\\
A więc $$x \in \{-1\} \cup [16,20].$$
\rozwStop
\odpStart
$x \in \{-1\} \cup [16,20]$
\odpStop
\testStart
A.$x \in \{-1\} \cup [16,20]$\\
B.$x \in \{1\} \cup (16,20)$\\
C.$x \in \{-1\} \cup (16,20]$\\
D.$x \in \{1\} \cup (16,20]$\\
E.$x \in \{-1\} \cup [16,20)$\\
F.$x \in \{1\} \cup [16,20)$\\
G.$x \in \{-1\} \cup (16,20)$\\
H.$x \in \{1\} \cup [16,20]$
\testStop
\kluczStart
A
\kluczStop



\zadStart{Zadanie z Wikieł Z 1.62 c) moja wersja nr 985}

Rozwiązać nierówności $(16-x)(x+2)^{2}(17-x)^{3}\le0$.
\zadStop
\rozwStart{Patryk Wirkus}{}
Miejsca zerowe naszego wielomianu to: $16, -2, 17$.\\
Wielomian jest stopnia parzystego, ponadto znak współczynnika przy\linebreak najwyższej potędze x jest ujemny.\\ W związku z tym wykres wielomianu zaczyna się od lewej strony powyżej osi OX.\\
Ponadto w punkcie $-2$ wykres odbija się od osi poziomej.\\
A więc $$x \in \{-2\} \cup [16,17].$$
\rozwStop
\odpStart
$x \in \{-2\} \cup [16,17]$
\odpStop
\testStart
A.$x \in \{-2\} \cup [16,17]$\\
B.$x \in \{2\} \cup (16,17)$\\
C.$x \in \{-2\} \cup (16,17]$\\
D.$x \in \{2\} \cup (16,17]$\\
E.$x \in \{-2\} \cup [16,17)$\\
F.$x \in \{2\} \cup [16,17)$\\
G.$x \in \{-2\} \cup (16,17)$\\
H.$x \in \{2\} \cup [16,17]$
\testStop
\kluczStart
A
\kluczStop



\zadStart{Zadanie z Wikieł Z 1.62 c) moja wersja nr 986}

Rozwiązać nierówności $(16-x)(x+2)^{2}(18-x)^{3}\le0$.
\zadStop
\rozwStart{Patryk Wirkus}{}
Miejsca zerowe naszego wielomianu to: $16, -2, 18$.\\
Wielomian jest stopnia parzystego, ponadto znak współczynnika przy\linebreak najwyższej potędze x jest ujemny.\\ W związku z tym wykres wielomianu zaczyna się od lewej strony powyżej osi OX.\\
Ponadto w punkcie $-2$ wykres odbija się od osi poziomej.\\
A więc $$x \in \{-2\} \cup [16,18].$$
\rozwStop
\odpStart
$x \in \{-2\} \cup [16,18]$
\odpStop
\testStart
A.$x \in \{-2\} \cup [16,18]$\\
B.$x \in \{2\} \cup (16,18)$\\
C.$x \in \{-2\} \cup (16,18]$\\
D.$x \in \{2\} \cup (16,18]$\\
E.$x \in \{-2\} \cup [16,18)$\\
F.$x \in \{2\} \cup [16,18)$\\
G.$x \in \{-2\} \cup (16,18)$\\
H.$x \in \{2\} \cup [16,18]$
\testStop
\kluczStart
A
\kluczStop



\zadStart{Zadanie z Wikieł Z 1.62 c) moja wersja nr 987}

Rozwiązać nierówności $(16-x)(x+2)^{2}(19-x)^{3}\le0$.
\zadStop
\rozwStart{Patryk Wirkus}{}
Miejsca zerowe naszego wielomianu to: $16, -2, 19$.\\
Wielomian jest stopnia parzystego, ponadto znak współczynnika przy\linebreak najwyższej potędze x jest ujemny.\\ W związku z tym wykres wielomianu zaczyna się od lewej strony powyżej osi OX.\\
Ponadto w punkcie $-2$ wykres odbija się od osi poziomej.\\
A więc $$x \in \{-2\} \cup [16,19].$$
\rozwStop
\odpStart
$x \in \{-2\} \cup [16,19]$
\odpStop
\testStart
A.$x \in \{-2\} \cup [16,19]$\\
B.$x \in \{2\} \cup (16,19)$\\
C.$x \in \{-2\} \cup (16,19]$\\
D.$x \in \{2\} \cup (16,19]$\\
E.$x \in \{-2\} \cup [16,19)$\\
F.$x \in \{2\} \cup [16,19)$\\
G.$x \in \{-2\} \cup (16,19)$\\
H.$x \in \{2\} \cup [16,19]$
\testStop
\kluczStart
A
\kluczStop



\zadStart{Zadanie z Wikieł Z 1.62 c) moja wersja nr 988}

Rozwiązać nierówności $(16-x)(x+2)^{2}(20-x)^{3}\le0$.
\zadStop
\rozwStart{Patryk Wirkus}{}
Miejsca zerowe naszego wielomianu to: $16, -2, 20$.\\
Wielomian jest stopnia parzystego, ponadto znak współczynnika przy\linebreak najwyższej potędze x jest ujemny.\\ W związku z tym wykres wielomianu zaczyna się od lewej strony powyżej osi OX.\\
Ponadto w punkcie $-2$ wykres odbija się od osi poziomej.\\
A więc $$x \in \{-2\} \cup [16,20].$$
\rozwStop
\odpStart
$x \in \{-2\} \cup [16,20]$
\odpStop
\testStart
A.$x \in \{-2\} \cup [16,20]$\\
B.$x \in \{2\} \cup (16,20)$\\
C.$x \in \{-2\} \cup (16,20]$\\
D.$x \in \{2\} \cup (16,20]$\\
E.$x \in \{-2\} \cup [16,20)$\\
F.$x \in \{2\} \cup [16,20)$\\
G.$x \in \{-2\} \cup (16,20)$\\
H.$x \in \{2\} \cup [16,20]$
\testStop
\kluczStart
A
\kluczStop



\zadStart{Zadanie z Wikieł Z 1.62 c) moja wersja nr 989}

Rozwiązać nierówności $(16-x)(x+3)^{2}(17-x)^{3}\le0$.
\zadStop
\rozwStart{Patryk Wirkus}{}
Miejsca zerowe naszego wielomianu to: $16, -3, 17$.\\
Wielomian jest stopnia parzystego, ponadto znak współczynnika przy\linebreak najwyższej potędze x jest ujemny.\\ W związku z tym wykres wielomianu zaczyna się od lewej strony powyżej osi OX.\\
Ponadto w punkcie $-3$ wykres odbija się od osi poziomej.\\
A więc $$x \in \{-3\} \cup [16,17].$$
\rozwStop
\odpStart
$x \in \{-3\} \cup [16,17]$
\odpStop
\testStart
A.$x \in \{-3\} \cup [16,17]$\\
B.$x \in \{3\} \cup (16,17)$\\
C.$x \in \{-3\} \cup (16,17]$\\
D.$x \in \{3\} \cup (16,17]$\\
E.$x \in \{-3\} \cup [16,17)$\\
F.$x \in \{3\} \cup [16,17)$\\
G.$x \in \{-3\} \cup (16,17)$\\
H.$x \in \{3\} \cup [16,17]$
\testStop
\kluczStart
A
\kluczStop



\zadStart{Zadanie z Wikieł Z 1.62 c) moja wersja nr 990}

Rozwiązać nierówności $(16-x)(x+3)^{2}(18-x)^{3}\le0$.
\zadStop
\rozwStart{Patryk Wirkus}{}
Miejsca zerowe naszego wielomianu to: $16, -3, 18$.\\
Wielomian jest stopnia parzystego, ponadto znak współczynnika przy\linebreak najwyższej potędze x jest ujemny.\\ W związku z tym wykres wielomianu zaczyna się od lewej strony powyżej osi OX.\\
Ponadto w punkcie $-3$ wykres odbija się od osi poziomej.\\
A więc $$x \in \{-3\} \cup [16,18].$$
\rozwStop
\odpStart
$x \in \{-3\} \cup [16,18]$
\odpStop
\testStart
A.$x \in \{-3\} \cup [16,18]$\\
B.$x \in \{3\} \cup (16,18)$\\
C.$x \in \{-3\} \cup (16,18]$\\
D.$x \in \{3\} \cup (16,18]$\\
E.$x \in \{-3\} \cup [16,18)$\\
F.$x \in \{3\} \cup [16,18)$\\
G.$x \in \{-3\} \cup (16,18)$\\
H.$x \in \{3\} \cup [16,18]$
\testStop
\kluczStart
A
\kluczStop



\zadStart{Zadanie z Wikieł Z 1.62 c) moja wersja nr 991}

Rozwiązać nierówności $(16-x)(x+3)^{2}(19-x)^{3}\le0$.
\zadStop
\rozwStart{Patryk Wirkus}{}
Miejsca zerowe naszego wielomianu to: $16, -3, 19$.\\
Wielomian jest stopnia parzystego, ponadto znak współczynnika przy\linebreak najwyższej potędze x jest ujemny.\\ W związku z tym wykres wielomianu zaczyna się od lewej strony powyżej osi OX.\\
Ponadto w punkcie $-3$ wykres odbija się od osi poziomej.\\
A więc $$x \in \{-3\} \cup [16,19].$$
\rozwStop
\odpStart
$x \in \{-3\} \cup [16,19]$
\odpStop
\testStart
A.$x \in \{-3\} \cup [16,19]$\\
B.$x \in \{3\} \cup (16,19)$\\
C.$x \in \{-3\} \cup (16,19]$\\
D.$x \in \{3\} \cup (16,19]$\\
E.$x \in \{-3\} \cup [16,19)$\\
F.$x \in \{3\} \cup [16,19)$\\
G.$x \in \{-3\} \cup (16,19)$\\
H.$x \in \{3\} \cup [16,19]$
\testStop
\kluczStart
A
\kluczStop



\zadStart{Zadanie z Wikieł Z 1.62 c) moja wersja nr 992}

Rozwiązać nierówności $(16-x)(x+3)^{2}(20-x)^{3}\le0$.
\zadStop
\rozwStart{Patryk Wirkus}{}
Miejsca zerowe naszego wielomianu to: $16, -3, 20$.\\
Wielomian jest stopnia parzystego, ponadto znak współczynnika przy\linebreak najwyższej potędze x jest ujemny.\\ W związku z tym wykres wielomianu zaczyna się od lewej strony powyżej osi OX.\\
Ponadto w punkcie $-3$ wykres odbija się od osi poziomej.\\
A więc $$x \in \{-3\} \cup [16,20].$$
\rozwStop
\odpStart
$x \in \{-3\} \cup [16,20]$
\odpStop
\testStart
A.$x \in \{-3\} \cup [16,20]$\\
B.$x \in \{3\} \cup (16,20)$\\
C.$x \in \{-3\} \cup (16,20]$\\
D.$x \in \{3\} \cup (16,20]$\\
E.$x \in \{-3\} \cup [16,20)$\\
F.$x \in \{3\} \cup [16,20)$\\
G.$x \in \{-3\} \cup (16,20)$\\
H.$x \in \{3\} \cup [16,20]$
\testStop
\kluczStart
A
\kluczStop



\zadStart{Zadanie z Wikieł Z 1.62 c) moja wersja nr 993}

Rozwiązać nierówności $(16-x)(x+4)^{2}(17-x)^{3}\le0$.
\zadStop
\rozwStart{Patryk Wirkus}{}
Miejsca zerowe naszego wielomianu to: $16, -4, 17$.\\
Wielomian jest stopnia parzystego, ponadto znak współczynnika przy\linebreak najwyższej potędze x jest ujemny.\\ W związku z tym wykres wielomianu zaczyna się od lewej strony powyżej osi OX.\\
Ponadto w punkcie $-4$ wykres odbija się od osi poziomej.\\
A więc $$x \in \{-4\} \cup [16,17].$$
\rozwStop
\odpStart
$x \in \{-4\} \cup [16,17]$
\odpStop
\testStart
A.$x \in \{-4\} \cup [16,17]$\\
B.$x \in \{4\} \cup (16,17)$\\
C.$x \in \{-4\} \cup (16,17]$\\
D.$x \in \{4\} \cup (16,17]$\\
E.$x \in \{-4\} \cup [16,17)$\\
F.$x \in \{4\} \cup [16,17)$\\
G.$x \in \{-4\} \cup (16,17)$\\
H.$x \in \{4\} \cup [16,17]$
\testStop
\kluczStart
A
\kluczStop



\zadStart{Zadanie z Wikieł Z 1.62 c) moja wersja nr 994}

Rozwiązać nierówności $(16-x)(x+4)^{2}(18-x)^{3}\le0$.
\zadStop
\rozwStart{Patryk Wirkus}{}
Miejsca zerowe naszego wielomianu to: $16, -4, 18$.\\
Wielomian jest stopnia parzystego, ponadto znak współczynnika przy\linebreak najwyższej potędze x jest ujemny.\\ W związku z tym wykres wielomianu zaczyna się od lewej strony powyżej osi OX.\\
Ponadto w punkcie $-4$ wykres odbija się od osi poziomej.\\
A więc $$x \in \{-4\} \cup [16,18].$$
\rozwStop
\odpStart
$x \in \{-4\} \cup [16,18]$
\odpStop
\testStart
A.$x \in \{-4\} \cup [16,18]$\\
B.$x \in \{4\} \cup (16,18)$\\
C.$x \in \{-4\} \cup (16,18]$\\
D.$x \in \{4\} \cup (16,18]$\\
E.$x \in \{-4\} \cup [16,18)$\\
F.$x \in \{4\} \cup [16,18)$\\
G.$x \in \{-4\} \cup (16,18)$\\
H.$x \in \{4\} \cup [16,18]$
\testStop
\kluczStart
A
\kluczStop



\zadStart{Zadanie z Wikieł Z 1.62 c) moja wersja nr 995}

Rozwiązać nierówności $(16-x)(x+4)^{2}(19-x)^{3}\le0$.
\zadStop
\rozwStart{Patryk Wirkus}{}
Miejsca zerowe naszego wielomianu to: $16, -4, 19$.\\
Wielomian jest stopnia parzystego, ponadto znak współczynnika przy\linebreak najwyższej potędze x jest ujemny.\\ W związku z tym wykres wielomianu zaczyna się od lewej strony powyżej osi OX.\\
Ponadto w punkcie $-4$ wykres odbija się od osi poziomej.\\
A więc $$x \in \{-4\} \cup [16,19].$$
\rozwStop
\odpStart
$x \in \{-4\} \cup [16,19]$
\odpStop
\testStart
A.$x \in \{-4\} \cup [16,19]$\\
B.$x \in \{4\} \cup (16,19)$\\
C.$x \in \{-4\} \cup (16,19]$\\
D.$x \in \{4\} \cup (16,19]$\\
E.$x \in \{-4\} \cup [16,19)$\\
F.$x \in \{4\} \cup [16,19)$\\
G.$x \in \{-4\} \cup (16,19)$\\
H.$x \in \{4\} \cup [16,19]$
\testStop
\kluczStart
A
\kluczStop



\zadStart{Zadanie z Wikieł Z 1.62 c) moja wersja nr 996}

Rozwiązać nierówności $(16-x)(x+4)^{2}(20-x)^{3}\le0$.
\zadStop
\rozwStart{Patryk Wirkus}{}
Miejsca zerowe naszego wielomianu to: $16, -4, 20$.\\
Wielomian jest stopnia parzystego, ponadto znak współczynnika przy\linebreak najwyższej potędze x jest ujemny.\\ W związku z tym wykres wielomianu zaczyna się od lewej strony powyżej osi OX.\\
Ponadto w punkcie $-4$ wykres odbija się od osi poziomej.\\
A więc $$x \in \{-4\} \cup [16,20].$$
\rozwStop
\odpStart
$x \in \{-4\} \cup [16,20]$
\odpStop
\testStart
A.$x \in \{-4\} \cup [16,20]$\\
B.$x \in \{4\} \cup (16,20)$\\
C.$x \in \{-4\} \cup (16,20]$\\
D.$x \in \{4\} \cup (16,20]$\\
E.$x \in \{-4\} \cup [16,20)$\\
F.$x \in \{4\} \cup [16,20)$\\
G.$x \in \{-4\} \cup (16,20)$\\
H.$x \in \{4\} \cup [16,20]$
\testStop
\kluczStart
A
\kluczStop



\zadStart{Zadanie z Wikieł Z 1.62 c) moja wersja nr 997}

Rozwiązać nierówności $(16-x)(x+5)^{2}(17-x)^{3}\le0$.
\zadStop
\rozwStart{Patryk Wirkus}{}
Miejsca zerowe naszego wielomianu to: $16, -5, 17$.\\
Wielomian jest stopnia parzystego, ponadto znak współczynnika przy\linebreak najwyższej potędze x jest ujemny.\\ W związku z tym wykres wielomianu zaczyna się od lewej strony powyżej osi OX.\\
Ponadto w punkcie $-5$ wykres odbija się od osi poziomej.\\
A więc $$x \in \{-5\} \cup [16,17].$$
\rozwStop
\odpStart
$x \in \{-5\} \cup [16,17]$
\odpStop
\testStart
A.$x \in \{-5\} \cup [16,17]$\\
B.$x \in \{5\} \cup (16,17)$\\
C.$x \in \{-5\} \cup (16,17]$\\
D.$x \in \{5\} \cup (16,17]$\\
E.$x \in \{-5\} \cup [16,17)$\\
F.$x \in \{5\} \cup [16,17)$\\
G.$x \in \{-5\} \cup (16,17)$\\
H.$x \in \{5\} \cup [16,17]$
\testStop
\kluczStart
A
\kluczStop



\zadStart{Zadanie z Wikieł Z 1.62 c) moja wersja nr 998}

Rozwiązać nierówności $(16-x)(x+5)^{2}(18-x)^{3}\le0$.
\zadStop
\rozwStart{Patryk Wirkus}{}
Miejsca zerowe naszego wielomianu to: $16, -5, 18$.\\
Wielomian jest stopnia parzystego, ponadto znak współczynnika przy\linebreak najwyższej potędze x jest ujemny.\\ W związku z tym wykres wielomianu zaczyna się od lewej strony powyżej osi OX.\\
Ponadto w punkcie $-5$ wykres odbija się od osi poziomej.\\
A więc $$x \in \{-5\} \cup [16,18].$$
\rozwStop
\odpStart
$x \in \{-5\} \cup [16,18]$
\odpStop
\testStart
A.$x \in \{-5\} \cup [16,18]$\\
B.$x \in \{5\} \cup (16,18)$\\
C.$x \in \{-5\} \cup (16,18]$\\
D.$x \in \{5\} \cup (16,18]$\\
E.$x \in \{-5\} \cup [16,18)$\\
F.$x \in \{5\} \cup [16,18)$\\
G.$x \in \{-5\} \cup (16,18)$\\
H.$x \in \{5\} \cup [16,18]$
\testStop
\kluczStart
A
\kluczStop



\zadStart{Zadanie z Wikieł Z 1.62 c) moja wersja nr 999}

Rozwiązać nierówności $(16-x)(x+5)^{2}(19-x)^{3}\le0$.
\zadStop
\rozwStart{Patryk Wirkus}{}
Miejsca zerowe naszego wielomianu to: $16, -5, 19$.\\
Wielomian jest stopnia parzystego, ponadto znak współczynnika przy\linebreak najwyższej potędze x jest ujemny.\\ W związku z tym wykres wielomianu zaczyna się od lewej strony powyżej osi OX.\\
Ponadto w punkcie $-5$ wykres odbija się od osi poziomej.\\
A więc $$x \in \{-5\} \cup [16,19].$$
\rozwStop
\odpStart
$x \in \{-5\} \cup [16,19]$
\odpStop
\testStart
A.$x \in \{-5\} \cup [16,19]$\\
B.$x \in \{5\} \cup (16,19)$\\
C.$x \in \{-5\} \cup (16,19]$\\
D.$x \in \{5\} \cup (16,19]$\\
E.$x \in \{-5\} \cup [16,19)$\\
F.$x \in \{5\} \cup [16,19)$\\
G.$x \in \{-5\} \cup (16,19)$\\
H.$x \in \{5\} \cup [16,19]$
\testStop
\kluczStart
A
\kluczStop



\zadStart{Zadanie z Wikieł Z 1.62 c) moja wersja nr 1000}

Rozwiązać nierówności $(16-x)(x+5)^{2}(20-x)^{3}\le0$.
\zadStop
\rozwStart{Patryk Wirkus}{}
Miejsca zerowe naszego wielomianu to: $16, -5, 20$.\\
Wielomian jest stopnia parzystego, ponadto znak współczynnika przy\linebreak najwyższej potędze x jest ujemny.\\ W związku z tym wykres wielomianu zaczyna się od lewej strony powyżej osi OX.\\
Ponadto w punkcie $-5$ wykres odbija się od osi poziomej.\\
A więc $$x \in \{-5\} \cup [16,20].$$
\rozwStop
\odpStart
$x \in \{-5\} \cup [16,20]$
\odpStop
\testStart
A.$x \in \{-5\} \cup [16,20]$\\
B.$x \in \{5\} \cup (16,20)$\\
C.$x \in \{-5\} \cup (16,20]$\\
D.$x \in \{5\} \cup (16,20]$\\
E.$x \in \{-5\} \cup [16,20)$\\
F.$x \in \{5\} \cup [16,20)$\\
G.$x \in \{-5\} \cup (16,20)$\\
H.$x \in \{5\} \cup [16,20]$
\testStop
\kluczStart
A
\kluczStop



\zadStart{Zadanie z Wikieł Z 1.62 c) moja wersja nr 1001}

Rozwiązać nierówności $(16-x)(x+6)^{2}(17-x)^{3}\le0$.
\zadStop
\rozwStart{Patryk Wirkus}{}
Miejsca zerowe naszego wielomianu to: $16, -6, 17$.\\
Wielomian jest stopnia parzystego, ponadto znak współczynnika przy\linebreak najwyższej potędze x jest ujemny.\\ W związku z tym wykres wielomianu zaczyna się od lewej strony powyżej osi OX.\\
Ponadto w punkcie $-6$ wykres odbija się od osi poziomej.\\
A więc $$x \in \{-6\} \cup [16,17].$$
\rozwStop
\odpStart
$x \in \{-6\} \cup [16,17]$
\odpStop
\testStart
A.$x \in \{-6\} \cup [16,17]$\\
B.$x \in \{6\} \cup (16,17)$\\
C.$x \in \{-6\} \cup (16,17]$\\
D.$x \in \{6\} \cup (16,17]$\\
E.$x \in \{-6\} \cup [16,17)$\\
F.$x \in \{6\} \cup [16,17)$\\
G.$x \in \{-6\} \cup (16,17)$\\
H.$x \in \{6\} \cup [16,17]$
\testStop
\kluczStart
A
\kluczStop



\zadStart{Zadanie z Wikieł Z 1.62 c) moja wersja nr 1002}

Rozwiązać nierówności $(16-x)(x+6)^{2}(18-x)^{3}\le0$.
\zadStop
\rozwStart{Patryk Wirkus}{}
Miejsca zerowe naszego wielomianu to: $16, -6, 18$.\\
Wielomian jest stopnia parzystego, ponadto znak współczynnika przy\linebreak najwyższej potędze x jest ujemny.\\ W związku z tym wykres wielomianu zaczyna się od lewej strony powyżej osi OX.\\
Ponadto w punkcie $-6$ wykres odbija się od osi poziomej.\\
A więc $$x \in \{-6\} \cup [16,18].$$
\rozwStop
\odpStart
$x \in \{-6\} \cup [16,18]$
\odpStop
\testStart
A.$x \in \{-6\} \cup [16,18]$\\
B.$x \in \{6\} \cup (16,18)$\\
C.$x \in \{-6\} \cup (16,18]$\\
D.$x \in \{6\} \cup (16,18]$\\
E.$x \in \{-6\} \cup [16,18)$\\
F.$x \in \{6\} \cup [16,18)$\\
G.$x \in \{-6\} \cup (16,18)$\\
H.$x \in \{6\} \cup [16,18]$
\testStop
\kluczStart
A
\kluczStop



\zadStart{Zadanie z Wikieł Z 1.62 c) moja wersja nr 1003}

Rozwiązać nierówności $(16-x)(x+6)^{2}(19-x)^{3}\le0$.
\zadStop
\rozwStart{Patryk Wirkus}{}
Miejsca zerowe naszego wielomianu to: $16, -6, 19$.\\
Wielomian jest stopnia parzystego, ponadto znak współczynnika przy\linebreak najwyższej potędze x jest ujemny.\\ W związku z tym wykres wielomianu zaczyna się od lewej strony powyżej osi OX.\\
Ponadto w punkcie $-6$ wykres odbija się od osi poziomej.\\
A więc $$x \in \{-6\} \cup [16,19].$$
\rozwStop
\odpStart
$x \in \{-6\} \cup [16,19]$
\odpStop
\testStart
A.$x \in \{-6\} \cup [16,19]$\\
B.$x \in \{6\} \cup (16,19)$\\
C.$x \in \{-6\} \cup (16,19]$\\
D.$x \in \{6\} \cup (16,19]$\\
E.$x \in \{-6\} \cup [16,19)$\\
F.$x \in \{6\} \cup [16,19)$\\
G.$x \in \{-6\} \cup (16,19)$\\
H.$x \in \{6\} \cup [16,19]$
\testStop
\kluczStart
A
\kluczStop



\zadStart{Zadanie z Wikieł Z 1.62 c) moja wersja nr 1004}

Rozwiązać nierówności $(16-x)(x+6)^{2}(20-x)^{3}\le0$.
\zadStop
\rozwStart{Patryk Wirkus}{}
Miejsca zerowe naszego wielomianu to: $16, -6, 20$.\\
Wielomian jest stopnia parzystego, ponadto znak współczynnika przy\linebreak najwyższej potędze x jest ujemny.\\ W związku z tym wykres wielomianu zaczyna się od lewej strony powyżej osi OX.\\
Ponadto w punkcie $-6$ wykres odbija się od osi poziomej.\\
A więc $$x \in \{-6\} \cup [16,20].$$
\rozwStop
\odpStart
$x \in \{-6\} \cup [16,20]$
\odpStop
\testStart
A.$x \in \{-6\} \cup [16,20]$\\
B.$x \in \{6\} \cup (16,20)$\\
C.$x \in \{-6\} \cup (16,20]$\\
D.$x \in \{6\} \cup (16,20]$\\
E.$x \in \{-6\} \cup [16,20)$\\
F.$x \in \{6\} \cup [16,20)$\\
G.$x \in \{-6\} \cup (16,20)$\\
H.$x \in \{6\} \cup [16,20]$
\testStop
\kluczStart
A
\kluczStop



\zadStart{Zadanie z Wikieł Z 1.62 c) moja wersja nr 1005}

Rozwiązać nierówności $(16-x)(x+7)^{2}(17-x)^{3}\le0$.
\zadStop
\rozwStart{Patryk Wirkus}{}
Miejsca zerowe naszego wielomianu to: $16, -7, 17$.\\
Wielomian jest stopnia parzystego, ponadto znak współczynnika przy\linebreak najwyższej potędze x jest ujemny.\\ W związku z tym wykres wielomianu zaczyna się od lewej strony powyżej osi OX.\\
Ponadto w punkcie $-7$ wykres odbija się od osi poziomej.\\
A więc $$x \in \{-7\} \cup [16,17].$$
\rozwStop
\odpStart
$x \in \{-7\} \cup [16,17]$
\odpStop
\testStart
A.$x \in \{-7\} \cup [16,17]$\\
B.$x \in \{7\} \cup (16,17)$\\
C.$x \in \{-7\} \cup (16,17]$\\
D.$x \in \{7\} \cup (16,17]$\\
E.$x \in \{-7\} \cup [16,17)$\\
F.$x \in \{7\} \cup [16,17)$\\
G.$x \in \{-7\} \cup (16,17)$\\
H.$x \in \{7\} \cup [16,17]$
\testStop
\kluczStart
A
\kluczStop



\zadStart{Zadanie z Wikieł Z 1.62 c) moja wersja nr 1006}

Rozwiązać nierówności $(16-x)(x+7)^{2}(18-x)^{3}\le0$.
\zadStop
\rozwStart{Patryk Wirkus}{}
Miejsca zerowe naszego wielomianu to: $16, -7, 18$.\\
Wielomian jest stopnia parzystego, ponadto znak współczynnika przy\linebreak najwyższej potędze x jest ujemny.\\ W związku z tym wykres wielomianu zaczyna się od lewej strony powyżej osi OX.\\
Ponadto w punkcie $-7$ wykres odbija się od osi poziomej.\\
A więc $$x \in \{-7\} \cup [16,18].$$
\rozwStop
\odpStart
$x \in \{-7\} \cup [16,18]$
\odpStop
\testStart
A.$x \in \{-7\} \cup [16,18]$\\
B.$x \in \{7\} \cup (16,18)$\\
C.$x \in \{-7\} \cup (16,18]$\\
D.$x \in \{7\} \cup (16,18]$\\
E.$x \in \{-7\} \cup [16,18)$\\
F.$x \in \{7\} \cup [16,18)$\\
G.$x \in \{-7\} \cup (16,18)$\\
H.$x \in \{7\} \cup [16,18]$
\testStop
\kluczStart
A
\kluczStop



\zadStart{Zadanie z Wikieł Z 1.62 c) moja wersja nr 1007}

Rozwiązać nierówności $(16-x)(x+7)^{2}(19-x)^{3}\le0$.
\zadStop
\rozwStart{Patryk Wirkus}{}
Miejsca zerowe naszego wielomianu to: $16, -7, 19$.\\
Wielomian jest stopnia parzystego, ponadto znak współczynnika przy\linebreak najwyższej potędze x jest ujemny.\\ W związku z tym wykres wielomianu zaczyna się od lewej strony powyżej osi OX.\\
Ponadto w punkcie $-7$ wykres odbija się od osi poziomej.\\
A więc $$x \in \{-7\} \cup [16,19].$$
\rozwStop
\odpStart
$x \in \{-7\} \cup [16,19]$
\odpStop
\testStart
A.$x \in \{-7\} \cup [16,19]$\\
B.$x \in \{7\} \cup (16,19)$\\
C.$x \in \{-7\} \cup (16,19]$\\
D.$x \in \{7\} \cup (16,19]$\\
E.$x \in \{-7\} \cup [16,19)$\\
F.$x \in \{7\} \cup [16,19)$\\
G.$x \in \{-7\} \cup (16,19)$\\
H.$x \in \{7\} \cup [16,19]$
\testStop
\kluczStart
A
\kluczStop



\zadStart{Zadanie z Wikieł Z 1.62 c) moja wersja nr 1008}

Rozwiązać nierówności $(16-x)(x+7)^{2}(20-x)^{3}\le0$.
\zadStop
\rozwStart{Patryk Wirkus}{}
Miejsca zerowe naszego wielomianu to: $16, -7, 20$.\\
Wielomian jest stopnia parzystego, ponadto znak współczynnika przy\linebreak najwyższej potędze x jest ujemny.\\ W związku z tym wykres wielomianu zaczyna się od lewej strony powyżej osi OX.\\
Ponadto w punkcie $-7$ wykres odbija się od osi poziomej.\\
A więc $$x \in \{-7\} \cup [16,20].$$
\rozwStop
\odpStart
$x \in \{-7\} \cup [16,20]$
\odpStop
\testStart
A.$x \in \{-7\} \cup [16,20]$\\
B.$x \in \{7\} \cup (16,20)$\\
C.$x \in \{-7\} \cup (16,20]$\\
D.$x \in \{7\} \cup (16,20]$\\
E.$x \in \{-7\} \cup [16,20)$\\
F.$x \in \{7\} \cup [16,20)$\\
G.$x \in \{-7\} \cup (16,20)$\\
H.$x \in \{7\} \cup [16,20]$
\testStop
\kluczStart
A
\kluczStop



\zadStart{Zadanie z Wikieł Z 1.62 c) moja wersja nr 1009}

Rozwiązać nierówności $(16-x)(x+8)^{2}(17-x)^{3}\le0$.
\zadStop
\rozwStart{Patryk Wirkus}{}
Miejsca zerowe naszego wielomianu to: $16, -8, 17$.\\
Wielomian jest stopnia parzystego, ponadto znak współczynnika przy\linebreak najwyższej potędze x jest ujemny.\\ W związku z tym wykres wielomianu zaczyna się od lewej strony powyżej osi OX.\\
Ponadto w punkcie $-8$ wykres odbija się od osi poziomej.\\
A więc $$x \in \{-8\} \cup [16,17].$$
\rozwStop
\odpStart
$x \in \{-8\} \cup [16,17]$
\odpStop
\testStart
A.$x \in \{-8\} \cup [16,17]$\\
B.$x \in \{8\} \cup (16,17)$\\
C.$x \in \{-8\} \cup (16,17]$\\
D.$x \in \{8\} \cup (16,17]$\\
E.$x \in \{-8\} \cup [16,17)$\\
F.$x \in \{8\} \cup [16,17)$\\
G.$x \in \{-8\} \cup (16,17)$\\
H.$x \in \{8\} \cup [16,17]$
\testStop
\kluczStart
A
\kluczStop



\zadStart{Zadanie z Wikieł Z 1.62 c) moja wersja nr 1010}

Rozwiązać nierówności $(16-x)(x+8)^{2}(18-x)^{3}\le0$.
\zadStop
\rozwStart{Patryk Wirkus}{}
Miejsca zerowe naszego wielomianu to: $16, -8, 18$.\\
Wielomian jest stopnia parzystego, ponadto znak współczynnika przy\linebreak najwyższej potędze x jest ujemny.\\ W związku z tym wykres wielomianu zaczyna się od lewej strony powyżej osi OX.\\
Ponadto w punkcie $-8$ wykres odbija się od osi poziomej.\\
A więc $$x \in \{-8\} \cup [16,18].$$
\rozwStop
\odpStart
$x \in \{-8\} \cup [16,18]$
\odpStop
\testStart
A.$x \in \{-8\} \cup [16,18]$\\
B.$x \in \{8\} \cup (16,18)$\\
C.$x \in \{-8\} \cup (16,18]$\\
D.$x \in \{8\} \cup (16,18]$\\
E.$x \in \{-8\} \cup [16,18)$\\
F.$x \in \{8\} \cup [16,18)$\\
G.$x \in \{-8\} \cup (16,18)$\\
H.$x \in \{8\} \cup [16,18]$
\testStop
\kluczStart
A
\kluczStop



\zadStart{Zadanie z Wikieł Z 1.62 c) moja wersja nr 1011}

Rozwiązać nierówności $(16-x)(x+8)^{2}(19-x)^{3}\le0$.
\zadStop
\rozwStart{Patryk Wirkus}{}
Miejsca zerowe naszego wielomianu to: $16, -8, 19$.\\
Wielomian jest stopnia parzystego, ponadto znak współczynnika przy\linebreak najwyższej potędze x jest ujemny.\\ W związku z tym wykres wielomianu zaczyna się od lewej strony powyżej osi OX.\\
Ponadto w punkcie $-8$ wykres odbija się od osi poziomej.\\
A więc $$x \in \{-8\} \cup [16,19].$$
\rozwStop
\odpStart
$x \in \{-8\} \cup [16,19]$
\odpStop
\testStart
A.$x \in \{-8\} \cup [16,19]$\\
B.$x \in \{8\} \cup (16,19)$\\
C.$x \in \{-8\} \cup (16,19]$\\
D.$x \in \{8\} \cup (16,19]$\\
E.$x \in \{-8\} \cup [16,19)$\\
F.$x \in \{8\} \cup [16,19)$\\
G.$x \in \{-8\} \cup (16,19)$\\
H.$x \in \{8\} \cup [16,19]$
\testStop
\kluczStart
A
\kluczStop



\zadStart{Zadanie z Wikieł Z 1.62 c) moja wersja nr 1012}

Rozwiązać nierówności $(16-x)(x+8)^{2}(20-x)^{3}\le0$.
\zadStop
\rozwStart{Patryk Wirkus}{}
Miejsca zerowe naszego wielomianu to: $16, -8, 20$.\\
Wielomian jest stopnia parzystego, ponadto znak współczynnika przy\linebreak najwyższej potędze x jest ujemny.\\ W związku z tym wykres wielomianu zaczyna się od lewej strony powyżej osi OX.\\
Ponadto w punkcie $-8$ wykres odbija się od osi poziomej.\\
A więc $$x \in \{-8\} \cup [16,20].$$
\rozwStop
\odpStart
$x \in \{-8\} \cup [16,20]$
\odpStop
\testStart
A.$x \in \{-8\} \cup [16,20]$\\
B.$x \in \{8\} \cup (16,20)$\\
C.$x \in \{-8\} \cup (16,20]$\\
D.$x \in \{8\} \cup (16,20]$\\
E.$x \in \{-8\} \cup [16,20)$\\
F.$x \in \{8\} \cup [16,20)$\\
G.$x \in \{-8\} \cup (16,20)$\\
H.$x \in \{8\} \cup [16,20]$
\testStop
\kluczStart
A
\kluczStop



\zadStart{Zadanie z Wikieł Z 1.62 c) moja wersja nr 1013}

Rozwiązać nierówności $(16-x)(x+9)^{2}(17-x)^{3}\le0$.
\zadStop
\rozwStart{Patryk Wirkus}{}
Miejsca zerowe naszego wielomianu to: $16, -9, 17$.\\
Wielomian jest stopnia parzystego, ponadto znak współczynnika przy\linebreak najwyższej potędze x jest ujemny.\\ W związku z tym wykres wielomianu zaczyna się od lewej strony powyżej osi OX.\\
Ponadto w punkcie $-9$ wykres odbija się od osi poziomej.\\
A więc $$x \in \{-9\} \cup [16,17].$$
\rozwStop
\odpStart
$x \in \{-9\} \cup [16,17]$
\odpStop
\testStart
A.$x \in \{-9\} \cup [16,17]$\\
B.$x \in \{9\} \cup (16,17)$\\
C.$x \in \{-9\} \cup (16,17]$\\
D.$x \in \{9\} \cup (16,17]$\\
E.$x \in \{-9\} \cup [16,17)$\\
F.$x \in \{9\} \cup [16,17)$\\
G.$x \in \{-9\} \cup (16,17)$\\
H.$x \in \{9\} \cup [16,17]$
\testStop
\kluczStart
A
\kluczStop



\zadStart{Zadanie z Wikieł Z 1.62 c) moja wersja nr 1014}

Rozwiązać nierówności $(16-x)(x+9)^{2}(18-x)^{3}\le0$.
\zadStop
\rozwStart{Patryk Wirkus}{}
Miejsca zerowe naszego wielomianu to: $16, -9, 18$.\\
Wielomian jest stopnia parzystego, ponadto znak współczynnika przy\linebreak najwyższej potędze x jest ujemny.\\ W związku z tym wykres wielomianu zaczyna się od lewej strony powyżej osi OX.\\
Ponadto w punkcie $-9$ wykres odbija się od osi poziomej.\\
A więc $$x \in \{-9\} \cup [16,18].$$
\rozwStop
\odpStart
$x \in \{-9\} \cup [16,18]$
\odpStop
\testStart
A.$x \in \{-9\} \cup [16,18]$\\
B.$x \in \{9\} \cup (16,18)$\\
C.$x \in \{-9\} \cup (16,18]$\\
D.$x \in \{9\} \cup (16,18]$\\
E.$x \in \{-9\} \cup [16,18)$\\
F.$x \in \{9\} \cup [16,18)$\\
G.$x \in \{-9\} \cup (16,18)$\\
H.$x \in \{9\} \cup [16,18]$
\testStop
\kluczStart
A
\kluczStop



\zadStart{Zadanie z Wikieł Z 1.62 c) moja wersja nr 1015}

Rozwiązać nierówności $(16-x)(x+9)^{2}(19-x)^{3}\le0$.
\zadStop
\rozwStart{Patryk Wirkus}{}
Miejsca zerowe naszego wielomianu to: $16, -9, 19$.\\
Wielomian jest stopnia parzystego, ponadto znak współczynnika przy\linebreak najwyższej potędze x jest ujemny.\\ W związku z tym wykres wielomianu zaczyna się od lewej strony powyżej osi OX.\\
Ponadto w punkcie $-9$ wykres odbija się od osi poziomej.\\
A więc $$x \in \{-9\} \cup [16,19].$$
\rozwStop
\odpStart
$x \in \{-9\} \cup [16,19]$
\odpStop
\testStart
A.$x \in \{-9\} \cup [16,19]$\\
B.$x \in \{9\} \cup (16,19)$\\
C.$x \in \{-9\} \cup (16,19]$\\
D.$x \in \{9\} \cup (16,19]$\\
E.$x \in \{-9\} \cup [16,19)$\\
F.$x \in \{9\} \cup [16,19)$\\
G.$x \in \{-9\} \cup (16,19)$\\
H.$x \in \{9\} \cup [16,19]$
\testStop
\kluczStart
A
\kluczStop



\zadStart{Zadanie z Wikieł Z 1.62 c) moja wersja nr 1016}

Rozwiązać nierówności $(16-x)(x+9)^{2}(20-x)^{3}\le0$.
\zadStop
\rozwStart{Patryk Wirkus}{}
Miejsca zerowe naszego wielomianu to: $16, -9, 20$.\\
Wielomian jest stopnia parzystego, ponadto znak współczynnika przy\linebreak najwyższej potędze x jest ujemny.\\ W związku z tym wykres wielomianu zaczyna się od lewej strony powyżej osi OX.\\
Ponadto w punkcie $-9$ wykres odbija się od osi poziomej.\\
A więc $$x \in \{-9\} \cup [16,20].$$
\rozwStop
\odpStart
$x \in \{-9\} \cup [16,20]$
\odpStop
\testStart
A.$x \in \{-9\} \cup [16,20]$\\
B.$x \in \{9\} \cup (16,20)$\\
C.$x \in \{-9\} \cup (16,20]$\\
D.$x \in \{9\} \cup (16,20]$\\
E.$x \in \{-9\} \cup [16,20)$\\
F.$x \in \{9\} \cup [16,20)$\\
G.$x \in \{-9\} \cup (16,20)$\\
H.$x \in \{9\} \cup [16,20]$
\testStop
\kluczStart
A
\kluczStop



\zadStart{Zadanie z Wikieł Z 1.62 c) moja wersja nr 1017}

Rozwiązać nierówności $(16-x)(x+10)^{2}(17-x)^{3}\le0$.
\zadStop
\rozwStart{Patryk Wirkus}{}
Miejsca zerowe naszego wielomianu to: $16, -10, 17$.\\
Wielomian jest stopnia parzystego, ponadto znak współczynnika przy\linebreak najwyższej potędze x jest ujemny.\\ W związku z tym wykres wielomianu zaczyna się od lewej strony powyżej osi OX.\\
Ponadto w punkcie $-10$ wykres odbija się od osi poziomej.\\
A więc $$x \in \{-10\} \cup [16,17].$$
\rozwStop
\odpStart
$x \in \{-10\} \cup [16,17]$
\odpStop
\testStart
A.$x \in \{-10\} \cup [16,17]$\\
B.$x \in \{10\} \cup (16,17)$\\
C.$x \in \{-10\} \cup (16,17]$\\
D.$x \in \{10\} \cup (16,17]$\\
E.$x \in \{-10\} \cup [16,17)$\\
F.$x \in \{10\} \cup [16,17)$\\
G.$x \in \{-10\} \cup (16,17)$\\
H.$x \in \{10\} \cup [16,17]$
\testStop
\kluczStart
A
\kluczStop



\zadStart{Zadanie z Wikieł Z 1.62 c) moja wersja nr 1018}

Rozwiązać nierówności $(16-x)(x+10)^{2}(18-x)^{3}\le0$.
\zadStop
\rozwStart{Patryk Wirkus}{}
Miejsca zerowe naszego wielomianu to: $16, -10, 18$.\\
Wielomian jest stopnia parzystego, ponadto znak współczynnika przy\linebreak najwyższej potędze x jest ujemny.\\ W związku z tym wykres wielomianu zaczyna się od lewej strony powyżej osi OX.\\
Ponadto w punkcie $-10$ wykres odbija się od osi poziomej.\\
A więc $$x \in \{-10\} \cup [16,18].$$
\rozwStop
\odpStart
$x \in \{-10\} \cup [16,18]$
\odpStop
\testStart
A.$x \in \{-10\} \cup [16,18]$\\
B.$x \in \{10\} \cup (16,18)$\\
C.$x \in \{-10\} \cup (16,18]$\\
D.$x \in \{10\} \cup (16,18]$\\
E.$x \in \{-10\} \cup [16,18)$\\
F.$x \in \{10\} \cup [16,18)$\\
G.$x \in \{-10\} \cup (16,18)$\\
H.$x \in \{10\} \cup [16,18]$
\testStop
\kluczStart
A
\kluczStop



\zadStart{Zadanie z Wikieł Z 1.62 c) moja wersja nr 1019}

Rozwiązać nierówności $(16-x)(x+10)^{2}(19-x)^{3}\le0$.
\zadStop
\rozwStart{Patryk Wirkus}{}
Miejsca zerowe naszego wielomianu to: $16, -10, 19$.\\
Wielomian jest stopnia parzystego, ponadto znak współczynnika przy\linebreak najwyższej potędze x jest ujemny.\\ W związku z tym wykres wielomianu zaczyna się od lewej strony powyżej osi OX.\\
Ponadto w punkcie $-10$ wykres odbija się od osi poziomej.\\
A więc $$x \in \{-10\} \cup [16,19].$$
\rozwStop
\odpStart
$x \in \{-10\} \cup [16,19]$
\odpStop
\testStart
A.$x \in \{-10\} \cup [16,19]$\\
B.$x \in \{10\} \cup (16,19)$\\
C.$x \in \{-10\} \cup (16,19]$\\
D.$x \in \{10\} \cup (16,19]$\\
E.$x \in \{-10\} \cup [16,19)$\\
F.$x \in \{10\} \cup [16,19)$\\
G.$x \in \{-10\} \cup (16,19)$\\
H.$x \in \{10\} \cup [16,19]$
\testStop
\kluczStart
A
\kluczStop



\zadStart{Zadanie z Wikieł Z 1.62 c) moja wersja nr 1020}

Rozwiązać nierówności $(16-x)(x+10)^{2}(20-x)^{3}\le0$.
\zadStop
\rozwStart{Patryk Wirkus}{}
Miejsca zerowe naszego wielomianu to: $16, -10, 20$.\\
Wielomian jest stopnia parzystego, ponadto znak współczynnika przy\linebreak najwyższej potędze x jest ujemny.\\ W związku z tym wykres wielomianu zaczyna się od lewej strony powyżej osi OX.\\
Ponadto w punkcie $-10$ wykres odbija się od osi poziomej.\\
A więc $$x \in \{-10\} \cup [16,20].$$
\rozwStop
\odpStart
$x \in \{-10\} \cup [16,20]$
\odpStop
\testStart
A.$x \in \{-10\} \cup [16,20]$\\
B.$x \in \{10\} \cup (16,20)$\\
C.$x \in \{-10\} \cup (16,20]$\\
D.$x \in \{10\} \cup (16,20]$\\
E.$x \in \{-10\} \cup [16,20)$\\
F.$x \in \{10\} \cup [16,20)$\\
G.$x \in \{-10\} \cup (16,20)$\\
H.$x \in \{10\} \cup [16,20]$
\testStop
\kluczStart
A
\kluczStop



\zadStart{Zadanie z Wikieł Z 1.62 c) moja wersja nr 1021}

Rozwiązać nierówności $(16-x)(x+11)^{2}(17-x)^{3}\le0$.
\zadStop
\rozwStart{Patryk Wirkus}{}
Miejsca zerowe naszego wielomianu to: $16, -11, 17$.\\
Wielomian jest stopnia parzystego, ponadto znak współczynnika przy\linebreak najwyższej potędze x jest ujemny.\\ W związku z tym wykres wielomianu zaczyna się od lewej strony powyżej osi OX.\\
Ponadto w punkcie $-11$ wykres odbija się od osi poziomej.\\
A więc $$x \in \{-11\} \cup [16,17].$$
\rozwStop
\odpStart
$x \in \{-11\} \cup [16,17]$
\odpStop
\testStart
A.$x \in \{-11\} \cup [16,17]$\\
B.$x \in \{11\} \cup (16,17)$\\
C.$x \in \{-11\} \cup (16,17]$\\
D.$x \in \{11\} \cup (16,17]$\\
E.$x \in \{-11\} \cup [16,17)$\\
F.$x \in \{11\} \cup [16,17)$\\
G.$x \in \{-11\} \cup (16,17)$\\
H.$x \in \{11\} \cup [16,17]$
\testStop
\kluczStart
A
\kluczStop



\zadStart{Zadanie z Wikieł Z 1.62 c) moja wersja nr 1022}

Rozwiązać nierówności $(16-x)(x+11)^{2}(18-x)^{3}\le0$.
\zadStop
\rozwStart{Patryk Wirkus}{}
Miejsca zerowe naszego wielomianu to: $16, -11, 18$.\\
Wielomian jest stopnia parzystego, ponadto znak współczynnika przy\linebreak najwyższej potędze x jest ujemny.\\ W związku z tym wykres wielomianu zaczyna się od lewej strony powyżej osi OX.\\
Ponadto w punkcie $-11$ wykres odbija się od osi poziomej.\\
A więc $$x \in \{-11\} \cup [16,18].$$
\rozwStop
\odpStart
$x \in \{-11\} \cup [16,18]$
\odpStop
\testStart
A.$x \in \{-11\} \cup [16,18]$\\
B.$x \in \{11\} \cup (16,18)$\\
C.$x \in \{-11\} \cup (16,18]$\\
D.$x \in \{11\} \cup (16,18]$\\
E.$x \in \{-11\} \cup [16,18)$\\
F.$x \in \{11\} \cup [16,18)$\\
G.$x \in \{-11\} \cup (16,18)$\\
H.$x \in \{11\} \cup [16,18]$
\testStop
\kluczStart
A
\kluczStop



\zadStart{Zadanie z Wikieł Z 1.62 c) moja wersja nr 1023}

Rozwiązać nierówności $(16-x)(x+11)^{2}(19-x)^{3}\le0$.
\zadStop
\rozwStart{Patryk Wirkus}{}
Miejsca zerowe naszego wielomianu to: $16, -11, 19$.\\
Wielomian jest stopnia parzystego, ponadto znak współczynnika przy\linebreak najwyższej potędze x jest ujemny.\\ W związku z tym wykres wielomianu zaczyna się od lewej strony powyżej osi OX.\\
Ponadto w punkcie $-11$ wykres odbija się od osi poziomej.\\
A więc $$x \in \{-11\} \cup [16,19].$$
\rozwStop
\odpStart
$x \in \{-11\} \cup [16,19]$
\odpStop
\testStart
A.$x \in \{-11\} \cup [16,19]$\\
B.$x \in \{11\} \cup (16,19)$\\
C.$x \in \{-11\} \cup (16,19]$\\
D.$x \in \{11\} \cup (16,19]$\\
E.$x \in \{-11\} \cup [16,19)$\\
F.$x \in \{11\} \cup [16,19)$\\
G.$x \in \{-11\} \cup (16,19)$\\
H.$x \in \{11\} \cup [16,19]$
\testStop
\kluczStart
A
\kluczStop



\zadStart{Zadanie z Wikieł Z 1.62 c) moja wersja nr 1024}

Rozwiązać nierówności $(16-x)(x+11)^{2}(20-x)^{3}\le0$.
\zadStop
\rozwStart{Patryk Wirkus}{}
Miejsca zerowe naszego wielomianu to: $16, -11, 20$.\\
Wielomian jest stopnia parzystego, ponadto znak współczynnika przy\linebreak najwyższej potędze x jest ujemny.\\ W związku z tym wykres wielomianu zaczyna się od lewej strony powyżej osi OX.\\
Ponadto w punkcie $-11$ wykres odbija się od osi poziomej.\\
A więc $$x \in \{-11\} \cup [16,20].$$
\rozwStop
\odpStart
$x \in \{-11\} \cup [16,20]$
\odpStop
\testStart
A.$x \in \{-11\} \cup [16,20]$\\
B.$x \in \{11\} \cup (16,20)$\\
C.$x \in \{-11\} \cup (16,20]$\\
D.$x \in \{11\} \cup (16,20]$\\
E.$x \in \{-11\} \cup [16,20)$\\
F.$x \in \{11\} \cup [16,20)$\\
G.$x \in \{-11\} \cup (16,20)$\\
H.$x \in \{11\} \cup [16,20]$
\testStop
\kluczStart
A
\kluczStop



\zadStart{Zadanie z Wikieł Z 1.62 c) moja wersja nr 1025}

Rozwiązać nierówności $(16-x)(x+12)^{2}(17-x)^{3}\le0$.
\zadStop
\rozwStart{Patryk Wirkus}{}
Miejsca zerowe naszego wielomianu to: $16, -12, 17$.\\
Wielomian jest stopnia parzystego, ponadto znak współczynnika przy\linebreak najwyższej potędze x jest ujemny.\\ W związku z tym wykres wielomianu zaczyna się od lewej strony powyżej osi OX.\\
Ponadto w punkcie $-12$ wykres odbija się od osi poziomej.\\
A więc $$x \in \{-12\} \cup [16,17].$$
\rozwStop
\odpStart
$x \in \{-12\} \cup [16,17]$
\odpStop
\testStart
A.$x \in \{-12\} \cup [16,17]$\\
B.$x \in \{12\} \cup (16,17)$\\
C.$x \in \{-12\} \cup (16,17]$\\
D.$x \in \{12\} \cup (16,17]$\\
E.$x \in \{-12\} \cup [16,17)$\\
F.$x \in \{12\} \cup [16,17)$\\
G.$x \in \{-12\} \cup (16,17)$\\
H.$x \in \{12\} \cup [16,17]$
\testStop
\kluczStart
A
\kluczStop



\zadStart{Zadanie z Wikieł Z 1.62 c) moja wersja nr 1026}

Rozwiązać nierówności $(16-x)(x+12)^{2}(18-x)^{3}\le0$.
\zadStop
\rozwStart{Patryk Wirkus}{}
Miejsca zerowe naszego wielomianu to: $16, -12, 18$.\\
Wielomian jest stopnia parzystego, ponadto znak współczynnika przy\linebreak najwyższej potędze x jest ujemny.\\ W związku z tym wykres wielomianu zaczyna się od lewej strony powyżej osi OX.\\
Ponadto w punkcie $-12$ wykres odbija się od osi poziomej.\\
A więc $$x \in \{-12\} \cup [16,18].$$
\rozwStop
\odpStart
$x \in \{-12\} \cup [16,18]$
\odpStop
\testStart
A.$x \in \{-12\} \cup [16,18]$\\
B.$x \in \{12\} \cup (16,18)$\\
C.$x \in \{-12\} \cup (16,18]$\\
D.$x \in \{12\} \cup (16,18]$\\
E.$x \in \{-12\} \cup [16,18)$\\
F.$x \in \{12\} \cup [16,18)$\\
G.$x \in \{-12\} \cup (16,18)$\\
H.$x \in \{12\} \cup [16,18]$
\testStop
\kluczStart
A
\kluczStop



\zadStart{Zadanie z Wikieł Z 1.62 c) moja wersja nr 1027}

Rozwiązać nierówności $(16-x)(x+12)^{2}(19-x)^{3}\le0$.
\zadStop
\rozwStart{Patryk Wirkus}{}
Miejsca zerowe naszego wielomianu to: $16, -12, 19$.\\
Wielomian jest stopnia parzystego, ponadto znak współczynnika przy\linebreak najwyższej potędze x jest ujemny.\\ W związku z tym wykres wielomianu zaczyna się od lewej strony powyżej osi OX.\\
Ponadto w punkcie $-12$ wykres odbija się od osi poziomej.\\
A więc $$x \in \{-12\} \cup [16,19].$$
\rozwStop
\odpStart
$x \in \{-12\} \cup [16,19]$
\odpStop
\testStart
A.$x \in \{-12\} \cup [16,19]$\\
B.$x \in \{12\} \cup (16,19)$\\
C.$x \in \{-12\} \cup (16,19]$\\
D.$x \in \{12\} \cup (16,19]$\\
E.$x \in \{-12\} \cup [16,19)$\\
F.$x \in \{12\} \cup [16,19)$\\
G.$x \in \{-12\} \cup (16,19)$\\
H.$x \in \{12\} \cup [16,19]$
\testStop
\kluczStart
A
\kluczStop



\zadStart{Zadanie z Wikieł Z 1.62 c) moja wersja nr 1028}

Rozwiązać nierówności $(16-x)(x+12)^{2}(20-x)^{3}\le0$.
\zadStop
\rozwStart{Patryk Wirkus}{}
Miejsca zerowe naszego wielomianu to: $16, -12, 20$.\\
Wielomian jest stopnia parzystego, ponadto znak współczynnika przy\linebreak najwyższej potędze x jest ujemny.\\ W związku z tym wykres wielomianu zaczyna się od lewej strony powyżej osi OX.\\
Ponadto w punkcie $-12$ wykres odbija się od osi poziomej.\\
A więc $$x \in \{-12\} \cup [16,20].$$
\rozwStop
\odpStart
$x \in \{-12\} \cup [16,20]$
\odpStop
\testStart
A.$x \in \{-12\} \cup [16,20]$\\
B.$x \in \{12\} \cup (16,20)$\\
C.$x \in \{-12\} \cup (16,20]$\\
D.$x \in \{12\} \cup (16,20]$\\
E.$x \in \{-12\} \cup [16,20)$\\
F.$x \in \{12\} \cup [16,20)$\\
G.$x \in \{-12\} \cup (16,20)$\\
H.$x \in \{12\} \cup [16,20]$
\testStop
\kluczStart
A
\kluczStop



\zadStart{Zadanie z Wikieł Z 1.62 c) moja wersja nr 1029}

Rozwiązać nierówności $(16-x)(x+13)^{2}(17-x)^{3}\le0$.
\zadStop
\rozwStart{Patryk Wirkus}{}
Miejsca zerowe naszego wielomianu to: $16, -13, 17$.\\
Wielomian jest stopnia parzystego, ponadto znak współczynnika przy\linebreak najwyższej potędze x jest ujemny.\\ W związku z tym wykres wielomianu zaczyna się od lewej strony powyżej osi OX.\\
Ponadto w punkcie $-13$ wykres odbija się od osi poziomej.\\
A więc $$x \in \{-13\} \cup [16,17].$$
\rozwStop
\odpStart
$x \in \{-13\} \cup [16,17]$
\odpStop
\testStart
A.$x \in \{-13\} \cup [16,17]$\\
B.$x \in \{13\} \cup (16,17)$\\
C.$x \in \{-13\} \cup (16,17]$\\
D.$x \in \{13\} \cup (16,17]$\\
E.$x \in \{-13\} \cup [16,17)$\\
F.$x \in \{13\} \cup [16,17)$\\
G.$x \in \{-13\} \cup (16,17)$\\
H.$x \in \{13\} \cup [16,17]$
\testStop
\kluczStart
A
\kluczStop



\zadStart{Zadanie z Wikieł Z 1.62 c) moja wersja nr 1030}

Rozwiązać nierówności $(16-x)(x+13)^{2}(18-x)^{3}\le0$.
\zadStop
\rozwStart{Patryk Wirkus}{}
Miejsca zerowe naszego wielomianu to: $16, -13, 18$.\\
Wielomian jest stopnia parzystego, ponadto znak współczynnika przy\linebreak najwyższej potędze x jest ujemny.\\ W związku z tym wykres wielomianu zaczyna się od lewej strony powyżej osi OX.\\
Ponadto w punkcie $-13$ wykres odbija się od osi poziomej.\\
A więc $$x \in \{-13\} \cup [16,18].$$
\rozwStop
\odpStart
$x \in \{-13\} \cup [16,18]$
\odpStop
\testStart
A.$x \in \{-13\} \cup [16,18]$\\
B.$x \in \{13\} \cup (16,18)$\\
C.$x \in \{-13\} \cup (16,18]$\\
D.$x \in \{13\} \cup (16,18]$\\
E.$x \in \{-13\} \cup [16,18)$\\
F.$x \in \{13\} \cup [16,18)$\\
G.$x \in \{-13\} \cup (16,18)$\\
H.$x \in \{13\} \cup [16,18]$
\testStop
\kluczStart
A
\kluczStop



\zadStart{Zadanie z Wikieł Z 1.62 c) moja wersja nr 1031}

Rozwiązać nierówności $(16-x)(x+13)^{2}(19-x)^{3}\le0$.
\zadStop
\rozwStart{Patryk Wirkus}{}
Miejsca zerowe naszego wielomianu to: $16, -13, 19$.\\
Wielomian jest stopnia parzystego, ponadto znak współczynnika przy\linebreak najwyższej potędze x jest ujemny.\\ W związku z tym wykres wielomianu zaczyna się od lewej strony powyżej osi OX.\\
Ponadto w punkcie $-13$ wykres odbija się od osi poziomej.\\
A więc $$x \in \{-13\} \cup [16,19].$$
\rozwStop
\odpStart
$x \in \{-13\} \cup [16,19]$
\odpStop
\testStart
A.$x \in \{-13\} \cup [16,19]$\\
B.$x \in \{13\} \cup (16,19)$\\
C.$x \in \{-13\} \cup (16,19]$\\
D.$x \in \{13\} \cup (16,19]$\\
E.$x \in \{-13\} \cup [16,19)$\\
F.$x \in \{13\} \cup [16,19)$\\
G.$x \in \{-13\} \cup (16,19)$\\
H.$x \in \{13\} \cup [16,19]$
\testStop
\kluczStart
A
\kluczStop



\zadStart{Zadanie z Wikieł Z 1.62 c) moja wersja nr 1032}

Rozwiązać nierówności $(16-x)(x+13)^{2}(20-x)^{3}\le0$.
\zadStop
\rozwStart{Patryk Wirkus}{}
Miejsca zerowe naszego wielomianu to: $16, -13, 20$.\\
Wielomian jest stopnia parzystego, ponadto znak współczynnika przy\linebreak najwyższej potędze x jest ujemny.\\ W związku z tym wykres wielomianu zaczyna się od lewej strony powyżej osi OX.\\
Ponadto w punkcie $-13$ wykres odbija się od osi poziomej.\\
A więc $$x \in \{-13\} \cup [16,20].$$
\rozwStop
\odpStart
$x \in \{-13\} \cup [16,20]$
\odpStop
\testStart
A.$x \in \{-13\} \cup [16,20]$\\
B.$x \in \{13\} \cup (16,20)$\\
C.$x \in \{-13\} \cup (16,20]$\\
D.$x \in \{13\} \cup (16,20]$\\
E.$x \in \{-13\} \cup [16,20)$\\
F.$x \in \{13\} \cup [16,20)$\\
G.$x \in \{-13\} \cup (16,20)$\\
H.$x \in \{13\} \cup [16,20]$
\testStop
\kluczStart
A
\kluczStop



\zadStart{Zadanie z Wikieł Z 1.62 c) moja wersja nr 1033}

Rozwiązać nierówności $(16-x)(x+14)^{2}(17-x)^{3}\le0$.
\zadStop
\rozwStart{Patryk Wirkus}{}
Miejsca zerowe naszego wielomianu to: $16, -14, 17$.\\
Wielomian jest stopnia parzystego, ponadto znak współczynnika przy\linebreak najwyższej potędze x jest ujemny.\\ W związku z tym wykres wielomianu zaczyna się od lewej strony powyżej osi OX.\\
Ponadto w punkcie $-14$ wykres odbija się od osi poziomej.\\
A więc $$x \in \{-14\} \cup [16,17].$$
\rozwStop
\odpStart
$x \in \{-14\} \cup [16,17]$
\odpStop
\testStart
A.$x \in \{-14\} \cup [16,17]$\\
B.$x \in \{14\} \cup (16,17)$\\
C.$x \in \{-14\} \cup (16,17]$\\
D.$x \in \{14\} \cup (16,17]$\\
E.$x \in \{-14\} \cup [16,17)$\\
F.$x \in \{14\} \cup [16,17)$\\
G.$x \in \{-14\} \cup (16,17)$\\
H.$x \in \{14\} \cup [16,17]$
\testStop
\kluczStart
A
\kluczStop



\zadStart{Zadanie z Wikieł Z 1.62 c) moja wersja nr 1034}

Rozwiązać nierówności $(16-x)(x+14)^{2}(18-x)^{3}\le0$.
\zadStop
\rozwStart{Patryk Wirkus}{}
Miejsca zerowe naszego wielomianu to: $16, -14, 18$.\\
Wielomian jest stopnia parzystego, ponadto znak współczynnika przy\linebreak najwyższej potędze x jest ujemny.\\ W związku z tym wykres wielomianu zaczyna się od lewej strony powyżej osi OX.\\
Ponadto w punkcie $-14$ wykres odbija się od osi poziomej.\\
A więc $$x \in \{-14\} \cup [16,18].$$
\rozwStop
\odpStart
$x \in \{-14\} \cup [16,18]$
\odpStop
\testStart
A.$x \in \{-14\} \cup [16,18]$\\
B.$x \in \{14\} \cup (16,18)$\\
C.$x \in \{-14\} \cup (16,18]$\\
D.$x \in \{14\} \cup (16,18]$\\
E.$x \in \{-14\} \cup [16,18)$\\
F.$x \in \{14\} \cup [16,18)$\\
G.$x \in \{-14\} \cup (16,18)$\\
H.$x \in \{14\} \cup [16,18]$
\testStop
\kluczStart
A
\kluczStop



\zadStart{Zadanie z Wikieł Z 1.62 c) moja wersja nr 1035}

Rozwiązać nierówności $(16-x)(x+14)^{2}(19-x)^{3}\le0$.
\zadStop
\rozwStart{Patryk Wirkus}{}
Miejsca zerowe naszego wielomianu to: $16, -14, 19$.\\
Wielomian jest stopnia parzystego, ponadto znak współczynnika przy\linebreak najwyższej potędze x jest ujemny.\\ W związku z tym wykres wielomianu zaczyna się od lewej strony powyżej osi OX.\\
Ponadto w punkcie $-14$ wykres odbija się od osi poziomej.\\
A więc $$x \in \{-14\} \cup [16,19].$$
\rozwStop
\odpStart
$x \in \{-14\} \cup [16,19]$
\odpStop
\testStart
A.$x \in \{-14\} \cup [16,19]$\\
B.$x \in \{14\} \cup (16,19)$\\
C.$x \in \{-14\} \cup (16,19]$\\
D.$x \in \{14\} \cup (16,19]$\\
E.$x \in \{-14\} \cup [16,19)$\\
F.$x \in \{14\} \cup [16,19)$\\
G.$x \in \{-14\} \cup (16,19)$\\
H.$x \in \{14\} \cup [16,19]$
\testStop
\kluczStart
A
\kluczStop



\zadStart{Zadanie z Wikieł Z 1.62 c) moja wersja nr 1036}

Rozwiązać nierówności $(16-x)(x+14)^{2}(20-x)^{3}\le0$.
\zadStop
\rozwStart{Patryk Wirkus}{}
Miejsca zerowe naszego wielomianu to: $16, -14, 20$.\\
Wielomian jest stopnia parzystego, ponadto znak współczynnika przy\linebreak najwyższej potędze x jest ujemny.\\ W związku z tym wykres wielomianu zaczyna się od lewej strony powyżej osi OX.\\
Ponadto w punkcie $-14$ wykres odbija się od osi poziomej.\\
A więc $$x \in \{-14\} \cup [16,20].$$
\rozwStop
\odpStart
$x \in \{-14\} \cup [16,20]$
\odpStop
\testStart
A.$x \in \{-14\} \cup [16,20]$\\
B.$x \in \{14\} \cup (16,20)$\\
C.$x \in \{-14\} \cup (16,20]$\\
D.$x \in \{14\} \cup (16,20]$\\
E.$x \in \{-14\} \cup [16,20)$\\
F.$x \in \{14\} \cup [16,20)$\\
G.$x \in \{-14\} \cup (16,20)$\\
H.$x \in \{14\} \cup [16,20]$
\testStop
\kluczStart
A
\kluczStop



\zadStart{Zadanie z Wikieł Z 1.62 c) moja wersja nr 1037}

Rozwiązać nierówności $(16-x)(x+15)^{2}(17-x)^{3}\le0$.
\zadStop
\rozwStart{Patryk Wirkus}{}
Miejsca zerowe naszego wielomianu to: $16, -15, 17$.\\
Wielomian jest stopnia parzystego, ponadto znak współczynnika przy\linebreak najwyższej potędze x jest ujemny.\\ W związku z tym wykres wielomianu zaczyna się od lewej strony powyżej osi OX.\\
Ponadto w punkcie $-15$ wykres odbija się od osi poziomej.\\
A więc $$x \in \{-15\} \cup [16,17].$$
\rozwStop
\odpStart
$x \in \{-15\} \cup [16,17]$
\odpStop
\testStart
A.$x \in \{-15\} \cup [16,17]$\\
B.$x \in \{15\} \cup (16,17)$\\
C.$x \in \{-15\} \cup (16,17]$\\
D.$x \in \{15\} \cup (16,17]$\\
E.$x \in \{-15\} \cup [16,17)$\\
F.$x \in \{15\} \cup [16,17)$\\
G.$x \in \{-15\} \cup (16,17)$\\
H.$x \in \{15\} \cup [16,17]$
\testStop
\kluczStart
A
\kluczStop



\zadStart{Zadanie z Wikieł Z 1.62 c) moja wersja nr 1038}

Rozwiązać nierówności $(16-x)(x+15)^{2}(18-x)^{3}\le0$.
\zadStop
\rozwStart{Patryk Wirkus}{}
Miejsca zerowe naszego wielomianu to: $16, -15, 18$.\\
Wielomian jest stopnia parzystego, ponadto znak współczynnika przy\linebreak najwyższej potędze x jest ujemny.\\ W związku z tym wykres wielomianu zaczyna się od lewej strony powyżej osi OX.\\
Ponadto w punkcie $-15$ wykres odbija się od osi poziomej.\\
A więc $$x \in \{-15\} \cup [16,18].$$
\rozwStop
\odpStart
$x \in \{-15\} \cup [16,18]$
\odpStop
\testStart
A.$x \in \{-15\} \cup [16,18]$\\
B.$x \in \{15\} \cup (16,18)$\\
C.$x \in \{-15\} \cup (16,18]$\\
D.$x \in \{15\} \cup (16,18]$\\
E.$x \in \{-15\} \cup [16,18)$\\
F.$x \in \{15\} \cup [16,18)$\\
G.$x \in \{-15\} \cup (16,18)$\\
H.$x \in \{15\} \cup [16,18]$
\testStop
\kluczStart
A
\kluczStop



\zadStart{Zadanie z Wikieł Z 1.62 c) moja wersja nr 1039}

Rozwiązać nierówności $(16-x)(x+15)^{2}(19-x)^{3}\le0$.
\zadStop
\rozwStart{Patryk Wirkus}{}
Miejsca zerowe naszego wielomianu to: $16, -15, 19$.\\
Wielomian jest stopnia parzystego, ponadto znak współczynnika przy\linebreak najwyższej potędze x jest ujemny.\\ W związku z tym wykres wielomianu zaczyna się od lewej strony powyżej osi OX.\\
Ponadto w punkcie $-15$ wykres odbija się od osi poziomej.\\
A więc $$x \in \{-15\} \cup [16,19].$$
\rozwStop
\odpStart
$x \in \{-15\} \cup [16,19]$
\odpStop
\testStart
A.$x \in \{-15\} \cup [16,19]$\\
B.$x \in \{15\} \cup (16,19)$\\
C.$x \in \{-15\} \cup (16,19]$\\
D.$x \in \{15\} \cup (16,19]$\\
E.$x \in \{-15\} \cup [16,19)$\\
F.$x \in \{15\} \cup [16,19)$\\
G.$x \in \{-15\} \cup (16,19)$\\
H.$x \in \{15\} \cup [16,19]$
\testStop
\kluczStart
A
\kluczStop



\zadStart{Zadanie z Wikieł Z 1.62 c) moja wersja nr 1040}

Rozwiązać nierówności $(16-x)(x+15)^{2}(20-x)^{3}\le0$.
\zadStop
\rozwStart{Patryk Wirkus}{}
Miejsca zerowe naszego wielomianu to: $16, -15, 20$.\\
Wielomian jest stopnia parzystego, ponadto znak współczynnika przy\linebreak najwyższej potędze x jest ujemny.\\ W związku z tym wykres wielomianu zaczyna się od lewej strony powyżej osi OX.\\
Ponadto w punkcie $-15$ wykres odbija się od osi poziomej.\\
A więc $$x \in \{-15\} \cup [16,20].$$
\rozwStop
\odpStart
$x \in \{-15\} \cup [16,20]$
\odpStop
\testStart
A.$x \in \{-15\} \cup [16,20]$\\
B.$x \in \{15\} \cup (16,20)$\\
C.$x \in \{-15\} \cup (16,20]$\\
D.$x \in \{15\} \cup (16,20]$\\
E.$x \in \{-15\} \cup [16,20)$\\
F.$x \in \{15\} \cup [16,20)$\\
G.$x \in \{-15\} \cup (16,20)$\\
H.$x \in \{15\} \cup [16,20]$
\testStop
\kluczStart
A
\kluczStop



\zadStart{Zadanie z Wikieł Z 1.62 c) moja wersja nr 1041}

Rozwiązać nierówności $(17-x)(x+1)^{2}(18-x)^{3}\le0$.
\zadStop
\rozwStart{Patryk Wirkus}{}
Miejsca zerowe naszego wielomianu to: $17, -1, 18$.\\
Wielomian jest stopnia parzystego, ponadto znak współczynnika przy\linebreak najwyższej potędze x jest ujemny.\\ W związku z tym wykres wielomianu zaczyna się od lewej strony powyżej osi OX.\\
Ponadto w punkcie $-1$ wykres odbija się od osi poziomej.\\
A więc $$x \in \{-1\} \cup [17,18].$$
\rozwStop
\odpStart
$x \in \{-1\} \cup [17,18]$
\odpStop
\testStart
A.$x \in \{-1\} \cup [17,18]$\\
B.$x \in \{1\} \cup (17,18)$\\
C.$x \in \{-1\} \cup (17,18]$\\
D.$x \in \{1\} \cup (17,18]$\\
E.$x \in \{-1\} \cup [17,18)$\\
F.$x \in \{1\} \cup [17,18)$\\
G.$x \in \{-1\} \cup (17,18)$\\
H.$x \in \{1\} \cup [17,18]$
\testStop
\kluczStart
A
\kluczStop



\zadStart{Zadanie z Wikieł Z 1.62 c) moja wersja nr 1042}

Rozwiązać nierówności $(17-x)(x+1)^{2}(19-x)^{3}\le0$.
\zadStop
\rozwStart{Patryk Wirkus}{}
Miejsca zerowe naszego wielomianu to: $17, -1, 19$.\\
Wielomian jest stopnia parzystego, ponadto znak współczynnika przy\linebreak najwyższej potędze x jest ujemny.\\ W związku z tym wykres wielomianu zaczyna się od lewej strony powyżej osi OX.\\
Ponadto w punkcie $-1$ wykres odbija się od osi poziomej.\\
A więc $$x \in \{-1\} \cup [17,19].$$
\rozwStop
\odpStart
$x \in \{-1\} \cup [17,19]$
\odpStop
\testStart
A.$x \in \{-1\} \cup [17,19]$\\
B.$x \in \{1\} \cup (17,19)$\\
C.$x \in \{-1\} \cup (17,19]$\\
D.$x \in \{1\} \cup (17,19]$\\
E.$x \in \{-1\} \cup [17,19)$\\
F.$x \in \{1\} \cup [17,19)$\\
G.$x \in \{-1\} \cup (17,19)$\\
H.$x \in \{1\} \cup [17,19]$
\testStop
\kluczStart
A
\kluczStop



\zadStart{Zadanie z Wikieł Z 1.62 c) moja wersja nr 1043}

Rozwiązać nierówności $(17-x)(x+1)^{2}(20-x)^{3}\le0$.
\zadStop
\rozwStart{Patryk Wirkus}{}
Miejsca zerowe naszego wielomianu to: $17, -1, 20$.\\
Wielomian jest stopnia parzystego, ponadto znak współczynnika przy\linebreak najwyższej potędze x jest ujemny.\\ W związku z tym wykres wielomianu zaczyna się od lewej strony powyżej osi OX.\\
Ponadto w punkcie $-1$ wykres odbija się od osi poziomej.\\
A więc $$x \in \{-1\} \cup [17,20].$$
\rozwStop
\odpStart
$x \in \{-1\} \cup [17,20]$
\odpStop
\testStart
A.$x \in \{-1\} \cup [17,20]$\\
B.$x \in \{1\} \cup (17,20)$\\
C.$x \in \{-1\} \cup (17,20]$\\
D.$x \in \{1\} \cup (17,20]$\\
E.$x \in \{-1\} \cup [17,20)$\\
F.$x \in \{1\} \cup [17,20)$\\
G.$x \in \{-1\} \cup (17,20)$\\
H.$x \in \{1\} \cup [17,20]$
\testStop
\kluczStart
A
\kluczStop



\zadStart{Zadanie z Wikieł Z 1.62 c) moja wersja nr 1044}

Rozwiązać nierówności $(17-x)(x+2)^{2}(18-x)^{3}\le0$.
\zadStop
\rozwStart{Patryk Wirkus}{}
Miejsca zerowe naszego wielomianu to: $17, -2, 18$.\\
Wielomian jest stopnia parzystego, ponadto znak współczynnika przy\linebreak najwyższej potędze x jest ujemny.\\ W związku z tym wykres wielomianu zaczyna się od lewej strony powyżej osi OX.\\
Ponadto w punkcie $-2$ wykres odbija się od osi poziomej.\\
A więc $$x \in \{-2\} \cup [17,18].$$
\rozwStop
\odpStart
$x \in \{-2\} \cup [17,18]$
\odpStop
\testStart
A.$x \in \{-2\} \cup [17,18]$\\
B.$x \in \{2\} \cup (17,18)$\\
C.$x \in \{-2\} \cup (17,18]$\\
D.$x \in \{2\} \cup (17,18]$\\
E.$x \in \{-2\} \cup [17,18)$\\
F.$x \in \{2\} \cup [17,18)$\\
G.$x \in \{-2\} \cup (17,18)$\\
H.$x \in \{2\} \cup [17,18]$
\testStop
\kluczStart
A
\kluczStop



\zadStart{Zadanie z Wikieł Z 1.62 c) moja wersja nr 1045}

Rozwiązać nierówności $(17-x)(x+2)^{2}(19-x)^{3}\le0$.
\zadStop
\rozwStart{Patryk Wirkus}{}
Miejsca zerowe naszego wielomianu to: $17, -2, 19$.\\
Wielomian jest stopnia parzystego, ponadto znak współczynnika przy\linebreak najwyższej potędze x jest ujemny.\\ W związku z tym wykres wielomianu zaczyna się od lewej strony powyżej osi OX.\\
Ponadto w punkcie $-2$ wykres odbija się od osi poziomej.\\
A więc $$x \in \{-2\} \cup [17,19].$$
\rozwStop
\odpStart
$x \in \{-2\} \cup [17,19]$
\odpStop
\testStart
A.$x \in \{-2\} \cup [17,19]$\\
B.$x \in \{2\} \cup (17,19)$\\
C.$x \in \{-2\} \cup (17,19]$\\
D.$x \in \{2\} \cup (17,19]$\\
E.$x \in \{-2\} \cup [17,19)$\\
F.$x \in \{2\} \cup [17,19)$\\
G.$x \in \{-2\} \cup (17,19)$\\
H.$x \in \{2\} \cup [17,19]$
\testStop
\kluczStart
A
\kluczStop



\zadStart{Zadanie z Wikieł Z 1.62 c) moja wersja nr 1046}

Rozwiązać nierówności $(17-x)(x+2)^{2}(20-x)^{3}\le0$.
\zadStop
\rozwStart{Patryk Wirkus}{}
Miejsca zerowe naszego wielomianu to: $17, -2, 20$.\\
Wielomian jest stopnia parzystego, ponadto znak współczynnika przy\linebreak najwyższej potędze x jest ujemny.\\ W związku z tym wykres wielomianu zaczyna się od lewej strony powyżej osi OX.\\
Ponadto w punkcie $-2$ wykres odbija się od osi poziomej.\\
A więc $$x \in \{-2\} \cup [17,20].$$
\rozwStop
\odpStart
$x \in \{-2\} \cup [17,20]$
\odpStop
\testStart
A.$x \in \{-2\} \cup [17,20]$\\
B.$x \in \{2\} \cup (17,20)$\\
C.$x \in \{-2\} \cup (17,20]$\\
D.$x \in \{2\} \cup (17,20]$\\
E.$x \in \{-2\} \cup [17,20)$\\
F.$x \in \{2\} \cup [17,20)$\\
G.$x \in \{-2\} \cup (17,20)$\\
H.$x \in \{2\} \cup [17,20]$
\testStop
\kluczStart
A
\kluczStop



\zadStart{Zadanie z Wikieł Z 1.62 c) moja wersja nr 1047}

Rozwiązać nierówności $(17-x)(x+3)^{2}(18-x)^{3}\le0$.
\zadStop
\rozwStart{Patryk Wirkus}{}
Miejsca zerowe naszego wielomianu to: $17, -3, 18$.\\
Wielomian jest stopnia parzystego, ponadto znak współczynnika przy\linebreak najwyższej potędze x jest ujemny.\\ W związku z tym wykres wielomianu zaczyna się od lewej strony powyżej osi OX.\\
Ponadto w punkcie $-3$ wykres odbija się od osi poziomej.\\
A więc $$x \in \{-3\} \cup [17,18].$$
\rozwStop
\odpStart
$x \in \{-3\} \cup [17,18]$
\odpStop
\testStart
A.$x \in \{-3\} \cup [17,18]$\\
B.$x \in \{3\} \cup (17,18)$\\
C.$x \in \{-3\} \cup (17,18]$\\
D.$x \in \{3\} \cup (17,18]$\\
E.$x \in \{-3\} \cup [17,18)$\\
F.$x \in \{3\} \cup [17,18)$\\
G.$x \in \{-3\} \cup (17,18)$\\
H.$x \in \{3\} \cup [17,18]$
\testStop
\kluczStart
A
\kluczStop



\zadStart{Zadanie z Wikieł Z 1.62 c) moja wersja nr 1048}

Rozwiązać nierówności $(17-x)(x+3)^{2}(19-x)^{3}\le0$.
\zadStop
\rozwStart{Patryk Wirkus}{}
Miejsca zerowe naszego wielomianu to: $17, -3, 19$.\\
Wielomian jest stopnia parzystego, ponadto znak współczynnika przy\linebreak najwyższej potędze x jest ujemny.\\ W związku z tym wykres wielomianu zaczyna się od lewej strony powyżej osi OX.\\
Ponadto w punkcie $-3$ wykres odbija się od osi poziomej.\\
A więc $$x \in \{-3\} \cup [17,19].$$
\rozwStop
\odpStart
$x \in \{-3\} \cup [17,19]$
\odpStop
\testStart
A.$x \in \{-3\} \cup [17,19]$\\
B.$x \in \{3\} \cup (17,19)$\\
C.$x \in \{-3\} \cup (17,19]$\\
D.$x \in \{3\} \cup (17,19]$\\
E.$x \in \{-3\} \cup [17,19)$\\
F.$x \in \{3\} \cup [17,19)$\\
G.$x \in \{-3\} \cup (17,19)$\\
H.$x \in \{3\} \cup [17,19]$
\testStop
\kluczStart
A
\kluczStop



\zadStart{Zadanie z Wikieł Z 1.62 c) moja wersja nr 1049}

Rozwiązać nierówności $(17-x)(x+3)^{2}(20-x)^{3}\le0$.
\zadStop
\rozwStart{Patryk Wirkus}{}
Miejsca zerowe naszego wielomianu to: $17, -3, 20$.\\
Wielomian jest stopnia parzystego, ponadto znak współczynnika przy\linebreak najwyższej potędze x jest ujemny.\\ W związku z tym wykres wielomianu zaczyna się od lewej strony powyżej osi OX.\\
Ponadto w punkcie $-3$ wykres odbija się od osi poziomej.\\
A więc $$x \in \{-3\} \cup [17,20].$$
\rozwStop
\odpStart
$x \in \{-3\} \cup [17,20]$
\odpStop
\testStart
A.$x \in \{-3\} \cup [17,20]$\\
B.$x \in \{3\} \cup (17,20)$\\
C.$x \in \{-3\} \cup (17,20]$\\
D.$x \in \{3\} \cup (17,20]$\\
E.$x \in \{-3\} \cup [17,20)$\\
F.$x \in \{3\} \cup [17,20)$\\
G.$x \in \{-3\} \cup (17,20)$\\
H.$x \in \{3\} \cup [17,20]$
\testStop
\kluczStart
A
\kluczStop



\zadStart{Zadanie z Wikieł Z 1.62 c) moja wersja nr 1050}

Rozwiązać nierówności $(17-x)(x+4)^{2}(18-x)^{3}\le0$.
\zadStop
\rozwStart{Patryk Wirkus}{}
Miejsca zerowe naszego wielomianu to: $17, -4, 18$.\\
Wielomian jest stopnia parzystego, ponadto znak współczynnika przy\linebreak najwyższej potędze x jest ujemny.\\ W związku z tym wykres wielomianu zaczyna się od lewej strony powyżej osi OX.\\
Ponadto w punkcie $-4$ wykres odbija się od osi poziomej.\\
A więc $$x \in \{-4\} \cup [17,18].$$
\rozwStop
\odpStart
$x \in \{-4\} \cup [17,18]$
\odpStop
\testStart
A.$x \in \{-4\} \cup [17,18]$\\
B.$x \in \{4\} \cup (17,18)$\\
C.$x \in \{-4\} \cup (17,18]$\\
D.$x \in \{4\} \cup (17,18]$\\
E.$x \in \{-4\} \cup [17,18)$\\
F.$x \in \{4\} \cup [17,18)$\\
G.$x \in \{-4\} \cup (17,18)$\\
H.$x \in \{4\} \cup [17,18]$
\testStop
\kluczStart
A
\kluczStop



\zadStart{Zadanie z Wikieł Z 1.62 c) moja wersja nr 1051}

Rozwiązać nierówności $(17-x)(x+4)^{2}(19-x)^{3}\le0$.
\zadStop
\rozwStart{Patryk Wirkus}{}
Miejsca zerowe naszego wielomianu to: $17, -4, 19$.\\
Wielomian jest stopnia parzystego, ponadto znak współczynnika przy\linebreak najwyższej potędze x jest ujemny.\\ W związku z tym wykres wielomianu zaczyna się od lewej strony powyżej osi OX.\\
Ponadto w punkcie $-4$ wykres odbija się od osi poziomej.\\
A więc $$x \in \{-4\} \cup [17,19].$$
\rozwStop
\odpStart
$x \in \{-4\} \cup [17,19]$
\odpStop
\testStart
A.$x \in \{-4\} \cup [17,19]$\\
B.$x \in \{4\} \cup (17,19)$\\
C.$x \in \{-4\} \cup (17,19]$\\
D.$x \in \{4\} \cup (17,19]$\\
E.$x \in \{-4\} \cup [17,19)$\\
F.$x \in \{4\} \cup [17,19)$\\
G.$x \in \{-4\} \cup (17,19)$\\
H.$x \in \{4\} \cup [17,19]$
\testStop
\kluczStart
A
\kluczStop



\zadStart{Zadanie z Wikieł Z 1.62 c) moja wersja nr 1052}

Rozwiązać nierówności $(17-x)(x+4)^{2}(20-x)^{3}\le0$.
\zadStop
\rozwStart{Patryk Wirkus}{}
Miejsca zerowe naszego wielomianu to: $17, -4, 20$.\\
Wielomian jest stopnia parzystego, ponadto znak współczynnika przy\linebreak najwyższej potędze x jest ujemny.\\ W związku z tym wykres wielomianu zaczyna się od lewej strony powyżej osi OX.\\
Ponadto w punkcie $-4$ wykres odbija się od osi poziomej.\\
A więc $$x \in \{-4\} \cup [17,20].$$
\rozwStop
\odpStart
$x \in \{-4\} \cup [17,20]$
\odpStop
\testStart
A.$x \in \{-4\} \cup [17,20]$\\
B.$x \in \{4\} \cup (17,20)$\\
C.$x \in \{-4\} \cup (17,20]$\\
D.$x \in \{4\} \cup (17,20]$\\
E.$x \in \{-4\} \cup [17,20)$\\
F.$x \in \{4\} \cup [17,20)$\\
G.$x \in \{-4\} \cup (17,20)$\\
H.$x \in \{4\} \cup [17,20]$
\testStop
\kluczStart
A
\kluczStop



\zadStart{Zadanie z Wikieł Z 1.62 c) moja wersja nr 1053}

Rozwiązać nierówności $(17-x)(x+5)^{2}(18-x)^{3}\le0$.
\zadStop
\rozwStart{Patryk Wirkus}{}
Miejsca zerowe naszego wielomianu to: $17, -5, 18$.\\
Wielomian jest stopnia parzystego, ponadto znak współczynnika przy\linebreak najwyższej potędze x jest ujemny.\\ W związku z tym wykres wielomianu zaczyna się od lewej strony powyżej osi OX.\\
Ponadto w punkcie $-5$ wykres odbija się od osi poziomej.\\
A więc $$x \in \{-5\} \cup [17,18].$$
\rozwStop
\odpStart
$x \in \{-5\} \cup [17,18]$
\odpStop
\testStart
A.$x \in \{-5\} \cup [17,18]$\\
B.$x \in \{5\} \cup (17,18)$\\
C.$x \in \{-5\} \cup (17,18]$\\
D.$x \in \{5\} \cup (17,18]$\\
E.$x \in \{-5\} \cup [17,18)$\\
F.$x \in \{5\} \cup [17,18)$\\
G.$x \in \{-5\} \cup (17,18)$\\
H.$x \in \{5\} \cup [17,18]$
\testStop
\kluczStart
A
\kluczStop



\zadStart{Zadanie z Wikieł Z 1.62 c) moja wersja nr 1054}

Rozwiązać nierówności $(17-x)(x+5)^{2}(19-x)^{3}\le0$.
\zadStop
\rozwStart{Patryk Wirkus}{}
Miejsca zerowe naszego wielomianu to: $17, -5, 19$.\\
Wielomian jest stopnia parzystego, ponadto znak współczynnika przy\linebreak najwyższej potędze x jest ujemny.\\ W związku z tym wykres wielomianu zaczyna się od lewej strony powyżej osi OX.\\
Ponadto w punkcie $-5$ wykres odbija się od osi poziomej.\\
A więc $$x \in \{-5\} \cup [17,19].$$
\rozwStop
\odpStart
$x \in \{-5\} \cup [17,19]$
\odpStop
\testStart
A.$x \in \{-5\} \cup [17,19]$\\
B.$x \in \{5\} \cup (17,19)$\\
C.$x \in \{-5\} \cup (17,19]$\\
D.$x \in \{5\} \cup (17,19]$\\
E.$x \in \{-5\} \cup [17,19)$\\
F.$x \in \{5\} \cup [17,19)$\\
G.$x \in \{-5\} \cup (17,19)$\\
H.$x \in \{5\} \cup [17,19]$
\testStop
\kluczStart
A
\kluczStop



\zadStart{Zadanie z Wikieł Z 1.62 c) moja wersja nr 1055}

Rozwiązać nierówności $(17-x)(x+5)^{2}(20-x)^{3}\le0$.
\zadStop
\rozwStart{Patryk Wirkus}{}
Miejsca zerowe naszego wielomianu to: $17, -5, 20$.\\
Wielomian jest stopnia parzystego, ponadto znak współczynnika przy\linebreak najwyższej potędze x jest ujemny.\\ W związku z tym wykres wielomianu zaczyna się od lewej strony powyżej osi OX.\\
Ponadto w punkcie $-5$ wykres odbija się od osi poziomej.\\
A więc $$x \in \{-5\} \cup [17,20].$$
\rozwStop
\odpStart
$x \in \{-5\} \cup [17,20]$
\odpStop
\testStart
A.$x \in \{-5\} \cup [17,20]$\\
B.$x \in \{5\} \cup (17,20)$\\
C.$x \in \{-5\} \cup (17,20]$\\
D.$x \in \{5\} \cup (17,20]$\\
E.$x \in \{-5\} \cup [17,20)$\\
F.$x \in \{5\} \cup [17,20)$\\
G.$x \in \{-5\} \cup (17,20)$\\
H.$x \in \{5\} \cup [17,20]$
\testStop
\kluczStart
A
\kluczStop



\zadStart{Zadanie z Wikieł Z 1.62 c) moja wersja nr 1056}

Rozwiązać nierówności $(17-x)(x+6)^{2}(18-x)^{3}\le0$.
\zadStop
\rozwStart{Patryk Wirkus}{}
Miejsca zerowe naszego wielomianu to: $17, -6, 18$.\\
Wielomian jest stopnia parzystego, ponadto znak współczynnika przy\linebreak najwyższej potędze x jest ujemny.\\ W związku z tym wykres wielomianu zaczyna się od lewej strony powyżej osi OX.\\
Ponadto w punkcie $-6$ wykres odbija się od osi poziomej.\\
A więc $$x \in \{-6\} \cup [17,18].$$
\rozwStop
\odpStart
$x \in \{-6\} \cup [17,18]$
\odpStop
\testStart
A.$x \in \{-6\} \cup [17,18]$\\
B.$x \in \{6\} \cup (17,18)$\\
C.$x \in \{-6\} \cup (17,18]$\\
D.$x \in \{6\} \cup (17,18]$\\
E.$x \in \{-6\} \cup [17,18)$\\
F.$x \in \{6\} \cup [17,18)$\\
G.$x \in \{-6\} \cup (17,18)$\\
H.$x \in \{6\} \cup [17,18]$
\testStop
\kluczStart
A
\kluczStop



\zadStart{Zadanie z Wikieł Z 1.62 c) moja wersja nr 1057}

Rozwiązać nierówności $(17-x)(x+6)^{2}(19-x)^{3}\le0$.
\zadStop
\rozwStart{Patryk Wirkus}{}
Miejsca zerowe naszego wielomianu to: $17, -6, 19$.\\
Wielomian jest stopnia parzystego, ponadto znak współczynnika przy\linebreak najwyższej potędze x jest ujemny.\\ W związku z tym wykres wielomianu zaczyna się od lewej strony powyżej osi OX.\\
Ponadto w punkcie $-6$ wykres odbija się od osi poziomej.\\
A więc $$x \in \{-6\} \cup [17,19].$$
\rozwStop
\odpStart
$x \in \{-6\} \cup [17,19]$
\odpStop
\testStart
A.$x \in \{-6\} \cup [17,19]$\\
B.$x \in \{6\} \cup (17,19)$\\
C.$x \in \{-6\} \cup (17,19]$\\
D.$x \in \{6\} \cup (17,19]$\\
E.$x \in \{-6\} \cup [17,19)$\\
F.$x \in \{6\} \cup [17,19)$\\
G.$x \in \{-6\} \cup (17,19)$\\
H.$x \in \{6\} \cup [17,19]$
\testStop
\kluczStart
A
\kluczStop



\zadStart{Zadanie z Wikieł Z 1.62 c) moja wersja nr 1058}

Rozwiązać nierówności $(17-x)(x+6)^{2}(20-x)^{3}\le0$.
\zadStop
\rozwStart{Patryk Wirkus}{}
Miejsca zerowe naszego wielomianu to: $17, -6, 20$.\\
Wielomian jest stopnia parzystego, ponadto znak współczynnika przy\linebreak najwyższej potędze x jest ujemny.\\ W związku z tym wykres wielomianu zaczyna się od lewej strony powyżej osi OX.\\
Ponadto w punkcie $-6$ wykres odbija się od osi poziomej.\\
A więc $$x \in \{-6\} \cup [17,20].$$
\rozwStop
\odpStart
$x \in \{-6\} \cup [17,20]$
\odpStop
\testStart
A.$x \in \{-6\} \cup [17,20]$\\
B.$x \in \{6\} \cup (17,20)$\\
C.$x \in \{-6\} \cup (17,20]$\\
D.$x \in \{6\} \cup (17,20]$\\
E.$x \in \{-6\} \cup [17,20)$\\
F.$x \in \{6\} \cup [17,20)$\\
G.$x \in \{-6\} \cup (17,20)$\\
H.$x \in \{6\} \cup [17,20]$
\testStop
\kluczStart
A
\kluczStop



\zadStart{Zadanie z Wikieł Z 1.62 c) moja wersja nr 1059}

Rozwiązać nierówności $(17-x)(x+7)^{2}(18-x)^{3}\le0$.
\zadStop
\rozwStart{Patryk Wirkus}{}
Miejsca zerowe naszego wielomianu to: $17, -7, 18$.\\
Wielomian jest stopnia parzystego, ponadto znak współczynnika przy\linebreak najwyższej potędze x jest ujemny.\\ W związku z tym wykres wielomianu zaczyna się od lewej strony powyżej osi OX.\\
Ponadto w punkcie $-7$ wykres odbija się od osi poziomej.\\
A więc $$x \in \{-7\} \cup [17,18].$$
\rozwStop
\odpStart
$x \in \{-7\} \cup [17,18]$
\odpStop
\testStart
A.$x \in \{-7\} \cup [17,18]$\\
B.$x \in \{7\} \cup (17,18)$\\
C.$x \in \{-7\} \cup (17,18]$\\
D.$x \in \{7\} \cup (17,18]$\\
E.$x \in \{-7\} \cup [17,18)$\\
F.$x \in \{7\} \cup [17,18)$\\
G.$x \in \{-7\} \cup (17,18)$\\
H.$x \in \{7\} \cup [17,18]$
\testStop
\kluczStart
A
\kluczStop



\zadStart{Zadanie z Wikieł Z 1.62 c) moja wersja nr 1060}

Rozwiązać nierówności $(17-x)(x+7)^{2}(19-x)^{3}\le0$.
\zadStop
\rozwStart{Patryk Wirkus}{}
Miejsca zerowe naszego wielomianu to: $17, -7, 19$.\\
Wielomian jest stopnia parzystego, ponadto znak współczynnika przy\linebreak najwyższej potędze x jest ujemny.\\ W związku z tym wykres wielomianu zaczyna się od lewej strony powyżej osi OX.\\
Ponadto w punkcie $-7$ wykres odbija się od osi poziomej.\\
A więc $$x \in \{-7\} \cup [17,19].$$
\rozwStop
\odpStart
$x \in \{-7\} \cup [17,19]$
\odpStop
\testStart
A.$x \in \{-7\} \cup [17,19]$\\
B.$x \in \{7\} \cup (17,19)$\\
C.$x \in \{-7\} \cup (17,19]$\\
D.$x \in \{7\} \cup (17,19]$\\
E.$x \in \{-7\} \cup [17,19)$\\
F.$x \in \{7\} \cup [17,19)$\\
G.$x \in \{-7\} \cup (17,19)$\\
H.$x \in \{7\} \cup [17,19]$
\testStop
\kluczStart
A
\kluczStop



\zadStart{Zadanie z Wikieł Z 1.62 c) moja wersja nr 1061}

Rozwiązać nierówności $(17-x)(x+7)^{2}(20-x)^{3}\le0$.
\zadStop
\rozwStart{Patryk Wirkus}{}
Miejsca zerowe naszego wielomianu to: $17, -7, 20$.\\
Wielomian jest stopnia parzystego, ponadto znak współczynnika przy\linebreak najwyższej potędze x jest ujemny.\\ W związku z tym wykres wielomianu zaczyna się od lewej strony powyżej osi OX.\\
Ponadto w punkcie $-7$ wykres odbija się od osi poziomej.\\
A więc $$x \in \{-7\} \cup [17,20].$$
\rozwStop
\odpStart
$x \in \{-7\} \cup [17,20]$
\odpStop
\testStart
A.$x \in \{-7\} \cup [17,20]$\\
B.$x \in \{7\} \cup (17,20)$\\
C.$x \in \{-7\} \cup (17,20]$\\
D.$x \in \{7\} \cup (17,20]$\\
E.$x \in \{-7\} \cup [17,20)$\\
F.$x \in \{7\} \cup [17,20)$\\
G.$x \in \{-7\} \cup (17,20)$\\
H.$x \in \{7\} \cup [17,20]$
\testStop
\kluczStart
A
\kluczStop



\zadStart{Zadanie z Wikieł Z 1.62 c) moja wersja nr 1062}

Rozwiązać nierówności $(17-x)(x+8)^{2}(18-x)^{3}\le0$.
\zadStop
\rozwStart{Patryk Wirkus}{}
Miejsca zerowe naszego wielomianu to: $17, -8, 18$.\\
Wielomian jest stopnia parzystego, ponadto znak współczynnika przy\linebreak najwyższej potędze x jest ujemny.\\ W związku z tym wykres wielomianu zaczyna się od lewej strony powyżej osi OX.\\
Ponadto w punkcie $-8$ wykres odbija się od osi poziomej.\\
A więc $$x \in \{-8\} \cup [17,18].$$
\rozwStop
\odpStart
$x \in \{-8\} \cup [17,18]$
\odpStop
\testStart
A.$x \in \{-8\} \cup [17,18]$\\
B.$x \in \{8\} \cup (17,18)$\\
C.$x \in \{-8\} \cup (17,18]$\\
D.$x \in \{8\} \cup (17,18]$\\
E.$x \in \{-8\} \cup [17,18)$\\
F.$x \in \{8\} \cup [17,18)$\\
G.$x \in \{-8\} \cup (17,18)$\\
H.$x \in \{8\} \cup [17,18]$
\testStop
\kluczStart
A
\kluczStop



\zadStart{Zadanie z Wikieł Z 1.62 c) moja wersja nr 1063}

Rozwiązać nierówności $(17-x)(x+8)^{2}(19-x)^{3}\le0$.
\zadStop
\rozwStart{Patryk Wirkus}{}
Miejsca zerowe naszego wielomianu to: $17, -8, 19$.\\
Wielomian jest stopnia parzystego, ponadto znak współczynnika przy\linebreak najwyższej potędze x jest ujemny.\\ W związku z tym wykres wielomianu zaczyna się od lewej strony powyżej osi OX.\\
Ponadto w punkcie $-8$ wykres odbija się od osi poziomej.\\
A więc $$x \in \{-8\} \cup [17,19].$$
\rozwStop
\odpStart
$x \in \{-8\} \cup [17,19]$
\odpStop
\testStart
A.$x \in \{-8\} \cup [17,19]$\\
B.$x \in \{8\} \cup (17,19)$\\
C.$x \in \{-8\} \cup (17,19]$\\
D.$x \in \{8\} \cup (17,19]$\\
E.$x \in \{-8\} \cup [17,19)$\\
F.$x \in \{8\} \cup [17,19)$\\
G.$x \in \{-8\} \cup (17,19)$\\
H.$x \in \{8\} \cup [17,19]$
\testStop
\kluczStart
A
\kluczStop



\zadStart{Zadanie z Wikieł Z 1.62 c) moja wersja nr 1064}

Rozwiązać nierówności $(17-x)(x+8)^{2}(20-x)^{3}\le0$.
\zadStop
\rozwStart{Patryk Wirkus}{}
Miejsca zerowe naszego wielomianu to: $17, -8, 20$.\\
Wielomian jest stopnia parzystego, ponadto znak współczynnika przy\linebreak najwyższej potędze x jest ujemny.\\ W związku z tym wykres wielomianu zaczyna się od lewej strony powyżej osi OX.\\
Ponadto w punkcie $-8$ wykres odbija się od osi poziomej.\\
A więc $$x \in \{-8\} \cup [17,20].$$
\rozwStop
\odpStart
$x \in \{-8\} \cup [17,20]$
\odpStop
\testStart
A.$x \in \{-8\} \cup [17,20]$\\
B.$x \in \{8\} \cup (17,20)$\\
C.$x \in \{-8\} \cup (17,20]$\\
D.$x \in \{8\} \cup (17,20]$\\
E.$x \in \{-8\} \cup [17,20)$\\
F.$x \in \{8\} \cup [17,20)$\\
G.$x \in \{-8\} \cup (17,20)$\\
H.$x \in \{8\} \cup [17,20]$
\testStop
\kluczStart
A
\kluczStop



\zadStart{Zadanie z Wikieł Z 1.62 c) moja wersja nr 1065}

Rozwiązać nierówności $(17-x)(x+9)^{2}(18-x)^{3}\le0$.
\zadStop
\rozwStart{Patryk Wirkus}{}
Miejsca zerowe naszego wielomianu to: $17, -9, 18$.\\
Wielomian jest stopnia parzystego, ponadto znak współczynnika przy\linebreak najwyższej potędze x jest ujemny.\\ W związku z tym wykres wielomianu zaczyna się od lewej strony powyżej osi OX.\\
Ponadto w punkcie $-9$ wykres odbija się od osi poziomej.\\
A więc $$x \in \{-9\} \cup [17,18].$$
\rozwStop
\odpStart
$x \in \{-9\} \cup [17,18]$
\odpStop
\testStart
A.$x \in \{-9\} \cup [17,18]$\\
B.$x \in \{9\} \cup (17,18)$\\
C.$x \in \{-9\} \cup (17,18]$\\
D.$x \in \{9\} \cup (17,18]$\\
E.$x \in \{-9\} \cup [17,18)$\\
F.$x \in \{9\} \cup [17,18)$\\
G.$x \in \{-9\} \cup (17,18)$\\
H.$x \in \{9\} \cup [17,18]$
\testStop
\kluczStart
A
\kluczStop



\zadStart{Zadanie z Wikieł Z 1.62 c) moja wersja nr 1066}

Rozwiązać nierówności $(17-x)(x+9)^{2}(19-x)^{3}\le0$.
\zadStop
\rozwStart{Patryk Wirkus}{}
Miejsca zerowe naszego wielomianu to: $17, -9, 19$.\\
Wielomian jest stopnia parzystego, ponadto znak współczynnika przy\linebreak najwyższej potędze x jest ujemny.\\ W związku z tym wykres wielomianu zaczyna się od lewej strony powyżej osi OX.\\
Ponadto w punkcie $-9$ wykres odbija się od osi poziomej.\\
A więc $$x \in \{-9\} \cup [17,19].$$
\rozwStop
\odpStart
$x \in \{-9\} \cup [17,19]$
\odpStop
\testStart
A.$x \in \{-9\} \cup [17,19]$\\
B.$x \in \{9\} \cup (17,19)$\\
C.$x \in \{-9\} \cup (17,19]$\\
D.$x \in \{9\} \cup (17,19]$\\
E.$x \in \{-9\} \cup [17,19)$\\
F.$x \in \{9\} \cup [17,19)$\\
G.$x \in \{-9\} \cup (17,19)$\\
H.$x \in \{9\} \cup [17,19]$
\testStop
\kluczStart
A
\kluczStop



\zadStart{Zadanie z Wikieł Z 1.62 c) moja wersja nr 1067}

Rozwiązać nierówności $(17-x)(x+9)^{2}(20-x)^{3}\le0$.
\zadStop
\rozwStart{Patryk Wirkus}{}
Miejsca zerowe naszego wielomianu to: $17, -9, 20$.\\
Wielomian jest stopnia parzystego, ponadto znak współczynnika przy\linebreak najwyższej potędze x jest ujemny.\\ W związku z tym wykres wielomianu zaczyna się od lewej strony powyżej osi OX.\\
Ponadto w punkcie $-9$ wykres odbija się od osi poziomej.\\
A więc $$x \in \{-9\} \cup [17,20].$$
\rozwStop
\odpStart
$x \in \{-9\} \cup [17,20]$
\odpStop
\testStart
A.$x \in \{-9\} \cup [17,20]$\\
B.$x \in \{9\} \cup (17,20)$\\
C.$x \in \{-9\} \cup (17,20]$\\
D.$x \in \{9\} \cup (17,20]$\\
E.$x \in \{-9\} \cup [17,20)$\\
F.$x \in \{9\} \cup [17,20)$\\
G.$x \in \{-9\} \cup (17,20)$\\
H.$x \in \{9\} \cup [17,20]$
\testStop
\kluczStart
A
\kluczStop



\zadStart{Zadanie z Wikieł Z 1.62 c) moja wersja nr 1068}

Rozwiązać nierówności $(17-x)(x+10)^{2}(18-x)^{3}\le0$.
\zadStop
\rozwStart{Patryk Wirkus}{}
Miejsca zerowe naszego wielomianu to: $17, -10, 18$.\\
Wielomian jest stopnia parzystego, ponadto znak współczynnika przy\linebreak najwyższej potędze x jest ujemny.\\ W związku z tym wykres wielomianu zaczyna się od lewej strony powyżej osi OX.\\
Ponadto w punkcie $-10$ wykres odbija się od osi poziomej.\\
A więc $$x \in \{-10\} \cup [17,18].$$
\rozwStop
\odpStart
$x \in \{-10\} \cup [17,18]$
\odpStop
\testStart
A.$x \in \{-10\} \cup [17,18]$\\
B.$x \in \{10\} \cup (17,18)$\\
C.$x \in \{-10\} \cup (17,18]$\\
D.$x \in \{10\} \cup (17,18]$\\
E.$x \in \{-10\} \cup [17,18)$\\
F.$x \in \{10\} \cup [17,18)$\\
G.$x \in \{-10\} \cup (17,18)$\\
H.$x \in \{10\} \cup [17,18]$
\testStop
\kluczStart
A
\kluczStop



\zadStart{Zadanie z Wikieł Z 1.62 c) moja wersja nr 1069}

Rozwiązać nierówności $(17-x)(x+10)^{2}(19-x)^{3}\le0$.
\zadStop
\rozwStart{Patryk Wirkus}{}
Miejsca zerowe naszego wielomianu to: $17, -10, 19$.\\
Wielomian jest stopnia parzystego, ponadto znak współczynnika przy\linebreak najwyższej potędze x jest ujemny.\\ W związku z tym wykres wielomianu zaczyna się od lewej strony powyżej osi OX.\\
Ponadto w punkcie $-10$ wykres odbija się od osi poziomej.\\
A więc $$x \in \{-10\} \cup [17,19].$$
\rozwStop
\odpStart
$x \in \{-10\} \cup [17,19]$
\odpStop
\testStart
A.$x \in \{-10\} \cup [17,19]$\\
B.$x \in \{10\} \cup (17,19)$\\
C.$x \in \{-10\} \cup (17,19]$\\
D.$x \in \{10\} \cup (17,19]$\\
E.$x \in \{-10\} \cup [17,19)$\\
F.$x \in \{10\} \cup [17,19)$\\
G.$x \in \{-10\} \cup (17,19)$\\
H.$x \in \{10\} \cup [17,19]$
\testStop
\kluczStart
A
\kluczStop



\zadStart{Zadanie z Wikieł Z 1.62 c) moja wersja nr 1070}

Rozwiązać nierówności $(17-x)(x+10)^{2}(20-x)^{3}\le0$.
\zadStop
\rozwStart{Patryk Wirkus}{}
Miejsca zerowe naszego wielomianu to: $17, -10, 20$.\\
Wielomian jest stopnia parzystego, ponadto znak współczynnika przy\linebreak najwyższej potędze x jest ujemny.\\ W związku z tym wykres wielomianu zaczyna się od lewej strony powyżej osi OX.\\
Ponadto w punkcie $-10$ wykres odbija się od osi poziomej.\\
A więc $$x \in \{-10\} \cup [17,20].$$
\rozwStop
\odpStart
$x \in \{-10\} \cup [17,20]$
\odpStop
\testStart
A.$x \in \{-10\} \cup [17,20]$\\
B.$x \in \{10\} \cup (17,20)$\\
C.$x \in \{-10\} \cup (17,20]$\\
D.$x \in \{10\} \cup (17,20]$\\
E.$x \in \{-10\} \cup [17,20)$\\
F.$x \in \{10\} \cup [17,20)$\\
G.$x \in \{-10\} \cup (17,20)$\\
H.$x \in \{10\} \cup [17,20]$
\testStop
\kluczStart
A
\kluczStop



\zadStart{Zadanie z Wikieł Z 1.62 c) moja wersja nr 1071}

Rozwiązać nierówności $(17-x)(x+11)^{2}(18-x)^{3}\le0$.
\zadStop
\rozwStart{Patryk Wirkus}{}
Miejsca zerowe naszego wielomianu to: $17, -11, 18$.\\
Wielomian jest stopnia parzystego, ponadto znak współczynnika przy\linebreak najwyższej potędze x jest ujemny.\\ W związku z tym wykres wielomianu zaczyna się od lewej strony powyżej osi OX.\\
Ponadto w punkcie $-11$ wykres odbija się od osi poziomej.\\
A więc $$x \in \{-11\} \cup [17,18].$$
\rozwStop
\odpStart
$x \in \{-11\} \cup [17,18]$
\odpStop
\testStart
A.$x \in \{-11\} \cup [17,18]$\\
B.$x \in \{11\} \cup (17,18)$\\
C.$x \in \{-11\} \cup (17,18]$\\
D.$x \in \{11\} \cup (17,18]$\\
E.$x \in \{-11\} \cup [17,18)$\\
F.$x \in \{11\} \cup [17,18)$\\
G.$x \in \{-11\} \cup (17,18)$\\
H.$x \in \{11\} \cup [17,18]$
\testStop
\kluczStart
A
\kluczStop



\zadStart{Zadanie z Wikieł Z 1.62 c) moja wersja nr 1072}

Rozwiązać nierówności $(17-x)(x+11)^{2}(19-x)^{3}\le0$.
\zadStop
\rozwStart{Patryk Wirkus}{}
Miejsca zerowe naszego wielomianu to: $17, -11, 19$.\\
Wielomian jest stopnia parzystego, ponadto znak współczynnika przy\linebreak najwyższej potędze x jest ujemny.\\ W związku z tym wykres wielomianu zaczyna się od lewej strony powyżej osi OX.\\
Ponadto w punkcie $-11$ wykres odbija się od osi poziomej.\\
A więc $$x \in \{-11\} \cup [17,19].$$
\rozwStop
\odpStart
$x \in \{-11\} \cup [17,19]$
\odpStop
\testStart
A.$x \in \{-11\} \cup [17,19]$\\
B.$x \in \{11\} \cup (17,19)$\\
C.$x \in \{-11\} \cup (17,19]$\\
D.$x \in \{11\} \cup (17,19]$\\
E.$x \in \{-11\} \cup [17,19)$\\
F.$x \in \{11\} \cup [17,19)$\\
G.$x \in \{-11\} \cup (17,19)$\\
H.$x \in \{11\} \cup [17,19]$
\testStop
\kluczStart
A
\kluczStop



\zadStart{Zadanie z Wikieł Z 1.62 c) moja wersja nr 1073}

Rozwiązać nierówności $(17-x)(x+11)^{2}(20-x)^{3}\le0$.
\zadStop
\rozwStart{Patryk Wirkus}{}
Miejsca zerowe naszego wielomianu to: $17, -11, 20$.\\
Wielomian jest stopnia parzystego, ponadto znak współczynnika przy\linebreak najwyższej potędze x jest ujemny.\\ W związku z tym wykres wielomianu zaczyna się od lewej strony powyżej osi OX.\\
Ponadto w punkcie $-11$ wykres odbija się od osi poziomej.\\
A więc $$x \in \{-11\} \cup [17,20].$$
\rozwStop
\odpStart
$x \in \{-11\} \cup [17,20]$
\odpStop
\testStart
A.$x \in \{-11\} \cup [17,20]$\\
B.$x \in \{11\} \cup (17,20)$\\
C.$x \in \{-11\} \cup (17,20]$\\
D.$x \in \{11\} \cup (17,20]$\\
E.$x \in \{-11\} \cup [17,20)$\\
F.$x \in \{11\} \cup [17,20)$\\
G.$x \in \{-11\} \cup (17,20)$\\
H.$x \in \{11\} \cup [17,20]$
\testStop
\kluczStart
A
\kluczStop



\zadStart{Zadanie z Wikieł Z 1.62 c) moja wersja nr 1074}

Rozwiązać nierówności $(17-x)(x+12)^{2}(18-x)^{3}\le0$.
\zadStop
\rozwStart{Patryk Wirkus}{}
Miejsca zerowe naszego wielomianu to: $17, -12, 18$.\\
Wielomian jest stopnia parzystego, ponadto znak współczynnika przy\linebreak najwyższej potędze x jest ujemny.\\ W związku z tym wykres wielomianu zaczyna się od lewej strony powyżej osi OX.\\
Ponadto w punkcie $-12$ wykres odbija się od osi poziomej.\\
A więc $$x \in \{-12\} \cup [17,18].$$
\rozwStop
\odpStart
$x \in \{-12\} \cup [17,18]$
\odpStop
\testStart
A.$x \in \{-12\} \cup [17,18]$\\
B.$x \in \{12\} \cup (17,18)$\\
C.$x \in \{-12\} \cup (17,18]$\\
D.$x \in \{12\} \cup (17,18]$\\
E.$x \in \{-12\} \cup [17,18)$\\
F.$x \in \{12\} \cup [17,18)$\\
G.$x \in \{-12\} \cup (17,18)$\\
H.$x \in \{12\} \cup [17,18]$
\testStop
\kluczStart
A
\kluczStop



\zadStart{Zadanie z Wikieł Z 1.62 c) moja wersja nr 1075}

Rozwiązać nierówności $(17-x)(x+12)^{2}(19-x)^{3}\le0$.
\zadStop
\rozwStart{Patryk Wirkus}{}
Miejsca zerowe naszego wielomianu to: $17, -12, 19$.\\
Wielomian jest stopnia parzystego, ponadto znak współczynnika przy\linebreak najwyższej potędze x jest ujemny.\\ W związku z tym wykres wielomianu zaczyna się od lewej strony powyżej osi OX.\\
Ponadto w punkcie $-12$ wykres odbija się od osi poziomej.\\
A więc $$x \in \{-12\} \cup [17,19].$$
\rozwStop
\odpStart
$x \in \{-12\} \cup [17,19]$
\odpStop
\testStart
A.$x \in \{-12\} \cup [17,19]$\\
B.$x \in \{12\} \cup (17,19)$\\
C.$x \in \{-12\} \cup (17,19]$\\
D.$x \in \{12\} \cup (17,19]$\\
E.$x \in \{-12\} \cup [17,19)$\\
F.$x \in \{12\} \cup [17,19)$\\
G.$x \in \{-12\} \cup (17,19)$\\
H.$x \in \{12\} \cup [17,19]$
\testStop
\kluczStart
A
\kluczStop



\zadStart{Zadanie z Wikieł Z 1.62 c) moja wersja nr 1076}

Rozwiązać nierówności $(17-x)(x+12)^{2}(20-x)^{3}\le0$.
\zadStop
\rozwStart{Patryk Wirkus}{}
Miejsca zerowe naszego wielomianu to: $17, -12, 20$.\\
Wielomian jest stopnia parzystego, ponadto znak współczynnika przy\linebreak najwyższej potędze x jest ujemny.\\ W związku z tym wykres wielomianu zaczyna się od lewej strony powyżej osi OX.\\
Ponadto w punkcie $-12$ wykres odbija się od osi poziomej.\\
A więc $$x \in \{-12\} \cup [17,20].$$
\rozwStop
\odpStart
$x \in \{-12\} \cup [17,20]$
\odpStop
\testStart
A.$x \in \{-12\} \cup [17,20]$\\
B.$x \in \{12\} \cup (17,20)$\\
C.$x \in \{-12\} \cup (17,20]$\\
D.$x \in \{12\} \cup (17,20]$\\
E.$x \in \{-12\} \cup [17,20)$\\
F.$x \in \{12\} \cup [17,20)$\\
G.$x \in \{-12\} \cup (17,20)$\\
H.$x \in \{12\} \cup [17,20]$
\testStop
\kluczStart
A
\kluczStop



\zadStart{Zadanie z Wikieł Z 1.62 c) moja wersja nr 1077}

Rozwiązać nierówności $(17-x)(x+13)^{2}(18-x)^{3}\le0$.
\zadStop
\rozwStart{Patryk Wirkus}{}
Miejsca zerowe naszego wielomianu to: $17, -13, 18$.\\
Wielomian jest stopnia parzystego, ponadto znak współczynnika przy\linebreak najwyższej potędze x jest ujemny.\\ W związku z tym wykres wielomianu zaczyna się od lewej strony powyżej osi OX.\\
Ponadto w punkcie $-13$ wykres odbija się od osi poziomej.\\
A więc $$x \in \{-13\} \cup [17,18].$$
\rozwStop
\odpStart
$x \in \{-13\} \cup [17,18]$
\odpStop
\testStart
A.$x \in \{-13\} \cup [17,18]$\\
B.$x \in \{13\} \cup (17,18)$\\
C.$x \in \{-13\} \cup (17,18]$\\
D.$x \in \{13\} \cup (17,18]$\\
E.$x \in \{-13\} \cup [17,18)$\\
F.$x \in \{13\} \cup [17,18)$\\
G.$x \in \{-13\} \cup (17,18)$\\
H.$x \in \{13\} \cup [17,18]$
\testStop
\kluczStart
A
\kluczStop



\zadStart{Zadanie z Wikieł Z 1.62 c) moja wersja nr 1078}

Rozwiązać nierówności $(17-x)(x+13)^{2}(19-x)^{3}\le0$.
\zadStop
\rozwStart{Patryk Wirkus}{}
Miejsca zerowe naszego wielomianu to: $17, -13, 19$.\\
Wielomian jest stopnia parzystego, ponadto znak współczynnika przy\linebreak najwyższej potędze x jest ujemny.\\ W związku z tym wykres wielomianu zaczyna się od lewej strony powyżej osi OX.\\
Ponadto w punkcie $-13$ wykres odbija się od osi poziomej.\\
A więc $$x \in \{-13\} \cup [17,19].$$
\rozwStop
\odpStart
$x \in \{-13\} \cup [17,19]$
\odpStop
\testStart
A.$x \in \{-13\} \cup [17,19]$\\
B.$x \in \{13\} \cup (17,19)$\\
C.$x \in \{-13\} \cup (17,19]$\\
D.$x \in \{13\} \cup (17,19]$\\
E.$x \in \{-13\} \cup [17,19)$\\
F.$x \in \{13\} \cup [17,19)$\\
G.$x \in \{-13\} \cup (17,19)$\\
H.$x \in \{13\} \cup [17,19]$
\testStop
\kluczStart
A
\kluczStop



\zadStart{Zadanie z Wikieł Z 1.62 c) moja wersja nr 1079}

Rozwiązać nierówności $(17-x)(x+13)^{2}(20-x)^{3}\le0$.
\zadStop
\rozwStart{Patryk Wirkus}{}
Miejsca zerowe naszego wielomianu to: $17, -13, 20$.\\
Wielomian jest stopnia parzystego, ponadto znak współczynnika przy\linebreak najwyższej potędze x jest ujemny.\\ W związku z tym wykres wielomianu zaczyna się od lewej strony powyżej osi OX.\\
Ponadto w punkcie $-13$ wykres odbija się od osi poziomej.\\
A więc $$x \in \{-13\} \cup [17,20].$$
\rozwStop
\odpStart
$x \in \{-13\} \cup [17,20]$
\odpStop
\testStart
A.$x \in \{-13\} \cup [17,20]$\\
B.$x \in \{13\} \cup (17,20)$\\
C.$x \in \{-13\} \cup (17,20]$\\
D.$x \in \{13\} \cup (17,20]$\\
E.$x \in \{-13\} \cup [17,20)$\\
F.$x \in \{13\} \cup [17,20)$\\
G.$x \in \{-13\} \cup (17,20)$\\
H.$x \in \{13\} \cup [17,20]$
\testStop
\kluczStart
A
\kluczStop



\zadStart{Zadanie z Wikieł Z 1.62 c) moja wersja nr 1080}

Rozwiązać nierówności $(17-x)(x+14)^{2}(18-x)^{3}\le0$.
\zadStop
\rozwStart{Patryk Wirkus}{}
Miejsca zerowe naszego wielomianu to: $17, -14, 18$.\\
Wielomian jest stopnia parzystego, ponadto znak współczynnika przy\linebreak najwyższej potędze x jest ujemny.\\ W związku z tym wykres wielomianu zaczyna się od lewej strony powyżej osi OX.\\
Ponadto w punkcie $-14$ wykres odbija się od osi poziomej.\\
A więc $$x \in \{-14\} \cup [17,18].$$
\rozwStop
\odpStart
$x \in \{-14\} \cup [17,18]$
\odpStop
\testStart
A.$x \in \{-14\} \cup [17,18]$\\
B.$x \in \{14\} \cup (17,18)$\\
C.$x \in \{-14\} \cup (17,18]$\\
D.$x \in \{14\} \cup (17,18]$\\
E.$x \in \{-14\} \cup [17,18)$\\
F.$x \in \{14\} \cup [17,18)$\\
G.$x \in \{-14\} \cup (17,18)$\\
H.$x \in \{14\} \cup [17,18]$
\testStop
\kluczStart
A
\kluczStop



\zadStart{Zadanie z Wikieł Z 1.62 c) moja wersja nr 1081}

Rozwiązać nierówności $(17-x)(x+14)^{2}(19-x)^{3}\le0$.
\zadStop
\rozwStart{Patryk Wirkus}{}
Miejsca zerowe naszego wielomianu to: $17, -14, 19$.\\
Wielomian jest stopnia parzystego, ponadto znak współczynnika przy\linebreak najwyższej potędze x jest ujemny.\\ W związku z tym wykres wielomianu zaczyna się od lewej strony powyżej osi OX.\\
Ponadto w punkcie $-14$ wykres odbija się od osi poziomej.\\
A więc $$x \in \{-14\} \cup [17,19].$$
\rozwStop
\odpStart
$x \in \{-14\} \cup [17,19]$
\odpStop
\testStart
A.$x \in \{-14\} \cup [17,19]$\\
B.$x \in \{14\} \cup (17,19)$\\
C.$x \in \{-14\} \cup (17,19]$\\
D.$x \in \{14\} \cup (17,19]$\\
E.$x \in \{-14\} \cup [17,19)$\\
F.$x \in \{14\} \cup [17,19)$\\
G.$x \in \{-14\} \cup (17,19)$\\
H.$x \in \{14\} \cup [17,19]$
\testStop
\kluczStart
A
\kluczStop



\zadStart{Zadanie z Wikieł Z 1.62 c) moja wersja nr 1082}

Rozwiązać nierówności $(17-x)(x+14)^{2}(20-x)^{3}\le0$.
\zadStop
\rozwStart{Patryk Wirkus}{}
Miejsca zerowe naszego wielomianu to: $17, -14, 20$.\\
Wielomian jest stopnia parzystego, ponadto znak współczynnika przy\linebreak najwyższej potędze x jest ujemny.\\ W związku z tym wykres wielomianu zaczyna się od lewej strony powyżej osi OX.\\
Ponadto w punkcie $-14$ wykres odbija się od osi poziomej.\\
A więc $$x \in \{-14\} \cup [17,20].$$
\rozwStop
\odpStart
$x \in \{-14\} \cup [17,20]$
\odpStop
\testStart
A.$x \in \{-14\} \cup [17,20]$\\
B.$x \in \{14\} \cup (17,20)$\\
C.$x \in \{-14\} \cup (17,20]$\\
D.$x \in \{14\} \cup (17,20]$\\
E.$x \in \{-14\} \cup [17,20)$\\
F.$x \in \{14\} \cup [17,20)$\\
G.$x \in \{-14\} \cup (17,20)$\\
H.$x \in \{14\} \cup [17,20]$
\testStop
\kluczStart
A
\kluczStop



\zadStart{Zadanie z Wikieł Z 1.62 c) moja wersja nr 1083}

Rozwiązać nierówności $(17-x)(x+15)^{2}(18-x)^{3}\le0$.
\zadStop
\rozwStart{Patryk Wirkus}{}
Miejsca zerowe naszego wielomianu to: $17, -15, 18$.\\
Wielomian jest stopnia parzystego, ponadto znak współczynnika przy\linebreak najwyższej potędze x jest ujemny.\\ W związku z tym wykres wielomianu zaczyna się od lewej strony powyżej osi OX.\\
Ponadto w punkcie $-15$ wykres odbija się od osi poziomej.\\
A więc $$x \in \{-15\} \cup [17,18].$$
\rozwStop
\odpStart
$x \in \{-15\} \cup [17,18]$
\odpStop
\testStart
A.$x \in \{-15\} \cup [17,18]$\\
B.$x \in \{15\} \cup (17,18)$\\
C.$x \in \{-15\} \cup (17,18]$\\
D.$x \in \{15\} \cup (17,18]$\\
E.$x \in \{-15\} \cup [17,18)$\\
F.$x \in \{15\} \cup [17,18)$\\
G.$x \in \{-15\} \cup (17,18)$\\
H.$x \in \{15\} \cup [17,18]$
\testStop
\kluczStart
A
\kluczStop



\zadStart{Zadanie z Wikieł Z 1.62 c) moja wersja nr 1084}

Rozwiązać nierówności $(17-x)(x+15)^{2}(19-x)^{3}\le0$.
\zadStop
\rozwStart{Patryk Wirkus}{}
Miejsca zerowe naszego wielomianu to: $17, -15, 19$.\\
Wielomian jest stopnia parzystego, ponadto znak współczynnika przy\linebreak najwyższej potędze x jest ujemny.\\ W związku z tym wykres wielomianu zaczyna się od lewej strony powyżej osi OX.\\
Ponadto w punkcie $-15$ wykres odbija się od osi poziomej.\\
A więc $$x \in \{-15\} \cup [17,19].$$
\rozwStop
\odpStart
$x \in \{-15\} \cup [17,19]$
\odpStop
\testStart
A.$x \in \{-15\} \cup [17,19]$\\
B.$x \in \{15\} \cup (17,19)$\\
C.$x \in \{-15\} \cup (17,19]$\\
D.$x \in \{15\} \cup (17,19]$\\
E.$x \in \{-15\} \cup [17,19)$\\
F.$x \in \{15\} \cup [17,19)$\\
G.$x \in \{-15\} \cup (17,19)$\\
H.$x \in \{15\} \cup [17,19]$
\testStop
\kluczStart
A
\kluczStop



\zadStart{Zadanie z Wikieł Z 1.62 c) moja wersja nr 1085}

Rozwiązać nierówności $(17-x)(x+15)^{2}(20-x)^{3}\le0$.
\zadStop
\rozwStart{Patryk Wirkus}{}
Miejsca zerowe naszego wielomianu to: $17, -15, 20$.\\
Wielomian jest stopnia parzystego, ponadto znak współczynnika przy\linebreak najwyższej potędze x jest ujemny.\\ W związku z tym wykres wielomianu zaczyna się od lewej strony powyżej osi OX.\\
Ponadto w punkcie $-15$ wykres odbija się od osi poziomej.\\
A więc $$x \in \{-15\} \cup [17,20].$$
\rozwStop
\odpStart
$x \in \{-15\} \cup [17,20]$
\odpStop
\testStart
A.$x \in \{-15\} \cup [17,20]$\\
B.$x \in \{15\} \cup (17,20)$\\
C.$x \in \{-15\} \cup (17,20]$\\
D.$x \in \{15\} \cup (17,20]$\\
E.$x \in \{-15\} \cup [17,20)$\\
F.$x \in \{15\} \cup [17,20)$\\
G.$x \in \{-15\} \cup (17,20)$\\
H.$x \in \{15\} \cup [17,20]$
\testStop
\kluczStart
A
\kluczStop



\zadStart{Zadanie z Wikieł Z 1.62 c) moja wersja nr 1086}

Rozwiązać nierówności $(17-x)(x+16)^{2}(18-x)^{3}\le0$.
\zadStop
\rozwStart{Patryk Wirkus}{}
Miejsca zerowe naszego wielomianu to: $17, -16, 18$.\\
Wielomian jest stopnia parzystego, ponadto znak współczynnika przy\linebreak najwyższej potędze x jest ujemny.\\ W związku z tym wykres wielomianu zaczyna się od lewej strony powyżej osi OX.\\
Ponadto w punkcie $-16$ wykres odbija się od osi poziomej.\\
A więc $$x \in \{-16\} \cup [17,18].$$
\rozwStop
\odpStart
$x \in \{-16\} \cup [17,18]$
\odpStop
\testStart
A.$x \in \{-16\} \cup [17,18]$\\
B.$x \in \{16\} \cup (17,18)$\\
C.$x \in \{-16\} \cup (17,18]$\\
D.$x \in \{16\} \cup (17,18]$\\
E.$x \in \{-16\} \cup [17,18)$\\
F.$x \in \{16\} \cup [17,18)$\\
G.$x \in \{-16\} \cup (17,18)$\\
H.$x \in \{16\} \cup [17,18]$
\testStop
\kluczStart
A
\kluczStop



\zadStart{Zadanie z Wikieł Z 1.62 c) moja wersja nr 1087}

Rozwiązać nierówności $(17-x)(x+16)^{2}(19-x)^{3}\le0$.
\zadStop
\rozwStart{Patryk Wirkus}{}
Miejsca zerowe naszego wielomianu to: $17, -16, 19$.\\
Wielomian jest stopnia parzystego, ponadto znak współczynnika przy\linebreak najwyższej potędze x jest ujemny.\\ W związku z tym wykres wielomianu zaczyna się od lewej strony powyżej osi OX.\\
Ponadto w punkcie $-16$ wykres odbija się od osi poziomej.\\
A więc $$x \in \{-16\} \cup [17,19].$$
\rozwStop
\odpStart
$x \in \{-16\} \cup [17,19]$
\odpStop
\testStart
A.$x \in \{-16\} \cup [17,19]$\\
B.$x \in \{16\} \cup (17,19)$\\
C.$x \in \{-16\} \cup (17,19]$\\
D.$x \in \{16\} \cup (17,19]$\\
E.$x \in \{-16\} \cup [17,19)$\\
F.$x \in \{16\} \cup [17,19)$\\
G.$x \in \{-16\} \cup (17,19)$\\
H.$x \in \{16\} \cup [17,19]$
\testStop
\kluczStart
A
\kluczStop



\zadStart{Zadanie z Wikieł Z 1.62 c) moja wersja nr 1088}

Rozwiązać nierówności $(17-x)(x+16)^{2}(20-x)^{3}\le0$.
\zadStop
\rozwStart{Patryk Wirkus}{}
Miejsca zerowe naszego wielomianu to: $17, -16, 20$.\\
Wielomian jest stopnia parzystego, ponadto znak współczynnika przy\linebreak najwyższej potędze x jest ujemny.\\ W związku z tym wykres wielomianu zaczyna się od lewej strony powyżej osi OX.\\
Ponadto w punkcie $-16$ wykres odbija się od osi poziomej.\\
A więc $$x \in \{-16\} \cup [17,20].$$
\rozwStop
\odpStart
$x \in \{-16\} \cup [17,20]$
\odpStop
\testStart
A.$x \in \{-16\} \cup [17,20]$\\
B.$x \in \{16\} \cup (17,20)$\\
C.$x \in \{-16\} \cup (17,20]$\\
D.$x \in \{16\} \cup (17,20]$\\
E.$x \in \{-16\} \cup [17,20)$\\
F.$x \in \{16\} \cup [17,20)$\\
G.$x \in \{-16\} \cup (17,20)$\\
H.$x \in \{16\} \cup [17,20]$
\testStop
\kluczStart
A
\kluczStop



\zadStart{Zadanie z Wikieł Z 1.62 c) moja wersja nr 1089}

Rozwiązać nierówności $(18-x)(x+1)^{2}(19-x)^{3}\le0$.
\zadStop
\rozwStart{Patryk Wirkus}{}
Miejsca zerowe naszego wielomianu to: $18, -1, 19$.\\
Wielomian jest stopnia parzystego, ponadto znak współczynnika przy\linebreak najwyższej potędze x jest ujemny.\\ W związku z tym wykres wielomianu zaczyna się od lewej strony powyżej osi OX.\\
Ponadto w punkcie $-1$ wykres odbija się od osi poziomej.\\
A więc $$x \in \{-1\} \cup [18,19].$$
\rozwStop
\odpStart
$x \in \{-1\} \cup [18,19]$
\odpStop
\testStart
A.$x \in \{-1\} \cup [18,19]$\\
B.$x \in \{1\} \cup (18,19)$\\
C.$x \in \{-1\} \cup (18,19]$\\
D.$x \in \{1\} \cup (18,19]$\\
E.$x \in \{-1\} \cup [18,19)$\\
F.$x \in \{1\} \cup [18,19)$\\
G.$x \in \{-1\} \cup (18,19)$\\
H.$x \in \{1\} \cup [18,19]$
\testStop
\kluczStart
A
\kluczStop



\zadStart{Zadanie z Wikieł Z 1.62 c) moja wersja nr 1090}

Rozwiązać nierówności $(18-x)(x+1)^{2}(20-x)^{3}\le0$.
\zadStop
\rozwStart{Patryk Wirkus}{}
Miejsca zerowe naszego wielomianu to: $18, -1, 20$.\\
Wielomian jest stopnia parzystego, ponadto znak współczynnika przy\linebreak najwyższej potędze x jest ujemny.\\ W związku z tym wykres wielomianu zaczyna się od lewej strony powyżej osi OX.\\
Ponadto w punkcie $-1$ wykres odbija się od osi poziomej.\\
A więc $$x \in \{-1\} \cup [18,20].$$
\rozwStop
\odpStart
$x \in \{-1\} \cup [18,20]$
\odpStop
\testStart
A.$x \in \{-1\} \cup [18,20]$\\
B.$x \in \{1\} \cup (18,20)$\\
C.$x \in \{-1\} \cup (18,20]$\\
D.$x \in \{1\} \cup (18,20]$\\
E.$x \in \{-1\} \cup [18,20)$\\
F.$x \in \{1\} \cup [18,20)$\\
G.$x \in \{-1\} \cup (18,20)$\\
H.$x \in \{1\} \cup [18,20]$
\testStop
\kluczStart
A
\kluczStop



\zadStart{Zadanie z Wikieł Z 1.62 c) moja wersja nr 1091}

Rozwiązać nierówności $(18-x)(x+2)^{2}(19-x)^{3}\le0$.
\zadStop
\rozwStart{Patryk Wirkus}{}
Miejsca zerowe naszego wielomianu to: $18, -2, 19$.\\
Wielomian jest stopnia parzystego, ponadto znak współczynnika przy\linebreak najwyższej potędze x jest ujemny.\\ W związku z tym wykres wielomianu zaczyna się od lewej strony powyżej osi OX.\\
Ponadto w punkcie $-2$ wykres odbija się od osi poziomej.\\
A więc $$x \in \{-2\} \cup [18,19].$$
\rozwStop
\odpStart
$x \in \{-2\} \cup [18,19]$
\odpStop
\testStart
A.$x \in \{-2\} \cup [18,19]$\\
B.$x \in \{2\} \cup (18,19)$\\
C.$x \in \{-2\} \cup (18,19]$\\
D.$x \in \{2\} \cup (18,19]$\\
E.$x \in \{-2\} \cup [18,19)$\\
F.$x \in \{2\} \cup [18,19)$\\
G.$x \in \{-2\} \cup (18,19)$\\
H.$x \in \{2\} \cup [18,19]$
\testStop
\kluczStart
A
\kluczStop



\zadStart{Zadanie z Wikieł Z 1.62 c) moja wersja nr 1092}

Rozwiązać nierówności $(18-x)(x+2)^{2}(20-x)^{3}\le0$.
\zadStop
\rozwStart{Patryk Wirkus}{}
Miejsca zerowe naszego wielomianu to: $18, -2, 20$.\\
Wielomian jest stopnia parzystego, ponadto znak współczynnika przy\linebreak najwyższej potędze x jest ujemny.\\ W związku z tym wykres wielomianu zaczyna się od lewej strony powyżej osi OX.\\
Ponadto w punkcie $-2$ wykres odbija się od osi poziomej.\\
A więc $$x \in \{-2\} \cup [18,20].$$
\rozwStop
\odpStart
$x \in \{-2\} \cup [18,20]$
\odpStop
\testStart
A.$x \in \{-2\} \cup [18,20]$\\
B.$x \in \{2\} \cup (18,20)$\\
C.$x \in \{-2\} \cup (18,20]$\\
D.$x \in \{2\} \cup (18,20]$\\
E.$x \in \{-2\} \cup [18,20)$\\
F.$x \in \{2\} \cup [18,20)$\\
G.$x \in \{-2\} \cup (18,20)$\\
H.$x \in \{2\} \cup [18,20]$
\testStop
\kluczStart
A
\kluczStop



\zadStart{Zadanie z Wikieł Z 1.62 c) moja wersja nr 1093}

Rozwiązać nierówności $(18-x)(x+3)^{2}(19-x)^{3}\le0$.
\zadStop
\rozwStart{Patryk Wirkus}{}
Miejsca zerowe naszego wielomianu to: $18, -3, 19$.\\
Wielomian jest stopnia parzystego, ponadto znak współczynnika przy\linebreak najwyższej potędze x jest ujemny.\\ W związku z tym wykres wielomianu zaczyna się od lewej strony powyżej osi OX.\\
Ponadto w punkcie $-3$ wykres odbija się od osi poziomej.\\
A więc $$x \in \{-3\} \cup [18,19].$$
\rozwStop
\odpStart
$x \in \{-3\} \cup [18,19]$
\odpStop
\testStart
A.$x \in \{-3\} \cup [18,19]$\\
B.$x \in \{3\} \cup (18,19)$\\
C.$x \in \{-3\} \cup (18,19]$\\
D.$x \in \{3\} \cup (18,19]$\\
E.$x \in \{-3\} \cup [18,19)$\\
F.$x \in \{3\} \cup [18,19)$\\
G.$x \in \{-3\} \cup (18,19)$\\
H.$x \in \{3\} \cup [18,19]$
\testStop
\kluczStart
A
\kluczStop



\zadStart{Zadanie z Wikieł Z 1.62 c) moja wersja nr 1094}

Rozwiązać nierówności $(18-x)(x+3)^{2}(20-x)^{3}\le0$.
\zadStop
\rozwStart{Patryk Wirkus}{}
Miejsca zerowe naszego wielomianu to: $18, -3, 20$.\\
Wielomian jest stopnia parzystego, ponadto znak współczynnika przy\linebreak najwyższej potędze x jest ujemny.\\ W związku z tym wykres wielomianu zaczyna się od lewej strony powyżej osi OX.\\
Ponadto w punkcie $-3$ wykres odbija się od osi poziomej.\\
A więc $$x \in \{-3\} \cup [18,20].$$
\rozwStop
\odpStart
$x \in \{-3\} \cup [18,20]$
\odpStop
\testStart
A.$x \in \{-3\} \cup [18,20]$\\
B.$x \in \{3\} \cup (18,20)$\\
C.$x \in \{-3\} \cup (18,20]$\\
D.$x \in \{3\} \cup (18,20]$\\
E.$x \in \{-3\} \cup [18,20)$\\
F.$x \in \{3\} \cup [18,20)$\\
G.$x \in \{-3\} \cup (18,20)$\\
H.$x \in \{3\} \cup [18,20]$
\testStop
\kluczStart
A
\kluczStop



\zadStart{Zadanie z Wikieł Z 1.62 c) moja wersja nr 1095}

Rozwiązać nierówności $(18-x)(x+4)^{2}(19-x)^{3}\le0$.
\zadStop
\rozwStart{Patryk Wirkus}{}
Miejsca zerowe naszego wielomianu to: $18, -4, 19$.\\
Wielomian jest stopnia parzystego, ponadto znak współczynnika przy\linebreak najwyższej potędze x jest ujemny.\\ W związku z tym wykres wielomianu zaczyna się od lewej strony powyżej osi OX.\\
Ponadto w punkcie $-4$ wykres odbija się od osi poziomej.\\
A więc $$x \in \{-4\} \cup [18,19].$$
\rozwStop
\odpStart
$x \in \{-4\} \cup [18,19]$
\odpStop
\testStart
A.$x \in \{-4\} \cup [18,19]$\\
B.$x \in \{4\} \cup (18,19)$\\
C.$x \in \{-4\} \cup (18,19]$\\
D.$x \in \{4\} \cup (18,19]$\\
E.$x \in \{-4\} \cup [18,19)$\\
F.$x \in \{4\} \cup [18,19)$\\
G.$x \in \{-4\} \cup (18,19)$\\
H.$x \in \{4\} \cup [18,19]$
\testStop
\kluczStart
A
\kluczStop



\zadStart{Zadanie z Wikieł Z 1.62 c) moja wersja nr 1096}

Rozwiązać nierówności $(18-x)(x+4)^{2}(20-x)^{3}\le0$.
\zadStop
\rozwStart{Patryk Wirkus}{}
Miejsca zerowe naszego wielomianu to: $18, -4, 20$.\\
Wielomian jest stopnia parzystego, ponadto znak współczynnika przy\linebreak najwyższej potędze x jest ujemny.\\ W związku z tym wykres wielomianu zaczyna się od lewej strony powyżej osi OX.\\
Ponadto w punkcie $-4$ wykres odbija się od osi poziomej.\\
A więc $$x \in \{-4\} \cup [18,20].$$
\rozwStop
\odpStart
$x \in \{-4\} \cup [18,20]$
\odpStop
\testStart
A.$x \in \{-4\} \cup [18,20]$\\
B.$x \in \{4\} \cup (18,20)$\\
C.$x \in \{-4\} \cup (18,20]$\\
D.$x \in \{4\} \cup (18,20]$\\
E.$x \in \{-4\} \cup [18,20)$\\
F.$x \in \{4\} \cup [18,20)$\\
G.$x \in \{-4\} \cup (18,20)$\\
H.$x \in \{4\} \cup [18,20]$
\testStop
\kluczStart
A
\kluczStop



\zadStart{Zadanie z Wikieł Z 1.62 c) moja wersja nr 1097}

Rozwiązać nierówności $(18-x)(x+5)^{2}(19-x)^{3}\le0$.
\zadStop
\rozwStart{Patryk Wirkus}{}
Miejsca zerowe naszego wielomianu to: $18, -5, 19$.\\
Wielomian jest stopnia parzystego, ponadto znak współczynnika przy\linebreak najwyższej potędze x jest ujemny.\\ W związku z tym wykres wielomianu zaczyna się od lewej strony powyżej osi OX.\\
Ponadto w punkcie $-5$ wykres odbija się od osi poziomej.\\
A więc $$x \in \{-5\} \cup [18,19].$$
\rozwStop
\odpStart
$x \in \{-5\} \cup [18,19]$
\odpStop
\testStart
A.$x \in \{-5\} \cup [18,19]$\\
B.$x \in \{5\} \cup (18,19)$\\
C.$x \in \{-5\} \cup (18,19]$\\
D.$x \in \{5\} \cup (18,19]$\\
E.$x \in \{-5\} \cup [18,19)$\\
F.$x \in \{5\} \cup [18,19)$\\
G.$x \in \{-5\} \cup (18,19)$\\
H.$x \in \{5\} \cup [18,19]$
\testStop
\kluczStart
A
\kluczStop



\zadStart{Zadanie z Wikieł Z 1.62 c) moja wersja nr 1098}

Rozwiązać nierówności $(18-x)(x+5)^{2}(20-x)^{3}\le0$.
\zadStop
\rozwStart{Patryk Wirkus}{}
Miejsca zerowe naszego wielomianu to: $18, -5, 20$.\\
Wielomian jest stopnia parzystego, ponadto znak współczynnika przy\linebreak najwyższej potędze x jest ujemny.\\ W związku z tym wykres wielomianu zaczyna się od lewej strony powyżej osi OX.\\
Ponadto w punkcie $-5$ wykres odbija się od osi poziomej.\\
A więc $$x \in \{-5\} \cup [18,20].$$
\rozwStop
\odpStart
$x \in \{-5\} \cup [18,20]$
\odpStop
\testStart
A.$x \in \{-5\} \cup [18,20]$\\
B.$x \in \{5\} \cup (18,20)$\\
C.$x \in \{-5\} \cup (18,20]$\\
D.$x \in \{5\} \cup (18,20]$\\
E.$x \in \{-5\} \cup [18,20)$\\
F.$x \in \{5\} \cup [18,20)$\\
G.$x \in \{-5\} \cup (18,20)$\\
H.$x \in \{5\} \cup [18,20]$
\testStop
\kluczStart
A
\kluczStop



\zadStart{Zadanie z Wikieł Z 1.62 c) moja wersja nr 1099}

Rozwiązać nierówności $(18-x)(x+6)^{2}(19-x)^{3}\le0$.
\zadStop
\rozwStart{Patryk Wirkus}{}
Miejsca zerowe naszego wielomianu to: $18, -6, 19$.\\
Wielomian jest stopnia parzystego, ponadto znak współczynnika przy\linebreak najwyższej potędze x jest ujemny.\\ W związku z tym wykres wielomianu zaczyna się od lewej strony powyżej osi OX.\\
Ponadto w punkcie $-6$ wykres odbija się od osi poziomej.\\
A więc $$x \in \{-6\} \cup [18,19].$$
\rozwStop
\odpStart
$x \in \{-6\} \cup [18,19]$
\odpStop
\testStart
A.$x \in \{-6\} \cup [18,19]$\\
B.$x \in \{6\} \cup (18,19)$\\
C.$x \in \{-6\} \cup (18,19]$\\
D.$x \in \{6\} \cup (18,19]$\\
E.$x \in \{-6\} \cup [18,19)$\\
F.$x \in \{6\} \cup [18,19)$\\
G.$x \in \{-6\} \cup (18,19)$\\
H.$x \in \{6\} \cup [18,19]$
\testStop
\kluczStart
A
\kluczStop



\zadStart{Zadanie z Wikieł Z 1.62 c) moja wersja nr 1100}

Rozwiązać nierówności $(18-x)(x+6)^{2}(20-x)^{3}\le0$.
\zadStop
\rozwStart{Patryk Wirkus}{}
Miejsca zerowe naszego wielomianu to: $18, -6, 20$.\\
Wielomian jest stopnia parzystego, ponadto znak współczynnika przy\linebreak najwyższej potędze x jest ujemny.\\ W związku z tym wykres wielomianu zaczyna się od lewej strony powyżej osi OX.\\
Ponadto w punkcie $-6$ wykres odbija się od osi poziomej.\\
A więc $$x \in \{-6\} \cup [18,20].$$
\rozwStop
\odpStart
$x \in \{-6\} \cup [18,20]$
\odpStop
\testStart
A.$x \in \{-6\} \cup [18,20]$\\
B.$x \in \{6\} \cup (18,20)$\\
C.$x \in \{-6\} \cup (18,20]$\\
D.$x \in \{6\} \cup (18,20]$\\
E.$x \in \{-6\} \cup [18,20)$\\
F.$x \in \{6\} \cup [18,20)$\\
G.$x \in \{-6\} \cup (18,20)$\\
H.$x \in \{6\} \cup [18,20]$
\testStop
\kluczStart
A
\kluczStop



\zadStart{Zadanie z Wikieł Z 1.62 c) moja wersja nr 1101}

Rozwiązać nierówności $(18-x)(x+7)^{2}(19-x)^{3}\le0$.
\zadStop
\rozwStart{Patryk Wirkus}{}
Miejsca zerowe naszego wielomianu to: $18, -7, 19$.\\
Wielomian jest stopnia parzystego, ponadto znak współczynnika przy\linebreak najwyższej potędze x jest ujemny.\\ W związku z tym wykres wielomianu zaczyna się od lewej strony powyżej osi OX.\\
Ponadto w punkcie $-7$ wykres odbija się od osi poziomej.\\
A więc $$x \in \{-7\} \cup [18,19].$$
\rozwStop
\odpStart
$x \in \{-7\} \cup [18,19]$
\odpStop
\testStart
A.$x \in \{-7\} \cup [18,19]$\\
B.$x \in \{7\} \cup (18,19)$\\
C.$x \in \{-7\} \cup (18,19]$\\
D.$x \in \{7\} \cup (18,19]$\\
E.$x \in \{-7\} \cup [18,19)$\\
F.$x \in \{7\} \cup [18,19)$\\
G.$x \in \{-7\} \cup (18,19)$\\
H.$x \in \{7\} \cup [18,19]$
\testStop
\kluczStart
A
\kluczStop



\zadStart{Zadanie z Wikieł Z 1.62 c) moja wersja nr 1102}

Rozwiązać nierówności $(18-x)(x+7)^{2}(20-x)^{3}\le0$.
\zadStop
\rozwStart{Patryk Wirkus}{}
Miejsca zerowe naszego wielomianu to: $18, -7, 20$.\\
Wielomian jest stopnia parzystego, ponadto znak współczynnika przy\linebreak najwyższej potędze x jest ujemny.\\ W związku z tym wykres wielomianu zaczyna się od lewej strony powyżej osi OX.\\
Ponadto w punkcie $-7$ wykres odbija się od osi poziomej.\\
A więc $$x \in \{-7\} \cup [18,20].$$
\rozwStop
\odpStart
$x \in \{-7\} \cup [18,20]$
\odpStop
\testStart
A.$x \in \{-7\} \cup [18,20]$\\
B.$x \in \{7\} \cup (18,20)$\\
C.$x \in \{-7\} \cup (18,20]$\\
D.$x \in \{7\} \cup (18,20]$\\
E.$x \in \{-7\} \cup [18,20)$\\
F.$x \in \{7\} \cup [18,20)$\\
G.$x \in \{-7\} \cup (18,20)$\\
H.$x \in \{7\} \cup [18,20]$
\testStop
\kluczStart
A
\kluczStop



\zadStart{Zadanie z Wikieł Z 1.62 c) moja wersja nr 1103}

Rozwiązać nierówności $(18-x)(x+8)^{2}(19-x)^{3}\le0$.
\zadStop
\rozwStart{Patryk Wirkus}{}
Miejsca zerowe naszego wielomianu to: $18, -8, 19$.\\
Wielomian jest stopnia parzystego, ponadto znak współczynnika przy\linebreak najwyższej potędze x jest ujemny.\\ W związku z tym wykres wielomianu zaczyna się od lewej strony powyżej osi OX.\\
Ponadto w punkcie $-8$ wykres odbija się od osi poziomej.\\
A więc $$x \in \{-8\} \cup [18,19].$$
\rozwStop
\odpStart
$x \in \{-8\} \cup [18,19]$
\odpStop
\testStart
A.$x \in \{-8\} \cup [18,19]$\\
B.$x \in \{8\} \cup (18,19)$\\
C.$x \in \{-8\} \cup (18,19]$\\
D.$x \in \{8\} \cup (18,19]$\\
E.$x \in \{-8\} \cup [18,19)$\\
F.$x \in \{8\} \cup [18,19)$\\
G.$x \in \{-8\} \cup (18,19)$\\
H.$x \in \{8\} \cup [18,19]$
\testStop
\kluczStart
A
\kluczStop



\zadStart{Zadanie z Wikieł Z 1.62 c) moja wersja nr 1104}

Rozwiązać nierówności $(18-x)(x+8)^{2}(20-x)^{3}\le0$.
\zadStop
\rozwStart{Patryk Wirkus}{}
Miejsca zerowe naszego wielomianu to: $18, -8, 20$.\\
Wielomian jest stopnia parzystego, ponadto znak współczynnika przy\linebreak najwyższej potędze x jest ujemny.\\ W związku z tym wykres wielomianu zaczyna się od lewej strony powyżej osi OX.\\
Ponadto w punkcie $-8$ wykres odbija się od osi poziomej.\\
A więc $$x \in \{-8\} \cup [18,20].$$
\rozwStop
\odpStart
$x \in \{-8\} \cup [18,20]$
\odpStop
\testStart
A.$x \in \{-8\} \cup [18,20]$\\
B.$x \in \{8\} \cup (18,20)$\\
C.$x \in \{-8\} \cup (18,20]$\\
D.$x \in \{8\} \cup (18,20]$\\
E.$x \in \{-8\} \cup [18,20)$\\
F.$x \in \{8\} \cup [18,20)$\\
G.$x \in \{-8\} \cup (18,20)$\\
H.$x \in \{8\} \cup [18,20]$
\testStop
\kluczStart
A
\kluczStop



\zadStart{Zadanie z Wikieł Z 1.62 c) moja wersja nr 1105}

Rozwiązać nierówności $(18-x)(x+9)^{2}(19-x)^{3}\le0$.
\zadStop
\rozwStart{Patryk Wirkus}{}
Miejsca zerowe naszego wielomianu to: $18, -9, 19$.\\
Wielomian jest stopnia parzystego, ponadto znak współczynnika przy\linebreak najwyższej potędze x jest ujemny.\\ W związku z tym wykres wielomianu zaczyna się od lewej strony powyżej osi OX.\\
Ponadto w punkcie $-9$ wykres odbija się od osi poziomej.\\
A więc $$x \in \{-9\} \cup [18,19].$$
\rozwStop
\odpStart
$x \in \{-9\} \cup [18,19]$
\odpStop
\testStart
A.$x \in \{-9\} \cup [18,19]$\\
B.$x \in \{9\} \cup (18,19)$\\
C.$x \in \{-9\} \cup (18,19]$\\
D.$x \in \{9\} \cup (18,19]$\\
E.$x \in \{-9\} \cup [18,19)$\\
F.$x \in \{9\} \cup [18,19)$\\
G.$x \in \{-9\} \cup (18,19)$\\
H.$x \in \{9\} \cup [18,19]$
\testStop
\kluczStart
A
\kluczStop



\zadStart{Zadanie z Wikieł Z 1.62 c) moja wersja nr 1106}

Rozwiązać nierówności $(18-x)(x+9)^{2}(20-x)^{3}\le0$.
\zadStop
\rozwStart{Patryk Wirkus}{}
Miejsca zerowe naszego wielomianu to: $18, -9, 20$.\\
Wielomian jest stopnia parzystego, ponadto znak współczynnika przy\linebreak najwyższej potędze x jest ujemny.\\ W związku z tym wykres wielomianu zaczyna się od lewej strony powyżej osi OX.\\
Ponadto w punkcie $-9$ wykres odbija się od osi poziomej.\\
A więc $$x \in \{-9\} \cup [18,20].$$
\rozwStop
\odpStart
$x \in \{-9\} \cup [18,20]$
\odpStop
\testStart
A.$x \in \{-9\} \cup [18,20]$\\
B.$x \in \{9\} \cup (18,20)$\\
C.$x \in \{-9\} \cup (18,20]$\\
D.$x \in \{9\} \cup (18,20]$\\
E.$x \in \{-9\} \cup [18,20)$\\
F.$x \in \{9\} \cup [18,20)$\\
G.$x \in \{-9\} \cup (18,20)$\\
H.$x \in \{9\} \cup [18,20]$
\testStop
\kluczStart
A
\kluczStop



\zadStart{Zadanie z Wikieł Z 1.62 c) moja wersja nr 1107}

Rozwiązać nierówności $(18-x)(x+10)^{2}(19-x)^{3}\le0$.
\zadStop
\rozwStart{Patryk Wirkus}{}
Miejsca zerowe naszego wielomianu to: $18, -10, 19$.\\
Wielomian jest stopnia parzystego, ponadto znak współczynnika przy\linebreak najwyższej potędze x jest ujemny.\\ W związku z tym wykres wielomianu zaczyna się od lewej strony powyżej osi OX.\\
Ponadto w punkcie $-10$ wykres odbija się od osi poziomej.\\
A więc $$x \in \{-10\} \cup [18,19].$$
\rozwStop
\odpStart
$x \in \{-10\} \cup [18,19]$
\odpStop
\testStart
A.$x \in \{-10\} \cup [18,19]$\\
B.$x \in \{10\} \cup (18,19)$\\
C.$x \in \{-10\} \cup (18,19]$\\
D.$x \in \{10\} \cup (18,19]$\\
E.$x \in \{-10\} \cup [18,19)$\\
F.$x \in \{10\} \cup [18,19)$\\
G.$x \in \{-10\} \cup (18,19)$\\
H.$x \in \{10\} \cup [18,19]$
\testStop
\kluczStart
A
\kluczStop



\zadStart{Zadanie z Wikieł Z 1.62 c) moja wersja nr 1108}

Rozwiązać nierówności $(18-x)(x+10)^{2}(20-x)^{3}\le0$.
\zadStop
\rozwStart{Patryk Wirkus}{}
Miejsca zerowe naszego wielomianu to: $18, -10, 20$.\\
Wielomian jest stopnia parzystego, ponadto znak współczynnika przy\linebreak najwyższej potędze x jest ujemny.\\ W związku z tym wykres wielomianu zaczyna się od lewej strony powyżej osi OX.\\
Ponadto w punkcie $-10$ wykres odbija się od osi poziomej.\\
A więc $$x \in \{-10\} \cup [18,20].$$
\rozwStop
\odpStart
$x \in \{-10\} \cup [18,20]$
\odpStop
\testStart
A.$x \in \{-10\} \cup [18,20]$\\
B.$x \in \{10\} \cup (18,20)$\\
C.$x \in \{-10\} \cup (18,20]$\\
D.$x \in \{10\} \cup (18,20]$\\
E.$x \in \{-10\} \cup [18,20)$\\
F.$x \in \{10\} \cup [18,20)$\\
G.$x \in \{-10\} \cup (18,20)$\\
H.$x \in \{10\} \cup [18,20]$
\testStop
\kluczStart
A
\kluczStop



\zadStart{Zadanie z Wikieł Z 1.62 c) moja wersja nr 1109}

Rozwiązać nierówności $(18-x)(x+11)^{2}(19-x)^{3}\le0$.
\zadStop
\rozwStart{Patryk Wirkus}{}
Miejsca zerowe naszego wielomianu to: $18, -11, 19$.\\
Wielomian jest stopnia parzystego, ponadto znak współczynnika przy\linebreak najwyższej potędze x jest ujemny.\\ W związku z tym wykres wielomianu zaczyna się od lewej strony powyżej osi OX.\\
Ponadto w punkcie $-11$ wykres odbija się od osi poziomej.\\
A więc $$x \in \{-11\} \cup [18,19].$$
\rozwStop
\odpStart
$x \in \{-11\} \cup [18,19]$
\odpStop
\testStart
A.$x \in \{-11\} \cup [18,19]$\\
B.$x \in \{11\} \cup (18,19)$\\
C.$x \in \{-11\} \cup (18,19]$\\
D.$x \in \{11\} \cup (18,19]$\\
E.$x \in \{-11\} \cup [18,19)$\\
F.$x \in \{11\} \cup [18,19)$\\
G.$x \in \{-11\} \cup (18,19)$\\
H.$x \in \{11\} \cup [18,19]$
\testStop
\kluczStart
A
\kluczStop



\zadStart{Zadanie z Wikieł Z 1.62 c) moja wersja nr 1110}

Rozwiązać nierówności $(18-x)(x+11)^{2}(20-x)^{3}\le0$.
\zadStop
\rozwStart{Patryk Wirkus}{}
Miejsca zerowe naszego wielomianu to: $18, -11, 20$.\\
Wielomian jest stopnia parzystego, ponadto znak współczynnika przy\linebreak najwyższej potędze x jest ujemny.\\ W związku z tym wykres wielomianu zaczyna się od lewej strony powyżej osi OX.\\
Ponadto w punkcie $-11$ wykres odbija się od osi poziomej.\\
A więc $$x \in \{-11\} \cup [18,20].$$
\rozwStop
\odpStart
$x \in \{-11\} \cup [18,20]$
\odpStop
\testStart
A.$x \in \{-11\} \cup [18,20]$\\
B.$x \in \{11\} \cup (18,20)$\\
C.$x \in \{-11\} \cup (18,20]$\\
D.$x \in \{11\} \cup (18,20]$\\
E.$x \in \{-11\} \cup [18,20)$\\
F.$x \in \{11\} \cup [18,20)$\\
G.$x \in \{-11\} \cup (18,20)$\\
H.$x \in \{11\} \cup [18,20]$
\testStop
\kluczStart
A
\kluczStop



\zadStart{Zadanie z Wikieł Z 1.62 c) moja wersja nr 1111}

Rozwiązać nierówności $(18-x)(x+12)^{2}(19-x)^{3}\le0$.
\zadStop
\rozwStart{Patryk Wirkus}{}
Miejsca zerowe naszego wielomianu to: $18, -12, 19$.\\
Wielomian jest stopnia parzystego, ponadto znak współczynnika przy\linebreak najwyższej potędze x jest ujemny.\\ W związku z tym wykres wielomianu zaczyna się od lewej strony powyżej osi OX.\\
Ponadto w punkcie $-12$ wykres odbija się od osi poziomej.\\
A więc $$x \in \{-12\} \cup [18,19].$$
\rozwStop
\odpStart
$x \in \{-12\} \cup [18,19]$
\odpStop
\testStart
A.$x \in \{-12\} \cup [18,19]$\\
B.$x \in \{12\} \cup (18,19)$\\
C.$x \in \{-12\} \cup (18,19]$\\
D.$x \in \{12\} \cup (18,19]$\\
E.$x \in \{-12\} \cup [18,19)$\\
F.$x \in \{12\} \cup [18,19)$\\
G.$x \in \{-12\} \cup (18,19)$\\
H.$x \in \{12\} \cup [18,19]$
\testStop
\kluczStart
A
\kluczStop



\zadStart{Zadanie z Wikieł Z 1.62 c) moja wersja nr 1112}

Rozwiązać nierówności $(18-x)(x+12)^{2}(20-x)^{3}\le0$.
\zadStop
\rozwStart{Patryk Wirkus}{}
Miejsca zerowe naszego wielomianu to: $18, -12, 20$.\\
Wielomian jest stopnia parzystego, ponadto znak współczynnika przy\linebreak najwyższej potędze x jest ujemny.\\ W związku z tym wykres wielomianu zaczyna się od lewej strony powyżej osi OX.\\
Ponadto w punkcie $-12$ wykres odbija się od osi poziomej.\\
A więc $$x \in \{-12\} \cup [18,20].$$
\rozwStop
\odpStart
$x \in \{-12\} \cup [18,20]$
\odpStop
\testStart
A.$x \in \{-12\} \cup [18,20]$\\
B.$x \in \{12\} \cup (18,20)$\\
C.$x \in \{-12\} \cup (18,20]$\\
D.$x \in \{12\} \cup (18,20]$\\
E.$x \in \{-12\} \cup [18,20)$\\
F.$x \in \{12\} \cup [18,20)$\\
G.$x \in \{-12\} \cup (18,20)$\\
H.$x \in \{12\} \cup [18,20]$
\testStop
\kluczStart
A
\kluczStop



\zadStart{Zadanie z Wikieł Z 1.62 c) moja wersja nr 1113}

Rozwiązać nierówności $(18-x)(x+13)^{2}(19-x)^{3}\le0$.
\zadStop
\rozwStart{Patryk Wirkus}{}
Miejsca zerowe naszego wielomianu to: $18, -13, 19$.\\
Wielomian jest stopnia parzystego, ponadto znak współczynnika przy\linebreak najwyższej potędze x jest ujemny.\\ W związku z tym wykres wielomianu zaczyna się od lewej strony powyżej osi OX.\\
Ponadto w punkcie $-13$ wykres odbija się od osi poziomej.\\
A więc $$x \in \{-13\} \cup [18,19].$$
\rozwStop
\odpStart
$x \in \{-13\} \cup [18,19]$
\odpStop
\testStart
A.$x \in \{-13\} \cup [18,19]$\\
B.$x \in \{13\} \cup (18,19)$\\
C.$x \in \{-13\} \cup (18,19]$\\
D.$x \in \{13\} \cup (18,19]$\\
E.$x \in \{-13\} \cup [18,19)$\\
F.$x \in \{13\} \cup [18,19)$\\
G.$x \in \{-13\} \cup (18,19)$\\
H.$x \in \{13\} \cup [18,19]$
\testStop
\kluczStart
A
\kluczStop



\zadStart{Zadanie z Wikieł Z 1.62 c) moja wersja nr 1114}

Rozwiązać nierówności $(18-x)(x+13)^{2}(20-x)^{3}\le0$.
\zadStop
\rozwStart{Patryk Wirkus}{}
Miejsca zerowe naszego wielomianu to: $18, -13, 20$.\\
Wielomian jest stopnia parzystego, ponadto znak współczynnika przy\linebreak najwyższej potędze x jest ujemny.\\ W związku z tym wykres wielomianu zaczyna się od lewej strony powyżej osi OX.\\
Ponadto w punkcie $-13$ wykres odbija się od osi poziomej.\\
A więc $$x \in \{-13\} \cup [18,20].$$
\rozwStop
\odpStart
$x \in \{-13\} \cup [18,20]$
\odpStop
\testStart
A.$x \in \{-13\} \cup [18,20]$\\
B.$x \in \{13\} \cup (18,20)$\\
C.$x \in \{-13\} \cup (18,20]$\\
D.$x \in \{13\} \cup (18,20]$\\
E.$x \in \{-13\} \cup [18,20)$\\
F.$x \in \{13\} \cup [18,20)$\\
G.$x \in \{-13\} \cup (18,20)$\\
H.$x \in \{13\} \cup [18,20]$
\testStop
\kluczStart
A
\kluczStop



\zadStart{Zadanie z Wikieł Z 1.62 c) moja wersja nr 1115}

Rozwiązać nierówności $(18-x)(x+14)^{2}(19-x)^{3}\le0$.
\zadStop
\rozwStart{Patryk Wirkus}{}
Miejsca zerowe naszego wielomianu to: $18, -14, 19$.\\
Wielomian jest stopnia parzystego, ponadto znak współczynnika przy\linebreak najwyższej potędze x jest ujemny.\\ W związku z tym wykres wielomianu zaczyna się od lewej strony powyżej osi OX.\\
Ponadto w punkcie $-14$ wykres odbija się od osi poziomej.\\
A więc $$x \in \{-14\} \cup [18,19].$$
\rozwStop
\odpStart
$x \in \{-14\} \cup [18,19]$
\odpStop
\testStart
A.$x \in \{-14\} \cup [18,19]$\\
B.$x \in \{14\} \cup (18,19)$\\
C.$x \in \{-14\} \cup (18,19]$\\
D.$x \in \{14\} \cup (18,19]$\\
E.$x \in \{-14\} \cup [18,19)$\\
F.$x \in \{14\} \cup [18,19)$\\
G.$x \in \{-14\} \cup (18,19)$\\
H.$x \in \{14\} \cup [18,19]$
\testStop
\kluczStart
A
\kluczStop



\zadStart{Zadanie z Wikieł Z 1.62 c) moja wersja nr 1116}

Rozwiązać nierówności $(18-x)(x+14)^{2}(20-x)^{3}\le0$.
\zadStop
\rozwStart{Patryk Wirkus}{}
Miejsca zerowe naszego wielomianu to: $18, -14, 20$.\\
Wielomian jest stopnia parzystego, ponadto znak współczynnika przy\linebreak najwyższej potędze x jest ujemny.\\ W związku z tym wykres wielomianu zaczyna się od lewej strony powyżej osi OX.\\
Ponadto w punkcie $-14$ wykres odbija się od osi poziomej.\\
A więc $$x \in \{-14\} \cup [18,20].$$
\rozwStop
\odpStart
$x \in \{-14\} \cup [18,20]$
\odpStop
\testStart
A.$x \in \{-14\} \cup [18,20]$\\
B.$x \in \{14\} \cup (18,20)$\\
C.$x \in \{-14\} \cup (18,20]$\\
D.$x \in \{14\} \cup (18,20]$\\
E.$x \in \{-14\} \cup [18,20)$\\
F.$x \in \{14\} \cup [18,20)$\\
G.$x \in \{-14\} \cup (18,20)$\\
H.$x \in \{14\} \cup [18,20]$
\testStop
\kluczStart
A
\kluczStop



\zadStart{Zadanie z Wikieł Z 1.62 c) moja wersja nr 1117}

Rozwiązać nierówności $(18-x)(x+15)^{2}(19-x)^{3}\le0$.
\zadStop
\rozwStart{Patryk Wirkus}{}
Miejsca zerowe naszego wielomianu to: $18, -15, 19$.\\
Wielomian jest stopnia parzystego, ponadto znak współczynnika przy\linebreak najwyższej potędze x jest ujemny.\\ W związku z tym wykres wielomianu zaczyna się od lewej strony powyżej osi OX.\\
Ponadto w punkcie $-15$ wykres odbija się od osi poziomej.\\
A więc $$x \in \{-15\} \cup [18,19].$$
\rozwStop
\odpStart
$x \in \{-15\} \cup [18,19]$
\odpStop
\testStart
A.$x \in \{-15\} \cup [18,19]$\\
B.$x \in \{15\} \cup (18,19)$\\
C.$x \in \{-15\} \cup (18,19]$\\
D.$x \in \{15\} \cup (18,19]$\\
E.$x \in \{-15\} \cup [18,19)$\\
F.$x \in \{15\} \cup [18,19)$\\
G.$x \in \{-15\} \cup (18,19)$\\
H.$x \in \{15\} \cup [18,19]$
\testStop
\kluczStart
A
\kluczStop



\zadStart{Zadanie z Wikieł Z 1.62 c) moja wersja nr 1118}

Rozwiązać nierówności $(18-x)(x+15)^{2}(20-x)^{3}\le0$.
\zadStop
\rozwStart{Patryk Wirkus}{}
Miejsca zerowe naszego wielomianu to: $18, -15, 20$.\\
Wielomian jest stopnia parzystego, ponadto znak współczynnika przy\linebreak najwyższej potędze x jest ujemny.\\ W związku z tym wykres wielomianu zaczyna się od lewej strony powyżej osi OX.\\
Ponadto w punkcie $-15$ wykres odbija się od osi poziomej.\\
A więc $$x \in \{-15\} \cup [18,20].$$
\rozwStop
\odpStart
$x \in \{-15\} \cup [18,20]$
\odpStop
\testStart
A.$x \in \{-15\} \cup [18,20]$\\
B.$x \in \{15\} \cup (18,20)$\\
C.$x \in \{-15\} \cup (18,20]$\\
D.$x \in \{15\} \cup (18,20]$\\
E.$x \in \{-15\} \cup [18,20)$\\
F.$x \in \{15\} \cup [18,20)$\\
G.$x \in \{-15\} \cup (18,20)$\\
H.$x \in \{15\} \cup [18,20]$
\testStop
\kluczStart
A
\kluczStop



\zadStart{Zadanie z Wikieł Z 1.62 c) moja wersja nr 1119}

Rozwiązać nierówności $(18-x)(x+16)^{2}(19-x)^{3}\le0$.
\zadStop
\rozwStart{Patryk Wirkus}{}
Miejsca zerowe naszego wielomianu to: $18, -16, 19$.\\
Wielomian jest stopnia parzystego, ponadto znak współczynnika przy\linebreak najwyższej potędze x jest ujemny.\\ W związku z tym wykres wielomianu zaczyna się od lewej strony powyżej osi OX.\\
Ponadto w punkcie $-16$ wykres odbija się od osi poziomej.\\
A więc $$x \in \{-16\} \cup [18,19].$$
\rozwStop
\odpStart
$x \in \{-16\} \cup [18,19]$
\odpStop
\testStart
A.$x \in \{-16\} \cup [18,19]$\\
B.$x \in \{16\} \cup (18,19)$\\
C.$x \in \{-16\} \cup (18,19]$\\
D.$x \in \{16\} \cup (18,19]$\\
E.$x \in \{-16\} \cup [18,19)$\\
F.$x \in \{16\} \cup [18,19)$\\
G.$x \in \{-16\} \cup (18,19)$\\
H.$x \in \{16\} \cup [18,19]$
\testStop
\kluczStart
A
\kluczStop



\zadStart{Zadanie z Wikieł Z 1.62 c) moja wersja nr 1120}

Rozwiązać nierówności $(18-x)(x+16)^{2}(20-x)^{3}\le0$.
\zadStop
\rozwStart{Patryk Wirkus}{}
Miejsca zerowe naszego wielomianu to: $18, -16, 20$.\\
Wielomian jest stopnia parzystego, ponadto znak współczynnika przy\linebreak najwyższej potędze x jest ujemny.\\ W związku z tym wykres wielomianu zaczyna się od lewej strony powyżej osi OX.\\
Ponadto w punkcie $-16$ wykres odbija się od osi poziomej.\\
A więc $$x \in \{-16\} \cup [18,20].$$
\rozwStop
\odpStart
$x \in \{-16\} \cup [18,20]$
\odpStop
\testStart
A.$x \in \{-16\} \cup [18,20]$\\
B.$x \in \{16\} \cup (18,20)$\\
C.$x \in \{-16\} \cup (18,20]$\\
D.$x \in \{16\} \cup (18,20]$\\
E.$x \in \{-16\} \cup [18,20)$\\
F.$x \in \{16\} \cup [18,20)$\\
G.$x \in \{-16\} \cup (18,20)$\\
H.$x \in \{16\} \cup [18,20]$
\testStop
\kluczStart
A
\kluczStop



\zadStart{Zadanie z Wikieł Z 1.62 c) moja wersja nr 1121}

Rozwiązać nierówności $(18-x)(x+17)^{2}(19-x)^{3}\le0$.
\zadStop
\rozwStart{Patryk Wirkus}{}
Miejsca zerowe naszego wielomianu to: $18, -17, 19$.\\
Wielomian jest stopnia parzystego, ponadto znak współczynnika przy\linebreak najwyższej potędze x jest ujemny.\\ W związku z tym wykres wielomianu zaczyna się od lewej strony powyżej osi OX.\\
Ponadto w punkcie $-17$ wykres odbija się od osi poziomej.\\
A więc $$x \in \{-17\} \cup [18,19].$$
\rozwStop
\odpStart
$x \in \{-17\} \cup [18,19]$
\odpStop
\testStart
A.$x \in \{-17\} \cup [18,19]$\\
B.$x \in \{17\} \cup (18,19)$\\
C.$x \in \{-17\} \cup (18,19]$\\
D.$x \in \{17\} \cup (18,19]$\\
E.$x \in \{-17\} \cup [18,19)$\\
F.$x \in \{17\} \cup [18,19)$\\
G.$x \in \{-17\} \cup (18,19)$\\
H.$x \in \{17\} \cup [18,19]$
\testStop
\kluczStart
A
\kluczStop



\zadStart{Zadanie z Wikieł Z 1.62 c) moja wersja nr 1122}

Rozwiązać nierówności $(18-x)(x+17)^{2}(20-x)^{3}\le0$.
\zadStop
\rozwStart{Patryk Wirkus}{}
Miejsca zerowe naszego wielomianu to: $18, -17, 20$.\\
Wielomian jest stopnia parzystego, ponadto znak współczynnika przy\linebreak najwyższej potędze x jest ujemny.\\ W związku z tym wykres wielomianu zaczyna się od lewej strony powyżej osi OX.\\
Ponadto w punkcie $-17$ wykres odbija się od osi poziomej.\\
A więc $$x \in \{-17\} \cup [18,20].$$
\rozwStop
\odpStart
$x \in \{-17\} \cup [18,20]$
\odpStop
\testStart
A.$x \in \{-17\} \cup [18,20]$\\
B.$x \in \{17\} \cup (18,20)$\\
C.$x \in \{-17\} \cup (18,20]$\\
D.$x \in \{17\} \cup (18,20]$\\
E.$x \in \{-17\} \cup [18,20)$\\
F.$x \in \{17\} \cup [18,20)$\\
G.$x \in \{-17\} \cup (18,20)$\\
H.$x \in \{17\} \cup [18,20]$
\testStop
\kluczStart
A
\kluczStop



\zadStart{Zadanie z Wikieł Z 1.62 c) moja wersja nr 1123}

Rozwiązać nierówności $(19-x)(x+1)^{2}(20-x)^{3}\le0$.
\zadStop
\rozwStart{Patryk Wirkus}{}
Miejsca zerowe naszego wielomianu to: $19, -1, 20$.\\
Wielomian jest stopnia parzystego, ponadto znak współczynnika przy\linebreak najwyższej potędze x jest ujemny.\\ W związku z tym wykres wielomianu zaczyna się od lewej strony powyżej osi OX.\\
Ponadto w punkcie $-1$ wykres odbija się od osi poziomej.\\
A więc $$x \in \{-1\} \cup [19,20].$$
\rozwStop
\odpStart
$x \in \{-1\} \cup [19,20]$
\odpStop
\testStart
A.$x \in \{-1\} \cup [19,20]$\\
B.$x \in \{1\} \cup (19,20)$\\
C.$x \in \{-1\} \cup (19,20]$\\
D.$x \in \{1\} \cup (19,20]$\\
E.$x \in \{-1\} \cup [19,20)$\\
F.$x \in \{1\} \cup [19,20)$\\
G.$x \in \{-1\} \cup (19,20)$\\
H.$x \in \{1\} \cup [19,20]$
\testStop
\kluczStart
A
\kluczStop



\zadStart{Zadanie z Wikieł Z 1.62 c) moja wersja nr 1124}

Rozwiązać nierówności $(19-x)(x+2)^{2}(20-x)^{3}\le0$.
\zadStop
\rozwStart{Patryk Wirkus}{}
Miejsca zerowe naszego wielomianu to: $19, -2, 20$.\\
Wielomian jest stopnia parzystego, ponadto znak współczynnika przy\linebreak najwyższej potędze x jest ujemny.\\ W związku z tym wykres wielomianu zaczyna się od lewej strony powyżej osi OX.\\
Ponadto w punkcie $-2$ wykres odbija się od osi poziomej.\\
A więc $$x \in \{-2\} \cup [19,20].$$
\rozwStop
\odpStart
$x \in \{-2\} \cup [19,20]$
\odpStop
\testStart
A.$x \in \{-2\} \cup [19,20]$\\
B.$x \in \{2\} \cup (19,20)$\\
C.$x \in \{-2\} \cup (19,20]$\\
D.$x \in \{2\} \cup (19,20]$\\
E.$x \in \{-2\} \cup [19,20)$\\
F.$x \in \{2\} \cup [19,20)$\\
G.$x \in \{-2\} \cup (19,20)$\\
H.$x \in \{2\} \cup [19,20]$
\testStop
\kluczStart
A
\kluczStop



\zadStart{Zadanie z Wikieł Z 1.62 c) moja wersja nr 1125}

Rozwiązać nierówności $(19-x)(x+3)^{2}(20-x)^{3}\le0$.
\zadStop
\rozwStart{Patryk Wirkus}{}
Miejsca zerowe naszego wielomianu to: $19, -3, 20$.\\
Wielomian jest stopnia parzystego, ponadto znak współczynnika przy\linebreak najwyższej potędze x jest ujemny.\\ W związku z tym wykres wielomianu zaczyna się od lewej strony powyżej osi OX.\\
Ponadto w punkcie $-3$ wykres odbija się od osi poziomej.\\
A więc $$x \in \{-3\} \cup [19,20].$$
\rozwStop
\odpStart
$x \in \{-3\} \cup [19,20]$
\odpStop
\testStart
A.$x \in \{-3\} \cup [19,20]$\\
B.$x \in \{3\} \cup (19,20)$\\
C.$x \in \{-3\} \cup (19,20]$\\
D.$x \in \{3\} \cup (19,20]$\\
E.$x \in \{-3\} \cup [19,20)$\\
F.$x \in \{3\} \cup [19,20)$\\
G.$x \in \{-3\} \cup (19,20)$\\
H.$x \in \{3\} \cup [19,20]$
\testStop
\kluczStart
A
\kluczStop



\zadStart{Zadanie z Wikieł Z 1.62 c) moja wersja nr 1126}

Rozwiązać nierówności $(19-x)(x+4)^{2}(20-x)^{3}\le0$.
\zadStop
\rozwStart{Patryk Wirkus}{}
Miejsca zerowe naszego wielomianu to: $19, -4, 20$.\\
Wielomian jest stopnia parzystego, ponadto znak współczynnika przy\linebreak najwyższej potędze x jest ujemny.\\ W związku z tym wykres wielomianu zaczyna się od lewej strony powyżej osi OX.\\
Ponadto w punkcie $-4$ wykres odbija się od osi poziomej.\\
A więc $$x \in \{-4\} \cup [19,20].$$
\rozwStop
\odpStart
$x \in \{-4\} \cup [19,20]$
\odpStop
\testStart
A.$x \in \{-4\} \cup [19,20]$\\
B.$x \in \{4\} \cup (19,20)$\\
C.$x \in \{-4\} \cup (19,20]$\\
D.$x \in \{4\} \cup (19,20]$\\
E.$x \in \{-4\} \cup [19,20)$\\
F.$x \in \{4\} \cup [19,20)$\\
G.$x \in \{-4\} \cup (19,20)$\\
H.$x \in \{4\} \cup [19,20]$
\testStop
\kluczStart
A
\kluczStop



\zadStart{Zadanie z Wikieł Z 1.62 c) moja wersja nr 1127}

Rozwiązać nierówności $(19-x)(x+5)^{2}(20-x)^{3}\le0$.
\zadStop
\rozwStart{Patryk Wirkus}{}
Miejsca zerowe naszego wielomianu to: $19, -5, 20$.\\
Wielomian jest stopnia parzystego, ponadto znak współczynnika przy\linebreak najwyższej potędze x jest ujemny.\\ W związku z tym wykres wielomianu zaczyna się od lewej strony powyżej osi OX.\\
Ponadto w punkcie $-5$ wykres odbija się od osi poziomej.\\
A więc $$x \in \{-5\} \cup [19,20].$$
\rozwStop
\odpStart
$x \in \{-5\} \cup [19,20]$
\odpStop
\testStart
A.$x \in \{-5\} \cup [19,20]$\\
B.$x \in \{5\} \cup (19,20)$\\
C.$x \in \{-5\} \cup (19,20]$\\
D.$x \in \{5\} \cup (19,20]$\\
E.$x \in \{-5\} \cup [19,20)$\\
F.$x \in \{5\} \cup [19,20)$\\
G.$x \in \{-5\} \cup (19,20)$\\
H.$x \in \{5\} \cup [19,20]$
\testStop
\kluczStart
A
\kluczStop



\zadStart{Zadanie z Wikieł Z 1.62 c) moja wersja nr 1128}

Rozwiązać nierówności $(19-x)(x+6)^{2}(20-x)^{3}\le0$.
\zadStop
\rozwStart{Patryk Wirkus}{}
Miejsca zerowe naszego wielomianu to: $19, -6, 20$.\\
Wielomian jest stopnia parzystego, ponadto znak współczynnika przy\linebreak najwyższej potędze x jest ujemny.\\ W związku z tym wykres wielomianu zaczyna się od lewej strony powyżej osi OX.\\
Ponadto w punkcie $-6$ wykres odbija się od osi poziomej.\\
A więc $$x \in \{-6\} \cup [19,20].$$
\rozwStop
\odpStart
$x \in \{-6\} \cup [19,20]$
\odpStop
\testStart
A.$x \in \{-6\} \cup [19,20]$\\
B.$x \in \{6\} \cup (19,20)$\\
C.$x \in \{-6\} \cup (19,20]$\\
D.$x \in \{6\} \cup (19,20]$\\
E.$x \in \{-6\} \cup [19,20)$\\
F.$x \in \{6\} \cup [19,20)$\\
G.$x \in \{-6\} \cup (19,20)$\\
H.$x \in \{6\} \cup [19,20]$
\testStop
\kluczStart
A
\kluczStop



\zadStart{Zadanie z Wikieł Z 1.62 c) moja wersja nr 1129}

Rozwiązać nierówności $(19-x)(x+7)^{2}(20-x)^{3}\le0$.
\zadStop
\rozwStart{Patryk Wirkus}{}
Miejsca zerowe naszego wielomianu to: $19, -7, 20$.\\
Wielomian jest stopnia parzystego, ponadto znak współczynnika przy\linebreak najwyższej potędze x jest ujemny.\\ W związku z tym wykres wielomianu zaczyna się od lewej strony powyżej osi OX.\\
Ponadto w punkcie $-7$ wykres odbija się od osi poziomej.\\
A więc $$x \in \{-7\} \cup [19,20].$$
\rozwStop
\odpStart
$x \in \{-7\} \cup [19,20]$
\odpStop
\testStart
A.$x \in \{-7\} \cup [19,20]$\\
B.$x \in \{7\} \cup (19,20)$\\
C.$x \in \{-7\} \cup (19,20]$\\
D.$x \in \{7\} \cup (19,20]$\\
E.$x \in \{-7\} \cup [19,20)$\\
F.$x \in \{7\} \cup [19,20)$\\
G.$x \in \{-7\} \cup (19,20)$\\
H.$x \in \{7\} \cup [19,20]$
\testStop
\kluczStart
A
\kluczStop



\zadStart{Zadanie z Wikieł Z 1.62 c) moja wersja nr 1130}

Rozwiązać nierówności $(19-x)(x+8)^{2}(20-x)^{3}\le0$.
\zadStop
\rozwStart{Patryk Wirkus}{}
Miejsca zerowe naszego wielomianu to: $19, -8, 20$.\\
Wielomian jest stopnia parzystego, ponadto znak współczynnika przy\linebreak najwyższej potędze x jest ujemny.\\ W związku z tym wykres wielomianu zaczyna się od lewej strony powyżej osi OX.\\
Ponadto w punkcie $-8$ wykres odbija się od osi poziomej.\\
A więc $$x \in \{-8\} \cup [19,20].$$
\rozwStop
\odpStart
$x \in \{-8\} \cup [19,20]$
\odpStop
\testStart
A.$x \in \{-8\} \cup [19,20]$\\
B.$x \in \{8\} \cup (19,20)$\\
C.$x \in \{-8\} \cup (19,20]$\\
D.$x \in \{8\} \cup (19,20]$\\
E.$x \in \{-8\} \cup [19,20)$\\
F.$x \in \{8\} \cup [19,20)$\\
G.$x \in \{-8\} \cup (19,20)$\\
H.$x \in \{8\} \cup [19,20]$
\testStop
\kluczStart
A
\kluczStop



\zadStart{Zadanie z Wikieł Z 1.62 c) moja wersja nr 1131}

Rozwiązać nierówności $(19-x)(x+9)^{2}(20-x)^{3}\le0$.
\zadStop
\rozwStart{Patryk Wirkus}{}
Miejsca zerowe naszego wielomianu to: $19, -9, 20$.\\
Wielomian jest stopnia parzystego, ponadto znak współczynnika przy\linebreak najwyższej potędze x jest ujemny.\\ W związku z tym wykres wielomianu zaczyna się od lewej strony powyżej osi OX.\\
Ponadto w punkcie $-9$ wykres odbija się od osi poziomej.\\
A więc $$x \in \{-9\} \cup [19,20].$$
\rozwStop
\odpStart
$x \in \{-9\} \cup [19,20]$
\odpStop
\testStart
A.$x \in \{-9\} \cup [19,20]$\\
B.$x \in \{9\} \cup (19,20)$\\
C.$x \in \{-9\} \cup (19,20]$\\
D.$x \in \{9\} \cup (19,20]$\\
E.$x \in \{-9\} \cup [19,20)$\\
F.$x \in \{9\} \cup [19,20)$\\
G.$x \in \{-9\} \cup (19,20)$\\
H.$x \in \{9\} \cup [19,20]$
\testStop
\kluczStart
A
\kluczStop



\zadStart{Zadanie z Wikieł Z 1.62 c) moja wersja nr 1132}

Rozwiązać nierówności $(19-x)(x+10)^{2}(20-x)^{3}\le0$.
\zadStop
\rozwStart{Patryk Wirkus}{}
Miejsca zerowe naszego wielomianu to: $19, -10, 20$.\\
Wielomian jest stopnia parzystego, ponadto znak współczynnika przy\linebreak najwyższej potędze x jest ujemny.\\ W związku z tym wykres wielomianu zaczyna się od lewej strony powyżej osi OX.\\
Ponadto w punkcie $-10$ wykres odbija się od osi poziomej.\\
A więc $$x \in \{-10\} \cup [19,20].$$
\rozwStop
\odpStart
$x \in \{-10\} \cup [19,20]$
\odpStop
\testStart
A.$x \in \{-10\} \cup [19,20]$\\
B.$x \in \{10\} \cup (19,20)$\\
C.$x \in \{-10\} \cup (19,20]$\\
D.$x \in \{10\} \cup (19,20]$\\
E.$x \in \{-10\} \cup [19,20)$\\
F.$x \in \{10\} \cup [19,20)$\\
G.$x \in \{-10\} \cup (19,20)$\\
H.$x \in \{10\} \cup [19,20]$
\testStop
\kluczStart
A
\kluczStop



\zadStart{Zadanie z Wikieł Z 1.62 c) moja wersja nr 1133}

Rozwiązać nierówności $(19-x)(x+11)^{2}(20-x)^{3}\le0$.
\zadStop
\rozwStart{Patryk Wirkus}{}
Miejsca zerowe naszego wielomianu to: $19, -11, 20$.\\
Wielomian jest stopnia parzystego, ponadto znak współczynnika przy\linebreak najwyższej potędze x jest ujemny.\\ W związku z tym wykres wielomianu zaczyna się od lewej strony powyżej osi OX.\\
Ponadto w punkcie $-11$ wykres odbija się od osi poziomej.\\
A więc $$x \in \{-11\} \cup [19,20].$$
\rozwStop
\odpStart
$x \in \{-11\} \cup [19,20]$
\odpStop
\testStart
A.$x \in \{-11\} \cup [19,20]$\\
B.$x \in \{11\} \cup (19,20)$\\
C.$x \in \{-11\} \cup (19,20]$\\
D.$x \in \{11\} \cup (19,20]$\\
E.$x \in \{-11\} \cup [19,20)$\\
F.$x \in \{11\} \cup [19,20)$\\
G.$x \in \{-11\} \cup (19,20)$\\
H.$x \in \{11\} \cup [19,20]$
\testStop
\kluczStart
A
\kluczStop



\zadStart{Zadanie z Wikieł Z 1.62 c) moja wersja nr 1134}

Rozwiązać nierówności $(19-x)(x+12)^{2}(20-x)^{3}\le0$.
\zadStop
\rozwStart{Patryk Wirkus}{}
Miejsca zerowe naszego wielomianu to: $19, -12, 20$.\\
Wielomian jest stopnia parzystego, ponadto znak współczynnika przy\linebreak najwyższej potędze x jest ujemny.\\ W związku z tym wykres wielomianu zaczyna się od lewej strony powyżej osi OX.\\
Ponadto w punkcie $-12$ wykres odbija się od osi poziomej.\\
A więc $$x \in \{-12\} \cup [19,20].$$
\rozwStop
\odpStart
$x \in \{-12\} \cup [19,20]$
\odpStop
\testStart
A.$x \in \{-12\} \cup [19,20]$\\
B.$x \in \{12\} \cup (19,20)$\\
C.$x \in \{-12\} \cup (19,20]$\\
D.$x \in \{12\} \cup (19,20]$\\
E.$x \in \{-12\} \cup [19,20)$\\
F.$x \in \{12\} \cup [19,20)$\\
G.$x \in \{-12\} \cup (19,20)$\\
H.$x \in \{12\} \cup [19,20]$
\testStop
\kluczStart
A
\kluczStop



\zadStart{Zadanie z Wikieł Z 1.62 c) moja wersja nr 1135}

Rozwiązać nierówności $(19-x)(x+13)^{2}(20-x)^{3}\le0$.
\zadStop
\rozwStart{Patryk Wirkus}{}
Miejsca zerowe naszego wielomianu to: $19, -13, 20$.\\
Wielomian jest stopnia parzystego, ponadto znak współczynnika przy\linebreak najwyższej potędze x jest ujemny.\\ W związku z tym wykres wielomianu zaczyna się od lewej strony powyżej osi OX.\\
Ponadto w punkcie $-13$ wykres odbija się od osi poziomej.\\
A więc $$x \in \{-13\} \cup [19,20].$$
\rozwStop
\odpStart
$x \in \{-13\} \cup [19,20]$
\odpStop
\testStart
A.$x \in \{-13\} \cup [19,20]$\\
B.$x \in \{13\} \cup (19,20)$\\
C.$x \in \{-13\} \cup (19,20]$\\
D.$x \in \{13\} \cup (19,20]$\\
E.$x \in \{-13\} \cup [19,20)$\\
F.$x \in \{13\} \cup [19,20)$\\
G.$x \in \{-13\} \cup (19,20)$\\
H.$x \in \{13\} \cup [19,20]$
\testStop
\kluczStart
A
\kluczStop



\zadStart{Zadanie z Wikieł Z 1.62 c) moja wersja nr 1136}

Rozwiązać nierówności $(19-x)(x+14)^{2}(20-x)^{3}\le0$.
\zadStop
\rozwStart{Patryk Wirkus}{}
Miejsca zerowe naszego wielomianu to: $19, -14, 20$.\\
Wielomian jest stopnia parzystego, ponadto znak współczynnika przy\linebreak najwyższej potędze x jest ujemny.\\ W związku z tym wykres wielomianu zaczyna się od lewej strony powyżej osi OX.\\
Ponadto w punkcie $-14$ wykres odbija się od osi poziomej.\\
A więc $$x \in \{-14\} \cup [19,20].$$
\rozwStop
\odpStart
$x \in \{-14\} \cup [19,20]$
\odpStop
\testStart
A.$x \in \{-14\} \cup [19,20]$\\
B.$x \in \{14\} \cup (19,20)$\\
C.$x \in \{-14\} \cup (19,20]$\\
D.$x \in \{14\} \cup (19,20]$\\
E.$x \in \{-14\} \cup [19,20)$\\
F.$x \in \{14\} \cup [19,20)$\\
G.$x \in \{-14\} \cup (19,20)$\\
H.$x \in \{14\} \cup [19,20]$
\testStop
\kluczStart
A
\kluczStop



\zadStart{Zadanie z Wikieł Z 1.62 c) moja wersja nr 1137}

Rozwiązać nierówności $(19-x)(x+15)^{2}(20-x)^{3}\le0$.
\zadStop
\rozwStart{Patryk Wirkus}{}
Miejsca zerowe naszego wielomianu to: $19, -15, 20$.\\
Wielomian jest stopnia parzystego, ponadto znak współczynnika przy\linebreak najwyższej potędze x jest ujemny.\\ W związku z tym wykres wielomianu zaczyna się od lewej strony powyżej osi OX.\\
Ponadto w punkcie $-15$ wykres odbija się od osi poziomej.\\
A więc $$x \in \{-15\} \cup [19,20].$$
\rozwStop
\odpStart
$x \in \{-15\} \cup [19,20]$
\odpStop
\testStart
A.$x \in \{-15\} \cup [19,20]$\\
B.$x \in \{15\} \cup (19,20)$\\
C.$x \in \{-15\} \cup (19,20]$\\
D.$x \in \{15\} \cup (19,20]$\\
E.$x \in \{-15\} \cup [19,20)$\\
F.$x \in \{15\} \cup [19,20)$\\
G.$x \in \{-15\} \cup (19,20)$\\
H.$x \in \{15\} \cup [19,20]$
\testStop
\kluczStart
A
\kluczStop



\zadStart{Zadanie z Wikieł Z 1.62 c) moja wersja nr 1138}

Rozwiązać nierówności $(19-x)(x+16)^{2}(20-x)^{3}\le0$.
\zadStop
\rozwStart{Patryk Wirkus}{}
Miejsca zerowe naszego wielomianu to: $19, -16, 20$.\\
Wielomian jest stopnia parzystego, ponadto znak współczynnika przy\linebreak najwyższej potędze x jest ujemny.\\ W związku z tym wykres wielomianu zaczyna się od lewej strony powyżej osi OX.\\
Ponadto w punkcie $-16$ wykres odbija się od osi poziomej.\\
A więc $$x \in \{-16\} \cup [19,20].$$
\rozwStop
\odpStart
$x \in \{-16\} \cup [19,20]$
\odpStop
\testStart
A.$x \in \{-16\} \cup [19,20]$\\
B.$x \in \{16\} \cup (19,20)$\\
C.$x \in \{-16\} \cup (19,20]$\\
D.$x \in \{16\} \cup (19,20]$\\
E.$x \in \{-16\} \cup [19,20)$\\
F.$x \in \{16\} \cup [19,20)$\\
G.$x \in \{-16\} \cup (19,20)$\\
H.$x \in \{16\} \cup [19,20]$
\testStop
\kluczStart
A
\kluczStop



\zadStart{Zadanie z Wikieł Z 1.62 c) moja wersja nr 1139}

Rozwiązać nierówności $(19-x)(x+17)^{2}(20-x)^{3}\le0$.
\zadStop
\rozwStart{Patryk Wirkus}{}
Miejsca zerowe naszego wielomianu to: $19, -17, 20$.\\
Wielomian jest stopnia parzystego, ponadto znak współczynnika przy\linebreak najwyższej potędze x jest ujemny.\\ W związku z tym wykres wielomianu zaczyna się od lewej strony powyżej osi OX.\\
Ponadto w punkcie $-17$ wykres odbija się od osi poziomej.\\
A więc $$x \in \{-17\} \cup [19,20].$$
\rozwStop
\odpStart
$x \in \{-17\} \cup [19,20]$
\odpStop
\testStart
A.$x \in \{-17\} \cup [19,20]$\\
B.$x \in \{17\} \cup (19,20)$\\
C.$x \in \{-17\} \cup (19,20]$\\
D.$x \in \{17\} \cup (19,20]$\\
E.$x \in \{-17\} \cup [19,20)$\\
F.$x \in \{17\} \cup [19,20)$\\
G.$x \in \{-17\} \cup (19,20)$\\
H.$x \in \{17\} \cup [19,20]$
\testStop
\kluczStart
A
\kluczStop



\zadStart{Zadanie z Wikieł Z 1.62 c) moja wersja nr 1140}

Rozwiązać nierówności $(19-x)(x+18)^{2}(20-x)^{3}\le0$.
\zadStop
\rozwStart{Patryk Wirkus}{}
Miejsca zerowe naszego wielomianu to: $19, -18, 20$.\\
Wielomian jest stopnia parzystego, ponadto znak współczynnika przy\linebreak najwyższej potędze x jest ujemny.\\ W związku z tym wykres wielomianu zaczyna się od lewej strony powyżej osi OX.\\
Ponadto w punkcie $-18$ wykres odbija się od osi poziomej.\\
A więc $$x \in \{-18\} \cup [19,20].$$
\rozwStop
\odpStart
$x \in \{-18\} \cup [19,20]$
\odpStop
\testStart
A.$x \in \{-18\} \cup [19,20]$\\
B.$x \in \{18\} \cup (19,20)$\\
C.$x \in \{-18\} \cup (19,20]$\\
D.$x \in \{18\} \cup (19,20]$\\
E.$x \in \{-18\} \cup [19,20)$\\
F.$x \in \{18\} \cup [19,20)$\\
G.$x \in \{-18\} \cup (19,20)$\\
H.$x \in \{18\} \cup [19,20]$
\testStop
\kluczStart
A
\kluczStop





\end{document}
