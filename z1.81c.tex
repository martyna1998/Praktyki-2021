\documentclass[12pt, a4paper]{article}
\usepackage[utf8]{inputenc}
\usepackage{polski}

\usepackage{amsthm}  %pakiet do tworzenia twierdzeń itp.
\usepackage{amsmath} %pakiet do niektórych symboli matematycznych
\usepackage{amssymb} %pakiet do symboli mat., np. \nsubseteq
\usepackage{amsfonts}
\usepackage{graphicx} %obsługa plików graficznych z rozszerzeniem png, jpg
\theoremstyle{definition} %styl dla definicji
\newtheorem{zad}{} 
\title{Multizestaw zadań}
\author{Jacek Jabłoński}
%\date{\today}
\date{}
\newcounter{liczniksekcji}
\newcommand{\kategoria}[1]{\section{#1}} %olreślamy nazwę kateforii zadań
\newcommand{\zadStart}[1]{\begin{zad}#1\newline} %oznaczenie początku zadania
\newcommand{\zadStop}{\end{zad}}   %oznaczenie końca zadania
%Makra opcjonarne (nie muszą występować):
\newcommand{\rozwStart}[2]{\noindent \textbf{Rozwiązanie (autor #1 , recenzent #2): }\newline} %oznaczenie początku rozwiązania, opcjonarnie można wprowadzić informację o autorze rozwiązania zadania i recenzencie poprawności wykonania rozwiązania zadania
\newcommand{\rozwStop}{\newline}                                            %oznaczenie końca rozwiązania
\newcommand{\odpStart}{\noindent \textbf{Odpowiedź:}\newline}    %oznaczenie początku odpowiedzi końcowej (wypisanie wyniku)
\newcommand{\odpStop}{\newline}                                             %oznaczenie końca odpowiedzi końcowej (wypisanie wyniku)
\newcommand{\testStart}{\noindent \textbf{Test:}\newline} %ewentualne możliwe opcje odpowiedzi testowej: A. ? B. ? C. ? D. ? itd.
\newcommand{\testStop}{\newline} %koniec wprowadzania odpowiedzi testowych
\newcommand{\kluczStart}{\noindent \textbf{Test poprawna odpowiedź:}\newline} %klucz, poprawna odpowiedź pytania testowego (jedna literka): A lub B lub C lub D itd.
\newcommand{\kluczStop}{\newline} %koniec poprawnej odpowiedzi pytania testowego 
\newcommand{\wstawGrafike}[2]{\begin{figure}[h] \includegraphics[scale=#2] {#1} \end{figure}} %gdyby była potrzeba wstawienia obrazka, parametry: nazwa pliku, skala (jak nie wiesz co wpisać, to wpisz 1)

\begin{document}
\maketitle


\kategoria{Wikieł/z1.81c}
\zadStart{Zadanie z Wikieł z.1.81c) moja wersja nr [nrWersji]}
%[p1]:[2,3]
%[p2]:[2,3,4,5]
%[p3]:[2,3,4,5]
%[p4]:[2,3,4,5]
%[p5]:[2,3,4,5]
%[r1]=[p2]*[p4]
%[r2]=[p5]*[p4]
%[r3]=[r1]*2
%[Delta]=abs(int((math.pow([r2],2)) - 4*[r1]*[p3]))
%[PDelta]=int(math.pow([Delta],(1/2)))
%[t1]=([r2]-[PDelta])
%[t2]=([r2]+[PDelta])
%[c1]=math.sqrt([Delta])
%[c2]=math.isqrt([Delta])
%[c3]=math.gcd([t1],[r3])
%[c4]=math.gcd([t2],[r3])
%[c5]=[r2]*[r2]-4*[r1]*[p3]
%[x1]=int([t1]/[c3])
%[x2]=int([t2]/[c4])
%[x11]=int([r3]/[c3])
%[x22]=int([r3]/[c4])
%[f1]=[x1]+1
%[f2]=[x1]+2
%[f3]=[x1]+3
%[f4]=[x1]+4
%[f5]=[x2]+1
%[f6]=[x2]+2
%[f7]=[x2]+3
%[f8]=[x2]+4
%[p3]!=[p4] and not([c1]!=[c2]) and [x2]!=[x22] and [x1]!=[x11] and [x22]!=1 and [x11]!=1 and [c5]>0
Wyznaczyć wartości zmiennej x, dla których funkcje f i g mają równe wartości.
c)$f(x)=[p1]^{\frac{[p3]}{[p4]}-[p5]x} \ \ \ \ g(x)=(\frac{1}{[p1]})^{[p2]x^2}$
\zadStop
\rozwStart{Jacek Jabłoński}{}
$$f(x)=g(x)$$
$$[p1]^{\frac{[p3]}{[p4]}-[p5]x} = (\frac{1}{[p1]})^{[p2]x^2}$$
$$[p1]^{\frac{[p3]}{[p4]}-[p5]x} = [p1]^{-[p2]x^2}$$
$$\frac{[p3]}{[p4]}-[p5]x = -[p2]x^2$$
$$[p2]x^2-[p5]x+\frac{[p3]}{[p4]} = 0$$
$$[r1]x^2-[r2]x+[p3] = 0$$
$$\Delta = [Delta]$$
$$\sqrt{\Delta} = [PDelta]$$
$$x_1 = \frac{[r2]-[PDelta]}{[r3]} = \frac{[t1]}{[r3]}$$
$$x_2 = \frac{[r2]+[PDelta]}{[r3]} = \frac{[t2]}{[r3]}$$
$$x = \frac{[x1]}{[x11]} \ \ lub \ \ x = \frac{[x2]}{[x22]}$$
\rozwStop
\odpStart
$$x = \frac{[x1]}{[x11]} \ \ lub \ \ x = \frac{[x2]}{[x22]}$$
\odpStop
\testStart
A. $$x = \frac{[x1]}{[x11]} \ \ lub \ \ x = \frac{[x2]}{[x22]}$$
B. $$x = \frac{[f1]}{[x11]} \ \ lub \ \ x = \frac{[x2]}{[x22]}$$
C. $$x = \frac{[f2]}{[x11]} \ \ lub \ \ x = \frac{[x2]}{[x22]}$$
D. $$x = \frac{[x1]}{[x11]} \ \ lub \ \ x = \frac{[f3]}{[x22]}$$
E. $$x = \frac{[x1]}{[x11]} \ \ lub \ \ x = \frac{[f4]}{[x22]}$$
F. $$x = \frac{[f5]}{[x11]} \ \ lub \ \ x = \frac{[f6]}{[x22]}$$
G. $$x = \frac{[f7]}{[x11]} \ \ lub \ \ x = \frac{[f8]}{[x22]}$$
H. $$x = [x1]$$
I. $$x = [x2]$$
\testStop
\kluczStart
A
\kluczStop



\end{document}