\documentclass[12pt, a4paper]{article}
\usepackage[utf8]{inputenc}
\usepackage{polski}

\usepackage{amsthm}  %pakiet do tworzenia twierdzeń itp.
\usepackage{amsmath} %pakiet do niektórych symboli matematycznych
\usepackage{amssymb} %pakiet do symboli mat., np. \nsubseteq
\usepackage{amsfonts}
\usepackage{graphicx} %obsługa plików graficznych z rozszerzeniem png, jpg
\theoremstyle{definition} %styl dla definicji
\newtheorem{zad}{} 
\title{Multizestaw zadań}
\author{Robert Fidytek}
%\date{\today}
\date{}
\newcounter{liczniksekcji}
\newcommand{\kategoria}[1]{\section{#1}} %olreślamy nazwę kateforii zadań
\newcommand{\zadStart}[1]{\begin{zad}#1\newline} %oznaczenie początku zadania
\newcommand{\zadStop}{\end{zad}}   %oznaczenie końca zadania
%Makra opcjonarne (nie muszą występować):
\newcommand{\rozwStart}[2]{\noindent \textbf{Rozwiązanie (autor #1 , recenzent #2): }\newline} %oznaczenie początku rozwiązania, opcjonarnie można wprowadzić informację o autorze rozwiązania zadania i recenzencie poprawności wykonania rozwiązania zadania
\newcommand{\rozwStop}{\newline}                                            %oznaczenie końca rozwiązania
\newcommand{\odpStart}{\noindent \textbf{Odpowiedź:}\newline}    %oznaczenie początku odpowiedzi końcowej (wypisanie wyniku)
\newcommand{\odpStop}{\newline}                                             %oznaczenie końca odpowiedzi końcowej (wypisanie wyniku)
\newcommand{\testStart}{\noindent \textbf{Test:}\newline} %ewentualne możliwe opcje odpowiedzi testowej: A. ? B. ? C. ? D. ? itd.
\newcommand{\testStop}{\newline} %koniec wprowadzania odpowiedzi testowych
\newcommand{\kluczStart}{\noindent \textbf{Test poprawna odpowiedź:}\newline} %klucz, poprawna odpowiedź pytania testowego (jedna literka): A lub B lub C lub D itd.
\newcommand{\kluczStop}{\newline} %koniec poprawnej odpowiedzi pytania testowego 
\newcommand{\wstawGrafike}[2]{\begin{figure}[h] \includegraphics[scale=#2] {#1} \end{figure}} %gdyby była potrzeba wstawienia obrazka, parametry: nazwa pliku, skala (jak nie wiesz co wpisać, to wpisz 1)

\begin{document}
\maketitle


\kategoria{Wikieł/Z5.5l}
\zadStart{Zadanie z Wikieł Z 5.5 l) moja wersja nr [nrWersji]}
%[a]:[2,3,4,5,6,7,8,9]
%[b]:[2,3,4,5,6,7,8,9]
%[c]=random.randint(2,10)
%[d]=random.randint(2,10)
%[e1]=2*[a]*[c]
%[e2]=2*[a]*[d]
%[e3]=[a]*[d]
%[e4]=[d]*[b]
%[e5]=[e1]+[e4]
%[e6]=[e2]-[e3]
%[e7]=int([e5]/2)
%[e8]=int([e6]/2)
%math.gcd([e5],2)!=1 and math.gcd([e6],2)!=1
Wyznacz pochodną funkcji \\ $f(x)=\frac{[a]x+[b]}{\sqrt{[c]-[d]x}}$.
\zadStop
\rozwStart{Joanna Świerzbin}{}
$$f(x)=\frac{[a]x+[b]}{\sqrt{[c]-[d]x}}$$
$$f'(x)= \left( \frac{[a]x+[b]}{\sqrt{[c]-[d]x}} \right)' =$$
$$ = \frac{\left([a]x+[b]\right)'\left(\sqrt{[c]-[d]x}\right)-\left([a]x+[b]\right)\left(\sqrt{[c]-[d]x}\right)'}{\left(\sqrt{[c]-[d]x}\right)^2}  =$$
$$ = \frac{[a]\left(\sqrt{[c]-[d]x}\right)-\left([a]x+[b]\right)\left( \frac {-[d]}{2\sqrt{[c]-[d]x}}\right)}{\left(\sqrt{[c]-[d]x}\right)^2} =$$
$$ = \frac{[a]\sqrt{[c]-[d]x} + \frac {[d]\left([a]x+[b]\right)}{2\sqrt{[c]-[d]x}}}{\left(\sqrt{[c]-[d]x}\right)^2} =$$
$$ = \frac{\frac{\left([a]\sqrt{[c]-[d]x}\right)\left(2\sqrt{[c]-[d]x}\right)}{2\sqrt{[c]-[d]x}}+ \frac {[d]\cdot[a]x+[d]\cdot[b]}{2\sqrt{[c]-[d]x}}}{\left(\sqrt{[c]-[d]x}\right)^2} =$$
$$ = \frac{\frac{\left([a]\sqrt{[c]-[d]x}\right)\left(2\sqrt{[c]-[d]x}\right)+ [d]\cdot[a]x+[d]\cdot[b]}{2\sqrt{[c]-[d]x}}}{\left(\sqrt{[c]-[d]x}\right)^2} =$$
$$ = \frac{2\cdot[a]([c]-[d]x) + [d]\cdot[a]x+[d]\cdot[b]}{2\cdot \left(\sqrt{[c]-[d]x}\right)^3} =$$
$$ = \frac{2\cdot[a]\cdot[c]-2\cdot[a]\cdot[d]x + [d]\cdot[a]x+[d]\cdot[b]}{2\cdot \left(\sqrt{[c]-[d]x}\right)^3} =$$
$$ = \frac{[e1]-[e2]x+[e3]x+[e4]}{2 \cdot \left(\sqrt{[c]-[d]x}\right)^3} $$
$$ = \frac{[e7]-[e8]x}{\left(\sqrt{[c]-[d]x}\right)^3} $$
\rozwStop
\odpStart
$ f'(x) = \frac{[e7]-[e8]x}{\left(\sqrt{[c]-[d]x}\right)^3} $
\odpStop
\testStart
A. $ f'(x)= \frac{[e7]-[e8]x}{\left(\sqrt{[c]-[d]x}\right)^3} $\\
B. $ f'(x)= \frac{[e7]x}{\left(\sqrt{[c]-[d]x}\right)^3} $ \\
C. $ f'(x)= \frac{[e8]x}{\left(\sqrt{[c]-[d]x}\right)^3}$ \\
D. $ f'(x)= \frac{[e7]-[e8]x}{\left(\sqrt{[c]-[d]x}\right)^2}$\\
E. $ f'(x)= \frac{[e7]-[e8]x}{\left(\sqrt{[c]-[d]x}\right)} $\\
F. $ f'(x)= \frac{1}{\left(\sqrt{[c]-[d]x}\right)^3} $
\testStop
\kluczStart
A
\kluczStop



\end{document}