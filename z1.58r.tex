\documentclass[12pt, a4paper]{article}
\usepackage[utf8]{inputenc}
\usepackage{polski}

\usepackage{amsthm}  %pakiet do tworzenia twierdzeń itp.
\usepackage{amsmath} %pakiet do niektórych symboli matematycznych
\usepackage{amssymb} %pakiet do symboli mat., np. \nsubseteq
\usepackage{amsfonts}
\usepackage{graphicx} %obsługa plików graficznych z rozszerzeniem png, jpg
\theoremstyle{definition} %styl dla definicji
\newtheorem{zad}{} 
\title{Multizestaw zadań}
\author{Laura Mieczkowska}
%\date{\today}
\date{}
\newcounter{liczniksekcji}
\newcommand{\kategoria}[1]{\section{#1}} %olreślamy nazwę kateforii zadań
\newcommand{\zadStart}[1]{\begin{zad}#1\newline} %oznaczenie początku zadania
\newcommand{\zadStop}{\end{zad}}   %oznaczenie końca zadania
%Makra opcjonarne (nie muszą występować):
\newcommand{\rozwStart}[2]{\noindent \textbf{Rozwiązanie (autor #1 , recenzent #2): }\newline} %oznaczenie początku rozwiązania, opcjonarnie można wprowadzić informację o autorze rozwiązania zadania i recenzencie poprawności wykonania rozwiązania zadania
\newcommand{\rozwStop}{\newline}                                            %oznaczenie końca rozwiązania
\newcommand{\odpStart}{\noindent \textbf{Odpowiedź:}\newline}    %oznaczenie początku odpowiedzi końcowej (wypisanie wyniku)
\newcommand{\odpStop}{\newline}                                             %oznaczenie końca odpowiedzi końcowej (wypisanie wyniku)
\newcommand{\testStart}{\noindent \textbf{Test:}\newline} %ewentualne możliwe opcje odpowiedzi testowej: A. ? B. ? C. ? D. ? itd.
\newcommand{\testStop}{\newline} %koniec wprowadzania odpowiedzi testowych
\newcommand{\kluczStart}{\noindent \textbf{Test poprawna odpowiedź:}\newline} %klucz, poprawna odpowiedź pytania testowego (jedna literka): A lub B lub C lub D itd.
\newcommand{\kluczStop}{\newline} %koniec poprawnej odpowiedzi pytania testowego 
\newcommand{\wstawGrafike}[2]{\begin{figure}[h] \includegraphics[scale=#2] {#1} \end{figure}} %gdyby była potrzeba wstawienia obrazka, parametry: nazwa pliku, skala (jak nie wiesz co wpisać, to wpisz 1)

\begin{document}
\maketitle


\kategoria{Wikieł/Z1.58r}
\zadStart{Zadanie z Wikieł Z 1.58 r) moja wersja nr [nrWersji]}
%[a]:[2,3,4,5,6,7,8,9,10,11,12,13,14,15,16,17,18,19,20,21,22,23,24,25,26,27,28,29,30]
%[b]:[2,3,4,5,6,7,8,9,10,11,12,13,14,15,16,17,18,19,20,21,22,23,24,25,26,27,28,29,30]
%[delta]=[a]**2-4*[b]
%[pierw2]=pow([delta],1/2)
%[pierw1]=[pierw2].real
%[pierw]=int([pierw1])
%[f]=[a]-[pierw]
%[e]=[a]+[pierw]
%[u1]=[f]/2
%[u2]=[e]/2
%[uu1]=int([u1])
%[uu2]=int([u2])
%[z]=pow([uu1],1/3)
%[v]=pow([uu2],1/3)
%[a]**2>4*[b] and [u1].is_integer()==True and [u2].is_integer()==True and [z].is_integer()==False and [v].is_integer()==False and [pierw2].is_integer()==True
Rozwiązać nierówność $x^6-[a]x^3+[b]=0$.
\zadStop
\rozwStart{Mirella Narewska}{}
$$x^6-[a]x^3+[b]=0$$
$$t=x^3 \Rightarrow t^2-[a]t+[b]=0$$ 
$$\triangle=[a]^2-4\cdot [b]=[delta] \Rightarrow \sqrt{\triangle}=[pierw]$$
$$t=\frac{[f]}{2} \vee t=\frac{[e]}{2}$$
$$x^3=[uu1] \vee x^3=[uu2]$$
$$x=\sqrt[3]{[uu1]} \vee x=\sqrt[3]{[uu2]}$$
\odpStart
$x=\sqrt[3]{[uu1]} \vee x=\sqrt[3]{[uu2]}$
\odpStop
\testStart
A. $x=\sqrt[3]{[uu1]} \vee x=\sqrt[3]{[uu2]}$ \\
B. $x=[uu1] \vee x=[uu2]$ \\
C. $x\in\emptyset$ \\
D. $x=\sqrt[3]{[uu1]}$ 
\testStop
\kluczStart
A
\kluczStop



\end{document}