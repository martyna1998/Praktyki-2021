\documentclass[12pt, a4paper]{article}
\usepackage[utf8]{inputenc}
\usepackage{polski}

\usepackage{amsthm}  %pakiet do tworzenia twierdzeń itp.
\usepackage{amsmath} %pakiet do niektórych symboli matematycznych
\usepackage{amssymb} %pakiet do symboli mat., np. \nsubseteq
\usepackage{amsfonts}
\usepackage{graphicx} %obsługa plików graficznych z rozszerzeniem png, jpg
\theoremstyle{definition} %styl dla definicji
\newtheorem{zad}{} 
\title{Multizestaw zadań}
\author{Robert Fidytek}
%\date{\today}
\date{}
\newcounter{liczniksekcji}
\newcommand{\kategoria}[1]{\section{#1}} %olreślamy nazwę kateforii zadań
\newcommand{\zadStart}[1]{\begin{zad}#1\newline} %oznaczenie początku zadania
\newcommand{\zadStop}{\end{zad}}   %oznaczenie końca zadania
%Makra opcjonarne (nie muszą występować):
\newcommand{\rozwStart}[2]{\noindent \textbf{Rozwiązanie (autor #1 , recenzent #2): }\newline} %oznaczenie początku rozwiązania, opcjonarnie można wprowadzić informację o autorze rozwiązania zadania i recenzencie poprawności wykonania rozwiązania zadania
\newcommand{\rozwStop}{\newline}                                            %oznaczenie końca rozwiązania
\newcommand{\odpStart}{\noindent \textbf{Odpowiedź:}\newline}    %oznaczenie początku odpowiedzi końcowej (wypisanie wyniku)
\newcommand{\odpStop}{\newline}                                             %oznaczenie końca odpowiedzi końcowej (wypisanie wyniku)
\newcommand{\testStart}{\noindent \textbf{Test:}\newline} %ewentualne możliwe opcje odpowiedzi testowej: A. ? B. ? C. ? D. ? itd.
\newcommand{\testStop}{\newline} %koniec wprowadzania odpowiedzi testowych
\newcommand{\kluczStart}{\noindent \textbf{Test poprawna odpowiedź:}\newline} %klucz, poprawna odpowiedź pytania testowego (jedna literka): A lub B lub C lub D itd.
\newcommand{\kluczStop}{\newline} %koniec poprawnej odpowiedzi pytania testowego 
\newcommand{\wstawGrafike}[2]{\begin{figure}[h] \includegraphics[scale=#2] {#1} \end{figure}} %gdyby była potrzeba wstawienia obrazka, parametry: nazwa pliku, skala (jak nie wiesz co wpisać, to wpisz 1)

\begin{document}
\maketitle


\kategoria{Wikieł/Z2.32}
\zadStart{Zadanie z Wikieł Z 2.32 moja wersja nr [nrWersji]}
%[p1]:[2,3,4,5,6,7,8,9,10]
%[p2]:[2,3,4,5,6,7,8,9,10]
%[p3]:[2,3,4,5,6,7,8,9,10]
%[p4]=random.randint(2,10)
%[pp]:[1]
%[p31]=[p3]-[pp]
%[u]=[p2]-[p1]
%[pr]=[p31]*[p2]+[u]
%[a]=[p4]*[p1]
%[b]=[p2]*[p3]-[a]
%[pb]=-[b]
%math.gcd([p4],[p2])==1 and math.gcd([p1],[p2])==1 and [p1]<[p2]

Dana jest prosta w postaci parametrycznej $\left\{ \begin{array}{ll}
x=-[p1]+[p2]t & \quad \quad t\in\mathbb{R}\\
y=[p3]-[p4]t
\end{array} \right.$ Znaleźć równanie ogólne i kierunkowe tej prostej.

\zadStop

\rozwStart{Maja Szabłowska}{}
Równanie kierunkowe:
$$x=-[p1]+[p2]t \iff x+[p1]=[p2]t \iff t=\frac{x+[p1]}{[p2]}$$
$$y=[p3]-[p4]\cdot \frac{x+[p1]}{[p2]}=-\frac{[p4]}{[p2]}x+[p3]-\frac{[a]}{[p2]}=-\frac{[p4]}{[p2]}x+\frac{[b]}{[p2]}$$
$$y=-\frac{[p4]}{[p2]}x+\frac{[b]}{[p2]}$$

Równanie ogólne:
$$y=-\frac{[p4]}{[p2]}x+\frac{[b]}{[p2]} \iff [p2]y=-[p4]x+([b])$$
$$[p2]y+[p4]x+[pb]=0$$
\rozwStop


\odpStart
$y=-\frac{[p4]}{[p2]}x+\frac{[b]}{[p2]}, \quad [p2]y+[p4]x+[pb]=0$
\odpStop
\testStart
A.$y=-\frac{[p4]}{[p2]}x+\frac{[b]}{[p2]}, \quad [p2]y+[p4]x+[pb]=0$\\
B.$y=[p4]x+[p31]\frac{[u]}{[p2]}, \quad [p2]y+[p4]x-[pr]=0$\\
C.$y=\frac{[p4]}{[p2]}x+[p31]\frac{[u]}{[p2]}, \quad [p2]y-[p4]x+[pr]=0$\\
D.$y=\frac{[p1]}{[p2]}x-[p3]\frac{[u]}{[p2]}, \quad -[p2]y-[p4]x+[pr]=0$\\
E.$y=-\frac{[p4]}{[p2]}x+[p31], \quad [p2]y+[p4]x-[pr]=0$\\
F.$y=-\frac{[p4]}{[p2]}x, \quad [p2]y+[p4]x-[pr]=0$\\
G.$y=-\frac{[p4]}{[p2]}x-\frac{[u]}{[p2]}, \quad -[p2]y+[p4]x-[pr]=0$\\
\testStop
\kluczStart
A
\kluczStop



\end{document}
