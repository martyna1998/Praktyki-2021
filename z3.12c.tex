\documentclass[12pt, a4paper]{article}
\usepackage[utf8]{inputenc}
\usepackage{polski}
\usepackage{amsthm}  %pakiet do tworzenia twierdzeń itp.
\usepackage{amsmath} %pakiet do niektórych symboli matematycznych
\usepackage{amssymb} %pakiet do symboli mat., np. \nsubseteq
\usepackage{amsfonts}
\usepackage{graphicx} %obsługa plików graficznych z rozszerzeniem png, jpg
\theoremstyle{definition} %styl dla definicji
\newtheorem{zad}{} 
\title{Multizestaw zadań}
\author{Patryk Wirkus}
%\date{\today}
\date{}
\newcommand{\kategoria}[1]{\section{#1}}
\newcommand{\zadStart}[1]{\begin{zad}#1\newline}
\newcommand{\zadStop}{\end{zad}}
\newcommand{\rozwStart}[2]{\noindent \textbf{Rozwiązanie (autor #1 , recenzent #2): }\newline}
\newcommand{\rozwStop}{\newline}                                           
\newcommand{\odpStart}{\noindent \textbf{Odpowiedź:}\newline}
\newcommand{\odpStop}{\newline}
\newcommand{\testStart}{\noindent \textbf{Test:}\newline}
\newcommand{\testStop}{\newline}
\newcommand{\kluczStart}{\noindent \textbf{Test poprawna odpowiedź:}\newline}
\newcommand{\kluczStop}{\newline}
\newcommand{\wstawGrafike}[2]{\begin{figure}[h] \includegraphics[scale=#2] {#1} \end{figure}}

\begin{document}
\maketitle

\kategoria{Wikieł/Z3.12c}


\zadStart{Zadanie z Wikieł Z 3.12 c) moja wersja nr 1}
Obliczyć granicę ciągu $a_{n}=\frac{(n+1+1)!+(n+1)!}{(n+1+2)!}$.
\zadStop
\rozwStart{Patryk Wirkus}{}
$$\lim\limits_{n\to\infty}\frac{(n+1+1)!+(n+1)!}{(n+1+2)!}=$$
$$\lim\limits_{n\to\infty}\frac{(n+1)!(n+1+1+1)}{(n+1)!(n+1+2)(n+1+1)}=$$
$$\lim\limits_{n\to\infty}\frac{1}{n+1+1}= 0$$
\rozwStop
\odpStart
$0$
\odpStop
\testStart
A.$0$
B.$n+2$
C.$n+3$
D.$2$
E.$3$
F.$-2$
G.$-3$
H.$\infty$
I.$-\infty$
\testStop
\kluczStart
A
\kluczStop



\zadStart{Zadanie z Wikieł Z 3.12 c) moja wersja nr 2}
Obliczyć granicę ciągu $a_{n}=\frac{(n+2+1)!+(n+2)!}{(n+2+2)!}$.
\zadStop
\rozwStart{Patryk Wirkus}{}
$$\lim\limits_{n\to\infty}\frac{(n+2+1)!+(n+2)!}{(n+2+2)!}=$$
$$\lim\limits_{n\to\infty}\frac{(n+2)!(n+2+1+1)}{(n+2)!(n+2+2)(n+2+1)}=$$
$$\lim\limits_{n\to\infty}\frac{1}{n+2+1}= 0$$
\rozwStop
\odpStart
$0$
\odpStop
\testStart
A.$0$
B.$n+3$
C.$n+4$
D.$3$
E.$4$
F.$-3$
G.$-4$
H.$\infty$
I.$-\infty$
\testStop
\kluczStart
A
\kluczStop



\zadStart{Zadanie z Wikieł Z 3.12 c) moja wersja nr 3}
Obliczyć granicę ciągu $a_{n}=\frac{(n+3+1)!+(n+3)!}{(n+3+2)!}$.
\zadStop
\rozwStart{Patryk Wirkus}{}
$$\lim\limits_{n\to\infty}\frac{(n+3+1)!+(n+3)!}{(n+3+2)!}=$$
$$\lim\limits_{n\to\infty}\frac{(n+3)!(n+3+1+1)}{(n+3)!(n+3+2)(n+3+1)}=$$
$$\lim\limits_{n\to\infty}\frac{1}{n+3+1}= 0$$
\rozwStop
\odpStart
$0$
\odpStop
\testStart
A.$0$
B.$n+4$
C.$n+5$
D.$4$
E.$5$
F.$-4$
G.$-5$
H.$\infty$
I.$-\infty$
\testStop
\kluczStart
A
\kluczStop



\zadStart{Zadanie z Wikieł Z 3.12 c) moja wersja nr 4}
Obliczyć granicę ciągu $a_{n}=\frac{(n+4+1)!+(n+4)!}{(n+4+2)!}$.
\zadStop
\rozwStart{Patryk Wirkus}{}
$$\lim\limits_{n\to\infty}\frac{(n+4+1)!+(n+4)!}{(n+4+2)!}=$$
$$\lim\limits_{n\to\infty}\frac{(n+4)!(n+4+1+1)}{(n+4)!(n+4+2)(n+4+1)}=$$
$$\lim\limits_{n\to\infty}\frac{1}{n+4+1}= 0$$
\rozwStop
\odpStart
$0$
\odpStop
\testStart
A.$0$
B.$n+5$
C.$n+6$
D.$5$
E.$6$
F.$-5$
G.$-6$
H.$\infty$
I.$-\infty$
\testStop
\kluczStart
A
\kluczStop



\zadStart{Zadanie z Wikieł Z 3.12 c) moja wersja nr 5}
Obliczyć granicę ciągu $a_{n}=\frac{(n+5+1)!+(n+5)!}{(n+5+2)!}$.
\zadStop
\rozwStart{Patryk Wirkus}{}
$$\lim\limits_{n\to\infty}\frac{(n+5+1)!+(n+5)!}{(n+5+2)!}=$$
$$\lim\limits_{n\to\infty}\frac{(n+5)!(n+5+1+1)}{(n+5)!(n+5+2)(n+5+1)}=$$
$$\lim\limits_{n\to\infty}\frac{1}{n+5+1}= 0$$
\rozwStop
\odpStart
$0$
\odpStop
\testStart
A.$0$
B.$n+6$
C.$n+7$
D.$6$
E.$7$
F.$-6$
G.$-7$
H.$\infty$
I.$-\infty$
\testStop
\kluczStart
A
\kluczStop



\zadStart{Zadanie z Wikieł Z 3.12 c) moja wersja nr 6}
Obliczyć granicę ciągu $a_{n}=\frac{(n+6+1)!+(n+6)!}{(n+6+2)!}$.
\zadStop
\rozwStart{Patryk Wirkus}{}
$$\lim\limits_{n\to\infty}\frac{(n+6+1)!+(n+6)!}{(n+6+2)!}=$$
$$\lim\limits_{n\to\infty}\frac{(n+6)!(n+6+1+1)}{(n+6)!(n+6+2)(n+6+1)}=$$
$$\lim\limits_{n\to\infty}\frac{1}{n+6+1}= 0$$
\rozwStop
\odpStart
$0$
\odpStop
\testStart
A.$0$
B.$n+7$
C.$n+8$
D.$7$
E.$8$
F.$-7$
G.$-8$
H.$\infty$
I.$-\infty$
\testStop
\kluczStart
A
\kluczStop



\zadStart{Zadanie z Wikieł Z 3.12 c) moja wersja nr 7}
Obliczyć granicę ciągu $a_{n}=\frac{(n+7+1)!+(n+7)!}{(n+7+2)!}$.
\zadStop
\rozwStart{Patryk Wirkus}{}
$$\lim\limits_{n\to\infty}\frac{(n+7+1)!+(n+7)!}{(n+7+2)!}=$$
$$\lim\limits_{n\to\infty}\frac{(n+7)!(n+7+1+1)}{(n+7)!(n+7+2)(n+7+1)}=$$
$$\lim\limits_{n\to\infty}\frac{1}{n+7+1}= 0$$
\rozwStop
\odpStart
$0$
\odpStop
\testStart
A.$0$
B.$n+8$
C.$n+9$
D.$8$
E.$9$
F.$-8$
G.$-9$
H.$\infty$
I.$-\infty$
\testStop
\kluczStart
A
\kluczStop



\zadStart{Zadanie z Wikieł Z 3.12 c) moja wersja nr 8}
Obliczyć granicę ciągu $a_{n}=\frac{(n+8+1)!+(n+8)!}{(n+8+2)!}$.
\zadStop
\rozwStart{Patryk Wirkus}{}
$$\lim\limits_{n\to\infty}\frac{(n+8+1)!+(n+8)!}{(n+8+2)!}=$$
$$\lim\limits_{n\to\infty}\frac{(n+8)!(n+8+1+1)}{(n+8)!(n+8+2)(n+8+1)}=$$
$$\lim\limits_{n\to\infty}\frac{1}{n+8+1}= 0$$
\rozwStop
\odpStart
$0$
\odpStop
\testStart
A.$0$
B.$n+9$
C.$n+10$
D.$9$
E.$10$
F.$-9$
G.$-10$
H.$\infty$
I.$-\infty$
\testStop
\kluczStart
A
\kluczStop



\zadStart{Zadanie z Wikieł Z 3.12 c) moja wersja nr 9}
Obliczyć granicę ciągu $a_{n}=\frac{(n+9+1)!+(n+9)!}{(n+9+2)!}$.
\zadStop
\rozwStart{Patryk Wirkus}{}
$$\lim\limits_{n\to\infty}\frac{(n+9+1)!+(n+9)!}{(n+9+2)!}=$$
$$\lim\limits_{n\to\infty}\frac{(n+9)!(n+9+1+1)}{(n+9)!(n+9+2)(n+9+1)}=$$
$$\lim\limits_{n\to\infty}\frac{1}{n+9+1}= 0$$
\rozwStop
\odpStart
$0$
\odpStop
\testStart
A.$0$
B.$n+10$
C.$n+11$
D.$10$
E.$11$
F.$-10$
G.$-11$
H.$\infty$
I.$-\infty$
\testStop
\kluczStart
A
\kluczStop



\zadStart{Zadanie z Wikieł Z 3.12 c) moja wersja nr 10}
Obliczyć granicę ciągu $a_{n}=\frac{(n+10+1)!+(n+10)!}{(n+10+2)!}$.
\zadStop
\rozwStart{Patryk Wirkus}{}
$$\lim\limits_{n\to\infty}\frac{(n+10+1)!+(n+10)!}{(n+10+2)!}=$$
$$\lim\limits_{n\to\infty}\frac{(n+10)!(n+10+1+1)}{(n+10)!(n+10+2)(n+10+1)}=$$
$$\lim\limits_{n\to\infty}\frac{1}{n+10+1}= 0$$
\rozwStop
\odpStart
$0$
\odpStop
\testStart
A.$0$
B.$n+11$
C.$n+12$
D.$11$
E.$12$
F.$-11$
G.$-12$
H.$\infty$
I.$-\infty$
\testStop
\kluczStart
A
\kluczStop



\zadStart{Zadanie z Wikieł Z 3.12 c) moja wersja nr 11}
Obliczyć granicę ciągu $a_{n}=\frac{(n+11+1)!+(n+11)!}{(n+11+2)!}$.
\zadStop
\rozwStart{Patryk Wirkus}{}
$$\lim\limits_{n\to\infty}\frac{(n+11+1)!+(n+11)!}{(n+11+2)!}=$$
$$\lim\limits_{n\to\infty}\frac{(n+11)!(n+11+1+1)}{(n+11)!(n+11+2)(n+11+1)}=$$
$$\lim\limits_{n\to\infty}\frac{1}{n+11+1}= 0$$
\rozwStop
\odpStart
$0$
\odpStop
\testStart
A.$0$
B.$n+12$
C.$n+13$
D.$12$
E.$13$
F.$-12$
G.$-13$
H.$\infty$
I.$-\infty$
\testStop
\kluczStart
A
\kluczStop



\zadStart{Zadanie z Wikieł Z 3.12 c) moja wersja nr 12}
Obliczyć granicę ciągu $a_{n}=\frac{(n+12+1)!+(n+12)!}{(n+12+2)!}$.
\zadStop
\rozwStart{Patryk Wirkus}{}
$$\lim\limits_{n\to\infty}\frac{(n+12+1)!+(n+12)!}{(n+12+2)!}=$$
$$\lim\limits_{n\to\infty}\frac{(n+12)!(n+12+1+1)}{(n+12)!(n+12+2)(n+12+1)}=$$
$$\lim\limits_{n\to\infty}\frac{1}{n+12+1}= 0$$
\rozwStop
\odpStart
$0$
\odpStop
\testStart
A.$0$
B.$n+13$
C.$n+14$
D.$13$
E.$14$
F.$-13$
G.$-14$
H.$\infty$
I.$-\infty$
\testStop
\kluczStart
A
\kluczStop



\zadStart{Zadanie z Wikieł Z 3.12 c) moja wersja nr 13}
Obliczyć granicę ciągu $a_{n}=\frac{(n+13+1)!+(n+13)!}{(n+13+2)!}$.
\zadStop
\rozwStart{Patryk Wirkus}{}
$$\lim\limits_{n\to\infty}\frac{(n+13+1)!+(n+13)!}{(n+13+2)!}=$$
$$\lim\limits_{n\to\infty}\frac{(n+13)!(n+13+1+1)}{(n+13)!(n+13+2)(n+13+1)}=$$
$$\lim\limits_{n\to\infty}\frac{1}{n+13+1}= 0$$
\rozwStop
\odpStart
$0$
\odpStop
\testStart
A.$0$
B.$n+14$
C.$n+15$
D.$14$
E.$15$
F.$-14$
G.$-15$
H.$\infty$
I.$-\infty$
\testStop
\kluczStart
A
\kluczStop



\zadStart{Zadanie z Wikieł Z 3.12 c) moja wersja nr 14}
Obliczyć granicę ciągu $a_{n}=\frac{(n+14+1)!+(n+14)!}{(n+14+2)!}$.
\zadStop
\rozwStart{Patryk Wirkus}{}
$$\lim\limits_{n\to\infty}\frac{(n+14+1)!+(n+14)!}{(n+14+2)!}=$$
$$\lim\limits_{n\to\infty}\frac{(n+14)!(n+14+1+1)}{(n+14)!(n+14+2)(n+14+1)}=$$
$$\lim\limits_{n\to\infty}\frac{1}{n+14+1}= 0$$
\rozwStop
\odpStart
$0$
\odpStop
\testStart
A.$0$
B.$n+15$
C.$n+16$
D.$15$
E.$16$
F.$-15$
G.$-16$
H.$\infty$
I.$-\infty$
\testStop
\kluczStart
A
\kluczStop



\zadStart{Zadanie z Wikieł Z 3.12 c) moja wersja nr 15}
Obliczyć granicę ciągu $a_{n}=\frac{(n+15+1)!+(n+15)!}{(n+15+2)!}$.
\zadStop
\rozwStart{Patryk Wirkus}{}
$$\lim\limits_{n\to\infty}\frac{(n+15+1)!+(n+15)!}{(n+15+2)!}=$$
$$\lim\limits_{n\to\infty}\frac{(n+15)!(n+15+1+1)}{(n+15)!(n+15+2)(n+15+1)}=$$
$$\lim\limits_{n\to\infty}\frac{1}{n+15+1}= 0$$
\rozwStop
\odpStart
$0$
\odpStop
\testStart
A.$0$
B.$n+16$
C.$n+17$
D.$16$
E.$17$
F.$-16$
G.$-17$
H.$\infty$
I.$-\infty$
\testStop
\kluczStart
A
\kluczStop



\zadStart{Zadanie z Wikieł Z 3.12 c) moja wersja nr 16}
Obliczyć granicę ciągu $a_{n}=\frac{(n+16+1)!+(n+16)!}{(n+16+2)!}$.
\zadStop
\rozwStart{Patryk Wirkus}{}
$$\lim\limits_{n\to\infty}\frac{(n+16+1)!+(n+16)!}{(n+16+2)!}=$$
$$\lim\limits_{n\to\infty}\frac{(n+16)!(n+16+1+1)}{(n+16)!(n+16+2)(n+16+1)}=$$
$$\lim\limits_{n\to\infty}\frac{1}{n+16+1}= 0$$
\rozwStop
\odpStart
$0$
\odpStop
\testStart
A.$0$
B.$n+17$
C.$n+18$
D.$17$
E.$18$
F.$-17$
G.$-18$
H.$\infty$
I.$-\infty$
\testStop
\kluczStart
A
\kluczStop



\zadStart{Zadanie z Wikieł Z 3.12 c) moja wersja nr 17}
Obliczyć granicę ciągu $a_{n}=\frac{(n+17+1)!+(n+17)!}{(n+17+2)!}$.
\zadStop
\rozwStart{Patryk Wirkus}{}
$$\lim\limits_{n\to\infty}\frac{(n+17+1)!+(n+17)!}{(n+17+2)!}=$$
$$\lim\limits_{n\to\infty}\frac{(n+17)!(n+17+1+1)}{(n+17)!(n+17+2)(n+17+1)}=$$
$$\lim\limits_{n\to\infty}\frac{1}{n+17+1}= 0$$
\rozwStop
\odpStart
$0$
\odpStop
\testStart
A.$0$
B.$n+18$
C.$n+19$
D.$18$
E.$19$
F.$-18$
G.$-19$
H.$\infty$
I.$-\infty$
\testStop
\kluczStart
A
\kluczStop



\zadStart{Zadanie z Wikieł Z 3.12 c) moja wersja nr 18}
Obliczyć granicę ciągu $a_{n}=\frac{(n+18+1)!+(n+18)!}{(n+18+2)!}$.
\zadStop
\rozwStart{Patryk Wirkus}{}
$$\lim\limits_{n\to\infty}\frac{(n+18+1)!+(n+18)!}{(n+18+2)!}=$$
$$\lim\limits_{n\to\infty}\frac{(n+18)!(n+18+1+1)}{(n+18)!(n+18+2)(n+18+1)}=$$
$$\lim\limits_{n\to\infty}\frac{1}{n+18+1}= 0$$
\rozwStop
\odpStart
$0$
\odpStop
\testStart
A.$0$
B.$n+19$
C.$n+20$
D.$19$
E.$20$
F.$-19$
G.$-20$
H.$\infty$
I.$-\infty$
\testStop
\kluczStart
A
\kluczStop



\zadStart{Zadanie z Wikieł Z 3.12 c) moja wersja nr 19}
Obliczyć granicę ciągu $a_{n}=\frac{(n+19+1)!+(n+19)!}{(n+19+2)!}$.
\zadStop
\rozwStart{Patryk Wirkus}{}
$$\lim\limits_{n\to\infty}\frac{(n+19+1)!+(n+19)!}{(n+19+2)!}=$$
$$\lim\limits_{n\to\infty}\frac{(n+19)!(n+19+1+1)}{(n+19)!(n+19+2)(n+19+1)}=$$
$$\lim\limits_{n\to\infty}\frac{1}{n+19+1}= 0$$
\rozwStop
\odpStart
$0$
\odpStop
\testStart
A.$0$
B.$n+20$
C.$n+21$
D.$20$
E.$21$
F.$-20$
G.$-21$
H.$\infty$
I.$-\infty$
\testStop
\kluczStart
A
\kluczStop



\zadStart{Zadanie z Wikieł Z 3.12 c) moja wersja nr 20}
Obliczyć granicę ciągu $a_{n}=\frac{(n+20+1)!+(n+20)!}{(n+20+2)!}$.
\zadStop
\rozwStart{Patryk Wirkus}{}
$$\lim\limits_{n\to\infty}\frac{(n+20+1)!+(n+20)!}{(n+20+2)!}=$$
$$\lim\limits_{n\to\infty}\frac{(n+20)!(n+20+1+1)}{(n+20)!(n+20+2)(n+20+1)}=$$
$$\lim\limits_{n\to\infty}\frac{1}{n+20+1}= 0$$
\rozwStop
\odpStart
$0$
\odpStop
\testStart
A.$0$
B.$n+21$
C.$n+22$
D.$21$
E.$22$
F.$-21$
G.$-22$
H.$\infty$
I.$-\infty$
\testStop
\kluczStart
A
\kluczStop



\zadStart{Zadanie z Wikieł Z 3.12 c) moja wersja nr 21}
Obliczyć granicę ciągu $a_{n}=\frac{(n+21+1)!+(n+21)!}{(n+21+2)!}$.
\zadStop
\rozwStart{Patryk Wirkus}{}
$$\lim\limits_{n\to\infty}\frac{(n+21+1)!+(n+21)!}{(n+21+2)!}=$$
$$\lim\limits_{n\to\infty}\frac{(n+21)!(n+21+1+1)}{(n+21)!(n+21+2)(n+21+1)}=$$
$$\lim\limits_{n\to\infty}\frac{1}{n+21+1}= 0$$
\rozwStop
\odpStart
$0$
\odpStop
\testStart
A.$0$
B.$n+22$
C.$n+23$
D.$22$
E.$23$
F.$-22$
G.$-23$
H.$\infty$
I.$-\infty$
\testStop
\kluczStart
A
\kluczStop



\zadStart{Zadanie z Wikieł Z 3.12 c) moja wersja nr 22}
Obliczyć granicę ciągu $a_{n}=\frac{(n+22+1)!+(n+22)!}{(n+22+2)!}$.
\zadStop
\rozwStart{Patryk Wirkus}{}
$$\lim\limits_{n\to\infty}\frac{(n+22+1)!+(n+22)!}{(n+22+2)!}=$$
$$\lim\limits_{n\to\infty}\frac{(n+22)!(n+22+1+1)}{(n+22)!(n+22+2)(n+22+1)}=$$
$$\lim\limits_{n\to\infty}\frac{1}{n+22+1}= 0$$
\rozwStop
\odpStart
$0$
\odpStop
\testStart
A.$0$
B.$n+23$
C.$n+24$
D.$23$
E.$24$
F.$-23$
G.$-24$
H.$\infty$
I.$-\infty$
\testStop
\kluczStart
A
\kluczStop



\zadStart{Zadanie z Wikieł Z 3.12 c) moja wersja nr 23}
Obliczyć granicę ciągu $a_{n}=\frac{(n+23+1)!+(n+23)!}{(n+23+2)!}$.
\zadStop
\rozwStart{Patryk Wirkus}{}
$$\lim\limits_{n\to\infty}\frac{(n+23+1)!+(n+23)!}{(n+23+2)!}=$$
$$\lim\limits_{n\to\infty}\frac{(n+23)!(n+23+1+1)}{(n+23)!(n+23+2)(n+23+1)}=$$
$$\lim\limits_{n\to\infty}\frac{1}{n+23+1}= 0$$
\rozwStop
\odpStart
$0$
\odpStop
\testStart
A.$0$
B.$n+24$
C.$n+25$
D.$24$
E.$25$
F.$-24$
G.$-25$
H.$\infty$
I.$-\infty$
\testStop
\kluczStart
A
\kluczStop



\zadStart{Zadanie z Wikieł Z 3.12 c) moja wersja nr 24}
Obliczyć granicę ciągu $a_{n}=\frac{(n+24+1)!+(n+24)!}{(n+24+2)!}$.
\zadStop
\rozwStart{Patryk Wirkus}{}
$$\lim\limits_{n\to\infty}\frac{(n+24+1)!+(n+24)!}{(n+24+2)!}=$$
$$\lim\limits_{n\to\infty}\frac{(n+24)!(n+24+1+1)}{(n+24)!(n+24+2)(n+24+1)}=$$
$$\lim\limits_{n\to\infty}\frac{1}{n+24+1}= 0$$
\rozwStop
\odpStart
$0$
\odpStop
\testStart
A.$0$
B.$n+25$
C.$n+26$
D.$25$
E.$26$
F.$-25$
G.$-26$
H.$\infty$
I.$-\infty$
\testStop
\kluczStart
A
\kluczStop



\zadStart{Zadanie z Wikieł Z 3.12 c) moja wersja nr 25}
Obliczyć granicę ciągu $a_{n}=\frac{(n+25+1)!+(n+25)!}{(n+25+2)!}$.
\zadStop
\rozwStart{Patryk Wirkus}{}
$$\lim\limits_{n\to\infty}\frac{(n+25+1)!+(n+25)!}{(n+25+2)!}=$$
$$\lim\limits_{n\to\infty}\frac{(n+25)!(n+25+1+1)}{(n+25)!(n+25+2)(n+25+1)}=$$
$$\lim\limits_{n\to\infty}\frac{1}{n+25+1}= 0$$
\rozwStop
\odpStart
$0$
\odpStop
\testStart
A.$0$
B.$n+26$
C.$n+27$
D.$26$
E.$27$
F.$-26$
G.$-27$
H.$\infty$
I.$-\infty$
\testStop
\kluczStart
A
\kluczStop



\zadStart{Zadanie z Wikieł Z 3.12 c) moja wersja nr 26}
Obliczyć granicę ciągu $a_{n}=\frac{(n+26+1)!+(n+26)!}{(n+26+2)!}$.
\zadStop
\rozwStart{Patryk Wirkus}{}
$$\lim\limits_{n\to\infty}\frac{(n+26+1)!+(n+26)!}{(n+26+2)!}=$$
$$\lim\limits_{n\to\infty}\frac{(n+26)!(n+26+1+1)}{(n+26)!(n+26+2)(n+26+1)}=$$
$$\lim\limits_{n\to\infty}\frac{1}{n+26+1}= 0$$
\rozwStop
\odpStart
$0$
\odpStop
\testStart
A.$0$
B.$n+27$
C.$n+28$
D.$27$
E.$28$
F.$-27$
G.$-28$
H.$\infty$
I.$-\infty$
\testStop
\kluczStart
A
\kluczStop



\zadStart{Zadanie z Wikieł Z 3.12 c) moja wersja nr 27}
Obliczyć granicę ciągu $a_{n}=\frac{(n+27+1)!+(n+27)!}{(n+27+2)!}$.
\zadStop
\rozwStart{Patryk Wirkus}{}
$$\lim\limits_{n\to\infty}\frac{(n+27+1)!+(n+27)!}{(n+27+2)!}=$$
$$\lim\limits_{n\to\infty}\frac{(n+27)!(n+27+1+1)}{(n+27)!(n+27+2)(n+27+1)}=$$
$$\lim\limits_{n\to\infty}\frac{1}{n+27+1}= 0$$
\rozwStop
\odpStart
$0$
\odpStop
\testStart
A.$0$
B.$n+28$
C.$n+29$
D.$28$
E.$29$
F.$-28$
G.$-29$
H.$\infty$
I.$-\infty$
\testStop
\kluczStart
A
\kluczStop



\zadStart{Zadanie z Wikieł Z 3.12 c) moja wersja nr 28}
Obliczyć granicę ciągu $a_{n}=\frac{(n+28+1)!+(n+28)!}{(n+28+2)!}$.
\zadStop
\rozwStart{Patryk Wirkus}{}
$$\lim\limits_{n\to\infty}\frac{(n+28+1)!+(n+28)!}{(n+28+2)!}=$$
$$\lim\limits_{n\to\infty}\frac{(n+28)!(n+28+1+1)}{(n+28)!(n+28+2)(n+28+1)}=$$
$$\lim\limits_{n\to\infty}\frac{1}{n+28+1}= 0$$
\rozwStop
\odpStart
$0$
\odpStop
\testStart
A.$0$
B.$n+29$
C.$n+30$
D.$29$
E.$30$
F.$-29$
G.$-30$
H.$\infty$
I.$-\infty$
\testStop
\kluczStart
A
\kluczStop



\zadStart{Zadanie z Wikieł Z 3.12 c) moja wersja nr 29}
Obliczyć granicę ciągu $a_{n}=\frac{(n+29+1)!+(n+29)!}{(n+29+2)!}$.
\zadStop
\rozwStart{Patryk Wirkus}{}
$$\lim\limits_{n\to\infty}\frac{(n+29+1)!+(n+29)!}{(n+29+2)!}=$$
$$\lim\limits_{n\to\infty}\frac{(n+29)!(n+29+1+1)}{(n+29)!(n+29+2)(n+29+1)}=$$
$$\lim\limits_{n\to\infty}\frac{1}{n+29+1}= 0$$
\rozwStop
\odpStart
$0$
\odpStop
\testStart
A.$0$
B.$n+30$
C.$n+31$
D.$30$
E.$31$
F.$-30$
G.$-31$
H.$\infty$
I.$-\infty$
\testStop
\kluczStart
A
\kluczStop



\zadStart{Zadanie z Wikieł Z 3.12 c) moja wersja nr 30}
Obliczyć granicę ciągu $a_{n}=\frac{(n+30+1)!+(n+30)!}{(n+30+2)!}$.
\zadStop
\rozwStart{Patryk Wirkus}{}
$$\lim\limits_{n\to\infty}\frac{(n+30+1)!+(n+30)!}{(n+30+2)!}=$$
$$\lim\limits_{n\to\infty}\frac{(n+30)!(n+30+1+1)}{(n+30)!(n+30+2)(n+30+1)}=$$
$$\lim\limits_{n\to\infty}\frac{1}{n+30+1}= 0$$
\rozwStop
\odpStart
$0$
\odpStop
\testStart
A.$0$
B.$n+31$
C.$n+32$
D.$31$
E.$32$
F.$-31$
G.$-32$
H.$\infty$
I.$-\infty$
\testStop
\kluczStart
A
\kluczStop



\zadStart{Zadanie z Wikieł Z 3.12 c) moja wersja nr 31}
Obliczyć granicę ciągu $a_{n}=\frac{(n+21+1)!+(n+21)!}{(n+21+2)!}$.
\zadStop
\rozwStart{Patryk Wirkus}{}
$$\lim\limits_{n\to\infty}\frac{(n+21+1)!+(n+21)!}{(n+21+2)!}=$$
$$\lim\limits_{n\to\infty}\frac{(n+21)!(n+21+1+1)}{(n+21)!(n+21+2)(n+21+1)}=$$
$$\lim\limits_{n\to\infty}\frac{1}{n+21+1}= 0$$
\rozwStop
\odpStart
$0$
\odpStop
\testStart
A.$0$
B.$n+22$
C.$n+23$
D.$22$
E.$23$
F.$-22$
G.$-23$
H.$\infty$
I.$-\infty$
\testStop
\kluczStart
A
\kluczStop



\zadStart{Zadanie z Wikieł Z 3.12 c) moja wersja nr 32}
Obliczyć granicę ciągu $a_{n}=\frac{(n+32+1)!+(n+32)!}{(n+32+2)!}$.
\zadStop
\rozwStart{Patryk Wirkus}{}
$$\lim\limits_{n\to\infty}\frac{(n+32+1)!+(n+32)!}{(n+32+2)!}=$$
$$\lim\limits_{n\to\infty}\frac{(n+32)!(n+32+1+1)}{(n+32)!(n+32+2)(n+32+1)}=$$
$$\lim\limits_{n\to\infty}\frac{1}{n+32+1}= 0$$
\rozwStop
\odpStart
$0$
\odpStop
\testStart
A.$0$
B.$n+33$
C.$n+34$
D.$33$
E.$34$
F.$-33$
G.$-34$
H.$\infty$
I.$-\infty$
\testStop
\kluczStart
A
\kluczStop



\zadStart{Zadanie z Wikieł Z 3.12 c) moja wersja nr 33}
Obliczyć granicę ciągu $a_{n}=\frac{(n+33+1)!+(n+33)!}{(n+33+2)!}$.
\zadStop
\rozwStart{Patryk Wirkus}{}
$$\lim\limits_{n\to\infty}\frac{(n+33+1)!+(n+33)!}{(n+33+2)!}=$$
$$\lim\limits_{n\to\infty}\frac{(n+33)!(n+33+1+1)}{(n+33)!(n+33+2)(n+33+1)}=$$
$$\lim\limits_{n\to\infty}\frac{1}{n+33+1}= 0$$
\rozwStop
\odpStart
$0$
\odpStop
\testStart
A.$0$
B.$n+34$
C.$n+35$
D.$34$
E.$35$
F.$-34$
G.$-35$
H.$\infty$
I.$-\infty$
\testStop
\kluczStart
A
\kluczStop



\zadStart{Zadanie z Wikieł Z 3.12 c) moja wersja nr 34}
Obliczyć granicę ciągu $a_{n}=\frac{(n+34+1)!+(n+34)!}{(n+34+2)!}$.
\zadStop
\rozwStart{Patryk Wirkus}{}
$$\lim\limits_{n\to\infty}\frac{(n+34+1)!+(n+34)!}{(n+34+2)!}=$$
$$\lim\limits_{n\to\infty}\frac{(n+34)!(n+34+1+1)}{(n+34)!(n+34+2)(n+34+1)}=$$
$$\lim\limits_{n\to\infty}\frac{1}{n+34+1}= 0$$
\rozwStop
\odpStart
$0$
\odpStop
\testStart
A.$0$
B.$n+35$
C.$n+36$
D.$35$
E.$36$
F.$-35$
G.$-36$
H.$\infty$
I.$-\infty$
\testStop
\kluczStart
A
\kluczStop



\zadStart{Zadanie z Wikieł Z 3.12 c) moja wersja nr 35}
Obliczyć granicę ciągu $a_{n}=\frac{(n+35+1)!+(n+35)!}{(n+35+2)!}$.
\zadStop
\rozwStart{Patryk Wirkus}{}
$$\lim\limits_{n\to\infty}\frac{(n+35+1)!+(n+35)!}{(n+35+2)!}=$$
$$\lim\limits_{n\to\infty}\frac{(n+35)!(n+35+1+1)}{(n+35)!(n+35+2)(n+35+1)}=$$
$$\lim\limits_{n\to\infty}\frac{1}{n+35+1}= 0$$
\rozwStop
\odpStart
$0$
\odpStop
\testStart
A.$0$
B.$n+36$
C.$n+37$
D.$36$
E.$37$
F.$-36$
G.$-37$
H.$\infty$
I.$-\infty$
\testStop
\kluczStart
A
\kluczStop



\zadStart{Zadanie z Wikieł Z 3.12 c) moja wersja nr 36}
Obliczyć granicę ciągu $a_{n}=\frac{(n+36+1)!+(n+36)!}{(n+36+2)!}$.
\zadStop
\rozwStart{Patryk Wirkus}{}
$$\lim\limits_{n\to\infty}\frac{(n+36+1)!+(n+36)!}{(n+36+2)!}=$$
$$\lim\limits_{n\to\infty}\frac{(n+36)!(n+36+1+1)}{(n+36)!(n+36+2)(n+36+1)}=$$
$$\lim\limits_{n\to\infty}\frac{1}{n+36+1}= 0$$
\rozwStop
\odpStart
$0$
\odpStop
\testStart
A.$0$
B.$n+37$
C.$n+38$
D.$37$
E.$38$
F.$-37$
G.$-38$
H.$\infty$
I.$-\infty$
\testStop
\kluczStart
A
\kluczStop



\zadStart{Zadanie z Wikieł Z 3.12 c) moja wersja nr 37}
Obliczyć granicę ciągu $a_{n}=\frac{(n+37+1)!+(n+37)!}{(n+37+2)!}$.
\zadStop
\rozwStart{Patryk Wirkus}{}
$$\lim\limits_{n\to\infty}\frac{(n+37+1)!+(n+37)!}{(n+37+2)!}=$$
$$\lim\limits_{n\to\infty}\frac{(n+37)!(n+37+1+1)}{(n+37)!(n+37+2)(n+37+1)}=$$
$$\lim\limits_{n\to\infty}\frac{1}{n+37+1}= 0$$
\rozwStop
\odpStart
$0$
\odpStop
\testStart
A.$0$
B.$n+38$
C.$n+39$
D.$38$
E.$39$
F.$-38$
G.$-39$
H.$\infty$
I.$-\infty$
\testStop
\kluczStart
A
\kluczStop



\zadStart{Zadanie z Wikieł Z 3.12 c) moja wersja nr 38}
Obliczyć granicę ciągu $a_{n}=\frac{(n+38+1)!+(n+38)!}{(n+38+2)!}$.
\zadStop
\rozwStart{Patryk Wirkus}{}
$$\lim\limits_{n\to\infty}\frac{(n+38+1)!+(n+38)!}{(n+38+2)!}=$$
$$\lim\limits_{n\to\infty}\frac{(n+38)!(n+38+1+1)}{(n+38)!(n+38+2)(n+38+1)}=$$
$$\lim\limits_{n\to\infty}\frac{1}{n+38+1}= 0$$
\rozwStop
\odpStart
$0$
\odpStop
\testStart
A.$0$
B.$n+39$
C.$n+40$
D.$39$
E.$40$
F.$-39$
G.$-40$
H.$\infty$
I.$-\infty$
\testStop
\kluczStart
A
\kluczStop



\zadStart{Zadanie z Wikieł Z 3.12 c) moja wersja nr 39}
Obliczyć granicę ciągu $a_{n}=\frac{(n+39+1)!+(n+39)!}{(n+39+2)!}$.
\zadStop
\rozwStart{Patryk Wirkus}{}
$$\lim\limits_{n\to\infty}\frac{(n+39+1)!+(n+39)!}{(n+39+2)!}=$$
$$\lim\limits_{n\to\infty}\frac{(n+39)!(n+39+1+1)}{(n+39)!(n+39+2)(n+39+1)}=$$
$$\lim\limits_{n\to\infty}\frac{1}{n+39+1}= 0$$
\rozwStop
\odpStart
$0$
\odpStop
\testStart
A.$0$
B.$n+40$
C.$n+41$
D.$40$
E.$41$
F.$-40$
G.$-41$
H.$\infty$
I.$-\infty$
\testStop
\kluczStart
A
\kluczStop



\zadStart{Zadanie z Wikieł Z 3.12 c) moja wersja nr 40}
Obliczyć granicę ciągu $a_{n}=\frac{(n+40+1)!+(n+40)!}{(n+40+2)!}$.
\zadStop
\rozwStart{Patryk Wirkus}{}
$$\lim\limits_{n\to\infty}\frac{(n+40+1)!+(n+40)!}{(n+40+2)!}=$$
$$\lim\limits_{n\to\infty}\frac{(n+40)!(n+40+1+1)}{(n+40)!(n+40+2)(n+40+1)}=$$
$$\lim\limits_{n\to\infty}\frac{1}{n+40+1}= 0$$
\rozwStop
\odpStart
$0$
\odpStop
\testStart
A.$0$
B.$n+41$
C.$n+42$
D.$41$
E.$42$
F.$-41$
G.$-42$
H.$\infty$
I.$-\infty$
\testStop
\kluczStart
A
\kluczStop



\zadStart{Zadanie z Wikieł Z 3.12 c) moja wersja nr 41}
Obliczyć granicę ciągu $a_{n}=\frac{(n+41+1)!+(n+41)!}{(n+41+2)!}$.
\zadStop
\rozwStart{Patryk Wirkus}{}
$$\lim\limits_{n\to\infty}\frac{(n+41+1)!+(n+41)!}{(n+41+2)!}=$$
$$\lim\limits_{n\to\infty}\frac{(n+41)!(n+41+1+1)}{(n+41)!(n+41+2)(n+41+1)}=$$
$$\lim\limits_{n\to\infty}\frac{1}{n+41+1}= 0$$
\rozwStop
\odpStart
$0$
\odpStop
\testStart
A.$0$
B.$n+42$
C.$n+43$
D.$42$
E.$43$
F.$-42$
G.$-43$
H.$\infty$
I.$-\infty$
\testStop
\kluczStart
A
\kluczStop



\zadStart{Zadanie z Wikieł Z 3.12 c) moja wersja nr 42}
Obliczyć granicę ciągu $a_{n}=\frac{(n+42+1)!+(n+42)!}{(n+42+2)!}$.
\zadStop
\rozwStart{Patryk Wirkus}{}
$$\lim\limits_{n\to\infty}\frac{(n+42+1)!+(n+42)!}{(n+42+2)!}=$$
$$\lim\limits_{n\to\infty}\frac{(n+42)!(n+42+1+1)}{(n+42)!(n+42+2)(n+42+1)}=$$
$$\lim\limits_{n\to\infty}\frac{1}{n+42+1}= 0$$
\rozwStop
\odpStart
$0$
\odpStop
\testStart
A.$0$
B.$n+43$
C.$n+44$
D.$43$
E.$44$
F.$-43$
G.$-44$
H.$\infty$
I.$-\infty$
\testStop
\kluczStart
A
\kluczStop



\zadStart{Zadanie z Wikieł Z 3.12 c) moja wersja nr 43}
Obliczyć granicę ciągu $a_{n}=\frac{(n+43+1)!+(n+43)!}{(n+43+2)!}$.
\zadStop
\rozwStart{Patryk Wirkus}{}
$$\lim\limits_{n\to\infty}\frac{(n+43+1)!+(n+43)!}{(n+43+2)!}=$$
$$\lim\limits_{n\to\infty}\frac{(n+43)!(n+43+1+1)}{(n+43)!(n+43+2)(n+43+1)}=$$
$$\lim\limits_{n\to\infty}\frac{1}{n+43+1}= 0$$
\rozwStop
\odpStart
$0$
\odpStop
\testStart
A.$0$
B.$n+44$
C.$n+45$
D.$44$
E.$45$
F.$-44$
G.$-45$
H.$\infty$
I.$-\infty$
\testStop
\kluczStart
A
\kluczStop



\zadStart{Zadanie z Wikieł Z 3.12 c) moja wersja nr 44}
Obliczyć granicę ciągu $a_{n}=\frac{(n+44+1)!+(n+44)!}{(n+44+2)!}$.
\zadStop
\rozwStart{Patryk Wirkus}{}
$$\lim\limits_{n\to\infty}\frac{(n+44+1)!+(n+44)!}{(n+44+2)!}=$$
$$\lim\limits_{n\to\infty}\frac{(n+44)!(n+44+1+1)}{(n+44)!(n+44+2)(n+44+1)}=$$
$$\lim\limits_{n\to\infty}\frac{1}{n+44+1}= 0$$
\rozwStop
\odpStart
$0$
\odpStop
\testStart
A.$0$
B.$n+45$
C.$n+46$
D.$45$
E.$46$
F.$-45$
G.$-46$
H.$\infty$
I.$-\infty$
\testStop
\kluczStart
A
\kluczStop



\zadStart{Zadanie z Wikieł Z 3.12 c) moja wersja nr 45}
Obliczyć granicę ciągu $a_{n}=\frac{(n+45+1)!+(n+45)!}{(n+45+2)!}$.
\zadStop
\rozwStart{Patryk Wirkus}{}
$$\lim\limits_{n\to\infty}\frac{(n+45+1)!+(n+45)!}{(n+45+2)!}=$$
$$\lim\limits_{n\to\infty}\frac{(n+45)!(n+45+1+1)}{(n+45)!(n+45+2)(n+45+1)}=$$
$$\lim\limits_{n\to\infty}\frac{1}{n+45+1}= 0$$
\rozwStop
\odpStart
$0$
\odpStop
\testStart
A.$0$
B.$n+46$
C.$n+47$
D.$46$
E.$47$
F.$-46$
G.$-47$
H.$\infty$
I.$-\infty$
\testStop
\kluczStart
A
\kluczStop



\zadStart{Zadanie z Wikieł Z 3.12 c) moja wersja nr 46}
Obliczyć granicę ciągu $a_{n}=\frac{(n+46+1)!+(n+46)!}{(n+46+2)!}$.
\zadStop
\rozwStart{Patryk Wirkus}{}
$$\lim\limits_{n\to\infty}\frac{(n+46+1)!+(n+46)!}{(n+46+2)!}=$$
$$\lim\limits_{n\to\infty}\frac{(n+46)!(n+46+1+1)}{(n+46)!(n+46+2)(n+46+1)}=$$
$$\lim\limits_{n\to\infty}\frac{1}{n+46+1}= 0$$
\rozwStop
\odpStart
$0$
\odpStop
\testStart
A.$0$
B.$n+47$
C.$n+48$
D.$47$
E.$48$
F.$-47$
G.$-48$
H.$\infty$
I.$-\infty$
\testStop
\kluczStart
A
\kluczStop



\zadStart{Zadanie z Wikieł Z 3.12 c) moja wersja nr 47}
Obliczyć granicę ciągu $a_{n}=\frac{(n+47+1)!+(n+47)!}{(n+47+2)!}$.
\zadStop
\rozwStart{Patryk Wirkus}{}
$$\lim\limits_{n\to\infty}\frac{(n+47+1)!+(n+47)!}{(n+47+2)!}=$$
$$\lim\limits_{n\to\infty}\frac{(n+47)!(n+47+1+1)}{(n+47)!(n+47+2)(n+47+1)}=$$
$$\lim\limits_{n\to\infty}\frac{1}{n+47+1}= 0$$
\rozwStop
\odpStart
$0$
\odpStop
\testStart
A.$0$
B.$n+48$
C.$n+49$
D.$48$
E.$49$
F.$-48$
G.$-49$
H.$\infty$
I.$-\infty$
\testStop
\kluczStart
A
\kluczStop



\zadStart{Zadanie z Wikieł Z 3.12 c) moja wersja nr 48}
Obliczyć granicę ciągu $a_{n}=\frac{(n+48+1)!+(n+48)!}{(n+48+2)!}$.
\zadStop
\rozwStart{Patryk Wirkus}{}
$$\lim\limits_{n\to\infty}\frac{(n+48+1)!+(n+48)!}{(n+48+2)!}=$$
$$\lim\limits_{n\to\infty}\frac{(n+48)!(n+48+1+1)}{(n+48)!(n+48+2)(n+48+1)}=$$
$$\lim\limits_{n\to\infty}\frac{1}{n+48+1}= 0$$
\rozwStop
\odpStart
$0$
\odpStop
\testStart
A.$0$
B.$n+49$
C.$n+50$
D.$49$
E.$50$
F.$-49$
G.$-50$
H.$\infty$
I.$-\infty$
\testStop
\kluczStart
A
\kluczStop



\zadStart{Zadanie z Wikieł Z 3.12 c) moja wersja nr 49}
Obliczyć granicę ciągu $a_{n}=\frac{(n+49+1)!+(n+49)!}{(n+49+2)!}$.
\zadStop
\rozwStart{Patryk Wirkus}{}
$$\lim\limits_{n\to\infty}\frac{(n+49+1)!+(n+49)!}{(n+49+2)!}=$$
$$\lim\limits_{n\to\infty}\frac{(n+49)!(n+49+1+1)}{(n+49)!(n+49+2)(n+49+1)}=$$
$$\lim\limits_{n\to\infty}\frac{1}{n+49+1}= 0$$
\rozwStop
\odpStart
$0$
\odpStop
\testStart
A.$0$
B.$n+50$
C.$n+51$
D.$50$
E.$51$
F.$-50$
G.$-51$
H.$\infty$
I.$-\infty$
\testStop
\kluczStart
A
\kluczStop



\zadStart{Zadanie z Wikieł Z 3.12 c) moja wersja nr 50}
Obliczyć granicę ciągu $a_{n}=\frac{(n+50+1)!+(n+50)!}{(n+50+2)!}$.
\zadStop
\rozwStart{Patryk Wirkus}{}
$$\lim\limits_{n\to\infty}\frac{(n+50+1)!+(n+50)!}{(n+50+2)!}=$$
$$\lim\limits_{n\to\infty}\frac{(n+50)!(n+50+1+1)}{(n+50)!(n+50+2)(n+50+1)}=$$
$$\lim\limits_{n\to\infty}\frac{1}{n+50+1}= 0$$
\rozwStop
\odpStart
$0$
\odpStop
\testStart
A.$0$
B.$n+51$
C.$n+52$
D.$51$
E.$52$
F.$-51$
G.$-52$
H.$\infty$
I.$-\infty$
\testStop
\kluczStart
A
\kluczStop





\end{document}
