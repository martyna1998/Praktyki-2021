\documentclass[12pt, a4paper]{article}
\usepackage[utf8]{inputenc}
\usepackage{polski}

\usepackage{amsthm}  %pakiet do tworzenia twierdzeń itp.
\usepackage{amsmath} %pakiet do niektórych symboli matematycznych
\usepackage{amssymb} %pakiet do symboli mat., np. \nsubseteq
\usepackage{amsfonts}
\usepackage{graphicx} %obsługa plików graficznych z rozszerzeniem png, jpg
\theoremstyle{definition} %styl dla definicji
\newtheorem{zad}{} 
\title{Multizestaw zadań}
\author{Jacek Jabłoński}
%\date{\today}
\date{}
\newcounter{liczniksekcji}
\newcommand{\kategoria}[1]{\section{#1}} %olreślamy nazwę kateforii zadań
\newcommand{\zadStart}[1]{\begin{zad}#1\newline} %oznaczenie początku zadania
\newcommand{\zadStop}{\end{zad}}   %oznaczenie końca zadania
%Makra opcjonarne (nie muszą występować):
\newcommand{\rozwStart}[2]{\noindent \textbf{Rozwiązanie (autor #1 , recenzent #2): }\newline} %oznaczenie początku rozwiązania, opcjonarnie można wprowadzić informację o autorze rozwiązania zadania i recenzencie poprawności wykonania rozwiązania zadania
\newcommand{\rozwStop}{\newline}                                            %oznaczenie końca rozwiązania
\newcommand{\odpStart}{\noindent \textbf{Odpowiedź:}\newline}    %oznaczenie początku odpowiedzi końcowej (wypisanie wyniku)
\newcommand{\odpStop}{\newline}                                             %oznaczenie końca odpowiedzi końcowej (wypisanie wyniku)
\newcommand{\testStart}{\noindent \textbf{Test:}\newline} %ewentualne możliwe opcje odpowiedzi testowej: A. ? B. ? C. ? D. ? itd.
\newcommand{\testStop}{\newline} %koniec wprowadzania odpowiedzi testowych
\newcommand{\kluczStart}{\noindent \textbf{Test poprawna odpowiedź:}\newline} %klucz, poprawna odpowiedź pytania testowego (jedna literka): A lub B lub C lub D itd.
\newcommand{\kluczStop}{\newline} %koniec poprawnej odpowiedzi pytania testowego 
\newcommand{\wstawGrafike}[2]{\begin{figure}[h] \includegraphics[scale=#2] {#1} \end{figure}} %gdyby była potrzeba wstawienia obrazka, parametry: nazwa pliku, skala (jak nie wiesz co wpisać, to wpisz 1)

\begin{document}
\maketitle


\kategoria{Wikieł/z1.83a}
\zadStart{Zadanie z Wikieł z1.83a) moja wersja nr [nrWersji]}
%[p1]:[0,1,2,3,4,5,6,7]
%[p2]:[2,3,4,5,6,7,8,9,10,11,12]
%[r1]=int(math.pow(2,[p1]))
%[r2]=[p2]+[p1]
%[delta]=abs(1-4*(-[r2]))
%[pdelta]=int(math.pow([delta],(1/2)))
%[c1]=math.sqrt([delta])
%[c2]=math.isqrt([delta])
%[x1]=int((1-[pdelta])/2)
%[x2]=int((1+[pdelta])/2)
%[f1a]=[x1]-1
%[f2a]=[x1]-2
%[f3a]=[x1]-3
%[f1b]=[x2]+1
%[f2b]=[x2]+2
%[f3b]=[x2]+3
%not([c1]!=[c2]) and [delta]>0
Wyznaczyć wartośći x, dla których funkcja $y=2^{x^2-x-[p2]}$ przyjmuje wartość [r1]
\zadStop
\rozwStart{Jacek Jabłoński}{}
$$2^{x^2-x-[p2]} = [r1] $$
$$2^{x^2-x-[p2]} = 2^{[p1]}$$
$$x^2-x-[p2] = [p1]$$
$$x^2-x-[r2] = 0$$
$$\Delta = [delta]$$
$$\sqrt{\Delta} = [pdelta]$$
$$x_1 = \frac{1 - [pdelta]}{2} = [x1]$$
$$x_2 = \frac{1 + [pdelta]}{2} = [x2]$$
\rozwStop
\odpStart
$$x=\{[x1],[x2]\}$$
\odpStop
\testStart
A. $$x=\{[x1],[x2]\}$$
B. $$x=\{[x1],[f1b]\}$$
C. $$x=\{[x1],[f2b]\}$$
D. $$x=\{[x1],[f3b]\}$$
E. $$x=\{[f1a],[x2]\}$$
F. $$x=\{[f2a],[x2]\}$$
G. $$x=\{[f3a],[x2]\}$$
H. $$x=\{[f3a],[f3b]\}$$
I. $$x=\{[f2a],[f2b]\}$$
\testStop
\kluczStart
A
\kluczStop



\end{document}