\documentclass[12pt, a4paper]{article}
\usepackage[utf8]{inputenc}
\usepackage{polski}

\usepackage{amsthm}  %pakiet do tworzenia twierdzeń itp.
\usepackage{amsmath} %pakiet do niektórych symboli matematycznych
\usepackage{amssymb} %pakiet do symboli mat., np. \nsubseteq
\usepackage{amsfonts}
\usepackage{graphicx} %obsługa plików graficznych z rozszerzeniem png, jpg
\theoremstyle{definition} %styl dla definicji
\newtheorem{zad}{} 
\title{Multizestaw zadań}
\author{Robert Fidytek}
%\date{\today}
\date{}
\newcounter{liczniksekcji}
\newcommand{\kategoria}[1]{\section{#1}} %olreślamy nazwę kateforii zadań
\newcommand{\zadStart}[1]{\begin{zad}#1\newline} %oznaczenie początku zadania
\newcommand{\zadStop}{\end{zad}}   %oznaczenie końca zadania
%Makra opcjonarne (nie muszą występować):
\newcommand{\rozwStart}[2]{\noindent \textbf{Rozwiązanie (autor #1 , recenzent #2): }\newline} %oznaczenie początku rozwiązania, opcjonarnie można wprowadzić informację o autorze rozwiązania zadania i recenzencie poprawności wykonania rozwiązania zadania
\newcommand{\rozwStop}{\newline}                                            %oznaczenie końca rozwiązania
\newcommand{\odpStart}{\noindent \textbf{Odpowiedź:}\newline}    %oznaczenie początku odpowiedzi końcowej (wypisanie wyniku)
\newcommand{\odpStop}{\newline}                                             %oznaczenie końca odpowiedzi końcowej (wypisanie wyniku)
\newcommand{\testStart}{\noindent \textbf{Test:}\newline} %ewentualne możliwe opcje odpowiedzi testowej: A. ? B. ? C. ? D. ? itd.
\newcommand{\testStop}{\newline} %koniec wprowadzania odpowiedzi testowych
\newcommand{\kluczStart}{\noindent \textbf{Test poprawna odpowiedź:}\newline} %klucz, poprawna odpowiedź pytania testowego (jedna literka): A lub B lub C lub D itd.
\newcommand{\kluczStop}{\newline} %koniec poprawnej odpowiedzi pytania testowego 
\newcommand{\wstawGrafike}[2]{\begin{figure}[h] \includegraphics[scale=#2] {#1} \end{figure}} %gdyby była potrzeba wstawienia obrazka, parametry: nazwa pliku, skala (jak nie wiesz co wpisać, to wpisz 1)

\begin{document}
\maketitle


\kategoria{Wikieł/Z3.16a}
\zadStart{Zadanie z Wikieł Z 3.16 a) moja wersja nr [nrWersji]}
%[a]:[2,3,4,5,6,7,8]
%[b]:[2,3,4,5,6,7,8]
%[c]:[2,3,4,5,6,7,8]
%[d]:[2,3,4,5,6,7,8]
%[g]:[2,3,4,5,6,7,8]
%[gc]=[g]*[c]
%[agc]=[a]+[gc]
%[wynik]=int([agc]/[g])
%[wynik]!=[c] and math.gcd([agc],[g])==[g] and [a]!=[b] and [b]!=[c] and [c]!=[d] and [a]!=[c] and [b]!=[d] and [a]!=[d] and [wynik]!=[g] and math.gcd([a],[c])==1 and [c]!=[g]
Wyznaczyć wartość parametru $p\in\mathbb{R}$ tak, aby ciąg o wyrazie ogólnym $a_n$ miał granicę równą $g$, jeśli
$$a_n=\frac{[a]n-[b]}{(p-[c])n+[d]},\qquad g=[g].$$
\zadStop
\rozwStart{Adrianna Stobiecka}{}
Zaczniemy od obliczenia granicy ciągu $a_n$.
$$\lim_{n\to\infty}\frac{[a]n-[b]}{(p-[c])n+[d]}=\lim_{n\to\infty}\frac{n([a]-\frac{[b]}{n})}{n(p-[c]+\frac{[d]}{n})}=\lim_{n\to\infty}\frac{[a]-\frac{[b]}{n}}{p-[c]+\frac{[d]}{n}}=(*)$$
Wiemy, że
$$\lim_{n\to\infty}\frac{[b]}{n}=0$$
oraz
$$\lim_{n\to\infty}\frac{[d]}{n}=0.$$
Zatem:
$$(*)=\frac{[a]}{p-[c]}$$
Mianownik nie może być równy $0$, a zatem $p\ne[c]$. Wiemy, że granica ciągu $a_n$ jest równa $g=[g]$. Zatem mamy:
$$\frac{[a]}{p-[c]}=[g]\Leftrightarrow [g](p-[c])=[a]\Leftrightarrow [g]p-[gc]=[a]$$
$$\Leftrightarrow[g]p=[a]+[gc]\Leftrightarrow[g]p=[agc]\Leftrightarrow p=\frac{[agc]}{[g]}=[wynik]$$
Otrzymujemy stąd, że $p=[wynik]$.
\rozwStop
\odpStart
$p=[wynik]$
\odpStop
\testStart
A.$p=[agc]$
B.$p=-[g]$
C.$p=-[wynik]$
D.$p=[c]$
E.$p=[g]$
F.$p=[wynik]$
G.$p=0$
H.$p=-[c]$
I.$p=-[agc]$
\testStop
\kluczStart
F
\kluczStop



\end{document}
