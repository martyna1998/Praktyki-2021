\documentclass[12pt, a4paper]{article}
\usepackage[utf8]{inputenc}
\usepackage{polski}

\usepackage{amsthm}  %pakiet do tworzenia twierdzeń itp.
\usepackage{amsmath} %pakiet do niektórych symboli matematycznych
\usepackage{amssymb} %pakiet do symboli mat., np. \nsubseteq
\usepackage{amsfonts}
\usepackage{graphicx} %obsługa plików graficznych z rozszerzeniem png, jpg
\theoremstyle{definition} %styl dla definicji
\newtheorem{zad}{} 
\title{Multizestaw zadań}
\author{Mirella Narewska}
%\date{\today}
\date{}
\newcounter{liczniksekcji}
\newcommand{\kategoria}[1]{\section{#1}} %olreślamy nazwę kateforii zadań
\newcommand{\zadStart}[1]{\begin{zad}#1\newline} %oznaczenie początku zadania
\newcommand{\zadStop}{\end{zad}}   %oznaczenie końca zadania
%Makra opcjonarne (nie muszą występować):
\newcommand{\rozwStart}[2]{\noindent \textbf{Rozwiązanie (autor #1 , recenzent #2): }\newline} %oznaczenie początku rozwiązania, opcjonarnie można wprowadzić informację o autorze rozwiązania zadania i recenzencie poprawności wykonania rozwiązania zadania
\newcommand{\rozwStop}{\newline}                                            %oznaczenie końca rozwiązania
\newcommand{\odpStart}{\noindent \textbf{Odpowiedź:}\newline}    %oznaczenie początku odpowiedzi końcowej (wypisanie wyniku)
\newcommand{\odpStop}{\newline}                                             %oznaczenie końca odpowiedzi końcowej (wypisanie wyniku)
\newcommand{\testStart}{\noindent \textbf{Test:}\newline} %ewentualne możliwe opcje odpowiedzi testowej: A. ? B. ? C. ? D. ? itd.
\newcommand{\testStop}{\newline} %koniec wprowadzania odpowiedzi testowych
\newcommand{\kluczStart}{\noindent \textbf{Test poprawna odpowiedź:}\newline} %klucz, poprawna odpowiedź pytania testowego (jedna literka): A lub B lub C lub D itd.
\newcommand{\kluczStop}{\newline} %koniec poprawnej odpowiedzi pytania testowego 
\newcommand{\wstawGrafike}[2]{\begin{figure}[h] \includegraphics[scale=#2] {#1} \end{figure}} %gdyby była potrzeba wstawienia obrazka, parametry: nazwa pliku, skala (jak nie wiesz co wpisać, to wpisz 1)

\begin{document}
\maketitle


\kategoria{Wikieł/z1.14m}
\zadStart{Zadanie z Wikieł z1.40  moja wersja nr [nrWersji]}
%[a]:[2,3,4,5,6,7,8,9,10,11,12,13,14,15,16,17,18,19,20]
%[c]:[8,9,10,11,12,13]
%[b]:[2,3,4,5,6,7]
%[d]=[a] -1
%[e]=[c]-[b]
%[c]>[b] 
%[p]=math.sqrt([d])
%[z]=int([p])
%[s]=math.sqrt([e])
%[f]=int([s])
%[t]=[f]/[z]
%[e]!=0 and [p].is_integer()==True and [s].is_integer()==True and [t].is_integer()==False and math.gcd([f],[z])==1
Obliczyć, dla jakich x wartości funkcji $f(x) =[a]x^2 +[b]$ są większe od wartości funkcji $g(x)=x^2 + [c]$.
\zadStop
\rozwStart{Mirella Narewska}{}
$$$$
$$f(x)>g(x)$$
$$[a]x^2 + [b] >x^2 + [c]$$
$$[d]x^2 - [e] >0$$
$$([z]x + [f])([z]x - [f])>0$$
$$x \in(\infty ,-\frac{[f]}{[z]}) \cup x \in(\frac{[f]}{[z]},\infty) $$
\rozwStop
\odpStart
$x \in(\infty ,-\frac{[f]}{[z]}) \cup x \in(\frac{[f]}{[z]},\infty)$
\odpStop
\testStart
A.$x \in(0,-\frac{[f]}{[z]}) \cup x \in(\frac{[f]}{[z]},2)$
\\
B.$ \in(-[f],[f])$
\\
C.$x \in(\infty ,-\frac{[d]}{[z]}) \cup x \in(\frac{[f]}{[a]},\infty)$
\\
D.$x \in(\infty ,-\frac{[f]}) \cup x \in(\frac{[f]}{[z]},\infty)$
\\
E.$x \in(\infty ,-\frac{[f]}{[z]}) \cup x \in(\frac{[f]}{[z]},\infty)$
\testStop
\kluczStart
E
\kluczStop



\end{document}