\documentclass[12pt, a4paper]{article}
\usepackage[utf8]{inputenc}
\usepackage{polski}

\usepackage{amsthm}  %pakiet do tworzenia twierdzeń itp.
\usepackage{amsmath} %pakiet do niektórych symboli matematycznych
\usepackage{amssymb} %pakiet do symboli mat., np. \nsubseteq
\usepackage{amsfonts}
\usepackage{graphicx} %obsługa plików graficznych z rozszerzeniem png, jpg
\theoremstyle{definition} %styl dla definicji
\newtheorem{zad}{} 
\title{Multizestaw zadań}
\author{Robert Fidytek}
%\date{\today}
\date{}
\newcounter{liczniksekcji}
\newcommand{\kategoria}[1]{\section{#1}} %olreślamy nazwę kateforii zadań
\newcommand{\zadStart}[1]{\begin{zad}#1\newline} %oznaczenie początku zadania
\newcommand{\zadStop}{\end{zad}}   %oznaczenie końca zadania
%Makra opcjonarne (nie muszą występować):
\newcommand{\rozwStart}[2]{\noindent \textbf{Rozwiązanie (autor #1 , recenzent #2): }\newline} %oznaczenie początku rozwiązania, opcjonarnie można wprowadzić informację o autorze rozwiązania zadania i recenzencie poprawności wykonania rozwiązania zadania
\newcommand{\rozwStop}{\newline}                                            %oznaczenie końca rozwiązania
\newcommand{\odpStart}{\noindent \textbf{Odpowiedź:}\newline}    %oznaczenie początku odpowiedzi końcowej (wypisanie wyniku)
\newcommand{\odpStop}{\newline}                                             %oznaczenie końca odpowiedzi końcowej (wypisanie wyniku)
\newcommand{\testStart}{\noindent \textbf{Test:}\newline} %ewentualne możliwe opcje odpowiedzi testowej: A. ? B. ? C. ? D. ? itd.
\newcommand{\testStop}{\newline} %koniec wprowadzania odpowiedzi testowych
\newcommand{\kluczStart}{\noindent \textbf{Test poprawna odpowiedź:}\newline} %klucz, poprawna odpowiedź pytania testowego (jedna literka): A lub B lub C lub D itd.
\newcommand{\kluczStop}{\newline} %koniec poprawnej odpowiedzi pytania testowego 
\newcommand{\wstawGrafike}[2]{\begin{figure}[h] \includegraphics[scale=#2] {#1} \end{figure}} %gdyby była potrzeba wstawienia obrazka, parametry: nazwa pliku, skala (jak nie wiesz co wpisać, to wpisz 1)

\begin{document}
\maketitle


\kategoria{Wikieł/Z1.79i}
\zadStart{Zadanie z Wikieł Z 1.79 i) moja wersja nr [nrWersji]}
%[b]:[3,4,12,24,40,8,15,5,60,35,9,16,6]
%[a]:[9,16,49,64,81,121,144]
%[a1]=int(math.sqrt([a]))
%[b1]=[b]*[b]
%[b2]=[b1]+[a]
%[p]=int(math.sqrt([b2]))
%[p]-(math.sqrt([b2]))==0 
Rozwiązać nierówność $\sqrt{x^2-[a]}<[b]$
\zadStop
\rozwStart{Barbara Bączek}{}
Zaczniemy od wyznaczenia dziedziny.
$$D:x^2-[a] \geq 0$$
$$D: x \in (-\infty, -[a1]] \cup [[a1], \infty)$$
Obie strony nierówności wyjściowej są nieujemne.
$$\sqrt{x^2-[a]}<[b]$$
$$x^2 - [a] < [b1]$$
$$x^2<[b2] \Leftrightarrow x \in (-[p],[p])$$
Uwzględniając dziedzinę:  
$x \in (-[p],-[a1]] \cup [[a1], [p])$
\rozwStop
\odpStart
$x \in (-[p],-[a1]] \cup [[a1], [p])$
\odpStop
\testStart
A.$x \in [-[a1],0)$
B.$x \in (-[p],-[a1]] \cup [[a1], [p])$
C.$x \in (-[p], \infty)$
D.$x \in [[a1], \infty)$
E.$x \in [-[a1],[a1])$
G.$x \in x \in (-[p],-[a1]) \cup ([a1], [p])$
H.$x \in (-[p],0) \cup ([p], \infty)$
\testStop
\kluczStart
B
\kluczStop



\end{document}