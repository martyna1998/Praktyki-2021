\documentclass[12pt, a4paper]{article}
\usepackage[utf8]{inputenc}
\usepackage{polski}

\usepackage{amsthm}  %pakiet do tworzenia twierdzeń itp.
\usepackage{amsmath} %pakiet do niektórych symboli matematycznych
\usepackage{amssymb} %pakiet do symboli mat., np. \nsubseteq
\usepackage{amsfonts}
\usepackage{graphicx} %obsługa plików graficznych z rozszerzeniem png, jpg
\theoremstyle{definition} %styl dla definicji
\newtheorem{zad}{} 
\title{Multizestaw zadań}
\author{Robert Fidytek}
%\date{\today}
\date{}
\newcounter{liczniksekcji}
\newcommand{\kategoria}[1]{\section{#1}} %olreślamy nazwę kateforii zadań
\newcommand{\zadStart}[1]{\begin{zad}#1\newline} %oznaczenie początku zadania
\newcommand{\zadStop}{\end{zad}}   %oznaczenie końca zadania
%Makra opcjonarne (nie muszą występować):
\newcommand{\rozwStart}[2]{\noindent \textbf{Rozwiązanie (autor #1 , recenzent #2): }\newline} %oznaczenie początku rozwiązania, opcjonarnie można wprowadzić informację o autorze rozwiązania zadania i recenzencie poprawności wykonania rozwiązania zadania
\newcommand{\rozwStop}{\newline}                                            %oznaczenie końca rozwiązania
\newcommand{\odpStart}{\noindent \textbf{Odpowiedź:}\newline}    %oznaczenie początku odpowiedzi końcowej (wypisanie wyniku)
\newcommand{\odpStop}{\newline}                                             %oznaczenie końca odpowiedzi końcowej (wypisanie wyniku)
\newcommand{\testStart}{\noindent \textbf{Test:}\newline} %ewentualne możliwe opcje odpowiedzi testowej: A. ? B. ? C. ? D. ? itd.
\newcommand{\testStop}{\newline} %koniec wprowadzania odpowiedzi testowych
\newcommand{\kluczStart}{\noindent \textbf{Test poprawna odpowiedź:}\newline} %klucz, poprawna odpowiedź pytania testowego (jedna literka): A lub B lub C lub D itd.
\newcommand{\kluczStop}{\newline} %koniec poprawnej odpowiedzi pytania testowego 
\newcommand{\wstawGrafike}[2]{\begin{figure}[h] \includegraphics[scale=#2] {#1} \end{figure}} %gdyby była potrzeba wstawienia obrazka, parametry: nazwa pliku, skala (jak nie wiesz co wpisać, to wpisz 1)

\begin{document}
\maketitle


\kategoria{Wikieł/Z4.20j}
\zadStart{Zadanie z Wikieł Z 4.20j) moja wersja nr [nrWersji]}
%[b]:[2,3,4,5,6,7,8,9,10,11,12,13,14,15]
%[c]:[2,3,4,5,6,7,8,9,10,11,12,13,14,15]
%[bc]=[b]*[c]
%math.gcd([b],[c])==1 and math.gcd([bc],4)==1 and math.gcd([bc],9)==1
Wyznaczyć wartości parametru tak, aby funkcja $
f(x) = \left\{ \begin{array}{ll}
\frac{\sin([b]x)}{[c]x} & \textrm{dla $x\neq 0$}\\
a^{2} & \textrm{dla $x=0$}
\end{array} \right.
$ była ciągła.
\zadStop
\rozwStart{Justyna Chojecka}{}
Dla dowolnej wartości parametru $a$ funkcja jest ciągła w przedziałach $(-\infty,0)$ i $(0,\infty)$. Jedynym punktem, w którym funkcja mogłaby być nieciągła, jest punkt $x_{0}=0$.\\
Wartość funkcji w tym punkcie jest równa $f(0)=a^{2}$.\\
Obliczamy następującą granicę:
$$\lim\limits_{x\to 0}\frac{\sin([b]x)}{[c]x}=\frac{1}{[c]}\lim\limits_{x\to 0}\frac{\sin([b]x)}{x}\overset{l'H}{=}\frac{1}{[c]}\lim\limits_{x\to 0}\frac{[b]\cos([b]x)}{1}=\frac{1}{[c]}\cdot\frac{[b]\cos([b]\cdot 0)}{1}=\frac{[b]}{[c]}.$$
A zatem
$$f(0)=\lim\limits_{x\to 0}\frac{\sin([b]x)}{[c]x}\iff a^{2}=\frac{[b]}{[c]} \iff a=-\frac{\sqrt{[bc]}}{[c]}\lor a=\frac{\sqrt{[bc]}}{[c]}.$$
Stąd dla wartości parametru $a\in\left\{-\frac{\sqrt{[bc]}}{[c]},\frac{\sqrt{[bc]}}{[c]}\right\}$ funkcja jest ciągła dla wszystkich $x\in\mathbb{R}.$
\rozwStop
\odpStart
$a\in\left\{-\frac{\sqrt{[bc]}}{[c]},\frac{\sqrt{[bc]}}{[c]}\right\}$
\odpStop
\testStart
A.$a\in\left\{-\frac{\sqrt{[bc]}}{[c]},\frac{\sqrt{[bc]}}{[c]}\right\}$\\
B.$a=-\frac{\sqrt{[c]}}{[bc]}$\\
C.$a=\frac{\sqrt{[c]}}{[bc]}$\\
D.$a=-\frac{\sqrt{[bc]}}{[c]}$\\
E.$a\in\left\{-\frac{\sqrt{[c]}}{[bc]},\frac{\sqrt{[c]}}{[bc]}\right\}$\\
F.$a\in\left\{-\frac{\sqrt{[bc]}}{[c]},\frac{\sqrt{[c]}}{[bc]}\right\}$\\
G.$a\in\{-[c],[c]\}$\\
H.$a=\frac{\sqrt{[bc]}}{[c]}$\\
I.$a\in\left\{-\frac{\sqrt{[c]}}{[bc]},\frac{\sqrt{[bc]}}{[c]}\right\}$\\
\testStop
\kluczStart
A
\kluczStop



\end{document}