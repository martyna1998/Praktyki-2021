\documentclass[12pt, a4paper]{article}
\usepackage[utf8]{inputenc}
\usepackage{polski}

\usepackage{amsthm}  %pakiet do tworzenia twierdzeń itp.
\usepackage{amsmath} %pakiet do niektórych symboli matematycznych
\usepackage{amssymb} %pakiet do symboli mat., np. \nsubseteq
\usepackage{amsfonts}
\usepackage{graphicx} %obsługa plików graficznych z rozszerzeniem png, jpg
\theoremstyle{definition} %styl dla definicji
\newtheorem{zad}{} 
\title{Multizestaw zadań}
\author{Mirella Narewska}
%\date{\today}
\date{}
\newcounter{liczniksekcji}
\newcommand{\kategoria}[1]{\section{#1}} %olreślamy nazwę kateforii zadań
\newcommand{\zadStart}[1]{\begin{zad}#1\newline} %oznaczenie początku zadania
\newcommand{\zadStop}{\end{zad}}   %oznaczenie końca zadania
%Makra opcjonarne (nie muszą występować):
\newcommand{\rozwStart}[2]{\noindent \textbf{Rozwiązanie (autor #1 , recenzent #2): }\newline} %oznaczenie początku rozwiązania, opcjonarnie można wprowadzić informację o autorze rozwiązania zadania i recenzencie poprawności wykonania rozwiązania zadania
\newcommand{\rozwStop}{\newline}                                            %oznaczenie końca rozwiązania
\newcommand{\odpStart}{\noindent \textbf{Odpowiedź:}\newline}    %oznaczenie początku odpowiedzi końcowej (wypisanie wyniku)
\newcommand{\odpStop}{\newline}                                             %oznaczenie końca odpowiedzi końcowej (wypisanie wyniku)
\newcommand{\testStart}{\noindent \textbf{Test:}\newline} %ewentualne możliwe opcje odpowiedzi testowej: A. ? B. ? C. ? D. ? itd.
\newcommand{\testStop}{\newline} %koniec wprowadzania odpowiedzi testowych
\newcommand{\kluczStart}{\noindent \textbf{Test poprawna odpowiedź:}\newline} %klucz, poprawna odpowiedź pytania testowego (jedna literka): A lub B lub C lub D itd.
\newcommand{\kluczStop}{\newline} %koniec poprawnej odpowiedzi pytania testowego 
\newcommand{\wstawGrafike}[2]{\begin{figure}[h] \includegraphics[scale=#2] {#1} \end{figure}} %gdyby była potrzeba wstawienia obrazka, parametry: nazwa pliku, skala (jak nie wiesz co wpisać, to wpisz 1)

\begin{document}
\maketitle


\kategoria{Wikieł/Z1.58n}
\zadStart{Zadanie z Wikieł Z 1.58 n) moja wersja nr [nrWersji]}
%[a]:[1,2,3,4,5,6,7,8,9,10,11,12,13,14,15,16,17,18,20]
%[b]:[1,2,3,4,5,6,7,8,9,10,11,12,13,14,15]
%[c]=[a]-[b]
%[delta]=[c]**2-4*[a]*[a]
%[pierw2]=pow([delta],1/2)
%[pierw1]=[pierw2].real
%[pierw]=int([pierw1])
%[m]=2*[a]
%[licz1]=-[c]-[pierw]
%[licz2]=-[c]+[pierw]
%[uu1]=[licz1]/[m]
%[uu2]=[licz2]/[m]
%[u1]=int([uu1])
%[u2]=int([uu2])
%[a]>[b] and [c]**2>4*[a]*[a] and [licz1]!=0 and [licz2]!=0 and [pierw2].is_integer()==True and [uu1].is_integer()==True and [uu2].is_integer()==True 

Rozwiązać równanie $[a]x^3-[b]x^2+[b]x-[a]=0$.
\zadStop
\rozwStart{Mirella Narewska}{}
$$[a]x^3-[b]x^2+[b]x-[a]=0$$
$$[a](x^3-1)-[b]x(x-1)=0$$
$$[a](x-1)(x^2+x+1)-[b]x(x-1)=0$$
$$(x-1)([a]x^2+[a]x+[a]-[b]x)=0$$
$$(x-1)([a]x^2+[c]x+[a]=0$$
$$x=1 \vee [a]x^2+[c]x+[a]=0$$
$$\triangle=[c]^2-4\cdot [a] \cdot [a]=[delta] \Rightarrow \sqrt{\triangle}=[pierw]$$
$$x=1 \vee x=\frac{-[c]-[pierw]}{2\cdot[a]} \vee x=\frac{-[c]+[pierw]}{2\cdot[a]}$$
$$x=1 \vee x=\frac{[licz1]}{[m]} \vee x=\frac{[licz2]}{[m]}$$
$$x=1 \vee x=[u1] \vee x=[u2]$$
\odpStart
$x=1 \vee x=[u1] \vee x=[u2]$
\odpStop
\testStart
A. $x=1 \vee x=[u1] \vee x=[u2]$ \\
B. $x=1 \vee x=[u1] \vee x=[u2]$ \\
C. $x=1 \vee x=[u1] \vee x=[u2]$ \\
D. $x=1 \vee x=[u1] \vee x=[u2]$ 
\testStop
\kluczStart
A
\kluczStop



\end{document}