\documentclass[12pt, a4paper]{article}
\usepackage[utf8]{inputenc}
\usepackage{polski}

\usepackage{amsthm}  %pakiet do tworzenia twierdzeń itp.
\usepackage{amsmath} %pakiet do niektórych symboli matematycznych
\usepackage{amssymb} %pakiet do symboli mat., np. \nsubseteq
\usepackage{amsfonts}
\usepackage{graphicx} %obsługa plików graficznych z rozszerzeniem png, jpg
\theoremstyle{definition} %styl dla definicji
\newtheorem{zad}{} 
\title{Multizestaw zadań}
\author{Robert Fidytek}
%\date{\today}
\date{}
\newcounter{liczniksekcji}
\newcommand{\kategoria}[1]{\section{#1}} %olreślamy nazwę kateforii zadań
\newcommand{\zadStart}[1]{\begin{zad}#1\newline} %oznaczenie początku zadania
\newcommand{\zadStop}{\end{zad}}   %oznaczenie końca zadania
%Makra opcjonarne (nie muszą występować):
\newcommand{\rozwStart}[2]{\noindent \textbf{Rozwiązanie (autor #1 , recenzent #2): }\newline} %oznaczenie początku rozwiązania, opcjonarnie można wprowadzić informację o autorze rozwiązania zadania i recenzencie poprawności wykonania rozwiązania zadania
\newcommand{\rozwStop}{\newline}                                            %oznaczenie końca rozwiązania
\newcommand{\odpStart}{\noindent \textbf{Odpowiedź:}\newline}    %oznaczenie początku odpowiedzi końcowej (wypisanie wyniku)
\newcommand{\odpStop}{\newline}                                             %oznaczenie końca odpowiedzi końcowej (wypisanie wyniku)
\newcommand{\testStart}{\noindent \textbf{Test:}\newline} %ewentualne możliwe opcje odpowiedzi testowej: A. ? B. ? C. ? D. ? itd.
\newcommand{\testStop}{\newline} %koniec wprowadzania odpowiedzi testowych
\newcommand{\kluczStart}{\noindent \textbf{Test poprawna odpowiedź:}\newline} %klucz, poprawna odpowiedź pytania testowego (jedna literka): A lub B lub C lub D itd.
\newcommand{\kluczStop}{\newline} %koniec poprawnej odpowiedzi pytania testowego 
\newcommand{\wstawGrafike}[2]{\begin{figure}[h] \includegraphics[scale=#2] {#1} \end{figure}} %gdyby była potrzeba wstawienia obrazka, parametry: nazwa pliku, skala (jak nie wiesz co wpisać, to wpisz 1)

\begin{document}
\maketitle


\kategoria{Wikieł/Z1.70f}
\zadStart{Zadanie z Wikieł Z 1.70 f) moja wersja nr [nrWersji]}
%[a]:[2,3,4,5,6]
%[b]:[2,3,4,5,6]
%[c]:[2,3,4,5,6]
%[d]:[2,3,4,5,6]
%[a]=random.randint(2,6)
%[b]=random.randint(2,6)
%[c]=random.randint(2,6)
%[d]=random.randint(2,6)
%[ab]=[a]-[b]
%[adbc]=([a]*[d])+([b]*[c])
%([a]-[b])==1
Rozwiązać nierówność: $\frac{[a]}{x+[c]}\leq\frac{[b]}{x-[d]}$
\zadStop
\rozwStart{Pascal Nawrocki}{Jakub Ulrych}
Zaczniemy od wyznaczenia dziedziny. Zauważmy, że nierówność nam się psuje dla x, których mianowniki zerują się, nie chcemy otrzymać przecież dzielenia przez zero. Stąd widzimy, że $x\in\mathbb{R}\symbol{92}\{-[c],[d]\}$. Mając dziedzinę zabieramy się za rozwiązywanie tej nierówności.
$$\frac{[a]}{x+[c]}\leq\frac{[b]}{x-[d]}$$
$$\frac{[a]}{x+[c]}-\frac{[b]}{x-[d]}\leq0$$
$$\frac{[a](x-[d])-[b](x+[c])}{(x+[c])(x-[d])}\leq0$$
$$\frac{x-[adbc]}{(x+[c])(x-[d])}\leq0$$
Możemy teraz rozwiązać zadanie w sposób standardowy czyli przemnożyc przez mianownik do kwadratu i policzyć nierówność wielomianową, ale to dużo liczenia i łatwo się pomylić. Zastosujemy sztuczkę. Zauważmy dwie rzeczy:
\begin{itemize}
\item Zero otrzymamy tylko gdy licznik nam się wyzeruje.
\item Spełnienie warunku bycia mniejszym od zera otrzymamy gdy licznik \textbf{albo} (alternatywa wykluczająca) mianownik będą mniejsze od zera. 
\end{itemize}
Więc rozłożymy sobie naszą nierównośc na dwa przypadki:
\begin{enumerate}
\item $(x+[c])(x-[d])>0 \wedge x-[adbc]\leq0$
\item  $(x+[c])(x-[d])<0 \wedge x-[adbc]\geq0$
\end{enumerate}
Rozwiążmy:
\begin{enumerate}
\item 
$$(x+[c])(x-[d])>0\Leftrightarrow x\in(-\infty,-[c])\cup([d],\infty)$$
$$x-[adbc]\leq0\Leftrightarrow x\leq[adbc]$$
Bierzemy część wspólną rozwiązań i otrzymujemy: $x\in(-\infty,-[c])\cup([d],[adbc]]$.
\item 
$$(x+[c])(x-[d])<0\Leftrightarrow x\in(-[c],[d])$$
$$ x-[adbc]\geq0\Leftrightarrow  x\geq[adbc]$$
Bierzemy część wspólną rozwiązań i otrzymujemy: $x\in\emptyset$.
\end{enumerate}
Teraz sumujemy oba przypadki i dostajemy odpowiedź:
$$x\in(-\infty,-[c])\cup([d],[adbc]]$$.
\odpStart
$x\in(-\infty,-[c])\cup([d],[adbc]]$
\odpStop
\testStart
A.$x\in(-\infty,-[c])\cup([d],[adbc]]$
\\
B.$x\in([b],\infty)$
\\
C.$x\in\emptyset$
\\
D.$x\in(-\infty,-[a])$
\testStop
\kluczStart
A
\kluczStop
\end{document}