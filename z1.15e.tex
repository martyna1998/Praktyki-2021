\documentclass[12pt, a4paper]{article}
\usepackage[utf8]{inputenc}
\usepackage{polski}

\usepackage{amsthm}  %pakiet do tworzenia twierdzeń itp.
\usepackage{amsmath} %pakiet do niektórych symboli matematycznych
\usepackage{amssymb} %pakiet do symboli mat., np. \nsubseteq
\usepackage{amsfonts}
\usepackage{graphicx} %obsługa plików graficznych z rozszerzeniem png, jpg
\theoremstyle{definition} %styl dla definicji
\newtheorem{zad}{} 
\title{Multizestaw zadań}
\author{Robert Fidytek}
%\date{\today}
\date{}
\newcounter{liczniksekcji}
\newcommand{\kategoria}[1]{\section{#1}} %olreślamy nazwę kateforii zadań
\newcommand{\zadStart}[1]{\begin{zad}#1\newline} %oznaczenie początku zadania
\newcommand{\zadStop}{\end{zad}}   %oznaczenie końca zadania
%Makra opcjonarne (nie muszą występować):
\newcommand{\rozwStart}[2]{\noindent \textbf{Rozwiązanie (autor #1 , recenzent #2): }\newline} %oznaczenie początku rozwiązania, opcjonarnie można wprowadzić informację o autorze rozwiązania zadania i recenzencie poprawności wykonania rozwiązania zadania
\newcommand{\rozwStop}{\newline}                                            %oznaczenie końca rozwiązania
\newcommand{\odpStart}{\noindent \textbf{Odpowiedź:}\newline}    %oznaczenie początku odpowiedzi końcowej (wypisanie wyniku)
\newcommand{\odpStop}{\newline}                                             %oznaczenie końca odpowiedzi końcowej (wypisanie wyniku)
\newcommand{\testStart}{\noindent \textbf{Test:}\newline} %ewentualne możliwe opcje odpowiedzi testowej: A. ? B. ? C. ? D. ? itd.
\newcommand{\testStop}{\newline} %koniec wprowadzania odpowiedzi testowych
\newcommand{\kluczStart}{\noindent \textbf{Test poprawna odpowiedź:}\newline} %klucz, poprawna odpowiedź pytania testowego (jedna literka): A lub B lub C lub D itd.
\newcommand{\kluczStop}{\newline} %koniec poprawnej odpowiedzi pytania testowego 
\newcommand{\wstawGrafike}[2]{\begin{figure}[h] \includegraphics[scale=#2] {#1} \end{figure}} %gdyby była potrzeba wstawienia obrazka, parametry: nazwa pliku, skala (jak nie wiesz co wpisać, to wpisz 1)

\begin{document}
\maketitle


\kategoria{Wikieł/Z1.15e}
\zadStart{Zadanie z Wikieł Z 1.15 e) moja wersja nr [nrWersji]}
%[a]:[2,3,4,5,6]
%[b]:[2,4,6,8,10]
%[c]:[3,5,7,11,13]
%[d]:[2,3,4,5,6]
%[ca]=[c]*[a]
%[cb]=[c]*[b]
%[cd]=[c]*[d]
%[cdcb1]=[cd]-[cb]
%[cdcb2]=-[cd]-[cb]
%[ca1]=[ca]-1
%[ca2]=[ca]+1
%[w1]=-[cdcb1]
%[w2]=-[cdcb2]
%[cd]>[cb] and math.gcd([cdcb2],[ca2])==1 and math.gcd([cdcb2],[ca2])==1
Rozwiąż nierówność $|[a]x+[b]|\leq\frac{x}{[c]}+[d]$.
\zadStop
\rozwStart{Adrianna Stobiecka}{}
$$|[a]x+[b]|\leq\frac{x}{[c]}+[d]$$
$$[a]x+[b]\leq\frac{x}{[c]}+[d]~~\bigg|\cdot[c]\qquad\land\qquad[a]x+[b]\geq-\frac{x}{[c]}-[d]~~\bigg|\cdot[c]$$
$$[c]\cdot[a]x+[c]\cdot[b]\leq[c]\cdot\frac{x}{[c]}+[c]\cdot[d]\qquad\land\qquad[c]\cdot[a]x+[c]\cdot[b]\geq-[c]\cdot\frac{x}{[c]}-[c]\cdot[d]$$
$$[ca]x+[cb]\leq x+[cd]\qquad\land\qquad[ca]x+[cb]\geq-x-[cd]$$
$$[ca]x-x\leq[cd]-[cb]\qquad\land\qquad[ca]x+x\geq-[cd]-[cb]$$
$$[ca1]x\leq[cdcb1]~~\bigg|:[ca1]\qquad\land\qquad[ca2]x\geq[cdcb2]~~\bigg|:[ca2]$$
$$x\leq\frac{[cdcb1]}{[ca1]}\qquad\land\qquad x\geq\frac{[cdcb2]}{[ca2]}$$
$$x\in\bigg[\frac{[cdcb2]}{[ca2]},\frac{[cdcb1]}{[ca1]}\bigg]$$
\rozwStop
\odpStart
$x\in\bigg[\frac{[cdcb2]}{[ca2]},\frac{[cdcb1]}{[ca1]}\bigg]$
\odpStop
\testStart
A.$x\in([w1],[w2])$
B.$x\in\emptyset$
C.$x\in\mathbb{R}$
D.$x\in\bigg[\frac{[cdcb2]}{[ca2]},\frac{[cdcb1]}{[ca1]}\bigg]$
E.$x\in[[w1],[w2]]$
F.$x\in\bigg(\frac{[cdcb2]}{[ca2]},\frac{[cdcb1]}{[ca1]}\bigg)$
G.$x\in[[cdcb2],[cdcb1]]$
H.$x\in\bigg(-\infty,\frac{[cdcb1]}{[ca1]}\bigg]$
I.$x\in([cdcb2],[cdcb1])$
\testStop
\kluczStart
D
\kluczStop



\end{document}