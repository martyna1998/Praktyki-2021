\documentclass[12pt, a4paper]{article}
\usepackage[utf8]{inputenc}
\usepackage{polski}

\usepackage{amsthm}  %pakiet do tworzenia twierdzeń itp.
\usepackage{amsmath} %pakiet do niektórych symboli matematycznych
\usepackage{amssymb} %pakiet do symboli mat., np. \nsubseteq
\usepackage{amsfonts}
\usepackage{graphicx} %obsługa plików graficznych z rozszerzeniem png, jpg
\theoremstyle{definition} %styl dla definicji
\newtheorem{zad}{} 
\title{Multizestaw zadań}
\author{Robert Fidytek}
%\date{\today}
\date{}
\newcounter{liczniksekcji}
\newcommand{\kategoria}[1]{\section{#1}} %olreślamy nazwę kateforii zadań
\newcommand{\zadStart}[1]{\begin{zad}#1\newline} %oznaczenie początku zadania
\newcommand{\zadStop}{\end{zad}}   %oznaczenie końca zadania
%Makra opcjonarne (nie muszą występować):
\newcommand{\rozwStart}[2]{\noindent \textbf{Rozwiązanie (autor #1 , recenzent #2): }\newline} %oznaczenie początku rozwiązania, opcjonarnie można wprowadzić informację o autorze rozwiązania zadania i recenzencie poprawności wykonania rozwiązania zadania
\newcommand{\rozwStop}{\newline}                                            %oznaczenie końca rozwiązania
\newcommand{\odpStart}{\noindent \textbf{Odpowiedź:}\newline}    %oznaczenie początku odpowiedzi końcowej (wypisanie wyniku)
\newcommand{\odpStop}{\newline}                                             %oznaczenie końca odpowiedzi końcowej (wypisanie wyniku)
\newcommand{\testStart}{\noindent \textbf{Test:}\newline} %ewentualne możliwe opcje odpowiedzi testowej: A. ? B. ? C. ? D. ? itd.
\newcommand{\testStop}{\newline} %koniec wprowadzania odpowiedzi testowych
\newcommand{\kluczStart}{\noindent \textbf{Test poprawna odpowiedź:}\newline} %klucz, poprawna odpowiedź pytania testowego (jedna literka): A lub B lub C lub D itd.
\newcommand{\kluczStop}{\newline} %koniec poprawnej odpowiedzi pytania testowego 
\newcommand{\wstawGrafike}[2]{\begin{figure}[h] \includegraphics[scale=#2] {#1} \end{figure}} %gdyby była potrzeba wstawienia obrazka, parametry: nazwa pliku, skala (jak nie wiesz co wpisać, to wpisz 1)

\begin{document}
\maketitle


\kategoria{Wikieł/Z5.56c}
\zadStart{Zadanie z Wikieł Z 5.56 c) moja wersja nr [nrWersji]}
%[a]:[2,3,4,5,6,7,8,9,10,11,12,13,14,15]
%[2a]=[a]*2
%[a21]=[a]*[a]-1
%[ap1]=[a]+1
Wyznaczyć równania asymptot wykresu funkcji:\\
c) $f(x)=\big( x+\frac{1}{x+[a]}\big) arcctg(x)$
\zadStop
\rozwStart{Wojciech Przybylski}{Pascal Nawrocki}
1. Wyznaczamy dziedzinę funkcji.
$$[a]+x\neq0 \Rightarrow x\neq-[a] \Rightarrow x\in\mathbb{R}\setminus\{-[a]\}$$
$$D_{f}=(-\infty,-[a])\cup(-[a],\infty)$$
2. Wyznaczamy granicę. 
$$\lim_{x\to\infty}\big( x+\frac{1}{x+[a]}\big) arcctg(x)=\lim_{x\to\infty}\frac{arcctg(x)}{\frac{1}{x+\frac{1}{x+[a]}}}=\lim_{x\to\infty}\frac{arcctg(x)}{\frac{x+[a]}{x^{2}+[a]x+1}}\stackrel{\text{H}}{=}$$
$$\stackrel{\text{H}}{=}\lim_{x\to\infty}\frac{\frac{-1}{x^{2}+1}}{\frac{x^{2}+[a]x+1-(x+[a])\cdot(2x+[a])}{(x^{2}+[a]x+1)^{2}}}=\lim_{x\to\infty}\frac{\frac{-1}{x^{2}+1}}{\frac{-(x^{2}+[2a]x+[a21])}{(x^{2}+[a]x+1)^{2}}}=$$
$$=\lim_{x\to\infty}\frac{(x^{2}+[a]x+1)^{2}}{(x^{2}+1)\cdot(x^{2}+[2a]x+[a21])}=$$
$$=\lim_{x\to\infty}\frac{x^{4}(1+\frac{[a]}{x}+\frac{1}{x^{2}})^{2}}{x^{4}(1+\frac{1}{x^{2}})\cdot(1+\frac{[2a]}{x}+\frac{[a21]}{x^{2}})}=1$$
$$\lim_{x\to-\infty}\big( x+\frac{1}{x+[a]}\big) arcctg(x)=\lim_{x\to-\infty}(-\infty+0)\cdot\pi=-\infty$$
$$\lim_{x\to-[a]^{-}}\big( x+\frac{1}{x+[a]}\big) arcctg(x)=-\infty$$
$$\lim_{x\to-[a]^{+}}\big( x+\frac{1}{x+[a]}\big) arcctg(x)=\infty$$
Istnieje asymptota pionowa obustronna w $x=-[a]$ oraz asymptota pozioma prawostronna y=1.\\
3. Sprawdzamy, czy istnieje asymtota ukośna.
$$y=ax+b,\hspace{3mm}a=\lim_{x\to-\infty}\frac{f(x)}{x},\hspace{3mm}b=\lim_{x\to-\infty}[f(x)-ax]$$
$$a=\lim_{x\to-\infty}\frac{\big( x+\frac{1}{x+[a]}\big) arcctg(x)}{x}=\lim_{x\to-\infty}\frac{(x^{2}+[a]x+1)arcctg(x)}{x^{2}+[a]x}=$$
$$=\lim_{x\to-\infty}\frac{x^{2}(1+\frac{[a]}{x}+\frac{1}{x^{2}})arcctg(x)}{x^{2}(1+\frac{[a]}{x})}=arcctg(-\infty)=\pi$$
$$b=\lim_{x\to-\infty}\big( x+\frac{1}{x+[a]}\big)arcctg(x)-\pi x=$$
$$=\lim_{x\to-\infty}\frac{x^{2}(arcctg(x)-\pi)+x([a]arcctg(x)-[a]\pi)+arcctg(x)}{x+[a]}=$$
$$=\lim_{x\to-\infty}\frac{arcctg(x)}{x+3}+\lim_{x\to-\infty}\frac{x([a]arcctg(x)-[a]\pi)}{x(1+\frac{[a]}{x})}+\lim_{x\to-\infty}\frac{x^{2}(arcctg(x)-\pi)}{x+3}=$$
$$=\lim_{x\to-\infty}\frac{arcctg(x)-\pi}{\frac{x+3}{x^{2}}}+0+0\stackrel{\text{H}}{=}\lim_{x\to-\infty}\frac{\frac{-1}{x^{2}+1}}{\frac{x^{2}-2x^{2}-[2a]x}{x^{4}}}=$$
$$=\lim_{x\to-\infty}\frac{-x^{4}}{(x^{2}+1)(-x^{2}-[2a]x)}=1$$
Istnieje asymptota ukośna lewostronna $y=\pi x+1$.
\rozwStop
\odpStart
pionowa obustronna $x=-[a]$; pozioma prawostronna $y=1$; ukośna obustronna: $y=\pi x+1$.
\odpStop
\testStart
A. pionowa obustronna $x=-[a]$; pozioma prawostronna $y=1$; ukośna obustronna: $y=\pi x+1$.\\
B. pionowa obustronna $x=-[a]$; pozioma prawostronna $y=1$; ukośna obustronna: $y=\pi x$.\\
C. pionowa lewostronna $x=[a]$; ukośna obustronna: $y=\pi x-1$.\\
D. pionowa prawostronna $x=0$; pozioma: $y=0$.\\
E. pionowa obustronna $x=-[a]$; pozioma obustronna: $y=1$.\\
F. Wykres funkcji $f(x)$ nie ma asymptot.
\testStop
\kluczStart
A
\kluczStop



\end{document}