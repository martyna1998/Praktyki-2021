\documentclass[12pt, a4paper]{article}
\usepackage[utf8]{inputenc}
\usepackage{polski}
\usepackage{amsthm}  %pakiet do tworzenia twierdzeń itp.
\usepackage{amsmath} %pakiet do niektórych symboli matematycznych
\usepackage{amssymb} %pakiet do symboli mat., np. \nsubseteq
\usepackage{amsfonts}
\usepackage{graphicx} %obsługa plików graficznych z rozszerzeniem png, jpg
\theoremstyle{definition} %styl dla definicji
\newtheorem{zad}{} 
\title{Multizestaw zadań}
\author{Patryk Wirkus}
%\date{\today}
\date{}
\newcommand{\kategoria}[1]{\section{#1}}
\newcommand{\zadStart}[1]{\begin{zad}#1\newline}
\newcommand{\zadStop}{\end{zad}}
\newcommand{\rozwStart}[2]{\noindent \textbf{Rozwiązanie (autor #1 , recenzent #2): }\newline}
\newcommand{\rozwStop}{\newline}                                           
\newcommand{\odpStart}{\noindent \textbf{Odpowiedź:}\newline}
\newcommand{\odpStop}{\newline}
\newcommand{\testStart}{\noindent \textbf{Test:}\newline}
\newcommand{\testStop}{\newline}
\newcommand{\kluczStart}{\noindent \textbf{Test poprawna odpowiedź:}\newline}
\newcommand{\kluczStop}{\newline}
\newcommand{\wstawGrafike}[2]{\begin{figure}[h] \includegraphics[scale=#2] {#1} \end{figure}}

\begin{document}
\maketitle

\kategoria{Wikieł/1.101b}


\zadStart{Zadanie z Wikieł Z 1.101 b) moja wersja nr 1}

Oblicz wartości podanych wyrażeń $sin \frac{19\pi}{6}$.
\zadStop
\rozwStart{Patryk Wirkus}{Laura Mieczkowska}
$$sin \frac{19\pi}{6} = cos \frac{7\pi}{6} = -\frac{1}{2}$$
\rozwStop
\odpStart
$-\frac{1}{2}$
\odpStop
\testStart
A.$-\frac{1}{2}$\\
B.$\frac{\sqrt{3}}{2}$\\
C.$-\frac{\sqrt{2}}{2}$\\
D.$\frac{\sqrt{3}}{2}$\\
E.$\frac{1}{2}$\\
F.$-\frac{\sqrt{3}}{2}$\\
G.$1$\\
H.$-1$
\testStop
\kluczStart
A
\kluczStop



\zadStart{Zadanie z Wikieł Z 1.101 b) moja wersja nr 2}

Oblicz wartości podanych wyrażeń $sin \frac{31\pi}{6}$.
\zadStop
\rozwStart{Patryk Wirkus}{Laura Mieczkowska}
$$sin \frac{31\pi}{6} = cos \frac{7\pi}{6} = -\frac{1}{2}$$
\rozwStop
\odpStart
$-\frac{1}{2}$
\odpStop
\testStart
A.$-\frac{1}{2}$\\
B.$\frac{\sqrt{3}}{2}$\\
C.$-\frac{\sqrt{2}}{2}$\\
D.$\frac{\sqrt{3}}{2}$\\
E.$\frac{1}{2}$\\
F.$-\frac{\sqrt{3}}{2}$\\
G.$1$\\
H.$-1$
\testStop
\kluczStart
A
\kluczStop



\zadStart{Zadanie z Wikieł Z 1.101 b) moja wersja nr 3}

Oblicz wartości podanych wyrażeń $sin \frac{43\pi}{6}$.
\zadStop
\rozwStart{Patryk Wirkus}{Laura Mieczkowska}
$$sin \frac{43\pi}{6} = cos \frac{7\pi}{6} = -\frac{1}{2}$$
\rozwStop
\odpStart
$-\frac{1}{2}$
\odpStop
\testStart
A.$-\frac{1}{2}$\\
B.$\frac{\sqrt{3}}{2}$\\
C.$-\frac{\sqrt{2}}{2}$\\
D.$\frac{\sqrt{3}}{2}$\\
E.$\frac{1}{2}$\\
F.$-\frac{\sqrt{3}}{2}$\\
G.$1$\\
H.$-1$
\testStop
\kluczStart
A
\kluczStop



\zadStart{Zadanie z Wikieł Z 1.101 b) moja wersja nr 4}

Oblicz wartości podanych wyrażeń $sin \frac{55\pi}{6}$.
\zadStop
\rozwStart{Patryk Wirkus}{Laura Mieczkowska}
$$sin \frac{55\pi}{6} = cos \frac{7\pi}{6} = -\frac{1}{2}$$
\rozwStop
\odpStart
$-\frac{1}{2}$
\odpStop
\testStart
A.$-\frac{1}{2}$\\
B.$\frac{\sqrt{3}}{2}$\\
C.$-\frac{\sqrt{2}}{2}$\\
D.$\frac{\sqrt{3}}{2}$\\
E.$\frac{1}{2}$\\
F.$-\frac{\sqrt{3}}{2}$\\
G.$1$\\
H.$-1$
\testStop
\kluczStart
A
\kluczStop



\zadStart{Zadanie z Wikieł Z 1.101 b) moja wersja nr 5}

Oblicz wartości podanych wyrażeń $sin \frac{67\pi}{6}$.
\zadStop
\rozwStart{Patryk Wirkus}{Laura Mieczkowska}
$$sin \frac{67\pi}{6} = cos \frac{7\pi}{6} = -\frac{1}{2}$$
\rozwStop
\odpStart
$-\frac{1}{2}$
\odpStop
\testStart
A.$-\frac{1}{2}$\\
B.$\frac{\sqrt{3}}{2}$\\
C.$-\frac{\sqrt{2}}{2}$\\
D.$\frac{\sqrt{3}}{2}$\\
E.$\frac{1}{2}$\\
F.$-\frac{\sqrt{3}}{2}$\\
G.$1$\\
H.$-1$
\testStop
\kluczStart
A
\kluczStop



\zadStart{Zadanie z Wikieł Z 1.101 b) moja wersja nr 6}

Oblicz wartości podanych wyrażeń $sin \frac{79\pi}{6}$.
\zadStop
\rozwStart{Patryk Wirkus}{Laura Mieczkowska}
$$sin \frac{79\pi}{6} = cos \frac{7\pi}{6} = -\frac{1}{2}$$
\rozwStop
\odpStart
$-\frac{1}{2}$
\odpStop
\testStart
A.$-\frac{1}{2}$\\
B.$\frac{\sqrt{3}}{2}$\\
C.$-\frac{\sqrt{2}}{2}$\\
D.$\frac{\sqrt{3}}{2}$\\
E.$\frac{1}{2}$\\
F.$-\frac{\sqrt{3}}{2}$\\
G.$1$\\
H.$-1$
\testStop
\kluczStart
A
\kluczStop



\zadStart{Zadanie z Wikieł Z 1.101 b) moja wersja nr 7}

Oblicz wartości podanych wyrażeń $sin \frac{91\pi}{6}$.
\zadStop
\rozwStart{Patryk Wirkus}{Laura Mieczkowska}
$$sin \frac{91\pi}{6} = cos \frac{7\pi}{6} = -\frac{1}{2}$$
\rozwStop
\odpStart
$-\frac{1}{2}$
\odpStop
\testStart
A.$-\frac{1}{2}$\\
B.$\frac{\sqrt{3}}{2}$\\
C.$-\frac{\sqrt{2}}{2}$\\
D.$\frac{\sqrt{3}}{2}$\\
E.$\frac{1}{2}$\\
F.$-\frac{\sqrt{3}}{2}$\\
G.$1$\\
H.$-1$
\testStop
\kluczStart
A
\kluczStop



\zadStart{Zadanie z Wikieł Z 1.101 b) moja wersja nr 8}

Oblicz wartości podanych wyrażeń $sin \frac{103\pi}{6}$.
\zadStop
\rozwStart{Patryk Wirkus}{Laura Mieczkowska}
$$sin \frac{103\pi}{6} = cos \frac{7\pi}{6} = -\frac{1}{2}$$
\rozwStop
\odpStart
$-\frac{1}{2}$
\odpStop
\testStart
A.$-\frac{1}{2}$\\
B.$\frac{\sqrt{3}}{2}$\\
C.$-\frac{\sqrt{2}}{2}$\\
D.$\frac{\sqrt{3}}{2}$\\
E.$\frac{1}{2}$\\
F.$-\frac{\sqrt{3}}{2}$\\
G.$1$\\
H.$-1$
\testStop
\kluczStart
A
\kluczStop



\zadStart{Zadanie z Wikieł Z 1.101 b) moja wersja nr 9}

Oblicz wartości podanych wyrażeń $sin \frac{115\pi}{6}$.
\zadStop
\rozwStart{Patryk Wirkus}{Laura Mieczkowska}
$$sin \frac{115\pi}{6} = cos \frac{7\pi}{6} = -\frac{1}{2}$$
\rozwStop
\odpStart
$-\frac{1}{2}$
\odpStop
\testStart
A.$-\frac{1}{2}$\\
B.$\frac{\sqrt{3}}{2}$\\
C.$-\frac{\sqrt{2}}{2}$\\
D.$\frac{\sqrt{3}}{2}$\\
E.$\frac{1}{2}$\\
F.$-\frac{\sqrt{3}}{2}$\\
G.$1$\\
H.$-1$
\testStop
\kluczStart
A
\kluczStop



\zadStart{Zadanie z Wikieł Z 1.101 b) moja wersja nr 10}

Oblicz wartości podanych wyrażeń $sin \frac{127\pi}{6}$.
\zadStop
\rozwStart{Patryk Wirkus}{Laura Mieczkowska}
$$sin \frac{127\pi}{6} = cos \frac{7\pi}{6} = -\frac{1}{2}$$
\rozwStop
\odpStart
$-\frac{1}{2}$
\odpStop
\testStart
A.$-\frac{1}{2}$\\
B.$\frac{\sqrt{3}}{2}$\\
C.$-\frac{\sqrt{2}}{2}$\\
D.$\frac{\sqrt{3}}{2}$\\
E.$\frac{1}{2}$\\
F.$-\frac{\sqrt{3}}{2}$\\
G.$1$\\
H.$-1$
\testStop
\kluczStart
A
\kluczStop



\zadStart{Zadanie z Wikieł Z 1.101 b) moja wersja nr 11}

Oblicz wartości podanych wyrażeń $sin \frac{139\pi}{6}$.
\zadStop
\rozwStart{Patryk Wirkus}{Laura Mieczkowska}
$$sin \frac{139\pi}{6} = cos \frac{7\pi}{6} = -\frac{1}{2}$$
\rozwStop
\odpStart
$-\frac{1}{2}$
\odpStop
\testStart
A.$-\frac{1}{2}$\\
B.$\frac{\sqrt{3}}{2}$\\
C.$-\frac{\sqrt{2}}{2}$\\
D.$\frac{\sqrt{3}}{2}$\\
E.$\frac{1}{2}$\\
F.$-\frac{\sqrt{3}}{2}$\\
G.$1$\\
H.$-1$
\testStop
\kluczStart
A
\kluczStop



\zadStart{Zadanie z Wikieł Z 1.101 b) moja wersja nr 12}

Oblicz wartości podanych wyrażeń $sin \frac{151\pi}{6}$.
\zadStop
\rozwStart{Patryk Wirkus}{Laura Mieczkowska}
$$sin \frac{151\pi}{6} = cos \frac{7\pi}{6} = -\frac{1}{2}$$
\rozwStop
\odpStart
$-\frac{1}{2}$
\odpStop
\testStart
A.$-\frac{1}{2}$\\
B.$\frac{\sqrt{3}}{2}$\\
C.$-\frac{\sqrt{2}}{2}$\\
D.$\frac{\sqrt{3}}{2}$\\
E.$\frac{1}{2}$\\
F.$-\frac{\sqrt{3}}{2}$\\
G.$1$\\
H.$-1$
\testStop
\kluczStart
A
\kluczStop





\end{document}
