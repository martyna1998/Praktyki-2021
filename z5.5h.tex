\documentclass[12pt, a4paper]{article}
\usepackage[utf8]{inputenc}
\usepackage{polski}

\usepackage{amsthm}  %pakiet do tworzenia twierdzeń itp.
\usepackage{amsmath} %pakiet do niektórych symboli matematycznych
\usepackage{amssymb} %pakiet do symboli mat., np. \nsubseteq
\usepackage{amsfonts}
\usepackage{graphicx} %obsługa plików graficznych z rozszerzeniem png, jpg
\theoremstyle{definition} %styl dla definicji
\newtheorem{zad}{} 
\title{Multizestaw zadań}
\author{Robert Fidytek}
%\date{\today}
\date{}
\newcounter{liczniksekcji}
\newcommand{\kategoria}[1]{\section{#1}} %olreślamy nazwę kateforii zadań
\newcommand{\zadStart}[1]{\begin{zad}#1\newline} %oznaczenie początku zadania
\newcommand{\zadStop}{\end{zad}}   %oznaczenie końca zadania
%Makra opcjonarne (nie muszą występować):
\newcommand{\rozwStart}[2]{\noindent \textbf{Rozwiązanie (autor #1 , recenzent #2): }\newline} %oznaczenie początku rozwiązania, opcjonarnie można wprowadzić informację o autorze rozwiązania zadania i recenzencie poprawności wykonania rozwiązania zadania
\newcommand{\rozwStop}{\newline}                                            %oznaczenie końca rozwiązania
\newcommand{\odpStart}{\noindent \textbf{Odpowiedź:}\newline}    %oznaczenie początku odpowiedzi końcowej (wypisanie wyniku)
\newcommand{\odpStop}{\newline}                                             %oznaczenie końca odpowiedzi końcowej (wypisanie wyniku)
\newcommand{\testStart}{\noindent \textbf{Test:}\newline} %ewentualne możliwe opcje odpowiedzi testowej: A. ? B. ? C. ? D. ? itd.
\newcommand{\testStop}{\newline} %koniec wprowadzania odpowiedzi testowych
\newcommand{\kluczStart}{\noindent \textbf{Test poprawna odpowiedź:}\newline} %klucz, poprawna odpowiedź pytania testowego (jedna literka): A lub B lub C lub D itd.
\newcommand{\kluczStop}{\newline} %koniec poprawnej odpowiedzi pytania testowego 
\newcommand{\wstawGrafike}[2]{\begin{figure}[h] \includegraphics[scale=#2] {#1} \end{figure}} %gdyby była potrzeba wstawienia obrazka, parametry: nazwa pliku, skala (jak nie wiesz co wpisać, to wpisz 1)

\begin{document}
\maketitle


\kategoria{Wikieł/Z5.5h}
\zadStart{Zadanie z Wikieł Z 5.5 h) moja wersja nr [nrWersji]}
%[a]:[2,3,4,5,6,7,8,9]
%[a1]=5*[a]
%[a2]=int([a1]/2)
%[b]:[2,3,4,5,6,7,8,9]
%[b1]=4*[b]
%[b2]=int([b1]/2)
%[c]:[2,3,4,5,6,7,8,9]
%[c1]=3*[c]
%[c2]=int([c1]/2)
%[d]=random.randint(2,10)
%[d1]=2*[d]
%[d2]=int([d1]/2)
%[e]=random.randint(2,10)
%[e2]=int([e]/2)
%[f]=random.randint(2,10)
%math.gcd([a1],2)!=1 and math.gcd([b1],2)!=1 and math.gcd([c1],2)!=1 and math.gcd([d1],2)!=1 and math.gcd([e],2)!=1 
Wyznacz pochodną funkcji \\ $f(x)=\sqrt{-[a]x^5-[b]x^4+[c]x^3+[d]x^2-[e]x-[f]} $.
\zadStop
\rozwStart{Joanna Świerzbin}{}
$$f(x)=\sqrt{-[a]x^5-[b]x^4+[c]x^3+[d]x^2-[e]x-[f]} $$
$$u(x)=\sqrt{x} \ \ \ \ \ \ u'(x)=\frac{1}{2\sqrt{x}}$$
$$g(x)=-[a]x^5-[b]x^4+[c]x^3+[d]x^2-[e]x-[f] $$ $$ g'(x)=-5\cdot [a]x^4-4\cdot[b]x^3+3\cdot[c]x^2+2\cdot[d]x-[e]$$
$$f'(x)=(u(g(x)))'=u'(g(x))\cdot g'(x) = $$
 $$= \frac{-5\cdot [a]x^4-4\cdot[b]x^3+3\cdot[c]x^2+2\cdot[d]x-[e]}{2\sqrt{-[a]x^5-[b]x^4+[c]x^3+[d]x^2-[e]x-[f]}}=$$
 $$= \frac{-[a2]x^4-[b2]x^3+[c2]x^2+[d2]x-[e2]}{\sqrt{-[a]x^5-[b]x^4+[c]x^3+[d]x^2-[e]x-[f]}}$$
\rozwStop
\odpStart
$ f'(x) = \frac{-[a2]x^4-[b2]x^3+[c2]x^2+[d2]x-[e2]}{\sqrt{-[a]x^5-[b]x^4+[c]x^3+[d]x^2-[e]x-[f]}}$
\odpStop
\testStart
A. $ f'(x) = \frac{-[a2]x^4-[b2]x^3+[c2]x^2+[d2]x-[e2]}{\sqrt{-[a]x^5-[b]x^4+[c]x^3+[d]x^2-[e]x-[f]}}$\\
B. $ f'(x) = {-[a1]x^4-[b1]x^3+[c1]x^2+[d1]x-[e]}$ \\
C. $ f'(x) = \frac{1}{2\sqrt{-[a]x^5-[b]x^4+[c]x^3+[d]x^2-[e]x-[f]}}$ \\
D. $ f'(x) = \frac{-[a1]x^4-[b1]x^3+[c1]x^2+[d1]x-[e]}{\sqrt{-[a]x^5-[b]x^4+[c]x^3+[d]x^2-[e]x-[f]}}$\\
E. $ f'(x) = \frac{-[a1]x^4-[b1]x^3+[c1]x^2+[d1]x-[e]}{2(-[a]x^5-[b]x^4+[c]x^3+[d]x^2-[e]x-[f])}$\\
F. $ f'(x) = \frac{[c1]x^2+[d1]x-[e]}{2\sqrt{-[a]x^5-[b]x^4+[c]x^3+[d]x^2-[e]x-[f]}}$
\testStop
\kluczStart
A
\kluczStop
\end{document}