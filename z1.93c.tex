\documentclass[12pt, a4paper]{article}
\usepackage[utf8]{inputenc}
\usepackage{polski}

\usepackage{amsthm}  %pakiet do tworzenia twierdzeń itp.
\usepackage{amsmath} %pakiet do niektórych symboli matematycznych
\usepackage{amssymb} %pakiet do symboli mat., np. \nsubseteq
\usepackage{amsfonts}
\usepackage{graphicx} %obsługa plików graficznych z rozszerzeniem png, jpg
\theoremstyle{definition} %styl dla definicji
\newtheorem{zad}{} 
\title{Multizestaw zadań}
\author{Robert Fidytek}
%\date{\today}
\date{}
\newcounter{liczniksekcji}
\newcommand{\kategoria}[1]{\section{#1}} %olreślamy nazwę kateforii zadań
\newcommand{\zadStart}[1]{\begin{zad}#1\newline} %oznaczenie początku zadania
\newcommand{\zadStop}{\end{zad}}   %oznaczenie końca zadania
%Makra opcjonarne (nie muszą występować):
\newcommand{\rozwStart}[2]{\noindent \textbf{Rozwiązanie (autor #1 , recenzent #2): }\newline} %oznaczenie początku rozwiązania, opcjonarnie można wprowadzić informację o autorze rozwiązania zadania i recenzencie poprawności wykonania rozwiązania zadania
\newcommand{\rozwStop}{\newline}                                            %oznaczenie końca rozwiązania
\newcommand{\odpStart}{\noindent \textbf{Odpowiedź:}\newline}    %oznaczenie początku odpowiedzi końcowej (wypisanie wyniku)
\newcommand{\odpStop}{\newline}                                             %oznaczenie końca odpowiedzi końcowej (wypisanie wyniku)
\newcommand{\testStart}{\noindent \textbf{Test:}\newline} %ewentualne możliwe opcje odpowiedzi testowej: A. ? B. ? C. ? D. ? itd.
\newcommand{\testStop}{\newline} %koniec wprowadzania odpowiedzi testowych
\newcommand{\kluczStart}{\noindent \textbf{Test poprawna odpowiedź:}\newline} %klucz, poprawna odpowiedź pytania testowego (jedna literka): A lub B lub C lub D itd.
\newcommand{\kluczStop}{\newline} %koniec poprawnej odpowiedzi pytania testowego 
\newcommand{\wstawGrafike}[2]{\begin{figure}[h] \includegraphics[scale=#2] {#1} \end{figure}} %gdyby była potrzeba wstawienia obrazka, parametry: nazwa pliku, skala (jak nie wiesz co wpisać, to wpisz 1)

\begin{document}
\maketitle


\kategoria{Wikieł/Z1.93c}
\zadStart{Zadanie z Wikieł Z 1.93 c) moja wersja nr [nrWersji]}
%[p]:[2,3]
%[r]:[2,3,4,5]
%[a]:[2,3,4,5,6]
%[b]=random.randint(1,12)
%[c]:[2,3,4,5,6]
%[d]=random.randint(1,12)
%[a2]=[a]*[c]
%[b2]=[b]*[c]
%[e]=[d]-[b2]
%[pr]=pow([p],[r])
%[prd]=[pr]*[d]
%[prc]=[pr]*[c]
%[m]=[a]+[prc]
%[l]=[prd]-[b]
%[w]=[l]/[m]
%[pr]<30 and [m]!=0 and (([a]*[w]+[b])/([d]-[c]*[w]))>0 and math.gcd([a],[b])==1 and [b]!=[e]
Rozwiązać równanie $\log_{[p]}{([a]x+[b])} - \log_{[p]}{([d]-[c]x)} = [r]$
\zadStop
\rozwStart{Małgorzata Ugowska}{}
$$\log_{[p]}{([a]x+[b])} - \log_{[p]}{([d]-[c]x)} = [r] \Longleftrightarrow \log_{[p]}{\frac{[a]x+[b]}{[d]-[c]x}}= [r]$$
$$\Longleftrightarrow \frac{[a]x+[b]}{[d]-[c]x}=[p]^{[r]} \Longleftrightarrow  [a]x+[b]=[pr]([d]-[c]x)$$
$$\Longleftrightarrow [a]x+[b]=[prd]-[prc]x \Longleftrightarrow [m]x=[l] \Longleftrightarrow x=\frac{[l]}{[m]}$$
\rozwStop
\odpStart
$x=\frac{[l]}{[m]}$
\odpStop
\testStart
A. $\frac{[e]}{[a2]}$
B. $\frac{[b]}{[a2]}$
C. $\frac{[m]}{[l]}$
D. $\frac{[l]}{[m]}$
E. $[prd]$
\testStop
\kluczStart
D
\kluczStop



\end{document}