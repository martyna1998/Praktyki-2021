\documentclass[12pt, a4paper]{article}
\usepackage[utf8]{inputenc}
\usepackage{polski}

\usepackage{amsthm}  %pakiet do tworzenia twierdzeń itp.
\usepackage{amsmath} %pakiet do niektórych symboli matematycznych
\usepackage{amssymb} %pakiet do symboli mat., np. \nsubseteq
\usepackage{amsfonts}
\usepackage{graphicx} %obsługa plików graficznych z rozszerzeniem png, jpg
\theoremstyle{definition} %styl dla definicji
\newtheorem{zad}{} 
\title{Multizestaw zadań}
\author{Robert Fidytek}
%\date{\today}
\date{}
\newcounter{liczniksekcji}
\newcommand{\kategoria}[1]{\section{#1}} %olreślamy nazwę kateforii zadań
\newcommand{\zadStart}[1]{\begin{zad}#1\newline} %oznaczenie początku zadania
\newcommand{\zadStop}{\end{zad}}   %oznaczenie końca zadania
%Makra opcjonarne (nie muszą występować):
\newcommand{\rozwStart}[2]{\noindent \textbf{Rozwiązanie (autor #1 , recenzent #2): }\newline} %oznaczenie początku rozwiązania, opcjonarnie można wprowadzić informację o autorze rozwiązania zadania i recenzencie poprawności wykonania rozwiązania zadania
\newcommand{\rozwStop}{\newline}                                            %oznaczenie końca rozwiązania
\newcommand{\odpStart}{\noindent \textbf{Odpowiedź:}\newline}    %oznaczenie początku odpowiedzi końcowej (wypisanie wyniku)
\newcommand{\odpStop}{\newline}                                             %oznaczenie końca odpowiedzi końcowej (wypisanie wyniku)
\newcommand{\testStart}{\noindent \textbf{Test:}\newline} %ewentualne możliwe opcje odpowiedzi testowej: A. ? B. ? C. ? D. ? itd.
\newcommand{\testStop}{\newline} %koniec wprowadzania odpowiedzi testowych
\newcommand{\kluczStart}{\noindent \textbf{Test poprawna odpowiedź:}\newline} %klucz, poprawna odpowiedź pytania testowego (jedna literka): A lub B lub C lub D itd.
\newcommand{\kluczStop}{\newline} %koniec poprawnej odpowiedzi pytania testowego 
\newcommand{\wstawGrafike}[2]{\begin{figure}[h] \includegraphics[scale=#2] {#1} \end{figure}} %gdyby była potrzeba wstawienia obrazka, parametry: nazwa pliku, skala (jak nie wiesz co wpisać, to wpisz 1)

\begin{document}
\maketitle


\kategoria{Wikieł/Z2.16}
\zadStart{Zadanie z Wikieł Z 2.16 moja wersja nr [nrWersji]}
%[v1]:[1,2,3]
%[v2]:[1,2,3]
%[v3]:[1,2,3]
%[u1]:[1,2,3]
%[u2]:[1,2,3]
%[u3]:[1,2,3]
%[ku1]=[u1]*[u1]
%[ku2]=[u2]*[u2]
%[ku3]=[u3]*[u3]
%[kv1]=[v1]*[v1]
%[kv2]=[v2]*[v2]
%[kv3]=[v3]*[v3]
%[a]=[u1]*[v1]
%[b]=[u2]*[v2]
%[c]=[u3]*[v3]
%[A]=[a]+[b]+[c]
%[kA]=[A]*[A]
%[U]=[ku1]+[ku2]+[ku3]
%[V]=[kv1]+[kv2]+[kv3]
%[VU]=[V]*[U]
%[x]=[VU]-[kA]
%[VU]>[kA]
Dane są wektory $\overrightarrow{AB}=[[v1],[v2],[v3]]$, $\overrightarrow{AC}=[[u1],[u2],[u3]]$. Obliczyć pole trójkąta ABC.
\zadStop
\rozwStart{Aleksandra Pasińska}{}
$$S=\frac{1}{2}|\overrightarrow{AB}|\cdot |\overrightarrow{AC}|\cdot sin\angle$$
$$\overrightarrow{AB}\circ \overrightarrow{AC}=[v1]\cdot [u1]+[v2]\cdot [u2]+[v3]\cdot [u3]=[A]$$
$$|\overrightarrow{AB}|\cdot |\overrightarrow{AC}|=\sqrt{[v1]^2+[v2]^2+[v3]^2}\cdot \sqrt{[u1]^2+[u2]^2+[u3]^2}=\sqrt{[V]}\cdot\sqrt{[U]}$$
$$cos\angle=\frac{\overrightarrow{AB}\circ \overrightarrow{AC}}{|\overrightarrow{AB}|\cdot |\overrightarrow{AC}|} =\frac{[A]}{\sqrt{[V]}\cdot \sqrt{[U]}}$$
$$sin\angle=\sqrt{1-cos^2\angle}=\sqrt{1-\frac{[kA]}{[V]\cdot [U]}}=\sqrt{\frac{[x]}{[V]\cdot [U]}}$$
$$S=\frac{1}{2}\sqrt{[V]}\cdot\sqrt{[U]}\frac{\sqrt{[x]}}{\sqrt{[V]}\cdot \sqrt{[U]}}=\frac{\sqrt{[x]}}{2}$$
\rozwStop
\odpStart
$S=\frac{\sqrt{[x]}}{2}$\\
\odpStop
\testStart
A.$S=\frac{\sqrt{[x]}}{2}$
B.$S=90$
C.$S=23$
D.$S=1$
E.$S=9$
F.$S=e$
G.$S=4$
H.$S=7$
I.$S=17$
\testStop
\kluczStart
A
\kluczStop



\end{document}