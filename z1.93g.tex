\documentclass[12pt, a4paper]{article}
\usepackage[utf8]{inputenc}
\usepackage{polski}

\usepackage{amsthm}  %pakiet do tworzenia twierdzeń itp.
\usepackage{amsmath} %pakiet do niektórych symboli matematycznych
\usepackage{amssymb} %pakiet do symboli mat., np. \nsubseteq
\usepackage{amsfonts}
\usepackage{graphicx} %obsługa plików graficznych z rozszerzeniem png, jpg
\theoremstyle{definition} %styl dla definicji
\newtheorem{zad}{} 
\title{Multizestaw zadań}
\author{Robert Fidytek}
%\date{\today}
\date{}
\newcounter{liczniksekcji}
\newcommand{\kategoria}[1]{\section{#1}} %olreślamy nazwę kateforii zadań
\newcommand{\zadStart}[1]{\begin{zad}#1\newline} %oznaczenie początku zadania
\newcommand{\zadStop}{\end{zad}}   %oznaczenie końca zadania
%Makra opcjonarne (nie muszą występować):
\newcommand{\rozwStart}[2]{\noindent \textbf{Rozwiązanie (autor #1 , recenzent #2): }\newline} %oznaczenie początku rozwiązania, opcjonarnie można wprowadzić informację o autorze rozwiązania zadania i recenzencie poprawności wykonania rozwiązania zadania
\newcommand{\rozwStop}{\newline}                                            %oznaczenie końca rozwiązania
\newcommand{\odpStart}{\noindent \textbf{Odpowiedź:}\newline}    %oznaczenie początku odpowiedzi końcowej (wypisanie wyniku)
\newcommand{\odpStop}{\newline}                                             %oznaczenie końca odpowiedzi końcowej (wypisanie wyniku)
\newcommand{\testStart}{\noindent \textbf{Test:}\newline} %ewentualne możliwe opcje odpowiedzi testowej: A. ? B. ? C. ? D. ? itd.
\newcommand{\testStop}{\newline} %koniec wprowadzania odpowiedzi testowych
\newcommand{\kluczStart}{\noindent \textbf{Test poprawna odpowiedź:}\newline} %klucz, poprawna odpowiedź pytania testowego (jedna literka): A lub B lub C lub D itd.
\newcommand{\kluczStop}{\newline} %koniec poprawnej odpowiedzi pytania testowego 
\newcommand{\wstawGrafike}[2]{\begin{figure}[h] \includegraphics[scale=#2] {#1} \end{figure}} %gdyby była potrzeba wstawienia obrazka, parametry: nazwa pliku, skala (jak nie wiesz co wpisać, to wpisz 1)

\begin{document}
\maketitle


\kategoria{Wikieł/Z1.93g}
\zadStart{Zadanie z Wikieł Z 1.93 g) moja wersja nr [nrWersji]}
%[p]:[2,3,4,5,6]
%[r]:[1,2,3,4,5]
%[b]:[1,2,3,4,5,6,7,8,9,10,11,12]
%[c]:[2,4,5,7]
%[d]:[1,2,3,4,5,6,7,8,9,10,11,12]
%[r2]=(pow([p],[r]))
%[b2]=[b]*[d]
%[a]=[d]-[b]*[c]
%[b3]=[r2]-[b2]
%[delta]=[a]**2-4*[c]*[b3]
%[pr2]=(pow([delta],(1/2)))
%[pr1]=[pr2].real
%[pr]=int([pr1])
%[z1]=round(([a]-[pr])/(2*[c]),2)
%[z2]=round(([b]+[pr])/(2*[c]),2)
%[r2]<50 and [r2]>[b2] and [delta]>0 and [pr2].is_integer()==True
Rozwiązać równanie $\log_{\frac{1}{[p]}}{(x+[b])} + \log_{\frac{1}{[p]}}{([d]-[c]x)} = -[r]$
\zadStop
\rozwStart{Małgorzata Ugowska}{}
$$\log_{\frac{1}{[p]}}{(x+[b])} + \log_{\frac{1}{[p]}}{([d]-[c]x)} = -[r] \quad \Longleftrightarrow \quad \log_{\frac{1}{[p]}}{(x+[b])([d]-[c]x)}= -[r]$$
$$\Longleftrightarrow \quad (x+[b])([d]-[c]x)= \Big(\frac{1}{[p]}\Big)^{-[r]} \quad \Longleftrightarrow \quad [d]x-[c]x^2+[b] \cdot [d]-[b] \cdot [c]x= [p]^{[r]} $$
$$\Longleftrightarrow \quad [a]x-[c]x^2+[b2]= [r2] \quad \Longleftrightarrow \quad -[c]x^2+[a]x-[b3]= 0 \quad \Longleftrightarrow \quad [c]x^2-[a]x+[b3]= 0$$
Szukamy miejsc zerowych funkcji $[c]x^2-[a]x+[b3]$:
$$ \bigtriangleup = [a]^2-4 \cdot [c] \cdot [b3] = [delta] \quad  \Longrightarrow \quad x_1=\frac{[a]-\sqrt{\bigtriangleup}}{2\cdot [c]} = [z1], \quad x_2=\frac{[a]+\sqrt{\bigtriangleup}}{2\cdot [c]} = [z2]$$
$$ \quad \Longleftrightarrow \quad x \in \{[z1],[z2]\}$$
\rozwStop
\odpStart
$x \in \{[z1],[z2]\}$
\odpStop
\testStart
A. $x \in \{[delta],[c]\}$\\
B. $x \in \{-1,1\}$\\
C. $x \in \{[z1],[z2]\}$\\
D. $[z1]$\\
E. $x \in \{4,5\}$
\testStop
\kluczStart
C
\kluczStop



\end{document}