\documentclass[12pt, a4paper]{article}
\usepackage[utf8]{inputenc}
\usepackage{polski}

\usepackage{amsthm}  %pakiet do tworzenia twierdzeń itp.
\usepackage{amsmath} %pakiet do niektórych symboli matematycznych
\usepackage{amssymb} %pakiet do symboli mat., np. \nsubseteq
\usepackage{amsfonts}
\usepackage{graphicx} %obsługa plików graficznych z rozszerzeniem png, jpg
\theoremstyle{definition} %styl dla definicji
\newtheorem{zad}{} 
\title{Multizestaw zadań}
\author{Robert Fidytek}
%\date{\today}
\date{}
\newcounter{liczniksekcji}
\newcommand{\kategoria}[1]{\section{#1}} %olreślamy nazwę kateforii zadań
\newcommand{\zadStart}[1]{\begin{zad}#1\newline} %oznaczenie początku zadania
\newcommand{\zadStop}{\end{zad}}   %oznaczenie końca zadania
%Makra opcjonarne (nie muszą występować):
\newcommand{\rozwStart}[2]{\noindent \textbf{Rozwiązanie (autor #1 , recenzent #2): }\newline} %oznaczenie początku rozwiązania, opcjonarnie można wprowadzić informację o autorze rozwiązania zadania i recenzencie poprawności wykonania rozwiązania zadania
\newcommand{\rozwStop}{\newline}                                            %oznaczenie końca rozwiązania
\newcommand{\odpStart}{\noindent \textbf{Odpowiedź:}\newline}    %oznaczenie początku odpowiedzi końcowej (wypisanie wyniku)
\newcommand{\odpStop}{\newline}                                             %oznaczenie końca odpowiedzi końcowej (wypisanie wyniku)
\newcommand{\testStart}{\noindent \textbf{Test:}\newline} %ewentualne możliwe opcje odpowiedzi testowej: A. ? B. ? C. ? D. ? itd.
\newcommand{\testStop}{\newline} %koniec wprowadzania odpowiedzi testowych
\newcommand{\kluczStart}{\noindent \textbf{Test poprawna odpowiedź:}\newline} %klucz, poprawna odpowiedź pytania testowego (jedna literka): A lub B lub C lub D itd.
\newcommand{\kluczStop}{\newline} %koniec poprawnej odpowiedzi pytania testowego 
\newcommand{\wstawGrafike}[2]{\begin{figure}[h] \includegraphics[scale=#2] {#1} \end{figure}} %gdyby była potrzeba wstawienia obrazka, parametry: nazwa pliku, skala (jak nie wiesz co wpisać, to wpisz 1)

\begin{document}
\maketitle


\kategoria{Wikieł/Z1.6b}
\zadStart{Zadanie z Wikieł Z 1.6 b) moja wersja nr [nrWersji]}
%[p1]:[5,8,11,13]
%[p2]:[3,7,10,12,14]
%[a]=random.randint(2,16)
%[p11]=[p1]-1
%[p21]=[p2]-1
%[p2p1]=[p2]+1
%[c]=[p21]*[a]-[p11]*[p2p1]
%[d]=([p21]*[a])**(2)*[p1]
%[e]=([p11]*2)**(2)*[p2]
%[c]>0 and [p1]!=[p2] and [d]>[e]
Sprawdzić, która z liczb jest większa: $A$ czy $B$.
b) $A=\frac{[a]}{\sqrt{[p1]}-1}$, $B=\frac{\sqrt{[p2]}+1}{\sqrt{[p2]}-1}$
\zadStop
\rozwStart{Wojciech Przybylski}{Natalia Danieluk}
$$A=\frac{[a]}{\sqrt{[p1]}-1}\cdot \frac{\sqrt{[p1]}+1}{\sqrt{[p1]}+1}=\frac{(\sqrt{[p1]}+1)\cdot [a]}{[p11]}=\frac{[a]\sqrt{[p1]}+[a]}{[p11]}$$
$$B=\frac{\sqrt{[p2]}+1}{\sqrt{[p2]}-1}\cdot \frac{\sqrt{[p2]}+1}{\sqrt{[p2]}+1}=\frac{(\sqrt{[p2]}+1)^{2}}{[p21]}=\frac{[p2p1]+2\sqrt{[p2]}}{[p21]}$$
$$A-B=[p21]\cdot([a]\sqrt{[p1]}+[a])-[p11]\cdot([p2p1]+2\sqrt{[p2]})=$$
$$=[c]+(\sqrt{[d]}-\sqrt{[e]})>0 \mbox{ stąd wiemy że }A>B$$
\rozwStop
\odpStart
$A>B$
\odpStop
\testStart
A. $A>B$
B. $B>A$
C. $A=B$
D. Nie da się sprawdzić, co jest większe.
\testStop
\kluczStart
A
\kluczStop



\end{document}