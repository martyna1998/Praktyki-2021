\documentclass[12pt, a4paper]{article}
\usepackage[utf8]{inputenc}
\usepackage{polski}

\usepackage{amsthm}  %pakiet do tworzenia twierdzeń itp.
\usepackage{amsmath} %pakiet do niektórych symboli matematycznych
\usepackage{amssymb} %pakiet do symboli mat., np. \nsubseteq
\usepackage{amsfonts}
\usepackage{graphicx} %obsługa plików graficznych z rozszerzeniem png, jpg
\theoremstyle{definition} %styl dla definicji
\newtheorem{zad}{} 
\title{Multizestaw zadań}
\author{Jacek Jabłoński}
%\date{\today}
\date{}
\newcounter{liczniksekcji}
\newcommand{\kategoria}[1]{\section{#1}} %olreślamy nazwę kateforii zadań
\newcommand{\zadStart}[1]{\begin{zad}#1\newline} %oznaczenie początku zadania
\newcommand{\zadStop}{\end{zad}}   %oznaczenie końca zadania
%Makra opcjonarne (nie muszą występować):
\newcommand{\rozwStart}[2]{\noindent \textbf{Rozwiązanie (autor #1 , recenzent #2): }\newline} %oznaczenie początku rozwiązania, opcjonarnie można wprowadzić informację o autorze rozwiązania zadania i recenzencie poprawności wykonania rozwiązania zadania
\newcommand{\rozwStop}{\newline}                                            %oznaczenie końca rozwiązania
\newcommand{\odpStart}{\noindent \textbf{Odpowiedź:}\newline}    %oznaczenie początku odpowiedzi końcowej (wypisanie wyniku)
\newcommand{\odpStop}{\newline}                                             %oznaczenie końca odpowiedzi końcowej (wypisanie wyniku)
\newcommand{\testStart}{\noindent \textbf{Test:}\newline} %ewentualne możliwe opcje odpowiedzi testowej: A. ? B. ? C. ? D. ? itd.
\newcommand{\testStop}{\newline} %koniec wprowadzania odpowiedzi testowych
\newcommand{\kluczStart}{\noindent \textbf{Test poprawna odpowiedź:}\newline} %klucz, poprawna odpowiedź pytania testowego (jedna literka): A lub B lub C lub D itd.
\newcommand{\kluczStop}{\newline} %koniec poprawnej odpowiedzi pytania testowego 
\newcommand{\wstawGrafike}[2]{\begin{figure}[h] \includegraphics[scale=#2] {#1} \end{figure}} %gdyby była potrzeba wstawienia obrazka, parametry: nazwa pliku, skala (jak nie wiesz co wpisać, to wpisz 1)

\begin{document}
\maketitle


\kategoria{Wikieł/z1.86c}
\zadStart{Zadanie z Wikieł z1.86c) moja wersja nr [nrWersji]}
%[p1]:[2,3,4,5,6,7,8]
%[p2]:[2,3,4,5,6,7,8]
%[w1]=int(math.pow(2,[p1]))
%[w2]=[p2]
%[r1]=[p1]-1
%[r2]=1-[p1]
%[r4]=[p1]*[w2]
%[r5]=[r4]-[w2]
%[r6]=[w2]-[r4]
%[r7]=-[w2]+[p1]
%[delta]=abs(int([r6]*[r6] - 4*[r5]*[r7]))
%[Pdelta]=int(math.pow([delta],(1/2)))
%[x1]=int((-[r6]-[Pdelta])/(2*[r5]))
%[x2]=int((-[r6]+[Pdelta])/(2*[r5]))
%[c4]=math.sqrt([delta])
%[c5]=math.isqrt([delta])
%[f1a]=[x1]-1
%[f2a]=[x1]-2
%[f1b]=[x2]+1
%[f2b]=[x2]+2
%[delta]>0 and [p2]<=[p1] and not([c4]!=[c5])
Rozwiązać nierówność:
c) $(\frac{1}{2})^{[p1]x^2+ x-1} > (\frac{1}{[w1]})^{\frac{1}{[p1]}x^2 +x-\frac{1}{[w2]}}$
\zadStop
\rozwStart{Jacek Jabłoński}{}
$$(\frac{1}{2})^{[p1]x^2+ x-1} > (\frac{1}{[w1]})^{\frac{1}{[p1]}x^2 +x-\frac{1}{[w2]}}$$
$$(\frac{1}{2})^{[p1]x^2+ x-1} > (\frac{1}{2^{[p1]}})^{\frac{1}{[p1]}x^2 +x-\frac{1}{[w2]}}$$
$$(\frac{1}{2})^{[p1]x^2+ x-1} > (\frac{1}{2})^{[p1](\frac{1}{[p1]}x^2 +x-\frac{1}{[w2]})}$$
$$(\frac{1}{2})^{[p1]x^2+ x-1} > (\frac{1}{2})^{x^2 +[p1]x- \frac{[p1]}{[w2]}}$$
$$[p1]x^2+ x-1 < x^2 +[p1]x-\frac{[p1]}{[w2]}$$
$$[r4]x^2 + [w2]x -[w2] < [w2]x^2 + [r4]x - [p1]$$
$$[r5]x^2 + ([r6])x + [r7] < 0$$
$$\Delta = [delta]$$
$$\sqrt{\Delta} = [Pdelta]$$
$$x_1=\frac{-([r6])-[Pdelta]}{2 \cdot [r5]} = [x1] $$
$$x_1=\frac{-([r6])+[Pdelta]}{2 \cdot [r5]} = [x2] $$
$$x \in ([x1] ; [x2])$$
\rozwStop
\odpStart
$$x \in ([x1] ; [x2])$$
\odpStop
\testStart
A. $$x \in ([x1] ; [x2])$$
B. $$x \in ([f1a] ; [x2])$$
C. $$x \in ([x1] ; [f1b])$$
D. $$x \in ([f2a] ; [x2])$$
E. $$x \in ([x1] ; [f2b])$$
F. $$x \in ([f1a] ; [f1b])$$
G. $$x \in ([f2a] ; [f2b])$$
H. $$x < [x2]$$
I. $$x > [x1]$$
\testStop
\kluczStart
A
\kluczStop



\end{document}