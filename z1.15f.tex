\documentclass[12pt, a4paper]{article}
\usepackage[utf8]{inputenc}
\usepackage{polski}

\usepackage{amsthm}  %pakiet do tworzenia twierdzeń itp.
\usepackage{amsmath} %pakiet do niektórych symboli matematycznych
\usepackage{amssymb} %pakiet do symboli mat., np. \nsubseteq
\usepackage{amsfonts}
\usepackage{graphicx} %obsługa plików graficznych z rozszerzeniem png, jpg
\theoremstyle{definition} %styl dla definicji
\newtheorem{zad}{} 
\title{Multizestaw zadań}
\author{Robert Fidytek}
%\date{\today}
\date{}
\newcounter{liczniksekcji}
\newcommand{\kategoria}[1]{\section{#1}} %olreślamy nazwę kateforii zadań
\newcommand{\zadStart}[1]{\begin{zad}#1\newline} %oznaczenie początku zadania
\newcommand{\zadStop}{\end{zad}}   %oznaczenie końca zadania
%Makra opcjonarne (nie muszą występować):
\newcommand{\rozwStart}[2]{\noindent \textbf{Rozwiązanie (autor #1 , recenzent #2): }\newline} %oznaczenie początku rozwiązania, opcjonarnie można wprowadzić informację o autorze rozwiązania zadania i recenzencie poprawności wykonania rozwiązania zadania
\newcommand{\rozwStop}{\newline}                                            %oznaczenie końca rozwiązania
\newcommand{\odpStart}{\noindent \textbf{Odpowiedź:}\newline}    %oznaczenie początku odpowiedzi końcowej (wypisanie wyniku)
\newcommand{\odpStop}{\newline}                                             %oznaczenie końca odpowiedzi końcowej (wypisanie wyniku)
\newcommand{\testStart}{\noindent \textbf{Test:}\newline} %ewentualne możliwe opcje odpowiedzi testowej: A. ? B. ? C. ? D. ? itd.
\newcommand{\testStop}{\newline} %koniec wprowadzania odpowiedzi testowych
\newcommand{\kluczStart}{\noindent \textbf{Test poprawna odpowiedź:}\newline} %klucz, poprawna odpowiedź pytania testowego (jedna literka): A lub B lub C lub D itd.
\newcommand{\kluczStop}{\newline} %koniec poprawnej odpowiedzi pytania testowego 
\newcommand{\wstawGrafike}[2]{\begin{figure}[h] \includegraphics[scale=#2] {#1} \end{figure}} %gdyby była potrzeba wstawienia obrazka, parametry: nazwa pliku, skala (jak nie wiesz co wpisać, to wpisz 1)

\begin{document}
\maketitle


\kategoria{Wikieł/Z1.15f}
\zadStart{Zadanie z Wikieł Z 1.15 f) moja wersja nr [nrWersji]}
%[a]:[2,3,4,5,6,7]
%[b]:[2,3,4,5,6,7]
%[c]:[2,3,4,5,6,7]
%[d]:[2,3,4,5,6,7]
%[e]=random.randint(2,10)
%[f]=random.randint(2,10)
%[da]=[d]-[a]
%[dab]=-[da]-[b]
%[dabb]=[da]-[b]
%[w1]=[dab]/-[c]
%[w2]=[dabb]/-[c]
%[w11]=int([w1])
%[w22]=int([w2])
%math.gcd([dab],[c])==[c] and math.gcd([dabb],[c])==[c] and [w11]<[w22]
Rozwiązać nierówność $[a]-|[b]-[c]x|>[d]$.
\zadStop
\rozwStart{Jakub Ulrych}{Pascal Nawrocki}
$$[a]-|[b]-[c]x|>[d]$$ 
$$-|[b]-[c]x|>[da]$$
$$|[b]-[c]x|<-([da])$$
$$[b]-[c]x< -([da]) \land [b]-[c]x>[da]$$
$$-([c]x)< [dab] \land -([c]x)> [dabb]$$
$$x> [w11] \land x< [w22]$$
$$x\in([w11],[w22])$$
\rozwStop
\odpStart
$$x\in([w11],[w22])$$
\odpStop
\testStart
A.$x\in([w11],[w22])$
B.$x=[w11]$
C.$x\in(-\infty,[w22])$
D.$x\in([w11],\infty)$
\testStop
\kluczStart
A
\kluczStop



\end{document}